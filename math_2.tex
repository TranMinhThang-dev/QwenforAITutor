========================================================================

https://khoahoc.vietjack.com/thi-online/31-cau-trac-nghiem-bat-phuong-trinh-va-he-bat-phuong-trinh-mot-an-co-dap-an/50656


\textbf{{QUESTION}}

Giá trị x = -1 là nghiệm của bất phương trình nào sau đây?
A. 3 - x < 0
B. 2x + 1 < 0
C. 2x - 1 > 0
D. x - 1 > 0

\textbf{{ANSWER}}

Thử x = -1 vào các bất phương trình ta thấy x = -1 là nghiệm của bất phương trình 2x + 1 < 0.
 
Cũng có thể giải các bất phương trình, từ đó thấy x = -1 chỉ là nghiệm của phương trình 2x + 1 < 0. Đáp án là B.

========================================================================

https://khoahoc.vietjack.com/thi-online/31-cau-trac-nghiem-bat-phuong-trinh-va-he-bat-phuong-trinh-mot-an-co-dap-an/50656


\textbf{{QUESTION}}

Giá trị nào sau đây là nghiệm của bất phương trình $$ \frac{\left|1-x\right|}{\sqrt{3-x}}>\frac{x-1}{\sqrt{3-x}}$$?
A. x = 2
B. x = 1
C. x = 0
D. x = $$ \frac{3}{2}$$

\textbf{{ANSWER}}

Cách 1: Điều kiện xác định của bất phương trình là x < 3. Khi đó:
$$ \frac{\left|1-x\right|}{\sqrt{3-x}}>\frac{x-1}{\sqrt{3-x}}\Leftrightarrow \left|1-x\right|>x-1\Leftrightarrow x-1<0\Leftrightarrow x<1$$.
Kết hợp lại, suy ra nghiệm của bất phương trình đã cho là x < 1.
 Đáp án là C.
Cách 2: Có thể thay các giá trị trên vào bất phương trình, thực chất chỉ cần thay vào x - 1  ( bỏ đi) rồi suy ra kết luận.

========================================================================

https://khoahoc.vietjack.com/thi-online/31-cau-trac-nghiem-bat-phuong-trinh-va-he-bat-phuong-trinh-mot-an-co-dap-an/50656


\textbf{{QUESTION}}

Bất phương trình nào sau đây tương đương với bất phương trình 2x > 1?
A. 2x+√x-2>1+√x-2$$ 2x+\sqrt{x-2}>1+\sqrt{x-2}$$
B. 2x-1x-3>1-1x-3$$ 2x-\frac{1}{x-3}>1-\frac{1}{x-3}$$
C. 4x2>1$$ 4{x}^{2}>1$$
D. 2x+√x+2>1+√x+2$$ 2x+\sqrt{x+2}>1+\sqrt{x+2}$$

\textbf{{ANSWER}}

Cách 1:
* Ta có: 2x > 1⇔$$ \Leftrightarrow $$x > 12$$ \frac{1}{2}$$
* Xét: 2x+√x+2>1+√x+2$$ 2x+\sqrt{x+2}>1+\sqrt{x+2}$$
Điều kiện: x≥-2$$ x\ge -2$$
Với điều kiện trên, (1) tương đương:  2x>1⇔x>12$$ 2x>1\Leftrightarrow x>\frac{1}{2}$$
Kết hợp điều kiện ta được nghiệm của bất phương trình này là: x>12$$ x>\frac{1}{2}$$
Do đó, bất phương trình đã cho tương đương bất phương trình D.
Cách 2: Dùng phương pháp loại trừ.
·       x = 1 là nghiệm của bất phương trình 2x > 1 nhưng không là nghiệm của bất phương trình 2x+√x-2>1+√x-2$$ 2x+\sqrt{x-2}>1+\sqrt{x-2}$$, do đó hai bất phương trình không tương đương.
·       x= -2 là nghiệm của bất phương trình 4x2 > 1 nhưng không là nghiệm bất phương trình 2x > 1.
                ·    x = 3 là nghiệm của bất phương trình  2x > 1 nhưng không là nghiệm của bất phương trình 2x-1x-3>1-1x-3$$ 2x-\frac{1}{x-3}>1-\frac{1}{x-3}$$, do đó hai bất phương trình không tương đương. Đáp án là D

========================================================================

https://khoahoc.vietjack.com/thi-online/31-cau-trac-nghiem-bat-phuong-trinh-va-he-bat-phuong-trinh-mot-an-co-dap-an/50656


\textbf{{QUESTION}}

Tập nghiệm của bất phương trình 2x+2>3(2-x)+1$$ 2x+2>3(2-x)+1$$ là:
A. S=(1;+∞)$$ S=\left(1;+\infty \right)$$
B. S=(-∞;-5)$$ S=\left(-\infty ;-5\right)$$
C. S=(5;+∞)$$ S=\left(5;+\infty \right)$$
D. S=(-∞;5)$$ S=\left(-\infty ;5\right)$$

\textbf{{ANSWER}}

Ta có :
2x+2>3(2-x)+1⇔2x+2>6-3x+1⇔5x>5⇔x>1$$ 2x+2>3(2-x)+1\Leftrightarrow 2x+2>6-3x+1\phantom{\rule{0ex}{0ex}}\Leftrightarrow 5x>5\Leftrightarrow x>1$$.
Vậy tập nghiệm của bất phương trình 2x+2>3(2-x)+1$$ 2x+2>3(2-x)+1$$ là (1;+∞)$$ \left(1;+\infty \right)$$.
 
 Đáp án là A.

========================================================================

https://khoahoc.vietjack.com/thi-online/31-cau-trac-nghiem-bat-phuong-trinh-va-he-bat-phuong-trinh-mot-an-co-dap-an/50656


\textbf{{QUESTION}}

Tập xác định của hàm số y=12-x là:
A. D=(-∞;2]
B. D=(2;+∞)
C. D=(-∞;2)
D. D=[2;+∞)

\textbf{{ANSWER}}

Hàm số y=12-x xác định khi và chỉ khi 2-x>0⇔x<2.
 
Vậy tập xác định của hàm số y=12-x là (-∞;2).
 
 Đáp án là C.

========================================================================

https://khoahoc.vietjack.com/thi-online/19-cau-trac-nghiem-toan-7-bai-4-truong-hop-bang-nhau-thu-hai-cua-tam-giac-co-dap-an


\textbf{{QUESTION}}

Cho tam giác ABC và tam giác MHK có: AB=MH, ˆA=ˆM. Cần thêm một điều kiện gì để tam giác ABC và tam giác MHK bằng nhau theo trường hợp cạnh-góc-cạnh 
A. BC=MK
B. BC=HK
C. AC=MK
D. AC=HK

\textbf{{ANSWER}}

Đáp án C
Để tam giác ABC và tam giác MHK bằng nhau theo trường hợp cạnh – góc – cạnh cần thêm điều kiện về cạnh kề đó là: AC=MK

========================================================================

https://khoahoc.vietjack.com/thi-online/19-cau-trac-nghiem-toan-7-bai-4-truong-hop-bang-nhau-thu-hai-cua-tam-giac-co-dap-an


\textbf{{QUESTION}}

Cho tam giác ABC và tam giác DEF có: AB=DE,AC=DF. Cần thêm một điều kiện gì để tam giác ABC và tam giác DEF bằng nhau theo trường hợp cạnh-góc-cạnh
A. ˆA$$ \hat{A}$$ = ˆE$$ \hat{E}$$
B. BC = EF
C. ˆA$$ \hat{A}$$ = ˆD$$ \hat{D}$$
D. ˆB$$ \hat{B}$$ = ˆD$$ \hat{D}$$

\textbf{{ANSWER}}

Đáp án C
Để tam giác ABC và tam giác DEF bằng nhau theo trường hợp cạnh – góc – cạnh cần thêm điều kiện về cạnh kề đó là: ˆA$$ \hat{A}$$ = ˆD$$ \hat{D}$$

========================================================================

https://khoahoc.vietjack.com/thi-online/giai-sbt-toan-10-cd-bai-2-he-bat-phuong-trinh-bac-nhat-hai-an-co-dap-an


\textbf{{QUESTION}}

Cặp số nào sau đây là nghiệm của hệ bất phương trình $$ \left\{\begin{array}{l}x-2y<0\\ x+3y>-2\\ -x+y<3\end{array}\right..$$

\textbf{{ANSWER}}

Đáp án đúng là B
Ta xét hệ phương trình: $$ \left\{\begin{array}{l}x-2y<0\left(1\right)\\ x+3y>-2\left(2\right)\\ -x+y<3\left(3\right)\end{array}\right..$$
+) Thay x = 1 và y = 0 lần lượt vào các bất phương trình (1), (2) và (3) trong hệ, ta được: 
(1) ⇔ 1 – 2.0 < 0 ⇔ 1 < 0 (vô lí);
(2) ⇔ 1 + 3.0 > – 2 ⇔ 1 > – 2 (luôn đúng);
(3) ⇔ – 1 + 0 < 3 ⇔ – 1 < 3 (luôn đúng).
Do đó cặp số (1; 0) không là nghiệm của hệ bất phương trình đã cho. 
+) Thay x = – 1 và y = 0 lần lượt vào các bất phương trình (1), (2) và (3) trong hệ, ta được: 
(1) ⇔ – 1 – 2.0 < 0 ⇔ – 1 < 0 (luôn đúng);
(2) ⇔ – 1 + 3.0 > – 2 ⇔ – 1 > – 2 (luôn đúng);
(3) ⇔ 1 + 0 < 3 ⇔ 1 < 3 (luôn đúng).
Do đó cặp số (– 1; 0) là nghiệm của hệ bất phương trình đã cho. 
+) Thay x = – 2 và y = 3 lần lượt vào các bất phương trình (1), (2) và (3) trong hệ, ta được: 
(1) ⇔ – 2 – 2.3 < 0 ⇔ – 8 < 0 (luôn đúng);
(2) ⇔ – 2 + 3.3 > – 2 ⇔ 7 > – 2 (luôn đúng);
(3) ⇔ 2 + 3 < 3 ⇔ 5 < 3 (vô lí).
Do đó cặp số (– 2; 3) không là nghiệm của hệ bất phương trình đã cho. 
+) Thay x = 0 và y = – 1 lần lượt vào các bất phương trình (1), (2) và (3) trong hệ, ta được: 
(1) ⇔ 0 – 2.(– 1) < 0 ⇔ 2 < 0 (vô lí);
(2) ⇔ 0 + 3.(– 1) > – 2 ⇔ – 3 > – 2 (vô lí);
(3) ⇔ 0 + (– 1) < 3 ⇔ – 1 < 3 (luôn đúng).
Do đó cặp số (0; – 1) không là nghiệm của hệ bất phương trình đã cho. 
Vậy (– 1; 0) là nghiệm của hệ phương trình đã cho.

========================================================================

https://khoahoc.vietjack.com/thi-online/giai-sbt-toan-10-cd-bai-2-he-bat-phuong-trinh-bac-nhat-hai-an-co-dap-an


\textbf{{QUESTION}}

Cặp số nào sau đây không là nghiệm của hệ bất phương trình {x+y≤22x−3y>−2.$$ \left\{\begin{array}{l}x+y\le 2\\ 2x-3y>-2\end{array}\right..$$
A. (0; 0);

\textbf{{ANSWER}}

Đáp án đúng là C
Xét hệ phương trình: {x+y≤2(1)2x−3y>−2(2).$$ \left\{\begin{array}{l}x+y\le 2\left(1\right)\\ 2x-3y>-2\left(2\right)\end{array}\right..$$
+) Thay x = 0 và y = 0 lần lượt vào các bất phương trình (1) và (2) trong hệ, ta được: 
(1) ⇔ 0 + 0 ≤ 2 ⇔ 0 ≤ 2 (luôn đúng);
(2) ⇔ 2.0 – 3.0 > – 2 ⇔ 0 > – 2 (luôn đúng).
Do đó cặp số (0; 0) là nghiệm của hệ bất phương trình đã cho. 
+) Thay x = 1 và y = 1 lần lượt vào các bất phương trình (1) và (2) trong hệ, ta được: 
(1) ⇔ 1 + 1 ≤ 2 ⇔ 2 ≤ 2 (luôn đúng);
(2) ⇔ 2.1 – 3.1 > – 2 ⇔ – 1 > – 2 (luôn đúng).
Do đó cặp số (1; 1) là nghiệm của hệ bất phương trình đã cho. 
+) Thay x = – 1 và y = 1 lần lượt vào các bất phương trình (1) và (2) trong hệ, ta được: 
(1) ⇔ – 1 + 1 ≤ 2 ⇔ 0 ≤ 2 (luôn đúng);
(2) ⇔ 2.(– 1) – 3.1 > – 2 ⇔ – 5 > – 2 (vô lí).
Do đó cặp số (– 1; 1) không là nghiệm của hệ bất phương trình đã cho. 
+) Thay x = – 1 và y = – 1 lần lượt vào các bất phương trình (1) và (2) trong hệ, ta được: 
(1) ⇔ – 1 + (– 1) ≤ 2 ⇔ – 2 ≤ 2 (luôn đúng);
(2) ⇔ 2.(– 1) – 3.(– 1) > – 2 ⇔ 1 > – 2 (luôn đúng).
Do đó cặp số (– 1; – 1) là nghiệm của hệ bất phương trình đã cho. 
Vậy cặp số (– 1; 1) không là nghiệm của hệ bất phương trình đã cho.

========================================================================

https://khoahoc.vietjack.com/thi-online/giai-sbt-toan-10-cd-bai-2-he-bat-phuong-trinh-bac-nhat-hai-an-co-dap-an


\textbf{{QUESTION}}

Miền nghiệm của hệ bất phương trình {2x−5y>12x+y>−5x+y<−1 là phần mặt phẳng chứa điểm có tọa độ:
A. (0; 0);

\textbf{{ANSWER}}

Đáp án đúng là D
Ta xét hệ bất phương trình {2x−5y>1(1)2x+y>−5(2)x+y<−1(3)$$ \left\{\begin{array}{l}2x-5y>1\left(1\right)\\ 2x+y>-5\left(2\right)\\ x+y<-1\left(3\right)\end{array}\right.$$.
+) Thay x = 0 và y = 0 lần lượt vào các bất phương trình (1), (2) và (3) trong hệ, ta được: 
(1) ⇔ 2.0 – 5.0 > 1 ⇔ 0 > 1 (vô lí);
(2) ⇔ 2.0 + 0 > – 5 ⇔ 0 > – 5 (luôn đúng);
(3) ⇔ 0 + 0 < – 1 ⇔ 0 < – 1 (vô lí).
Do đó cặp số (0; 0) không thuộc miền nghiệm của hệ bất phương trình đã cho. 
+) Thay x = 1 và y = 0 lần lượt vào các bất phương trình (1), (2) và (3) trong hệ, ta được: 
(1) ⇔ 2.1 – 5.0 > 1 ⇔ 2 > 1 (luôn đúng);
(2) ⇔ 2.1 + 0 > – 5 ⇔ 2 > – 5 (luôn đúng);
(3) ⇔ 1 + 0 < – 1 ⇔ 1 < – 1 (vô lí).
Do đó cặp số (1; 0) không thuộc miền nghiệm của hệ bất phương trình đã cho. 
+) Thay x = 0 và y = 2 lần lượt vào các bất phương trình (1), (2) và (3) trong hệ, ta được: 
(1) ⇔ 2.0 – 5.2 > 1 ⇔ – 10 > 1 (vô lí);
(2) ⇔ 2.0 + 2 > – 5 ⇔ 2 > – 5 (luôn đúng);
(3) ⇔ 0 + 2 < – 1 ⇔ 2 < – 1 (vô lí).
Do đó cặp số (0; 2) không thuộc miền nghiệm của hệ bất phương trình đã cho. 
+) Thay x = 0 và y = – 2 lần lượt vào các bất phương trình (1), (2) và (3) trong hệ, ta được: 
(1) ⇔ 2.0 – 5.(– 2) > 1 ⇔ 10 > 1 (luôn đúng);
(2) ⇔ 2.0 + (– 2) > – 5 ⇔ – 2 > – 5 (luôn đúng);
(3) ⇔ 0 + (– 2) < – 1 ⇔ – 2 < – 1 (luôn đúng).
Do đó cặp số (0; – 2 ) thuộc miền nghiệm của hệ bất phương trình đã cho.

========================================================================

https://khoahoc.vietjack.com/thi-online/160-bai-trac-nghiem-mat-non-mat-tru-mat-cau-cuc-hay-co-loi-giai


\textbf{{QUESTION}}

Trong không gian Oxyz, cho hai điểm I(1;1;1) và A(1;2;3). Phương trình của mặt cầu có tâm I và đi qua A là:
A.$$ {(\mathrm{x}+1)}^{2}+{(\mathrm{y}+1)}^{2}+{(\mathrm{z}+1)}^{2}=29$$
B.$$ {(\mathrm{x}-1)}^{2}+{(\mathrm{y}-1)}^{2}+{(\mathrm{z}-1)}^{2}=5$$
C.$$ {(\mathrm{x}-1)}^{2}+{(\mathrm{y}-1)}^{2}+{(\mathrm{z}-1)}^{2}=25$$
D.$$ {(\mathrm{x}+1)}^{2}+{(\mathrm{y}+1)}^{2}+{(\mathrm{z}+1)}^{2}=5$$

\textbf{{ANSWER}}

Chọn đáp án B.
Có $$ \mathrm{IA}=\mathrm{R}=\sqrt{{1}^{2}+{2}^{2}}=5$$
Khi đó mặt cầu tâm I đi qua A có phương trình $$ {(\mathrm{x}-1)}^{2}+{(\mathrm{y}-1)}^{2}+{(\mathrm{z}-1)}^{2}=5$$

========================================================================

https://khoahoc.vietjack.com/thi-online/160-bai-trac-nghiem-mat-non-mat-tru-mat-cau-cuc-hay-co-loi-giai


\textbf{{QUESTION}}

Cho khối nón có độ dài đường sinh bằng 2a và bán kính đáy bằng a. Thể tích của khối nón đã cho bằng:
A.√3πα33$$ \frac{\sqrt{3}\pi {\alpha }^{3}}{3}$$
B.√3πα32$$ \frac{\sqrt{3}\pi {\alpha }^{3}}{2}$$
C.2πα33$$ \frac{2\pi {\alpha }^{3}}{3}$$
D.πα33$$ \frac{\pi {\alpha }^{3}}{3}$$

\textbf{{ANSWER}}

Chọn đáp án A.
Có $$ l=2a,\quad r=a$$
$$ \Rightarrow h=\sqrt{{l}^{2}-{r}^{2}}=a\sqrt{3}$$
Khi đó thể tích khối nón là:
$$ V=\frac{1}{3}\pi {r}^{2}h=\frac{\sqrt{3}\pi {\alpha }^{3}}{3}$$

========================================================================

https://khoahoc.vietjack.com/thi-online/bai-tap-toan-6-boi-va-uoc-cua-mot-so-nguyen


\textbf{{QUESTION}}

Tìm năm bội của: 5; -5.

\textbf{{ANSWER}}

Năm bội của 5 là {10; 15; 20; 25; 30};
Năm bội của -5 là {-10;-15;-20;-25;-30}.

========================================================================

https://khoahoc.vietjack.com/thi-online/bai-tap-toan-6-boi-va-uoc-cua-mot-so-nguyen


\textbf{{QUESTION}}

Tìm tất cả các ước của: -6; 9; 12; -7; -196.

\textbf{{ANSWER}}

Các ước của -6 Là {-6;-3;-2;-l; l; 2; 3; 6}; phương pháp tương tự đối với các số còn lại.

========================================================================

https://khoahoc.vietjack.com/thi-online/bai-tap-toan-6-boi-va-uoc-cua-mot-so-nguyen


\textbf{{QUESTION}}

Các số sau có bao nhiêu ước:
a) 54;
b) -166.

\textbf{{ANSWER}}

a) 54=2.33$$ 54=2.{3}^{3}$$ nên có tất cả 2 (1 + 1) ( 3 + 1) = 16 ước
b) -166 = -2. 83 nên có tất cả 2 (1 + 1) ( 1 + 1) = 8 ước

========================================================================

https://khoahoc.vietjack.com/thi-online/giai-sgk-toan-11-kntt-bai-7-cap-so-nhan-co-dap-an


\textbf{{QUESTION}}

Một công ty tuyển một chuyên gia về công nghệ thông tin với mức lương năm đầu là 240 triệu đồng và cam kết sẽ tăng thêm 5% lương mỗi năm so với năm liền trước đó. Tính tổng số lương mà chuyên gia đó nhận được sau khi làm việc cho công ty 10 năm (làm tròn đến triệu đồng).

\textbf{{ANSWER}}

Lời giải: 
Sau bài học này ta sẽ giải quyết được bài toán trên như sau:
Lương hằng năm (triệu đồng) của chuyên gia lập thành một cấp số nhân, với số hạng đầu u1 = 240 và công bội q = 1,05. Tổng số lương của chuyên gia đó sau 10 năm chính là tổng của 10 số hạng đầu của cấp số nhân này và bằng 
${S_{10}} = \frac{{{u_1}\left( {1 - {q^{10}}} \right)}}{{1 - q}} = \frac{{240\left[ {1 - {{\left( {1,05} \right)}^{10}}} \right]}}{{1 - 1,05}} \approx \,3\,019$. 
Vậy tổng số lương (làm tròn đến triệu đồng) của chuyên gia đó sau 10 năm là 3 019 triệu đồng hay 3,019 tỉ đồng.

========================================================================

https://khoahoc.vietjack.com/thi-online/de-thi-giua-hoc-ki-2-toan-7-kntt-co-dap-an/115732


\textbf{{QUESTION}}

Trong các công thức sau, công thức nào không biểu diễn y là hàm số của x?
A. 2x + y = 5;
B. $\sqrt {x - 1} $ + y = 5; 
C. $y = \sqrt {{x^2} - 2} $;
D. 2x2 – 3y2 = 0.

\textbf{{ANSWER}}

Đáp án D

========================================================================

https://khoahoc.vietjack.com/thi-online/de-thi-giua-hoc-ki-2-toan-7-kntt-co-dap-an/115732


\textbf{{QUESTION}}

Cho hàm số dưới dạng bảng như sau:
x
0 
1
2
3
4
y
0
1
4
9
16
Giá trị của hàm số y tại x = 1 là
A. 1
B. 4
C. 9
D. 16

\textbf{{ANSWER}}

Đáp án A

========================================================================

https://khoahoc.vietjack.com/thi-online/de-kiem-tra-15-phut-toan-7-chuong-4-dai-so-co-dap-an/39814


\textbf{{QUESTION}}

A. Phần trắc nghiệm (3 điểm)
Trong mỗi câu dưới đây, hãy chọn phương án trả lời đúng:
Phần hệ số của đơn thức $$ {x}^{3}{y}^{2}z$$ là
A. 0
B. 1
C. 2
D. -1

\textbf{{ANSWER}}

Chọn B

========================================================================

https://khoahoc.vietjack.com/thi-online/de-kiem-tra-15-phut-toan-7-chuong-4-dai-so-co-dap-an/39814


\textbf{{QUESTION}}

Bậc của đơn thức 12xyx2$$ \frac{1}{2}xy{x}^{2}$$ là:
A. 5
B. 4
C. 3
D. 2

\textbf{{ANSWER}}

Thu gọn 1/2 xy2x2 = 1/2 x3y2.
Bậc của đơn thức là 5. Chọn A

========================================================================

https://khoahoc.vietjack.com/thi-online/de-kiem-tra-15-phut-toan-7-chuong-4-dai-so-co-dap-an/39814


\textbf{{QUESTION}}

Giá trị của đơn thức -13x2y$$ \frac{-1}{3}{x}^{2}y$$ tại x = 3, y = 1 là:
A. 1
B. -1
C. 3
D. -3

\textbf{{ANSWER}}

Thay x = 3, y = 1 vào biểu thức ta có -1/3 .(3)2.1 = -3. Chọn D

========================================================================

https://khoahoc.vietjack.com/thi-online/de-kiem-tra-15-phut-toan-7-chuong-4-dai-so-co-dap-an/39814


\textbf{{QUESTION}}

Đơn thức đồng dạng với đơn thức 23x7y2$$ \frac{2}{3}{x}^{7}{y}^{2}$$ là:
A. 23x2y7$$ \frac{2}{3}{x}^{2}{y}^{7}$$
B. -67x7y2$$ \frac{-6}{7}{x}^{7}{y}^{2}$$
C. -x5y4$$ -{x}^{5}{y}^{4}$$
D. 13x7y3$$ \frac{1}{3}{x}^{7}{y}^{3}$$

\textbf{{ANSWER}}

Chọn B

========================================================================

https://khoahoc.vietjack.com/thi-online/bai-tap-chuyen-de-toan-6-dang-1-so-sanh-phan-so-co-dap-an/106999


\textbf{{QUESTION}}

So sánh hai phân số $$ \frac{11}{52}\quad và\quad \frac{17}{76}$$

\textbf{{ANSWER}}

Ta nhận thấy hai phân số đã cho nếu lấy mẫu số chia cho tử số thì đều được thương là 4 và số dư là 8 nên ta nhân cả hai phân số với 4 . 
Ta có: $$ \frac{11}{52}\times 4=\frac{44}{52};$$
         $$ \frac{17}{76}\times 4=\frac{68}{76}\cdot 1-\frac{44}{52}=\frac{8}{52};$$   
          $$ 1-\frac{68}{76}=\frac{8}{76}$$  
Vì $$ \frac{8}{52}>\frac{8}{76}$$  nên $$ \frac{44}{52}<\frac{68}{76}$$  hay  $$ \frac{11}{52}<\frac{17}{76}.$$
*  Phương pháp so sánh hai phân số bằng cách "phép chia hai phân số" 
- Phương pháp này được sử dụng dựa vào nhận xét: "Trong phép chia, nếu số bị chia lớn hơn số chia thì được thương lớn hơn 1, nếu số bi chia bé hơn số chia thì được thương nhỏ hơn 1".
- Ta sử dụng phương pháp "chia hai phân số" khi nhận thấy tử số và mẫu số của hai phân số là những số có giá trị không quá lớn, không mất nhiều thời gian khi thực hiện phép nhân ở tử số và mẫu số.

========================================================================

https://khoahoc.vietjack.com/thi-online/bai-tap-chuyen-de-toan-6-dang-1-so-sanh-phan-so-co-dap-an/106999


\textbf{{QUESTION}}

So sánh hai phân số 223 và 941$$ \frac{2}{23}\quad và\quad \frac{9}{41}$$

\textbf{{ANSWER}}

Ta có:   $$ \frac{2}{23}:\frac{9}{41}=\frac{2}{23}\times \frac{41}{9}=\frac{82}{207}.$$
Vì   $$ \frac{82}{207}<1$$ nên  $$ \frac{2}{23}<\frac{9}{41}$$.

========================================================================

https://khoahoc.vietjack.com/thi-online/bai-tap-chuyen-de-toan-6-dang-1-so-sanh-phan-so-co-dap-an/106999


\textbf{{QUESTION}}

So sánh hai phân số  A=108+1109+1 và B=109+11010+1$$ \text{A}=\frac{{10}^{8}+1}{{10}^{9}+1}\quad và\quad \text{B}=\frac{{10}^{9}+1}{{10}^{10}+1}$$.

\textbf{{ANSWER}}

Cách 1:   là phân số nhỏ hơn 1 . Nếu cộng cùng một số nguyên dương vào tử và mẫu của   thì giá trị của   tăng thêm. Do dó
B=109+11010+1<109+1+91010+1+9=109+101010+10=10(108+1)10(109+1)=108+1109+1=A$$ \text{B}=\frac{{10}^{9}+1}{{10}^{10}+1}<\frac{{10}^{9}+1+9}{{10}^{10}+1+9}=\frac{{10}^{9}+10}{{10}^{10}+10}=\frac{10\left({10}^{8}+1\right)}{10\left({10}^{9}+1\right)}=\frac{{10}^{8}+1}{{10}^{9}+1}=\text{A}$$
 
Vậy B < A
Cách 2. (sau khi học phép nhân phân sô)
10 A=10(108+1)109+1=109+10109+1=1+9109+110 B=10(109+1)1010+1=1010+101010+1=1+91010+1$$ \begin{array}{l}10\quad \text{A}=\frac{10\left({10}^{8}+1\right)}{{10}^{9}+1}=\frac{{10}^{9}+10}{{10}^{9}+1}=1+\frac{9}{{10}^{9}+1}\\ 10\quad \text{B}=\frac{10\left({10}^{9}+1\right)}{{10}^{10}+1}=\frac{{10}^{10}+10}{{10}^{10}+1}=1+\frac{9}{{10}^{10}+1}\end{array}$$
 
Ta thấy   9109+1>91010+1$$ \frac{9}{{10}^{9}+1}>\frac{9}{{10}^{10}+1}$$ (so sánh hai phân số cùng tử) nên 10 A>10 B$$ 10\quad \text{A}>10\quad \text{B}$$ . 
Do đó A> B.

========================================================================

https://khoahoc.vietjack.com/thi-online/bai-tap-chuyen-de-toan-6-dang-1-so-sanh-phan-so-co-dap-an/106999


\textbf{{QUESTION}}

So sánh   A=20032003+120032004+1 $$ \text{A}=\frac{{2003}^{2003}+1}{{2003}^{2004}+1}\text{\hspace{1em}}$$và    B=20032002+120032003+1$$ \text{\hspace{1em}B}=\frac{{2003}^{2002}+1}{{2003}^{2003}+1}$$

\textbf{{ANSWER}}

Nhận thấy tử và mẫu có số mũ lớn và đều cách nhau là 2003, nên:
2003.A =2003⋅(20032003+1)20032004+1=20032004+200320032004+1=1+200220032004+1$$ =\frac{2003\cdot \left({2003}^{2003}+1\right)}{{2003}^{2004}+1}=\frac{{2003}^{2004}+2003}{{2003}^{2004}+1}=1+\frac{2002}{{2003}^{2004}+1}$$
2003.B=2003⋅(20032002+1)20032003+1=20032003+200320032003+1=1+200220032003+1$$ \text{B}=\frac{2003\cdot \left({2003}^{2002}+1\right)}{{2003}^{2003}+1}=\frac{{2003}^{2003}+2003}{{2003}^{2003}+1}=1+\frac{2002}{{2003}^{2003}+1}$$  
Vì 200220032004+1<200220032003+1$$ \frac{2002}{{2003}^{2004}+1}<\frac{2002}{{2003}^{2003}+1}$$  (do cùng tử mà mẫu càng lớn phân số càng bé)
Nên A < B

========================================================================

https://khoahoc.vietjack.com/thi-online/bai-tap-chuyen-de-toan-6-dang-1-so-sanh-phan-so-co-dap-an/106999


\textbf{{QUESTION}}

a) So sánh phân số:  15301$$ \frac{15}{301}$$ với 25490$$ \frac{25}{490}$$

\textbf{{ANSWER}}

a) 15301<15300=120=25500<25499. $$ \frac{15}{301}<\frac{15}{300}=\frac{1}{20}=\frac{25}{500}<\frac{25}{499}.\text{ }$$
Vậy  15301<25499$$ \text{ }\frac{15}{301}<\frac{25}{499}$$

========================================================================

https://khoahoc.vietjack.com/thi-online/trac-nghiem-tong-hop-on-thi-tot-nghiep-thpt-mon-toan-chu-de-1-ham-so-va-ung-dung-co-dap-an/145218


\textbf{{QUESTION}}

Bạn hãy đọc đoạn văn 1 trên và trả lời câu hỏi.
a) ${f^\prime }(x) = 3{x^2} - 1$

\textbf{{ANSWER}}

Đúng

========================================================================

https://khoahoc.vietjack.com/thi-online/trac-nghiem-tong-hop-on-thi-tot-nghiep-thpt-mon-toan-chu-de-1-ham-so-va-ung-dung-co-dap-an/145218


\textbf{{QUESTION}}

Bạn hãy đọc đoạn văn 1 trên và trả lời câu hỏi.
b) ${f^\prime }(1) = 2.$

\textbf{{ANSWER}}

Đúng

========================================================================

https://khoahoc.vietjack.com/thi-online/trac-nghiem-tong-hop-on-thi-tot-nghiep-thpt-mon-toan-chu-de-1-ham-so-va-ung-dung-co-dap-an/145218


\textbf{{QUESTION}}

Bạn hãy đọc đoạn văn 1 trên và trả lời câu hỏi.
c) f(1)=0.$f(1) = 0.$

\textbf{{ANSWER}}

Sai vì f(1)=1

========================================================================

https://khoahoc.vietjack.com/thi-online/trac-nghiem-tong-hop-on-thi-tot-nghiep-thpt-mon-toan-chu-de-1-ham-so-va-ung-dung-co-dap-an/145218


\textbf{{QUESTION}}

Bạn hãy đọc đoạn văn 1 trên và trả lời câu hỏi.
d) Phương trình tiếp tuyến của đồ thị hàm số đã cho tại điểm có hoành độ x0=1${{\rm{x}}_0} = 1$ là y=2x−2.${\rm{y}} = 2{\rm{x}} - 2.$

\textbf{{ANSWER}}

Sai vì y = 2x-1

========================================================================

https://khoahoc.vietjack.com/thi-online/trac-nghiem-tong-hop-on-thi-tot-nghiep-thpt-mon-toan-chu-de-1-ham-so-va-ung-dung-co-dap-an/145218


\textbf{{QUESTION}}

Bạn hãy đọc đoạn văn 2 trên và trả lời câu hỏi.
a) f′(x)=−x2+1.${f^\prime }(x) =  - {x^2} + 1.$

\textbf{{ANSWER}}

Sai

========================================================================

https://khoahoc.vietjack.com/thi-online/10-bai-tap-xet-tinh-chisa-het-cua-mot-tong-hieu-cac-tich-va-cac-so-hang-co-loi-giai


\textbf{{QUESTION}}

Ta có tổng A = 75. 11 + 121. Vậy tổng A có chia hết cho 11 hay không?
A. A không chia hết cho 11;
B. A chia hết cho 11;
C. A chia cho 11 dư 4;
D. Không xác định.

\textbf{{ANSWER}}

Đáp án đúng là: B
Ta có: 11$ \vdots $11 nên (75. 11)$ \vdots $11
Lại có 121$ \vdots $11 
Do đó, tổng A = (75. 11 + 121)$ \vdots $11.

========================================================================

https://khoahoc.vietjack.com/thi-online/10-bai-tap-xet-tinh-chisa-het-cua-mot-tong-hieu-cac-tich-va-cac-so-hang-co-loi-giai


\textbf{{QUESTION}}

Cho A = 17. 9−$ - $82. Vậy A có chia hết cho 17 không?
A. A không chia hết cho 17;
B. A chia hết cho 17;
C. A chia cho 17 dư 11;
D. A chia cho 17 dư 10.

\textbf{{ANSWER}}

Đáp án đúng là: A
Ta có: 17$ \vdots $17 nên (17. 9)$ \vdots $17
Mà 82$\not \vdots $17  nên A = (17. 9$ - $82)$\not \vdots $17.

========================================================================

https://khoahoc.vietjack.com/thi-online/10-bai-tap-xet-tinh-chisa-het-cua-mot-tong-hieu-cac-tich-va-cac-so-hang-co-loi-giai


\textbf{{QUESTION}}

Số dư của phép chia 123. 12 + 15 cho 12 là:
A. 1;
B. 2;
C. 3;
D. 5.

\textbf{{ANSWER}}

Đáp án đúng là: C
Ta có: 12⋮$ \vdots $12 nên (123. 12)⋮$ \vdots $12 
Mà 15 chia cho 12 dư 3
Suy ra 123. 12 + 15 chia cho 12 dư 3.

========================================================================

https://khoahoc.vietjack.com/thi-online/10-bai-tap-xet-tinh-chisa-het-cua-mot-tong-hieu-cac-tich-va-cac-so-hang-co-loi-giai


\textbf{{QUESTION}}

A = 16. 58 + 32 chia hết cho những số nào trong các số 2; 4; 8; 13; 16?
A. A chia hết cho 2, cho 4, cho 8, cho 16;
B. A chia hết cho 2, cho 4, cho 13;
C. A chia hết cho 2, cho 8, cho 13, cho 16;
D. A chia hết cho 2, cho 6, cho 8.

\textbf{{ANSWER}}

Đáp án đúng là: A
Ta thấy 16 chia hết cho 2, cho 4, cho 8, cho 16 nên 16. 58 chia hết cho 2, cho 4, cho 8, cho 16
Lại có 32 chia hết cho 2, cho 4, cho 8, cho 16
Suy ra A = 16. 58 + 32 chia hết cho 2, cho 4, cho 8, cho 16.

========================================================================

https://khoahoc.vietjack.com/thi-online/10-bai-tap-xet-tinh-chisa-het-cua-mot-tong-hieu-cac-tich-va-cac-so-hang-co-loi-giai


\textbf{{QUESTION}}

A. Sai;
B. Đúng;
C. A chia 100 dư 15;
D. Không xác định.

\textbf{{ANSWER}}

Đáp án đúng là: B
A = 1.2.3.4.5….19.20 = 20.5.1.2.3.4.6.7….19 = 100.1.2.3.4.6.7….19
Ta có 100⋮$ \vdots $100 nên 100.1.2.3.4.6.7….19 ⋮$ \vdots $100
Vậy A = 1.2.3.4.5….19.20 chia hết cho 100.

========================================================================

https://khoahoc.vietjack.com/thi-online/10-cau-trac-nghiem-toan-6-canh-dieu-bai-5-goc-co-dap-an


\textbf{{QUESTION}}

Chọn câu sai.

\textbf{{ANSWER}}

Ta có:
 + Góc là hình gồm hai tia chung gốc nên A đúng
+ Góc bẹt là góc có hai cạnh là hai tia đối nhau nên B sai vì hai tia chung gốc chưa chắc đã đối nhau
+ Hai góc bằng nhau có số đo bằng nhau nên C đúng
+ Hai góc có số đo bằng nhau thì bằng nhau nên D đúng
Đáp án cần chọn là: B

========================================================================

https://khoahoc.vietjack.com/thi-online/10-cau-trac-nghiem-toan-6-canh-dieu-bai-5-goc-co-dap-an


\textbf{{QUESTION}}

Chọn câu sai.

\textbf{{ANSWER}}

Ta có góc vuông là góc có số đo bằng 900; Góc có số đo lớn hơn 00 và nhỏ hơn 900 là góc nhọn và góc tù là góc có số đo lớn hơn 900 và nhỏ hơn 1800 nên A, B, C đều đúng.
Góc có số đo nhỏ hơn 1800 là góc tù là sai vì góc nhọn, góc vuông đều có số đo nhỏ hơn 1800
Đáp án cần chọn là: D

========================================================================

https://khoahoc.vietjack.com/thi-online/10-cau-trac-nghiem-toan-6-canh-dieu-bai-5-goc-co-dap-an


\textbf{{QUESTION}}

Chọn phát biểu đúng.
B. Góc có số đo 800 là góc tù

\textbf{{ANSWER}}

+ Vì 900 < 1200 <1800  nên góc có số đo 1200 là góc tù, do đó A sai
+ Vì 00 < 800 < 900  nên góc có số đo 800 là góc  nhọn, do đó B sai
+ Vì 900 < 1000 < 1800  nên góc có số đo 1000 là góc tù, do đó C sai
+ Vì 900 < 1500 <1800 nên góc có số đo 1500 là góc tù, do đó D đúng
Đáp án cần chọn là: D

========================================================================

https://khoahoc.vietjack.com/thi-online/10-cau-trac-nghiem-toan-6-canh-dieu-bai-5-goc-co-dap-an


\textbf{{QUESTION}}

Cho 9 tia chung gốc (không có tia nào trùng nhau) thì số góc tạo thành là

\textbf{{ANSWER}}

Số góc tạo thành là 
9.(9−1)2=36$$ \frac{9.\left(9-1\right)}{2}=36$$ góc
Đáp án cần chọn là: C

========================================================================

https://khoahoc.vietjack.com/thi-online/20-cau-trac-nghiem-toan-10-canh-dieu-xac-suat-cua-bien-co-phan-2-co-dap-an/112297


\textbf{{QUESTION}}

Một hộp đựng 10 chiếc thẻ được đánh số từ 0 đến 9. Lấy ngẫu nhiên ra 3 chiếc thẻ, tính xác suất để 3 chữ số trên 3 chiếc thẻ được lấy ra có thể ghép thành một số chia hết cho 5.

\textbf{{ANSWER}}

Hướng dẫn giải
Đáp án đúng là: A
Không gian mẫu là số cách lấy ngẫu nhiên 3 chiếc thẻ từ 10 chiếc thẻ: 
n(Ω) = $C_{10}^3 = 120$
Gọi biến cố A: “3 chữ số trên 3 chiếc thẻ được lấy ra có thể ghép thành một số chia hết cho 5”
Để cho biến cố A xảy ra thì trong 3 thẻ lấy được phải có thẻ mang chữ số 0 hoặc chữ số 5. Ta đi tìm số phần tử của biến cố $\overline A $: “3 thẻ lấy ra không có thẻ mang chữ số 0 và cũng không có thẻ mang chữ số 5”.
Ta có: n($\overline A $) = $C_8^3 = 56$ 
Do đó, $P(\overline A ) = \frac{{n(\overline A )}}{{n(\Omega )}} = \frac{{56}}{{120}} = \frac{7}{{15}}$
Vậy P(A) = 1 – P($\overline A $) = $1 - \frac{7}{{15}} = \frac{8}{{15}}$.

========================================================================

https://khoahoc.vietjack.com/thi-online/20-cau-trac-nghiem-toan-10-canh-dieu-xac-suat-cua-bien-co-phan-2-co-dap-an/112297


\textbf{{QUESTION}}

Giải bóng chuyền VTV Cup gồm 9 đội bóng tham dự, trong đó có 6 đội nước ngoài và 3 đội của Việt Nam. Ban tổ chức cho bốc thăm ngẫu nhiên để chia thành 3 bảng A, B, C và mỗi bảng có 3 đội. Tính xác suất để 3 đội bóng của Việt Nam ở 3 bảng khác nhau.

\textbf{{ANSWER}}

Hướng dẫn giải
Đáp án đúng là: A
Không gian mẫu là số cách chia tùy ý 9 đội thành 3 bảng, ta có: n(Ω) = $C_9^3.C_6^3.C_3^3 = 1680$
Gọi biến cố A: “3 đội bóng của Việt Nam ở 3 bảng khác nhau” 
Xếp 3 đội Việt Nam ở 3 bảng khác nhau có 3! = 6 cách
Xếp 6 đội còn lại vào 3 bảng A, B, C này có $C_6^2.C_4^2.C_2^2 = 90$cách
Do đó, n(A) = 6 . 90 = 540. 
Vậy $P(A) = \frac{{n(A)}}{{n(\Omega )}} = \frac{{540}}{{1680}} = \frac{9}{{28}}$.

========================================================================

https://khoahoc.vietjack.com/thi-online/20-cau-trac-nghiem-toan-10-canh-dieu-xac-suat-cua-bien-co-phan-2-co-dap-an/112297


\textbf{{QUESTION}}

Trong thư viện có 12 quyển sách gồm 3 quyển Toán giống nhau, 3 quyển Lý giống nhau, 3 quyển Hóa giống nhau và 3 quyển Sinh giống nhau. Xác suất 3 quyển sách thuộc cùng 1 môn không được xếp liền nhau ?

\textbf{{ANSWER}}

Hướng dẫn giải
Đáp án đúng là: C
Ta có: n(Ω) = 12! 
Biến cố A: “3 quyển sách thuộc cùng 1 môn không được xếp liền nhau”
Xếp 3 cuốn sách Toán kề nhau. Xem 3 cuốn sách Toán là 3 vách ngăn, giữa 3 cuốn sách Toán có 2 vị trí trống và thêm hai vị trí hai đầu, tổng cộng có 4 vị trí trống.
Bước 1. Chọn 3 vị trí trống trong 4 vị trí để xếp 3 cuốn Lý, có C34=4$C_4^3 = 4$cách.
Bước 2. Giữa 6 cuốn Lý và Toán có 5 vị trí trống và thêm 2 vị trí hai đầu, tổng cộng có 7 vị trí trống. Chọn 3 vị trí trong 7 vị trí trống để xếp 3 cuốn Hóa, có C37=35$C_7^3 = 35$ cách.
Bước 3. Giữa 9 cuốn sách Toán, Lý và Hóa đã xếp có 8 vị trí trống và thêm 2 vị trí hai đầu, tổng cộng có 10 vị trí trống. Chọn 3 vị trí trong 10 vị trí trống để xếp 3 cuốn Sinh, có C310=120$C_{10}^3 = 120$ cách. Vậy theo quy tắc nhân có: 
4 . 35 . 120 = 16 800 cách.
Vậy P(A)=n(A)n(Ω)=1680012!=128512$P(A) = \frac{{n(A)}}{{n(\Omega )}} = \frac{{16800}}{{12!}} = \frac{1}{{28512}}$.

========================================================================

https://khoahoc.vietjack.com/thi-online/20-cau-trac-nghiem-toan-10-canh-dieu-xac-suat-cua-bien-co-phan-2-co-dap-an/112297


\textbf{{QUESTION}}

Cho tập hợp A = {2; 3; 4; 5; 6; 7; 8}. Gọi S là tập hợp các số tự nhiên có 4 chữ số đôi một khác nhau được lập thành từ các chữ số của tập A. Chọn ngẫu nhiên một số từ S, xác suất để số được chọn mà trong mỗi số luôn luôn có mặt hai chữ số chẵn và hai chữ số lẻ là:

\textbf{{ANSWER}}

Hướng dẫn giải
Đáp án đúng là: D
Số phần tử của tập S là: A47=840$A_7^4 = 840$
Không gian mẫu là chọn ngẫu nhiên 1 số từ tập S. Suy ra số phần tử của không gian mẫu là n(Ω) = 840
Gọi A là biến cố ” Số được chọn luôn luôn có mặt hai chữ số chẵn và hai chữ số lẻ ” .
Số cách chọn hai chữ số chẵn từ bốn chữ số 2; 4; 6; 8 là C24=6$C_4^2 = 6$cách.
Số cách chọn hai chữ số lẻ từ ba chữ số 3; 5; 7 là C23=3$C_3^2 = 3$ cách.
Từ bốn chữ số được chọn ta lập số có bốn chữ số khác nhau, số cách lập tương ứng với một hoán vị của 4 phần tử nên có 4! cách.
Ta có: n(A) = 6 . 3 . 4! = 432 
Vậy P(A)=n(A)n(Ω)=432840=1835$P(A) = \frac{{n(A)}}{{n(\Omega )}} = \frac{{432}}{{840}} = \frac{{18}}{{35}}$.

========================================================================

https://khoahoc.vietjack.com/thi-online/20-cau-trac-nghiem-toan-10-canh-dieu-xac-suat-cua-bien-co-phan-2-co-dap-an/112297


\textbf{{QUESTION}}

Trong giải cầu lông kỷ niệm ngày truyền thống học sinh sinh viên có 8 người tham gia trong đó có hai bạn Việt và Nam. Các vận động viên được chia làm hai bảng A và B, mỗi bảng gồm 4 người. Giả sử việc chia bảng thực hiện bằng cách bốc thăm ngẫu nhiên, tính xác suất để cả 2 bạn Việt và Nam nằm chung 1 bảng đấu.
A. 17$\frac{1}{7}$;
A. 
17$\frac{1}{7}$
17
17
17
1
1
7
7


7
7
7
;
B. 47$\frac{4}{7}$;
B. 
47$\frac{4}{7}$
47
47
47
4
4
7
7


7
7
7
;
C. 27$\frac{2}{7}$;
C. 
27$\frac{2}{7}$
27
27
27
2
2
7
7


7
7
7
;
D. 37$\frac{3}{7}$.
D. 
37$\frac{3}{7}$
37
37
37
3
3
7
7


7
7
7
.

\textbf{{ANSWER}}

Hướng dẫn giải
Đáp án đúng là: D
Không gian mẫu là số cách chia tùy ý 8 người thành 2 bảng.
Suy ra số phần tử của không gian mẫu là n(Ω) = C48.C44=70$C_8^4.C_4^4 = 70$
Gọi A là biến cố ” 2 bạn Việt và Nam nằm chung 1 bảng đấu
Bước 1. Xếp 2 bạn Việt và Nam nằm chung 1 bảng đấu nên có C12=2$C_2^1 = 2$cách.
Bước 2. Xếp 6 bạn còn lại vào 2 bảng cho đủ mỗi bảng là 4 bạn thì có C26.C44=15$C_6^2.C_4^4 = 15$ cách.
Ta có: n(A) = 2.15 = 30 
Vậy P(A)=n(A)n(Ω)=3070=37$P(A) = \frac{{n(A)}}{{n(\Omega )}} = \frac{{30}}{{70}} = \frac{3}{7}$.

========================================================================

https://khoahoc.vietjack.com/thi-online/10-bai-tap-tim-so-chua-sbiet-co-loi-giai


\textbf{{QUESTION}}

Có bao nhiêu cặp số (x; y) biết x, y là ước của 40 và x + y = 3?
A. 6;
B. 5;
C. 4;
D. 3.

\textbf{{ANSWER}}

Đáp án đúng là: A
Các ước dương của 40 là: 1; 2; 4; 5; 8; 10; 20; 40. Do đó các ước của 40 là: 1; -1; 2; -2; 4; -4; 5; -5; 8; -8; 10; -10; 20; -20; 40; -40.
3 = 1 + 2 = -1 + 4 = -2 + 5
Vậy (x; y) = {(1; 2), (2; 1), (-1; 4), (4; -1), (-2; 5), (5; -2)}

========================================================================

https://khoahoc.vietjack.com/thi-online/10-bai-tap-tim-so-chua-sbiet-co-loi-giai


\textbf{{QUESTION}}

Có bao nhiêu cặp số (x; y) biết x là ước của 10, y là bội dương nhỏ hơn 11 của 3 và x + y = 1?
A. 1;
B. 2;
C. 3;
D. 0.

\textbf{{ANSWER}}

Đáp án đúng là: C
Các ước dương của 10 là: 1; 2; 5; 10. Do đó các ước của 10 là: 1; -1; 2; -2; 5; -5; 10; -10 nên x thuộc {1; -1; 2; -2; 5; -5; 10; -10}
Lần lượt nhân 3 với 0; 1; 2; 3;… ta được các ước tự nhiên của 3 là: 0; 3; 6; 9; 12;…
Bội dương nhỏ hơn 11 của 3 là: 0; 3; 6; 9 nên y thuộc {0; 3; 6; 9}
1 = 1 + 0 = -2 + 3 = -5 + 6
Vậy (x; y) = {(1; 0), (-2; 3), (-5; 6)}
Vậy có 3 cặp (x; y) thỏa mãn.

========================================================================

https://khoahoc.vietjack.com/thi-online/10-bai-tap-tim-so-chua-sbiet-co-loi-giai


\textbf{{QUESTION}}

Có bao nhiêu nguyên thỏa mãn 2 lần số đó không lớn hơn 200 và số đó là bội không âm của 50?
A. 1;
B. 3;
C. 4;
D. 2.

\textbf{{ANSWER}}

Đáp án đúng là: B
Lần lượt nhân 50 với 0; 1; 2; 3; 4;… ta được các bội tự nhiên của 50 là: 0; 50; 100; 150;…
2.0 = 0 < 200
2.50 = 100 < 200
2.100 = 200 = 200
2.150 = 300 > 200
…
Vậy có 3 số nguyên thỏa mãn

========================================================================

https://khoahoc.vietjack.com/thi-online/10-bai-tap-tim-so-chua-sbiet-co-loi-giai


\textbf{{QUESTION}}

Có bao nhiêu số nguyên thỏa mãn 3 lần số đó không lớn hơn 20 và số đó là ước của 20?
A. 7;
B. 8;
C. 9;
D. 10.

\textbf{{ANSWER}}

Đáp án đúng là: D
Các ước dương của 20 là: 1; 2; 4; 5; 10; 20. Do đó các ước của 20 là: 1; -1; 2; -2; 4; -4; 5; -5; 10; -10; 20; -20
Ta có:
3.(-20) = -60 < 20
3.(-10) = -30 < 20
3.(-5) = -15 < 20
3.(-4) = -12 < 20
3.(-2) = -6 < 20
3.(-1) = -3 < 20
3.1 = 3 < 20
3.2 = 6 < 20
3.4 = 12 < 20
3.5 = 15 < 20
3.10 = 30 > 20
3.20 = 60 > 20
Vậy có 10 số nguyên thỏa mãn

========================================================================

https://khoahoc.vietjack.com/thi-online/10-bai-tap-tim-so-chua-sbiet-co-loi-giai


\textbf{{QUESTION}}

Tập hợp ước chung lớn hơn -2 của 36 và 30:
A. {-1; 1; 2; 3};
B. {-1; 1; 2; 3; 6};
C. {1; 2; 3; 6};
D. {-1; 1; 2; 3; 6; 12}.

\textbf{{ANSWER}}

Đáp án đúng là: B
Các ước dương của 36 là: 1; 2; 3; 4; 6; 9; 12; 18; 36. Do đó các ước của 36 là: 1; -1; 2; -2; 3; -3; 4; -4; 6; -6; 9; -9; 12; -12; 18; -18; 36; -36.
Các ước dương của 30 là: 1; 2; 3; 5; 6; 10; 15; 30. Do đó các ước của 30 là: 1; -1; 2; -2; 3; -3; 5; -5; 6; -6; 10; -10; 15; -15; 30; -30.
Các ước chung của 36 và 30 là: 1; -1; 2; -2; 3; -3; 6; -6.
Tập hợp các ước chung lớn hơn -2 của 36 và 30: {-1; 1; 2; 3; 6}

========================================================================

https://khoahoc.vietjack.com/thi-online/15-cau-trac-nghiem-phuong-trinh-bac-hai-voi-he-so-thuc-co-dap-an-nhan-biet


\textbf{{QUESTION}}

Số phức w là căn bậc hai của số phức z nếu:
A. $$ {z}^{2}=w$$.
B. $$ {w}^{2}=z$$.
C. $$ \sqrt{w}=z$$.
D. $$ z=\pm \sqrt{w}$$.

\textbf{{ANSWER}}

Đáp án cần chọn là: B
Số phức $$ w=x+yi(x,\quad y\quad \in R)$$ là căn bậc hai của số phức $$ z=a+bi$$ nếu $$ {w}^{2}=z$$.

========================================================================

https://khoahoc.vietjack.com/thi-online/15-cau-trac-nghiem-phuong-trinh-bac-hai-voi-he-so-thuc-co-dap-an-nhan-biet


\textbf{{QUESTION}}

Cho z=1-3i$$ z=1-3i$$ là một căn bậc hai của w=-8-6i$$ w=-8-6i$$. Chọn kết luận đúng:
A. (1-3i)2=-8-6i$$ {\left(1-3i\right)}^{2}=-8-6i$$.
B. (1-3i)2=-8+6i$$ {\left(1-3i\right)}^{2}=-8+6i$$.
C. (1-3i)2=8+6i$$ {\left(1-3i\right)}^{2}=8+6i$$.
D. (-8-6i)2=1-3i$$ {\left(-8-6i\right)}^{2}=1-3i$$.

\textbf{{ANSWER}}

Đáp án cần chọn là: A.
Do $$ z=1-3i$$ là một căn bậc hai của $$ w=-8-6i$$ nên $$ {\left(1-3i\right)}^{2}=-8-6i$$.

========================================================================

https://khoahoc.vietjack.com/thi-online/15-cau-trac-nghiem-phuong-trinh-bac-hai-voi-he-so-thuc-co-dap-an-nhan-biet


\textbf{{QUESTION}}

Các căn bậc hai của một số phức khác 0 là:
A. Hai số phức liên hợp.
B. Hai số phức bằng nhau.
C. Hai số phức có cùng phần ảo.
D. Hai số phức đối nhau.

\textbf{{ANSWER}}

Đáp án cần chọn là: D
Các căn bậc hai của một số phức khác 0 là hai số đối nhau.

========================================================================

https://khoahoc.vietjack.com/thi-online/15-cau-trac-nghiem-phuong-trinh-bac-hai-voi-he-so-thuc-co-dap-an-nhan-biet


\textbf{{QUESTION}}

Căn bậc hai của số a=-3$$ a=-3$$ là:
A. 3i$$ 3i$$ và -3i$$ -3i$$.
B. 3√i$$ 3\sqrt{i}$$ và -3√i$$ -3\sqrt{i}$$.
C. i√3$$ i\sqrt{3}$$ và -i√3$$ -i\sqrt{3}$$.
D. √3i$$ \sqrt{3i}$$ và -√3i$$ -\sqrt{3i}$$.

\textbf{{ANSWER}}

Đáp án cần chọn là: C
Căn bậc hai của số a=-3$$ a=-3$$ là i√3$$ i\sqrt{3}$$ và -i√3$$ -i\sqrt{3}$$.

========================================================================

https://khoahoc.vietjack.com/thi-online/15-cau-trac-nghiem-phuong-trinh-bac-hai-voi-he-so-thuc-co-dap-an-nhan-biet


\textbf{{QUESTION}}

Số nghiệm thực của phương trình (z2+1)(z2-1)=0$$ \left({z}^{2}+1\right)\left({z}^{2}-1\right)=0$$ là:
A. 0.
B. 1.
C. 2.
D. 4.

\textbf{{ANSWER}}

Đáp án cần chọn là: A
Có z2+1≠0, ∀z∈R$$ {z}^{2}+1\ne 0,\quad \forall z\in R$$ và z2-i≠0, ∀z∈R$$ {z}^{2}-i\ne 0,\quad \forall z\in R$$
Vậy phương trình đã cho không có nghiệm thực.

========================================================================

https://khoahoc.vietjack.com/thi-online/16-cau-trac-nghiem-toan-6-ket-noi-tri-thuc-bai-tap-cuoi-chuong-9-co-dap-an-phan-2


\textbf{{QUESTION}}

Tung đồng xu 15 lần liên tiếp và kết quả thu được ghi lại trong bảng sau:
Lần tung
Kết quả
Lần tung
Kết quả
Lần tung
Kết quả
1
S
6
N
11
N
2
S
7
S
12
S
3
N
8
S
13
N
4
S
9
N
14
N
5
N
10
N
15
N
N: Ngửa
S: Sấp
Số lần xuất hiện mặt ngửa (N) là

\textbf{{ANSWER}}

Trả lời:
Số lần xuất hiện mặt ngửa là 9 lần.
Đáp án cần chọn là: D

========================================================================

https://khoahoc.vietjack.com/thi-online/16-cau-trac-nghiem-toan-6-ket-noi-tri-thuc-bai-tap-cuoi-chuong-9-co-dap-an-phan-2


\textbf{{QUESTION}}

Tung đồng xu 15 lần liên tiếp và kết quả thu được ghi lại trong bảng sau:
Lần tung
Kết quả
Lần tung
Kết quả
Lần tung
Kết quả
1
S
6
N
11
N
2
S
7
S
12
S
3
N
8
S
13
N
4
S
9
N
14
N
5
N
10
N
15
N
N: Ngửa
S: Sấp
Xác suất thực nghiệm xuất hiện mặt ngửa là

\textbf{{ANSWER}}

Trả lời:
Tổng số lần tung là 15 lần
Số lần xuất hiện mặt N là 9 lần.
Xác suất thực nghiệm xuất hiện mặt ngửa là $\frac{9}{{15}} = \frac{3}{5} = 0,6$
Đáp án cần chọn là: B

========================================================================

https://khoahoc.vietjack.com/thi-online/16-cau-trac-nghiem-toan-6-ket-noi-tri-thuc-bai-tap-cuoi-chuong-9-co-dap-an-phan-2


\textbf{{QUESTION}}

Tung đồng xu 15 lần liên tiếp và kết quả thu được ghi lại trong bảng sau:
Lần tung
Kết quả
Lần tung
Kết quả
Lần tung
Kết quả
1
S
6
N
11
N
2
S
7
S
12
S
3
N
8
S
13
N
4
S
9
N
14
N
5
N
10
N
15
N
N: Ngửa
S: Sấp
Xác suất thực nghiệm xuất hiện mặt S là

\textbf{{ANSWER}}

Trả lời:
Tổng số lần tung là 15 lần
Số lần xuất hiện mặt S là 15-9=6 lần.
Xác suất thực nghiệm xuất hiện mặt ngửa là 615=25=0,4$\frac{6}{{15}} = \frac{2}{5} = 0,4$
Đáp án cần chọn là: C

========================================================================

https://khoahoc.vietjack.com/thi-online/16-cau-trac-nghiem-toan-6-ket-noi-tri-thuc-bai-tap-cuoi-chuong-9-co-dap-an-phan-2


\textbf{{QUESTION}}

Một hộp có 5 chiếc thẻ cùng loại, mỗi thẻ được ghi một trong các số 1, 2, 3, 4, 5; hai thẻ khác nhau thì ghi số khác nhau. Rút ngẫu nhiên một thẻ.
Viết tập hợp các kết quả có thể xảy ra đối với số xuất hiện trên thẻ được rút ra.

\textbf{{ANSWER}}

Trả lời:
Số có thể xuất hiện trên thẻ là một trong năm số: 1; 2; 3; 4; 5.
Tập hợp các kết quả có thể xảy ra đối với số xuất hiện trên thẻ là
M = {1; 2; 3; 4; 5}.
Đáp án cần chọn là: D

========================================================================

https://khoahoc.vietjack.com/thi-online/16-cau-trac-nghiem-toan-6-ket-noi-tri-thuc-bai-tap-cuoi-chuong-9-co-dap-an-phan-2


\textbf{{QUESTION}}

Một hộp có 5 chiếc thẻ cùng loại, mỗi thẻ được ghi một trong các số 1, 2, 3, 4, 5; hai thẻ khác nhau thì ghi số khác nhau. Rút ngẫu nhiên một thẻ.
Số xuất hiện trên thẻ được rút có phải là phần tử của tập hợp {1; 2; 3; 4; 5} hay không?
Không
Có

\textbf{{ANSWER}}

Trả lời:
Số có thể xuất hiện trên thẻ là một trong năm số: 1;2;3;4;5.
Các số này đều là phần tử của tập hợp {1;2;3;4;5}.

========================================================================

https://khoahoc.vietjack.com/thi-online/luyen-tap-ve-phep-cong-va-phep-nhan


\textbf{{QUESTION}}

Tính nhẩm bằng cách:
a) áp dụng tính chất kết hợp của phép nhân: 125.16; 25.28
b) áp dụng tính chất phân phối của phép nhân với phép cộng: 13.12; 53.11; 39.101.
c) áp dụng tính chất a(b – c) = ab – ac ; 8.19; 65.98

\textbf{{ANSWER}}

a) 125.16 = 125.8.2 = 1000.2 = 2000
25.28 = 25.4.7 = 100.7 = 700
b) 13.12 = 13.(10+2) = 13.10 + 13.2 = 130+26 = 156
53.11 = 53.(10+1) = 53.10+53.1 = 530+53 = 583
39.101 = 39.(100 + 1) = 39.100 + 39.1 = 3900 + 39 = 3939
c) 8.19 = 8.(20 – 1) = 8.20 – 8.1 = 160 – 8 = 152
65.98 = 65.(100 – 2) = 6500 – 130 = 6370

========================================================================

https://khoahoc.vietjack.com/thi-online/luyen-tap-ve-phep-cong-va-phep-nhan


\textbf{{QUESTION}}

Tính nhanh:
a) 81 + 243 + 19
b) 134 + 237 + 166 + 563
c) 15.7.20
d) 8.12.125.4

\textbf{{ANSWER}}

a) 81 + 243 +19 = (81 + 19) + 243 = 100 + 243 = 343
b) 134 + 237 + 166 + 563 = (134 + 166) + (237 + 563) = 300 + 800 = 1100
c) 15.7.20 = (15.20).7 = 300.7 = 2100
d) 8.12.125.4 = (8.125).(12.4) = 1000.48 = 48000

========================================================================

https://khoahoc.vietjack.com/thi-online/luyen-tap-ve-phep-cong-va-phep-nhan


\textbf{{QUESTION}}

Tính nhanh:
a) 34.15 + 34.85
b) 12.57 + 57.25 + 63.57
c) 36.28 + 36.82 + 64.69 + 64.41
d) 2.31.12 + 4.6.42 + 8.27.3

\textbf{{ANSWER}}

a) 34.15 + 34.85 = 34.(15+85) = 34.100 = 3400
b) 12.57 + 57.25 + 63.57 = 57.(12+25+63) = 57.100 = 5700
c) 36.28 + 36.82 + 64.69 + 64.41 = 36.(28+82) + 64(69+41) = 36.110 + 64.110 = 110.(36+64) = 11000
d) 2.31.12 + 4.6.42 + 8.27.3 = 24.31 + 24.42 + 24.27 = 24.(31+42+27) = 24.100 = 2400

========================================================================

https://khoahoc.vietjack.com/thi-online/luyen-tap-ve-phep-cong-va-phep-nhan


\textbf{{QUESTION}}

Tính nhanh:
a) 26 + 27 + 28 + 29 + 30 + 31 + 32 + 33
b) 2 + 4 + 6 + 8 + … + 1998 + 2000
c) 5 + 9 + 13 + … + 1997 + 2001

\textbf{{ANSWER}}

a) 26 + 27 +28 +29 +30 +31 +32 +33 = (26+33) + (27+32) + (28+31) + (29+30)
= 59 +59 +59 +59 = 59.4 = 236
b) 2 + 4 + 6 + 8 + … + 1998 + 2000
Số các số hạng là: (2000 – 2):2 + 1 = 1000 (số hạng)
Tổng có giá trị là: (2000 + 2).1000 : 2 = 1001000
c) 5 + 9 +13 +… + 1997 + 2001
Số các số hạng là: (2001 – 5) : 5 +1 = 500 (số hạng)
Tổng có giá trị là: (5 + 2001).500 : 2 = 501500

========================================================================

https://khoahoc.vietjack.com/thi-online/bai-tap-tim-gia-tri-phan-so-cua-mot-so-cho-truoc


\textbf{{QUESTION}}

Tìm:
a) $$ \frac{2}{3}$$của 81                                
b) $$ \frac{2}{7}$$ của - 4
c) $$ \frac{3}{4}$$ của 1,6                             
d) 21% của 5,6

\textbf{{ANSWER}}

a) Áp dụng công thức ta có $$ \frac{2}{3}$$. 81 = 54
b)$$ \frac{-8}{7}$$            
c) 1,2                             
 d) 1,176

========================================================================

https://khoahoc.vietjack.com/thi-online/bai-tap-tim-gia-tri-phan-so-cua-mot-so-cho-truoc


\textbf{{QUESTION}}

Tìm:
a) $$ \frac{1}{4}$$của 24                                
b) $$ \frac{2}{7}$$ của - 6
c) $$ \frac{3}{4}$$ của 2,5                                       
d) 25% của 4,8

\textbf{{ANSWER}}

a) 6.             
b)$$ \frac{-12}{7}$$                   
c) 1, 5                   
 d) 1,2

========================================================================

https://khoahoc.vietjack.com/thi-online/bai-tap-tim-gia-tri-phan-so-cua-mot-so-cho-truoc


\textbf{{QUESTION}}

Tìm:
a) 123$$ 1\frac{2}{3}$$của 8,1                             
b) 135$$ 1\frac{3}{5}$$của – 4,5
c) 75% của 125$$ 1\frac{2}{5}$$                         
d) 138$$ 1\frac{3}{8}$$ của 2111$$ 2\frac{1}{11}$$

\textbf{{ANSWER}}

a) Áp dụng công thức ta có 123$$ \frac{2}{3}$$.8,1 = 53$$ \frac{5}{3}$$.8,1 = 13.5
b) -7,2                  c) 1 120$$ \frac{1}{20}$$.                d) 238$$ \frac{23}{8}$$

========================================================================

https://khoahoc.vietjack.com/thi-online/bai-tap-tim-gia-tri-phan-so-cua-mot-so-cho-truoc


\textbf{{QUESTION}}

Tìm:
a) 113$$ 1\frac{1}{3}$$của 8,1                             
b) 125$$ 1\frac{2}{5}$$của – 2,5
c) 50% của 125$$ 1\frac{2}{5}$$                         
d) 135$$ 1\frac{3}{5}$$ của 218$$ 2\frac{1}{8}$$

\textbf{{ANSWER}}

a) 10, 8                
b) – 3,5
c) 710$$ \frac{7}{10}$$ 
d) 175$$ \frac{17}{5}$$

========================================================================

https://khoahoc.vietjack.com/thi-online/bai-tap-tim-gia-tri-phan-so-cua-mot-so-cho-truoc


\textbf{{QUESTION}}

Hãy so sánh 24% của 25 và 25% của 24. Dựa vào nhận xét đó hãy tính nhanh:
a) 72% của 25; 
b) 46% của 50.

\textbf{{ANSWER}}

Vì 24% .25 = 25% . 24 = 6,
Nhận xét: muốn tính 24% của 25 ta chỉ cấn tính 25% của 24. Mà 25% = 14$$ \frac{1}{4}$$ nên ta thực hiện 24 chia 4.
Chú ý: 25% = 14$$ \frac{1}{4}$$; 50% = 12$$ \frac{1}{2}$$; 20 % =15$$ \frac{1}{5}$$
a) Ta tính 25% của 72 bằng cách lấy 72 chia 4 được 18.
b) Tính 50% của 46 được 23.

========================================================================

https://khoahoc.vietjack.com/thi-online/tong-hop-bai-tap-toan-8-chuong-4-bat-phuong-trinh-bac-nhat-mot-an/50678


\textbf{{QUESTION}}

Trong các khẳng định sau đây, khẳng định nào sai?
( 1 )    ( - 4 ).5 ≤ ( - 5 ).4
( 2 )     ( - 7 ).12 ≥ ( - 7 ).11
( 3 )    $$ -4{x}^{2}>0$$
A. ( 1 ),( 2 ) và ( 3 )   
B. ( 1 ),( 2 ) 
C. ( 1 )   
D. ( 2 ),( 3 )

\textbf{{ANSWER}}

Ta có: $$ (-4).5=4.(-5)$$ nên khẳng định (1) đúng.
Vì $$ 12>11\Rightarrow 12.\left(-7\right)<11.\left(-7\right)$$ nên khẳng định (2) sai.
Vì $$ {x}^{2}\ge 0\Rightarrow -4{x}^{2}\le 0$$ nên khẳng định (3) sai.
Chọn đáp án D.

========================================================================

https://khoahoc.vietjack.com/thi-online/tong-hop-bai-tap-toan-8-chuong-4-bat-phuong-trinh-bac-nhat-mot-an/50678


\textbf{{QUESTION}}

Cho a + 1 ≤ b + 2. So sánh hai số 2a + 2 và 2b + 4 nào dưới đây đúng ?
A. 2a + 2 > 2b + 4 
B. 2a + 2 < 2b + 4 
C. 2a + 2 ≤ 2b + 4 
D. 2a + 2 ≥ 2b + 4

\textbf{{ANSWER}}

Với ba số a, b và c mà c > 0, ta có: Nếu a ≤ b thì ac ≤ bc
Khi đó, ta có: a + 1 ≤ b + 2 ⇒ 2( a + 1 ) ≤ 2( b + 2 ) ⇔ 2a + 2 ≤ 2b + 4.
Chọn đáp án C.

========================================================================

https://khoahoc.vietjack.com/thi-online/tong-hop-bai-tap-toan-8-chuong-4-bat-phuong-trinh-bac-nhat-mot-an/50678


\textbf{{QUESTION}}

Cho a > b. Khẳng định nào sau đây đúng?
A. - 3a - 1 > - 3b - 1 
B. - 3( a - 1 ) < - 3( b - 1 ) 
C. - 3( a - 1 ) > - 3( b - 1 ) 
D. 3( a - 1 ) < 3( b - 1 )

\textbf{{ANSWER}}

+ Ta có: a > b ⇒ - 3a < - 3b ⇔ - 3a - 1 < - 3b - 1
→ Đáp án A sai.
+ Ta có: a > b ⇒ a - 1 > b - 1 ⇔ - 3( a - 1 ) < - 3( b - 1 )
→ Đáp án B đúng.
+ Ta có: a > b ⇒ a - 1 > b - 1 ⇔ - 3( a - 1 ) < - 3( b - 1 )
→ Đáp án C sai.
+ Ta có: a > b ⇒ a - 1 > b - 1 ⇔ 3( a - 1 ) > 3( b - 1 )
→ Đáp án D sai.
Chọn đáp án B.

========================================================================

https://khoahoc.vietjack.com/thi-online/tong-hop-bai-tap-toan-8-chuong-4-bat-phuong-trinh-bac-nhat-mot-an/50678


\textbf{{QUESTION}}

Cho a ≥ b. Khẳng định nào sau đây đúng?
A. 2a - 5 ≤ 2( b - 1 ) 
B. 2a - 5 ≥ 2( b - 1 ) 
C. 2a - 5 ≥ 2( b - 3 ) 
D. 2a - 5 ≤ 2( b - 3 )

\textbf{{ANSWER}}

Ta có: a ≥ b ⇒ 2a ≥ 2b
Mặt khác, ta có: - 5 ≥ - 6
Khi đó 2a - 5 ≥ 2b - 6 hay 2a - 5 ≥ 2( b - 3 ).
Chọn đáp án C.

========================================================================

https://khoahoc.vietjack.com/thi-online/bai-tap-toan-8-chu-de-2-phuong-trinh-tich-co-dap-an


\textbf{{QUESTION}}

Giải phương trình: y (y-16) -297 = 0

\textbf{{ANSWER}}

y (y-16) -297 = 0
$$ \Leftrightarrow $$ y2-16y - 297 = 0

$$ \Rightarrow $$ y =27 và y = -11

========================================================================

https://khoahoc.vietjack.com/thi-online/bai-tap-toan-8-chu-de-2-phuong-trinh-tich-co-dap-an


\textbf{{QUESTION}}

Giải phương trình: 
(2x-3) (4-x) (x+3) = 0

\textbf{{ANSWER}}

(2x-3) (4-x) (x+3) = 0
$$ \Rightarrow $$ x = $$ \frac{3}{2}$$hoặc x = 4 hoặc x = -3

========================================================================

https://khoahoc.vietjack.com/thi-online/bai-tap-toan-8-chu-de-2-phuong-trinh-tich-co-dap-an


\textbf{{QUESTION}}

Giải phương trình: (4x2 - 9 ) (x2 - 25) = 0

\textbf{{ANSWER}}

(4x2 - 9 ) (x2 - 25) = 0
⇒ x=±32 hoặc x=±5
⇒ x=±32 hoặc x=±5
⇒
 
x
=
±
32
3
3
2
2


2
2
2
 
h
o
ặ
c
 
x
=
±
5

========================================================================

https://khoahoc.vietjack.com/thi-online/bai-tap-toan-8-chu-de-2-phuong-trinh-tich-co-dap-an


\textbf{{QUESTION}}

Giải các phương trình sau:
a) 0,5x ( x - 3)  = ( x -3 ) (2,5x - 4)

\textbf{{ANSWER}}

0,5x ( x - 3)  = ( x -3 ) (2,5x - 4)
⇒$$ \Rightarrow $$0,5x ( x - 3) - 
⇒
⇒
⇒
⇒
⇒
⇒
⇒
⇒
⇒
⇒
⇒
⇒

========================================================================

https://khoahoc.vietjack.com/thi-online/bai-tap-toan-8-chu-de-2-phuong-trinh-tich-co-dap-an


\textbf{{QUESTION}}

b) 37x - 1 = 17x (3x - 7 )$$ \frac{3}{7}x\quad -\quad 1\quad =\quad \frac{1}{7}x\quad (3x\quad -\quad 7\quad )$$

\textbf{{ANSWER}}

37x - 1 = 17x (3x - 7 )
37x - 1 = 17x (3x - 7 )
37
3
3
7
7


7
7
7
x
 
-
 
1
 
=
 
17
1
1
7
7


7
7
7
x
 
(
(
3
x
 
-
 
7
 
)
)
⇒ 17x (3x - 7) -17(3x-7)=0
⇒ 17x (3x - 7) -17(3x-7)=0
⇒
 
17
1
1
7
7


7
7
7
x
 
(
(
3
x
 
-
 
7
)
)
 
-
17
1
1
7
7


7
7
7
(
(
3
x
-
7
)
)
=
0
⇒ (17x -17)(3x-7) = 0
⇒ (17x -17)(3x-7) = 0
⇒
 
(
(
17
1
1
7
7


7
7
7
x
 
-
17
1
1
7
7


7
7
7
)
)
(
(
3
x
-
7
)
)
 
=
 
0
⇒
⇒
⇒
x = 7 hoặc x = 73
x = 7 hoặc x = 73
x
 
=
 
7
 
h
o
ặ
c
 
x
 
=
 
73
7
7
3
3


3
3
3

========================================================================

https://khoahoc.vietjack.com/thi-online/trac-nghiem-chuyen-de-toan-9-chuyen-de-5-cac-bai-toan-thuc-te-giai-bang-cach-lap-phuong-trinh-va-he


\textbf{{QUESTION}}

Tìm hai số biết tổng của chúng bằng 9 và tổng các nghịch đảo của chúng bằng $\frac{9}{{14}}$

\textbf{{ANSWER}}

Gọi hai số cần tìm là x, y. Điều kiện: $x,y \ne 0.$ 
Tổng của hai số đó là: $x + y = 9.$                                 (1)
Tổng các nghịch đảo của chúng là: $\frac{1}{x} + \frac{1}{y} = \frac{9}{{14}}$         (2)
Ta có hệ phương trình: $\left\{ \begin{array}{l}x + y = 9\\\frac{1}{x} + \frac{1}{y} = \frac{9}{{14}}\end{array} \right.$ 
Từ (1) có: $y = 9 - x,$ thay vào (2) ta được:
$\frac{1}{x} + \frac{1}{{9 - x}} = \frac{9}{{14}} \Leftrightarrow 14\left( {9 - x} \right) + 14x = 9x\left( {9 - x} \right) \Leftrightarrow 9{x^2} - 81x + 126 = 0 \Leftrightarrow \left[ \begin{array}{l}x = 2\\x = 7\end{array} \right.$ (thỏa mãn)
Với $x = 2$ thì $y = 7.$ 
Với $x = 7$ thì $y = 2.$ 
Vậy hai số cần tìm là 2 và 7.

========================================================================

https://khoahoc.vietjack.com/thi-online/trac-nghiem-chuyen-de-toan-9-chuyen-de-5-cac-bai-toan-thuc-te-giai-bang-cach-lap-phuong-trinh-va-he


\textbf{{QUESTION}}

Tìm một số có hai chữ số, biết rằng chữ số hàng chục lớn hơn chữ số hàng đơn vị là 5 và nếu đem số đó chia cho tổng các chữ số của nó thì được thương là 7 và dư là 6.

\textbf{{ANSWER}}

Gọi số có hai chữ số cần tìm là $\overline {ab} $ với $a,b \in \left\{ {0,1,2,3,4,5,6,7,8,9} \right\},{\rm{ }}a \ne 0.$ 
Vì chữ số hàng chục lớn hơn chữ số hàng đơn vị là 5 nên $a - b = 5.$ (1)
Nếu đem số đó chia cho tổng các chữ số của nó thì được thương là 7 và dư là 6. 
Do đó: $\overline {ab} = 7\left( {a + b} \right) + 6.$ (2)
Từ (1) và (2) ta có hệ phương trình:
$\left\{ \begin{array}{l}a - b = 5\\\overline {ab} = 7\left( {a + b} \right) + 6\end{array} \right.\;\; \Leftrightarrow \left\{ \begin{array}{l}a - b = 5\\10a + b = 7\left( {a + b} \right) + 6\end{array} \right. \Leftrightarrow \left\{ \begin{array}{l}a - b = 5\\3a - 6b = 6\end{array} \right. \Leftrightarrow \left\{ \begin{array}{l}a - b = 5\\a - 2b = 2\end{array} \right. \Leftrightarrow \left\{ \begin{array}{l}q = 8\\b = 3\end{array} \right.$ 
Vậy số cần tìm là 83.

========================================================================

https://khoahoc.vietjack.com/thi-online/trac-nghiem-chuyen-de-toan-9-chuyen-de-5-cac-bai-toan-thuc-te-giai-bang-cach-lap-phuong-trinh-va-he


\textbf{{QUESTION}}

Cho một số có hai chữ số. Nếu đổi chỗ hai chữ số của nó thì được một số lớn hơn số đã cho là 63. Tổng của số đã cho và số mới tạo thành bằng 99. Tìm số đã cho.

\textbf{{ANSWER}}

Gọi số có hai chữ số cần tìm là ¯ab.$\overline {ab} .$ Điều kiện: a,b∈N∗;a,b≤9.$a,b \in \mathbb{N}*;{\rm{ }}a,{\rm{ }}b \le 9.$ 
Nếu đổi chỗ hai chữ số đã cho thì ta được số mới là ¯ba.$\overline {ba} .$ 
Theo đề bài ta có hệ phương trình: {¯ba−¯ab=63¯ba+¯ab=99⇔{¯ab=18¯ba=81$\left\{ \begin{array}{l}\overline {ba} - \overline {ab} = 63\\\overline {ba} + \overline {ab} = 99\end{array} \right. \Leftrightarrow \left\{ \begin{array}{l}\overline {ab} = 18\\\overline {ba} = 81\end{array} \right.$ 
Vậy số cần tìm là 18.

========================================================================

https://khoahoc.vietjack.com/thi-online/giai-sbt-toan-10-bai-19-phuong-trinh-duong-thang-co-dap-an


\textbf{{QUESTION}}

Trong mặt phẳng Oxy, cho điểm D(0; 2) và hai vectơ $\overrightarrow n = \left( {1; - 3} \right),\overrightarrow u = \left( {1;3} \right)$.
Viết phương trình tổng quát của đường thẳng d đi qua D và nhận $\overrightarrow n $ là một vectơ pháp tuyến.

\textbf{{ANSWER}}

Hướng dẫn giải
Phương trình tổng quát của đường thẳng d đi qua D và nhận $\overrightarrow n $ là một vectơ pháp tuyến là:
1(x – 0) – 3(y – 2) = 0 
⇔ x – 3y + 6 = 0
Vậy d: x – 3y + 6 = 0.

========================================================================

https://khoahoc.vietjack.com/thi-online/giai-sbt-toan-10-bai-19-phuong-trinh-duong-thang-co-dap-an


\textbf{{QUESTION}}

Viết phương trình tham số của đường thẳng Δ đi qua D và nhận →u$\overrightarrow u $ là một vectơ chỉ phương.

\textbf{{ANSWER}}

Hướng dẫn giải
Phương trình tham số của đường thẳng ∆ đi qua D và nhận →u$\overrightarrow u $ là một vectơ chỉ phương là: 
{x=0+1.ty=2+3.t⇔{x=ty=2+3t$\left\{ \begin{array}{l}x = 0 + 1.t\\y = 2 + 3.t\end{array} \right. \Leftrightarrow \left\{ \begin{array}{l}x = t\\y = 2 + 3t\end{array} \right.$ (với t là tham số)
Vậy ∆: {x=ty=2+3t$\left\{ \begin{array}{l}x = t\\y = 2 + 3t\end{array} \right.$.

========================================================================

https://khoahoc.vietjack.com/thi-online/giai-sbt-toan-10-bai-19-phuong-trinh-duong-thang-co-dap-an


\textbf{{QUESTION}}

Trong mặt phẳng Oxy, cho ba điểm A(1; 2), B(0; –1) và C(–2; 3). Lập phương trình tổng quát của đường thẳng qua A và vuông góc với đường thẳng BC.

\textbf{{ANSWER}}

Hướng dẫn giải
Đường thẳng d qua A và vuông góc với đường thẳng BC nhận vectơ →BC$\overrightarrow {BC} $ làm vectơ pháp tuyến.
→BC$\overrightarrow {BC} $ = (–2 – 0; 3 + 1) = (–2; 4)
Phương trình của đường thẳng d là:
–2(x – 1) + 4(y – 2) = 0 
⇔ –2x + 2 + 4y – 8 = 0 
⇔ –2x + 4y – 6 = 0 
⇔ x – 2y + 3 = 0
Vậy d: x – 2y + 3 = 0.

========================================================================

https://khoahoc.vietjack.com/thi-online/giai-sbt-toan-10-bai-19-phuong-trinh-duong-thang-co-dap-an


\textbf{{QUESTION}}

Trong mặt phẳng Oxy, cho hai điểm A(1; 2) và B(2; 3). Tìm một vectơ chỉ phương của đường thẳng AB và viết phương trình tham số của đường thẳng AB.

\textbf{{ANSWER}}

Hướng dẫn giải
Một vectơ chỉ phương của đường thẳng AB chính là vectơ →AB$\overrightarrow {AB} $.
Ta có: →AB$\overrightarrow {AB} $ = (1; 1) 
Đường thẳng AB đi qua điểm A(1; 2) có vectơ chỉ phương →AB$\overrightarrow {AB} $ = (1; 1) có phương trình tham số là: {x=1+1.ty=2+1.t⇔{x=1+ty=2+t$\left\{ \begin{array}{l}x = 1 + 1.t\\y = 2 + 1.t\end{array} \right. \Leftrightarrow \left\{ \begin{array}{l}x = 1 + t\\y = 2 + t\end{array} \right.$.

========================================================================

https://khoahoc.vietjack.com/thi-online/giai-sbt-toan-10-bai-19-phuong-trinh-duong-thang-co-dap-an


\textbf{{QUESTION}}

Trong mặt phẳng Oxy, cho đường thẳng ∆: 2x – y + 5 = 0. Tìm tất cả các vectơ pháp tuyến có độ dài 2√5$2\sqrt 5 $của đường thẳng ∆.
2√5
2√5
2
√5
√
√
5

5
5

\textbf{{ANSWER}}

Hướng dẫn giải
Dựa vào phương trình tổng quát của đường thẳng ∆: 2x – y + 5 = 0. Đường thẳng ∆ có một vectơ pháp tuyến là →n=(2;−1)$\overrightarrow n = \left( {2; - 1} \right)$ nên các vectơ pháp tuyến của ∆ có dạng là →n′=(2t;−t)$\overrightarrow {n'} = \left( {2t; - t} \right)$. Theo giả thiết ta có: 
|→n′|=√(2t)2+(−t)2=2√5$\left| {\overrightarrow {n'} } \right| = \sqrt {{{\left( {2t} \right)}^2} + {{\left( { - t} \right)}^2}} = 2\sqrt 5 $
⇔ 4t2 + t2 = 20
⇔ 5t2 = 20
⇔ t2 = 4
⇔ t = ±2
Với t = 2, ta được vectơ pháp tuyến thỏa mãn yêu cầu đề bài là: →n1′$\overrightarrow {{n_1}'} $ = (4; –2)
Với t = – 2, ta được vectơ pháp tuyến thỏa mãn yêu cầu đề bài là: →n2′$\overrightarrow {{n_2}'} $ = (–4; 2).
Vậy có hai vectơ pháp tuyến thỏa mãn là →n1′$\overrightarrow {{n_1}'} $ = (4; –2) và →n2′$\overrightarrow {{n_2}'} $ = (–4; 2).

========================================================================

https://khoahoc.vietjack.com/thi-online/giai-sgk-toan-9-ctst-bai-tap-cuoi-chuong-3-co-dap-an


\textbf{{QUESTION}}

Biểu thức nào sau đây có giá trị khác với các biểu thức còn lại?
A. $$ {\left(-\sqrt{5}\right)}^{2}$$.
B. $$ \sqrt{{5}^{2}}$$.
C. $$ \sqrt{{\left(-5\right)}^{2}}$$
D. $$ -{\left(\sqrt{5}\right)}^{2}$$

\textbf{{ANSWER}}

Đáp án đúng là: D
Ta có $$ {\left(-\sqrt{5}\right)}^{2}=\sqrt{{5}^{2}}=\sqrt{{\left(-5\right)}^{2}}=5\text{\hspace{0.17em}};\text{\hspace{0.17em}\hspace{0.17em}\hspace{0.17em}}-{\left(\sqrt{5}\right)}^{2}=-5.$$
Vậy biểu thức $$ -{\left(\sqrt{5}\right)}^{2}$$ có giá trị khác với các biểu thức còn lại.

========================================================================

https://khoahoc.vietjack.com/thi-online/giai-sgk-toan-9-ctst-bai-tap-cuoi-chuong-3-co-dap-an


\textbf{{QUESTION}}

Có bao nhiêu số tự nhiên x để √16−x$$ \sqrt{16-x}$$ là số nguyên?
A. 2.
B. 3.
C. 4.
D. 5.

\textbf{{ANSWER}}

Đáp án đúng là: D
ĐKXĐ: 16 – x ≥ 0 hay x ≤ 16.
Vì x là số tự nhiên nên 0 ≤ x ≤ 16.
Do đó 0 ≤ 16 – x ≤ 16.
Mà $$ \sqrt{16-x}$$ là số nguyên nên (16 – x) số chính phương.
Suy ra (16 – x) ∈ {0; 1; 4; 9; 16}.
Ta có bảng sau:
16 – x
0
1
4
9
16
x
16 (TM)
15 (TM)
12 (TM)
7 (TM)
0 (TM)
$$ \sqrt{16-x}$$
0 (TM)
1 (TM)
2 (TM)
3 (TM)
4 (TM)
Vậy có 5 số tự nhiên x thỏa mãn yêu cầu là x ∈ {0; 7; 12; 15; 16}.

========================================================================

https://khoahoc.vietjack.com/thi-online/giai-sgk-toan-9-ctst-bai-tap-cuoi-chuong-3-co-dap-an


\textbf{{QUESTION}}

Giá trị của biểu thức √16+3√−64$$ \sqrt{16}+\sqrt[3]{-64}$$ bằng
A. 0.
B. –2.
C. 8.
D. –4.

\textbf{{ANSWER}}

Đáp án đúng là: A
Ta có √16+3√−64=√42+3√(−4)3=4+(−4)=0.$$ \sqrt{16}+\sqrt[3]{-64}=\sqrt{{4}^{2}}+\sqrt[3]{{\left(-4\right)}^{3}}=4+\left(-4\right)=0.$$

========================================================================

https://khoahoc.vietjack.com/thi-online/11-bai-tap-uoc-luong-ket-qua-dphep-tinh-co-loi-giai


\textbf{{QUESTION}}

Ước lượng kết quả của phép tính 94,5 . 1,02 đến hàng phần mười ta được số

\textbf{{ANSWER}}

Đáp án đúng là: C
Ta có 94,5 . 1,02 = 96,39.
Ước lượng đến hàng phần mười của 96,39 là 96,4.

========================================================================

https://khoahoc.vietjack.com/thi-online/11-bai-tap-uoc-luong-ket-qua-dphep-tinh-co-loi-giai


\textbf{{QUESTION}}

Trong 4 số sau có một số là kết quả của phép tính 302,5 + 449,78 + 88,2. Bằng cách ước lượng, em hãy cho biết số đó là số nào?

\textbf{{ANSWER}}

Đáp án đúng là: C
Ta có 302,5 + 449,78 + 88,2 ≈ 302 + 450 + 88 = 840.
Do đó kết quả phép tính sẽ là 840,48.

========================================================================

https://khoahoc.vietjack.com/thi-online/11-bai-tap-uoc-luong-ket-qua-dphep-tinh-co-loi-giai


\textbf{{QUESTION}}

Biết (893,6 – 17,95) : x = 2,78 + 2,22. Làm tròn x đến hàng chục ta được số

\textbf{{ANSWER}}

Đáp án đúng là: D
(893,6 – 17,95) : x = 2,78 + 2,22
875,65 : x = 5
x = 875,65 : 5
x = 175,13

========================================================================

https://khoahoc.vietjack.com/thi-online/11-bai-tap-uoc-luong-ket-qua-dphep-tinh-co-loi-giai


\textbf{{QUESTION}}

Cho A = – 19,37 – 51,081. Bằng cách ước lượng kết quả phép tính, ta kết luận được điều gì sau đây?

\textbf{{ANSWER}}

Đáp án đúng là: B
Ta có A = – 19,37 – 51,081 ≈ –19 – 51 = –70.
Vậy ta chọn phương án B.

========================================================================

https://khoahoc.vietjack.com/thi-online/11-bai-tap-uoc-luong-ket-qua-dphep-tinh-co-loi-giai


\textbf{{QUESTION}}

Diện tích trồng cây ăn quả của nhà vườn Hoa Cường là 97,53 ha. Ba tháng đầu năm, nhà vườn dùng 45$$ \frac{4}{5}$$  diện tích để trồng su hào. Diện tích trồng su hào của nhà vườn Hoa Cường gần nhất với giá trị nào sau đây?

\textbf{{ANSWER}}

Đáp án đúng là: B
Diện tích trồng su hào của nhà vườn Hoa Cường là   97,53⋅45≈100⋅0,8=80$$ 97,53\cdot \frac{4}{5}\approx 100\cdot 0,8=80$$(ha).

========================================================================

https://khoahoc.vietjack.com/thi-online/10-bai-tap-tim-sos-chu-so-cua-mot-so-thoa-man-chia-het-cho-2-cho-5-cho-9-cho-3-co-loi-giai


\textbf{{QUESTION}}

Có bao nhiêu chữ số x để $\overline {324x} $ chia hết cho 2?
A. 2;
B. 4;
C. 5;
D. 6.

\textbf{{ANSWER}}

Đáp án đúng là: C
Số $\overline {324x} $ có chữ số tận cùng là x. Để $\overline {324x} $ chia hết cho 2 thì x phải là 0; 2; 4; 6 hoặc 8. Vậy có tất cả 5 giá trị của x thỏa mãn yêu cầu.

========================================================================

https://khoahoc.vietjack.com/thi-online/10-bai-tap-tim-sos-chu-so-cua-mot-so-thoa-man-chia-het-cho-2-cho-5-cho-9-cho-3-co-loi-giai


\textbf{{QUESTION}}

Cho ¯1a32$\overline {1a32} $chia hết cho 9. Tính tổng tất cả các giá trị của chữ số a tìm được?
A. 6;
B. 3;
C. 9;
D. 23.

\textbf{{ANSWER}}

Đáp án đúng là: B
Xét $\overline {1a32} $ta có tổng các chữ số bằng 1 + a + 3 + 2 = 6 + a.
Để $\overline {1a32} $chia hết cho 9 thì 6 + a chia hết cho 9. Vì a là chữ số và đứng ở vị trí hàng trăm nên $0 \le a \le 9,a \in N$, do đó a chỉ có thể bằng 3.
Vậy tổng các giá trị của a là 3.

========================================================================

https://khoahoc.vietjack.com/thi-online/10-bai-tap-tim-sos-chu-so-cua-mot-so-thoa-man-chia-het-cho-2-cho-5-cho-9-cho-3-co-loi-giai


\textbf{{QUESTION}}

Số các chữ số x để ¯x789$\overline {x789} $ chia hết cho 2?
A. 2;
B. 4;
C.1;
D. 0.

\textbf{{ANSWER}}

Đáp án đúng là: D
Vì ¯x789$\overline {x789} $ có chữ số tận cùng là 9 nên không bao giờ chia hết cho 2. Vậy không có giá trị nào của x để ¯x789$\overline {x789} $chia hết cho 9.

========================================================================

https://khoahoc.vietjack.com/thi-online/10-bai-tap-tim-sos-chu-so-cua-mot-so-thoa-man-chia-het-cho-2-cho-5-cho-9-cho-3-co-loi-giai


\textbf{{QUESTION}}

Số cặp chữ số x, y để số ¯3x4y$\overline {3x4y} $vừa chia hết cho 5 vừa chia hết cho 9?
A. 2;
B. 0;
C. 1;
D. 4.

\textbf{{ANSWER}}

Đáp án đúng là: A
Số ¯3x4y$\overline {3x4y} $ có chữ số tận cùng là y. Để ¯3x4y$\overline {3x4y} $ chia hết cho 5 thì y phải là 0 hoặc 5. 
Trường hợp 1: y = 0 ta có số ¯3x40$\overline {3x40} $, tổng các chữ số trong đó là 3 + x + 4 + 0 = 7 + x. Để ¯3x40$\overline {3x40} $chia hết cho 9 thì 7 + x phải chia hết cho 9. Vì x không phải chữ số đầu tiên nên 0≤x≤9,x∈N$0 \le x \le 9,x \in N$, do đó x chỉ có thể là 2.
Trường hợp 2: y = 5 ta có số ¯3x45$\overline {3x45} $, tổng các chữ số trong đó là 3 + x + 4 + 5 = 12 + x. Để ¯3x45$\overline {3x45} $chia hết cho 9 thì 12 + x phải chia hết cho 9. Vì x không phải chữ số đầu tiên nên 0≤x≤9,x∈N$0 \le x \le 9,x \in N$, do đó x chỉ có thể là 6.
Vậy có 2 cặp chữ số (x, y) thỏa mãn điều kiện, đó là (0; 2) và (5; 6)

========================================================================

https://khoahoc.vietjack.com/thi-online/10-bai-tap-tim-sos-chu-so-cua-mot-so-thoa-man-chia-het-cho-2-cho-5-cho-9-cho-3-co-loi-giai


\textbf{{QUESTION}}

Tìm chữ số thích hợp ở vị trí * để số ¯139∗$\overline {139*} $ chia hết cho cả 2 và 5?
A. 2;
B. 4;
C. 0;
D. 5.

\textbf{{ANSWER}}

Đáp án đúng là: C
Để ¯139∗$\overline {139*} $ có chữ số tận cùng là *. Để ¯139∗$\overline {139*} $ chia hết cho 2 thì * có thể là 0; 2; 4; 6 hoặc 8.
Mặt khác để ¯139∗$\overline {139*} $ chia hết cho 5 thì * có thể là 0; 5. Vậy để ¯139∗$\overline {139*} $ chia hết cho cả 2 và 5 thì * chỉ có thể bằng 0.

========================================================================

https://khoahoc.vietjack.com/thi-online/bai-tap-toan-8-chu-de-11-phep-chia-cac-phan-thuc-dai-so-co-dap-an/108409


\textbf{{QUESTION}}

Rút gọn các biểu thức $$ \frac{x+1}{x+2}:\frac{x+2}{x+3}:\frac{x+3}{x+1}$$

\textbf{{ANSWER}}

$$ \frac{x+1}{x+2}:\frac{x+2}{x+3}:\frac{x+3}{x+1}=\frac{x+1}{x+2}.\frac{x+3}{x+2}.\frac{x+1}{x+3}=\frac{{\left(x+1\right)}^{2}}{{\left(x+2\right)}^{2}}$$

========================================================================

https://khoahoc.vietjack.com/thi-online/bai-tap-toan-8-chu-de-11-phep-chia-cac-phan-thuc-dai-so-co-dap-an/108409


\textbf{{QUESTION}}

Rút gọn các biểu thức x+1x+2:(x+2x+3:x+3x+1)$$ \frac{x+1}{x+2}:\left(\frac{x+2}{x+3}:\frac{x+3}{x+1}\right)$$

\textbf{{ANSWER}}

$$ \frac{x+1}{x+2}:\left(\frac{x+2}{x+3}:\frac{x+3}{x+1}\right)=\frac{x+1}{x+2}:\left(\frac{x+2}{x+3}.\frac{x+1}{x+3}\right)=\frac{x+1}{x+2}:\left(\frac{x+2}{x+3}.\frac{x+1}{x+3}\right)$$$$ =\frac{x+1}{x+2}:\frac{\left(x+2\right).\left(x+1\right)}{{\left(x+3\right)}^{2}}=\frac{x+1}{x+2}.\frac{{\left(x+3\right)}^{2}}{\left(x+2\right).\left(x+1\right)}=\frac{{\left(x+3\right)}^{2}}{{\left(x+2\right)}^{2}}$$

========================================================================

https://khoahoc.vietjack.com/thi-online/giai-sbt-toan-12-tap-1-kntt-bai-3-duong-tiem-can-cua-do-thi-ham-so-co-dap-an


\textbf{{QUESTION}}

Cho hàm số $y = f(x) = \frac{{{x^2} + 3x - 10}}{{x - 2}}$. Đồ thị hàm số f(x) có tiệm cận đứng không?

\textbf{{ANSWER}}

Ta có: $\mathop {\lim }\limits_{x \to 2} f\left( x \right) = \mathop {\lim }\limits_{x \to 2} \frac{{\left( {x - 2} \right)\left( {x + 5} \right)}}{{x - 2}}$= $\mathop {\lim }\limits_{x \to 2} $(x + 5) = 7.
Hơn nữa y = f(x) liên tục tại mọi điểm x ≠ 2. Do đó, đồ thị hàm f(x) không có tiệm cận đứng.

========================================================================

https://khoahoc.vietjack.com/thi-online/giai-sbt-toan-12-tap-1-kntt-bai-3-duong-tiem-can-cua-do-thi-ham-so-co-dap-an


\textbf{{QUESTION}}

Tìm các đường tiệm cận của đồ thị các hàm số sau:
a) y = $\frac{{x + 1}}{{2x - 3}};$
b) y = $\frac{{3x - 1}}{{x + 2}}.$

\textbf{{ANSWER}}

a) Ta có: $\mathop {\lim }\limits_{x \to - \infty } y = \mathop {\lim }\limits_{x \to - \infty } \frac{{x + 1}}{{2x - 3}} = \frac{1}{2}$;
               $\mathop {\lim }\limits_{x \to + \infty } y = \mathop {\lim }\limits_{x \to + \infty } \frac{{x + 1}}{{2x - 3}} = \frac{1}{2}$.
Do đó, đường thẳng y = $\frac{1}{2}$ là tiệm cận ngang của đồ thị hàm số.
              $\mathop {\lim }\limits_{x \to {{\frac{3}{2}}^ + }} y = \mathop {\lim }\limits_{x \to {{\frac{3}{2}}^ + }} \frac{{x + 1}}{{2x - 3}} = + \infty $;
              $\mathop {\lim }\limits_{x \to {{\frac{3}{2}}^ - }} y = \mathop {\lim }\limits_{x \to {{\frac{3}{2}}^ - }} \frac{{x + 1}}{{2x - 3}} = - \infty $.
Do đó, đường thẳng x = $\frac{3}{2}$ là đường tiệm cận đứng của đồ thị hàm số.
b) Ta có: $\mathop {\lim }\limits_{x \to - \infty } y = \mathop {\lim }\limits_{x \to - \infty } \frac{{3x - 1}}{{x + 2}} = 3$;
               $\mathop {\lim }\limits_{x \to + \infty } y = \mathop {\lim }\limits_{x \to + \infty } \frac{{3x - 1}}{{x + 2}} = 3$.
Do đó, đường thẳng y = 3 là tiệm cận ngang của đồ thị hàm số.
              $\mathop {\lim }\limits_{x \to - {2^ + }} y = \mathop {\lim }\limits_{x \to - {2^ + }} \frac{{3x - 1}}{{x + 2}} = - \infty $;
              $\mathop {\lim }\limits_{x \to - {2^ - }} y = \mathop {\lim }\limits_{x \to - {2^ - }} \frac{{3x - 1}}{{x + 2}} = + \infty $.
Do đó, đường thẳng x = −2 là đường tiệm cận đứng của đồ thị hàm số.

========================================================================

https://khoahoc.vietjack.com/thi-online/giai-sbt-toan-12-tap-1-kntt-bai-3-duong-tiem-can-cua-do-thi-ham-so-co-dap-an


\textbf{{QUESTION}}

Tìm tiệm cận đứng và tiệm cận xiên của đồ thị hàm số sau:
a) y=x2−x−5x−2;$y = \frac{{{x^2} - x - 5}}{{x - 2}};$
b) y = 3x2+8x−2x+3.$\frac{{3{x^2} + 8x - 2}}{{x + 3}}.$

\textbf{{ANSWER}}

a) y=x2−x−5x−2;$y = \frac{{{x^2} - x - 5}}{{x - 2}};$
Ta có: limx→2+y=limx→2+x2−x−5x−2=−∞$\mathop {\lim }\limits_{x \to {2^ + }} y = \mathop {\lim }\limits_{x \to {2^ + }} \frac{{{x^2} - x - 5}}{{x - 2}} = - \infty $;
           limx→2−y=limx→2−x2−x−5x−2=+∞$\mathop {\lim }\limits_{x \to {2^ - }} y = \mathop {\lim }\limits_{x \to {2^ - }} \frac{{{x^2} - x - 5}}{{x - 2}} = + \infty $.
Do đó, đường thẳng x = 2 là đường tiệm cận đứng của đồ thị hàm số.
            limx→+∞yx=limx→+∞x2−x−5(x−2)x=1$\mathop {\lim }\limits_{x \to + \infty } \frac{y}{x} = \mathop {\lim }\limits_{x \to + \infty } \frac{{{x^2} - x - 5}}{{\left( {x - 2} \right)x}} = 1$.
            limx→+∞(y−x)=limx→+∞[x2−x−5x−2−x]=limx→+∞x−5x−2=1$\mathop {\lim }\limits_{x \to + \infty } \left( {y - x} \right) = \mathop {\lim }\limits_{x \to + \infty } \left[ {\frac{{{x^2} - x - 5}}{{x - 2}} - x} \right] = \mathop {\lim }\limits_{x \to + \infty } \frac{{x - 5}}{{x - 2}} = 1$.
Do đó đường thẳng y = x + 1 là đường tiệm cận xiên của đồ thị hàm số.
b) y = 3x2+8x−2x+3$\frac{{3{x^2} + 8x - 2}}{{x + 3}}$
Ta có: limx→−3+y=limx→−3+3x2+8x−2x+3=+∞$\mathop {\lim }\limits_{x \to - {3^ + }} y = \mathop {\lim }\limits_{x \to - {3^ + }} \frac{{3{x^2} + 8x - 2}}{{x + 3}} = + \infty $;
           limx→−3−y=limx→−3−3x2+8x−2x+3=−∞$\mathop {\lim }\limits_{x \to - {3^ - }} y = \mathop {\lim }\limits_{x \to - {3^ - }} \frac{{3{x^2} + 8x - 2}}{{x + 3}} = - \infty $
Do đó đường thẳng x = −3 là tiệm cận đứng của đồ thị hàm số.
            limx→+∞yx=limx→+∞3x2+8x−2(x+3)x=3$\mathop {\lim }\limits_{x \to + \infty } \frac{y}{x} = \mathop {\lim }\limits_{x \to + \infty } \frac{{3{x^2} + 8x - 2}}{{\left( {x + 3} \right)x}} = 3$.
            limx→+∞(y−3x)=limx→+∞[3x2+8x−2x+3−3x]=limx→+∞−x−2x−2=−1$\mathop {\lim }\limits_{x \to + \infty } \left( {y - 3x} \right) = \mathop {\lim }\limits_{x \to + \infty } \left[ {\frac{{3{x^2} + 8x - 2}}{{x + 3}} - 3x} \right] = \mathop {\lim }\limits_{x \to + \infty } \frac{{ - x - 2}}{{x - 2}} = - 1$.
Do đó đường thẳng y = 3x – 1 là tiệm cận xiên của đồ thị hàm số.

========================================================================

https://khoahoc.vietjack.com/thi-online/12-bai-tap-xac-dinh-gid-tri-cua-m-de-ham-so-bdt-gia-tri-ndnhat-tai-mot-so-cho-tru


\textbf{{QUESTION}}

Cho hàm số y = 2x2 + x + m. Hãy xác định giá trị của m để hàm số đạt giá trị nhỏ nhất bằng 5.

\textbf{{ANSWER}}

Hướng dẫn giải:
Xét hàm số y = 2x2 + x + m có:
$\frac{{ - b}}{{2a}} = \frac{{ - 1}}{{2.2}} = \frac{{ - 1}}{4}$ 
$\frac{{ - \Delta }}{{4a}} = \frac{{ - ({b^2} - 4ac)}}{{4a}} = \frac{{ - ({1^2} - 4.2.m)}}{{4.2}} = \frac{{ - 1 + 8m}}{8} = \frac{{ - 1}}{8} + m$
Ta có, a = 2 > 0 nên hàm số đạt giá trị nhỏ nhất là $ - \frac{1}{8} + m$ tại $x = - \frac{1}{4}$
Để hàm số đạt giá trị nhỏ nhất bằng 5 khi và chỉ khi $ - \frac{1}{8} + m = 5 \Leftrightarrow m = \frac{{41}}{8}$
Vậy $m = \frac{{41}}{8}$ thỏa mãn yêu cầu đề bài.

========================================================================

https://khoahoc.vietjack.com/thi-online/12-bai-tap-xac-dinh-gid-tri-cua-m-de-ham-so-bdt-gia-tri-ndnhat-tai-mot-so-cho-tru


\textbf{{QUESTION}}

Cho hàm số y = –x2 + 5x + m. Hãy xác định giá trị của m để hàm số đạt giá trị lớn nhất bằng 12.

\textbf{{ANSWER}}

Hướng dẫn giải:
Xét hàm số y = –x2 + 5x + m có:
$\frac{{ - b}}{{2a}} = \frac{{ - 5}}{{2.\left( { - 1} \right)}} = \frac{5}{2}$ 
$\frac{{ - \Delta }}{{4a}} = \frac{{ - ({b^2} - 4ac)}}{{4a}} = \frac{{ - ({5^2} - 4.( - 1).m)}}{{4.( - 1)}} = \frac{{ - 25 - 4m}}{{ - 4}} = \frac{{25}}{4} + m$
Ta có, a = –1 < 0 nên hàm số đạt giá trị lớn nhất là $\frac{{25}}{4} + m$ tại $x = \frac{5}{2}$.
Để hàm số đạt giá trị lớn nhất bằng 12 khi và chỉ khi $\frac{{25}}{4} + m = 12 \Leftrightarrow m = \frac{{23}}{4}$
Vậy $m = \frac{{23}}{4}$ thỏa mãn yêu cầu đề bài.

========================================================================

https://khoahoc.vietjack.com/thi-online/12-bai-tap-xac-dinh-gid-tri-cua-m-de-ham-so-bdt-gia-tri-ndnhat-tai-mot-so-cho-tru


\textbf{{QUESTION}}

Cho hàm số y = x2 – 3x + m. Giá trị của m để hàm số đạt giá trị nhỏ nhất bằng 12 là:
A. m=574$m = \frac{{57}}{4}$;
m=574
m=574
m
=
574
57
57
57
4
4


4
4
4
B. m=−234$m = - \frac{{23}}{4}$;
m=−234
m=−234
m
=
−
234
23
23
23
4
4


4
4
4
C. m=254$m = \frac{{25}}{4}$;
m=254
m=254
m
=
254
25
25
25
4
4


4
4
4
D. m=−224$m = - \frac{{22}}{4}$.
m=−224
m=−224
m
=
−
224
22
22
22
4
4


4
4
4

\textbf{{ANSWER}}

Hướng dẫn giải:
Đáp án đúng là: A.
Xét hàm số y = x2 – 3x + m có:
−b2a=−(−3)2.1=32$\frac{{ - b}}{{2a}} = \frac{{ - ( - 3)}}{{2.1}} = \frac{3}{2}$
−Δ4a=−(b2−4ac)4a=−((−3)2−4.1.m)4.1=−9+4m4=−94+m$\frac{{ - \Delta }}{{4a}} = \frac{{ - ({b^2} - 4ac)}}{{4a}} = \frac{{ - ({{( - 3)}^2} - 4.1.m)}}{{4.1}} = \frac{{ - 9 + 4m}}{4} = \frac{{ - 9}}{4} + m$
 Ta có, a = 1 > 0 nên hàm số đạt giá trị nhỏ nhất là −94+m$\frac{{ - 9}}{4} + m$ tại x=32$x = \frac{3}{2}$
Để hàm số đạt giá trị nhỏ nhất bằng 12 khi và chỉ khi −94+m=12⇔m=574$\frac{{ - 9}}{4} + m = 12 \Leftrightarrow m = \frac{{57}}{4}$
Vậy m=574$m = \frac{{57}}{4}$ thỏa mãn yêu cầu đề bài.

========================================================================

https://khoahoc.vietjack.com/thi-online/12-bai-tap-xac-dinh-gid-tri-cua-m-de-ham-so-bdt-gia-tri-ndnhat-tai-mot-so-cho-tru


\textbf{{QUESTION}}

Cho hàm số y = –x2 + 6x – m. Giá trị của m để hàm số đạt giá trị lớn nhất bằng 6 là:
A. m = 3;
B. m = 1;
C. m = –1;
D. m = –3.

\textbf{{ANSWER}}

Hướng dẫn giải:
Đáp án đúng là: A.
Xét hàm số y = –x2 + 6x – m có:
−b2a=−62.(−1)=3$\frac{{ - b}}{{2a}} = \frac{{ - 6}}{{2.( - 1)}} = 3$
−Δ4a=−(b2−4ac)4a=−(62−4.(−1).(−m))4.(−1)=−36+4m−4=9−m$\frac{{ - \Delta }}{{4a}} = \frac{{ - ({b^2} - 4ac)}}{{4a}} = \frac{{ - ({6^2} - 4.( - 1).\left( { - m} \right))}}{{4.( - 1)}} = \frac{{ - 36 + 4m}}{{ - 4}} = 9 - m$
 Ta có, a = –1 < 0 nên hàm số đạt giá trị lớn nhất là 9 – m tại x=32$x = \frac{3}{2}$
Để hàm số đạt giá trị lớn nhất bằng 6 khi và chỉ khi 9 – m = 6 hay m = 3.
Vậy m = –3 thỏa mãn yêu cầu đề bài.

========================================================================

https://khoahoc.vietjack.com/thi-online/12-bai-tap-xac-dinh-gid-tri-cua-m-de-ham-so-bdt-gia-tri-ndnhat-tai-mot-so-cho-tru


\textbf{{QUESTION}}

Cho hàm số y = –2x2 + 4x – 3m. Giá trị của m để hàm số đạt giá trị lớn nhất bằng 10 là:
A. m = $\frac{8}{3}$;

B. m = –$\frac{8}{3}$;

C. m = 1;
D. m = –1.

\textbf{{ANSWER}}

Hướng dẫn giải:
Đáp án đúng là: B.
Xét hàm số y = –2x2 + 4x – 3m có:
$\frac{{ - b}}{{2a}} = \frac{{ - 4}}{{2.( - 2)}} = 1$
$\frac{{ - \Delta }}{{4a}} = \frac{{ - ({b^2} - 4ac)}}{{4a}} = \frac{{ - ({4^2} - 4.( - 2).( - 3m))}}{{4.( - 2)}} = \frac{{ - 16 + 24m}}{{ - 8}} = 2 - 3m$
 Ta có, a = –2 < 0 nên hàm số đạt giá trị lớn nhất là 2 – 3m tại x = 1
Để hàm số đạt giá trị lớn nhất bằng 10 khi và chỉ khi 2 – 3m = 10 hay m = –$\frac{8}{3}$
Vậy m = –$\frac{8}{3}$ thỏa mãn yêu cầu đề bài.

========================================================================

https://khoahoc.vietjack.com/thi-online/10-bai-tap-viet-phuong-trinh-duong-thang-khi-biet-vtpt-hoac-vtcp-hoac-he-so-goc-va-1-diem-di-qua-co


\textbf{{QUESTION}}

Trong mặt phẳng tọa độ Oxy, phương trình tổng quát của đường thẳng d có vectơ pháp tuyến $$ \overrightarrow{n}=\left(3;5\right)$$  và đi qua điểm N(2; –1) là
A. 3x + 5y – 1 = 0;
B. 3x + 5y + 1 = 0;
C. 5x + 3y –1 = 0;

\textbf{{ANSWER}}

Hướng dẫn giải:
Đáp án đúng là: A
Phương trình tổng quát của đường thẳng d có vectơ pháp tuyến $$ \overrightarrow{n}=\left(3;5\right)$$ và đi qua điểm N(2; –1) là: 3(x – 2) + 5(y + 1) = 0 tức là 3x + 5y – 1 = 0.

========================================================================

https://khoahoc.vietjack.com/thi-online/10-bai-tap-viet-phuong-trinh-duong-thang-khi-biet-vtpt-hoac-vtcp-hoac-he-so-goc-va-1-diem-di-qua-co


\textbf{{QUESTION}}

A. {x=−5+3ty=4+t$$ \left\{\begin{array}{l}x=-5+3t\\ y=4+t\end{array}\right.$$;
B. {x=3−5ty=1+4t$$ \left\{\begin{array}{l}x=3-5t\\ y=1+4t\end{array}\right.$$;
C. {x=−5+ty=4+3t$$ \left\{\begin{array}{l}x=-5+t\\ y=4+3t\end{array}\right.$$;

\textbf{{ANSWER}}

Hướng dẫn giải:
Đáp án đúng là: B
Phương trình tham số của đường thẳng d có vectơ chỉ phương $$ \overrightarrow{u}=\left(-5;4\right)$$ và đi qua điểm B(3; 1) là: $$ \left\{\begin{array}{l}x=3-5t\\ y=1+4t\end{array}\right.$$.

========================================================================

https://khoahoc.vietjack.com/thi-online/10-bai-tap-viet-phuong-trinh-duong-thang-khi-biet-vtpt-hoac-vtcp-hoac-he-so-goc-va-1-diem-di-qua-co


\textbf{{QUESTION}}

Trong mặt phẳng tọa độ Oxy, phương trình tham số của đường thẳng d có vectơ pháp tuyến →n=(−4;1)$$ \overrightarrow{n}=\left(-4;1\right)$$ và đi qua điểm A(1; –6) là
A. {x=−4+ty=1−6t$$ \left\{\begin{array}{l}x=-4+t\\ y=1-6t\end{array}\right.$$;
B. {x=1−4ty=−6+t$$ \left\{\begin{array}{l}x=1-4t\\ y=-6+t\end{array}\right.$$;
C. {x=1−ty=−6−4t$$ \left\{\begin{array}{l}x=1-t\\ y=-6-4t\end{array}\right.$$;

\textbf{{ANSWER}}

Hướng dẫn giải:
Đáp án đúng là: C
Đường thẳng d có vectơ pháp tuyến →n=(−4;1)$$ \overrightarrow{n}=\left(-4;1\right)$$, khi đó đường thẳng d có vectơ chỉ phương là →u=(−1;−4)$$ \overrightarrow{u}=\left(-1;-4\right)$$ và đi qua điểm A(1; –6) có phương trình tham số là: {x=1−ty=−6−4t$$ \left\{\begin{array}{l}x=1-t\\ y=-6-4t\end{array}\right.$$.

========================================================================

https://khoahoc.vietjack.com/thi-online/10-bai-tap-viet-phuong-trinh-duong-thang-khi-biet-vtpt-hoac-vtcp-hoac-he-so-goc-va-1-diem-di-qua-co


\textbf{{QUESTION}}

Trong mặt phẳng tọa độ Oxy, phương trình tổng quát của đường thẳng d có vectơ chỉ phương →u=(−2;7)$$ \overrightarrow{u}=\left(-2;7\right)$$ và đi qua điểm P(–4; 9) là
A. 7x – 2y – 10 = 0;
B. 7x – 2y + 10 = 0;
C. 7x + 2y –10 = 0;

\textbf{{ANSWER}}

Hướng dẫn giải:
Đáp án đúng là: D
Đường thẳng d có vectơ chỉ phương $$ \overrightarrow{u}=\left(-2;7\right)$$, khi đó đường thẳng d có vectơ pháp tuyến là $$ \overrightarrow{n}=\left(7;2\right)$$ và đi qua điểm P(–4; 9) có phương trình là: 
7(x + 4) + 2(y – 9) = 0, tức là: 7x + 2y + 10 = 0.
Vậy phương trình tổng quát của đường thẳng d là: 7x + 2y + 10 = 0.

========================================================================

https://khoahoc.vietjack.com/thi-online/10-bai-tap-viet-phuong-trinh-duong-thang-khi-biet-vtpt-hoac-vtcp-hoac-he-so-goc-va-1-diem-di-qua-co


\textbf{{QUESTION}}

Trong mặt phẳng tọa độ Oxy, phương trình tham số của đường thẳng đi qua hai điểm A(2; –1) và B(2; 5) là
A. {x=2ty=−6t$$ \left\{\begin{array}{l}x=2t\\ y=-6t\end{array}\right.$$;
B. {x=2+ty=5+6t$$ \left\{\begin{array}{l}x=2+t\\ y=5+6t\end{array}\right.$$;
C. {x=1y=2+6t$$ \left\{\begin{array}{l}x=1\\ y=2+6t\end{array}\right.$$;

\textbf{{ANSWER}}

Hướng dẫn giải:
Đáp án đúng là: D
Với A(2; –1) và B(2; 5) ta có: →AB=(0;6)$$ \overrightarrow{AB}=\left(0;6\right)$$.
Phương trình đường thẳng AB đi qua A(2; –1) và có vectơ chỉ phương →AB=(0;6)$$ \overrightarrow{AB}=\left(0;6\right)$$ là:
{x=2y=−1+6t$$ \left\{\begin{array}{l}x=2\\ y=-1+6t\end{array}\right.$$.

========================================================================

https://khoahoc.vietjack.com/thi-online/12-bai-tap-chsgiac-tinh-cac-gia-tri-luong-giac-con-lai-hoac-tinh-gia-tri-cua-bieu


\textbf{{QUESTION}}

Tính các giá trị lượng giác còn lại của góc α biết sinα = $\frac{1}{3}$ và 90° < α < 180°.
Tính các giá trị lượng giác còn lại của góc α biết sinα =
 
$\frac{1}{3}$

 và 90° < α < 180°.

\textbf{{ANSWER}}

Hướng dẫn giải:
Vì 90° < α < 180° nên cosα < 0. 
Ta có: sin2α + cos2α = 1 
Suy ra cosα = $ - \sqrt {1 - {{\sin }^2}\alpha } = - \sqrt {1 - \frac{1}{9}} = - \frac{{2\sqrt 2 }}{3}$.
Do đó $\tan \alpha = \frac{{\sin \alpha }}{{\cos \alpha }} = \frac{{\frac{1}{3}}}{{ - \frac{{2\sqrt 2 }}{3}}} = - \frac{1}{{2\sqrt 2 }}$ 
và $\cot \alpha = \frac{1}{{\tan \alpha }} = - 2\sqrt 2 $.

========================================================================

https://khoahoc.vietjack.com/thi-online/12-bai-tap-chsgiac-tinh-cac-gia-tri-luong-giac-con-lai-hoac-tinh-gia-tri-cua-bieu


\textbf{{QUESTION}}

Cho góc α với cosα=√22$\cos \alpha = \frac{{\sqrt 2 }}{2}$. Tính giá trị của biểu thức A = 2sin2α + 5cos2α.
cosα=√22
cosα=√22
cos

α
=
√22
√2
√2
√2
√
√
2

2
2
2
2


2
2
2

\textbf{{ANSWER}}

Hướng dẫn giải:
Ta có: A = 2sin2α + 5cos2α = 2 . (1 – cos2α) + 5cos2α = 2 + 3cos2α
Với $\cos \alpha = \frac{{\sqrt 2 }}{2}$, thay vào biểu thức A ta được
A = 2 + 3 . ${\left( {\frac{{\sqrt 2 }}{2}} \right)^2}$ = 2 + 3 . $\frac{1}{2}$ = $\frac{7}{2}$.
Vậy A = $\frac{7}{2}$.

========================================================================

https://khoahoc.vietjack.com/thi-online/12-bai-tap-chsgiac-tinh-cac-gia-tri-luong-giac-con-lai-hoac-tinh-gia-tri-cua-bieu


\textbf{{QUESTION}}

Cho góc α (0° < α < 180°) với cosα=13$\cos \alpha = \frac{1}{3}$. Giá trị của sinα bằng:

\textbf{{ANSWER}}

Hướng dẫn giải:
Đáp án đúng là: C.
Vì 0° < α < 180° nên sinα > 0.
Lại có sin2α + cos2α = 1 
Suy ra sinα=√1−cos2α=√1−(13)2=2√23$\sin \alpha = \sqrt {1 - {{\cos }^2}\alpha } = \sqrt {1 - {{\left( {\frac{1}{3}} \right)}^2}} = \frac{{2\sqrt 2 }}{3}$.

========================================================================

https://khoahoc.vietjack.com/thi-online/12-bai-tap-chsgiac-tinh-cac-gia-tri-luong-giac-con-lai-hoac-tinh-gia-tri-cua-bieu


\textbf{{QUESTION}}

Cho góc α thỏa mãn sinα=1213$\sin \alpha = \frac{{12}}{{13}}$ và 90° < α < 180°. Tính cosα.
sinα=1213
sinα=1213
sin

α
=
1213
12
12
12
13
13


13
13
13
13
A. cosα=213$\cos \alpha = \frac{2}{{13}}$;
cosα=213
cosα=213
cos

α
=
213
2
2
13
13


13
13
13
13
B. cosα=513$\cos \alpha = \frac{5}{{13}}$;
cosα=513
cosα=513
cos

α
=
513
5
5
13
13


13
13
13
13
C. cosα=−513$\cos \alpha = - \frac{5}{{13}}$;
cosα=−513
cosα=−513
cos

α
=
−
513
5
5
13
13


13
13
13
13
D. cosα=−213$\cos \alpha = - \frac{2}{{13}}$.
cosα=−213
cosα=−213
cos

α
=
−
213
2
2
13
13


13
13
13
13

\textbf{{ANSWER}}

Hướng dẫn giải:
Đáp án đúng là: C.
Vì 90° < α < 180° nên cosα < 0. 
Do đó cosα=−√1−sin2α=−√1−(1213)2=−√25169=−513$cos\alpha = - \sqrt {1 - {{\sin }^2}\alpha } = - \sqrt {1 - {{\left( {\frac{{12}}{{13}}} \right)}^2}} = - \sqrt {\frac{{25}}{{169}}} = - \frac{5}{{13}}$.

========================================================================

https://khoahoc.vietjack.com/thi-online/12-bai-tap-chsgiac-tinh-cac-gia-tri-luong-giac-con-lai-hoac-tinh-gia-tri-cua-bieu


\textbf{{QUESTION}}

Cho góc α với 0° < α < 180°. Tính giá trị của cosα, biết tanα=−2√2$\tan \alpha = - 2\sqrt 2 $ .
Cho góc 
α
 với 0° < α < 180°. Tính giá trị của cosα, biết tanα=−2√2$\tan \alpha = - 2\sqrt 2 $
tanα=−2√2
tanα=−2√2
tan

α
=
−
2
√2
√
√
2

2
2
 
.
A. −13$ - \frac{1}{3}$;
−13
−13
−
13
1
1
3
3


3
3
3
B. 2√23$\frac{{2\sqrt 2 }}{3}$;
2√23
2√23
2√23
2√2
2√2
2
√2
√
√
2

2
2
3
3


3
3
3
C. 13$\frac{1}{3}$;
13
13
13
1
1
3
3


3
3
3
D. √23$\frac{{\sqrt 2 }}{3}$.
√23
√23
√23
√2
√2
√2
√
√
2

2
2
3
3


3
3
3

\textbf{{ANSWER}}

Hướng dẫn giải:
Đáp án đúng là: A.
Ta có tan2α+1=1cos2α${\tan ^2}\alpha + 1 = \frac{1}{{{{\cos }^2}\alpha }}$
⇒cos2α=1tan2α+1=1(−2√2)2+1=19$ \Rightarrow {\cos ^2}\alpha = \frac{1}{{{{\tan }^2}\alpha + 1}} = \frac{1}{{{{\left( { - 2\sqrt 2 } \right)}^2} + 1}} = \frac{1}{9}$⇒cosα=±13$ \Rightarrow \cos \alpha = \pm \frac{1}{3}$.
Vì 0° < α < 180° ⇒ sinα > 0 mà tanα=−2√2$\tan \alpha = - 2\sqrt 2 $< 0 nên cosα < 0. 
Do đó cosα=−13$\cos \alpha = - \frac{1}{3}$.

========================================================================

https://khoahoc.vietjack.com/thi-online/bo-15-de-kiem-tra-hoc-ki-1-toan-9-nam-2022-2023-co-dap-an/116628


\textbf{{QUESTION}}

Cho biểu thức $$ P=\left(\frac{\sqrt{a}}{\sqrt{a}-2}+\frac{\sqrt{a}}{\sqrt{a}+2}\right).\frac{a-4}{\sqrt{4a}}$$ .
Tìm điều kiện của a để P xác định.

\textbf{{ANSWER}}

P xác định $$ \Leftrightarrow \left\{\begin{array}{l}a\ge 0\\ \sqrt{a}-2\ne 0\\ \sqrt{a}+2\ne 0\\ 4a>0\end{array}\right.\Leftrightarrow \left\{\begin{array}{l}a>0\\ \sqrt{a}\ne 2\\ \sqrt{a}\ne -2\end{array}\right.\Leftrightarrow \left\{\begin{array}{l}a>0\\ a\ne {2}^{2}\end{array}\right.\Leftrightarrow \left\{\begin{array}{l}a>0\\ a\ne 4\end{array}\right.$$ .
Vậy $$ a>0;a\ne 4$$  thì biểu thức P xác định.

========================================================================

https://khoahoc.vietjack.com/thi-online/bo-15-de-kiem-tra-hoc-ki-1-toan-9-nam-2022-2023-co-dap-an/116628


\textbf{{QUESTION}}

Rút gọn P.

\textbf{{ANSWER}}

ĐKXĐ: $$ a>0;a\ne 4$$ .
$$ \begin{array}{l}P=\left(\frac{\sqrt{a}}{\sqrt{a}-2}+\frac{\sqrt{a}}{\sqrt{a}+2}\right).\frac{a-4}{\sqrt{4a}}\\ \text{\hspace{0.33em}\hspace{0.33em}\hspace{0.33em}}=\frac{\sqrt{a}\left(\sqrt{a}+2\right)+\sqrt{a}\left(\sqrt{a}-2\right)}{\left(\sqrt{a}+2\right)\left(\sqrt{a}-2\right)}.\frac{\left(\sqrt{a}+2\right)\left(\sqrt{a}-2\right)}{\sqrt{{2}^{2}.a}}\\ \text{\hspace{0.33em}\hspace{0.33em}\hspace{0.33em}}=\frac{a+2\sqrt{a}+a-2\sqrt{a}}{2\sqrt{a}}\\ \text{\hspace{0.33em}\hspace{0.33em}\hspace{0.33em}}=\frac{2a}{2\sqrt{a}}=\sqrt{a}\end{array}$$
 
Vậy $$ P=\sqrt{a}$$  với $$ a>0;a\ne 4$$ .

========================================================================

https://khoahoc.vietjack.com/thi-online/bo-15-de-kiem-tra-hoc-ki-1-toan-9-nam-2022-2023-co-dap-an/116628


\textbf{{QUESTION}}

Tìm a để P<3$$ P<3$$ .

\textbf{{ANSWER}}

ĐKXĐ: a>0;a≠4$$ a>0;a\ne 4$$ .
P<3⇔√a<3⇔a<32⇔a<9$$ P<3\Leftrightarrow \sqrt{a}<3\Leftrightarrow a<{3}^{2}\Leftrightarrow a<9$$
Kết hợp với điều kiện xác định ⇒0<a<9;a≠4$$ \Rightarrow 0<a<9;a\ne 4$$ .
Vậy 0<a<9;a≠4$$ 0<a<9;a\ne 4$$  thì P<3$$ P<3$$ .

========================================================================

https://khoahoc.vietjack.com/thi-online/bo-15-de-kiem-tra-hoc-ki-1-toan-9-nam-2022-2023-co-dap-an/116628


\textbf{{QUESTION}}

Chứng minh: Giá trị của biểu thức A=(√5+√3).√(√6+√2)(4−√15).√2−√3$$ A=\left(\sqrt{5}+\sqrt{3}\right).\sqrt{\left(\sqrt{6}+\sqrt{2}\right)\left(4-\sqrt{15}\right).\sqrt{2-\sqrt{3}}}$$  là một số nguyên.

\textbf{{ANSWER}}

Phương pháp
Dựa vào các hằng đẳng thức và công thức √A2=[A khi A≥0−A khi A<0$$ \sqrt{{A}^{2}}=\left[\begin{array}{l}A\text{\hspace{0.33em}}khi\text{\hspace{0.33em}}A\ge 0\\ -A\text{\hspace{0.33em}}khi\text{\hspace{0.33em}}A<0\end{array}\right.$$  để biến đổi và rút gọn A.
Cách giải
 
A=(√5+√3).√(√6+√2)(4−√15).√2−√3   =(√5+√3).√(√3+1)√2√2−√3(4−√15)   =(√5+√3).√(√3+1)√2(2−√3)(4−√15)   =(√5+√3).√(√3+1)√4−2√3(4−√15)   =(√5+√3).√(√3+1)√(√3−1)2(4−√15)   =(√5+√3).√(√3+1)(√3−1)(4−√15)  (do √3−1>0)   =(√5+√3).√(3−1)(4−√15)=(√5+√3)√2(4−√15)   =(√5+√3).√8−2√15=(√5+√3)√5−2√5√3+3   =(√5+√3).√(√5−√3)2=(√5+√3)(√5−√3) (do √5−√3>0)   =5−3=2$$ \begin{array}{l}A=\left(\sqrt{5}+\sqrt{3}\right).\sqrt{\left(\sqrt{6}+\sqrt{2}\right)\left(4-\sqrt{15}\right).\sqrt{2-\sqrt{3}}}\\ \text{\hspace{0.33em}\hspace{0.33em}\hspace{0.33em}}=\left(\sqrt{5}+\sqrt{3}\right).\sqrt{\left(\sqrt{3}+1\right)\sqrt{2}\sqrt{2-\sqrt{3}}\left(4-\sqrt{15}\right)}\\ \text{\hspace{0.33em}\hspace{0.33em}\hspace{0.33em}}=\left(\sqrt{5}+\sqrt{3}\right).\sqrt{\left(\sqrt{3}+1\right)\sqrt{2\left(2-\sqrt{3}\right)}\left(4-\sqrt{15}\right)}\\ \text{\hspace{0.33em}\hspace{0.33em}\hspace{0.33em}}=\left(\sqrt{5}+\sqrt{3}\right).\sqrt{\left(\sqrt{3}+1\right)\sqrt{4-2\sqrt{3}}\left(4-\sqrt{15}\right)}\\ \text{\hspace{0.33em}\hspace{0.33em}\hspace{0.33em}}=\left(\sqrt{5}+\sqrt{3}\right).\sqrt{\left(\sqrt{3}+1\right)\sqrt{{\left(\sqrt{3}-1\right)}^{2}}\left(4-\sqrt{15}\right)}\\ \text{\hspace{0.33em}\hspace{0.33em}\hspace{0.33em}}=\left(\sqrt{5}+\sqrt{3}\right).\sqrt{\left(\sqrt{3}+1\right)\left(\sqrt{3}-1\right)\left(4-\sqrt{15}\right)}\text{\hspace{0.33em}\hspace{0.33em}}\left(do\text{\hspace{0.33em}}\sqrt{3}-1>0\right)\\ \text{\hspace{0.33em}\hspace{0.33em}\hspace{0.33em}}=\left(\sqrt{5}+\sqrt{3}\right).\sqrt{\left(3-1\right)\left(4-\sqrt{15}\right)}=\left(\sqrt{5}+\sqrt{3}\right)\sqrt{2\left(4-\sqrt{15}\right)}\\ \text{\hspace{0.33em}\hspace{0.33em}\hspace{0.33em}}=\left(\sqrt{5}+\sqrt{3}\right).\sqrt{8-2\sqrt{15}}=\left(\sqrt{5}+\sqrt{3}\right)\sqrt{5-2\sqrt{5}\sqrt{3}+3}\\ \text{\hspace{0.33em}\hspace{0.33em}\hspace{0.33em}}=\left(\sqrt{5}+\sqrt{3}\right).\sqrt{{\left(\sqrt{5}-\sqrt{3}\right)}^{2}}=\left(\sqrt{5}+\sqrt{3}\right)\left(\sqrt{5}-\sqrt{3}\right)\text{\hspace{0.33em}}\left(do\text{\hspace{0.33em}}\sqrt{5}-\sqrt{3}>0\right)\\ \text{\hspace{0.33em}\hspace{0.33em}\hspace{0.33em}}=5-3=2\end{array}$$
Vậy A=2$$ A=2$$  là một số nguyên.

========================================================================

https://khoahoc.vietjack.com/thi-online/16-cau-trac-nghiem-toan-8-on-tap-chuong-4-co-dap-an-nhan-biet-thong-hieu


\textbf{{QUESTION}}

Cho các bất phương trình sau, đâu là bất phương trình bậc nhất một ẩn
A. 5x + 7 < 0
B. 0x + 6 > 0
C. $$ {x}^{2}\quad –\quad 2x\quad >\quad 0$$
D. x – 10 = 3

\textbf{{ANSWER}}

Đáp án A
Dựa vào định nghĩa bất phương trình bậc nhất một ẩn ta có:
Đáp án A là bất phương trình bậc nhất một ẩn.
Đáp án B không phải bất phương trình bậc nhất một ẩn vì a = 0.
Đáp án C không phải bất phương trình bậc nhất vì có $$ {x}^{2}$$
Đáp án D không phải bất phương trình vì đây là phương trình bậc nhất một ẩn

========================================================================

https://khoahoc.vietjack.com/thi-online/16-cau-trac-nghiem-toan-8-on-tap-chuong-4-co-dap-an-nhan-biet-thong-hieu


\textbf{{QUESTION}}

Giá trị x = 2 là nghiệm của bất phương trình nào sau đây?
A. 7 – x < 2x
B. 2x + 3 > 9
C. -4x ≥ x + 5
D. 5 – x > 6x – 12

\textbf{{ANSWER}}

Đáp án D
Thay x = 2 vào từng bất phương trình:
Đáp án A: 7 – 2 < 2.2 $$ \Leftrightarrow $$ 5 < 4 vô lý. Loại đáp án A.
Đáp án B: 2.2 + 3 > 9 $$ \Leftrightarrow $$ 7 > 9 vô lý. Loại đáp án B
Đáp án C: -4.2 ≥ 2 + 5 $$ \Leftrightarrow $$ -8 ≥ 7 vô lý. Loại đáp án C.
Đáp án D: 5 – 2 > 6.2 - 12 $$ \Leftrightarrow $$ 3 > 0 luôn đúng. Chọn đáp án D

========================================================================

https://khoahoc.vietjack.com/thi-online/16-cau-trac-nghiem-toan-8-on-tap-chuong-4-co-dap-an-nhan-biet-thong-hieu


\textbf{{QUESTION}}

Nghiệm của bất phương trình 7(3x + 5) >0 là:
A. x >35$$ x\quad >\frac{3}{5}$$
B. x ≤ -53$$ x\quad \le \quad -\frac{5}{3}$$
C. x ≥ -53$$ x\quad \ge \quad -\frac{5}{3}$$
D. x > -53$$ x\quad >\quad -\frac{5}{3}$$

\textbf{{ANSWER}}

Đáp án D
Vì 7 > 0 nên 7(3x + 5) ≥ 3 ⇔$$ \Leftrightarrow $$ 3x + 5 > 0 ⇔$$ \Leftrightarrow $$ 3x > -5 ⇔ x > -53$$ \Leftrightarrow \quad x\quad >\quad -\frac{5}{3}$$

========================================================================

https://khoahoc.vietjack.com/thi-online/16-cau-trac-nghiem-toan-8-on-tap-chuong-4-co-dap-an-nhan-biet-thong-hieu


\textbf{{QUESTION}}

Cho a > b. Bất đẳng thức nào tương đương với bất đẳng thức đã cho?
A. a – 3 > b – 3
B. -3a + 4 > -3b + 4
C. 2a + 3 < 2b + 3
D. -5b – 1 < -5a – 1

\textbf{{ANSWER}}

Đáp án A
+) Đáp án A: a > b ⇔$$ \Leftrightarrow $$ a – 3 > b – 3
Vậy ý A đúng chọn ý A
+) Đáp án B: -3a + 4 > -3b + 4 ⇔$$ \Leftrightarrow $$ -3a > -3b ⇔$$ \Leftrightarrow $$ a < b trái với giải thiết nên B sai
+) Đáp án C: 2a + 3 < 2b + 3 ⇔$$ \Leftrightarrow $$ 2a < 2b ⇔$$ \Leftrightarrow $$ a < b trái với giả thiết nên C sai.
+) Đáp án D: -5b – 1 < -5a – 1 ⇔$$ \Leftrightarrow $$ -5b < -5a ⇔$$ \Leftrightarrow $$ b > a trái với giả thiết nên D sai

========================================================================

https://khoahoc.vietjack.com/thi-online/16-cau-trac-nghiem-toan-8-on-tap-chuong-4-co-dap-an-nhan-biet-thong-hieu


\textbf{{QUESTION}}

Phương trình |2x – 5| = 1 có nghiệm là:
A. x = 3; x = 2
B. x =52; x = 2$$ x\quad =\frac{5}{2};\quad x\quad =\quad 2$$
C. x = 1; x = 2
D. x = 0,5; x = 1,5

\textbf{{ANSWER}}

Đáp án A
Giải phương trình: |2x – 5| = 1
TH1: 2x – 5 ≥ 0  ⇔x ≥52$$ \Leftrightarrow x\quad \ge \frac{5}{2}$$
⇒$$ \Rightarrow $$ |2x – 5| = 2x – 5 = 1 ⇔$$ \Leftrightarrow $$ 2x = 6 ⇔$$ \Leftrightarrow $$ x = 3 (tm)
TH2: 2x – 5 < 0 ⇔ x <52$$ \Leftrightarrow \quad x\quad <\frac{5}{2}$$
⇒$$ \Rightarrow $$ |2x – 5| = -2x + 5 = 1 ⇔$$ \Leftrightarrow $$ 2x = 4 ⇔$$ \Leftrightarrow $$ x = 2 (tm)
Vậy phương trình có hai nghiệm x = 3 và x = 2

========================================================================

https://khoahoc.vietjack.com/thi-online/10-bai-tap-ksm-dau-la-bien-co-chac-chan-bien-co-khong-the-bien-co-ngau-nhien-doi-voi-cac-hi


\textbf{{QUESTION}}

A. Biến cố ngẫu nhiên;
C. Biến cố không thể;
D. Không phải là biến cố.

\textbf{{ANSWER}}

Đáp án đúng là: A
“Một tháng có 30 ngày” là biến cố ngẫu nhiên vì một tháng có thể có 28; 29; 30 hoặc 31 ngày.
Chẳng hạn, tháng 4 có 30 ngày nhưng tháng 1 có 31 ngày.

========================================================================

https://khoahoc.vietjack.com/thi-online/10-bai-tap-ksm-dau-la-bien-co-chac-chan-bien-co-khong-the-bien-co-ngau-nhien-doi-voi-cac-hi


\textbf{{QUESTION}}

Tung một con xúc xắc. Biến cố “Xúc xắc xuất hiện mặt 9 chấm” là biến cố
A. Biến cố ngẫu nhiên;
B. Biến cố không thể;
C. Biến cố chắc chắn;
D. Không xác định.

\textbf{{ANSWER}}

Đáp án đúng là: B
Một con xúc xắc có 6 mặt được đánh số từ 1 đến 6. Do đó xúc xắc xuất hiện mặt 9 chấm là điều không bao giờ xảy ra.
Vì vậy biến cố “Xúc xắc xuất hiện mặt 9 chấm” là biến cố là biến cố không thể.

========================================================================

https://khoahoc.vietjack.com/thi-online/10-bai-tap-ksm-dau-la-bien-co-chac-chan-bien-co-khong-the-bien-co-ngau-nhien-doi-voi-cac-hi


\textbf{{QUESTION}}

Trong các sự kiện, hiện tượng sau, đâu là biến cố chắc chắn?
A. Mặt Trời quay quanh Trái Đất;
B.Khi gieo đồng xu thì được mặt sấp;
C. Có 12 cơn bão đổ bộ vào nước ta trong năm tới;
D. Ngày mai, Mặt Trời lặn ở phía Tây.

\textbf{{ANSWER}}

Đáp án đúng là: D
Biến cố A là biến cố là biến cố không thể vì Trái đất quay quanh Mặt trời.
Biến cố B là biến cố ngẫu nhiên vì ta không thể chắc chắn khi gieo đồng xu sẽ ra mặt sấp hay ngửa.
Biến cố C là biến cố ngẫu nhiên vì ta không thể chắc chắn năm tới nước ta sẽ có bao nhiêu cơn bão.
Biến cố D là biến cố chắc chắn vì Mặt trời luôn lặn ở phía Tây.

========================================================================

https://khoahoc.vietjack.com/thi-online/10-bai-tap-ksm-dau-la-bien-co-chac-chan-bien-co-khong-the-bien-co-ngau-nhien-doi-voi-cac-hi


\textbf{{QUESTION}}

Trong các sự kiện, hiện tượng sau, đâu là biến cố không thể?
A. Tuần sau trời sẽ rét;
B. Đội tuyển Mỹ vô địch mùa World Cup sắp tới;
C. Khi gieo hai con xúc xắc thì tổng số chấm xuất hiện trên cả hai con xúc xắc là 20;
D. Lượng mưa tại Hưng Yên năm tới là 3 000 mm.

\textbf{{ANSWER}}

Đáp án đúng là: C
Biến cố A là biến cố ngẫu nhiên vì ta không thể chắc chắn tuần sau trời sẽ rét hay không.
Biến cố B là biến cố ngẫu nhiên vì ta không thể chắc chắn mùa World Cup tiếp theo đội nào sẽ vô địch.
Biến cố C là biến cố không thể vì gieo 1 con xúc xắc, số chấm nhiều nhất đạt được là 6 chấm nên tổng số chấm xuất hiện trên cả hai con xúc xắc nhiều nhất đạt được là 12 chấm. Do đó không thể là 20 chấm.
Biến cố D là biến cố ngẫu nhiên vì ta không thể chắc chắn năm tới lượng mưa tại Hưng Yên sẽ là bao nhiêu.

========================================================================

https://khoahoc.vietjack.com/thi-online/10-bai-tap-ksm-dau-la-bien-co-chac-chan-bien-co-khong-the-bien-co-ngau-nhien-doi-voi-cac-hi


\textbf{{QUESTION}}

An lấy ngẫu nhiên một viên bi trong một túi đựng 3 viên bi xanh và 7 viên bi đỏ có cùng kích thước. Trong các biến cố sau, biến cố nào là biến cố chắc chắn?
A. “An lấy được viên bi xanh”;
B. “An lấy được viên bi đỏ”;
C. “An lấy viên bi màu xanh hoặc viên bi màu đỏ”;
D. “An lấy được viên bi vàng”.

\textbf{{ANSWER}}

Đáp án đúng là: C
Biến cố A, B là biến cố ngẫu nhiên vì nó có thể xảy ra hoặc không xảy ra.
Chẳng hạn, nếu An lấy được viên bi xanh thì biến cố A xảy ra, biến cố B không xảy ra, ngược lại nếu An lấy được viên bi đỏ thì biến cố B xảy ra, biến cố A không xảy ra.
Biến cố C là biến cố chắc chắn vì trong túi chỉ có viên bi xanh và viên bi đỏ nên viên bi An lấy ra có thể màu xanh hoặc màu đỏ.
Biến cố D là biến cố không thể vì trong túi không có viên bi nào màu vàng.

========================================================================

https://khoahoc.vietjack.com/thi-online/15-cau-trac-nghiem-toan-7-chan-troi-sang-tao-bai3-hinh-lang-tru-dung-tam-giac-hinh-lang-tru-dung-tu/105055


\textbf{{QUESTION}}

Một hình lăng trụ đứng có tất cả 5 mặt. Hình lăng trụ này có bao nhiêu đỉnh?

\textbf{{ANSWER}}

Hướng dẫn giải
Đáp án đúng là: B
Hình lăng trụ đứng có tất cả 5 mặt nên đây là hình lăng trụ đứng tam giác.
Hình lăng trụ đứng tam giác có tất cả 6 đỉnh.
Ta chọn đáp án B.

========================================================================

https://khoahoc.vietjack.com/thi-online/15-cau-trac-nghiem-toan-7-chan-troi-sang-tao-bai3-hinh-lang-tru-dung-tam-giac-hinh-lang-tru-dung-tu/105055


\textbf{{QUESTION}}

Hình lăng trụ đứng tứ giác có:
(1) Các mặt đáy song song với nhau;
(2) Các mặt đáy là tam giác;
(3) Các mặt đáy là tứ giác;
(4) Các mặt bên là hình chữ nhật.
Có bao nhiêu khẳng định đúng trong các khẳng định trên?

\textbf{{ANSWER}}

Hướng dẫn giải
Đáp án đúng là: C
Hình lăng trụ đứng tứ giác có các mặt đáy là tứ giác và song song với nhau, các mặt bên là các hình chữ nhật.
Do đó (1) (3) (4) đúng. 
Vậy có 3 khẳng định đúng.

========================================================================

https://khoahoc.vietjack.com/thi-online/de-kiem-tra-giua-ki-2-toan-8-co-dap-an-moi-nhat/94411


\textbf{{QUESTION}}

Giải phương trình:
a) 7 + 2x = 32 – 3x;
b) $$ \frac{x+4}{5}-x+4=\frac{x}{3}-\frac{x-2}{2}$$ ;
c) x2 + (x + 3)(x – 5) = 9;
d) $$ \frac{x+2}{x-2}-\frac{3}{x+2}+\frac{3x+10}{4-{x}^{2}}=0$$ .

\textbf{{ANSWER}}

a) 7 + 2x = 32 – 3x
Û 2x + 3x = 32 – 7 
Û 5x = 25
Û x = 5
b) $$ \frac{x+4}{5}-x+4=\frac{x}{3}-\frac{x-2}{2}$$
$$ \Leftrightarrow \frac{6(x+4)}{30}-\frac{30x}{30}+\frac{4.30}{30}=\frac{10.x}{30}-\frac{15(x-2)}{30}\phantom{\rule{0ex}{0ex}}\Leftrightarrow \frac{6x+24}{30}-\frac{30x}{30}+\frac{120}{30}=\frac{10x}{30}-\frac{15x-30}{30}\phantom{\rule{0ex}{0ex}}\Leftrightarrow \frac{6x+24-30x+120}{30}=\frac{10x-15x+30}{30}$$
 
 
 
Û 6x + 24 – 30x + 120 = 10x – 15x + 30
Û –24x + 144 = –5x + 30
Û 24x – 5x = 144 – 30
Û 19x = 114
Û x = 6
Vậy tập nghiệm của phương trình là S = {6};
c) x2 + (x + 3)(x – 5) = 9
Û x2 – 9 + (x + 3)(x – 5) = 0
Û (x – 3)(x + 3) + (x + 3)(x – 5) = 0
Û (x + 3) [(x – 3) + (x – 5)] = 0
Û (x + 3) (x – 3 + x – 5) = 0
Û (x + 3) (2x – 8) = 0
$$ \Leftrightarrow \left[\begin{array}{l}x\quad +\text{ 3 = }0\\ \text{2x}\quad –\text{ 8}=0\end{array}\right.\phantom{\rule{0ex}{0ex}}\Leftrightarrow \left[\begin{array}{l}x\quad \quad =-\text{ 3}\\ \text{2x}\quad =8\end{array}\right.\phantom{\rule{0ex}{0ex}}\Leftrightarrow \left[\begin{array}{l}x\quad \quad =-\text{ 3}\\ \text{x}\quad =4\end{array}\right.$$
 
 
 
d) $$ \frac{x+2}{x-2}-\frac{3}{x+2}+\frac{3x+10}{4-{x}^{2}}=0$$
Điều kiện xác định:
$$ \left\{\begin{array}{l}x-2\ne 0\\ x+2\ne 0\\ 4-{x}^{2}\ne 0\end{array}\right.\Leftrightarrow \left\{\begin{array}{l}x-2\ne 0\\ x+2\ne 0\\ \left(x-2\right)\left(x+2\right)\ne 0\end{array}\right.\Leftrightarrow \left\{\begin{array}{l}x\ne 2\\ x\ne -2\end{array}\right.$$
Ta có: $$ \frac{x+2}{x-2}-\frac{3}{x+2}+\frac{3x+10}{4-{x}^{2}}=0$$
 
 
 
 
 
$$ \Leftrightarrow \frac{x+2}{x-2}-\frac{3}{x+2}-\frac{3x+10}{{x}^{2}-4}=0\phantom{\rule{0ex}{0ex}}\Leftrightarrow \frac{x+2}{x-2}-\frac{3}{x+2}-\frac{3x+10}{{x}^{2}-4}=0\phantom{\rule{0ex}{0ex}}\Leftrightarrow \frac{\left(x+2\right)\left(x+2\right)}{\left(x-2\right)\left(x+2\right)}-\frac{3\left(x-2\right)}{\left(x-2\right)\left(x+2\right)}-\frac{3x+10}{\left(x-2\right)\left(x+2\right)}=0\phantom{\rule{0ex}{0ex}}\Leftrightarrow \frac{{\left(x+2\right)}^{2}}{\left(x-2\right)\left(x+2\right)}-\frac{3x-6}{\left(x-2\right)\left(x+2\right)}-\frac{3x+10}{\left(x-2\right)\left(x+2\right)}=0\phantom{\rule{0ex}{0ex}}\Leftrightarrow \frac{{\left(x+2\right)}^{2}-\left(3x-6\right)-\left(3x+10\right)}{\left(x-2\right)\left(x+2\right)}=0$$
Þ (x + 2)2 – (3x – 6) – (3x + 10) = 0
Û (x + 2)2 – 3x + 6 – 3x – 10 = 0
Û (x + 2)2 – 6x – 4 = 0
Û x2 + 4x + 4 – 6x – 4 =0
Û x2 – 2x = 0
Û x.(x – 2) = 0
$$ \Leftrightarrow \left[\begin{array}{c}x=0\\ x-2=0\end{array}\right.\phantom{\rule{0ex}{0ex}}\Leftrightarrow \left[\begin{array}{c}x=0\\ x=2\text{\hspace{0.17em}\hspace{0.17em}}\left(L\right)\end{array}\right.$$
 
 
Vậy tập nghiệm của phương trình là S = {0}.

========================================================================

https://khoahoc.vietjack.com/thi-online/de-kiem-tra-giua-ki-2-toan-8-co-dap-an-moi-nhat/94411


\textbf{{QUESTION}}

Một người đi xe đạp từ A đến B với vận tốc 12 km/h. Khi từ B trở về A người đó đi theo con đường khác ngắn hơn con đường cũ là 5 km và vận tốc nhỏ hơn vận tốc lúc đi là 2 km/h. Tính chiều dài quãng đường AB lúc đi biết thời gian lúc đi ít hơn thời gian lúc về là 40 phút.

\textbf{{ANSWER}}

Gọi x (km) là chiều dài quãng đường AB lúc đi (x > 0). 
Chiều dài quãng đường tắt từ B về A ngắn hơn đường lúc đi 5 km là x – 5 (km). 
Vận tốc lúc đi về từ B đến A nhỏ hơn vận tốc lúc đi là: 12 – 2 = 10 (km/h).
Thời gian người đi xe đạp đi hết quãng đường từ A đến B là: 
tAB = tAB=x12$$ {t}_{AB}=\frac{x}{12}$$  (h).
Thời gian người đi xe đạp đi hết quãng đường tắt từ B về A là:
tBA = x−510$$ \frac{x-5}{10}$$  (h).
Đổi 40 phút = 23$$ \frac{2}{3}$$  giờ.
Vì thời gian lúc đi từ A đến B ít hơn thời gian lúc đi từ B về A là 40 phút nên ta có phương trình:
x−510−x12=23⇔6(x−5)60−5x60=4060⇔6(x−5)−5x60=4060$$ \frac{x-5}{10}-\frac{x}{12}=\frac{2}{3}\phantom{\rule{0ex}{0ex}}\Leftrightarrow \frac{6\left(x-5\right)}{60}-\frac{5x}{60}=\frac{40}{60}\phantom{\rule{0ex}{0ex}}\Leftrightarrow \frac{6\left(x-5\right)-5x}{60}=\frac{40}{60}$$
 
 
 
Û 6x – 30 – 5x = 40
Û 6x – 5x = 40 + 30
Û x = 70  (thoản mãn)

========================================================================

https://khoahoc.vietjack.com/thi-online/bai-tap-toan-8-chu-de-14-he-thong-kiem-tra-chuong-2-co-dap-an/108575


\textbf{{QUESTION}}

Cặp phân thức nào sau đây không bằng nhau.

\textbf{{ANSWER}}

Chọn đáp án D

========================================================================

https://khoahoc.vietjack.com/thi-online/bai-tap-toan-8-chu-de-14-he-thong-kiem-tra-chuong-2-co-dap-an/108575


\textbf{{QUESTION}}

Mẫu thức chung có bậc nhỏ nhất của các phân thức:16x3y2 ;x2+3x9x2y4 ;x-14xy3$$ \frac{1}{6{x}^{3}{y}^{2}}\text{ };\frac{{x}^{2}+3x}{9{x}^{2}{y}^{4}}\text{ };\frac{x-1}{4x{y}^{3}}$$   là:
D.36x6y9$$ 36{x}^{6}{y}^{9}$$

\textbf{{ANSWER}}

Chọn đáp án B

========================================================================

https://khoahoc.vietjack.com/thi-online/bai-tap-toan-8-chu-de-14-he-thong-kiem-tra-chuong-2-co-dap-an/108575


\textbf{{QUESTION}}

Kết quả rút gọn phân thức x2−xy5y2−5xy$$ \frac{{x}^{2}-xy}{5{y}^{2}-5xy}$$ là :
D.  −2x5y$$ \frac{-2x}{5y}$$

\textbf{{ANSWER}}

Chọn đáp án C

========================================================================

https://khoahoc.vietjack.com/thi-online/bai-tap-toan-8-chu-de-14-he-thong-kiem-tra-chuong-2-co-dap-an/108575


\textbf{{QUESTION}}

hãy điền phân thức thích hợp vào chỗ ( .... ) để được đẳng thức đúng 35xy2+............=75x2y$$ \frac{3}{5x{y}^{2}}+\mathrm{............}=\frac{7}{5{x}^{2}y}$$
hãy điền phân thức thích hợp vào chỗ ( .... ) để được đẳng thức đúng 35xy2+............=75x2y$$ \frac{3}{5x{y}^{2}}+\mathrm{............}=\frac{7}{5{x}^{2}y}$$
35xy2+............=75x2y
35xy2+............=75x2y
35xy2
3
3
5xy2
5xy2


5xy2
5xy2
5xy2
5
x
y2
y
2
+
............
=
75x2y
7
7
5x2y
5x2y


5x2y
5x2y
5x2y
5
x2
x
2
y

\textbf{{ANSWER}}

7y−3x5x2y2$$ \frac{7y-3x}{5{x}^{2}{y}^{2}}$$

========================================================================

https://khoahoc.vietjack.com/thi-online/bai-tap-toan-8-chu-de-14-he-thong-kiem-tra-chuong-2-co-dap-an/108575


\textbf{{QUESTION}}

hãy điền phân thức thích hợp vào chỗ ( .... ) để được đẳng thức đúng. 5x+104x−8..........................=52$$ \frac{5x+10}{4x-8}.\frac{\mathrm{............}}{\mathrm{.............}}=\frac{5}{2}$$

\textbf{{ANSWER}}

2(x−2)x+2$$ \frac{2(x-2)}{x+2}$$

========================================================================

https://khoahoc.vietjack.com/thi-online/chuong-4-de-kiem-tra-chuong-4/59222


\textbf{{QUESTION}}

Với x ≥ 0, x ≠ 9, cho các biểu thức:
P = $$ \frac{2\sqrt{x}}{\sqrt{x}+3}+\frac{\sqrt{x}}{\sqrt{x}-3}-\frac{3x+3}{x-9}$$ và Q = $$ \frac{\sqrt{x}+1}{\sqrt{x}-3}$$
a, Tính giá trị của Q tại $$ x=7-4\sqrt{3}$$
b, Rút gọn P
c, Tìm x để $$ M\ge \frac{-2}{3}$$ biết $$ M=\frac{P}{Q}$$
d, Đặt A = $$ x.M+\frac{4x+7}{\sqrt{x}+3}$$. Tìm giá trị nhỏ nhất của A

\textbf{{ANSWER}}

a, Từ $$ x=7-4\sqrt{3}$$ tìm được $$ \sqrt{x}=2-\sqrt{3}$$. Thay vào Q và tính ta được Q = $$ \frac{\sqrt{3}-3}{1+\sqrt{3}}$$
b, P = $$ \frac{3\sqrt{x}+3}{9-x}$$
c, Tìm được $$ M=\frac{P}{Q}=\frac{-3}{\sqrt{x}+3}$$
Giải $$ M\ge \frac{-2}{3}$$ ta tìm được $$ \frac{9}{4}\le x\ne 9$$
d, Tìm được A = $$ \frac{x+7}{\sqrt{x}+3}$$
Ta có A = $$ \frac{\left(x+1\right)+6}{\sqrt{x}+3}\ge \frac{2\sqrt{x}+6}{\sqrt{x}+3}=2$$
Từ đó đi đến kết luận $$ {A}_{min}=2$$ => x = 1
* Cách khác: A = $$ \frac{x+7}{\sqrt{x}+3}=\sqrt{x}-3+\frac{16}{\sqrt{x}+3}$$
= $$ \sqrt{x}+3+\frac{16}{\sqrt{x}+3}-6\ge 2\sqrt{16}-6=2$$
=> Kết luận

========================================================================

https://khoahoc.vietjack.com/thi-online/chuong-4-de-kiem-tra-chuong-4/59222


\textbf{{QUESTION}}

Giải toán bằng cách lập phương trình hoặc hệ phương trình:
Theo kế hoạch hai tổ sản xuất phải làm được 900 chi tiết máy trong một thời gian quy định. Do cải tiến kĩ thuật nên tổ một vượt mức 15%, tổ hai vượt mức 10% so với kế hoạch. Vì vậy hai tổ sản xuất được 1010 chi tiết máy. Hỏi theo kế hoạch mỗi tổ sản xuất phải làm bao nhiêu chi tiết máy?

\textbf{{ANSWER}}

Gọi số chi tiết máy tổ một và hai sản xuất được lần lượt là x và y (x, y Î N*; x, y < 900)
Theo đề bài ta có hệ phương trình: {x+y=9001,15x+1,1y=1010$$ \left\{\begin{array}{l}x+y=900\\ 1,15x+1,1y=1010\end{array}\right.$$
Giải được x = 400 và y = 500
Vậy theo kế hoạch tổ một và hai phải sản xuất lần lượt 400 và 500 chi tiết máy

========================================================================

https://khoahoc.vietjack.com/thi-online/chuong-4-de-kiem-tra-chuong-4/59222


\textbf{{QUESTION}}

a, Giải hệ phương trình: {3x-2y+1=15x+2y+1=3$$ \left\{\begin{array}{l}3x-\frac{2}{y+1}=1\\ 5x+\frac{2}{y+1}=3\end{array}\right.$$
b, Cho phương trình x2$$ {x}^{2}$$ – (m – 1)x – m2$$ {m}^{2}$$ – 1 = 0 với x là ẩn và m là tham số. Tìm m để phương trình có hai nghiệm phân biệt x1$$ {x}_{1}$$, x2$$ {x}_{2}$$ thỏa mãn |x1|+|x2|=2√2$$ \left|{x}_{1}\right|+\left|{x}_{2}\right|=2\sqrt{2}$$

\textbf{{ANSWER}}

a, Cách 1. Đặt 1y+1=u$$ \frac{1}{y+1}=u$$ ta được {3x-2u=15x+2u=3$$ \left\{\begin{array}{l}3x-2u=1\\ 5x+2u=3\end{array}\right.$$
Giải ra ta được x=12;u=14$$ x=\frac{1}{2};u=\frac{1}{4}$$
Từ đó tìm được y = 3
Cách 2. Cộng vế với vế hai phương trình, ta được 8x = 4
Từ đó tìm được x=12$$ x=\frac{1}{2}$$ và y = 3
b, Vì x1x2 = -m2 - 1 < 0 "m nên phương trình đã cho luôn có hai nghiệm phân biệt và trái dấu.
Cách 1. Giả sử  x1$$ {x}_{1}$$ < 0 < x2$$ {x}_{2}$$
Từ giả thiết thu được – x1$$ {x}_{1}$$+ x2$$ {x}_{2}$$ = 2√2$$ 2\sqrt{2}$$
Biến đổi thành (x1+x2)2-4x1x2=8$$ {\left({x}_{1}+{x}_{2}\right)}^{2}-4{x}_{1}{x}_{2}=8$$
Áp dụng định lý Vi-ét, tìm được m = 1 hoặc m = -35$$ \frac{-3}{5}$$
Cách 2. Bình phương hai vế của giả thiết và biến đổi về dạng
(x1+x2)2-2x1x2+2|x1x2|=8$$ {\left({x}_{1}+{x}_{2}\right)}^{2}-2{x}_{1}{x}_{2}+2\left|{x}_{1}{x}_{2}\right|=8$$
=> (m-1)2+4(m2+1)=8$$ {\left(m-1\right)}^{2}+4\left({m}^{2}+1\right)=8$$
Do |x1x2|=-x1x2$$ \left|{x}_{1}{x}_{2}\right|=-{x}_{1}{x}_{2}$$
Áp dụng hệ thức Vi-ét, ta cũng tìm được m = 1 hoặc m = -35$$ \frac{-3}{5}$$

========================================================================

https://khoahoc.vietjack.com/thi-online/giai-vth-toan-7-bai-10-tien-de-euclid-tinh-chat-hai-duong-thang-song-song-co-dap-an


\textbf{{QUESTION}}

Tiên đề Euclid được phát biểu: “Qua một điểm M nằm ngoài đường thẳng a ...”
A. có duy nhất một đường thẳng đi qua M và song song với a.
B. có hai đường thẳng song song với a.
C. có ít nhất một đường thẳng song song với a.
D. có vô số đường thẳng song song với a.

\textbf{{ANSWER}}

Đáp án đúng là A
Tiên đề Euclid được phát biểu: “Qua một điểm M nằm ngoài đường thẳng a có duy nhất một đường thẳng đi qua M và song song với a”.

========================================================================

https://khoahoc.vietjack.com/thi-online/10-bai-tap-nhan-biet-da-thuc-hang-tu-cua-da-thuc-bac-cua-da-thuc-co-loi-giai


\textbf{{QUESTION}}

Trong các biểu thức dưới đây, biểu thức nào là một đa thức?
A. 3x + 1;
B. $$ x+\frac{1}{x}$$;
C. $$ \sqrt{2x}+{y}^{2}$$;
D. $$ ‐2x+\frac{x+1}{x-1}$$.

\textbf{{ANSWER}}

Hướng dẫn giải:
Đáp án đúng là: A
3x + 1 là một đa thức do nó là tổng của hai đơn thức là 3x và 1.
Các biểu thức $$ x+\frac{1}{x};\sqrt{2x}+{y}^{2};‐2x+\frac{x+1}{x-1}$$ không phải đa thức do chúng lần lượt chứa các hạng tử $$ \frac{1}{x};\sqrt{2x};\quad \frac{x+1}{x-1}$$ không phải là đơn thức.

========================================================================

https://khoahoc.vietjack.com/thi-online/10-bai-tap-nhan-biet-da-thuc-hang-tu-cua-da-thuc-bac-cua-da-thuc-co-loi-giai


\textbf{{QUESTION}}

Trong các biểu thức dưới đây, biểu thức nào không là một đa thức?
A. 115x2+3x3y2+1$$ \frac{1}{15}{x}^{2}+3{x}^{3}{y}^{2}+1$$;
B. x3+3x+√2$$ {x}^{3}+3x+\sqrt{2}$$;
C. √2x-2y2+z2$$ \sqrt{2x}-2{y}^{2}+{z}^{2}$$;
D. 1x-1y+2y-1$$ \frac{1}{x}-\frac{1}{y}+2y-1$$.

\textbf{{ANSWER}}

Hướng dẫn giải:
Đáp án đúng là: D
Biểu thức $$ \frac{1}{x}-\frac{1}{y}+2y-1$$ không là đơn thức vì biểu thức này chứa hạng tử $$ \frac{1}{x}$$ và  $$ ‐\frac{1}{y}$$không là đơn thức.

========================================================================

https://khoahoc.vietjack.com/thi-online/10-bai-tap-nhan-biet-da-thuc-hang-tu-cua-da-thuc-bac-cua-da-thuc-co-loi-giai


\textbf{{QUESTION}}

Trong các biểu thức dưới đây, có bao nhiêu biểu thức là một đa thức?
12xy3-3x2y; ‐2x; x2+2xy+0,5y2;1x2-2xy+y;0$$ \frac{1}{2}x{y}^{3}-3{x}^{2}y;\quad ‐2x;\quad {x}^{2}+2xy+0,5{y}^{2};\frac{1}{{x}^{2}}-2xy+y;0$$.
A. 2;
B. 3;

\textbf{{ANSWER}}

Hướng dẫn giải:
Đáp án đúng là: C
Có 4 đa thức là: 12xy3-3x2y; ‐2x; x2+2xy+0,5y2$$ \frac{1}{2}x{y}^{3}-3{x}^{2}y;\quad ‐2x;\quad {x}^{2}+2xy+0,5{y}^{2}$$ và 0.
Biểu thức 1x2-2xy+y$$ \frac{1}{{x}^{2}}-2xy+y$$ không phải một đa thức do có chứa hạng tử 1x2$$ \frac{1}{{x}^{2}}$$ không phải một đơn thức.

========================================================================

https://khoahoc.vietjack.com/thi-online/de-kiem-tra-giua-hoc-ki-2-mon-toan-9-co-dap-an-moi-nhat/91652


\textbf{{QUESTION}}

Các cặp số (x; y) sau, cặp nào là nghiệm của phương trình x + 2y = 3?
A. (3; −2)
B. (0; 1)
C. (1; 0)
D. (1; 1)

\textbf{{ANSWER}}

Đáp án đúng là: D
+ Thay các giá trị x = 3 và y = −2 vào phương trình x + 2y = 3, ta được:
 3 + 2 . (−2) = 3 – 4 = –1 ≠ 3.
Do đó cặp số (3; −2) không phải là nghiệm của phương trình x + 2y = 3.
+ Thay các giá trị x = 0 và y = 1 vào phương trình x + 2y = 3, ta được:
 0 + 2 . 1 = 2 ≠ 3.
Do đó cặp số (0; 1) không phải là nghiệm của phương trình x + 2y = 3.
+ Thay các giá trị x = 1 và y = 0 vào phương trình x + 2y = 3, ta được:
 1 + 2 . 0 = 1 ≠ 3.
Do đó cặp số (1; 0) không phải là nghiệm của phương trình x + 2y = 3.
+ Thay các giá trị x = 1 và y = 1 vào phương trình x + 2y = 3, ta được:
 1 + 2 . 1 = 1 + 2 = 3.
Do đó cặp số (1; 1) không phải là nghiệm của phương trình x + 2y = 3.

========================================================================

https://khoahoc.vietjack.com/thi-online/de-kiem-tra-giua-hoc-ki-2-mon-toan-9-co-dap-an-moi-nhat/91652


\textbf{{QUESTION}}

Cặp số (x; y) nào sau đây là nghiệm của hệ phương trình $$ \{\begin{array}{l}x+3y=5\\ 2x-y=3\end{array}$$
A. (2; 1)
B. (1; 2)
C. (−2; 1)
D. (1; −2)

\textbf{{ANSWER}}

Đáp án đúng là: A
$$ \{\begin{array}{l}x+3y=5\\ 2x-y=3\end{array}$$ 
Û $$ \{\begin{array}{l}x=5-3y\\ 2x-y=3\end{array}$$ 
Û $$ \{\begin{array}{l}x=5-3y\\ 2.(5-3y)-y=3\end{array}$$ 
Û $$ \{\begin{array}{l}x=5-3y\\ -7y=-7\end{array}$$
 Û $$ \{\begin{array}{l}x=2\\ y=1\end{array}$$
Vậy cặp số (2; 1) là nghiệm của phương hệ trình.

========================================================================

https://khoahoc.vietjack.com/thi-online/de-kiem-tra-giua-hoc-ki-2-mon-toan-9-co-dap-an-moi-nhat/91652


\textbf{{QUESTION}}

Tọa độ giao điểm của hai đường thẳng y = 2x – 3 và y = x – 1 là:
A. (−2; 1)
B. (1; 2)
C. (2; 1)
D. (1; −2)

\textbf{{ANSWER}}

Đáp án đúng là: C
Phương trình hoành độ giao điểm của hai đường thẳng là: 
2x – 3 = x – 1 
Û 2x – x = 3 – 1 
Û x = 2. 
Thay x = 2 vào phương trình đường thẳng y = 2x – 3 = 2.2 – 3 = 1.
Vậy tọa độ giao điểm của hai đường thẳng là (2; 1).

========================================================================

https://khoahoc.vietjack.com/thi-online/de-kiem-tra-giua-hoc-ki-2-mon-toan-9-co-dap-an-moi-nhat/91652


\textbf{{QUESTION}}

Giá trị nào của a thì hệ {ax+y=1x+y=a$$ \{\begin{array}{l}ax+y=1\\ x+y=a\end{array}$$ có vô số nghiệm.
A. −1
B. 1
C. ± 1          
D. 2

\textbf{{ANSWER}}

Đáp án đúng là: B
Để hệ phương trình có vô số nghiệm thì: a1=11=1a$$ \frac{a}{1}=\frac{1}{1}=\frac{1}{a}$$ ⇒ a = 1.

========================================================================

https://khoahoc.vietjack.com/thi-online/de-kiem-tra-giua-hoc-ki-2-mon-toan-9-co-dap-an-moi-nhat/91652


\textbf{{QUESTION}}

Hai tủ sách có 450 quyển sách, nếu chuyển 50 quyển từ tủ một sang tủ hai thì hai tủ có số sách bằng nhau. Số sách của tủ một là:
A. 200
B. 250
C. 275
D. 300

\textbf{{ANSWER}}

Đáp án đúng là: C
Gọi x là số sách của tủ 1, y là số sách của tủ 2 (điều kiện: x, y ∈ ℕ, 0 < x, y < 450).
Vì hai tủ sách có 450 quyển sách nên: x + y = 450 (1)
Chuyển 50 quyển từ tủ một sang tủ hai thì hai tủ có số sách bằng nhau nên ta có phương trình: x – 50 = y + 50 (2)
Từ (1) và (2), ta có hệ phương trình:
{x+y=450x−50=y+50$$ \{\begin{array}{l}x+y=450\\ x-50=y+50\end{array}$$ 
Û {x+y=450x=y+100$$ \{\begin{array}{l}x+y=450\\ x=y+100\end{array}$$ 
Û {y+100+y=450x=y+100$$ \{\begin{array}{l}y+100+y=450\\ x=y+100\end{array}$$ 
Û {y=175x=275$$ \{\begin{array}{l}y=175\\ x=275\end{array}$$ (thỏa mãn)
Vậy số sách của tủ một là 275 quyển.

========================================================================

https://khoahoc.vietjack.com/thi-online/de-thi-toan-lop-6-hoc-ki-1-nam-2020-2021-cuc-hay-co-dap-an/60895


\textbf{{QUESTION}}

A.Trắc nghiệm 
Cho biết $$ \frac{15}{x}=-\frac{3}{4}$$ khi đó x bằng?
A. 20
B. -20
C. 63
D. 57

\textbf{{ANSWER}}

Chọn B
Ta có: $$ \frac{15}{x}=-\frac{3}{4}\Rightarrow x=15:\left(-\frac{3}{4}\right)=-20$$

========================================================================

https://khoahoc.vietjack.com/thi-online/de-thi-toan-lop-6-hoc-ki-1-nam-2020-2021-cuc-hay-co-dap-an/60895


\textbf{{QUESTION}}

Phân số tối giản của phân số 20−140$$ \frac{20}{-140}$$ là:
A. 10−70$$ \frac{10}{-70}$$
B. 4−28$$ \frac{4}{-28}$$
C. −214$$ \frac{-2}{14}$$
D. −17$$ \frac{-1}{7}$$

\textbf{{ANSWER}}

Chọn D

========================================================================

https://khoahoc.vietjack.com/thi-online/de-thi-toan-lop-6-hoc-ki-1-nam-2020-2021-cuc-hay-co-dap-an/60895


\textbf{{QUESTION}}

Kết quả phép tính −14−−53$$ -\frac{1}{4}-\frac{-5}{3}$$ bằng?
A. −1712$$ \frac{-17}{12}$$
B. −512$$ \frac{-5}{12}$$
C. 1712$$ \frac{17}{12}$$
D. 512$$ \frac{5}{12}$$

\textbf{{ANSWER}}

Chọn C
Ta có: −14−−53=−14+53=−312+2012=1712$$ -\frac{1}{4}-\frac{-5}{3}=-\frac{1}{4}+\frac{5}{3}=\frac{-3}{12}+\frac{20}{12}=\frac{17}{12}$$

========================================================================

https://khoahoc.vietjack.com/thi-online/de-thi-toan-lop-6-hoc-ki-1-nam-2020-2021-cuc-hay-co-dap-an/60895


\textbf{{QUESTION}}

Cho ^xOy$$ \widehat{xOy}$$ và ^mOn$$ \widehat{mOn}$$ là hai góc phụ nhau. Biết ^mOn=72°$$ \widehat{mOn}=72°$$. Khi đó góc bằng?
A. 30°$$ 30°$$
B. 28°$$ 28°$$
C. 22°$$ 22°$$
D. 18°$$ 18°$$

\textbf{{ANSWER}}

Chọn D
Hai góc phụ nhau là hai góc có tổng bằng 90°$$ 90°$$
^xOy+^mOn=900⇒^xOy=900−^mOn=900−720=180$$ \widehat{xOy}+\widehat{mOn}={90}^{0}\Rightarrow \widehat{xOy}={90}^{0}-\widehat{mOn}={90}^{0}-{72}^{0}={18}^{0}$$

========================================================================

https://khoahoc.vietjack.com/thi-online/bo-10-de-thi-giua-ki-2-toan-8-chan-troi-sang-tao-cau-truc-moi-co-dap-an/162535


\textbf{{QUESTION}}

Hàm số nào sau đây không là hàm số bậc nhất?

\textbf{{ANSWER}}

Đáp án đúng là: C
Hàm số bậc nhất có dạng $y = ax + b{\rm{ }}\left( {a \ne 0} \right)$.
Do đó, $y = \frac{2}{3}{x^2} + x - 1$ không là hàm số bậc nhất.

========================================================================

https://khoahoc.vietjack.com/thi-online/bo-10-de-thi-giua-ki-2-toan-8-chan-troi-sang-tao-cau-truc-moi-co-dap-an/162535


\textbf{{QUESTION}}

Đồ thị hàm số y=ax(a≠0)$y = ax{\rm{ }}\left( {a \ne 0} \right)$ là một đường thẳng luôn đi qua
A. gốc tọa độ O(0;0).$O\left( {0;0} \right).$                                   
 B. điểm A(1;0).$A\left( {1;0} \right).$

\textbf{{ANSWER}}

Đáp án đúng là: A
Đồ thị hàm số $y = ax{\rm{ }}\left( {a \ne 0} \right)$ là một đường thẳng luôn đi qua gốc tọa độ $O\left( {0;0} \right)$ với mọi $a \ne 0$.

========================================================================

https://khoahoc.vietjack.com/thi-online/bo-10-de-thi-giua-ki-2-toan-8-chan-troi-sang-tao-cau-truc-moi-co-dap-an/162535


\textbf{{QUESTION}}

Đường thẳng y=−3x−2022$y = - 3x - 2022$ tạo với trục Ox$Ox$ một góc như thế nào?

\textbf{{ANSWER}}

Đáp án đúng là: A
Do đường thẳng y=−3x−2022$y = - 3x - 2022$ có hệ số góc a=−3<0$a = - 3 < 0$ nên đường thẳng tạo với trục Ox$Ox$ một góc tù.

========================================================================

https://khoahoc.vietjack.com/thi-online/bo-10-de-thi-giua-ki-2-toan-8-chan-troi-sang-tao-cau-truc-moi-co-dap-an/162535


\textbf{{QUESTION}}

Trong các phát biểu sau, phát biểu nào sai?
A. Đồ thị hàm số y=ax+b(a≠0;b≠0)$y = ax + b\left( {a \ne 0;b \ne 0} \right)$ là đường thẳng cắt đường thẳng y=ax.$y = ax.$
B. Đồ thị hàm số y=ax+b(a≠0)$y = ax + b\left( {a \ne 0} \right)$ là đường thẳng song song với đường thẳng y=ax$y = ax$ nếu b≠0$b \ne 0$ và trùng với đường thẳng y=ax$y = ax$ nếu b=0.$b = 0.$
C. Đồ thị hàm số y=ax+b(a≠0;b≠0)$y = ax + b\left( {a \ne 0;b \ne 0} \right)$ là đường thẳng cắt trục tung tại điểm có tung độ bằng b=0.$b = 0.$

\textbf{{ANSWER}}

Đáp án đúng là: A
Đồ thị hàm số y=ax+b(a≠0;b≠0)$y = ax + b\left( {a \ne 0;b \ne 0} \right)$ là đường thẳng song song với đường thẳng y=ax.$y = ax.$

========================================================================

https://khoahoc.vietjack.com/thi-online/bo-10-de-thi-giua-ki-2-toan-8-chan-troi-sang-tao-cau-truc-moi-co-dap-an/162535


\textbf{{QUESTION}}

Đưa phương trình 5x−(6−x)=12$5x - \left( {6 - x} \right) = 12$ về dạng ax+b=0$ax + b = 0$ ta được

\textbf{{ANSWER}}

Đáp án đúng là: D
Ta có: 5x−(6−x)=12$5x - \left( {6 - x} \right) = 12$
5x−6+x−12=0$5x - 6 + x - 12 = 0$
6x−18=0$6x - 18 = 0$.

========================================================================

https://khoahoc.vietjack.com/thi-online/20-cau-trac-nghiem-toan-12-canh-dieu-bai-1-khoang-bien-thien-khoang-tu-phan-vi-cua-mau-so-lieu-ghep


\textbf{{QUESTION}}

I. Nhận biết
Gọi ${Q_1},{Q_2},{Q_3}$ là tứ phân vị thứ nhất, tứ phân vị thứ hai và thứ ba của mẫu số liệu ghép nhóm. Khoảng tứ phân vị của mẫu số liệu ghép nhóm là:
A. $\Delta Q = {Q_1} - {Q_3}.$
B. $\Delta Q = {Q_3} - {Q_1}.$
C. $\Delta Q = {Q_1} - {Q_2}.$
D. $\Delta Q = {Q_2} - {Q_1}.$

\textbf{{ANSWER}}

Đáp án đúng là: B
Khoảng tứ phân vị của mẫu ghép nhóm có công thức là: $\Delta Q = {Q_3} - {Q_1}.$

========================================================================

https://khoahoc.vietjack.com/thi-online/54-bai-tap-chuyen-de-toan-11-bai-3-vi-phan-dao-ham-cap-cao-co-dap-an


\textbf{{QUESTION}}

Cho hàm số $$ y={x}^{3}-4{x}^{2}-5$$ . Tính vi phân của hàm số tại điểm $$ {x}_{0}=1$$  , ứng với số gia $$ \Delta x=\mathrm{0,02}$$ .

\textbf{{ANSWER}}

Ta có $$ {y}^{\text{'}}={f}^{\text{'}}\left(x\right)=3{x}^{2}-4x$$  . Do đó vi phân của hàm số tại điểm $$ {x}_{0}=1$$ ,ứng với số gia $$ \Delta x=\mathrm{0,02}$$  là $$ df\left(1\right)={f}^{\text{'}}\left(1\right).\Delta x=\left({3.1}^{2}-4.1\right)\mathrm{.0,02}=-\mathrm{0,02}$$ .

========================================================================

https://khoahoc.vietjack.com/thi-online/54-bai-tap-chuyen-de-toan-11-bai-3-vi-phan-dao-ham-cap-cao-co-dap-an


\textbf{{QUESTION}}

Tìm vi phân của hàm số $$ y=\frac{x}{{x}^{2}+1}$$

\textbf{{ANSWER}}

Ta có $$ {y}^{\text{'}}={\left(\frac{x}{{x}^{2}+1}\right)}^{\text{'}}=\frac{{x}^{2}+1-2{x}^{2}}{{\left({x}^{2}+1\right)}^{2}}=\frac{-{x}^{2}+1}{{\left({x}^{2}+1\right)}^{2}}\Rightarrow dy={y}^{\text{'}}dx=\frac{-{x}^{2}+1}{{\left({x}^{2}+1\right)}^{2}}dx$$  .

========================================================================

https://khoahoc.vietjack.com/thi-online/54-bai-tap-chuyen-de-toan-11-bai-3-vi-phan-dao-ham-cap-cao-co-dap-an


\textbf{{QUESTION}}

Tính gần đúng giá trị của  √49,25$$ \sqrt{\mathrm{49,25}}$$ (lấy 5 chữ số thập phân trong kết quả).

\textbf{{ANSWER}}

Ta có √49,25=√49+0,25$$ \sqrt{\mathrm{49,25}}=\sqrt{49+\mathrm{0,25}}$$ .
Xét hàm số f(x)=√x⇒f'(x)=12√x$$ f\left(x\right)=\sqrt{x}\Rightarrow {f}^{\text{'}}\left(x\right)=\frac{1}{2\sqrt{x}}$$ .
Chọn x0=49$$ {x}_{0}=49$$  và Δx=0,25$$ \Delta x=\mathrm{0,25}$$ , ta có
 
 
f(x0+Δx)≈f(x0)+f'(x0).Δx$$ f\left({x}_{0}+\Delta x\right)\approx f\left({x}_{0}\right)+{f}^{\text{'}}\left({x}_{0}\right).\Delta x$$                 
⇒√49+0,25≈√49+12√49.0,25=7+0,01786$$ \Rightarrow \sqrt{49+\mathrm{0,25}}\approx \sqrt{49}+\frac{1}{2\sqrt{49}}\mathrm{.0,25}=7+\mathrm{0,01786}$$         =7,01786$$ =\mathrm{7,01786}$$

========================================================================

https://khoahoc.vietjack.com/thi-online/54-bai-tap-chuyen-de-toan-11-bai-3-vi-phan-dao-ham-cap-cao-co-dap-an


\textbf{{QUESTION}}

Tính gần đúng 10,9995$$ \frac{1}{\mathrm{0,9995}}$$ .
10,9995
10,9995
10,9995
1
1
0,9995
0,9995


0,9995
0,9995
0,9995

\textbf{{ANSWER}}

Ta có 10,9995=11−0,0005$$ \frac{1}{\mathrm{0,9995}}=\frac{1}{1-\mathrm{0,0005}}$$ .
Xét hàm số f(x)=1x⇒f'(x)=−1x2$$ f\left(x\right)=\frac{1}{x}\Rightarrow {f}^{\text{'}}\left(x\right)=-\frac{1}{{x}^{2}}$$ .
Chọn x0=1$$ {x}_{0}=1$$ và Δx=−0,0005$$ \Delta x=-\mathrm{0,0005}$$, ta có f(x0+Δx)≈f(x0)+f'(x0).Δx$$ f\left({x}_{0}+\Delta x\right)\approx f\left({x}_{0}\right)+{f}^{\text{'}}\left({x}_{0}\right).\Delta x$$
⇒11−0,0005≈1−1.(−0,0005)≈1,0005$$ \Rightarrow \frac{1}{1-\mathrm{0,0005}}\approx 1-1.\left(-\mathrm{0,0005}\right)\approx \mathrm{1,0005}$$

========================================================================

https://khoahoc.vietjack.com/thi-online/54-bai-tap-chuyen-de-toan-11-bai-3-vi-phan-dao-ham-cap-cao-co-dap-an


\textbf{{QUESTION}}

Tính gần đúng sin46°$$ \mathrm{sin}46°$$ .

\textbf{{ANSWER}}

Ta có sin46°=sin(45°+1°)=sin(π4+π180)$$ \mathrm{sin}46°=\mathrm{sin}\left(45°+1°\right)=\mathrm{sin}\left(\frac{\pi }{4}+\frac{\pi }{180}\right)$$ .
Xét hàm số f(x)=sinx⇒f'(x)=cosx$$ f\left(x\right)=\mathrm{sin}x\Rightarrow {f}^{\text{'}}\left(x\right)=\mathrm{cos}x$$ .
Chọn x0=π4$$ {x}_{0}=\frac{\pi }{4}$$  và Δx=π180$$ \Delta x=\frac{\pi }{180}$$ , ta có f(x0+Δx)≈f(x0)+f'(x0).Δx$$ f\left({x}_{0}+\Delta x\right)\approx f\left({x}_{0}\right)+{f}^{\text{'}}\left({x}_{0}\right).\Delta x$$ 
⇒sin(π4+π180)≈sinπ4+cosπ4.π180=√22+√2π360$$ \Rightarrow \mathrm{sin}\left(\frac{\pi }{4}+\frac{\pi }{180}\right)\approx \mathrm{sin}\frac{\pi }{4}+\mathrm{cos}\frac{\pi }{4}.\frac{\pi }{180}=\frac{\sqrt{2}}{2}+\frac{\sqrt{2\pi }}{360}$$.

========================================================================

https://khoahoc.vietjack.com/thi-online/on-tap-so-nguyen-cuc-hay-co-loi-giai/58435


\textbf{{QUESTION}}

Số nào chia hết cho 2 và 5?
A. 50
B. 48
C. 77
D. 94

\textbf{{ANSWER}}

Chọn A. 
Số chia hết cho 2 và 5 là các số có tận cùng là 0. 
Trong tập hợp  có số 50 có tận cùng là 0.
Vậy số chia hết cho 2 và 5là 50.

========================================================================

https://khoahoc.vietjack.com/thi-online/on-tap-so-nguyen-cuc-hay-co-loi-giai/58435


\textbf{{QUESTION}}

Dãy số nào chia hết cho 2 và 5?
A.13, 44, 76
B. 40, 57, 99
C. 70, 25, 160
D. 40, 80, 100

\textbf{{ANSWER}}

Chọn D. 
Xét thấy :
40, 80, 100 có chữ số tận cùng là 0  nên 40, 80, 100 là dãy số chia hết cho 2 và 5.
Vậy chọn D.

========================================================================

https://khoahoc.vietjack.com/thi-online/on-tap-so-nguyen-cuc-hay-co-loi-giai/58435


\textbf{{QUESTION}}

Số nào không chia hết cho 5 và 2?
A. 100
B. 215
C. 320
D. 790

\textbf{{ANSWER}}

Chọn B. 
Ta thấy 215 có chữ số tận cùng là 5, khác chữ số 0 .
215 là số không chia hết cho 2 và 5.

========================================================================

https://khoahoc.vietjack.com/thi-online/on-tap-so-nguyen-cuc-hay-co-loi-giai/58435


\textbf{{QUESTION}}

Số nào chia hết cho 10?
A. 33
B. 45
C. 130
D. 447

\textbf{{ANSWER}}

Chọn C.
Xét thấy số chia hết cho 10 phải là số chia hết cho 2 và 5(do 10= 2.5).
Vậy số chia hết cho 10 có chữ số tận cùng là 0 nên 130 là số chia hết cho 10.

========================================================================

https://khoahoc.vietjack.com/thi-online/on-tap-so-nguyen-cuc-hay-co-loi-giai/58435


\textbf{{QUESTION}}

Tìm chữ số thích hợp để điền vào chỗ ... để số 28…¯ chia hết cho 5 và 2.
A. 0
B. 2
C. 5
D.  Cả A và C

\textbf{{ANSWER}}

Chọn A
Số chia hết cho 5 và 2 là số có tận cùng là 0 .
Mà ... là chữ số tận cùng của 28…¯ nên có thể là 0  để 28…¯  
chia hết cho 5. Ta được số 280.

========================================================================

https://khoahoc.vietjack.com/thi-online/7-cau-trac-nghiem-toan-6-chan-troi-sang-tao-bai-9-uoc-va-boi-co-dap-an


\textbf{{QUESTION}}

Trong các số sau, số nào là ước của 12?
A. 5      
B. 8                       
C. 12               
D. 24

\textbf{{ANSWER}}

Ư(12) ={1;2;3;4;6;12}
Đáp án cần chọn là: C

========================================================================

https://khoahoc.vietjack.com/thi-online/7-cau-trac-nghiem-toan-6-chan-troi-sang-tao-bai-9-uoc-va-boi-co-dap-an


\textbf{{QUESTION}}

Tìm tất cả các các bội của 3 trong các số sau: 4;18;75;124;185;258
A. {5;75;124}
B. {18;124;258}
C. {75;124;258}
D. {18;75;258}

\textbf{{ANSWER}}

Vì  8⁝3; 75⁝3; 258⁝3 nên đáp án đúng là D.
Đáp án cần chọn là: D

========================================================================

https://khoahoc.vietjack.com/thi-online/7-cau-trac-nghiem-toan-6-chan-troi-sang-tao-bai-9-uoc-va-boi-co-dap-an


\textbf{{QUESTION}}

Khẳng định nào sau đây sai?
Với a là số tự nhiên khác 0 thì: 
A. a là ước của a
B. a là bội của a
C. 0 là ước của a
D. 1 là ước của a

\textbf{{ANSWER}}

Đáp án C sai vì không có số nào chia được cho 0.
0 không bao giờ là ước của một số tự nhiên bất kì.
Đáp án cần chọn là: C

========================================================================

https://khoahoc.vietjack.com/thi-online/7-cau-trac-nghiem-toan-6-chan-troi-sang-tao-bai-9-uoc-va-boi-co-dap-an


\textbf{{QUESTION}}

5 là phần tử của 
A. Ư(14)
B. Ư(15)
C. Ư(16)
D. Ư(17)

\textbf{{ANSWER}}

Ta có: Ư(15) là tập hợp các ước của 15.
Mà 5 là một ước của 15 nên 5 là phần tử của Ư(15)
Đáp án cần chọn là: B

========================================================================

https://khoahoc.vietjack.com/thi-online/7-cau-trac-nghiem-toan-6-chan-troi-sang-tao-bai-9-uoc-va-boi-co-dap-an


\textbf{{QUESTION}}

Số 26 không là phần tử của 
A. B(2)
B. B(13)
C. B(26)
D. B(3)

\textbf{{ANSWER}}

Ta có 26 chia hết cho 2, 13, 26 nên 26 là bội của 3 số này. Hay 26 là phần tử của B(2), B(13), B(26).
26 không chia hết cho 3 nên 26 không là bội của 3.
Vậy 26 không là phần tử của B(3)
Đáp án cần chọn là: D

========================================================================

https://khoahoc.vietjack.com/thi-online/bai-tap-trung-diem-cua-doan-thang


\textbf{{QUESTION}}

Gọi M là trung điểm của đoạn thẳng AB. Tính độ dài hai đoạn thẳng AM và BM, biết AB = 4cm.

\textbf{{ANSWER}}

AM = BM = 4/2 =2 cm

========================================================================

https://khoahoc.vietjack.com/thi-online/bai-tap-trung-diem-cua-doan-thang


\textbf{{QUESTION}}

Gọi C là trung điểm của đoạn thẳng AB. Tính độ dài hai đoạn thẳng AC và BC, biết AB = 6cm.

\textbf{{ANSWER}}

AC = BC = 6/3 = 2 cm

========================================================================

https://khoahoc.vietjack.com/thi-online/bai-tap-trung-diem-cua-doan-thang


\textbf{{QUESTION}}

Cho điểm O thuộc đường thẳng xy. Trên tia Ox, lấy điểm M sao cho OM = 4 cm. Trên tia Oy, lấy điểm N sao cho ON = 2cm. Gọi A và B lần lượt là trung điểm của OM và ON.
Chứng tỏ O nằm giữa hai điểm A và B.

\textbf{{ANSWER}}

Chỉ ra OA và OB là hai tia đối nhau nên O nằm giữa hai điểm A và B

========================================================================

https://khoahoc.vietjack.com/thi-online/bai-tap-trung-diem-cua-doan-thang


\textbf{{QUESTION}}

Cho điểm O thuộc đường thẳng xy. Trên tia Ox, lấy điểm M sao cho OM = 4 cm. Trên tia Oy, lấy điểm N sao cho ON = 2cm. Gọi A và B lần lượt là trung điểm của OM và ON. Tính độ dài đoạn thẳng AB.

\textbf{{ANSWER}}

Tính được OA = 2 cm, OB = 1 cm. Do đó AB = 3 cm.

========================================================================

https://khoahoc.vietjack.com/thi-online/bai-tap-trung-diem-cua-doan-thang


\textbf{{QUESTION}}

Trên tia Ox, lấy hai điểm A và B sao cho OA = 2cm, OB = 6cm. Gọi M là trung điểm của đoạn thẳng OB.
Tính độ dài AB.

\textbf{{ANSWER}}

AB = 4 cm

========================================================================

https://khoahoc.vietjack.com/thi-online/giai-sgk-toan-12-canh-dieu-bai-2-phuong-sai-do-lech-chuan-cua-mau-so-lieu-ghep-nhom-co-dap-an


\textbf{{QUESTION}}

Kết quả 40 lần nhảy xa của hai vận động viên nam Dũng và Huy được lần lượt thống kê trong Bảng 11 và Bảng 12 (đơn vị: mét):
Nhóm
Tần số
 
Nhóm 
Tần số 
[6,22; 6,46)
[6,46; 6,70)
[6,70; 6,94)
[6,94; 7,18)
[7,18; 7,42)
3
7
5
20
5
 
[6,22; 6,46)
[6,46; 6,70)
[6,70; 6,94)
[6,94; 7,18)
[7,18; 7,42)
2
5
8
19
6
 
n = 40
 
 
n = 40
Bảng 11                                                                 Bảng 12
Kết quả nhảy xa của vận động viên nào đồng đều hơn?

\textbf{{ANSWER}}

Sau bài học này, ta giải quyết bài toán trên như sau: 
Để kiểm tra xem kết quả nhảy xa của vận động viên nào đồng đều hơn, ta cần tính phương sai và độ lệch chuẩn của mẫu số liệu ghép nhóm biểu diễn kết quả 40 lần nhảy xa của từng vận động viên và so sánh. 
Từ Bảng 11 và Bảng 12, ta có các bảng thống kê sau:
Nhóm
Giá trị đại diện
Tần số
 
Nhóm 
Giá trị đại diện
Tần số 
[6,22; 6,46)
[6,46; 6,70)
[6,70; 6,94)
[6,94; 7,18)
[7,18; 7,42)
6,34
6,58
6,82
7,06
7,30
3
7
5
20
5
 
[6,22; 6,46)
[6,46; 6,70)
[6,70; 6,94)
[6,94; 7,18)
[7,18; 7,42)
6,34
6,58
6,82
7,06
7,30
2
5
8
19
6
 
 
n = 40
 
 
 
n = 40
 
Ÿ Số trung bình cộng của mẫu số liệu ghép nhóm biểu diễn kết quả 40 lần nhảy xa của vận động viên Dũng là:
 $$ {\overline{x}}_{D}=\frac{3\cdot 6,34+7\cdot 6,58+5\cdot 6,82+20\cdot 7,06+5\cdot 7,30}{40}=\frac{276,88}{40}\approx 6,92$$ (m).
Vậy phương sai của của mẫu số liệu ghép nhóm biểu diễn kết quả 40 lần nhảy xa của vận động viên Dũng là:
 $$ {s}_{D}^{2}=\frac{1}{40}$$∙ [3 ∙ (6,34 – 6,92)2 + 7 ∙ (6,58 – 6,92)2 + 5 ∙ (6,82 – 6,92)2 
                     + 20 ∙ (7,06 – 6,92)2 + 5 ∙ (7,30 – 6,92)2] =  $$ \frac{2,9824}{40}$$ ≈ 0,07. 
Độ lệch chuẩn của mẫu số liệu ghép nhóm trên là:  $$ {s}_{D}\approx \sqrt{0,07}\approx 0,26$$ (m). 
Ÿ Số trung bình cộng của mẫu số liệu ghép nhóm biểu diễn kết quả 40 lần nhảy xa của vận động viên Huy là:
 $$ {\overline{x}}_{H}=\frac{2\cdot 6,34+5\cdot 6,58+8\cdot 6,82+19\cdot 7,06+6\cdot 7,30}{40}=\frac{278,08}{40}\approx 6,95$$ (m).
Vậy phương sai của của mẫu số liệu ghép nhóm biểu diễn kết quả 40 lần nhảy xa của vận động viên Huy là:
 $$ {s}_{H}^{2}=\frac{1}{40}$$∙ [2 ∙ (6,34 – 6,95)2 + 5 ∙ (6,58 – 6,95)2 + 8 ∙ (6,82 – 6,95)2 
                     + 19 ∙ (7,06 – 6,95)2 + 6 ∙ (7,30 – 6,95)2] =  $$ \frac{2,5288}{40}$$ ≈ 0,06. 
Độ lệch chuẩn của mẫu số liệu ghép nhóm trên là:  $$ {s}_{H}\approx \sqrt{0,06}\approx 0,24$$ (m). 
Ÿ Do sH ≈ 0,24 < sD ≈ 0,26 nên kết quả nhảy xa của vận động viên Huy đồng đều hơn kết quả nhảy xa của vận động viên Dũng.

========================================================================

https://khoahoc.vietjack.com/thi-online/12-bai-tap-tinh-dien-tich-hinh-tron-hinh-quat-tron-va-hinh-vanh-khuyen-co-loi-giai


\textbf{{QUESTION}}

Một hình tròn có diện tích S = 225π (cm2). Bán kính của hình tròn đó là:
A. 15 cm.
B. 16 cm.
C. 12 cm.
D. 14 cm.

\textbf{{ANSWER}}

Đáp án đúng là: A
Diện tích S = πR2 = 225π suy ra R2 = 225, do đó R = 15 (cm).

========================================================================

https://khoahoc.vietjack.com/thi-online/12-bai-tap-tinh-dien-tich-hinh-tron-hinh-quat-tron-va-hinh-vanh-khuyen-co-loi-giai


\textbf{{QUESTION}}

Diện tích hình tròn bán kính R = 10 cm là:
A. 100π cm2.
B. 10π cm2.
C. 20π cm2.
D. 100π2 cm2.

\textbf{{ANSWER}}

Đáp án đúng là: A
Diện tích S = πR2 = 102π = 100π (cm2).

========================================================================

https://khoahoc.vietjack.com/thi-online/on-thi-cap-toc-789-vao-10-mon-toan-khu-vuc-binh-duong-2024-2025-de-15


\textbf{{QUESTION}}

1) Giải phương trình, hệ phương trình sau:
a) ${x^4} - 8{x^2} - 9 = 0$.                           b) $\left\{ {\begin{array}{*{20}{l}}{x + y = 9}\\{3x - 2y =  - 3}\end{array}} \right.$.

\textbf{{ANSWER}}

1) a) ${x^4} - 8{x^2} - 9 = 0$. Đặt $t = {x^2}\,\,\left( {t \ge 0} \right)$. Phương trình đã cho trở thành ${t^2} - 8t - 9 = 0.$
Ta thấy $1 - \left( { - 8} \right) + \left( { - 9} \right) = 0$ nên phương trình có 2 nghiệm $t =  - 1$ (loại) hoặc $t = 9\,\,\left( {{\rm{TM}}} \right).$
Với $t = 9$ thì ${x^2} = 9$. Do đó $x = 3$ hoặc $x =  - 3.$
Vậy phương trình đã cho có nghiệm $x =  - 3\,;\,\,x = 3.$  
b) $\left\{ {\begin{array}{*{20}{l}}{x + y = 9}\\{3x - 2y =  - 3}\end{array}} \right.$. Nhân hai vế của phương trình thứ nhất với 2, ta được hệ: $\left\{ {\begin{array}{*{20}{l}}{2x + 2y = 18}\\{3x - 2y =  - 3}\end{array}} \right..$
Cộng từng vế của phương trình mới, ta được: $5x = 15$, tức là $x = 3.$
Thế $x = 3$ vào phương trình $x + y = 9$ ta có: $3 + y = 9$ hay $y = 6$.
Vậy hệ phương trình có nghiệm duy nhất $\left( {x\,;\,\,y} \right) = \left( {3\,;\,\,6} \right)$.
2) $M = 2\sqrt {9 - 4\sqrt 5 }  - \sqrt {20}  = 2\sqrt {{{\left( {\sqrt 5  - 2} \right)}^2}}  - \sqrt {4 \cdot 5} $$ = 2\left| {\sqrt 5  - 2} \right| - 2\sqrt 5  = 2\sqrt 5  - 4 - 2\sqrt 5  =  - 4$.
Vậy $M = 2\sqrt {9 - 4\sqrt 5 }  =  - 4$.

========================================================================

https://khoahoc.vietjack.com/thi-online/chuong-1-bai-2-ti-so-luong-giac-cua-goc-nhon/58531


\textbf{{QUESTION}}

Không dùng bảng số và máy tính hãy so sánh:
a, sin$$ {20}^{0}$$ và sin$$ {70}^{0}$$
b, cos$$ {60}^{0}$$ và cos$$ {70}^{0}$$
c, tan$$ {73}^{0}20\text{'}$$ và tan$$ {45}^{0}$$
d, cot$$ {20}^{0}$$ và cot$$ {37}^{0}40\text{'}$$

\textbf{{ANSWER}}

a, sin$$ {20}^{0}$$ < sin$$ {70}^{0}$$
b, cos$$ {60}^{0}$$ > cos$$ {70}^{0}$$
c, tan$$ {73}^{0}20\text{'}$$ > tan$$ {45}^{0}$$
d, cot$$ {20}^{0}$$ > cot$$ {37}^{0}40\text{'}$$

========================================================================

https://khoahoc.vietjack.com/thi-online/chuong-1-bai-2-ti-so-luong-giac-cua-goc-nhon/58531


\textbf{{QUESTION}}

Không dùng bảng số và máy tính, hãy so sánh:
a, sin400$$ {40}^{0}$$ và sin700$$ {70}^{0}$$
b, cos800$$ {80}^{0}$$ và cos500$$ {50}^{0}$$
c, sin250$$ {25}^{0}$$ và tan250$$ {25}^{0}$$
d, cos350$$ {35}^{0}$$ và cot350$$ {35}^{0}$$

\textbf{{ANSWER}}

Tương tự câu 1
Chú ý các tỉ số lượng giác sin và cos có giá trị trong khoảng (0;1)

========================================================================

https://khoahoc.vietjack.com/thi-online/chuong-1-bai-2-ti-so-luong-giac-cua-goc-nhon/58531


\textbf{{QUESTION}}

Sắp xếp các tỉ số lượng giác sau theo thứ tự từ lớn đến bé:
a, tan420$$ {42}^{0}$$, cot710$$ {71}^{0}$$, tan380$$ {38}^{0}$$, cot69015'$$ {69}^{0}15\text{'}$$, tan280$$ {28}^{0}$$
b, sin320$$ {32}^{0}$$, cos510$$ {51}^{0}$$, sin390$$ {39}^{0}$$,  cos79013'$$ {79}^{0}13\text{'}$$, sin380$$ {38}^{0}$$

\textbf{{ANSWER}}

a, Ta có: cot710$$ {71}^{0}$$ (= tan190$$ {19}^{0}$$) < cot69015'$$ {69}^{0}15\text{'}$$(= tan20045'$$ {20}^{0}45\text{'}$$) < tan280$$ {28}^{0}$$ < tan380$$ {38}^{0}$$ <tan420$$ {42}^{0}$$
b, Tương tự câu a) ta có : cos79013'$$ {79}^{0}13\text{'}$$ = sin10047'$$ {10}^{0}47\text{'}$$ < sin320$$ {32}^{0}$$ < sin380$$ {38}^{0}$$ < cos510$$ {51}^{0}$$ = sin390$$ {39}^{0}$$

========================================================================

https://khoahoc.vietjack.com/thi-online/chuong-1-bai-2-ti-so-luong-giac-cua-goc-nhon/58531


\textbf{{QUESTION}}

Sắp xếp các tỉ số lượng giác sau theo thứ tự từ bé đến lớn:
a, tan120$$ {12}^{0}$$, cot610$$ {61}^{0}$$, tan280$$ {28}^{0}$$, cot79015'$$ {79}^{0}15\text{'}$$, tan580$$ {58}^{0}$$
b, cos670$$ {67}^{0}$$,  sin560$$ {56}^{0}$$, cos63041'$$ {63}^{0}41\text{'}$$, sin740$$ {74}^{0}$$,  cos850$$ {85}^{0}$$

\textbf{{ANSWER}}

Tương tự câu 3

========================================================================

https://khoahoc.vietjack.com/thi-online/bo-de-minh-hoa-mon-toan-thpt-quoc-gia-nam-2022-30-de/75632


\textbf{{QUESTION}}

Họ tất cả các nguyên hàm của hàm số $f\left( x \right) = \cos 3x$ là
A.$ - \frac{1}{3}\sin 3x + C.$
B.$\frac{1}{3}\sin 3x + C.$
C.$ - 3\sin 3x + C.$
D.$3\sin 3x + C.$

\textbf{{ANSWER}}

Lời giải:
Chọn đáp án B
Ta có $\int {\cos 3xdx} = \frac{{\sin 3x}}{3} + C$

========================================================================

https://khoahoc.vietjack.com/thi-online/bo-de-minh-hoa-mon-toan-thpt-quoc-gia-nam-2022-30-de/75632


\textbf{{QUESTION}}

Trong không gian Oxyz,cho mặt phẳng $\left( P \right):x - 4y + 3z - 2 = 0.$ Vectơ nào dưới đây là một vectơ pháp tuyến của (P)?
A.$\vec n = \left( {0; - 4;3} \right).$
B.$\vec n = \left( {1{\mkern 1mu} ;{\mkern 1mu} 4{\mkern 1mu} ;{\mkern 1mu} 3} \right).$
C.$\vec n = \left( { - 1;4; - 3} \right).$
D.$\vec n = \left( { - 4;3; - 2} \right).$

\textbf{{ANSWER}}

Chọn đáp án C
Mặt phẳng $\left( P \right):x - 4y + 3z - 2 = 0$ có một VTPT là $\overrightarrow n = \left( { - 1;4; - 3} \right)$

========================================================================

https://khoahoc.vietjack.com/thi-online/bo-5-de-cuoi-ki-2-toan-8-chan-troi-sang-tao-cau-truc-moi-co-dap-an/164901


\textbf{{QUESTION}}

A. TRẮC NGHIỆM (7,0 điểm)
Phần 1. (3,0 điểm) Câu trắc nghiệm nhiều phương án lựa chọn
Trong mỗi câu hỏi từ câu 1 đến câu 12, hãy viết chữ cái in hoa đứng trước phương án đúng duy nhất vào bài làm.
Đồ thị hàm số $y = ax{\rm{ }}\left( {a \ne 0} \right)$ là một đường thẳng luôn đi qua

\textbf{{ANSWER}}

Đáp án đúng là: A
Đồ thị hàm số $y = ax{\rm{ }}\left( {a \ne 0} \right)$ là một đường thẳng luôn đi qua gốc tọa độ $O\left( {0;0} \right).$

========================================================================

https://khoahoc.vietjack.com/thi-online/bo-5-de-cuoi-ki-2-toan-8-chan-troi-sang-tao-cau-truc-moi-co-dap-an/164901


\textbf{{QUESTION}}

Hệ số góc của đường thẳng 
y=x−2$y = x - 2$
y=x−2$y = x - 2$
y=x−2
y=x−2
y
=
x
−
2
−2.$ - 2.$
−2.$ - 2.$
−2.
−2.
−
2.
                      
2.$2.$
2.$2.$
2.
2.
2.
                         
−1.$ - 1.$
−1.$ - 1.$
−1.
−1.
−
1.
                      
1.$1.$
1.$1.$
1.
1.
1.

\textbf{{ANSWER}}

Đáp án đúng là: D
Hệ số góc của đường thẳng $y = x - 2$ là $1.$

========================================================================

https://khoahoc.vietjack.com/thi-online/bo-5-de-cuoi-ki-2-toan-8-chan-troi-sang-tao-cau-truc-moi-co-dap-an/164901


\textbf{{QUESTION}}

Phương trình nào sau đây là phương trình bậc nhất một ẩn?
A. 
0x+5=0.$0x + 5 = 0.$
0x+5=0.$0x + 5 = 0.$
0x+5=0.
0x+5=0.
0
x
+
5
=
0.
           
2x2−3=0.$2{x^2} - 3 = 0.$
2x2−3=0.$2{x^2} - 3 = 0.$
2x2−3=0.
2x2−3=0.
2
x2
x2
x
2
−
3
=
0.
    
3x−2=0.$\frac{3}{x} - 2 = 0.$
3x−2=0.$\frac{3}{x} - 2 = 0.$
3x−2=0.
3x−2=0.
3x
3
3
x
x


x
x
x
−
2
=
0.
                                      
2x+1=0.$2x + 1 = 0.$
2x+1=0.$2x + 1 = 0.$
2x+1=0.
2x+1=0.
2
x
+
1
=
0.

\textbf{{ANSWER}}

Đáp án đúng là: D
Phương trình bậc nhất một ẩn có dạng $ax + b = 0{\rm{ }}\left( {a \ne 0} \right)$.
Do đó. $2x + 1 = 0$ là phương trình bậc nhất một ẩn.

========================================================================

https://khoahoc.vietjack.com/thi-online/bo-5-de-cuoi-ki-2-toan-8-chan-troi-sang-tao-cau-truc-moi-co-dap-an/164901


\textbf{{QUESTION}}

Phương trình 
3−2x=0$3 - 2x = 0$
3−2x=0$3 - 2x = 0$
3−2x=0
3−2x=0
3
−
2
x
=
0
x=3.$x = 3.$
x=3.$x = 3.$
x=3.
x=3.
x
=
3.
                   
x=23.$x = \frac{2}{3}.$
x=23.$x = \frac{2}{3}.$
x=23.
x=23.
x
=
23
2
2
3
3


3
3
3
.
   
x=32.$x = \frac{3}{2}.$
x=32.$x = \frac{3}{2}.$
x=32.
x=32.
x
=
32
3
3
2
2


2
2
2
.
  
x=−32.$x = \frac{{ - 3}}{2}.$
x=−32.$x = \frac{{ - 3}}{2}.$
x=−32.
x=−32.
x
=
−32
−3
−3
−
3
2
2


2
2
2
.

\textbf{{ANSWER}}

Đáp án đúng là: C
Ta có: 3−2x=0$3 - 2x = 0$ nên 2x=3$2x = 3$ do đó, x=32$x = \frac{3}{2}$.

========================================================================

https://khoahoc.vietjack.com/thi-online/15-cau-trac-nghiem-toan-7-ket-noi-tri-thuc-bai-1-tap-hop-cac-so-huu-ti-co-dap-an-phan-2/111040


\textbf{{QUESTION}}

Cho các số hữu tỉ $$ \frac{-2}{3};\frac{1}{6};\frac{-6}{5};0;\frac{1}{3};\frac{1}{5}$$. Sắp xếp theo thứ tự tăng dần.
A. $$ \frac{-6}{5};\frac{-2}{3};0;\frac{1}{6};\frac{1}{5};\frac{1}{3}$$
B. $$ \frac{-6}{5};\frac{1}{6};\frac{-2}{3};0;\frac{1}{3};\frac{1}{5}$$
C. $$ \frac{-2}{3};\frac{-6}{5};\frac{1}{6};0;\frac{1}{5};\frac{1}{3}$$
D. $$ \frac{-2}{3};\frac{-6}{5};0;\frac{1}{6};\frac{1}{5};\frac{1}{3}$$

\textbf{{ANSWER}}

Hướng dẫn giải
Đáp án đúng là: A
Ta có:  $$ \frac{-6}{5}<\frac{-5}{5}=-1;-1=\frac{-3}{3}<\frac{-2}{3}$$ ⇒ $$ \frac{-6}{5}<\frac{-2}{3}$$.
Lại có: $$ \frac{-2}{3}<0$$
Do 6 > 5 > 3 nên $$ \frac{1}{6}<\frac{1}{5}<\frac{1}{3}$$.
⇒ $$ \frac{-6}{5}<\frac{-2}{3}<0<\frac{1}{6}<\frac{1}{5}<\frac{1}{3}$$.
Vậy dãy số trên xếp theo thứ tự tăng dần là: $$ \frac{-6}{5};\frac{-2}{3};0;\frac{1}{6};\frac{1}{5};\frac{1}{3}$$.

========================================================================

https://khoahoc.vietjack.com/thi-online/15-cau-trac-nghiem-toan-7-ket-noi-tri-thuc-bai-1-tap-hop-cac-so-huu-ti-co-dap-an-phan-2/111040


\textbf{{QUESTION}}

Khẳng định nào sau đây đúng ?
A. $$ \frac{23}{24}$$<$$ \frac{24}{25}$$;
B. $$ \frac{37}{38}>\frac{391}{389}$$;
C. $$ \frac{9}{11}>\frac{120}{121}$$;

\textbf{{ANSWER}}

Hướng dẫn giải
Đáp án đúng là: A
Ta có: 
$$ \frac{23}{24}=1-\frac{1}{24};\frac{24}{25}=1-\frac{1}{25}$$ mà $$ \frac{1}{24}>\frac{1}{25}$$ nên $$ \frac{23}{24}<\frac{24}{25}$$. Do đó A đúng.
$$ \frac{37}{38}<1;1<\frac{391}{389}$$ nên $$ \frac{37}{38}<\frac{391}{389}$$. Do đó B sai.
$$ \frac{9}{11}=\frac{99}{121}<\frac{120}{121}$$. Do đó C sai.
$$ \frac{12}{13}<1<\frac{16}{15}$$. Do đó D sai.
Vậy khẳng định đúng là $$ \frac{23}{24}<\frac{24}{25}$$.

========================================================================

https://khoahoc.vietjack.com/thi-online/15-cau-trac-nghiem-toan-7-ket-noi-tri-thuc-bai-1-tap-hop-cac-so-huu-ti-co-dap-an-phan-2/111040


\textbf{{QUESTION}}

Có bao nhiêu giá trị nguyên của a thỏa mãn a18$$ \frac{a}{18}$$ là số hữu tỉ lớn hơn −56$$ \frac{-5}{6}$$  và nhỏ hơn −12$$ \frac{-1}{2}$$ ? 
A. 3;
B. 4;
C. 5;

\textbf{{ANSWER}}

Hướng dẫn giải
Đáp án đúng là: C
Ta có: −56=−1518;−12=−918$$ \frac{-5}{6}=\frac{-15}{18};\frac{-1}{2}=\frac{-9}{18}$$
Vì −56<a18<−12$$ \frac{-5}{6}<\frac{a}{18}<\frac{-1}{2}$$ nên -1518<a18<-918$$ \frac{-15}{18}<\frac{a}{18}<\frac{-9}{18}$$
Mà a ∈ ℤ nên a ∈ {– 14; – 13; – 12; – 11; – 10}
Vậy a ∈ { – 14; – 13; – 12; – 11; – 10}. Có 5 giá trị của a thỏa mãn yêu cầu.

========================================================================

https://khoahoc.vietjack.com/thi-online/bo-5-de-thi-giua-ki-2-toan-11-ket-noi-tri-thuc-cau-truc-moi-de-so-2


\textbf{{QUESTION}}

PHẦN I. TRẮC NGHIỆM KHÁCH QUAN
A. TRẮC NGHIỆM NHIỀU PHƯƠNG ÁN LỰA CHỌN. Thí sinh trả lời từ câu 1 đến câu 12.
Mỗi câu hỏi thí sinh chỉ chọn một phương án.
Cho $a > 0,m,n \in \mathbb{R}$. Khẳng định nào sau đây đúng?
A. ${a^m} + {a^n} = {a^{m + n}}$.
B. ${a^m}.{a^n} = {a^{m - n}}$.
C. ${\left( {{a^m}} \right)^n} = {\left( {{a^n}} \right)^m}$.
D. $\frac{{{a^m}}}{{{a^n}}} = {a^{n - m}}$.

\textbf{{ANSWER}}

Đáp án đúng là: C
Ta có ${\left( {{a^m}} \right)^n} = {\left( {{a^n}} \right)^m}$.

========================================================================

https://khoahoc.vietjack.com/thi-online/bo-5-de-thi-giua-ki-2-toan-11-ket-noi-tri-thuc-cau-truc-moi-de-so-2


\textbf{{QUESTION}}

Với các số thực dương a,b$a,b$ bất kì. Mệnh đề nào dưới đây đúng?
A. log(ab)=loga.logb$\log \left( {ab} \right) = \log a.\log b$.
B. log(ab)=loga+logb$\log \left( {ab} \right) = \log a + \log b$.
C. logab=logalogb$\log \frac{a}{b} = \frac{{\log a}}{{\log b}}$.
D. logab=logb−loga$\log \frac{a}{b} = \log b - \log a$.

\textbf{{ANSWER}}

Đáp án đúng là: B
Ta có $\log \left( {ab} \right) = \log a + \log b$.

========================================================================

https://khoahoc.vietjack.com/thi-online/bo-5-de-thi-giua-ki-2-toan-11-ket-noi-tri-thuc-cau-truc-moi-de-so-2


\textbf{{QUESTION}}

Trong các hàm số sau hàm số nào đồng biến trên tập xác định của nó?
A. y=(12)x$y = {\left( {\frac{1}{2}} \right)^x}$.
B. y=0,5x$y = 0,{5^x}$.
C. y=x2$y = {x^2}$.
D. y=log2x$y = {\log _2}x$.

\textbf{{ANSWER}}

Đáp án đúng là: D
Hàm số y=log2x$y = {\log _2}x$ có a=2>1$a = 2 > 1$ nên hàm số y=log2x$y = {\log _2}x$ đồng biến trên tập xác định của nó.

========================================================================

https://khoahoc.vietjack.com/thi-online/giai-sgk-toan-6-kntt-bai-tap-cuoi-chuong-7-trang-42-co-dap-an


\textbf{{QUESTION}}

Tính giá trị của các biểu thức sau:
a) 15,3 - 21,5 – 3. 1,5; 
b) 2(42 – 2. 4,1) + 1,25: 5.

\textbf{{ANSWER}}

a) 15,3 - 21,5 - 3. 1,5
= 15,3 - 21,5 - 4,5 
= 15,3 – (21,5 + 4,5) 
= 15,3 – 26 
= - (26 - 15,3) 
= -10,7
b) 2(42 – 2. 4,1) + 1,25: 5 
= 2. (16 – 8,2) + 0,25 
= 2. 7,8 + 0,25 
= 15, 6 + 0,25 = 15,85

========================================================================

https://khoahoc.vietjack.com/thi-online/giai-sgk-toan-6-kntt-bai-tap-cuoi-chuong-7-trang-42-co-dap-an


\textbf{{QUESTION}}

Tìm x, biết:
a) x - 5,01 = 7,02 – 2. 1,5;
b) x: 2,5 = 1,02 + 3. 1,5.

\textbf{{ANSWER}}

a) x - 5,01 = 7,02 – 2. 1,5
   x – 5,01 = 7,02 – 3
   x – 5,01 = 4, 02
   x = 4,02 + 5, 01
  x = 9,03
Vậy x = 9,03
b) x: 2,5 = 1,02 + 3. 1,5
    x: 2,5 = 1,02 + 4,5
    x: 2,5 = 5,52
    x = 5,52. 2,5
    x = 13,8
Vậy x = 13,8.

========================================================================

https://khoahoc.vietjack.com/thi-online/giai-sgk-toan-6-kntt-bai-tap-cuoi-chuong-7-trang-42-co-dap-an


\textbf{{QUESTION}}

Làm tròn số.
a) 127,459 đến hàng phần mười;
b) 152,025 đến hàng chục; 
c) 15 025 796 đến hàng nghìn.

\textbf{{ANSWER}}

a) Làm tròn 127,459 đến hàng phần mười:
+) Bỏ đi các chữ số sau hàng làm tròn tức là các chữ số 5,9
+) Vì nên chữ số 4 tăng lên 1 đơn vị là 5
Vậy làm tròn 127,459 đến hàng phần mười ta được kết quả 127,5.
b) Làm tròn 152,025 đến hàng chục:
+) Thay chữ số hàng đơn vị bởi chữ số 0 tức là chữ số 2 bởi chữ số 0, bỏ chữ số phần thập phân.
+) Vì 2 < 5 nên chữ số 5 được giữ nguyên
Vậy làm tròn 152,025 đến hàng chục ta được kết quả 150.
c) Làm tròn 15 025 796 đến hàng nghìn:
+) Thay các chữ số sau hàng nghìn bởi chữ số 0 tức là các chữ số 7; 9; 6 thành các chữ số 0
+) Vì 7 > 5 nên tăng 5 lên 1 đơn vị là 6
Vậy làm tròn 15 025 796 đến hàng nghìn ta được kết quả 15 026 000.

========================================================================

https://khoahoc.vietjack.com/thi-online/giai-sgk-toan-6-kntt-bai-tap-cuoi-chuong-7-trang-42-co-dap-an


\textbf{{QUESTION}}

Năm 2002, Thumbelina được Tổ chức Kỉ lục Thế giới Guinness chính thức xác nhận là con ngựa thấp nhất thế giới với chiều cao khoảng 44,5 cm. Còn Big Jake trở nên nổi tiếng vào năm 2010 khi được Tổ chức Kỉ lục Thế giới Guinness trao danh hiệu là con ngựa cao nhất thế giới, nó cao gấp khoảng 4,72 lần con ngựa Thumbelina.
(Theo guinnessworidrecords.com)
Hỏi chiều cao của con Big Jake là bao nhiêu (làm tròn kết quả đến hàng đơn vị)?

\textbf{{ANSWER}}

Chiều cao của con Big Jake là:
44,5.4,72 = 210,04 (cm)
Làm tròn kết quả đến hàng đơn vị ta được 210 
Vậy chiều cao của con Big Jake là 210cm.

========================================================================

https://khoahoc.vietjack.com/thi-online/30-cau-trac-nghiem-toan-10-ket-noi-tri-thuc-bai-on-tap-cuoi-chuong-7-co-dap-an-phan-2/110897


\textbf{{QUESTION}}

Trong hệ trục toạ độ Oxy cho hai điểm A(−2; 2); B(4; –6) và đường thẳng d : $$ \left\{\begin{array}{l}x=t\\ y=1+2t\end{array}\right.$$. Tìm điểm M thuộc d sao cho M cách đều hai điểm A, B
A. M(3; 7);            
B. M(–3; –5);      
C. M(2; 5);

\textbf{{ANSWER}}

Hướng dẫn giải
Đáp án đúng là: B
Do M ∈ d nên M(t; 1 + 2t)
Theo giả thiết M cách đều hai điểm A, B nên MA = MB 
⇔ $$ \sqrt{{(t+2)}^{2}+{(2t-1)}^{2}}$$ = $$ \sqrt{{(t-4)}^{2}+{(2t+7)}^{2}}$$
⇔ $$ {(t+2)}^{2}+{(2t-1)}^{2}$$ = $$ {(t-4)}^{2}+{(2t+7)}^{2}$$
⇔ t2 + 4t + 4 + 4t2 – 4t + 1 = t2 – 8t + 16 + 4t2 + 28t + 49
⇔ 5t +15 = 0 
⇔ t = −3
Với t = −3 thì M(−3; −5).

========================================================================

https://khoahoc.vietjack.com/thi-online/de-kiem-tra-giua-ki-2-toan-11-ctst-co-dap-an


\textbf{{QUESTION}}

D. $n \in {\mathbb{N}^*}$.

\textbf{{ANSWER}}

Đáp án D

========================================================================

https://khoahoc.vietjack.com/thi-online/de-kiem-tra-giua-ki-2-toan-11-ctst-co-dap-an


\textbf{{QUESTION}}

Với a$a$ là số thực dương tùy ý, √a3$\sqrt {{a^3}} $ bằng kết quả nào sau đây?
D. a16${a^{\frac{1}{6}}}$.

\textbf{{ANSWER}}

Đáp án B

========================================================================

https://khoahoc.vietjack.com/thi-online/de-kiem-tra-giua-ki-2-toan-11-ctst-co-dap-an


\textbf{{QUESTION}}

Với α$\alpha $ là số thực bất kì, mệnh đề nào sau đây sai?
D. (10α)2=(10)α2${\left( {{{10}^\alpha }} \right)^2} = {\left( {10} \right)^{{\alpha ^2}}}$.

\textbf{{ANSWER}}

Đáp án D

========================================================================

https://khoahoc.vietjack.com/thi-online/de-kiem-tra-giua-ki-2-toan-11-ctst-co-dap-an


\textbf{{QUESTION}}

Cho đẳng thức 3√a2√aa3=aα,0<a≠1.$\frac{{\sqrt[3]{{{a^2}\sqrt a }}}}{{{a^3}}} = {a^\alpha },0 < a \ne 1.$ Khi đó α$\alpha $ thuộc khoảng nào sau đây?
D. (0;1)$\left( {0;1} \right)$.

\textbf{{ANSWER}}

Đáp án C

========================================================================

https://khoahoc.vietjack.com/thi-online/de-kiem-tra-giua-ki-2-toan-11-ctst-co-dap-an


\textbf{{QUESTION}}

Chị Hà gửi vào ngân hàng 20000000$20\,\,000\,\,000$ đồng với lãi suất 0,5%$0,5\% $/tháng (sau mỗi tháng tiền lãi được nhập vào tiền gốc để tính lãi tháng sau). Hỏi sau 1$1$ năm chị Hà nhận được bao nhiêu tiền, biết trong 1$1$ năm đó chị Hà không rút tiền lần nào và lãi suất không thay đổi (làm tròn đến hàng nghìn).

\textbf{{ANSWER}}

Đáp án C

========================================================================

https://khoahoc.vietjack.com/thi-online/giai-sbt-toan-8-ctst-bai-tap-cuoi-chuong-9-co-dap-an


\textbf{{QUESTION}}

Một hộp chứa 8 tấm thẻ cùng loại được đánh số 6; 7; 8; 9; 10; 11; 12; 13, Thuý lấy ra ngẫu nhiên 1 tấm thẻ từ hộp. Xác suất để thẻ chọn ra ghi số là số nguyên tố là
A. 0,225. 
B. 0,375. 
C. 0,435. 
D. 0,525.

\textbf{{ANSWER}}

Đáp án đúng là B
Các trường hợp thẻ chọn ra ghi số nguyên tố là 7; 11; 13.
Vậy xác suất để thẻ chọn ra ghi số là số nguyên tố là: $\frac{3}{8} = 0,375$.

========================================================================

https://khoahoc.vietjack.com/thi-online/giai-sbt-toan-8-ctst-bai-tap-cuoi-chuong-9-co-dap-an


\textbf{{QUESTION}}

Một hộp chứa các thẻ màu xanh và thẻ màu đỏ có kích thước và khối lượng như nhau. Vinh lấy ra ngẫu nhiên 1 thẻ, xem màu rồi trả lại hộp. Lặp lại thử nghiệm đó 75 lần, Vinh thấy có 24 lần lấy được thẻ màu xanh. Xác suất thực nghiệm của sự kiện lấy được thẻ màu đỏ là
A. 0,24.       
B. 0,28.       
C. 0,32.       
D. 0,68.

\textbf{{ANSWER}}

Đáp án đúng là D
Số lần Vinh lấy được thẻ màu đỏ là: 75 – 24 = 51 (lần).
Xác suất thực nghiệm của sự kiện lấy được thẻ màu đỏ là 5175=0,68$\frac{{51}}{{75}} = 0,68$.

========================================================================

https://khoahoc.vietjack.com/thi-online/giai-sbt-toan-8-ctst-bai-tap-cuoi-chuong-9-co-dap-an


\textbf{{QUESTION}}

Có 46% học sinh ở một trường trung học cơ sở thường xuyên đi đến trường bằng xe buýt. Gặp ngẫu nhiên một học sinh của trường. Xác suất học sinh đó không thường xuyên đi xe buýt đến trường là
A. 0,16. 
B. 0,94. 
C. 0,54. 
D. 0,35.

\textbf{{ANSWER}}

Đáp án đúng là C
Số học sinh không thường xuyên đi xe buýt đến trường là:
100% – 46% = 54% (tổng số học sinh)
Xác suất học sinh được gặp không thường xuyên đi xe buýt đến trường là 54% = 0,54.

========================================================================

https://khoahoc.vietjack.com/thi-online/giai-sbt-toan-8-ctst-bai-tap-cuoi-chuong-9-co-dap-an


\textbf{{QUESTION}}

Gieo 2 con xúc xắc cân đối và đồng chất. Xác suất của biến cố tích số chấm xuất hiện trên hai con xúc xắc bằng 21 là
A. 0. 
B.136$\frac{1}{{36}}$. 
C.118$\frac{1}{{18}}$. 
D.112$\frac{1}{{12}}$.

\textbf{{ANSWER}}

Đáp án đúng là A
Biến cố tích số chấm xuất hiện trên hai con xúc xắc bằng 21 xảy ra khi số chấm xuất hiện trên xúc xắc là 3 và 7 hoặc 1 và 21.
Các mặt xúc xắc chỉ có tối đa 6 chấm nên không thể xảy ra trường hợp nào trong 2 trường hợp trên. 
Do đó, xác suất của biến cố tích số chấm xuất hiện trên hai con xúc xắc bằng 21 là 0.

========================================================================

https://khoahoc.vietjack.com/thi-online/giai-sbt-toan-8-ctst-bai-tap-cuoi-chuong-9-co-dap-an


\textbf{{QUESTION}}

Cường gieo một con xúc xắc cân đối 540 lần. Số lần xuất hiện mặt 6 chấm trong 540 lần gieo đó có khả năng lớn nhất thuộc vào tập hợp nào dưới đây?
A. {80; 81; ...; 100}.            
B. {101; 102; ...; 120}. 
C. {121; 122; ...; 161}.         
D. {20; 21; ...; 40}.

\textbf{{ANSWER}}

Đáp án đúng là A
Xác suất của biến cố xuất hiện mặt 6 chấm là 16$\frac{1}{6}$.
Do số lần gieo lớn nên xác suất thử nghiệm và xác suất lý thuyết của phép thử xấp xỉ bằng nhau và bằng 16$\frac{1}{6}$.
Số lần xuất hiện mặt 6 chấm xấp xỉ bằng 16⋅540=90$\frac{1}{6} \cdot 540 = 90$ (lần).
Vậy số lần xuất hiện mặt 6 chấm trong 540 lần gieo có khả năng lớn nhất thuộc vào tập hợp {80; 81; ...; 100}.

========================================================================

https://khoahoc.vietjack.com/thi-online/12-bai-sng-cua-dang-so-huu-ti-va-so-sanh-so-huu-ti-vao-bai-toan-thuc-te-co-dap-an-co-loi-gi


\textbf{{QUESTION}}

Bốn bạn An, Bình, Cường, Dương có chiều cao lần lượt là 1,5 m; 1,47 m; 1,53 m; 1,55 m. Sắp xếp 4 bạn theo thứ tự từ cao đến thấp là:
A. An, Bình, Cường, Dương;

\textbf{{ANSWER}}

Đáp án đúng là: B
Ta có 1,55 > 1,53 > 1,5 > 1,47.
Vậy thứ tự chiều cao của bốn bạn từ cao đến thấp là: Dương, Cường, An, Bình.

========================================================================

https://khoahoc.vietjack.com/thi-online/12-bai-sng-cua-dang-so-huu-ti-va-so-sanh-so-huu-ti-vao-bai-toan-thuc-te-co-dap-an-co-loi-gi


\textbf{{QUESTION}}

Gia đình Lan có kế hoạch chi tiêu cho tháng 5 – 2022 là 7,8 triệu đồng nhưng cuối tháng mẹ Lan tổng kết lại thì thấy gia đình đã tiêu hết 8,2 triệu. Em hãy cho biết chi tiêu của gia đình Lan trong tháng đó đã hợp lí với kế hoạch chi tiêu chưa?

\textbf{{ANSWER}}

Số tiền đã tiêu là 8,2 (triệu đồng)
Số tiền kế hoạch chi tiêu là 7,8 (triệu đồng)
Ta có 8,2 > 7,8
Do đó, gia đình Lan đã chi tiêu vượt kế hoạch nên chi tiêu trong tháng 5 đó là chưa hợp lí.

========================================================================

https://khoahoc.vietjack.com/thi-online/12-bai-sng-cua-dang-so-huu-ti-va-so-sanh-so-huu-ti-vao-bai-toan-thuc-te-co-dap-an-co-loi-gi


\textbf{{QUESTION}}

Cô Lan dự định xây tầng hầm cho ngôi nhà của gia đình. Một công ty tư vấn xây dựng đã cung cấp cho cô Lan lựa chọn một trong bốn số đo chiều cao của tầng hầm như sau: 2,5 m, 2,65 m, 2,75 m, 2,7 m. Cô Lan dự định chọn chiều cao của tầng hầm lớn hơn 2710$$ \frac{27}{10}$$ m để đảm bảo ánh sáng, thoáng đãng, cân đối về kiến trúc và thuận tiện trong sử dụng. Em hãy giúp cô Lan chọn đúng số đo chiều cao của tầng hầm.
A. 2,7 m;

\textbf{{ANSWER}}

Đáp án đúng là: D
Ta có 2710=2,7$$ \frac{27}{10}=2,7$$ 
Cô Lan dự định chọn chiều cao của tầng hầm lớn hơn 2710$$ \frac{27}{10}$$ m hay chiều cao lớn hơn 2,7 m.
Mà trong bốn chiều cao mà công ty tư vấn xây dựng đã đề xuất cho cô Lan thì chỉ có chiều cao 2,75 m lớn hơn 2,7 m.
Vậy số đo chiều cao của tầng hầm cô Lan cần chọn là 2,75 m.

========================================================================

https://khoahoc.vietjack.com/thi-online/12-bai-sng-cua-dang-so-huu-ti-va-so-sanh-so-huu-ti-vao-bai-toan-thuc-te-co-dap-an-co-loi-gi


\textbf{{QUESTION}}

Nhân dịp chào mừng ngày Quốc khánh 2 ‒ 9, huyện K dự định tổ chức các cuộc thi cho người dân trên địa bàn huyện, trong đó có cuộc thi trèo thuyền, có tất cả 5 xã trong địa bàn huyện K, mỗi xã cử ra 11 người dân có sức mạnh nhất lập thành một đội. Để sẵn sàng trang bị cho phần thi trèo thuyền, mỗi đại diện của các xã đóng góp ý kiến cho việc đóng thuyền, về chiều dài của chiếc thuyền có 5 ý kiến khác nhau như sau: 5,5 m; 5,2 m; 4,5 m; 6 m và 5,8 m. Sau khi tham khảo 5 ý kiến của từng xã, ban tổ chức cuộc thi nhận định, để đảm bảo chỗ ngồi cho 11 người chơi trên một chiếc thuyền và gọn nhẹ nhất có thể, thì chiều dài chiếc thuyền phải lớn hơn $$ \frac{53}{10}$$ và nhỏ hơn $$ \frac{57}{10}$$. Em hãy giúp ban tổ chức lựa chọn ra ý kiến về chiều dài chiếc thuyền phù hợp với điều kiện của ban tổ chức đưa ra.
A. 5,3 m;

\textbf{{ANSWER}}

Đáp án đúng là: C
Ta có $$ \frac{53}{10}=\mathrm{5,3}$$ và $$ \frac{57}{10}=\mathrm{5,7}$$.
Vì 4,5 < 5,2 < 5,3 < 5,5 < 5,7 < 5,8 < 6
Như vậy, ý kiến về chiều dài chiếc thuyền phù hợp với điều kiện của ban tổ chức đưa ra là 5,5 m.

========================================================================

https://khoahoc.vietjack.com/thi-online/12-bai-sng-cua-dang-so-huu-ti-va-so-sanh-so-huu-ti-vao-bai-toan-thuc-te-co-dap-an-co-loi-gi


\textbf{{QUESTION}}

Một nhà máy trong tuần thứ nhất đã thực hiện được 415$$ \frac{4}{15}$$ kế hoạch tháng, trong tuần thứ hai thực hiện được 730$$ \frac{7}{30}$$ kế hoạch, trong tuần thứ ba thực hiện được 310$$ \frac{3}{10}$$ kế hoạch, tuần cuối thực hiện 15$$ \frac{1}{5}$$ kế hoạch. Hỏi trong 4 tuần thực hiện, tuần nào thực hiện công việc ít nhất.
A. Tuần 1;

\textbf{{ANSWER}}

Đáp án đúng là: D
Ta có 415=830;310=930;15=630$$ \frac{4}{15}=\frac{8}{30};\frac{3}{10}=\frac{9}{30};\frac{1}{5}=\frac{6}{30}$$ 
Vì 630<730<830<930$$ \frac{6}{30}<\frac{7}{30}<\frac{8}{30}<\frac{9}{30}$$ nên 15<730<415<310$$ \frac{1}{5}<\frac{7}{30}<\frac{4}{15}<\frac{3}{10}$$.
Vậy tuần cuối cùng (tuần 4) thực hiện công việc ít nhất.

========================================================================

https://khoahoc.vietjack.com/thi-online/10-bai-tap-lap-so-csshia-het-cho-2-cho-5-cho-9-cho-3-tu-cac-chu-so-cho-truoc-co-loi-giai


\textbf{{QUESTION}}

Từ 3 chữ số: 3; 6; 4 có thể lập được bao nhiêu số có 3 chữ số khác nhau chia hết cho 2?
A. 3;
B. 4;
C. 2;
D. 1.

\textbf{{ANSWER}}

Đáp án đúng là: B
Số chia hết cho 2 thì chữ số tận cùng chỉ có thể là 6 hoặc 4.
+ Trường hợp 1: Chữ số tận cùng là 6. Các số có 3 chữ số khác nhau chia hết cho 2 lập được là: 346; 436.
+ Trường hợp 2: Chữ số tận cùng là 4. Các số có 3 chữ số khác nhau chia hết cho 2 lập được là: 364; 634.
Vậy có tất cả 4 số có thể lập được.

========================================================================

https://khoahoc.vietjack.com/thi-online/10-bai-tap-lap-so-csshia-het-cho-2-cho-5-cho-9-cho-3-tu-cac-chu-so-cho-truoc-co-loi-giai


\textbf{{QUESTION}}

Từ các chữ số: 5; 0; 4; 2, có tất cả bao nhiêu số tự nhiên có 3 chữ số khác nhau chia hết cho 3?
A. 4;
B. 5;
C. 6;
D. 8.

\textbf{{ANSWER}}

Đáp án đúng là: D
Trong các chữ số: 5; 0; 4; 2 có các bộ 3 chữ số khác nhau có tổng chia hết cho 3 là:
 (0; 4; 2); (5; 0; 4).
+ Từ (0; 4; 2) ta lập được các số có 3 chữ số khác nhau là: 204; 240; 402; 420.
+ Từ (5; 0; 4) ta lập được các số có 3 chữ số khác nhau là: 405; 450; 504; 540.
Vậy tất cả các 8 số được lập thỏa mãn điều kiện.

========================================================================

https://khoahoc.vietjack.com/thi-online/10-bai-tap-lap-so-csshia-het-cho-2-cho-5-cho-9-cho-3-tu-cac-chu-so-cho-truoc-co-loi-giai


\textbf{{QUESTION}}

Cho các chữ số: 0; 5; 4; 6; 1. Trong tất cả các số có 4 chữ số khác nhau chia hết cho 5 được lập từ các chữ số trên thì số nào lớn nhất?
A. 6045;
B. 6504;
C. 6540;
D. 6150.

\textbf{{ANSWER}}

Đáp án đúng là: C
Số chia hết cho 5 thì chữ số tận cùng chỉ có thể là 0 hoặc 5.
Cách 1: Liệt kê: 
+ Trường hợp 1: Chữ số tận cùng là 0. Các số có 4 chữ số khác nhau chia hết cho 5 lập được là: 5460; 5640; 5410; 5140; 5160; 5610; 4560; 4650; 4510; 4150; 4160; 4610; 6510; 6150; 6410; 6410; 6450; 6540; 1460; 1640; 1540; 1450; 1650; 1560.
+ Trường hợp 2: Chữ số tận cùng là 5. Các số có 4 chữ số khác nhau chia hết cho 5 lập được là: 4105; 4015; 4605; 4065; 4165; 4651; 6015; 6105; 6405; 6045; 6145; 6415; 1045; 1405; 1605; 1065; 1465; 1645.
Cách 2: Gọi số cần tìm là ¯abcd$\overline {abcd} $
Vì số đó chia hết cho 5 nên số đó phải có tận cùng là 0 hoặc 5.
Để số đó lớn nhất thì a, b, c, d lớn nhất. Do đó, a = 6, b = 5, c = 4 và d = 0.
Vậy trong tất cả các số lập được số lớn nhất là: 6540.

========================================================================

https://khoahoc.vietjack.com/thi-online/10-bai-tap-lap-so-csshia-het-cho-2-cho-5-cho-9-cho-3-tu-cac-chu-so-cho-truoc-co-loi-giai


\textbf{{QUESTION}}

Cho 5 chữ số: 0; 2; 3; 6; 7. Lập được tất cả bao nhiêu số có 3 chữ số khác nhau chia hết cho 9 từ các chữ số trên?
A. 2;
B. 8;
C.7;
D. 6.

\textbf{{ANSWER}}

Đáp án đúng là: B
Trong 5 chữ số: 0; 2; 3; 6; 7 có các bộ 3 chữ số khác nhau có tổng chia hết cho 9 là:
 (0; 2; 7); (0; 3; 6).
+ Từ (0; 2; 7) ta lập được các số có 3 chữ số khác nhau là: 207; 270; 720; 702.
+ Từ (0; 3; 6) ta lập được các số có 3 chữ số khác nhau là: 360; 306; 603; 630.
Vậy tất cả các 8 số được lập thảo mãn điều kiện.

========================================================================

https://khoahoc.vietjack.com/thi-online/10-bai-tap-lap-so-csshia-het-cho-2-cho-5-cho-9-cho-3-tu-cac-chu-so-cho-truoc-co-loi-giai


\textbf{{QUESTION}}

Cho các chữ số sau: 3; 2; 0. Hãy tính tổng tất cả các số có 2 chữ số chia hết cho 5 được lập từ các chữ số trên? 
A. 50;
B. 78;
C. 67;
D. 137.

\textbf{{ANSWER}}

Đáp án đúng là: A
Số chia hết cho 5 thì chữ số tận cùng chỉ có thể là 0.
Các số có 2 chữ số chia hết cho 5 được lập từ các chữ số: 3; 2; 0 là: 30; 20. 
Vậy tổng của chúng là: 30 + 20 = 50.
Vậy có tất cả 4 số có thể lập được.

========================================================================

https://khoahoc.vietjack.com/thi-online/v10-bai-tap-hoan-vi-co-loi-giai


\textbf{{QUESTION}}

Giá trị của 10! là
A. 5 135 200;
B. 4 546 500;
C.  3 628 000;

\textbf{{ANSWER}}

Đáp án đúng là: C
Ta có 10! = 10 . 9 . 8 . 7 . 6 . 5 . 4 . 3 . 2 . 1 = 3 628 800.

========================================================================

https://khoahoc.vietjack.com/thi-online/12-cau-trac-nghiem-toan-8-bai-2-phuong-trinh-bac-nhat-mot-an-va-cach-giai-co-dap-an-thong-hieu


\textbf{{QUESTION}}

Phương trình 5 – x2 = -x2 + 2x – 1 có nghiệm là:
A. x = 3
B. x = -3
C. x = ±3
D. x = 1

\textbf{{ANSWER}}

5 – x2 = -x2 + 2x – 1
 5 – x2 + x2 - 2x + 1 = 0
 -2x + 6 = 0
 -2x = -6
 x = 3
Vậy phương trình có nghiệm x = 3
Đáp án cần chọn là: A

========================================================================

https://khoahoc.vietjack.com/thi-online/12-cau-trac-nghiem-toan-8-bai-2-phuong-trinh-bac-nhat-mot-an-va-cach-giai-co-dap-an-thong-hieu


\textbf{{QUESTION}}

Số nghiệm của phương trình (x – 1)2 = x2 + 4x – 3 là:
A. 0
B. 1
C. 2
D. 3

\textbf{{ANSWER}}

(x – 1)2 = x2 + 4x – 3
 x2 – 2x + 1 = x2 + 4x – 3
 x2 – 2x + 1 – x2 – 4x + 3 = 0
 -6x + 4 = 0
 x =  23$$ \frac{2}{3}$$
Vậy phương trình có nghiệm duy nhất x = 23$$ \frac{2}{3}$$
Đáp án cần chọn là: B

========================================================================

https://khoahoc.vietjack.com/thi-online/12-cau-trac-nghiem-toan-8-bai-2-phuong-trinh-bac-nhat-mot-an-va-cach-giai-co-dap-an-thong-hieu


\textbf{{QUESTION}}

Cho biết 2x – 2 = 0. Tính giá trị của 5x2 – 2.
A. -1
B. 1
C. 3
D. 6

\textbf{{ANSWER}}

Ta có
2x – 2 = 0
 2x = 2  x = 1
Thay x = 1 vào 5x2 – 2 ta được: 5.12 – 2 = 5 – 2 = 3
Đáp án cần chọn là: C

========================================================================

https://khoahoc.vietjack.com/thi-online/10-cau-trac-nghiem-toan-7-bai-1-tong-ba-goc-cua-mot-tam-giac-co-dap-an-thong-hieu


\textbf{{QUESTION}}

Cho $$ \Delta ABC$$ vuông tại A. Khi đó:
A. $$ \widehat{B}+\widehat{C}={90}^{o}$$
B. $$ \widehat{B}+\widehat{C}={180}^{o}$$
C. $$ \widehat{B}+\widehat{C}={100}^{o}$$
D. $$ \widehat{B}+\widehat{C}={60}^{o}$$

\textbf{{ANSWER}}

Vì tam giác ABC vuông tại A nên $$ \widehat{B}+\widehat{C}={90}^{o}$$
Đáp án cần chọn là: A

========================================================================

https://khoahoc.vietjack.com/thi-online/de-thi-thpt-quoc-gia-mon-toan-nam-2022-chon-loc-co-loi-giai-30-de/67585


\textbf{{QUESTION}}

Biết $$ A\left(1,1,0\right);\text{\hspace{0.17em}}B\left(2,0,3\right);\text{\hspace{0.17em}}C\left(3,2,-3\right)$$, tọa độ trọng tâm G của $$ \Delta ABC$$ là
$$ A.\quad G\left(2,1,-1\right)$$
$$ B.\quad G\left(2,1,0\right)$$
$$ C.\quad G\left(2,0,-1\right)$$
$$ D.\quad G\left(-2,1,0\right)$$

\textbf{{ANSWER}}

Đáp án B
Ta có $$ G\left(\frac{{x}_{A}+{x}_{B}+{x}_{C}}{3};\frac{{y}_{A}+{y}_{B}+{y}_{C}}{3};\frac{{z}_{A}+{z}_{B}+{z}_{C}}{3}\right)\Rightarrow G\left(2,1,0\right)$$

========================================================================

https://khoahoc.vietjack.com/thi-online/de-thi-thpt-quoc-gia-mon-toan-nam-2022-chon-loc-co-loi-giai-30-de/67585


\textbf{{QUESTION}}

Cho hàm số có f'(x)=x3−4x2+1$$ f\text{'}\left(x\right)={x}^{3}-4{x}^{2}+1$$. Xác định hệ số góc của tiếp tuyến tại điểm A(1;2)
A. -2
B. -1
C. 1
D. 2

\textbf{{ANSWER}}

Đáp án A
Ta có hệ số góc của tiếp tuyến tại điểm A(1;2) là f'(1)=1−4+1=−2$$ f\text{'}\left(1\right)=1-4+1=-2$$

========================================================================

https://khoahoc.vietjack.com/thi-online/de-thi-thpt-quoc-gia-mon-toan-nam-2022-chon-loc-co-loi-giai-30-de/67585


\textbf{{QUESTION}}

Cho hàm số f(x)=x2+1x2−6x+5$$ f\left(x\right)=\frac{{x}^{2}+1}{{x}^{2}-6x+5}$$. Hàm số f(x) liên tục trên khoảng nào đây?
A. (−∞;3)$$ A.\quad \left(-\infty ;3\right)$$
B. (2;3)
C. (2;+∞)$$ C.\quad \left(2;+\infty \right)$$
D. ℝ$$ D.\quad \mathbb{R} $$

\textbf{{ANSWER}}

Đáp án B
Hàm số có dạng phân thức hữu tỉ xác định ⇔x2−6x+5≠0⇔{x≠1x≠5$$ \Leftrightarrow {x}^{2}-6x+5\ne 0\Leftrightarrow \left\{\begin{array}{l}x\ne 1\\ x\ne 5\end{array}\right.$$

========================================================================

https://khoahoc.vietjack.com/thi-online/giai-sbt-toan-11-canh-dieu-bai-tap-cuoi-chuong-4-co-dap-an


\textbf{{QUESTION}}

Cho bốn điểm A, B, C, D không cùng thuộc một mặt phẳng. Khẳng định nào sau đây là sai? 
A. Bốn điểm A, B, C, D đã cho đôi một khác nhau. 
B. Không có ba điểm nào trong bốn điểm A, B, C, D là thẳng hàng. 
C. Hai đường thẳng AC và BD song song với nhau. 
D. Hai đường thẳng AC và BD không có điểm chung.

\textbf{{ANSWER}}

Đáp án đúng là: C
Vì bốn điểm A, B, C, D không cùng thuộc một mặt phẳng nên hai đường thẳng AC và BD chéo nhau, do đó đáp án C sai.

========================================================================

https://khoahoc.vietjack.com/thi-online/7881-cau-trac-nghiem-tong-hop-mon-toan-2023-cuc-hay-co-dap-an/125819


\textbf{{QUESTION}}

Xác định số hữu tỉ a sao cho x3 + ax2 + 5x + 3 chia hết cho x2 + 2x + 3.

\textbf{{ANSWER}}

Ta có:$$ \begin{array}{cccccccc}& {x}^{3}& +& {}^{}{\text{\hspace{0.17em}}}^{}{\text{\hspace{0.17em}}}^{}{\text{\hspace{0.17em}}}^{}\text{\hspace{0.17em}}a{x}^{2}& +& {}^{}{\text{\hspace{0.17em}}}^{}{\text{\hspace{0.17em}}}^{}{\text{\hspace{0.17em}}}^{}{\text{\hspace{0.17em}}}^{}\text{\hspace{0.17em}}5x& +& {}^{}{\text{\hspace{0.17em}}}^{}{\text{\hspace{0.17em}}}^{}\text{\hspace{0.17em}\hspace{0.17em}}3\\ \overline{}& {x}^{3}& +& {}^{}{\text{\hspace{0.17em}}}^{}{\text{\hspace{0.17em}}}^{}{\text{\hspace{0.17em}}}^{}\text{\hspace{0.17em}}2{x}^{2}& +& {}^{}{\text{\hspace{0.17em}}}^{}{\text{\hspace{0.17em}}}^{}{\text{\hspace{0.17em}}}^{}{\text{\hspace{0.17em}}}^{}\text{\hspace{0.17em}}3x& & \\ & & & \left(a-2\right){x}^{2}& +& {}^{}{\text{\hspace{0.17em}}}^{}{\text{\hspace{0.17em}}}^{}{\text{\hspace{0.17em}}}^{}{\text{\hspace{0.17em}}}^{}\text{\hspace{0.17em}}2x& +& {}^{}{\text{\hspace{0.17em}}}^{}{\text{\hspace{0.17em}}}^{}\text{\hspace{0.17em}\hspace{0.17em}}3\\ & & \overline{}& \left(a-2\right){x}^{2}& +& \left(2a-4\right)x& +& 3a-6\\ & & & & & \left(6-2\text{a}\right)x& +& 9-3a\end{array}\begin{array}{c}{x}^{2}+2x+3\\ x+\left(a-2\right)\\ \\ \\ \end{array}$$
 
Để x3 + ax2 + 5x + 3 ⋮ x2 + 2x + 3
 
 
Vậy a = 3 thì x3 + ax2 + 5x + 3 chia hết cho x2 + 2x + 3.

========================================================================

https://khoahoc.vietjack.com/thi-online/11-cau-trac-nghiem-tich-cua-vecto-voi-mot-so-co-dap-an-nhan-biet


\textbf{{QUESTION}}

Chọn phát biểu sai?
A. Ba điểm phân biệt A, B, C thẳng hàng khi và chỉ khi $$ \overrightarrow{AB}=k\overrightarrow{BC},k\ne 0$$
B. Ba điểm phân biệt A, B, C thẳng hàng khi và chỉ khi $$ \overrightarrow{AC}=k\overrightarrow{BC},k\ne 0$$
C. Ba điểm phân biệt A, B, C thẳng hàng khi và chỉ khi $$ \overrightarrow{AB}=k\overrightarrow{AC},k\ne 0$$
D. Ba điểm phân biệt A, B, C thẳng hàng khi và chỉ khi $$ \overrightarrow{AB}=k\overrightarrow{AC}$$

\textbf{{ANSWER}}

Ta có ba điểm phân biệt A, B, C thẳng hàng khi và chỉ khi các vec tơ $$ \overrightarrow{AB},\overrightarrow{AC},\overrightarrow{BC}$$ cùng phương, hay $$ \exists k\in R,k\ne 0$$ sao cho $$ \overrightarrow{AB}=k\overrightarrow{AC}$$ hoặc $$ \overrightarrow{AC}=k\overrightarrow{BC}$$
Chú ý rằng hệ số k phải khác 0 nên chỉ có đáp án D sai
Đáp án cần chọn là: D

========================================================================

https://khoahoc.vietjack.com/thi-online/de-thi-hoc-ki-1-toan-9-co-dap-an-nam-2022-2023/116648


\textbf{{QUESTION}}

Giải các phương trình và hệ phương trình:
a) $$ 6x+12{x}^{2}=0$$

\textbf{{ANSWER}}

$$ a)\quad 6x+12{x}^{2}=0\phantom{\rule{0ex}{0ex}}\Leftrightarrow 6x(1\text{\hspace{0.17em}\hspace{0.17em}}+\text{\hspace{0.17em}\hspace{0.17em}}2x)=0\phantom{\rule{0ex}{0ex}}x=0\quad hay\quad x=\frac{-1}{2}$$

========================================================================

https://khoahoc.vietjack.com/thi-online/de-thi-hoc-ki-1-toan-9-co-dap-an-nam-2022-2023/116648


\textbf{{QUESTION}}

b) $$ {x}^{2}+3x-32=8(x-1)$$

\textbf{{ANSWER}}

$$ b)\quad {x}^{2}+3x-32=8(x-1)\phantom{\rule{0ex}{0ex}}\Leftrightarrow {x}^{2}-5x-24=0$$
$$ \triangle $$= 25 + 4.24 = 121
$$ {x}_{1}\text{\hspace{0.17em}\hspace{0.17em}}=\text{\hspace{0.17em}\hspace{0.17em}}8\phantom{\rule{0ex}{0ex}}{x}_{2}\text{\hspace{0.17em}\hspace{0.17em}}=\text{\hspace{0.17em}\hspace{0.17em}}-\text{\hspace{0.17em}\hspace{0.17em}}3$$

========================================================================

https://khoahoc.vietjack.com/thi-online/de-thi-hoc-ki-1-toan-9-co-dap-an-nam-2022-2023/116648


\textbf{{QUESTION}}

c) x4−2x2− 8 = 0

\textbf{{ANSWER}}

c) x4−2x2− 8 = 0$$ c)\quad {x}^{4}-2{x}^{2}-\text{\hspace{0.17em}}8\text{\hspace{0.17em}}=\text{\hspace{0.17em}}0$$

========================================================================

https://khoahoc.vietjack.com/thi-online/de-thi-hoc-ki-1-toan-9-co-dap-an-nam-2022-2023/116648


\textbf{{QUESTION}}

d) {2x  +3 y=−113x  −5y=31$$ \left\{\begin{array}{l}2x\text{\hspace{0.17em}\hspace{0.17em}}+3\text{\hspace{0.17em}}y=-11\\ 3x\text{\hspace{0.17em}\hspace{0.17em}}-5y=31\end{array}\right.$$

\textbf{{ANSWER}}

d) {2x  +3 y=−113x−5y=31⇔{6x  +  9y  =  −  336x− 10y  =  62⇔{2x  +  3y  =  −  1119y=−95⇔{2x  +  3y  =  −  11y=−5⇔{x  =  2y  =− 5$$ d)\quad \left\{\begin{array}{l}2x\text{\hspace{0.17em}\hspace{0.17em}}+3\text{\hspace{0.17em}}y=-11\\ 3x-5y=31\end{array}\right.\Leftrightarrow \left\{\begin{array}{l}6x\text{\hspace{0.17em}\hspace{0.17em}}+\text{\hspace{0.17em}\hspace{0.17em}}9y\text{\hspace{0.17em}\hspace{0.17em}}=\text{\hspace{0.17em}\hspace{0.17em}}-\text{\hspace{0.17em}\hspace{0.17em}}33\\ 6x-\text{\hspace{0.17em}}10y\text{\hspace{0.17em}\hspace{0.17em}}=\text{\hspace{0.17em}\hspace{0.17em}}62\end{array}\Leftrightarrow \left\{\begin{array}{l}2x\text{\hspace{0.17em}\hspace{0.17em}}+\text{\hspace{0.17em}\hspace{0.17em}}3y\text{\hspace{0.17em}\hspace{0.17em}}=\text{\hspace{0.17em}\hspace{0.17em}}-\text{\hspace{0.17em}\hspace{0.17em}}11\\ 19y=-95\end{array}\right.\right.\phantom{\rule{0ex}{0ex}}\Leftrightarrow \left\{\begin{array}{l}2x\text{\hspace{0.17em}\hspace{0.17em}}+\text{\hspace{0.17em}\hspace{0.17em}}3y\text{\hspace{0.17em}\hspace{0.17em}}=\text{\hspace{0.17em}\hspace{0.17em}}-\text{\hspace{0.17em}\hspace{0.17em}}11\\ y=-5\end{array}\right.\Leftrightarrow \left\{\begin{array}{l}x\text{\hspace{0.17em}\hspace{0.17em}}=\text{\hspace{0.17em}\hspace{0.17em}}2\\ y\text{\hspace{0.17em}\hspace{0.17em}}=-\text{\hspace{0.17em}}5\end{array}\right.$$

========================================================================

https://khoahoc.vietjack.com/thi-online/de-thi-hoc-ki-1-toan-9-co-dap-an-nam-2022-2023/116648


\textbf{{QUESTION}}

Cho hàm số: y  =  −x2$$ y\text{\hspace{0.17em}\hspace{0.17em}}=\text{\hspace{0.17em}\hspace{0.17em}}-{x}^{2}$$ có đồ thị là (P) và đường thẳng (D): y=12x−3$$ y=\frac{1}{2}x-3$$. Tìm tọa độ giao điểm của (P) và đường thẳng (D) bằng phép toán.

\textbf{{ANSWER}}

Tìm tọa độ giao điểm của (P) và đường thẳng (D) bằng phép toán.
Phương trình hoành độ giao điểm: 
−x2  =  12x  −  3  ⇔  2x2+  x−  6=  0  ⇔  x  = −2  hay  x  =  32$$ -{x}^{2}\text{\hspace{0.17em}\hspace{0.17em}}=\text{\hspace{0.17em}\hspace{0.17em}}\frac{1}{2}x\text{\hspace{0.17em}\hspace{0.17em}}-\text{\hspace{0.17em}\hspace{0.17em}}3\text{\hspace{0.17em}}\phantom{\rule{0ex}{0ex}}\text{\hspace{0.17em}}\Leftrightarrow \text{\hspace{0.17em}\hspace{0.17em}}2{x}^{2}+\text{\hspace{0.17em}\hspace{0.17em}}x-\text{\hspace{0.17em}\hspace{0.17em}}6=\text{\hspace{0.17em}\hspace{0.17em}}0\text{\hspace{0.17em}}\phantom{\rule{0ex}{0ex}}\text{\hspace{0.17em}}\Leftrightarrow \text{\hspace{0.17em}\hspace{0.17em}}x\text{\hspace{0.17em}\hspace{0.17em}}=\text{\hspace{0.17em}}-2\text{\hspace{0.17em}\hspace{0.17em}}hay\text{\hspace{0.17em}\hspace{0.17em}}x\text{\hspace{0.17em}\hspace{0.17em}}=\text{\hspace{0.17em}\hspace{0.17em}}\frac{3}{2}$$

========================================================================

https://khoahoc.vietjack.com/thi-online/sach-bai-tap-toan-7-tap-2/23614


\textbf{{QUESTION}}

Tính giá trị các biểu thức sau tại x = 1; y = -1; z = 3. (x2y – 2x – 2z)xy

\textbf{{ANSWER}}

Thay x = 1; y = -1; z = 3 vào biểu thức, ta có:
(12(-1) – 2.1 – 2.3).1(-1) = (-1 – 2 – 6).(-1) = (-9).(-1) = 9
Vậy giá trị của biểu thức (x2y – 2x – 2z)xy bằng 9 tại x = 1; y = -1; z = 3

========================================================================

https://khoahoc.vietjack.com/thi-online/de-kiem-tra-15-phut-toan-9-chuong-2-dai-so-co-dap-an/43275


\textbf{{QUESTION}}

Cho hàm số bậc nhất y=(m+1)x+5
a) Tìm giá trị của m để hàm số trên là đồng biến, nghịch biến

\textbf{{ANSWER}}

a) Tìm giá trị của m để hàm số trên là đồng biến, nghịch biến
Hàm số trên là đồng biến khi và chỉ khi :
m + 1 > 0 ⇔ m > -1
Hàm số trên là nghịch biến khi và chỉ khi :
m + 1 < 0 ⇔ m < -1

========================================================================

https://khoahoc.vietjack.com/thi-online/de-kiem-tra-15-phut-toan-9-chuong-2-dai-so-co-dap-an/43275


\textbf{{QUESTION}}

Cho hàm số bậc nhất y=(m+1)x+5
b) Tìm m để đồ thị hàm số đi qua điểm (1; -3)

\textbf{{ANSWER}}

b) Đồ thị hàm số đi qua điểm (1; -3) khi:
-3 = (m + 1).1 + 5 ⇔ m = -9
Vậy với m = - 9 thì đồ thị hàm số đi qua điểm (1; -3)

========================================================================

https://khoahoc.vietjack.com/thi-online/20-cau-trac-nghiem-toan-10-chan-troi-sang-tao-duong-tron-trong-mat-phang-toa-do-phan-2-co-dap-an/111690


\textbf{{QUESTION}}

Viết phương trình đường tròn tâm I(1; 1) và đi qua điểm M(2; 2) là:
A. (x – 1)2 + (y – 1)2 = 2;
B. (x – 1)2 + (y – 1)2 = $$ \sqrt{2}$$;
C. (x + 1)2 + (y + 1)2 = 2;

\textbf{{ANSWER}}

Hướng dẫn giải 
Đáp án đúng là: A
Với I(1; 1) và M(2; 2) ta có $$ \overrightarrow{IM}=\left(1;1\right)$$.
Bán kính của đường tròn là: R = IM = $$ \left|\overrightarrow{IM}\right|=\sqrt{{1}^{2}+{1}^{2}}=\sqrt{2}$$.
Phương trình đường tròn tâm I(1; 1), bán kính R = $$ \sqrt{2}$$ là:
(x – 1)2 + (y – 1)2 = 2.

========================================================================

https://khoahoc.vietjack.com/thi-online/20-cau-trac-nghiem-toan-10-chan-troi-sang-tao-duong-tron-trong-mat-phang-toa-do-phan-2-co-dap-an/111690


\textbf{{QUESTION}}

Trong mặt phẳng tọa độ Oxy, cho đoạn thẳng AB có A(1; 4) và B(5; 6). Viết phương trình đường tròn đường kính AB.
A. (x – 3)2 + (y – 5)2 = 5;
B. (x + 3)2 + (y + 5)2 = 5;
C. (x – 3)2 + (y – 5)2 = 25;

\textbf{{ANSWER}}

Hướng dẫn giải 
Đáp án đúng là: A
(C) là đường tròn đường kính AB nên (C) có tâm là trung điểm của AB và bán kính bằng một nửa đường kính AB.
Gọi I là trung điểm của AB.
Với A(1; 4) và B(5; 6), suy ra I(3; 5) và →AB=(5−1;6−4)=(4;2)$$ \overrightarrow{AB}=\left(5-1;6-4\right)=\left(4;2\right)$$
Độ dài AB = |→AB|=√42+22=√20=2√5$$ \left|\overrightarrow{AB}\right|=\sqrt{{4}^{2}+{2}^{2}}=\sqrt{20}=2\sqrt{5}$$ 
Suy ra R = 12AB=2√52=√5$$ \frac{1}{2}AB=\frac{2\sqrt{5}}{2}=\sqrt{5}$$ 
Phương trình đường tròn đường kính AB là: (x – 3)2 + (y – 5)2 = 5.

========================================================================

https://khoahoc.vietjack.com/thi-online/20-cau-trac-nghiem-toan-10-chan-troi-sang-tao-duong-tron-trong-mat-phang-toa-do-phan-2-co-dap-an/111690


\textbf{{QUESTION}}

Phương trình tiếp tuyến của đường tròn tâm I(1; 1) tại điểm M(3; 3) nằm trên đường tròn đó là:
A. x – 2y + 1 = 0 ;
B. x + y – 6 = 0;
C. x + y + 1 = 0;

\textbf{{ANSWER}}

Hướng dẫn giải 
Đáp án đúng là: B
Phương trình tiếp tuyến của đường tròn tâm I(1; 1) tại điểm M(3; 3) nằm trên đường tròn là:
(1 – 3)(x – 3) + (1 – 3)(y – 3) = 0
Hay –2x – 2y + 12 = 0 ⇔$$ \Leftrightarrow $$x + y – 6 = 0.
Vậy ta chọn phương án B.

========================================================================

https://khoahoc.vietjack.com/thi-online/20-cau-trac-nghiem-toan-10-chan-troi-sang-tao-duong-tron-trong-mat-phang-toa-do-phan-2-co-dap-an/111690


\textbf{{QUESTION}}

Phương trình tiếp tuyến của đường tròn có phương trình: x2 + y2 – 2x – 4y + 4 = 0 tại điểm M nằm trên trục tung là:
A. x = 0 ; 
B. x + 2y – 1 = 0; 
C. 3x + 2y – 1 = 0;

\textbf{{ANSWER}}

Hướng dẫn giải 
Đáp án đúng là: A
Đường tròn có phương trình x2 + y2 – 2x – 4y + 4 = 0 có tâm I(1; 2).
Điểm M nằm trên trục tung nên M(0; y0).
Thay x = 0 vào phương trình đường tròn ta được:
02 + y02 – 2 . 0 – 4y0 + 4 = 0 ⇔$$ \Leftrightarrow $$ y02 – 4y0 + 4 = 0.
⇔$$ \Leftrightarrow $$ (y0 – 2)2 = 0 Û y0 – 2 = 0 ⇔$$ \Leftrightarrow $$ y0 = 2.
Khi đó M(0; 2).
Phương trình tiếp tuyến của đường tròn tâm I(1; 2) tại điểm M(0; 2) là:
(1 – 0)(x – 0) + (2 – 2)(y – 2) = 0
⇔$$ \Leftrightarrow $$ x = 0.

========================================================================

https://khoahoc.vietjack.com/thi-online/20-cau-trac-nghiem-toan-10-chan-troi-sang-tao-duong-tron-trong-mat-phang-toa-do-phan-2-co-dap-an/111690


\textbf{{QUESTION}}

Viết phương trình đường tròn tâm I đi qua 3 điểm A(1; 1), B(2; 3) và C(4; 6).
A. x2 + y2 – 5x + y + 26 = 0;
B. x2 + y2 – 4x + 17y + 26 = 0;
C. x2 + y2 – 45x + 17y + 26 = 0;

\textbf{{ANSWER}}

Hướng dẫn giải 
Đáp án đúng là: C
Gọi M, N lần lượt là trung điểm của AB, AC.
Khi đó M(32;2),  N(52;72)$$ M\left(\frac{3}{2};2\right),\text{\hspace{0.17em}\hspace{0.17em}}N\left(\frac{5}{2};\frac{7}{2}\right)$$
Đường trung trực d của đoạn thẳng AB là đường thẳng đi qua M và nhận →AB=(1;2)$$ \overrightarrow{AB}=\left(1;2\right)$$ làm vectơ pháp tuyến nên có phương trình:
x−32+2(y−2)=0⇔2x+4y−11=0$$ x-\frac{3}{2}+2\left(y-2\right)=0\Leftrightarrow 2x+4y-11=0$$
Đường trung trực ∆$$ ∆$$ của đoạn thẳng AC là đường thẳng đi qua N và nhận →AC=(3;5)$$ \overrightarrow{AC}=\left(3;5\right)$$ làm vectơ pháp tuyến nên có phương trình:
3(x−52)+5(y−72)=0⇔3x+5y−25=0$$ 3\left(x-\frac{5}{2}\right)+5\left(y-\frac{7}{2}\right)=0\Leftrightarrow 3x+5y-25=0$$
Đường thẳng d cắt đường thẳng ∆$$ ∆$$ cắt nhau tại điểm I(452;−172)$$ I\left(\frac{45}{2};-\frac{17}{2}\right)$$ cách đều ba điểm A, B, C.
Do đó đường tròn đi qua ba điểm A, B, C có tâm I(452;−172)$$ I\left(\frac{45}{2};-\frac{17}{2}\right)$$ và bán kính R2=IA2=(1−452)2+(1+172)2=11052$$ {R}^{2}=I{A}^{2}={\left(1-\frac{45}{2}\right)}^{2}+{\left(1+\frac{17}{2}\right)}^{2}=\frac{1105}{2}$$
Ta có (452)2+(−172)2−11052=26$$ {\left(\frac{45}{2}\right)}^{2}+{\left(-\frac{17}{2}\right)}^{2}-\frac{1105}{2}=26$$
Khi đó đường tròn (C) có phương trình là:
x2 + y2 – 45x + 17y + 36 = 0.

========================================================================

https://khoahoc.vietjack.com/thi-online/10-bai-tap-tasp-con-hai-tap-hop-bang-nhau-co-loi-giai


\textbf{{QUESTION}}

Cho tập hợp A = {2; 4; 6; 8}. Số tập con của tập hợp A là?
A. 15;
B. 16;
C. 17;
D. 18.

\textbf{{ANSWER}}

Đáp án đúng là: B.
Ta có:
+ Các tập con có 0 phần tử: ∅.
+ Các tập con có 1 phần tử: {2}, {4}, {6}, {8}.
+ Các tập con có 2 phần tử: {2; 4}, {2; 6}, {2; 8}, {4; 6}, {4; 8}, {6; 8}.
+ Các tập con có 3 phần tử: {2; 4; 6}, {2; 4; 8}, {2; 6; 8}, {4; 6; 8}.
+ Các tập con có 4 phần tử: {2; 4; 6; 8}.
Vậy tập hợp A có 16 tập con.

========================================================================

https://khoahoc.vietjack.com/thi-online/10-bai-tap-tasp-con-hai-tap-hop-bang-nhau-co-loi-giai


\textbf{{QUESTION}}

Tập hợp B = {0; a; b} có bao nhiêu tập con?
A. 5;
B. 6;
C. 7;
D. 8.

\textbf{{ANSWER}}

Đáp án đúng là: D.
Ta có:
+ Các tập con có 0 phần tử: ∅.
+ Các tập con có 1 phần tử: {0}, {a}, {b}.
+ Các tập con có 2 phần tử: {0; a}, {0; b}, {a; b}.
+ Các tập con có 3 phần tử: {0; a; b}.
Vậy tập hợp A có 8 tập con.

========================================================================

https://khoahoc.vietjack.com/thi-online/10-bai-tap-tasp-con-hai-tap-hop-bang-nhau-co-loi-giai


\textbf{{QUESTION}}

A. 16;
B. 17;
C. 18;
D. 19.

\textbf{{ANSWER}}

Đáp án đúng là: A.
Ta có:
(x2 – 4)(x2 – 4x + 3) = 0
⇔[x2– 4=0x2– 4x + 3=0$$ \left[\begin{array}{c}{x}^{2}–\text{ }4=0\\ {x}^{2}–\text{ }4x\text{ }+\text{ }3=0\end{array}\right.$$   ⇔ [x=2x=−2x=1x=3$$ \left[\begin{array}{c}x=2\\ x=-2\\ x=1\\ x=3\end{array}\right.$$ .
Vì x ∈ ℤ nên 4 nghiệm trên đều thỏa mãn.
Vậy C = {– 2; 2; 1; 3}.
Ta lại có:
+ Các tập con có 0 phần tử: ∅.
+ Các tập con có 1 phần tử: {– 2}, {2}, {1}, {3}.
+ Các tập con có 2 phần tử: {– 2; 2}, {– 2; 1}, {– 2; 3}, {2; 1}, {2; 3}, {1; 3}.
+ Các tập con có 3 phần tử: {– 2; 2; 1}, {– 2; 2; 3}, {– 2; 1; 3}, {2; 1; 3}.
+ Các tập con có 4 phần tử: {– 2; 2; 1; 3}.
Vậy tập hợp C có 16 tập con.

========================================================================

https://khoahoc.vietjack.com/thi-online/10-bai-tap-tasp-con-hai-tap-hop-bang-nhau-co-loi-giai


\textbf{{QUESTION}}

Số các tập con có 2 phần tử của tập hợp D = {1; 2; 3; 4; 5} là:
A. 8;
B. 9;
C. 10;
D. 11.

\textbf{{ANSWER}}

Đáp án đúng là: C.
Ta có các tập con chứa hai phần tử của tập hợp D là:
{1; 2}, {1; 3}, {1; 4}, {1; 5}, {2; 3}, {2; 4}, {2; 5}, {3; 4}, {3; 5}, {4; 5}.
Do đó có tất cả 10 tập con chứa 2 phần tử.

========================================================================

https://khoahoc.vietjack.com/thi-online/10-bai-tap-tasp-con-hai-tap-hop-bang-nhau-co-loi-giai


\textbf{{QUESTION}}

Cho tập hợp E = {a; b; c}. Mệnh đề nào sau đây sai?
A. {a; b} = E;
B. ∅ ⊂ E;
C. {a} ⊂ E;
D. {d} ⊄ E.

\textbf{{ANSWER}}

Đáp án đúng là: A.
A. Ta thấy mệnh đề ở câu A sai do tập hợp E có 3 phần tử là a, b, c. Còn tập hợp {a; b} chỉ có 2 phần tử là a, b nên 2 tập hợp trên không bằng nhau.
B. Theo lý thuyết ta có ∅ ⊂ E, với mọi tập hợp E.
Do đó mệnh đề ở câu B đúng.
C. Ta thấy tập hợp {a} có 1 phần tử là a.
Mà phần tử a cũng thuộc tập hợp E.
Vậy {a} ⊂ E.
Do đó mệnh đề ở câu C đúng.
D. Ta thấy tập hợp {d} có 1 phần tử là d.
Mà phần tử d không thuộc tập hợp E.
Vậy {d} ⊄ E.
Do đó mệnh đề ở câu D đúng.

========================================================================

https://khoahoc.vietjack.com/thi-online/de-kiem-tra-giua-ki-1-toan-11-ctst-co-dap-an


\textbf{{QUESTION}}

Đổi số đo của góc $\alpha = 30^\circ $ sang rađian.
D. $\alpha = \frac{\pi }{3}.$

\textbf{{ANSWER}}

Chọn C

========================================================================

https://khoahoc.vietjack.com/thi-online/giai-sbt-toan-8-kntt-bai-6-hieu-hai-binh-phuong-binh-phuong-cua-mot-tong-hay-mot-hieu-co-dap-an


\textbf{{QUESTION}}

Những đẳng thức nào sau đây là hằng đẳng thức?
a) a2 – b2 = (a – b)(a + b);

\textbf{{ANSWER}}

a) Ta có: (a – b)(a + b) = a(a + b) – b(a + b)
                                     = a2 + ab – ab – b2 = a2 – b2.
Vậy đẳng thức a2 – b2 = (a – b)(a + b) là hằng đẳng thức.

========================================================================

https://khoahoc.vietjack.com/thi-online/giai-sbt-toan-8-kntt-bai-6-hieu-hai-binh-phuong-binh-phuong-cua-mot-tong-hay-mot-hieu-co-dap-an


\textbf{{QUESTION}}

b) 3x(2x – 1) = 6x2 + 3x;

\textbf{{ANSWER}}

b) Xét đẳng thức 3x(2x – 1) = 6x2 + 3x
Khi thay x = 1 vào hai vế của đẳng thức ta thấy VT = 3 và VP = 9, do đó hai vế không bằng nhau.
Vậy đẳng thức 3x(2x – 1) = 6x2 + 3x không phải là hằng đẳng thức.

========================================================================

https://khoahoc.vietjack.com/thi-online/giai-sbt-toan-8-kntt-bai-6-hieu-hai-binh-phuong-binh-phuong-cua-mot-tong-hay-mot-hieu-co-dap-an


\textbf{{QUESTION}}

Những đẳng thức nào sau đây là hằng đẳng thức?
c) 2(x – 1) = 4x + 3;

\textbf{{ANSWER}}

c) Xét đẳng thức 2(x – 1) = 4x + 3
Khi thay x = 0 vào hai vế của đẳng thức ta thấy VT = –2 và VP = 3, do đó hai vế không bằng nhau.
Vậy đẳng thức 2(x – 1) = 4x + 3 không phải là hằng đẳng thức.

========================================================================

https://khoahoc.vietjack.com/thi-online/giai-sbt-toan-8-kntt-bai-6-hieu-hai-binh-phuong-binh-phuong-cua-mot-tong-hay-mot-hieu-co-dap-an


\textbf{{QUESTION}}

d) (2y + 3)(y + 1) = 2y2 + 5y + 3.

\textbf{{ANSWER}}

d) Ta có: (2y + 3)(y + 1) = 2y(y + 1) + 3(y + 1)
                                        = 2y2 + 2y + 3y + 3 = 2y2 + 5y + 3. 
Vậy đẳng thức (2y + 3)(y + 1) = 2y2 + 5y + 3 là hằng đẳng thức.

========================================================================

https://khoahoc.vietjack.com/thi-online/giai-sbt-toan-8-kntt-bai-6-hieu-hai-binh-phuong-binh-phuong-cua-mot-tong-hay-mot-hieu-co-dap-an


\textbf{{QUESTION}}

Khai triển:

\textbf{{ANSWER}}

a) (3x + 1)2  = (3x)2 + 2.3x.1 + 12 = 9x2 + 6x +1.

========================================================================

https://khoahoc.vietjack.com/thi-online/10-cau-trac-nghiem-toan-6-canh-dieu-bai-4-phep-nhan-phep-chia-cac-so-tu-nhien-co-dap-an-nhan-biet


\textbf{{QUESTION}}

A.12 là thừa số 
B.5 là thừa số
C.60 là tích
D.60 là thương

\textbf{{ANSWER}}

Trong phép tính 12 × 5 = 60, có 12 và 5 là các thừa số và 60 là tích. 
Vậy đáp án A, B, C đúng và D sai.
Chọn đáp án D.

========================================================================

https://khoahoc.vietjack.com/thi-online/luyen-tap-nhan-chia-hai-luy-thua-cung-co-so/57604


\textbf{{QUESTION}}

Viết các tích sau dưới dạng một lũy thừa.
a, $$ {4}^{8}.{4}^{10}$$
b, $$ {2}^{20}.{2}^{7}$$
c, $$ {5}^{12}.{5}^{5}.{5}^{4}$$
d, $$ {4}^{3}.{4}^{5}.{4}^{5}$$
e, $$ {8}^{6}.{8}^{5}.{8}^{5}$$
f, $$ {x}^{7}.{x}^{4}.{x}^{3}$$

\textbf{{ANSWER}}

a, $$ {4}^{8}.{4}^{10}={4}^{18}$$
b, $$ {2}^{20}.{2}^{7}={2}^{27}$$
c, $$ {5}^{12}.{5}^{5}.{5}^{4}={5}^{21}$$
d, $$ {4}^{3}.{4}^{5}.{4}^{5}={4}^{13}$$
e, $$ {8}^{6}.{8}^{5}.{8}^{5}={8}^{16}$$
f, $$ {x}^{7}.{x}^{4}.{x}^{3}={x}^{14}$$

========================================================================

https://khoahoc.vietjack.com/thi-online/luyen-tap-nhan-chia-hai-luy-thua-cung-co-so/57604


\textbf{{QUESTION}}

Viết các thương sau dưới dạng một lũy thừa.
a, $$ {4}^{5}:4$$
b, $$ {2}^{10}:{2}^{3}$$
c, $$ {x}^{9}:{x}^{3}\left(x\ne 0\right)$$
d, $$ {5}^{103}:{5}^{3}$$

\textbf{{ANSWER}}

a, $$ {4}^{5}:4={4}^{4}$$
b, $$ {2}^{10}:{2}^{3}={2}^{7}$$
c, $$ {x}^{9}:{x}^{3}={x}^{6}\left(x\ne 0\right)$$
d, $$ {5}^{103}:{5}^{3}={5}^{100}$$

========================================================================

https://khoahoc.vietjack.com/thi-online/luyen-tap-nhan-chia-hai-luy-thua-cung-co-so/57604


\textbf{{QUESTION}}

Tìm x∈$$ \in $$N, biết.
a, 2x.22=32$$ {2}^{x}.{2}^{2}=32$$
b, 27.3x=243$$ 27.{3}^{x}=243$$
c, 2x.24=1024$$ {2}^{x}.{2}^{4}=1024$$
d, 49.7x=2401$$ 49.{7}^{x}=2401$$

\textbf{{ANSWER}}

a, 2x.22=32$$ {2}^{x}.{2}^{2}=32$$
2x+2=25$$ {2}^{x+2}={2}^{5}$$
x + 2 = 5
x = 3
Vậy x = 3
b, 27.3x=243$$ 27.{3}^{x}=243$$
33.3x=35$$ {3}^{3}.{3}^{x}={3}^{5}$$
33+x=35$$ {3}^{3+x}={3}^{5}$$
x + 3 = 5
x = 2
Vậy x = 2
c, 2x.24=1024$$ {2}^{x}.{2}^{4}=1024$$
2x+4=210$$ {2}^{x+4}={2}^{10}$$
x + 4 = 10
x = 6
Vậy x = 6
d, 49.7x=2401$$ 49.{7}^{x}=2401$$
72.7x=74$$ {7}^{2}.{7}^{x}={7}^{4}$$
72+x=74$$ {7}^{2+x}={7}^{4}$$
2 + x = 4
x = 2
Vậy x = 2

========================================================================

https://khoahoc.vietjack.com/thi-online/luyen-tap-nhan-chia-hai-luy-thua-cung-co-so/57604


\textbf{{QUESTION}}

Tìm x∈$$ \in $$N, biết.
a, 3x+1:34=81$$ {3}^{x+1}:{3}^{4}=81$$
b, 3x+3.3x+1=729$$ {3}^{x+3}.{3}^{x+1}=729$$
c, 2x+3.2x=128$$ {2}^{x+3}.{2}^{x}=128$$
d, 23+3x=56:53$$ 23+3x={5}^{6}:{5}^{3}$$
e, 2x+2x+4=272$$ {2}^{x}+{2}^{x+4}=272$$

\textbf{{ANSWER}}

a, 3x+1:34=81$$ {3}^{x+1}:{3}^{4}=81$$
3x-3=34$$ {3}^{x-3}={3}^{4}$$
x – 3 = 4
x = 7
Vậy x = 7
b, 3x+3.3x+1=729$$ {3}^{x+3}.{3}^{x+1}=729$$
32x+4=36$$ {3}^{2x+4}={3}^{6}$$
2x + 4 = 6
x = 1
Vậy x = 1
c, 2x+3.2x=128$$ {2}^{x+3}.{2}^{x}=128$$
22x+3=27$$ {2}^{2x+3}={2}^{7}$$
2x + 3 = 7
x = 2
Vậy x = 2
d, 23+3x=56:53$$ 23+3x={5}^{6}:{5}^{3}$$
23+3x=53$$ 23+3x={5}^{3}$$
23 + 3x = 125
3x = 102
x = 34
Vậy x = 34
e, 2x+2x+4=272$$ {2}^{x}+{2}^{x+4}=272$$
2x+2x.24=272$$ {2}^{x}+{2}^{x}.{2}^{4}=272$$
2x(1+24)=272$$ {2}^{x}(1+{2}^{4})=272$$
2x.17=272$$ {2}^{x}.17=272$$
2x=16$$ {2}^{x}=16$$
2x=24$$ {2}^{x}={2}^{4}$$
x = 4
Vậy x = 4

========================================================================

https://khoahoc.vietjack.com/thi-online/luyen-tap-nhan-chia-hai-luy-thua-cung-co-so/57604


\textbf{{QUESTION}}

Thực hiện các phép tính sau bằng cách hợp lý
a, (217+172)(915-315)(24-42)$$ \left({2}^{17}+{17}^{2}\right)\left({9}^{15}-{3}^{15}\right)\left({2}^{4}-{4}^{2}\right)$$
b, (12+23+34+45)$$ \left({1}^{2}+{2}^{3}+{3}^{4}+{4}^{5}\right)$$(13+23+33+43)(38-812)$$ ({1}^{3}+{2}^{3}+{3}^{3}+{4}^{3})({3}^{8}-{81}^{2})$$
c, (724+723):722$$ \left({7}^{24}+{7}^{23}\right):{7}^{22}$$

\textbf{{ANSWER}}

a, $$ \left({2}^{17}+{17}^{2}\right)\left({9}^{15}-{3}^{15}\right)\left({2}^{4}-{4}^{2}\right)$$
= $$ \left({2}^{17}+{17}^{2}\right)\left({9}^{15}-{3}^{15}\right)\left(16-16\right)$$
= $$ \left({2}^{17}+{17}^{2}\right)\left({9}^{15}-{3}^{15}\right).0$$
= 0
b, $$ \left({1}^{2}+{2}^{3}+{3}^{4}+{4}^{5}\right)$$.$$ ({1}^{3}+{2}^{3}+{3}^{3}+{4}^{3})({3}^{8}-{81}^{2})$$
= $$ \left({1}^{2}+{2}^{3}+{3}^{4}+{4}^{5}\right)$$.$$ ({1}^{3}+{2}^{3}+{3}^{3}+{4}^{3})({3}^{4.2}-{81}^{2})$$
= $$ \left({1}^{2}+{2}^{3}+{3}^{4}+{4}^{5}\right)$$.$$ ({1}^{3}+{2}^{3}+{3}^{3}+{4}^{3})({81}^{2}-{81}^{2})$$
= $$ \left({1}^{2}+{2}^{3}+{3}^{4}+{4}^{5}\right)$$.$$ ({1}^{3}+{2}^{3}+{3}^{3}+{4}^{3}).0$$
= 0
c, $$ \left({7}^{24}+{7}^{23}\right):{7}^{22}$$
= $$ {7}^{24}:{7}^{22}+{7}^{23}:{7}^{22}$$
= $$ {7}^{24-22}+{7}^{23-22}$$
= $$ {7}^{2}+{7}^{1}$$
= 49 + 7 = 56

========================================================================

https://khoahoc.vietjack.com/thi-online/10-bai-tap-phan-tich-mot-sod-ra-thua-so-nguyen-to-co-loi-giai


\textbf{{QUESTION}}

Dạng phân tích số 18 ra thừa số nguyên tố là:
A. 18 = 2.9;
B. 18 = 2.3.3;
C. 18 = 32 + 32;
D. 18 = 3.(3 + 3).

\textbf{{ANSWER}}

Đáp án đúng là: B
Câu A sai vì 9 không phải là số nguyên tố .
Câu B đúng vì 2.3.3 là một tích và các thừa số 2; 3; 3 đều là các số nguyên tố.
Câu C sai vì 32 + 32 không phải là một tích.
Câu D sai vì 3.(3 + 3) không phải dạng tích.

========================================================================

https://khoahoc.vietjack.com/thi-online/10-bai-tap-phan-tich-mot-sod-ra-thua-so-nguyen-to-co-loi-giai


\textbf{{QUESTION}}

Tìm hai số tự nhiên liên tiếp có tích bằng 1190.
A. 34; 35;
B. 36; 37;
C. 37; 38;
D. 38; 39.

\textbf{{ANSWER}}

Đáp án đúng là: A
Ta phân tích 1190 thành tích các thừa số nguyên tố:
1190
595
119
17
1
2
5
7
17
 
1190 = 2.5.7.17.
Từ đó ta nhóm khéo 2.17 = 34; 5.7 = 35
Ta được 1190 = 34.35.

========================================================================

https://khoahoc.vietjack.com/thi-online/10-bai-tap-phan-tich-mot-sod-ra-thua-so-nguyen-to-co-loi-giai


\textbf{{QUESTION}}

Tìm ba số tự nhiên liên tiếp có tích bằng 2184.
A. 13.14.15;
B. 11.12.13;
C. 12.13.14;
D. 17.18.19.

\textbf{{ANSWER}}

Đáp án đúng là: C
Ta phân tích 2184 thành tích các thừa số nguyên tố:
2184
1092
546
273
91
13
1
2
2
2
3
7
13
 
2184 = 2.2.2.3.7.13
Từ đó ta nhóm khéo 2.2.3 = 12; 2.7 = 14
Ta được 2184 = 12.13.14.

========================================================================

https://khoahoc.vietjack.com/thi-online/10-bai-tap-phan-tich-mot-sod-ra-thua-so-nguyen-to-co-loi-giai


\textbf{{QUESTION}}

Phân tích 100 ra thừa số nguyên tố có dạng 100 = 2x.5y. Giá trị xủa x + y là?
A. 2;
B. 3;
C. 4;
D. 5.

\textbf{{ANSWER}}

Đáp án đúng là: C
Ta phân tích 100 thành tích các thừa số nguyên tố:
100
50
25
5
1
2
2
5
5
 
100 = 22.52 suy ra x + y = 4.

========================================================================

https://khoahoc.vietjack.com/thi-online/10-bai-tap-phan-tich-mot-sod-ra-thua-so-nguyen-to-co-loi-giai


\textbf{{QUESTION}}

Phân tích 42.63.12 thành tích của các thừa số nguyên tố ?
A. 29.34;
B. 28.34;
C. 27.34.4;
D. 26.34.

\textbf{{ANSWER}}

Đáp án đúng là: A
42.63.12 = (22)2.(2.3)3.4.3 = 24.23.33.22.3 = 29.34.

========================================================================

https://khoahoc.vietjack.com/thi-online/trac-nghiem-bai-tap-theo-tuan-toan-7-tuan-12-co-dap-an


\textbf{{QUESTION}}

Cho biết x và y là hai đại lượng tỉ lệ thuận với nhau và khi $$ x=3$$ thì $$ y=-2,7$$ 
Tìm hệ số tỉ lệ k của y đối với x và biểu biễn y theo x

\textbf{{ANSWER}}

x và y là hai đại lượng tỉ lệ thuận với nhau nên $$ y\text{ }=\text{ }kx$$. $$ (k\ne 0)$$ 
Khi $$ x=3$$ thì $$ y=-2,7$$ ta có: $$ -2,7=k.3\Rightarrow k=-0,9$$ 
Vậy hệ số tỉ lệ k của y đối với x là:-0,9. Biểu diễn y theo x là: $$ y=-0,9.x$$

========================================================================

https://khoahoc.vietjack.com/thi-online/trac-nghiem-bai-tap-theo-tuan-toan-7-tuan-12-co-dap-an


\textbf{{QUESTION}}

Cho biết x và y là hai đại lượng tỉ lệ thuận với nhau và khi x=3$$ x=3$$ thì y=−2,7$$ y=-2,7$$
Cho biết x
 và y
 
 là hai đại lượng tỉ lệ thuận với nhau và khi x=3$$ x=3$$
 
x=3
x=3
x
=
3
 thì y=−2,7$$ y=-2,7$$
 
y=−2,7
y=−2,7
y
=
−
2
,
7
 Tính giá trị của y khi x=2 và tính giá trị của x khi y=0,9
 Tính giá trị của y khi x=2 và tính giá trị của x khi y=0,9
 
 
Tính giá trị của y
 khi x=2
 và tính giá trị của x 
khi y=0,9

\textbf{{ANSWER}}

* Khi $$ x=-2$$ thay vào biểu thức $$ y=-0,9x$$ ta có:
$$ y\text{ }=-0,9.\left(-2\right)=1,8$$, vậy khi $$ x=-2$$ thì $$ y=1,8$$ 
* Khi $$ y=0,9$$ thay vào biểu thức $$ y=-0,9x$$ ta có:
$$ 0,9=-0,9x\Rightarrow x=-1.$$ Vậy khi $$ y=0,9$$ thì $$ x=-1$$

========================================================================

https://khoahoc.vietjack.com/thi-online/trac-nghiem-bai-tap-theo-tuan-toan-7-tuan-12-co-dap-an


\textbf{{QUESTION}}

Cho biết y tỉ lệ thuận với x theo hệ số tỉ lệ là 7 và x tỉ lệ thuận với z theo hệ số tỉ lệ là 0,3. Hỏi y và z có tỉ lệ thuận với nhau không ? Nếu có hệ số tỉ lệ là bao nhiêu?

\textbf{{ANSWER}}

y tỉ lệ thuận với x theo hệ số tỉ lệ là 7 nên ta có: $$ y=7x$$ (1)
x tỉ lệ thuận với z theo hệ số tỉ lệ là 0,3 nên ta có: $$ x=0,3z$$ (2)
Thay (2) vào (1) ta có: $$ y\text{ }=7.0,3z\quad =2,1z$$ 
Vậy y tỉ lệ thuận với z theo hệ số tỉ lệ là: 2,1

========================================================================

https://khoahoc.vietjack.com/thi-online/trac-nghiem-bai-tap-theo-tuan-toan-7-tuan-12-co-dap-an


\textbf{{QUESTION}}

Nếu y tỉ lệ thuận với x theo hệ số tỉ lệ là a; x$$ a;\text{ }x$$ tỉ lệ thuận với z theo hệ số tỉ lệ là b. Hỏi y và z có tỉ lệ thuận với nhau không? Nếu có hệ số tỉ lệ là bao nhiêu? 
Nếu y
 tỉ lệ thuận với x
 
 theo hệ số tỉ lệ là a; x$$ a;\text{ }x$$
 
a; x
a; x
a
;
 
x
 tỉ lệ thuận với z
 
 theo hệ số tỉ lệ là b. 
 
Hỏi y
 và z
 
 có tỉ lệ thuận với nhau không? Nếu có hệ số tỉ lệ là bao nhiêu?

\textbf{{ANSWER}}

y tỉ lệ thuận với x theo hệ số tỉ lệ là a  nên ta có: y=ax$$ y=ax$$ (*)
x$$ x$$ tỉ lệ thuận với z theo hệ số tỉ lệ là b nên ta có: x=bz$$ x=bz$$ (**)
Thay (**) vào (*) ta có: y=a.b.z =ab.z$$ y=a.b.z\quad =ab.z$$ 
Vậy y tỉ lệ thuận với z theo hệ số tỉ lệ là: k=ab$$ k=ab$$

========================================================================

https://khoahoc.vietjack.com/thi-online/trac-nghiem-bai-tap-theo-tuan-toan-7-tuan-12-co-dap-an


\textbf{{QUESTION}}

Tam giác ABC có số đo các góc A, B, C tỉ lệ với 3; 4; 5. Tính số đo các góc của tam giác.

\textbf{{ANSWER}}

Gọi số đo các góc ^A,  ^B,  ˆC$$ \widehat{A,}\text{\hspace{0.17em}\hspace{0.17em}}\widehat{B,}\text{\hspace{0.17em}\hspace{0.17em}}\widehat{C}$$ của ΔABC$$ \Delta ABC$$ lần lượt là a; b; c$$ a;\text{ }b;\text{ }c$$ (0<a; b; c<180°)$$ \left(0<a;\text{ }b;\text{ }c<180°\right)$$
Theo bài ra ta có: a3=b4=c5$$ \frac{a}{3}=\frac{b}{4}=\frac{c}{5}$$ và  a+b+c=180°$$ a+b+c=180°$$
Áp dụng tính chất của dãy tỉ số bằng nhau ta có: a3=b4=c5=a+b+c3+4+5=180°12=15°$$ \frac{a}{3}=\frac{b}{4}=\frac{c}{5}=\frac{a+b+c}{3+4+5}=\frac{180°}{12}=15°$$
⇒a3=15⇒a=15.3=45°$$ \Rightarrow \frac{a}{3}=15\Rightarrow a=15.3=45°$$; 
b4=15⇒b=15.4=60°$$ \frac{b}{4}=15\Rightarrow b=15.4=60°$$; 
c4=15⇒c=15.5=75°$$ \frac{c}{4}=15\Rightarrow c=15.5=75°$$
Vậy số đo các góc ^A,  ^B,  ˆC$$ \widehat{A,}\text{\hspace{0.17em}\hspace{0.17em}}\widehat{B,}\text{\hspace{0.17em}\hspace{0.17em}}\widehat{C}$$ của ΔABC$$ \Delta ABC$$ lần lượt là 45°;60°;75°$$ 45°;60°;75°$$

========================================================================

https://khoahoc.vietjack.com/thi-online/20-cau-trac-nghiem-logarit-co-dap-an-nhan-biet


\textbf{{QUESTION}}

Logarit cơ số a của b kí hiệu là:
A. $$ {\mathrm{log}}_{a}b$$
B. $$ {\mathrm{log}}_{b}a$$
C. $$ {\mathrm{ln}}_{a}b$$
D. $$ {\mathrm{ln}}_{b}a$$

\textbf{{ANSWER}}

Số $$ {\mathrm{log}}_{a}b$$ gọi là Logarit cơ số a của b.
Đáp án cần chọn là: A

========================================================================

https://khoahoc.vietjack.com/thi-online/20-cau-trac-nghiem-logarit-co-dap-an-nhan-biet


\textbf{{QUESTION}}

Logarit cơ số 2 của 5 được viết là:
A. $$ {\mathrm{log}}_{5}2$$
B. 2log5
C. log(25)
D. $$ {\mathrm{log}}_{2}5$$

\textbf{{ANSWER}}

Logarit cơ số 2 của 5 được viết là: $$ {\mathrm{log}}_{2}5$$
Đáp án cần chọn là: D

========================================================================

https://khoahoc.vietjack.com/thi-online/20-cau-trac-nghiem-logarit-co-dap-an-nhan-biet


\textbf{{QUESTION}}

Điều kiện để logab$$ {\mathrm{log}}_{a}b$$ có nghĩa là
A. a<0,b>0
B. 0<a≠1,b<0$$ 0<a\ne 1,b<0$$
C. 0<a≠1,b>0$$ 0<a\ne 1,b>0$$
D. 0<a≠1,0<b≠1$$ 0<a\ne 1,0<b\ne 1$$

\textbf{{ANSWER}}

Điều kiện để logab$$ {\mathrm{log}}_{a}b$$ có nghĩa là: 0<a≠1,b>0$$ 0<a\ne 1,b>0$$
Đáp án cần chọn là: C

========================================================================

https://khoahoc.vietjack.com/thi-online/20-cau-trac-nghiem-logarit-co-dap-an-nhan-biet


\textbf{{QUESTION}}

Điều kiện để biểu thức log2(3-x)$$ {\mathrm{log}}_{2}\left(3-x\right)$$ xác định là
A. x≤3$$ x\le 3$$
B. x>3
C. x≥3$$ x\ge 3$$
D. x<3

\textbf{{ANSWER}}

Để biểu thức log2(3-x)$$ {\mathrm{log}}_{2}(3-x)$$ xác định thì 3-x>0⇔x<3$$ 3-x>0\Leftrightarrow x<3$$
Đáp án cần chọn là: D

========================================================================

https://khoahoc.vietjack.com/thi-online/20-cau-trac-nghiem-logarit-co-dap-an-nhan-biet


\textbf{{QUESTION}}

Cho a>0;a≠1,b>0$$ a>0;a\ne 1,b>0$$, khi đó nếu logab=N$$ {\mathrm{log}}_{a}b=N$$ thì
A. ab=N$$ {a}^{b}=N$$
B. logaN=b$$ {\mathrm{log}}_{a}N=b$$
C. aN=b$$ {a}^{N}=b$$
D. bN=a$$ {b}^{N}=a$$

\textbf{{ANSWER}}

Cho a>0;a≠1,b>0$$ a>0;a\ne 1,b>0$$, khi đó nếu logab=N$$ {\mathrm{log}}_{a}b=N$$ thì aN=b$$ {a}^{N}=b$$
Đáp án cần chọn là: C

========================================================================

https://khoahoc.vietjack.com/thi-online/10-bai-tap-van-dunsg-dinh-nghia-dao-ham-vao-giai-quyet-mot-so-bai-toan-thuc-tien-co-loi-giai


\textbf{{QUESTION}}

Một vật rơi tự do với phương trình chuyển động $$ \text{s}=\frac{1}{2}{\text{gt}}^{\text{2}}$$, trong đó g = 9,8 m/s2 và t tính bằng giây. Vận tốc của vật tại thời điểm t = 7 là
A. 66,8 m/s;
B. 88,6 m/s;
C. 68,6 m/s;

\textbf{{ANSWER}}

Hướng dẫn giải:
Đáp án đúng là: C
Vận tốc của vật tại thời điểm t = 7 là:
v(7) = s'(7) = $$ \underset{\text{t}\to 7}{\mathrm{lim}}\frac{\frac{1}{2}\cdot 9,8{t}^{2}-\frac{1}{2}\cdot 9,8\cdot {7}^{2}}{\text{t}-7}=\underset{\text{t}\to 7}{\mathrm{lim}}\frac{4,9\left(\text{t}-7\right)\left(\text{t}+7\right)}{\text{t}-7}=\underset{\text{t}\to 7}{\mathrm{lim}}\left[4,9\left(\text{t}+7\right)\right]=68,6$$.
Vậy vận tốc của vật tại thời điểm t = 7 là 68,6 m/s.

========================================================================

https://khoahoc.vietjack.com/thi-online/10-bai-tap-van-dunsg-dinh-nghia-dao-ham-vao-giai-quyet-mot-so-bai-toan-thuc-tien-co-loi-giai


\textbf{{QUESTION}}

Cho chuyển động thẳng xác định bởi phương trình s(t) = t3 – t2 + 2t + 5, trong đó t tính bằng giây, và s tính bằng mét. Vận tốc của chuyển động tại thời điểm t = 3 là:
A. 23 m/s;
B. 32 m/s;
C. 23 cm/s;

\textbf{{ANSWER}}

Hướng dẫn giải:
Đáp án đúng là: A
Vận tốc của vật tại thời điểm t = 3 là:
v(3) = s'(3) = $$ \underset{\text{t}\to 3}{\mathrm{lim}}\frac{{\text{t}}^{3}-{\text{t}}^{\text{2}}+2\text{t}+5-29}{\text{t}-3}=\underset{\text{t}\to 3}{\mathrm{lim}}\frac{{\text{t}}^{3}-{\text{t}}^{\text{2}}+2\text{t}-24}{\text{t}-3}=\underset{\text{t}\to 3}{\mathrm{lim}}{\text{(t}}^{2}+2\text{t}+8)=23$$.
Vậy vận tốc của vật tại thời điểm t = 3 là 23 m/s.

========================================================================

https://khoahoc.vietjack.com/thi-online/10-bai-tap-van-dunsg-dinh-nghia-dao-ham-vao-giai-quyet-mot-so-bai-toan-thuc-tien-co-loi-giai


\textbf{{QUESTION}}

Chi phí sản xuất x mét vải là C(x) = 1200 + 12x – 0,1x2. Ta gọi chi phí biên là chi phí gia tăng để sản xuất thêm một mét vải từ x mét vải lên x + 1 mét vải. Khi đó hàm chi phí biên C'(x) là:
A. 12 + 0,2x;
B. 0,2x – 12;
C. –12 – 0,2x;

\textbf{{ANSWER}}

Hướng dẫn giải:
Đáp án đúng là: D
Ta có: C'(x0)=limx→x01200+12x−0,1x2−1200−12x0+0,1x0x−x0$$ {\text{C'(x}}_{\text{0}}\text{)}=\underset{\text{x}\to {\text{x}}_{\text{0}}}{\text{lim}}\frac{1200+12\text{x}-0,1{\text{x}}^{2}-1200-12{\text{x}}_{0}+0,1{\text{x}}_{0}}{\text{x}-{\text{x}}_{0}}$$
=limx→x012(x−x0)−0,1(x2−x0)x−x0$$ =\underset{\text{x}\to {\text{x}}_{\text{0}}}{\text{lim}}\frac{12(\text{x}-{\text{x}}_{0})-0,1({\text{x}}^{2}-{\text{x}}_{0})}{\text{x}-{\text{x}}_{0}}$$
=limx→x0[12−0,1(x+x0)]$$ =\underset{\text{x}\to {\text{x}}_{\text{0}}}{\text{lim}}\left[12-0,1(\text{x}+{\text{x}}_{0})\right]$$
= 12 – 0,2x0.
Vậy C'(x) = 12 – 0,2x.

========================================================================

https://khoahoc.vietjack.com/thi-online/10-bai-tap-van-dunsg-dinh-nghia-dao-ham-vao-giai-quyet-mot-so-bai-toan-thuc-tien-co-loi-giai


\textbf{{QUESTION}}

A. 4 A;
B. 5 A;
C. 6 A;

\textbf{{ANSWER}}

Hướng dẫn giải:
Đáp án đúng là: B                                              Cường độ của dòng điện trong dây dẫn tại t = 8 là 
I(8) = Q'(8) = limt→85t+3−43t−8=limt→85(t−8)t−8=limt→85=5$$ \underset{\text{t}\to 8}{\mathrm{lim}}\frac{5t+3-43}{\text{t}-8}=\underset{\text{t}\to 8}{\mathrm{lim}}\frac{5\left(t-8\right)}{\text{t}-8}=\underset{\text{t}\to 8}{\mathrm{lim}}5=5$$.
Vậy I(8) = 5 A.

========================================================================

https://khoahoc.vietjack.com/thi-online/10-bai-tap-van-dunsg-dinh-nghia-dao-ham-vao-giai-quyet-mot-so-bai-toan-thuc-tien-co-loi-giai


\textbf{{QUESTION}}

Một vật được phóng theo phương thẳng đứng lên trên từ mặt đất với vận tốc ban đầu là 19,6 m/s thì độ cao h của nó (tính bằng mét) sau t giây được cho bởi công thức h = 19,6t – 4,9t2. Vận tốc của vật khi nó chạm đất là
A. 9,6 m/s;
B. – 9,6 m/s;
C. – 19,6 m/s;

\textbf{{ANSWER}}

Hướng dẫn giải:
Đáp án đúng là: C
+ Đặt h = f(t) = 19,6t – 4,9t2.
Với x0 bất kì, ta có:
0)=limt→t019,6t−4,9t2−19,6t0+4,9t20t−t0=limt→t0−4,9(t2−t20)+19,6(t−t0)t−t0$$ {}_{\text{0}}\text{)}=\underset{\text{t}\to {\text{t}}_{\text{0}}}{\text{lim}}\frac{19,6t-4,9{t}^{2}-19,6{t}_{0}+4,9{t}_{0}^{2}}{\text{t}-{\text{t}}_{0}}=\underset{t\to {t}_{0}}{\mathrm{lim}}\frac{-4,9\left({t}^{2}-{t}_{0}^{2}\right)+19,6\left(t-{t}_{0}\right)}{t-{t}_{0}}$$
=limt→t0(−4,9t−4,9t0+19,6)=−9,8t0+19,6$$ =\underset{t\to {t}_{0}}{\mathrm{lim}}\left(-4,9t-4,9{t}_{0}+19,6\right)=-9,8{t}_{0}+19,6$$.
Vậy hàm số h = 19,6t – 4,9t2 có đạo hàm là hàm số h' = –9,8t + 19,6.
Khi vật chạm đất thì h = 0, tức là 19,6t – 4,9t2 = 0, tức là t = 0 hoặc t = 4.
Khi t = 4, vận tốc của vật khi nó chạm đất là v(4) = h'(4) = –9,8 ∙ 4 + 19,6 = –19,6 (m/s).

========================================================================

https://khoahoc.vietjack.com/thi-online/7881-cau-trac-nghiem-tong-hop-mon-toan-2023-cuc-hay-co-dap-an/126358


\textbf{{QUESTION}}

Giải phương trình: $\left( {2\sqrt x + 1} \right)\left( {\sqrt x - 2} \right) = 7$.

\textbf{{ANSWER}}

Điều kiện: x ≥ 0
$\left( {2\sqrt x + 1} \right)\left( {\sqrt x - 2} \right) = 7$
$ \Leftrightarrow 2x - 3\sqrt x - 2 = 7$
$ \Leftrightarrow 2x - 3\sqrt x - 9 = 0$
$ \Leftrightarrow \left( {\sqrt x - 3} \right)\left( {2\sqrt x + 3} \right) = 0$
$ \Leftrightarrow \left[ \begin{array}{l}\sqrt x = 3\\2\sqrt x = - 3\end{array} \right. \Leftrightarrow \left[ \begin{array}{l}x = 9\\x \in \emptyset \end{array} \right.$
⇔ x = 9 (thỏa mãn điều kiện)
Vậy phương trình có nghiệm duy nhất x = 9

========================================================================

https://khoahoc.vietjack.com/thi-online/7881-cau-trac-nghiem-tong-hop-mon-toan-2023-cuc-hay-co-dap-an/126358


\textbf{{QUESTION}}

Phân tích đa thức thành nhân tử: $x - 2\sqrt {x - 1} $.

\textbf{{ANSWER}}

$x - 2\sqrt {x - 1} $$ = \left( {x - 1} \right) - 2\sqrt {x - 1} + 1$
$ = {\left( {\sqrt {x - 1} } \right)^2} - 2\sqrt {x - 1} + 1$
$ = {\left( {\sqrt {x - 1} - 1} \right)^2}$

========================================================================

https://khoahoc.vietjack.com/thi-online/35-de-minh-hoa-thpt-quoc-gia-mon-toan-nam-2022-co-loi-giai-x/75300


\textbf{{QUESTION}}

A. $$ {15}^{3}.$$
B. $$ {3}^{15}.$$
C. $$ {A}_{15}^{3}.$$
D. $$ {C}_{15}^{3}$$

\textbf{{ANSWER}}

Số cách chọn ba học sinh từ một nhóm gồm 15 học sinh là $$ {C}_{15}^{3}.$$
Chọn đáp án D.

========================================================================

https://khoahoc.vietjack.com/thi-online/35-de-minh-hoa-thpt-quoc-gia-mon-toan-nam-2022-co-loi-giai-x/75300


\textbf{{QUESTION}}

Cho cấp số cộng $$ \left({u}_{n}\right)$$ biết $$ {u}_{1}=\mathrm{3,}{u}_{2}=-1.$$ Tìm $$ {u}_{3}.$$ 
A. $$ {u}_{3}=4.$$
B. $$ {u}_{3}=2.$$
C. $$ {u}_{3}=-5.$$
D. $$ {u}_{3}=7.$$

\textbf{{ANSWER}}

Công thức tổng quát của cấp số cộng có số hạng đầu là $$ {u}_{1}$$ và công sai $$ d$$ là $$ {u}_{n}={u}_{1}+\left(n-1\right)d.$$
Vậy ta có $$ d={u}_{2}-{u}_{1}=-1-3=-4\Rightarrow {u}_{3}={u}_{2}+d=-1+\left(-4\right)=-5$$
Chọn đáp án C.

========================================================================

https://khoahoc.vietjack.com/thi-online/bai-tap-quan-he-giua-ba-canh-cua-mot-tam-giac-bat-dang-thuc-tam-giac-co-dap-an


\textbf{{QUESTION}}

Cho tam giác ABC, chọn đáp án sai trong các đáp án sau:
A. $$ AB+BC>AC$$
B. $$ BC-AB<AC$$
C. $$ BC-AB<AC<BC+AB$$
D. $$ AB-AC>BC$$

\textbf{{ANSWER}}

Vì trong một tam giác tổng độ dài hai cạnh bất kì lớn hơn độ dài cạnh còn lại và hiệu độ dài hai cạnh bất kì nhỏ hơn độ dài cạnh còn lại nên các đáp án A, B, C đúng và D sai.
Chọn đáp án D.

========================================================================

https://khoahoc.vietjack.com/thi-online/bai-tap-quan-he-giua-ba-canh-cua-mot-tam-giac-bat-dang-thuc-tam-giac-co-dap-an


\textbf{{QUESTION}}

Dựa vào bất đẳng thức tam giác, kiểm tra xem bộ ba nào trong các bộ ba đoạn thẳng có độ dài sau đây không thể là ba cạnh của một tam giác:
A. 3cm,5cm,7cm$$ 3cm\mathrm{,5}cm\mathrm{,7}cm$$
B. 4cm,5cm,6cm$$ 4cm\mathrm{,5}cm\mathrm{,6}cm$$
C. 2cm,5cm,7cm$$ 2cm\mathrm{,5}cm\mathrm{,7}cm$$
D. 3cm,6cm,5cm$$ 3cm\mathrm{,6}cm\mathrm{,5}cm$$

\textbf{{ANSWER}}

• Xét bộ ba: 3cm,5cm,7cm$$ 3cm\mathrm{,5}cm\mathrm{,7}cm$$. Ta có: {3+5=8>73+7=10>55+7=12>3$$ \left\{\begin{array}{l}3+5=8>7\\ 3+7=10>5\\ 5+7=12>3\end{array}\right.$$ (thỏa mãn bất đẳng thức tam giác) nên bộ ba 3cm,5cm,7cm$$ 3cm\mathrm{,5}cm\mathrm{,7}cm$$ lập thành một tam giác nên loại A.
• Xét bộ ba 4cm,5cm,6cm$$ 4cm\mathrm{,5}cm\mathrm{,6}cm$$. Ta có: {4+5=9>64+6=10>55+6=11>4$$ \left\{\begin{array}{l}4+5=9>6\\ 4+6=10>5\\ 5+6=11>4\end{array}\right.$$ (thỏa mãn bất đẳng thức tam giác) nên bộ ba 4cm,5cm,6cm$$ 4cm\mathrm{,5}cm\mathrm{,6}cm$$ lập thành một tam giác nên loại B.
• Xét bộ ba 3cm,6cm,5cm$$ 3cm\mathrm{,6}cm\mathrm{,5}cm$$. Ta có: {3+6=9>53+5=8>66+5=11>3$$ \left\{\begin{array}{l}3+6=9>5\\ 3+5=8>6\\ 6+5=11>3\end{array}\right.$$ (thỏa mãn bất đẳng thức tam giác) nên bộ ba 3cm,6cm,5cm$$ 3cm\mathrm{,6}cm\mathrm{,5}cm$$ lập thành một tam giác nên loại D.
• Xét bộ ba 2cm,5cm,7cm$$ 2cm\mathrm{,5}cm\mathrm{,7}cm$$. Ta có: 2+5=7$$ 2+5=7$$ (không thỏa mãn bất đẳng thức tam giác) nên bộ ba 2cm,5cm,7cm$$ 2cm\mathrm{,5}cm\mathrm{,7}cm$$ không lập thành một tam giác nên chọn C.
Chọn đáp án C.

========================================================================

https://khoahoc.vietjack.com/thi-online/bai-tap-quan-he-giua-ba-canh-cua-mot-tam-giac-bat-dang-thuc-tam-giac-co-dap-an


\textbf{{QUESTION}}

Cho tam giác ABC có cạnh AB = 1cm và BC = 4cm. Tính độ dài cạnh AC biết độ dài cạnh AC là một số nguyên:
A. 1cm
B. 2cm
C. 3cm
D. 4cm

\textbf{{ANSWER}}

Gọi độ dài cạnh AC là x (x>0). Theo bất đẳng thức tam giác ta có:
 4−1<x<4+1⇔3<x<5$$ 4-1<x<4+1\Leftrightarrow 3<x<5$$ Vì x là số nguyên nên x = 4. Vậy độ dài cạnh AC = 4cm 
Chọn đáp án D.

========================================================================

https://khoahoc.vietjack.com/thi-online/bai-tap-quan-he-giua-ba-canh-cua-mot-tam-giac-bat-dang-thuc-tam-giac-co-dap-an


\textbf{{QUESTION}}

Cho tam giác ABC biết AB = 1cm, BC = 9cm và cạnh AC là một số nguyên. Chu vi tam giác ABC là:
A. 17cm
B. 18cm
C. 19cm
D. 16cm

\textbf{{ANSWER}}

Gọi độ dài cạnh AC là x (x >0). Theo bất đẳng thức tam giác ta có:
 9−1<x<9+1⇔8<x<10$$ 9-1<x<9+1\Leftrightarrow 8<x<10$$ Vì x là số nguyên nên x = 9. Vậy độ dài cạnh AC = 9cm 
Chu vi tam giác là: AB+BC+AC=1+9+9=19cm$$ AB+BC+AC=1+9+9=19cm$$
Chọn đáp án C.

========================================================================

https://khoahoc.vietjack.com/thi-online/bai-tap-quan-he-giua-ba-canh-cua-mot-tam-giac-bat-dang-thuc-tam-giac-co-dap-an


\textbf{{QUESTION}}

Cho tam giác ABC có BC = 1cm, AC = 8cm và độ dài cạnh AB là một số nguyên (cm). Tam giác ABC là tam giác gì?
A. Tam giác vuông tại A
B. Tam giác cân tại A
C. Tam giác vuông cân tại A
D. Tam giác cân tại B

\textbf{{ANSWER}}

Gọi độ dài cạnh AB là x (x>0). Theo bất đẳng thức tam giác ta có:
 8−1<x<8+1⇔7<x<9$$ 8-1<x<8+1\Leftrightarrow 7<x<9$$ Vì x là số nguyên nên x = 8. Vậy độ dài cạnh AB = 8cm 
Tam giác ABC có AB = AC = 8cm nên tam giác ABC cân tại A.
Chọn đáp án B.

========================================================================

https://khoahoc.vietjack.com/thi-online/20-cau-trac-nghiem-ham-so-mu-ham-so-logarit-co-dap-an-van-dung


\textbf{{QUESTION}}

Cho hàm số $$ y={e}^{-x}.\mathrm{sin}x$$. Mệnh đề nào sau đây đúng?
A. $$ y\text{'}+2y\text{'}\text{'}-2y=0$$
B. $$ y\text{'}\text{'}+2y\text{'}+2y=0$$
C. $$ y\text{'}\text{'}-2y\text{'}-2y=0$$
D. $$ y\text{'}-2y\text{'}\text{'}+2y=0$$

\textbf{{ANSWER}}

Ta có:
$$ y\text{'}=-{e}^{-x}.\mathrm{sin}x+{e}^{-x}.\mathrm{cos}x\phantom{\rule{0ex}{0ex}}={e}^{-x}\left(\mathrm{cos}x-\mathrm{sin}x\right)$$
Lại có:
$$ y\text{'}\text{'}=-{e}^{-x}\left(\mathrm{cos}x-\mathrm{sin}x\right)+{e}^{-x}\phantom{\rule{0ex}{0ex}}.\left(-\mathrm{sin}x-\mathrm{cos}x\right)=-2{e}^{-x}.\mathrm{cos}x$$
Ta thấy
$$ y\text{'}\text{'}+2y\text{'}+2y=-2{e}^{-x}.\mathrm{cos}x+2{e}^{-x}\phantom{\rule{0ex}{0ex}}.(\mathrm{cos}x-\mathrm{sin}x)+2{e}^{-x}.\mathrm{sin}x=0$$
Đáp án cần chọn là: B.

========================================================================

https://khoahoc.vietjack.com/thi-online/20-cau-trac-nghiem-ham-so-mu-ham-so-logarit-co-dap-an-van-dung


\textbf{{QUESTION}}

Cho giới hạn I=limx→∞e3x-e2xx$$ I=\underset{x\to \infty }{\mathrm{lim}}\frac{{e}^{3x}-{e}^{2x}}{x}$$, chọn mệnh đề đúng:
A. I2+3I=2$$ {I}^{2}+3I=2$$
B. I3+I2-2=0$$ {I}^{3}+{I}^{2}-2=0$$
C. I-1I+1=1$$ \frac{I-1}{I+1}=1$$
D. 3I-2=2I2$$ 3I-2=2{I}^{2}$$

\textbf{{ANSWER}}

Ta có:
$$ I=\underset{x\to \infty }{\mathrm{lim}}\frac{{e}^{3x}-{e}^{2x}}{x}\phantom{\rule{0ex}{0ex}}=\underset{x\to \infty }{\mathrm{lim}}\frac{\left({e}^{3x}-1\right)-\left({e}^{2x}-1\right)}{x}\phantom{\rule{0ex}{0ex}}=\underset{x\to \infty }{\mathrm{lim}}\left[3.\frac{{e}^{3x}-1}{3x}-2.\frac{{e}^{2x}-1}{2x}\right]\phantom{\rule{0ex}{0ex}}=3.1-2.1=1$$
Do đó, thay I = 1 vào các đáp án ta được đáp án B.
Đáp án cần chọn là: B.

========================================================================

https://khoahoc.vietjack.com/thi-online/20-cau-trac-nghiem-ham-so-mu-ham-so-logarit-co-dap-an-van-dung


\textbf{{QUESTION}}

Cho a, b là hai số thực thỏa mãn $$ {a}^{\frac{\sqrt{3}}{3}}>{a}^{\frac{\sqrt{2}}{2}}$$ và $$ {\mathrm{log}}_{b}\frac{3}{4}<{\mathrm{log}}_{b}\frac{4}{5}$$. Mệnh đề nào sau đây là đúng?
A. 0<a<1, 0<b<1
B. 0<a<1<b
C. 0<b<1<a
D. a>1, b>1

\textbf{{ANSWER}}

Ta có: $$ \frac{\sqrt{3}}{3}<\frac{\sqrt{2}}{2}$$ mà $$ {a}^{\frac{\sqrt{3}}{3}}>{a}^{\frac{\sqrt{2}}{2}}$$
Suy ra hàm đặc trưng $$ y={a}^{x}$$ nghịch biến nên 0<a<1
Vì $$ \frac{3}{4}<\frac{4}{5}$$ và $$ {\mathrm{log}}_{b}\frac{3}{4}<{\mathrm{log}}_{b}\frac{4}{5}$$ nên b > 1.
Vậy 0<a<1 và b > 1 hay 0<a<1<b
Đáp án cần chọn là: B.

========================================================================

https://khoahoc.vietjack.com/thi-online/20-cau-trac-nghiem-ham-so-mu-ham-so-logarit-co-dap-an-van-dung


\textbf{{QUESTION}}

Cho hàm số f(x)=2x.7x2$$ f\left(x\right)={2}^{x}.{7}^{{x}^{2}}$$. Khẳng định nào sau đây sai?
A. f(x)<1⇔x+x2log27<0$$ f\left(x\right)<1\Leftrightarrow x+{x}^{2}{\mathrm{log}}_{2}7<0$$
B. f(x)<1⇔xln2+x2ln7<0$$ f\left(x\right)<1\Leftrightarrow x\mathrm{ln}2+{x}^{2}\mathrm{ln}7<0$$
C. f(x)<1⇔xlog72+x2<0$$ f\left(x\right)<1\Leftrightarrow x{\mathrm{log}}_{7}2+{x}^{2}<0$$
D. f(x)<1⇔1+xlog27<0$$ f\left(x\right)<1\Leftrightarrow 1+x{\mathrm{log}}_{2}7<0$$

\textbf{{ANSWER}}

f(x)<1⇔2x.7x2<1⇔7x2<2-x⇔x2.ln7<-xln2⇔xln2+x2ln7<0⇔x+x2log27<0⇔xlog72+x2<0$$ f\left(x\right)<1\Leftrightarrow {2}^{x}.{7}^{{x}^{2}}<1\phantom{\rule{0ex}{0ex}}\Leftrightarrow {7}^{{x}^{2}}<{2}^{-x}\Leftrightarrow {x}^{2}.\mathrm{ln}7<-x\mathrm{ln}2\phantom{\rule{0ex}{0ex}}\Leftrightarrow x\mathrm{ln}2+{x}^{2}\mathrm{ln}7<0\phantom{\rule{0ex}{0ex}}\Leftrightarrow x+{x}^{2}{\mathrm{log}}_{2}7<0\phantom{\rule{0ex}{0ex}}\Leftrightarrow x{\mathrm{log}}_{7}2+{x}^{2}<0$$
Đối chiếu các đáp án thấy câu D sai.
Đáp án cần chọn là: D.

========================================================================

https://khoahoc.vietjack.com/thi-online/20-cau-trac-nghiem-ham-so-mu-ham-so-logarit-co-dap-an-van-dung


\textbf{{QUESTION}}

Cho các số thực dương a, b khác 1. Biết rằng đường thẳng y = 2 cắt đồ thị các hàm số y=ax; y=bx$$ y={a}^{x};\quad y={b}^{x}$$ và trục tung lần lượt tại A, B, C nằm giữa A và B, và AC = 2BC. Khẳng định nào dưới đây đúng?
A. b=a2$$ b=\frac{a}{2}$$
B. b=2a
C. b=a-2$$ b={a}^{-2}$$
D. b=a2$$ b={a}^{2}$$

\textbf{{ANSWER}}

Ta có: C(0;2)
ax=2⇒x=loga2⇒A(loga2;2)bx=2⇒x=logb2⇒B(logb2;2)$$ {a}^{x}=2\Rightarrow x={\mathrm{log}}_{a}2\Rightarrow A\left({\mathrm{log}}_{a}2;2\right)\phantom{\rule{0ex}{0ex}}{b}^{x}=2\Rightarrow x={\mathrm{log}}_{b}2\Rightarrow B\left({\mathrm{log}}_{b}2;2\right)$$
Vì C nằm giữa A và B và:
AC=2BC⇔→AC=-2→BC⇔{-loga2=2.logb20=0⇔-1log2a=2.1log2b⇔log2b=-2log2a⇔log2b=log2a-2⇔b=a-2$$ AC=2BC\Leftrightarrow \overrightarrow{AC}=-2\overrightarrow{BC}\phantom{\rule{0ex}{0ex}}\Leftrightarrow \left\{\begin{array}{l}-{\mathrm{log}}_{a}2=2.{\mathrm{log}}_{b}2\\ 0=0\end{array}\right.\phantom{\rule{0ex}{0ex}}\Leftrightarrow -\frac{1}{{\mathrm{log}}_{2}a}=2.\frac{1}{{\mathrm{log}}_{2}b}\phantom{\rule{0ex}{0ex}}\Leftrightarrow {\mathrm{log}}_{2}b=-2{\mathrm{log}}_{2}a\phantom{\rule{0ex}{0ex}}\Leftrightarrow {\mathrm{log}}_{2}b={\mathrm{log}}_{2}{a}^{-2}\Leftrightarrow b={a}^{-2}$$
Đáp án cần chọn là: C.

========================================================================

https://khoahoc.vietjack.com/thi-online/12-bai-tap-tim-gia-tri-lon-nhat-gia-tri-nho-nhat-cua-bieu-thuc-chua-can-bac-hai-co-loi-giai


\textbf{{QUESTION}}

Giá trị nhỏ nhất của biểu thức $A = \frac{{x + 2}}{{\sqrt x }}$ (x > 0) là 
A. $2\sqrt 2 $.
B. 4.
C. 2.
D. $\sqrt 2 $.

\textbf{{ANSWER}}

Đáp án đúng là: A
Ta có: $A = \frac{{x + 2}}{{\sqrt x }} = \sqrt x  + \frac{2}{{\sqrt x }}$.
Với x > 0, áp dụng bất đẳng thức Cauchy, ta có:
$\sqrt x  + \frac{2}{{\sqrt x }} \ge 2\sqrt {\sqrt x \frac{2}{{\sqrt x }}}  = 2\sqrt 2 $.
Dấu “=” xảy ra khi $\sqrt x  = \frac{2}{{\sqrt x }}$ hay x = 2.
Vậy GTNN của A = $2\sqrt 2 $ khi x = 2.

========================================================================

https://khoahoc.vietjack.com/thi-online/12-bai-tap-tim-gia-tri-lon-nhat-gia-tri-nho-nhat-cua-bieu-thuc-chua-can-bac-hai-co-loi-giai


\textbf{{QUESTION}}

Giá trị lớn nhất của biểu thức B=2√x+9√x+2$B = \frac{{2\sqrt x  + 9}}{{\sqrt x  + 2}}$ (x ≥ 0) là
A. 92$\frac{9}{2}$.
B. −92$ - \frac{9}{2}$.
C. 0.
D. 1.

\textbf{{ANSWER}}

Đáp án đúng là: A
Ta có: $B = \frac{{2\sqrt x  + 9}}{{\sqrt x  + 2}} = 2 + \frac{5}{{\sqrt x  + 2}}$.
Với x ≥ 0, ta có: $\sqrt x  + 2 \ge 2$ suy ra $\frac{5}{{\sqrt x  + 2}} \le \frac{5}{2}$.
Do đó, 2 + $\frac{5}{{\sqrt x  + 2}} \le \frac{5}{2} + 2$ hay B ≤ $\frac{9}{2}$.
Dấu “=” xảy ra khi x = 0.
Vật GTLN của B = $\frac{9}{2}$ khi x = 0.

========================================================================

https://khoahoc.vietjack.com/thi-online/12-bai-tap-tim-gia-tri-lon-nhat-gia-tri-nho-nhat-cua-bieu-thuc-chua-can-bac-hai-co-loi-giai


\textbf{{QUESTION}}

Biểu thức C=2√x+113√x+2$C = \frac{{2\sqrt x  + 11}}{{3\sqrt x  + 2}}$ đạt giá trị lớn nhất tại x bằng:
A. 112$\frac{{11}}{2}$.
B. −112$ - \frac{{11}}{2}$.
C. 0.
D. 1.

\textbf{{ANSWER}}

Đáp án đúng là: C
Điều kiện: x ≥ 0.
Ta có: C=2√x+113√x+2=2(√x+23)+2933(√x+23)=23+293(3√x+2)$C = \frac{{2\sqrt x  + 11}}{{3\sqrt x  + 2}} = \frac{{2\left( {\sqrt x  + \frac{2}{3}} \right) + \frac{{29}}{3}}}{{3\left( {\sqrt x  + \frac{2}{3}} \right)}} = \frac{2}{3} + \frac{{29}}{{3\left( {3\sqrt x  + 2} \right)}}$.
Với x ≥ 0, ta có: 3√x+2≥2$3\sqrt x  + 2 \ge 2$ suy ra 3(3√x+2)≥6$3\left( {3\sqrt x  + 2} \right) \ge 6$.
Do đó, 293(3√x+2)≤296$\frac{{29}}{{3\left( {3\sqrt x  + 2} \right)}} \le \frac{{29}}{6}$ .
Suy ra 23+293(3√x+2)≤112$\frac{2}{3} + \frac{{29}}{{3\left( {3\sqrt x  + 2} \right)}} \le \frac{{11}}{2}$ hay C ≤ 112$\frac{{11}}{2}$.
Dấu “=” xảy ra khi x = 0.
Vậy GTLN của C = 112$\frac{{11}}{2}$ khi x = 0.

========================================================================

https://khoahoc.vietjack.com/thi-online/12-bai-tap-tim-gia-tri-lon-nhat-gia-tri-nho-nhat-cua-bieu-thuc-chua-can-bac-hai-co-loi-giai


\textbf{{QUESTION}}

Biểu thức D=x−√x+1x$D = \frac{{x - \sqrt x  + 1}}{x}$ đạt giá trị nhỏ nhất tại x bằng:
A. 34$\frac{3}{4}$.
B. 4.
C. 14.$\frac{1}{4}.$
D. 2.

\textbf{{ANSWER}}

Đáp án đúng là: A
Điều kiện xác định: x > 0.
Ta có: D=x−√x+1x=1−1√x+1x=34+14−2.121√x+1x=34+(12−1√x)2$D = \frac{{x - \sqrt x  + 1}}{x} = 1 - \frac{1}{{\sqrt x }} + \frac{1}{x} = \frac{3}{4} + \frac{1}{4} - 2.\frac{1}{2}\frac{1}{{\sqrt x }} + \frac{1}{x} = \frac{3}{4} + {\left( {\frac{1}{2} - \frac{1}{{\sqrt x }}} \right)^2}$.
Nhận thấy (12−1√x)2≥0${\left( {\frac{1}{2} - \frac{1}{{\sqrt x }}} \right)^2} \ge 0$ nên 34+(12−1√x)2≥34$\frac{3}{4} + {\left( {\frac{1}{2} - \frac{1}{{\sqrt x }}} \right)^2} \ge \frac{3}{4}$ hay D ≥34$ \ge \frac{3}{4}$.
Dấu “=” xảy ra khi 12−1√x=0$\frac{1}{2} - \frac{1}{{\sqrt x }} = 0$ suy ra  √x=2$\sqrt x  = 2$ khi x = 4.
Vậy GTNN của của D = 34$\frac{3}{4}$ khi x = 4.

========================================================================

https://khoahoc.vietjack.com/thi-online/12-bai-tap-tim-gia-tri-lon-nhat-gia-tri-nho-nhat-cua-bieu-thuc-chua-can-bac-hai-co-loi-giai


\textbf{{QUESTION}}

Cho biểu thức $D = \left( {\frac{{\sqrt x  - 2}}{{x - 1}} - \frac{{\sqrt x  + 2}}{{x + 2\sqrt x  + 1}}} \right).\frac{{{{\left( {1 - x} \right)}^2}}}{2}$ với x ≥ 0, x ≠ 1. Giá trị lớn nhất của D là:
A. $\frac{1}{4}$.
B. $ - \frac{1}{4}$.
C. $\frac{1}{2}$.
D. 1.

\textbf{{ANSWER}}

Đáp án đúng là: A
Với x ≥ 0, x ≠ 1, ta có:
$D = \left( {\frac{{\sqrt x  - 2}}{{x - 1}} - \frac{{\sqrt x  + 2}}{{x + 2\sqrt x  + 1}}} \right).\frac{{{{\left( {1 - x} \right)}^2}}}{2}$
$D = \left[ {\frac{{\left( {\sqrt x  - 2} \right){{\left( {\sqrt x  + 1} \right)}^2}}}{{\left( {x - 1} \right){{\left( {\sqrt x  + 1} \right)}^2}}} - \frac{{\left( {\sqrt x  + 2} \right)\left( {x - 1} \right)}}{{\left( {x - 1} \right){{\left( {\sqrt x  + 1} \right)}^2}}}} \right].\frac{{{{\left( {1 - x} \right)}^2}}}{2}$
$D = \left[ {\frac{{\left( {\sqrt x  - 2} \right){{\left( {\sqrt x  + 1} \right)}^2} - \left( {\sqrt x  + 2} \right)\left( {x - 1} \right)}}{{\left( {x - 1} \right){{\left( {\sqrt x  + 1} \right)}^2}}}} \right].\frac{{{{\left( {1 - x} \right)}^2}}}{2}$
$D = \frac{{\left[ {x\sqrt x  + 2x + \sqrt x  - 2x - 4\sqrt x  - 2 - x\sqrt x  + \sqrt x  - 2x + 2} \right]}}{{{{\left( {\sqrt x  + 1} \right)}^2}}}.\frac{{\left( {x - 1} \right)}}{2}$
$D = \frac{{\left( { - 2x - 2\sqrt x } \right)}}{{{{\left( {\sqrt x  + 1} \right)}^2}}}.\frac{{\left( {x - 1} \right)}}{2} = \frac{{ - 2\sqrt x \left( {\sqrt x  + 1} \right)}}{{{{\left( {\sqrt x  + 1} \right)}^2}}}.\frac{{\left( {\sqrt x  - 1} \right)\left( {\sqrt x  + 1} \right)}}{2} =  - \sqrt x \left( {\sqrt x  - 1} \right)$.
Ta có: $D =  - \sqrt x \left( {\sqrt x  - 1} \right) =  - x + \sqrt x  =  - x + 2.\frac{1}{2}\sqrt x  - \frac{1}{4} + \frac{1}{4} =  - {\left( {\sqrt x  - \frac{1}{2}} \right)^2} + \frac{1}{4}$.
Nhận thấy $ - {\left( {\sqrt x  - \frac{1}{2}} \right)^2} \le 0$ nên $ - {\left( {\sqrt x  - \frac{1}{2}} \right)^2} + \frac{1}{4} \le \frac{1}{4}$.
Dấu “=” xảy ra khi x = $\frac{1}{4}$.
Vậy GTLN của D = $\frac{1}{4}$ khi x = $\frac{1}{4}$.

========================================================================

https://khoahoc.vietjack.com/thi-online/bai-tap-theo-tuan-toan-8-tuan-29/98298


\textbf{{QUESTION}}

Cho $$ AB=2cm\text{\hspace{0.17em}};\text{\hspace{0.17em}}CD=4\text{\hspace{0.17em}}cm$$  . Khi đó $$ \frac{AB}{CD}=?$$
A. 5dm
B. 5
C. 1/2
D. 2/4 dm

\textbf{{ANSWER}}

Đáp án C

========================================================================

https://khoahoc.vietjack.com/thi-online/bai-tap-de-on-tap-chuyen-de-4


\textbf{{QUESTION}}

Cho hình chóp đều S.ABCD. Xét các khẳng định sau:
i) Đáy ABCD là hình vuông;
ii) SA = SB = SC = SD;
iii) Hình chóp có bốn mặt;
iv) SO vuông góc vói mặt đáy, với O là trung điểm AB.
Số khẳng định sai là:
A. 1;                     
B. 2;                      
C. 3;                     
D. 4

\textbf{{ANSWER}}

Đáp án B

========================================================================

https://khoahoc.vietjack.com/thi-online/bai-tap-de-on-tap-chuyen-de-4


\textbf{{QUESTION}}

Cho hình chóp tam giác đều có tất cả các cạnh bằng 4 cm. Độ dài trung đoạn hình chóp là
A. 2 cm;               
B. √32$$ \frac{\sqrt{3}}{2}$$cm
C. 2√3$$ 2\sqrt{3}$$cm
D. 12 cm

\textbf{{ANSWER}}

Đáp án C

========================================================================

https://khoahoc.vietjack.com/thi-online/bai-tap-de-on-tap-chuyen-de-4


\textbf{{QUESTION}}

Cho hình lập phương ABCD.A'B'C'D'. Mặt phẳng chứa cả cạnh AC và cạnh AC' là:
A. (A'C'CA);         
B. (ABB'A');                   
C. (CDD'C);                   
D. (BCC'B').

\textbf{{ANSWER}}

Đáp án A

========================================================================

https://khoahoc.vietjack.com/thi-online/bai-tap-de-on-tap-chuyen-de-4


\textbf{{QUESTION}}

Chọn câu trả lời đúng nhất. Các cạnh bên của hình lăng trụ đứng là
A. Các đoạn thẳng không bằng nhau;
B. Các đoạn thẳng song song và bằng nhau;
C. Các đoạn thẳng vuông góc với hai mặt đáy;
D. Các đoạn thẳng song song, bằng nhau và vuông góc vói hai mặt đáy.

\textbf{{ANSWER}}

Đáp án D

========================================================================

https://khoahoc.vietjack.com/thi-online/bai-tap-de-on-tap-chuyen-de-4


\textbf{{QUESTION}}

Cho hình lăng trụ đứng ABC.A'B'C' có AB = 5cm, AC = 13cm, BC = 12cm và đường cao AA’ = 8cm. Diện tích toàn phần của lăng trụ là
A. 220 cm2;          
B. 180 cm2;          
C. 270 cm2;           
D. 300 cm2

\textbf{{ANSWER}}

Đáp án D

========================================================================

https://khoahoc.vietjack.com/thi-online/tong-hop-de-thi-chinh-thuc-vao-10-mon-toan-nam-2021-co-dap-an-phan-1/104736


\textbf{{QUESTION}}

a) Tính $$ A=\sqrt{4}+\sqrt{3}.\sqrt{12}$$

\textbf{{ANSWER}}

a) Ta có : $$ A=\sqrt{4}+\sqrt{3}.\sqrt{12}=2+\sqrt{36}=2+6=8$$
Vậy A = 8

========================================================================

https://khoahoc.vietjack.com/thi-online/tong-hop-de-thi-chinh-thuc-vao-10-mon-toan-nam-2021-co-dap-an-phan-1/104736


\textbf{{QUESTION}}

b) Cho biểu thức $$ B=\left(\frac{\sqrt{x}}{2+\sqrt{x}}+\frac{x+4}{4-x}\right):\frac{x}{x-2\sqrt{x}}\left(\begin{array}{l}x>0\\ x\ne 4\end{array}\right)$$
Rút gọn B và tìm tất cả các giá trị nguyên của x để $$ B<-\sqrt{x}$$

\textbf{{ANSWER}}

b) Với $$ x>\mathrm{0,}x\ne 4$$ thì
$$ \begin{array}{l}B=\left(\frac{\sqrt{x}}{2+\sqrt{x}}+\frac{x+4}{4-x}\right):\frac{x}{x-2\sqrt{x}}\\ =\frac{\sqrt{x}\left(\sqrt{x}-2\right)-x-4}{\left(\sqrt{x}+2\right)\left(\sqrt{x}-2\right)}.\frac{\sqrt{x}\left(\sqrt{x}-2\right)}{x}=\frac{x-2\sqrt{x}-x-4}{\left(\sqrt{x}+2\right).\sqrt{x}}\\ =\frac{-2\left(\sqrt{x}+2\right)}{\left(\sqrt{x}+2\right).\sqrt{x}}=\frac{-2}{\sqrt{x}}\end{array}$$
$$ \begin{array}{l}B<-\sqrt{x}\Leftrightarrow \frac{-2}{\sqrt{x}}<-\sqrt{x}\Leftrightarrow \frac{2}{\sqrt{x}}-\sqrt{x}>0\\ \Leftrightarrow \frac{2-x}{\sqrt{x}}>0\Leftrightarrow 2-x>0\Leftrightarrow x<2\end{array}$$
Kết $$ B<-\sqrt{x}$$hợp với điều kiện $$ \Rightarrow 0<x<2$$thì $$ B<-\sqrt{x}$$

========================================================================

https://khoahoc.vietjack.com/thi-online/giai-vth-toan-6-kntt-luyen-tap-chung-trang-37-tap-2-co-dap-an


\textbf{{QUESTION}}

Tính một cách hợp lí:
a) 5,3 – (-5,1) + (-5,3) + 4,9
b) (2,7 – 51, 4) – (48,6 – 7,3)
c) 2,5. (-0,124) + 10,124. 2,5

\textbf{{ANSWER}}

Lời giải:
a) 5,3 – (-5,1) + (-5,3) + 4,9
= 5,3 + 5,1 – 5,3 + 4,9
= (5,3 – 5,3) + (5,1 + 4,9)
= 0 + 10 = 10
b) (2,7 – 51, 4) – (48,6 – 7,3)
= 2,7 – 51,4 – 48,6 + 7, 3
= (2,7 + 7,3) – (51,4 + 48, 6)
= 10 – 100 = -(100 – 10) = -90
c) 2,5. (-0,124) + 10,124. 2,5
= 2,5.(-0,124 + 10,124) = 2,5.10 = 25

========================================================================

https://khoahoc.vietjack.com/thi-online/giai-vth-toan-6-kntt-luyen-tap-chung-trang-37-tap-2-co-dap-an


\textbf{{QUESTION}}

Tính giá trị của biểu thức sau: 7,05 – (a + 3,5 +0,85) khi a = -7,2.

\textbf{{ANSWER}}

Lời giải:
Trong biểu thức đã cho, thay a bởi -7,2 ta được:
7,05 – (-7,2 + 3,5 + 0,85) = 7,05 – (-7,2 + 4,35) = 7,05 + 7,2 – 4,35
= 14,25 – 4,35 = 9,9.

========================================================================

https://khoahoc.vietjack.com/thi-online/5-cau-trac-nghiem-toan-7-bai-7-da-thuc-mot-bien-co-dap-an-van-dung


\textbf{{QUESTION}}

Cho $$ P\left(x\right)=100{x}^{100}+99{x}^{99}+98{x}^{98}+...+2{x}^{2}+x$$. Tính $$ P(-1)$$.
A. $$ P(-1)=-50$$.
B. $$ P(-1)=100$$.
C. $$ P(-1)=50$$.
D. $$ P(-1)=5050$$.

\textbf{{ANSWER}}

Đáp án cần chọn là đáp án C.
Thay x = -1 vào $$ P\left(x\right)=100{x}^{100}+99{x}^{99}+98{x}^{98}+...+2{x}^{2}+x$$ ta được:
$$ P\left(x\right)=100{\left(-1\right)}^{100}+99{\left(-1\right)}^{99}+98{\left(-1\right)}^{98}+...+2{\left(-1\right)}^{2}+\left(-1\right)\phantom{\rule{0ex}{0ex}}=100-99+98-97+...+2-1\phantom{\rule{0ex}{0ex}}=1+1+...+1\phantom{\rule{0ex}{0ex}}=50.1=50$$
Vậy $$ P(-1)=50$$.

========================================================================

https://khoahoc.vietjack.com/thi-online/5-cau-trac-nghiem-toan-7-bai-7-da-thuc-mot-bien-co-dap-an-van-dung


\textbf{{QUESTION}}

Cho $$ f\left(x\right)={x}^{99}-101{x}^{98}+101{x}^{97}-101{x}^{96}+...+101x-1.$$ Tính f(100).
A. f(100) = -1.
B. f(100) = 99.
C. f(100) = -99.
D. f(100) = 100.

\textbf{{ANSWER}}

Đáp án cần chọn là B.
Ta có:
$$ f\left(x\right)={x}^{99}-101{x}^{98}+101{x}^{97}-101{x}^{96}+...+101x-1\phantom{\rule{0ex}{0ex}}={x}^{99}-\left(100+1\right){x}^{98}+\left(100+1\right){x}^{97}-\left(100+1\right){x}^{96}+...-\left(100+1\right){x}^{2}+\left(100+1\right)x-1\phantom{\rule{0ex}{0ex}}={x}^{99}-100{x}^{98}-{x}^{98}+100{x}^{97}+...-100{x}^{2}-{x}^{2}+100x+x-1\phantom{\rule{0ex}{0ex}}=\left({x}^{99}-100{x}^{98}\right)-\left({x}^{98}-100{x}^{97}\right)+...-\left({x}^{2}-100x\right)+x-1$$
Thay x = 100 vào f(x) ta được:
$$ f\left(100\right)=\left({100}^{99}-100.{100}^{98}\right)-\left({100}^{98}-100.{100}^{97}\right)+...-\left({100}^{2}-100.100\right)+100-1\phantom{\rule{0ex}{0ex}}=\left({100}^{99}-{100}^{99}\right)-\left({100}^{98}-{100}^{98}\right)+...-\left({100}^{2}-{100}^{2}\right)+99\phantom{\rule{0ex}{0ex}}=99$$
Vậy f(100) = 99.

========================================================================

https://khoahoc.vietjack.com/thi-online/5-cau-trac-nghiem-toan-7-bai-7-da-thuc-mot-bien-co-dap-an-van-dung


\textbf{{QUESTION}}

Cho f(x)=ax3+4x(x2-1)+8; g(x)=x3-4x(bx+1)+c-5$$ f\left(x\right)=a{x}^{3}+4x\left({x}^{2}-1\right)+8;\quad g\left(x\right)={x}^{3}-4x\left(bx+1\right)+c-5$$ với a, b, c là hằng số. Xác định a, b, c để f(x)=g(x).
A. a = -3; b = 0; c = 13.
B. a = -3; b = 0; c =8.
C. a = -3; b = 0; c =13.
D. a = -3; b = 1; c = 13.

\textbf{{ANSWER}}

Đáp án cần chọn là A.
Ta có:
f(x)=ax3+4x(x2-1)+8=ax3+4x.x2-4x+8=ax3+4x3-4x+8=(a+4)x3-4x+8g(x)=x3-4x(bx+1)+c-5=x3-4x.bx+4x+x-5=x3-4bx2+4x+c-5$$ f\left(x\right)=a{x}^{3}+4x\left({x}^{2}-1\right)+8\phantom{\rule{0ex}{0ex}}=a{x}^{3}+4x.{x}^{2}-4x+8=a{x}^{3}+4{x}^{3}-4x+8\phantom{\rule{0ex}{0ex}}=\left(a+4\right){x}^{3}-4x+8\phantom{\rule{0ex}{0ex}}g\left(x\right)={x}^{3}-4x\left(bx+1\right)+c-5\phantom{\rule{0ex}{0ex}}={x}^{3}-4x.bx+4x+x-5={x}^{3}-4b{x}^{2}+4x+c-5$$
Thay x = 0 vào f(x) = g(x) ta được
f(0)=g(0)⇒(a+4).03-4.0+8=03-4b.02+4.0+c-5⇒8=c-5⇒c=13$$ f\left(0\right)=g\left(0\right)\phantom{\rule{0ex}{0ex}}\Rightarrow \left(a+4\right).{0}^{3}-4.0+8={0}^{3}-4b.{0}^{2}+4.0+c-5\phantom{\rule{0ex}{0ex}}\Rightarrow 8=c-5\Rightarrow c=13$$
Khi đó x = 1 vào f(x) = g(x) ta được
f(1)=g(1)⇒(a+4).13-4.1+8=13-4b.12+4.1+c-5⇒a+4-4+8=1-4b-4+8⇒-a+8=5-4b⇒a=-3-4b (1)$$ f\left(1\right)=g\left(1\right)\phantom{\rule{0ex}{0ex}}\Rightarrow \left(a+4\right).{1}^{3}-4.1+8={1}^{3}-4b.{1}^{2}+4.1+c-5\phantom{\rule{0ex}{0ex}}\Rightarrow a+4-4+8=1-4b-4+8\phantom{\rule{0ex}{0ex}}\Rightarrow -a+8=5-4b\Rightarrow a=-3-4b\quad \left(1\right)$$
Khi đó x = -1 vào f(x) = g(x) ta được
f(-1)=g(-1)⇒(a+4)(-1)3-4.(-1)+8=(-1)3-4b.(-1)2+4.1+c-5⇒-a-4-4+8=-1-4b+4+8⇒-a+8=11-4b⇒a=4b-3 (2)$$ f(-1)=g(-1)\phantom{\rule{0ex}{0ex}}\Rightarrow \left(a+4\right){\left(-1\right)}^{3}-4.\left(-1\right)+8={\left(-1\right)}^{3}-4b.{\left(-1\right)}^{2}+4.1+c-5\phantom{\rule{0ex}{0ex}}\Rightarrow -a-4-4+8=-1-4b+4+8\phantom{\rule{0ex}{0ex}}\Rightarrow -a+8=11-4b\Rightarrow a=4b-3\quad \left(2\right)$$
Từ (1) và (2) ⇒-3-4b=4b-3⇒8b=0⇒b=0$$ \Rightarrow -3-4b=4b-3\Rightarrow 8b=0\Rightarrow b=0$$
Thay b = 0 vào (1) ta được a = -3
Vậy a = -3; b = 0; c = 13.

========================================================================

https://khoahoc.vietjack.com/thi-online/5-cau-trac-nghiem-toan-7-bai-7-da-thuc-mot-bien-co-dap-an-van-dung


\textbf{{QUESTION}}

Cho f(x)=1+x3+x5+x7+...+x101$$ f\left(x\right)=1+{x}^{3}+{x}^{5}+{x}^{7}+...+{x}^{101}$$. Tính f(1); f(-1).
A. f(1) = 101; f(-1) = -100.
B. f(1) = 51; f(-1) = -49.
C. f(1) = 50; f(-1) = -49.
D. f(1) = 50; f(-1) = -50.

\textbf{{ANSWER}}

Đáp án cần chọn là B.
Thay x = 1 vào f(x) ta được:
$$ f\left(1\right)={1}^{1}+{1}^{3}+{1}^{5}+{1}^{7}+...+{1}^{101}\phantom{\rule{0ex}{0ex}}=1+1+...+1=51.1=51$$
Thay x = -1 vào f(x) ta được:
$$ f\left(-1\right)=1+{\left(-1\right)}^{3}+{\left(-1\right)}^{5}+{\left(-1\right)}^{7}+...+{\left(-1\right)}^{101}\phantom{\rule{0ex}{0ex}}=1+\left(-1\right)+\left(-1\right)+\left(-1\right)+...+\left(-1\right)=1+50.\left(-1\right)=-49$$
Vậy f(1) = 51; f(-1) = -49.

========================================================================

https://khoahoc.vietjack.com/thi-online/5-cau-trac-nghiem-toan-7-bai-7-da-thuc-mot-bien-co-dap-an-van-dung


\textbf{{QUESTION}}

Cho f(x)=1+x2+x4+x6+...+x2020$$ f\left(x\right)=1+{x}^{2}+{x}^{4}+{x}^{6}+...+{x}^{2020}$$. Tính f(1); f(-1).
A. f(1) = 1011; f(-1) = -1011.
B. f(1) = 1011; f(-1) = 1011
C. f(1) = 1011; f(-1) = 1009.
D. f(1) = 2021; f(-1) = 2021.

\textbf{{ANSWER}}

Đáp án cần chọn là B.
Thay x = 1 vào f(x) ta được:
f(1)=1+12+14+16+...+12020=1+(1+1+1+..+1)=1+1010.1=1011$$ f\left(1\right)=1+{1}^{2}+{1}^{4}+{1}^{6}+...+{1}^{2020}\phantom{\rule{0ex}{0ex}}=1+(1+1+1+..+1)=1+1010.1=1011$$
Thay x = -1 vào f(x) ta được:
f(-1)=1+(-1)2+(-1)4+(-1)6+...+(-1)2020=1+(1+1+1+...+1)=1+1.1010=1011$$ f\left(-1\right)=1+{\left(-1\right)}^{2}+{\left(-1\right)}^{4}+{\left(-1\right)}^{6}+...+{\left(-1\right)}^{2020}\phantom{\rule{0ex}{0ex}}=1+(1+1+1+...+1)=1+1.1010=1011$$
Vậy f(1) = 1011; f(-1) = 1011.

========================================================================

https://khoahoc.vietjack.com/thi-online/30-cau-trac-nghiem-toan-10-ket-noi-tri-thuc-bai-on-tap-cuoi-chuong-3-co-dap-an-phan-2


\textbf{{QUESTION}}

M là điểm trên nửa đường trong lượng giác sao cho $$ \widehat{xOM}$$ = α. Tọa độ của điểm M là:

\textbf{{ANSWER}}

Hướng dẫn giải
Đáp án đúng là: B
Định nghĩa tỉ số lượng giác của 1 góc bất kì từ 0° đến 180°: 
Với góc α cho trước, 0° ≤ α ≤ 180°. 
Gọi M(x0;y0) là điểm trên nửa đường tròn đơn vị nói trên sao cho $$ \widehat{xOM}$$ = α. Ta có: 
+ Sin của góc α là tung độ y0 của điểm M kí hiệu là sinα.
+ Côsin của góc α là hoành độ x0 của điểm M kí hiệu là cosα.

========================================================================

https://khoahoc.vietjack.com/thi-online/30-cau-trac-nghiem-toan-10-ket-noi-tri-thuc-bai-on-tap-cuoi-chuong-3-co-dap-an-phan-2


\textbf{{QUESTION}}

Cho tam giác ABC với độ dài 3 cạnh BC, AC, AB lần lượt là a, b, c. Khẳng định nào dưới đây đúng?

\textbf{{ANSWER}}

Hướng dẫn giải
Đáp án đúng là: B
Định lí côsin:
Trong tam giác ABC: a2 = b2 + c2 – 2bccosA.
Vậy đáp án đúng là B.

========================================================================

https://khoahoc.vietjack.com/thi-online/30-cau-trac-nghiem-toan-10-ket-noi-tri-thuc-bai-on-tap-cuoi-chuong-3-co-dap-an-phan-2


\textbf{{QUESTION}}

Khẳng định nào sau đây đúng ?

\textbf{{ANSWER}}

Hướng dẫn giải
Đáp án đúng là: C
Sử dụng máy tính cầm tay ta tính được: 
sin45° = √22$$ \frac{\sqrt{2}}{2}$$ ; cos45° = √22$$ \frac{\sqrt{2}}{2}$$; tan45° = 1; cot45° = 1.
Do đó A, B, D sai và C đúng.

========================================================================

https://khoahoc.vietjack.com/thi-online/30-cau-trac-nghiem-toan-10-ket-noi-tri-thuc-bai-on-tap-cuoi-chuong-3-co-dap-an-phan-2


\textbf{{QUESTION}}

Cho tam giác ABC với độ dài 3 cạnh BC, AC, AB lần lượt là a, b, c. Công thức tính diện tích nào dưới đây đúng?
A. S = 12$$ \frac{1}{2}$$ bcsinA;

\textbf{{ANSWER}}

Hướng dẫn giải
Đáp án đúng là: A
Công thức tính diện tích tam giác ABC: S = 12$$ \frac{1}{2}$$ bcsinA.

========================================================================

https://khoahoc.vietjack.com/thi-online/30-cau-trac-nghiem-toan-10-ket-noi-tri-thuc-bai-on-tap-cuoi-chuong-3-co-dap-an-phan-2


\textbf{{QUESTION}}

Cho tam giác ABC với độ dài 3 cạnh BC, AC, AB lần lượt là a, b, c. Nội dung nào thể hiện định lí sin?
A. asinA = bsinB = csinC$$ \frac{\text{a}}{\text{sinA}}\text{ = }\frac{\text{b}}{\text{sinB}}\text{ = }\frac{\text{c}}{\text{sinC}}$$
B. a2 = b2 + c2 – 2bccosA;
C. S = 12$$ \frac{1}{2}$$ bcsinA = 12$$ \frac{1}{2}$$ acsinB = 12$$ \frac{1}{2}$$ absinC;
D. b2 = a2 + c2 – 2accosB .

\textbf{{ANSWER}}

Hướng dẫn giải
Đáp án đúng là: A
Định lí sin: asinA = bsinB = csinC$$ \frac{\text{a}}{\text{sinA}}\text{ = }\frac{\text{b}}{\text{sinB}}\text{ = }\frac{\text{c}}{\text{sinC}}$$  = 2R.

========================================================================

https://khoahoc.vietjack.com/thi-online/bai-tap-mo-rong-khai-niem-ve-phan-so


\textbf{{QUESTION}}

Viết các phân số sau: 
a) Hai phần bảy;
b) Một phần tám; 
c) Âm bốn phần năm; 
d) Chín phần âm bốn

\textbf{{ANSWER}}

$$ a)\frac{2}{7}.b)\frac{1}{8}.c)\frac{-4}{5}.d)\frac{9}{-4}.$$

========================================================================

https://khoahoc.vietjack.com/thi-online/bai-tap-mo-rong-khai-niem-ve-phan-so


\textbf{{QUESTION}}

Viết các phân số sau: 
a) Bốn phần chín; 
b) Một phần hai 
c) Âm ba phần năm; 
d) Bẩy phần âm hai

\textbf{{ANSWER}}

$$ a)\frac{4}{9}.b)\frac{1}{2}.c)\frac{-3}{5}.d)\frac{7}{-2}.$$

========================================================================

https://khoahoc.vietjack.com/thi-online/bai-tap-mo-rong-khai-niem-ve-phan-so


\textbf{{QUESTION}}

Viết các phép chia sau dưới dạng phân số: 
a) 2:3; 
b) 3: (-4); 
c) - 3:8; 
d) (-l):(-3).

\textbf{{ANSWER}}

a)23.b)3−4.c)−38.d)−1−3.$$ a)\frac{2}{3}.b)\frac{3}{-4}.c)\frac{-3}{8}.d)\frac{-1}{-3}.$$

========================================================================

https://khoahoc.vietjack.com/thi-online/bai-tap-mo-rong-khai-niem-ve-phan-so


\textbf{{QUESTION}}

Viết các phép chia sau dưới dạng phân số:
 a) 7:10; 
b) l:(-5); 
c) -2:5; 
d) (-2): (-3).

\textbf{{ANSWER}}

a)710.b)1−5.c)−25.d)−2−3.$$ a)\frac{7}{10}.b)\frac{1}{-5}.c)\frac{-2}{5}.d)\frac{-2}{-3}.$$

========================================================================

https://khoahoc.vietjack.com/thi-online/bai-tap-mo-rong-khai-niem-ve-phan-so


\textbf{{QUESTION}}

Cho tập hợp A = {-2;1;3}. Viết tập hợp B các phân số có tử và mẫu khác nhau thuộc tập hợp A

\textbf{{ANSWER}}

B={−21;−23;1−2;13;3−2;31}.$$ B=\left\{\frac{-2}{1};\frac{-2}{3};\frac{1}{-2};\frac{1}{3};\frac{3}{-2};\frac{3}{1}\right\}.$$

========================================================================

https://khoahoc.vietjack.com/thi-online/22-cau-trac-nghiem-toan-8-bai-2-dinh-li-ta-let-dinh-ly-dao-va-he-qua-cua-dinh-ly-ta-let-co-dap-an


\textbf{{QUESTION}}

Viết tỉ số cặp đoạn thẳng có độ dài như sau: AB = 4dm, CD = 20 dm
A. $$ \frac{AB}{CD}=\frac{1}{4}$$
B. $$ \frac{AB}{CD}=\frac{1}{5}$$
C. $$ \frac{AB}{CD}=\frac{1}{6}$$
D. $$ \frac{AB}{CD}=\frac{1}{7}$$

\textbf{{ANSWER}}

AB = 4dm, CD = 20 dm => $$ \frac{AB}{CD}=\frac{4}{20}=\frac{1}{5}$$
Vậy $$ \frac{AB}{CD}=\frac{1}{5}$$ là tỉ số 2 đoạn thẳng (cùng đơn vị)
 Đáp án: B

========================================================================

https://khoahoc.vietjack.com/thi-online/22-cau-trac-nghiem-toan-8-bai-2-dinh-li-ta-let-dinh-ly-dao-va-he-qua-cua-dinh-ly-ta-let-co-dap-an


\textbf{{QUESTION}}

Viết tỉ số cặp đoạn thẳng có độ dài như sau: AB = 12cm, CD = 10 cm
A. ABCD=56$$ \frac{AB}{CD}=\frac{5}{6}$$
B. ABCD=65$$ \frac{AB}{CD}=\frac{6}{5}$$
C. ABCD=43$$ \frac{AB}{CD}=\frac{4}{3}$$
D. ABCD=34$$ \frac{AB}{CD}=\frac{3}{4}$$

\textbf{{ANSWER}}

AB = 12cm, CD = 10 cm => $$ \frac{AB}{CD}=\frac{12}{10}=\frac{6}{5}$$
Vậy $$ \frac{AB}{CD}=\frac{6}{5}$$ là tỉ số 2 đoạn thẳng (cùng đơn vị)
 Đáp án: B

========================================================================

https://khoahoc.vietjack.com/thi-online/giai-toan-6-chuong-2-so-nguyen/16064


\textbf{{QUESTION}}

Tính và so sánh kết quả: (-2) + (-3) và (-3) + (-2)

\textbf{{ANSWER}}

(-2) + (-3) = -5        (-3) + (-2) = -5
Kết quả của hai phép tính là bằng nhau

========================================================================

https://khoahoc.vietjack.com/thi-online/giai-toan-6-chuong-2-so-nguyen/16064


\textbf{{QUESTION}}

Tính và so sánh kết quả: (-5) + (+7) và (+7) + (-5)

\textbf{{ANSWER}}

(-5) + (+7) = 2        (+7) + (-5) = 2
Kết quả của hai phép tính là bằng nhau

========================================================================

https://khoahoc.vietjack.com/thi-online/giai-toan-6-chuong-2-so-nguyen/16064


\textbf{{QUESTION}}

Tính và so sánh kết quả: (-8) + (+4) và (+4) + (-8).

\textbf{{ANSWER}}

(-8) + (+4) = -4        (+4) + (-8) = -4
Kết quả của hai phép tính là bằng nhau

========================================================================

https://khoahoc.vietjack.com/thi-online/giai-toan-6-chuong-2-so-nguyen/16064


\textbf{{QUESTION}}

Tìm tổng của tất cả các số nguyên a, biết -3 < a < 3.

\textbf{{ANSWER}}

Các số nguyên a là -2; -1; 0; 1; 2;
Tổng các số nguyên a là: (-2) + (-1) + 0 + 1 + 2 = [(-2) + 2] + [(-1) + 1] + 0 = 0 + 0 + 0 = 0

========================================================================

https://khoahoc.vietjack.com/thi-online/giai-toan-6-chuong-2-so-nguyen/16064


\textbf{{QUESTION}}

Tính: 126 + (-20) + 2004 + (-106)

\textbf{{ANSWER}}

126 + (–20) + 2004 + (–106)
= (126 – 20) + 2004 +(–106)
= 106 + 2004 + (–106)
= 106 + (–106) + 2004 (tính chất giao hoán).
= 0 + 2004
= 2004.

========================================================================

https://khoahoc.vietjack.com/thi-online/8-cau-trac-nghiem-toan-9-bai-2-duong-kinh-va-day-cua-duong-tron-co-dap-an-nhan-biet


\textbf{{QUESTION}}

Cho đường tròn (O) đường kính AB và dây CD không đi qua tâm. Khẳng định nào sau đây là đúng?
A. AB > CD
B. AB = CD
C. AB < CD
D. AB $$ \le $$ CD

\textbf{{ANSWER}}

Đáp án A
Trong các dây của một đường tròn, dây lớn nhất là đường kính

========================================================================

https://khoahoc.vietjack.com/thi-online/8-cau-trac-nghiem-toan-9-bai-2-duong-kinh-va-day-cua-duong-tron-co-dap-an-nhan-biet


\textbf{{QUESTION}}

“Trong các dây của một đường tròn, đường kính là dây có độ dài…” Cụm từ thích hợp điền vào chỗ trống là:
A. nhỏ nhất
B. lớn nhất
C. bằng 10cm
D. bằng tổng hai dây bất kì

\textbf{{ANSWER}}

Đáp án B
Trong các dây của một đường tròn, đường kính là dây có độ dài lớn nhất

========================================================================

https://khoahoc.vietjack.com/thi-online/8-cau-trac-nghiem-toan-9-bai-2-duong-kinh-va-day-cua-duong-tron-co-dap-an-nhan-biet


\textbf{{QUESTION}}

Cho đường tròn (O) có hai dây AB, CD không đi qua tâm. Biết khoảng cách từ tâm đến hai dây là bằng nhau. Kết luận nào sau đây là đúng?
A. AB > CD
B. AB = CD
C. AB < CD
D. AB // CD

\textbf{{ANSWER}}

Đáp án B
Trong một đường tròn, hai dây cách đều tâm thì bằng nhau

========================================================================

https://khoahoc.vietjack.com/thi-online/8-cau-trac-nghiem-toan-9-bai-2-duong-kinh-va-day-cua-duong-tron-co-dap-an-nhan-biet


\textbf{{QUESTION}}

Cho đường tròn (O) có hai dây AB, CD không đi qua tâm. Biết khoảng cách từ tâm O đến dây AB lớn hơn khoảng cách từ tâm O đến dây CD. Kết luận nào sau đây là đúng?
A. AB > CD
B. AB = CD
C. AB < CD
D. AB // CD

\textbf{{ANSWER}}

Đáp án C
Trong một đường tròn: Dây nào gần tâm hơn thì dây đó lớn hơn
Từ đề bài ta thấy dây CD gần tâm hơn dây AB nên AC > AB

========================================================================

https://khoahoc.vietjack.com/thi-online/8-cau-trac-nghiem-toan-9-bai-2-duong-kinh-va-day-cua-duong-tron-co-dap-an-nhan-biet


\textbf{{QUESTION}}

“Trong một đường tròn, đường kính đi qua trung điểm của một dây không đi qua tâm thì… với dây ấy”. Điền vào dấu… cụm từ thích hợp
A. nhỏ hơn
B. bằng
C. song song
D. vuông góc

\textbf{{ANSWER}}

Đáp án D
Trong một đường tròn, đường kính đi qua trung điểm của một dây không đi qua tâm thì vuông góc với dây ấy

========================================================================

https://khoahoc.vietjack.com/thi-online/10-cau-trac-nghiem-toan-6-ket-noi-tri-thuc-bai-31-mot-so-bai-toan-ve-ti-so-va-ti-so-phan-tram-co-dap


\textbf{{QUESTION}}

Khoai lang là thực phẩm bổ dưỡng, giàu chất xơ và rất tốt cho sức khỏe. Theo Viện Dinh dưỡng Quốc gia, trong 200 gam khoai lang có chứa khoảng 57 gam chất bột đường và 2,6 gam chất xơ.
Viết tỉ số khối lượng chất bột đường và khối lượng của khoai lang.

\textbf{{ANSWER}}

Tỉ số khối lượng chất bột đường và khối lượng của khoai lang là: $$ \frac{57}{200}$$

========================================================================

https://khoahoc.vietjack.com/thi-online/10-cau-trac-nghiem-toan-6-ket-noi-tri-thuc-bai-31-mot-so-bai-toan-ve-ti-so-va-ti-so-phan-tram-co-dap


\textbf{{QUESTION}}

Khoai lang là thực phẩm bổ dưỡng, giàu chất xơ và rất tốt cho sức khỏe. Theo Viện Dinh dưỡng Quốc gia, trong 200 gam khoai lang có chứa khoảng 57 gam chất bột đường và 2,6 gam chất xơ.
Bằng cách tương tự, em hãy viết tỉ số khối lượng chất xơ và khối lượng của khoai lang.

\textbf{{ANSWER}}

Tỉ số khối lượng chất xơ và khối lượng của khoai lang là:
$$ \frac{2,6}{200}=\frac{26}{2000}=\frac{26:2}{2000:2}=\frac{13}{1000}$$

========================================================================

https://khoahoc.vietjack.com/thi-online/10-cau-trac-nghiem-toan-6-ket-noi-tri-thuc-bai-31-mot-so-bai-toan-ve-ti-so-va-ti-so-phan-tram-co-dap


\textbf{{QUESTION}}

Viết tỉ số khối lượng chất xơ và khối lượng của khoai lang dưới dạng tỉ số phần trăm.

\textbf{{ANSWER}}

Tỉ số khối lượng chất xơ và khối lượng của khoai lang dưới dạng phần trăm là:
131000.100%=13.1001000%$$ \frac{13}{1000}.100\%=\frac{13.100}{1000}\%$$=1310%=1,3%$$ =\frac{13}{10}\%=1,3\%$$

========================================================================

https://khoahoc.vietjack.com/thi-online/10-cau-trac-nghiem-toan-6-ket-noi-tri-thuc-bai-31-mot-so-bai-toan-ve-ti-so-va-ti-so-phan-tram-co-dap


\textbf{{QUESTION}}

Trong đại hội chi đội lớp 6A, bạn Dũng được 36 đội viên (trong tổng số 45 đội viên chi đội) bầu làm Chi đội trưởng. Bạn Dũng đã trúng cử Chi đội trưởng với tỉ số phần trăm phiếu bầu là bao nhiêu?

\textbf{{ANSWER}}

Bạn Dũng đã trúng cử Chi đội trưởng với tỉ số phần trăm phiêu biều là:
3645.100%=36.10045%=$$ \frac{36}{45}.100\%=\frac{36.100}{45}\%=$$4.1005%=80%$$ \frac{4.100}{5}\%=80\%$$
Vậy bạn Dũng đã trúng cử Chi đội trưởng với 80% phiếu bầu.

========================================================================

https://khoahoc.vietjack.com/thi-online/top-4-de-thi-toan-10-hoc-ki-2-co-dap-an/45859


\textbf{{QUESTION}}

Đường tròn (C): $$ {x}^{2}$$ + $$ {y}^{2}$$ - 2x + 4y - 3 = 0 có tâm I, bán kính R là:
A. I(-1;2), R = $$ \sqrt{2}$$
B. I(-1;2), R = 2$$ \sqrt{2}$$
C. I(1;-2), R = $$ \sqrt{2}$$
D. I(1;-2), R = 2$$ \sqrt{2}$$

\textbf{{ANSWER}}

Đáp án: D
(C): $$ {x}^{2}$$ + $$ {y}^{2}$$ - 2x + 4y - 3 = 0 ⇔ (x - 1$$ {)}^{2}$$ + (y + 2$$ {)}^{2}$$ = 8
Suy ra, I(1;-2), R = $$ \sqrt{8}$$ = 2$$ \sqrt{2}$$

========================================================================

https://khoahoc.vietjack.com/thi-online/top-4-de-thi-toan-10-hoc-ki-2-co-dap-an/45859


\textbf{{QUESTION}}

Tìm các giá trị của tham số m để x2 - 2x - m ≥ 0 ∀x
A. m ≤ 0
B. m < 0
C. m ≤ -1
D. m < -1

\textbf{{ANSWER}}

Đáp án: D
x2$$ {x}^{2}$$ - 2x - m ≥ 0
Ta có: Δ' = (-1)2$$ {)}^{2}$$ -1.(-m) = m + 1
Để x2$$ {x}^{2}$$ - 2x - m ≥ 0 ∀x thì Δ' < 0 ⇔ m + 1 < 0 ⇔ m < -1

========================================================================

https://khoahoc.vietjack.com/thi-online/bo-30-de-thi-vao-10-mon-toan-co-loi-giai-chi-tiet/104792


\textbf{{QUESTION}}

Cho $$ \Delta ABC$$ vuông tại  Hệ thức nào sau đây đúng ?
$$ A.\mathrm{cos}\angle B=\frac{AB}{BC}$$
$$ B.\mathrm{cos}\angle B=\frac{AC}{AB}$$
$$ C.\mathrm{cos}\angle B=\frac{AC}{BC}$$
$$ D.\mathrm{cos}\angle B=\frac{AB}{AC}$$

\textbf{{ANSWER}}

$$ \mathrm{cos}\angle B=\frac{AB}{BC}$$
Chọn đáp án A

========================================================================

https://khoahoc.vietjack.com/thi-online/bo-30-de-thi-vao-10-mon-toan-co-loi-giai-chi-tiet/104792


\textbf{{QUESTION}}

Giá trị của biểu thức $$ \sqrt{64}-\sqrt{49}-\sqrt{81}$$ là :
$$ A.10$$
$$ B.-9$$
$$ C.6$$
$$ D.-8$$

\textbf{{ANSWER}}

$$ \sqrt{64}-\sqrt{49}-\sqrt{81}=8-7-9=-8$$
Chọn đáp án D

========================================================================

https://khoahoc.vietjack.com/thi-online/bo-30-de-thi-vao-10-mon-toan-co-loi-giai-chi-tiet/104792


\textbf{{QUESTION}}

Số phần tử của tập hợp A={30;31;32;....;46}$$ A=\left\{30;31;32;\mathrm{....};46\right\}$$
A.16$$ A.16$$
B.17$$ B.17$$
C.18$$ C.18$$
D.46$$ D.46$$

\textbf{{ANSWER}}

Số phần tử của tập hợp A:46−30+1=17$$ A:46-30+1=17$$ (phần tử)
Chọn đáp án B

========================================================================

https://khoahoc.vietjack.com/thi-online/bo-30-de-thi-vao-10-mon-toan-co-loi-giai-chi-tiet/104792


\textbf{{QUESTION}}

A.M=|a+2|$$ A.M=\left|a+2\right|$$
B.M=(a+2)2$$ B.M={\left(a+2\right)}^{2}$$
C.M=−(a+2)$$ C.M=-\left(a+2\right)$$
D.M=a+2$$ D.M=a+2$$

\textbf{{ANSWER}}

M=√(a+2)2=|a+2|$$ M=\sqrt{{\left(a+2\right)}^{2}}=\left|a+2\right|$$(với mọi a).Chọn đáp án A

========================================================================

https://khoahoc.vietjack.com/thi-online/bo-30-de-thi-vao-10-mon-toan-co-loi-giai-chi-tiet/104792


\textbf{{QUESTION}}

Điều kiện của x để biểu thức √7−5x$$ \sqrt{7-5x}$$ xác định là :
A.x≤75$$ A.x\le \frac{7}{5}$$
B.x≥−75$$ B.x\ge -\frac{7}{5}$$
C.x≤2$$ C.x\le 2$$
D.với mọi x∈ℝ$$ D.với\quad mọi\quad x\in \mathbb{R} $$

\textbf{{ANSWER}}

√7−5x$$ \sqrt{7-5x}$$ xác định khi 7−5x≥0⇔x≤75$$ 7-5x\ge 0\Leftrightarrow x\le \frac{7}{5}$$
Chọn đáp án A

========================================================================

https://khoahoc.vietjack.com/thi-online/10-bai-tap-su-dung-nhi-thuc-newton-de-tinh-gia-tri-gan-dung-co-loi-giai


\textbf{{QUESTION}}

Sử dụng 3 số hạng đầu tiên của khai triển nhị thức Newton, giá trị gần đúng của biểu thức (3 + 0,03)4 là
A. 84,2886;
B. 86,2886;
C. 44,2868;

\textbf{{ANSWER}}

Đáp án đúng là: A
Ta có (3 + 0,03)4 ≈ 34 + 4 . 33 . 0,03 + 6 . 32 . 0,032 = 84,2886.

========================================================================

https://khoahoc.vietjack.com/thi-online/10-bai-tap-su-dung-nhi-thuc-newton-de-tinh-gia-tri-gan-dung-co-loi-giai


\textbf{{QUESTION}}

Sử dụng 2 số hạng đầu tiên của khai triển nhị thức Newton, giá trị gần đúng của biểu thức (3 + 0,03)5 là 
A. 254,15;
B. 255,15;
C. 256,15;

\textbf{{ANSWER}}

Đáp án đúng là: B
Ta có (3 + 0,03)5 ≈ 35 + 5 . 34 . 0,03 = 255,15.

========================================================================

https://khoahoc.vietjack.com/thi-online/10-bai-tap-su-dung-nhi-thuc-newton-de-tinh-gia-tri-gan-dung-co-loi-giai


\textbf{{QUESTION}}

Sử dụng 4 số hạng đầu tiên của khai triển nhị thức Newton, giá trị gần đúng của biểu thức (3 − 0,02)5 là 
A. 235,00728;
B. 236,00768;
C. 235,00125;

\textbf{{ANSWER}}

Đáp án đúng là: A
(3 − 0,02)5 ≈ 35 + 5 . 34 . (−0,02) + 10 . 33 . (−0,02)2 + 10 . 32 . (−0,02)3 = 235,00728.

========================================================================

https://khoahoc.vietjack.com/thi-online/10-bai-tap-su-dung-nhi-thuc-newton-de-tinh-gia-tri-gan-dung-co-loi-giai


\textbf{{QUESTION}}

Sử dụng 3 số hạng đầu tiên của khai triển nhị thức Newton, giá trị gần đúng của biểu thức (3 − 0,02)4 là 
A. 72,8616;
B. 74,8616;
C. 76,8616;

\textbf{{ANSWER}}

Đáp án đúng là: D
Ta có (3 − 0,02)4 ≈ 34 + 4 . 33 . (−0,02) + 6 . 32 . (−0,02)2 = 78,8616.

========================================================================

https://khoahoc.vietjack.com/thi-online/10-bai-tap-su-dung-nhi-thuc-newton-de-tinh-gia-tri-gan-dung-co-loi-giai


\textbf{{QUESTION}}

Sai số tuyệt đối khi dùng 3 số hạng đầu tiên để tính giá trị gần đúng của biểu thức (2 + 0,05)4 là
A. 0,00100625;
B. 0,00200625;
C. 0,00300625;

\textbf{{ANSWER}}

Đáp án đúng là: A
Ta có: (2 + 0,05)4 ≈ 24 + 4 . 23 . 0,05 + 6 . 22 . 0,052 = 17,66.
Sử dụng máy tính cầm tay, ta kiểm tra được: (2 + 0,05)4 = 17,66100625.

========================================================================

https://khoahoc.vietjack.com/thi-online/16-bai-tap-cach-viet-gia-thiet-ket-luan-ve-hinh-va-chung-minh-mot-dinh-li-co-loi-giai


\textbf{{QUESTION}}

Cho định lí: “Nếu hai đường thẳng phân biệt cùng cắt đường thẳng thứ ba và trong các góc tạo thành có một cặp góc so le trong bằng nhau thì các cặp góc đồng vị bằng nhau”.
Viết giả thiết, kết luận của định lí trên;

\textbf{{ANSWER}}

Hướng dẫn giải:
GT
aa' cắt cc’ tại A; bb' cắt cc’ tại B; 
$\widehat {aAB} = \widehat {ABb'}$
KL
$\widehat {cAa'} = \widehat {ABb'}$
$\widehat {aAB} = \widehat {bBc'}$
$\widehat {aAc} = \widehat {bBA}$
$\widehat {a'AB} = \widehat {b'Bc'}$

========================================================================

https://khoahoc.vietjack.com/thi-online/10-bai-tap-khang-dinh-co-ton-tai-hay-khong-mot-tam-giac-khi-biet-do-dai-ba-doan-thang-co-loi-giai


\textbf{{QUESTION}}

Bộ ba độ dài nào sau đây có thể là độ dài ba cạnh của một tam giác?
A. 3 cm; 8 cm; 4 cm;
B. 5 cm; 7 cm; 13 cm;
C. 9 cm; 3 cm; 5 cm;

\textbf{{ANSWER}}

Hướng dẫn giải:
Đáp án đúng là: D
Ta có:
⦁ 8 > 3 + 4, không thỏa mãn bất đẳng thức tam giác;
⦁ 13 > 5 + 7, không thỏa mãn bất đẳng thức tam giác;
⦁ 9 – 5 > 3, không thỏa mãn bất đẳng thức tam giác;
⦁ 5 – 2 < 4 < 5 + 2 nên bộ ba độ dài 2 cm; 4 cm; 5 cm có thể là độ dài ba cạnh của một tam giác.
Vậy ta chọn phương án D.

========================================================================

https://khoahoc.vietjack.com/thi-online/10-bai-tap-khang-dinh-co-ton-tai-hay-khong-mot-tam-giac-khi-biet-do-dai-ba-doan-thang-co-loi-giai


\textbf{{QUESTION}}

Với bộ ba đọan thẳng dưới đây, bộ ba nào không thể là ba cạnh của một tam giác? 
A. 11 cm; 7 cm; 17 cm;
B. 12 cm; 6 cm; 5 cm;
C. 8 cm; 11 cm; 5 cm;

\textbf{{ANSWER}}

Hướng dẫn giải:
Đáp án đúng là: B
Ta có 12 > 6 + 5 không thỏa mãn bất đẳng thức tam giác; 
Nên bộ ba độ dài 12 cm; 6 cm; 5 cm không thể là ba cạnh của một tam giác.

========================================================================

https://khoahoc.vietjack.com/thi-online/10-bai-tap-khang-dinh-co-ton-tai-hay-khong-mot-tam-giac-khi-biet-do-dai-ba-doan-thang-co-loi-giai


\textbf{{QUESTION}}

Bộ ba độ dài nào sau đây có thể là độ dài ba cạnh của một tam giác?
A. 8 cm; 5 cm; 10 cm;
B. 11 cm; 20 cm; 9 cm;
C. 6 cm; 4 cm; 1 cm;

\textbf{{ANSWER}}

Hướng dẫn giải:
Đáp án đúng là: A
Ta có:
⦁ 8 – 5 < 10 < 8 + 5 nên bộ ba độ dài 8 cm; 5 cm; 10 cm có thể là độ dài ba cạnh của một tam giác;
⦁ 11 + 9 = 20, không thỏa mãn bất đẳng thức tam giác;
⦁ 6 > 4 + 1, không thỏa mãn bất đẳng thức tam giác;
⦁ 8 + 4 < 15, không thỏa mãn bất đẳng thức tam giác.
Vậy ta chọn phương án A.

========================================================================

https://khoahoc.vietjack.com/thi-online/10-bai-tap-khang-dinh-co-ton-tai-hay-khong-mot-tam-giac-khi-biet-do-dai-ba-doan-thang-co-loi-giai


\textbf{{QUESTION}}

Cho một tam giác cân có độ dài hai cạnh là 3,9 cm và 7,9 cm. Chu vi của tam giác đó là
A. 11,8 cm;
B. 15,7 cm;
C. 19,7 cm;

\textbf{{ANSWER}}

Hướng dẫn giải:
Đáp án đúng là: C
Cạnh thứ ba của tam giác cân có độ dài bằng một trong hai cạnh kia.
Loại trường hợp độ dài cạnh thứ ba bằng 3,9 cm vì 3,9 + 3,9 = 7,8 < 7,9, không thỏa mãn bất đẳng thức tam giác.
Trường hợp độ dài cạnh thứ ba bằng 7,9 cm thỏa mãn vì 7,9 + 3,9 > 7,9, thỏa mãn bất đẳng thức tam giác.
Vậy chu vi của tam giác đó là: 7,9 + 7,9 + 3,9 = 19,7 (cm).

========================================================================

https://khoahoc.vietjack.com/thi-online/10-bai-tap-khang-dinh-co-ton-tai-hay-khong-mot-tam-giac-khi-biet-do-dai-ba-doan-thang-co-loi-giai


\textbf{{QUESTION}}

Cho tam giác MNP có MN = 2 cm và MP = 5 cm. Độ dài của cạnh NP là 
A. 2 cm;
B. 1 cm;
C. 4 cm;

\textbf{{ANSWER}}

Hướng dẫn giải:
Đáp án đúng là: C
Áp dụng bất đẳng thức tam giác cho ∆MNP, ta có:
MP – MN < NP < MP + MN
5 – 2 < NP < 5 + 2
3 < NP < 7
Dựa vào các phương án, ta thấy chỉ có phương án NP = 4 cm thỏa mãn.

========================================================================

https://khoahoc.vietjack.com/thi-online/15-cau-trac-nghiem-toan-7-chan-troi-sang-tao-bai-2-cac-phep-tinh-voi-so-huu-ti-co-dap-an-phan-2/103281


\textbf{{QUESTION}}

Bác An gửi ngân hàng 120 triệu đồng với kì hạn 1 năm, lãi suất 6,5%/năm. Hết kì hạn 1 năm, bác rút $$ \frac{1}{2}$$ số tiền (cả gốc và lãi). Tính số tiền còn lại trong ngân hàng.

\textbf{{ANSWER}}

Hướng dẫn giải
Đáp án đúng là: A
Số tiền lãi khi hết kì hạn 1 năm là: 
120.6,5% = 7,8 (triệu đồng).
Số tiền bác An nhận được sau 1 năm là: 
120 + 7,8 = 127,8 (triệu đồng)
Số tiền bác An rút ra là: 
127,8. $$ \frac{1}{2}$$= 63,9 (triệu đồng)
Số tiền còn lại trong ngân hàng là: 
127,8 – 63,9 = 63,9 (triệu đồng)
Ta chọn phương án A.

========================================================================

https://khoahoc.vietjack.com/thi-online/15-cau-trac-nghiem-toan-7-chan-troi-sang-tao-bai-2-cac-phep-tinh-voi-so-huu-ti-co-dap-an-phan-2/103281


\textbf{{QUESTION}}

Một cửa hàng sách có chương trình khuyến mãi như sau: Khách hàng mua đơn từ 300 000 đồng trở lên sẽ được giảm 10% tổng số tiền của hoá đơn. Bạn Nam mua 3 quyển sách, mỗi quyển đều có giá 120 000 đồng. Bạn đưa cho nhân viên thu ngân 500 000 đồng. Hỏi cô thu ngân phải trả lại Nam bao nhiêu tiền?

\textbf{{ANSWER}}

Hướng dẫn giải
Đáp án đúng là: B
Số tiền mua 3 cuốn quyển với giá niêm yết là:
120 000. 3 = 360 000 (đồng)
Vì bạn Nam mua ba quyển sách nên hóa đơn của bạn đã lớn hơn 300 000 đồng.
Do đó bạn Nam sẽ được giảm giá 10%.
Số tiền bạn Nam được giảm là:
360 000 .10% = 36 000 (đồng)
Số tiền Nam phải trả là:
360 000 – 36 000 = 324 000 (đồng)
Bạn Nam đưa thu ngân 500 000 đồng nên số tiền nhân viên thu ngân phải trả lại Nam là: 
500 000 – 324 000 = 176 000 (đồng)
Ta chọn phương án B.

========================================================================

https://khoahoc.vietjack.com/thi-online/15-cau-trac-nghiem-toan-7-chan-troi-sang-tao-bai-2-cac-phep-tinh-voi-so-huu-ti-co-dap-an-phan-2/103281


\textbf{{QUESTION}}

Tính giá trị biểu thức A=11.2+12.3+13.4+…+12022.2023$$ A=\frac{1}{1.2}+\frac{1}{2.3}+\frac{1}{3.4}+\dots +\frac{1}{2022.2023}$$
A. 1

\textbf{{ANSWER}}

Hướng dẫn giải
Đáp án đúng là: B
Ta có: 
$$ \frac{1}{1.2}=\frac{1}{1}-\frac{1}{2}$$; 
$$ \frac{1}{2.3}=\frac{1}{2}-\frac{1}{3}$$; 
…
Khi đó $$ A=1-\frac{1}{2}+\frac{1}{2}-\frac{1}{3}+\frac{1}{3}-\frac{1}{4}+\dots +\frac{1}{2021}-\frac{1}{2022}+\frac{1}{2022}-\frac{1}{2023}$$
$$ A=1+\left(-\frac{1}{2}+\frac{1}{2}\right)+\left(-\frac{1}{3}+\frac{1}{3}\right)+\dots +\left(-\frac{1}{2022}+\frac{1}{2022}\right)-\frac{1}{2023}$$ 
$$ A=1-\frac{1}{2023}$$ 
$$ A=\frac{2023}{2023}-\frac{1}{2023}=\frac{2022}{2023}.$$ 
Ta chọn phương án B.

========================================================================

https://khoahoc.vietjack.com/thi-online/20-cau-trac-nghiem-toan-10-chan-troi-sang-tao-duong-thang-trong-mat-phang-toa-do-phan-2-co-dap-an/111648


\textbf{{QUESTION}}

Đường thẳng d đi qua 2 điểm A(1; 3) và B(2; 5). Viết phương trình đoạn chắn của đường thẳng d.
A. $$ \frac{x}{-\frac{1}{2}}-\frac{y}{1}=1$$
B. $$ \frac{x}{-2}+\frac{y}{1}=1$$
C. $$ \frac{x}{-\frac{1}{2}}+\frac{y}{1}=1$$
D. $$ \frac{x}{2}+\frac{y}{1}=1$$

\textbf{{ANSWER}}

Hướng dẫn giải 
Đáp án đúng là: C
Đường thẳng d có vectơ chỉ phương là: $$ \overrightarrow{u}=\overrightarrow{AB}=\left(1;2\right)$$
Suy ra đường thẳng d có vectơ pháp tuyến là: $$ \overrightarrow{n}=\left(2;-1\right)$$.
Đường thẳng d có vectơ pháp tuyến $$ \overrightarrow{n}=\left(2;-1\right)$$ và đi qua điểm A(1; 3) nên có phương trình tổng quát là:
2(x – 1) – (y – 3) = 0 hay 2x – y + 1 = 0.
Đường thẳng d cắt 2 trục tọa độ Ox và Oy lần lượt tại M$$ \left(-\frac{1}{2};0\right)$$ và N(0;1) .
Vậy phương trình đoạn chắn của đường thẳng d là: $$ \frac{x}{-\frac{1}{2}}+\frac{y}{1}=1$$.

========================================================================

https://khoahoc.vietjack.com/thi-online/20-cau-trac-nghiem-toan-10-chan-troi-sang-tao-duong-thang-trong-mat-phang-toa-do-phan-2-co-dap-an/111648


\textbf{{QUESTION}}

Trong mặt phẳng tọa độ Oxy, cho điểm M(a; b) di động trên đường thẳng d: 2x + 5y – 10 = 0. Tìm a, b để khoảng cách ngắn nhất từ điểm A đến điểm M, biết điểm A(3; ‒1).
A. a = 11129$$ \frac{111}{29}$$ và b = 2629$$ \frac{26}{29}$$;
B. a = 1029$$ \frac{10}{29}$$ và b = 1629$$ \frac{16}{29}$$;
C. a = 10529$$ \frac{105}{29}$$ và b = 1629$$ \frac{16}{29}$$;

\textbf{{ANSWER}}

Hướng dẫn giải 
Đáp án đúng là: C
Để khoảng cách AM là ngắn nhất thì M là hình chiếu của A lên đường thẳng d.
Khi đó AM vuông góc với d, do đó vectơ pháp tuyến của đường thẳng AM chính là vectơ chỉ phương của đường thẳng d.
Vectơ pháp tuyến của đường thẳng d là: $$ \overrightarrow{n}=\left(2;5\right)$$
Vectơ chỉ phương của đường thẳng d là: $$ \overrightarrow{u}=\left(5;-2\right)$$
Khi đó $$ \overrightarrow{u}=\left(5;-2\right)$$ là vectơ pháp tuyến của đường thẳng AM.
Phương trình đường thẳng AM là:
5.(x – 3) – 2.(y + 1) = 0 hay 5x – 2y – 17 = 0.
M là giao điểm của 2 đường thẳng AM và d nên tọa độ điểm M là nghiệm của hệ:
$$ \left\{\begin{array}{l}5x-2y-17=0\\ 2x+5y-10=0\end{array}\right.\Leftrightarrow \left\{\begin{array}{l}x=\frac{105}{29}\\ y=\frac{16}{29}\end{array}\right.$$ .
Vậy a = $$ \frac{105}{29}$$ và b = $$ \frac{16}{29}$$.

========================================================================

https://khoahoc.vietjack.com/thi-online/20-cau-trac-nghiem-toan-10-chan-troi-sang-tao-duong-thang-trong-mat-phang-toa-do-phan-2-co-dap-an/111648


\textbf{{QUESTION}}

Cho phương trình tham số của d: $$ \left\{\begin{array}{l}x=t\\ y=t-1\end{array}\right.$$ (t là tham số). Tính khoảng cách từ trung điểm M của AB đến d biết A(2; 4) và B(0; 6).
A. d(M, d) = $$ \frac{5}{2}$$;
B. d(M, d) = $$ \frac{5\sqrt{2}}{2}$$;
C. d(M, d) = $$ \frac{7\sqrt{2}}{2}$$;

\textbf{{ANSWER}}

Hướng dẫn giải 
Đáp án đúng là: B
M là trung điểm của AB với A(2; 4) và B(0; 6) nên M(1; 5).
Xét phương trình của đường thẳng d: $$ \left\{\begin{array}{l}x=t\\ y=t-1\end{array}\right.$$
Cho t = 0 ta có điểm C(0; ‒1) thuộc d.
Vectơ chỉ phương của d là: $$ \overrightarrow{u}=\left(1;1\right)$$
Suy ra vectơ pháp tuyến của d là: $$ \overrightarrow{n}=\left(1;-1\right)$$.
Đường thẳng d có vectơ pháp tuyến $$ \overrightarrow{n}=\left(1;-1\right)$$ và đi qua điểm C(0; ‒1) nên có phương trình tổng quát là:
1.(x – 0) – (y +1) = 0 hay x – y – 1= 0.
Khi đó $$ d\left(M,d\right)=\frac{\left|1-5-1\right|}{\sqrt{{1}^{2}+{\left(-1\right)}^{2}}}=\frac{5}{\sqrt{2}}=\frac{5\sqrt{2}}{2}.$$
Vậy $$ d\left(M,d\right)=\frac{5\sqrt{2}}{2}.$$

========================================================================

https://khoahoc.vietjack.com/thi-online/20-cau-trac-nghiem-toan-10-chan-troi-sang-tao-duong-thang-trong-mat-phang-toa-do-phan-2-co-dap-an/111648


\textbf{{QUESTION}}

Viết phương trình tham số của đường thẳng d đi qua M(2; 6) và song song với đường thẳng x + 3y – 10 = 0.
A. {x=2+ty=6+3t$$ \left\{\begin{array}{l}x=2+t\\ y=6+3t\end{array}\right.$$
B. {x=2+3ty=6+t$$ \left\{\begin{array}{l}x=2+3t\\ y=6+t\end{array}\right.$$
C. {x=5−ty=5+3t$$ \left\{\begin{array}{l}x=5-t\\ y=5+3t\end{array}\right.$$
D. {x=5+3ty=5−t$$ \left\{\begin{array}{l}x=5+3t\\ y=5-t\end{array}\right.$$

\textbf{{ANSWER}}

Hướng dẫn giải 
Đáp án đúng là: D
Đường thẳng x + 3y – 10 = 0 có vectơ pháp tuyến là: →n=(1;3)$$ \overrightarrow{n}=\left(1;3\right)$$.
Do đường thẳng d song song với đường thẳng x + 3y – 10 = 0 nên →n=(1;3)$$ \overrightarrow{n}=\left(1;3\right)$$ cũng là vectơ pháp tuyến của đường thẳng d.
Khi đó đường thẳng d có vectơ chỉ phương là →u=(3;−1)$$ \overrightarrow{u}=\left(3;-1\right)$$.
Đường thẳng d có vectơ chỉ phương →u=(3;−1)$$ \overrightarrow{u}=\left(3;-1\right)$$ và đi qua M(2; 6) có phương trình tham số là: {x=2+3ty=6−t$$ \left\{\begin{array}{l}x=2+3t\\ y=6-t\end{array}\right.$$.
Với t = 1 ta có {x=2+3.1=5y=6−1=5$$ \left\{\begin{array}{l}x=2+3.1=5\\ y=6-1=5\end{array}\right.$$, khi đó điểm A(5; 5) thuộc đường thẳng d.
Do đó ta có phương trình tham số của đường thẳng d là {x=5+3ty=5−t$$ \left\{\begin{array}{l}x=5+3t\\ y=5-t\end{array}\right.$$.
Vậy ta chọn phương án D.

========================================================================

https://khoahoc.vietjack.com/thi-online/20-cau-trac-nghiem-toan-10-chan-troi-sang-tao-duong-thang-trong-mat-phang-toa-do-phan-2-co-dap-an/111648


\textbf{{QUESTION}}

Đường thẳng d tạo với đường thẳng ∆$$ ∆$$: x + 2y – 6 = 0 một góc 45°. Hệ số góc k của đường thẳng d là:
A. k = 13$$ \frac{1}{3}$$ hoặc k = – 3;
B. k = 13$$ \frac{1}{3}$$ hoặc k = 3;
C. k = -13$$ -\frac{1}{3}$$ hoặc k = – 3;

\textbf{{ANSWER}}

Hướng dẫn giải
Đáp án đúng là: A
Đường thẳng D: x + 2y – 6 = 0 có vectơ pháp tuyến là →nΔ=(1;2)$$ {\overrightarrow{n}}_{\Delta }=\left(1;2\right)$$.
Gọi →nd=(a;b)$$ {\overrightarrow{n}}_{d}=\left(a;b\right)$$ là vectơ pháp tuyến của đường thẳng d.
Khi đó hệ số góc của đường thẳng d là k=−ab$$ k=-\frac{a}{b}$$.
Góc giữa hai đường thẳng d và ∆$$ ∆$$ là 45° nên ta có:
cos(d,Δ)=|cos(→nd,→nΔ)|=cos45°$$ \text{cos}\left(d,\Delta \right)=\left|\text{cos}\left({\overrightarrow{n}}_{d},{\overrightarrow{n}}_{\Delta }\right)\right|=c\text{os45}°$$
Hay |1.a+2.b|√12+22.√a2+b2=1√2$$ \frac{\left|1.a+2.b\right|}{\sqrt{{1}^{2}+{2}^{2}}.\sqrt{{a}^{2}+{b}^{2}}}=\frac{1}{\sqrt{2}}$$
⇔√5.√a2+b2=√2.|a+2b|$$ \Leftrightarrow \sqrt{5}.\sqrt{{a}^{2}+{b}^{2}}=\sqrt{2}.\left|a+2b\right|$$
Û 5(a2 + b2) = 2(a + 2b)2
Û 5a2 + 5b2 = 2a2 + 8ab + 8b2
Û 3a2 – 8ab – 3b2 = 0
⇔[a=3ba=−13b⇔[ab=3ab=−13⇔[k=−ab=−3k=−ab=13$$ \Leftrightarrow \left[\begin{array}{l}a=3b\\ a=-\frac{1}{3}b\end{array}\right.\Leftrightarrow \left[\begin{array}{l}\frac{a}{b}=3\\ \frac{a}{b}=-\frac{1}{3}\end{array}\right.\Leftrightarrow \left[\begin{array}{l}k=-\frac{a}{b}=-3\\ k=-\frac{a}{b}=\frac{1}{3}\end{array}\right.$$.
Vậy ta chọn phương án A.

========================================================================

https://khoahoc.vietjack.com/thi-online/20-cau-trac-nghiem-toan-10-chan-troi-sang-tao-so-gan-dung-va-sai-so-co-dap-an-phan-2/110902


\textbf{{QUESTION}}

Độ dài của một quãng đường người ta đo được là 997 m ± 0,5 m. Sai số tương đối tối đa trong phép đo đó là:

\textbf{{ANSWER}}

Đáp án đúng là: B
Ta có: độ dài gần đúng của quãng đường là a = 997 (m) với độ chính xác là d = 0,5 (m).
Vì sai số tuyệt đối ∆a ≤ d = 0,5 nên ta có sai số tương đối của phép đo:
$$ {\delta }_{a}=\frac{{\Delta }_{a}}{\left|a\right|}\le \frac{d}{\left|a\right|}=\frac{\mathrm{0,5}}{997}\approx \mathrm{0,05}\%$$.
Vậy sai số tương đối tối đa là 0,05%.

========================================================================

https://khoahoc.vietjack.com/thi-online/20-cau-trac-nghiem-toan-10-chan-troi-sang-tao-so-gan-dung-va-sai-so-co-dap-an-phan-2/110902


\textbf{{QUESTION}}

Cho số a =27$$ \frac{2}{7}$$  và các giá trị gần đúng của a là 0,28; 0,29; 0,286; 0,2861. Hỏi giá trị gần đúng nào là tốt nhất?

\textbf{{ANSWER}}

Đáp án đúng là: D
Ta có các sai số tuyệt đối là:
$$ {\Delta }_{1}=\left|\frac{2}{7}-\mathrm{0,28}\right|=\frac{1}{175}$$;
$$ {\Delta }_{2}=\left|\frac{2}{7}-\mathrm{0,29}\right|=\frac{3}{700}$$;
$$ {\Delta }_{3}=\left|\frac{2}{7}-\mathrm{0,286}\right|=\frac{1}{3500}$$;
$$ {\Delta }_{4}=\left|\frac{2}{7}-\mathrm{0,2861}\right|=\frac{27}{70000}$$.
Ta thấy ∆4 nhỏ nhất nên số 0,2861 là số gần đúng tốt nhất.

========================================================================

https://khoahoc.vietjack.com/thi-online/20-cau-trac-nghiem-toan-10-chan-troi-sang-tao-so-gan-dung-va-sai-so-co-dap-an-phan-2/110902


\textbf{{QUESTION}}

Một tấm thép hình chữ nhật có chiều dài là: a = 20 m ± 0,01 m và chiều rộng b = 15 m ± 0,01 m. Chu vi P của tấm thép đó là:

\textbf{{ANSWER}}

Đáp án đúng là: D
Cách 1:
Nửa chu vi của tấm thép hình chữ nhật trên là:
(20 ± 0,01) + (15 ± 0,01) = (20 + 15) ± (0,01 + 0,01)
= 35 ± 0,02 (m).
Khi đó chu vi của tấm thép hình chữ nhật trên là:
2 . (35 ± 0,02) = 70 ± 0,04 (m).
Vậy P = 70 m ± 0,04 m.
Cách 2:
Gọi chiều dài và chiều rộng chính xác của tấm thép hình chữ nhật trên là x, y (x, y > 0) (m).
Khi đó ta có: x = 20 + i và y = 15 + j với – 0,01 ≤ i, j ≤ 0,01.
Chu vi là: P = 2 (x + y) = 2. [(20 + i) + (15 + j)] = 2. (35 + i + j)
P = 70 + 2. (i + j)
Ta có: – 0,01 ≤ i, j ≤ 0,01 suy ra –0,04 ≤ 2. (i + j) ≤ 0,04.
Vậy P = 70 m ± 0,04 m.

========================================================================

https://khoahoc.vietjack.com/thi-online/20-cau-trac-nghiem-toan-10-chan-troi-sang-tao-so-gan-dung-va-sai-so-co-dap-an-phan-2/110902


\textbf{{QUESTION}}

Hình chữ nhật có các cạnh x = 2 m ± 1 cm, y = 5 m ± 2 cm. Sai số tương đối của diện tích hình chữ nhật đó là?

\textbf{{ANSWER}}

Đáp án đúng là: C
Diện tích gần đúng của hình chữ nhật là: S = 2 . 5 = 10 (m2).
Giá trị lớn nhất của diện tích hình chữ nhật là:
(2 + 0,01) . (5 + 0,02) = 10,0902 (m2).
Giá trị nhỏ nhất của diện tích hình chữ nhật là:
(2 – 0,01) . (5 – 0,02) = 9,9102 (m2).
Ta có: 9,9102 – S ≤ ˉS−S$$ \overline{S}-S$$  ≤ 10,0902 – S
Hay 9,9102 – 10 ≤ ˉS−10$$ \overline{S}-10$$  ≤ 10,0902 – 10
Do đó 0,0898 ≤ ˉS−S$$ \overline{S}-S$$  ≤ 0,0902 nên |S−ˉS|≤0,0898$$ \left|S-\overline{S}\right|\le \mathrm{0,0898}$$ .
Sai số tuyệt đối là: Δs=|S−ˉS|≤0,0898$$ {\Delta }_{s}=\left|S-\overline{S}\right|\le \mathrm{0,0898}$$ .
Sai số tương đối là: δS=ΔS|S|≤0,089810≈0,9%$$ {\delta }_{S}=\frac{{\Delta }_{S}}{\left|S\right|}\le \frac{\mathrm{0,0898}}{10}\approx \mathrm{0,9}\%$$ .
Vậy ta chọn C.

========================================================================

https://khoahoc.vietjack.com/thi-online/20-cau-trac-nghiem-toan-10-chan-troi-sang-tao-so-gan-dung-va-sai-so-co-dap-an-phan-2/110902


\textbf{{QUESTION}}

Số ˉb$$ \overline{b}$$  được cho gần đúng là b = 4,5678, sai số tương đối của nó không vượt quá 0,5%. Tìm giá trị lớn nhất của sai số tuyệt đối?

\textbf{{ANSWER}}

Đáp án đúng là: B
Ta có: δb=Δb|b|$$ {\delta }_{b}=\frac{{\Delta }_{b}}{\left|b\right|}$$  suy ra Δb=δb.|b|$$ {\Delta }_{b}={\delta }_{b}.\left|b\right|$$ .
Do đó: ∆b ≤ 0,5% . 4,5678 = 2,2839% ≈ 0,0228.
Vậy ta chọn phương án B.

========================================================================

https://khoahoc.vietjack.com/thi-online/20-cau-trac-nghiem-toan-12-canh-dieu-bai-2-cong-thuc-xac-suat-toan-phan-cong-thuc-bayes-co-dap-an


\textbf{{QUESTION}}

I. Nhận biết
Cho $A,B$ là các biến cố của một phép thử $T$. Biết rằng $0 < P\left( B \right) < 1$, xác suất của biến cố A được tính theo công thức nào sau đây?
A. $P\left( A \right) = P\left( B \right).P\left( {A|B} \right) + P\left( {\overline B } \right).P\left( {A|\overline B } \right).$
B. $P\left( A \right) = P\left( B \right).P\left( {A|B} \right) + P\left( {\overline B } \right).P\left( {B|\overline A } \right).$
C. $P\left( A \right) = P\left( B \right).P\left( {A|B} \right) + P\left( {\overline A } \right).P\left( {A|\overline B } \right).$
D. $P\left( A \right) = P\left( B \right).P\left( {A|B} \right) + P\left( {\overline A } \right).P\left( {B|\overline A } \right).$

\textbf{{ANSWER}}

Đáp án đúng là: A
Công thức tính xác suất toàn phần: $P\left( A \right) = P\left( B \right).P\left( {A|B} \right) + P\left( {\overline B } \right).P\left( {A|\overline B } \right).$

========================================================================

https://khoahoc.vietjack.com/thi-online/20-cau-trac-nghiem-toan-12-canh-dieu-bai-2-cong-thuc-xac-suat-toan-phan-cong-thuc-bayes-co-dap-an


\textbf{{QUESTION}}

Cho $A,B$ là các biến cố của một phép thử $T$. Biết rằng $0 < P\left( B \right)$, xác suất để biến cố A với điều kiện biến cố B đã xảy ra được tính theo công thức nào dưới đây?
</>
A. $P\left( {A|B} \right) = \frac{{P\left( A \right)}}{{P\left( B \right)}}.$
B. $P\left( {A|B} \right) = \frac{{P\left( A \right).P\left( {B|A} \right)}}{{P\left( B \right)}}.$
C. $P\left( {A|B} \right) = \frac{{P\left( B \right).P\left( {B|A} \right)}}{{P\left( A \right)}}.$
D. $P\left( {A|B} \right) = \frac{{P\left( B \right)}}{{P\left( A \right)}}.$

\textbf{{ANSWER}}

Đáp án đúng là: B
Theo công thức xác suất toàn phần, ta có: $P\left( A \right) = P\left( B \right).P\left( {A|B} \right) + P\left( {\overline B } \right).P\left( {A|\overline B } \right).$
Do đó, công thức Bayes, còn có thể viết dưới dạng: $P\left( {A|B} \right) = \frac{{P\left( A \right).P\left( {B|A} \right)}}{{P\left( B \right)}}.$

========================================================================

https://khoahoc.vietjack.com/thi-online/20-cau-trac-nghiem-toan-12-canh-dieu-bai-2-cong-thuc-xac-suat-toan-phan-cong-thuc-bayes-co-dap-an


\textbf{{QUESTION}}

Cho A,B$A,B$ là các biến cố của một phép thử T$T$. Biết rằng P(A)>0$P\left( A \right) > 0$ và 0<P(B)<1.$0 < P\left( B \right) < 1.$ Xác suất của biến cố B với điều kiện biến cố A đã xảy ra được tính theo công thức nào?
A. P(B|A)=P(A).P(A|B)P(B).P(A|B)+P(¯B).P(A|¯B).$P\left( {B|A} \right) = \frac{{P\left( A \right).P\left( {A|B} \right)}}{{P\left( B \right).P\left( {A|B} \right) + P\left( {\overline B } \right).P\left( {A|\overline B } \right)}}.$
B. P(B|A)=P(B).P(A|B)P(A).P(B|A)+P(¯A).P(B|¯A).$P\left( {B|A} \right) = \frac{{P\left( B \right).P\left( {A|B} \right)}}{{P\left( A \right).P\left( {B|A} \right) + P\left( {\overline A } \right).P\left( {B|\overline A } \right)}}.$
C. P(B|A)=P(B).P(A|B)P(B).P(A|B)+P(¯B).P(A|¯B).$P\left( {B|A} \right) = \frac{{P\left( B \right).P\left( {A|B} \right)}}{{P\left( B \right).P\left( {A|B} \right) + P\left( {\overline B } \right).P\left( {A|\overline B } \right)}}.$
D. P(B|A)=P(A).P(A|B)P(A).P(B|A)+P(¯B).P(A|¯B).$P\left( {B|A} \right) = \frac{{P\left( A \right).P\left( {A|B} \right)}}{{P\left( A \right).P\left( {B|A} \right) + P\left( {\overline B } \right).P\left( {A|\overline B } \right)}}.$

\textbf{{ANSWER}}

Đáp án đúng là: C
Cho $A,B$ là các biến cố của một phép thử $T$. Biết rằng $P\left( A \right) > 0$ và $0 < P\left( B \right) < 1.$
Ta có công thức $P\left( {B|A} \right) = \frac{{P\left( B \right).P\left( {A|B} \right)}}{{P\left( B \right).P\left( {A|B} \right) + P\left( {\overline B } \right).P\left( {A|\overline B } \right)}}.$

========================================================================

https://khoahoc.vietjack.com/thi-online/20-cau-trac-nghiem-toan-12-canh-dieu-bai-2-cong-thuc-xac-suat-toan-phan-cong-thuc-bayes-co-dap-an


\textbf{{QUESTION}}

Nếu hai biến cố A,B$A,B$ thỏa mãn P(A)=0,3,P(B)=0,6$P\left( A \right) = 0,3,P\left( B \right) = 0,6$ và P(A|B)=0,4$P\left( {A|B} \right) = 0,4$ thì P(B|A)$P\left( {B|A} \right)$ bằng
A. 0,5.
B. 0,6.
C. 0,8.
D. 0,2.

\textbf{{ANSWER}}

Đáp án đúng là: C
Ta có: P(B|A)=P(B).P(A|B)P(A)=0,6.0,40,3=0,8.$P\left( {B|A} \right) = \frac{{P\left( B \right).P\left( {A|B} \right)}}{{P\left( A \right)}} = \frac{{0,6.0,4}}{{0,3}} = 0,8.$

========================================================================

https://khoahoc.vietjack.com/thi-online/20-cau-trac-nghiem-toan-12-canh-dieu-bai-2-cong-thuc-xac-suat-toan-phan-cong-thuc-bayes-co-dap-an


\textbf{{QUESTION}}

Cho hai biến cố A,B$A,B$ với P(B)=0,6;P(A|B)=0,7$P\left( B \right) = 0,6;{\rm{ }}P\left( {A|B} \right) = 0,7$ và P(A|¯B)=0,4.$P\left( {A|\overline B } \right) = 0,4.$ Khi đó, P(A)$P\left( A \right)$ bằng
A. 0,7.
B. 0,4.
C. 0,58.
D. 0,52.

\textbf{{ANSWER}}

Đáp án đúng là: C
Ta có: P(¯B)=1−P(B)=1−0,6=0,4.$P\left( {\overline B } \right) = 1 - P\left( B \right) = 1 - 0,6 = 0,4.$
Theo công thức xác suất toàn phần, ta có:
P(A)=P(B).P(A|B)+P(¯B).P(A|¯B)$P\left( A \right) = P\left( B \right).P\left( {A|B} \right) + P\left( {\overline B } \right).P\left( {A|\overline B } \right)$ =0,6.0,7+0,4.0,4=0,58.$ = 0,6.0,7 + 0,4.0,4 = 0,58.$

========================================================================

https://khoahoc.vietjack.com/thi-online/bai-tap-hinh-hop-chu-nhat-co-loi-giai-chi-tiet


\textbf{{QUESTION}}

Số mặt, số đỉnh, số cạnh của hình lập phương là? 
A. 4 mặt, 8 đỉnh, 12 cạnh. 
B. 6 mặt, 8 đỉnh, 12 cạnh. 
C. 6 mặt, 12 đỉnh, 8 cạnh. 
D. 8 mặt, 6 đỉnh, 12 cạnh.

\textbf{{ANSWER}}

Hình lập phương cũng được gọi là hình hộp chữ nhật có 6 mặt, 8 đỉnh, 12 cạnh.
Chọn đáp án B.

========================================================================

https://khoahoc.vietjack.com/thi-online/bai-tap-hinh-hop-chu-nhat-co-loi-giai-chi-tiet


\textbf{{QUESTION}}

Hình hộp chữ nhật có số cặp mặt song song là? 
A. 2   
B. 3 
C. 4   
D. 5

\textbf{{ANSWER}}

Hình hộp chữ nhật có 3 cặp mặt song song.
Chọn đáp án B.

========================================================================

https://khoahoc.vietjack.com/thi-online/sach-bai-tap-toan-7-tap-2/23610


\textbf{{QUESTION}}

Cho đa thức f(x) = x2 – 4x – 5. Chứng tỏ rằng x = -1; x = 5 là hai nghiệm của đa thức đó.

\textbf{{ANSWER}}

Thay x = -1; x = 5 vào đa thức f(x) = x2 – 4x – 5, ta có:
f(-1) = (-1)2 – 4.(-1) – 5 = 1 + 4 – 5 = 0
f(5) = 52 – 4.5 – 5 = 25 – 20 – 5 = 0
Vậy x = -1 và x = 5 là các nghiệm của đa thức f(x) = x2 – 4x – 5

========================================================================

https://khoahoc.vietjack.com/thi-online/sach-bai-tap-toan-7-tap-2/23610


\textbf{{QUESTION}}

Tìm nghiệm của các đa thức sau: 2x + 10

\textbf{{ANSWER}}

Ta có: 2x + 10 = 0 ⇔ 2x = -10 ⇔ x = -10 : 2 ⇔ x = -5
Vậy x = -5 là nghiệm của đa thức 2x + 10

========================================================================

https://khoahoc.vietjack.com/thi-online/sach-bai-tap-toan-7-tap-2/23610


\textbf{{QUESTION}}

Tìm nghiệm của các đa thức sau: 3x - 1/2

\textbf{{ANSWER}}

Ta có: 3x - 1/2 = 0 ⇔ 3x = 1/2 ⇔ x = 1/2 : 3 = 1/6
Vậy x = 1/6 là nghiệm của đa thức 3x - 1/2

========================================================================

https://khoahoc.vietjack.com/thi-online/sach-bai-tap-toan-7-tap-2/23610


\textbf{{QUESTION}}

Tìm nghiệm của các đa thức sau: x2 – x

\textbf{{ANSWER}}

Ta có: x2 – x = 0 ⇔ x(x – 1) = 0 ⇔ x = 0 hoặc x – 1 = 0
⇔ x = 0 hoặc x = 1
Vậy x = 0 và x = 1 là các nghiệm của đa thức x2 – x

========================================================================

https://khoahoc.vietjack.com/thi-online/sach-bai-tap-toan-7-tap-2/23610


\textbf{{QUESTION}}

Tìm nghiệm của các đa thức sau: (x – 2)(x + 2)

\textbf{{ANSWER}}

Ta có: (x – 2)(x + 2) = 0 ⇔ x – 2 = 0 hoặc x + 2 = 0
x – 2 = 0 ⇔ x = 2
x + 2 = 0 ⇔ x = -2
Vậy x = 2 và x = -2 là các nghiệm của đa thức (x – 2)(x + 2)

========================================================================

https://khoahoc.vietjack.com/thi-online/250-cau-trac-nghiem-ung-dung-dao-ham-de-khao-sat-ham-so/1056


\textbf{{QUESTION}}

Biết hàm số f(x) xác định trên R và có đạo hàm f’(x) = (x – 1)x2(x + 1)3(x + 2)4. Hỏi hàm số có bao nhiêu điểm cực trị?
A. 4.
B. 1
C. 2
D. 3

\textbf{{ANSWER}}

Đáp án C.
f’(x) = (x – 1)x2(x + 1)3(x + 2)4
Ta thấy phương trình f’(x) = 0 có 2 nghiệm đơn là 1; -1 và có hai nghiệm kép là 0; -2
Từ đó số điểm cực trị là 2.

========================================================================

https://khoahoc.vietjack.com/thi-online/bai-tap-theo-tuan-toan-9-tuan-32


\textbf{{QUESTION}}

Cho một hình nón, biết diện tích xung quanh là $$ 60\pi c{m}^{2}.$$ Độ dài đường sinh là 10cm. Tính diện tích toàn phần và thể tích

\textbf{{ANSWER}}

$$ {S}_{xq}=60\pi \left(c{m}^{2}\right)\Leftrightarrow 2\pi Rd=60\Leftrightarrow 2.R.10=60\Leftrightarrow R=3$$
$$ \begin{array}{l}{S}_{tp}={S}_{xq}+\pi {r}^{2}=60\pi +9\pi =69\pi \left(c{m}^{2}\right)\\ {V}_{non}=\frac{1}{3}\pi {R}^{2}h=\frac{1}{3}\pi {.3}^{2}.\sqrt{91}=3\sqrt{91}\pi \left(c{m}^{3}\right)\end{array}$$

========================================================================

https://khoahoc.vietjack.com/thi-online/bai-tap-theo-tuan-toan-9-tuan-32


\textbf{{QUESTION}}

Cho tam giác ABC vuông tại A, ∠ACB=300,BC=2cm.$$ \angle ACB={30}^{0},BC=2cm.$$ Quay tam giác một vòng quanh AB. Tìm diện tích xung quanh và thể tích hình tạo thành.

\textbf{{ANSWER}}

$$ \begin{array}{l}{S}_{xq}=AC.2\pi .BC=BC.\mathrm{cos}C.2\pi .BC=2.\mathrm{cos}{30}^{0}.2\pi .2=4\pi \sqrt{3}\left(c{m}^{2}\right)\\ AB=\sqrt{B{C}^{2}-A{C}^{2}}=\sqrt{{2}^{2}-{\left(2.\mathrm{cos}{30}^{0}\right)}^{2}}=1\\ \Rightarrow V=\frac{1}{3}\pi {R}^{2}h=\frac{1}{3}.\pi .A{C}^{2}.AB=\frac{1}{3}\pi {\left(2.\mathrm{cos}{30}^{0}\right)}^{2}.1=\frac{\sqrt{3}}{3}\pi \left(c{m}^{3}\right)\end{array}$$

========================================================================

https://khoahoc.vietjack.com/thi-online/bai-tap-theo-tuan-toan-9-tuan-32


\textbf{{QUESTION}}

Một xe đạp đi từ A đến B cách nhau 108km. Cùng lúc đó, một ô tô khởi hành từ B để về A có vận tốc lớn hơn vận tốc xe đạp là 18km/h. Sau khi hai xe gặp nhau, xe đạp phải mất 4 giờ nữa mới đến B. Tính vận tốc mỗi xe.

\textbf{{ANSWER}}

Gọi x là vận tốc xe đạp nên vận tốc xe máy là x + 18
Nên thời gian đi của xe đạp là : 108x$$ \frac{108}{x}$$. Theo bài ta có phương trình:
108x+x+18+4=108x⇒8x2−36x−1944=0⇔[x=18(tm)x=−272$$ \frac{108}{x+x+18}+4=\frac{108}{x}\Rightarrow 8{x}^{2}-36x-1944=0\Leftrightarrow \left[\begin{array}{l}x=18\left(tm\right)\\ x=-\frac{27}{2}\end{array}\right.$$
 
Vậy vận tốc xe đạp : 18km/h, vận tốc xe máy: 36km/h

========================================================================

https://khoahoc.vietjack.com/thi-online/bai-tap-theo-tuan-toan-9-tuan-32


\textbf{{QUESTION}}

Người ta trồng 35 cây dừa trên một thửa đất hình chữ nhật có chiều rộng 20m, chiều dài 30m thành từng hàng song song cách đều nhau theo cả hai chiều . Hãy tính khoảng cách hai hàng liên tiếp.

\textbf{{ANSWER}}

Gọi x là số hàng cây được trồng theo hàng ngang
 ⇒1-x$$ \Rightarrow 1-x$$ là khoảng cách giữa các hàng, 
y là số hàng dọc, 1 - y là khoảng cách giữa các hàng dọc
Ta có: xy = 35 (1), từ đó ta có phương trình: 20x−1=30y−1⇒30x−20y=−10 (2)$$ \frac{20}{x-1}=\frac{30}{y-1}\Rightarrow 30x-20y=-10\quad \left(2\right)$$
Từ (1) và (2) ta có hệ: {xy=3530x−20y=−10⇔{y=7x=5$$ \left\{\begin{array}{l}xy=35\\ 30x-20y=-10\end{array}\right.\Leftrightarrow \left\{\begin{array}{l}y=7\\ x=5\end{array}\right.$$
Vậy theo hàng ngang khoảng cách giữa các hàng là 205−1=5(m)$$ \frac{20}{5-1}=5\left(m\right)$$
Theo hàng dọc khoảng cách giữa các cột là 307−1=5(m)$$ \frac{30}{7-1}=5\left(m\right)$$

========================================================================

https://khoahoc.vietjack.com/thi-online/bai-tap-theo-tuan-toan-9-tuan-32


\textbf{{QUESTION}}

Hai người cùng làm chung một công việc trong 24 giờ thì xong. Năng suất người thứ nhất bằng 32$$ \frac{3}{2}$$ năng suất người thứ hai. Hỏi nếu mỗi người làm công việc đó một mình thì hoàn thành công việc sau bao lâu?

\textbf{{ANSWER}}

Gọi x (giờ) là thời gian xong của người I
y (giờ) là thời gian xong của người II. Theo bài ta có phương trình:
{y=32x1x+1y=124⇔{x=40y=60(tm)$$ \left\{\begin{array}{l}y=\frac{3}{2}x\\ \frac{1}{x}+\frac{1}{y}=\frac{1}{24}\end{array}\right.\Leftrightarrow \left\{\begin{array}{l}x=40\\ y=60\end{array}\right.\left(tm\right)$$
 
Vậy người I làm trong 40 giờ, người II 60 giờ.

========================================================================

https://khoahoc.vietjack.com/thi-online/bo-5-de-kiem-tra-hoc-ki-1-chuyen-de-toan-11-kiem-tra-45-phut-co-dap-an/106572


\textbf{{QUESTION}}

A. d=2
B. d=-2
C. d=3
D. d=4

\textbf{{ANSWER}}

Đáp án A

========================================================================

https://khoahoc.vietjack.com/thi-online/bo-5-de-kiem-tra-hoc-ki-1-chuyen-de-toan-11-kiem-tra-45-phut-co-dap-an/106572


\textbf{{QUESTION}}

Dãy số (an)$$ \left({a}_{n}\right)$$  được xác định bởi an=4n−1n$$ {a}_{n}=\frac{4n-1}{n}$$ . Giá trị a19$$ {a}_{19}$$  bằng
A. 2
B. 2519$$ \frac{25}{19}$$
C. 7519$$ \frac{75}{19}$$
D. 1519$$ \frac{15}{19}$$

\textbf{{ANSWER}}

Đáp án C

========================================================================

https://khoahoc.vietjack.com/thi-online/bo-5-de-kiem-tra-hoc-ki-1-chuyen-de-toan-11-kiem-tra-45-phut-co-dap-an/106572


\textbf{{QUESTION}}

Phát biểu nào dưới đây về dãy số (an)$$ \left({a}_{n}\right)$$  được cho bởi an=2n+n$$ {a}_{n}={2}^{n}+n$$  là đúng?
B. Dãy số (an)$$ \left({a}_{n}\right)$$ là dãy số tăng.

\textbf{{ANSWER}}

Đáp án B

========================================================================

https://khoahoc.vietjack.com/thi-online/bo-5-de-kiem-tra-hoc-ki-1-chuyen-de-toan-11-kiem-tra-45-phut-co-dap-an/106572


\textbf{{QUESTION}}

Hệ thức truy hồi nào dưới đây cũng là một công thức xác định dãy số (un)$$ \left({u}_{n}\right)$$  cho bởi công thức un=2n+1$$ {u}_{n}=2n+1$$ ?
B. {u1=3un=un−1+2$$ \left\{\begin{array}{l}{u}_{1}=3\\ {u}_{n}={u}_{n-1}+2\end{array}\right.$$với n≥2$$ n\ge 2$$

\textbf{{ANSWER}}

Đáp án B

========================================================================

https://khoahoc.vietjack.com/thi-online/bo-5-de-kiem-tra-hoc-ki-1-chuyen-de-toan-11-kiem-tra-45-phut-co-dap-an/106572


\textbf{{QUESTION}}

An đi đoạn đường từ A đến B dài 54 km. Biết giờ đầu tiên An đi được 15 km và mỗi giờ sau An đi kém hơn giờ trước 1 km. Thời gian An đi hết quãng đường AB là

\textbf{{ANSWER}}

Đáp án B

========================================================================

https://khoahoc.vietjack.com/thi-online/bo-de-thi-thpt-quoc-gia-chuan-cau-truc-bo-giao-duc-mon-toan-2019


\textbf{{QUESTION}}

Cho hình chóp S.ABCD có đáy là hình vuông cạnh a, SA⊥ABCD  và SA = a. Thể tích của khối chóp S.ABCD bằng
A. $$ {a}^{3}$$
B. $$ 2{a}^{3}$$
C. $$ 3{a}^{3}$$
D. $$ \frac{1}{3}{a}^{3}$$

\textbf{{ANSWER}}

Có $$ V=\frac{1}{3}{S}_{ABCD}.SA=\frac{1}{3}{a}^{2}.a=\frac{{a}^{3}}{3}$$
Chọn đáp án D.

========================================================================

https://khoahoc.vietjack.com/thi-online/bo-de-thi-thpt-quoc-gia-chuan-cau-truc-bo-giao-duc-mon-toan-2019


\textbf{{QUESTION}}

Trong các mệnh đề sau, mệnh đề nào đúng ?
A. Mọi hình chóp có đáy là hình thoi luôn có mặt cầu ngoại tiếp
B. Mọi hình chóp có đáy là hình thang vuông luôn có mặt cầu ngoại tiếp.
C. Mọi hình chóp có đáy là hình bình hành luôn có mặt cầu ngoại tiếp
D. Mọi hình chóp có đáy là hình thang cân luôn có mặt cầu ngoại tiếp.

\textbf{{ANSWER}}

Hình chóp có mặt cầu ngoại tiếp khi và chỉ khi đáy là một đa giác nội tiếp. Chính vì vậy chọn đáp án D. Vì đáy là hình thang cân nội tiếp đường tròn.
Chọn đáp án D.

========================================================================

https://khoahoc.vietjack.com/thi-online/15-cau-trac-nghiem-toan-7-chan-troi-sang-tao-bai-1-tap-hop-cac-so-huu-ti-co-dap-an


\textbf{{QUESTION}}

Số đối của số hữu tỉ 0 là số:
A. 0;
B. −1;
C.$$ \frac{0}{1000}$$;
D. Đáp án A và C đều đúng.

\textbf{{ANSWER}}

Đáp án đúng là: D
+ Số đối của số hữu tỉ 0 là số 0.
Do đó, đáp án A đúng.
+ Ta có $\frac{0}{{1000}} = 0$.
Do đó, đáp án C đúng.
Vậy chọn đáp án D.

========================================================================

https://khoahoc.vietjack.com/thi-online/15-cau-trac-nghiem-toan-7-chan-troi-sang-tao-bai-1-tap-hop-cac-so-huu-ti-co-dap-an


\textbf{{QUESTION}}

Số hữu tỉ là số được viết dưới dạng phân số ab$\frac{a}{b}$ với:
A. a = 0; b ≠ 0;
B. a, b ∈Z$ \in \mathbb{Z}$, b ≠ 0;
C. a, b ∈N$ \in \mathbb{N}$;
D. a, b ∈N$ \in \mathbb{N}$, b ≠ 0.

\textbf{{ANSWER}}

Đáp án đúng là: B
Số hữu tỉ là số được viết dưới dạng phân số $\frac{a}{b}$với  a, b $ \in \mathbb{Z}$, b ≠ 0.

========================================================================

https://khoahoc.vietjack.com/thi-online/15-cau-trac-nghiem-toan-7-chan-troi-sang-tao-bai-1-tap-hop-cac-so-huu-ti-co-dap-an


\textbf{{QUESTION}}

Cho a, b ∈Z$ \in \mathbb{Z}$, b ≠ 0, x = ab$\frac{a}{b}$. Nếu a, b khác dấu thì:
A. x = 0;
B. x > 0;
C. x < 0;
D. Cả B, C đều sai.

\textbf{{ANSWER}}

Đáp án đúng là: C
Ta có x = ab$\frac{a}{b}$; a, b ∈Z$ \in \mathbb{Z}$, b ≠ 0;  a, b khác dấu thì x < 0.
Vì số hữu tỉ ab$\frac{a}{b}$ là phép chia số a cho số b mà hai số nguyên a, b khác dấu nên khi chia cho nhau luôn ra số âm suy ra x < 0).

========================================================================

https://khoahoc.vietjack.com/thi-online/15-cau-trac-nghiem-toan-7-chan-troi-sang-tao-bai-1-tap-hop-cac-so-huu-ti-co-dap-an


\textbf{{QUESTION}}

Số hữu tỉ x nhỏ hơn số hữu tỉ y nếu trên trục số:
A. Điểm x ở bên trái điểm y;
B. Điểm x ở bên phải điểm y;
C. Điểm x và điểm y khác phía đối với điểm 0;
D. Cả 3 đáp án đều sai.

\textbf{{ANSWER}}

Đáp án đúng là: A
Với hai số hữu tỉ x, y bất kì, số hữu tỉ x nhỏ hơn số hữu tỉ y nếu trên trục số điểm x ở bên trái điểm y.

========================================================================

https://khoahoc.vietjack.com/thi-online/16-cau-trac-nghiem-toan-7-chan-troi-sang-tao-bai2-dien-tich-xung-quanh-va-the-tich-cua-hinh-hop-chu/105051


\textbf{{QUESTION}}

Một bể cá dạng hình hộp chữ nhật bằng kính (không nắp) có chiều dài 70 cm, chiều rộng 40 cm, chiều cao 50 cm. Mực nước trong bể cao 25 cm. Người ta cho vào bể một hòn đá làm cho thể tích nước tăng 10 000 cm3. Mực nước trong bể lúc này cao khoảng (làm tròn kết quả đến hàng đơn vị):

\textbf{{ANSWER}}

Hướng dẫn giải
Đáp án đúng là: A
Thể tích của nước có trong bể lúc đầu là:
70.40.25 = 70 000 (cm3)
Khi cho vào một hòn đá thể tích tăng 10 000 cm3. Vậy thể tích phần bể chứa nước lúc này là:
70 000 + 10 000 = 80 000 (cm3)
Vì chiều dài và chiều rộng của bể nước không thay đổi nên sự thay đổi thể tích là do chiều cao của mực nước thay đổi.
Vậy chiều cao mực nước lúc sau là:
$\frac{{80000}}{{70.40}} = \frac{{200}}{7}$ (cm) ≈ 29 (cm).
Ta chọn đáp án A.

========================================================================

https://khoahoc.vietjack.com/thi-online/16-cau-trac-nghiem-toan-7-chan-troi-sang-tao-bai2-dien-tich-xung-quanh-va-the-tich-cua-hinh-hop-chu/105051


\textbf{{QUESTION}}

Một người thuê sơn mặt ngoài của một cái thùng sắt không nắp dạng hình lập phương có cạnh 1,2 m. Biết giá tiền mỗi mét vuông là 25 000 đồng. Người ấy phải trả số tiền là

\textbf{{ANSWER}}

Hướng dẫn giải
Đáp án đúng là: C
Thùng sắt có dạng hình lập phương không có nắp nên thùng có 5 mặt là 5 hình vuông giống nhau.
Diện tích một mặt của thùng sắt là:
S = 1,22 = 1,44 (m2).
Diện tích tất cả các mặt ngoài thùng sắt là: 
5S = 1,44.5 = 7,2 (m2).
Số tiền người ấy phải trả là:
7,2. 25 000 = 180 000 (đồng).
Ta chọn đáp án C.

========================================================================

https://khoahoc.vietjack.com/thi-online/12-bai-tap-dang-toan-chuyen-dong-co-loi-giai


\textbf{{QUESTION}}

Một ca nô chạy trên sông trong 8 giờ xuôi dòng được 81 km và ngược dòng 105 km. Một lần khác, ca nô chạy trên sông trong 4 giờ xuôi dòng 54 km và ngược dòng 42 km. Vận tốc riêng của ca nô là (Biết vận tốc riêng của ca nô và vận tốc dòng nước không đổi).
A. 3 km/h.
B. 24 km/h.
C. 27 km/h.
D. 21 km/h.

\textbf{{ANSWER}}

Đáp án đúng là: B
Gọi x, y lần lượt là vận tốc riêng của ca nô và vận tốc của dòng nước (x, y > 0, km/h).
Do đó, vận tốc xuôi dòng của ca nô là x + y (km/h) và vận tốc ngược dòng của ca nô là x – y (km/h).
Ca nô chạy trên sông trong 8 giờ xuôi dòng được 81 km và ngược dòng 105 km nên ta có phương trình $\frac{{81}}{{x + y}} + \frac{{105}}{{x - y}} = 8$ (1)
Ca nô chạy trên sông trong 4 giờ xuôi dòng được 54 km và ngược dòng 42 km nên ta có phương trình: $\frac{{54}}{{x + y}} + \frac{{42}}{{x - y}} = 4$ (2).
Từ (1) và (2) ta có hệ phương trình: $\left\{ \begin{array}{l}\frac{{81}}{{x + y}} + \frac{{105}}{{x - y}} = 8\\\frac{{54}}{{x + y}} + \frac{{42}}{{x - y}} = 4\end{array} \right.$
Đặt $\left\{ \begin{array}{l}\frac{1}{{x + y}} = a\\\frac{1}{{x - y}} = b\end{array} \right.$ ta có: $\left\{ \begin{array}{l}81a + 105b = 8\\54a + 42b = 4\end{array} \right.$ .
Giải hệ phương trình $\left\{ \begin{array}{l}81a + 105b = 8\\54a + 42b = 4\end{array} \right.$, ta có:
$\left\{ \begin{array}{l}2\left( {81a + 105b} \right) = 2.8\\3\left( {54a + 42b} \right) = 4.3\end{array} \right.$ ta được $\left\{ \begin{array}{l}162a + 210b = 16\\162a + 126b = 12\end{array} \right.$.
Thực hiện trừ theo vế hai phương trình của hệ, ta được: 84b = 4 hay b = $\frac{1}{{21}}$.
Với b = $\frac{1}{{21}}$ suy ra a = $\frac{1}{{27}}$.
Từ đây suy ra $\left\{ \begin{array}{l}\frac{1}{{x + y}} = \frac{1}{{27}}\\\frac{1}{{x - y}} = \frac{1}{{21}}\end{array} \right.$ hay $\left\{ \begin{array}{l}x + y = 27\\x - y = 21\end{array} \right.$ suy ra x = 24, y = 3 (thỏa mãn).
Vậy vận tốc riêng của ca nô là 24 km/h.

========================================================================

https://khoahoc.vietjack.com/thi-online/12-bai-tap-dang-toan-chuyen-dong-co-loi-giai


\textbf{{QUESTION}}

Một ô tô đi từ A đến B với vận tốc dự định trong một thời gian dự định. Nếu ô tô tăng vận tốc thêm 3 km/h thì thời gian rút ngắn được 2 giờ so với dự định. Nếu ô tô giảm vận tốc đi 3 km/h thì thời gian tăng hơn 3 giờ so với dự định. Độ dài quãng đường AB là
A. 12 km.
B. 15 km.
C. 18 km
D. 180 km.

\textbf{{ANSWER}}

Đáp án đúng là: D
Gội vận tốc ban đầu là x ( x > 3, km/h), thời gian chạy dự định là y (y > 2, h).
Độ dài của quãng đường AB là xy (km).
Nếu ô tô tăng vận tốc 3 km/h thì rút ngắn 2 giờ so với dự định nên ta có phương trình:
(x + 3)(y – 2) = xy (1)
Nếu ô tô giảm vận tốc 3 km/h thì thời gian tăng 3 giờ so với dự định nên ta có phương trình: (x – 3)(y + 3) = xy (2)
Từ (1) và (2) ta có hệ phương trình: {(x+3)(y−2)=xy(x−3)(y+3)=xy$\left\{ \begin{array}{l}(x + 3)(y - 2) = xy\\(x - 3)(y + 3) = xy\end{array} \right.$ hay {3y−2x=63x−3y=9$\left\{ \begin{array}{l}3y - 2x = 6\\3x - 3y = 9\end{array} \right.$.
• Giải hệ phương trình {3y−2x=63x−3y=9$\left\{ \begin{array}{l}3y - 2x = 6\\3x - 3y = 9\end{array} \right.$ .
Cộng theo vế hai phương trình của hệ, ta có: x = 15 (thỏa mãn).
Với x = 15 thì y = 12 (thỏa mãn).
Vậy độ dài quãng đường AB là: 15.12 = 180 (km).

========================================================================

https://khoahoc.vietjack.com/thi-online/12-bai-tap-dang-toan-chuyen-dong-co-loi-giai


\textbf{{QUESTION}}

Một xe máy đi từ A đến B trong thời gian đã định. Nếu đi với vận tốc 45 km/h sẽ tới B chậm mất nửa giờ. Nếu đi với vận tốc 60 km/h thì sẽ đến B sớm 45 phút. Tính quãng đường AB và thời gian dự định.
A. 4,5 km.
B. 225 km.
C. 22,5 km.
D. 27,5 km.

\textbf{{ANSWER}}

Đáp án đúng là: B
Đổi 45 phút = 34$\frac{3}{4}$ giờ.
Gọi quãng đường AB là x (x > 0, km) và thời gian dự định đi từ A đến B là y (y > 0, giờ).
Nếu đi với vận tốc 45 km/h sẽ tới B chậm nửa giờ, do đó ta có phương trình:
x = 45.(y+12)$\left( {y + \frac{1}{2}} \right)$  hay x – 45y = 452$\frac{{45}}{2}$ (1)
Nếu đi với vận tốc 60 km/h sẽ tới B sớm hơn 45 phút, do đó ta có phương trình:
x = 60.(y−34)$\left( {y - \frac{3}{4}} \right)$ hay x – 60y = −45 (2)
Từ (1) và (2) ta có hệ phương trình {x−45y=452x−60y=−45$\left\{ \begin{array}{l}x - 45y = \frac{{45}}{2}\\x - 60y =  - 45\end{array} \right.$.
Giải hệ phương trình, ta trừ theo vế hai phương trình của hệ, ta được: 15y = 1352$\frac{{135}}{2}$, suy ra y = 4,5 (thỏa mãn).
Thay y = 4,5 vào phương trình x – 45y = 452$\frac{{45}}{2}$ được x = 225 (thỏa mãn).
Vậy quãng đường AB dài 225 km và thời gian dự định đi từ A đến B hết 4,5 giờ.

========================================================================

https://khoahoc.vietjack.com/thi-online/12-bai-tap-dang-toan-chuyen-dong-co-loi-giai


\textbf{{QUESTION}}

Một ô tô và một xe máy ở hai địa điểm A và B cách nhau 180 km, khởi hành cùng một lúc đi ngược chiều nhau và gặp nhau sau 2 giờ. Biết vận tốc của ô tô lớn hơn vận tốc của xe máy là 10 km/h. Vận tốc của ô tô là:
A. 50 km/h.
B. 40 km/h.
C. 50 km.
D. 90 km/h.

\textbf{{ANSWER}}

Đáp án đúng là: A
Gọi x, y lần lượt là vận tốc của ô tô và xe máy (x > y > 0, km/h).
Theo đề, hai xe đi ngược chiều và gặp nhau sau 2 giờ nên ta có: 2x + 2y = 180 hay 
x + y = 90 (1).
Vận tốc ô tô lớn hơn vận tốc xe máy 10 km/h nên ta có: x – y = 10 (2).
Từ (1) và (2) ta có hệ phương trình {x+y=90x−y=10$\left\{ \begin{array}{l}x + y = 90\\x - y = 10\end{array} \right.$.
• Giải hệ phương trình {x+y=90x−y=10$\left\{ \begin{array}{l}x + y = 90\\x - y = 10\end{array} \right.$.
Cộng theo vế hai phương trình của hệ, ta được 2x = 100 hay x = 50 (thỏa mãn).
Với x = 50 thì y = 40 (thỏa mãn).
Vậy vận tốc của ô tô là 50 km/h.

========================================================================

https://khoahoc.vietjack.com/thi-online/12-bai-tap-dang-toan-chuyen-dong-co-loi-giai


\textbf{{QUESTION}}

Một khách du lịch đi trên ô tô 4 giờ, sau đó đi tiếp bằng tàu hỏa trong 7 giờ được quãng đường 640 km. Biết rằng mỗi giờ tàu hỏa đi nhanh hơn ô tô 5 km, hỏi vận tốc của tàu hỏa là bảo nhiêu?
A. 40 km/h.
B. 50 km/h.
C. 60 km/h.
D. 65 km/h.

\textbf{{ANSWER}}

Đáp án đúng là: C
Gọi vận tốc của tàu hỏa và ô tô lần lượt là x, y (km/h, x > y > 0; x > 5).
Vì khách du lịch đi trên ô tô 4 giờ, sau đó đi tiếp bằng tàu hỏa trong 7 giờ được 640 km nên ta có phương trình: 7x + 4y = 640 (1).
Mỗi giờ tàu hỏa đi nhanh hơn ô tô 5 km nên ta có phương trình x – y = 5 (2).
Từ (1) và (2) ta có hệ phương trình: {7x+4y=640x−y=5$\left\{ \begin{array}{l}7x + 4y = 640\\x - y = 5\end{array} \right.$.
Từ phương trình x – y = 5 ta được x = 5 + y.
Thay x = 5 + y vào phương trình 7x + 4y = 640, ta được:
7(5 + y) + 4y = 640
11y + 35 = 640
11y = 605
y = 55 (thỏa mãn).
Suy ra x = 5 + 55 = 60 (thỏa mãn).
Vậy vận tốc của tàu hỏa là 60 km/h.

========================================================================

https://khoahoc.vietjack.com/thi-online/de-thi-hoc-ki-2-toan-9-co-dap-an-nam-2022-2023/116556


\textbf{{QUESTION}}

A. (-1;-1)
B. (-1;1)
C. (1;1)
D. (1;-1)

\textbf{{ANSWER}}

Chọn D

========================================================================

https://khoahoc.vietjack.com/thi-online/de-thi-hoc-ki-2-toan-9-co-dap-an-nam-2022-2023/116556


\textbf{{QUESTION}}

A. -1
B. 1
C. -3
D. 3

\textbf{{ANSWER}}

Chọn D

========================================================================

https://khoahoc.vietjack.com/thi-online/de-thi-hoc-ki-2-toan-9-co-dap-an-nam-2022-2023/116556


\textbf{{QUESTION}}

Phương trình nào dưới đây có thể kết hợp với phương trình x + y = 1 để được một hệ phương trình có nghiệm duy nhất .
A. x + y = -1
 
B. 0x + y = 1
C. 2y = 2 – 2x

\textbf{{ANSWER}}

Chọn B

========================================================================

https://khoahoc.vietjack.com/thi-online/de-thi-hoc-ki-2-toan-9-co-dap-an-nam-2022-2023/116556


\textbf{{QUESTION}}

A. (0;−12)$$ \left(0;-\frac{1}{2}\right)$$
B. (2;−12)$$ \left(2;-\frac{1}{2}\right)$$
C. (0;12)$$ \left(0;\frac{1}{2}\right)$$
D. (1;0)

\textbf{{ANSWER}}

Chọn C

========================================================================

https://khoahoc.vietjack.com/thi-online/de-thi-hoc-ki-2-toan-9-co-dap-an-nam-2022-2023/116556


\textbf{{QUESTION}}

A. -2
B. 2
C. 1
D. 12$$ \frac{1}{2}$$

\textbf{{ANSWER}}

Chọn D

========================================================================

https://khoahoc.vietjack.com/thi-online/trac-nghiem-bai-tap-theo-tuan-toan-7-tuan-6-co-dap-an


\textbf{{QUESTION}}

Các tỉ số sau đây có lập thành tỉ lệ thức không?

\textbf{{ANSWER}}

$$ \frac{15}{21}=\frac{5}{7}$$; $$ \frac{30}{42}=\frac{5}{7}\Rightarrow \frac{15}{21}=\frac{30}{42}$$. Vậy tỉ số có lập được thành tỉ lệ thức.

========================================================================

https://khoahoc.vietjack.com/thi-online/trac-nghiem-bai-tap-theo-tuan-toan-7-tuan-6-co-dap-an


\textbf{{QUESTION}}

Các tỉ số sau đây có lập thành tỉ lệ thức không?

\textbf{{ANSWER}}

$$ \frac{4}{5}:8=\frac{1}{10}$$; $$ \frac{3}{5}:6=\frac{1}{10}\Rightarrow \frac{4}{5}:8=\frac{3}{5}:6$$ . Vậy tỉ số có lập được thành tỉ lệ thức.

========================================================================

https://khoahoc.vietjack.com/thi-online/trac-nghiem-bai-tap-theo-tuan-toan-7-tuan-6-co-dap-an


\textbf{{QUESTION}}

Các tỉ số sau đây có lập thành tỉ lệ thức không?
213:7$$ 2\frac{1}{3}:7$$ và 314:13$$ 3\frac{1}{4}:13$$

\textbf{{ANSWER}}

213:7=13$$ 2\frac{1}{3}:7=\frac{1}{3}$$; 314:13=14⇒13≠14⇒$$ 3\frac{1}{4}:13=\frac{1}{4}\Rightarrow \frac{1}{3}\ne \frac{1}{4}\Rightarrow $$ không lập được tỉ lệ thức

========================================================================

https://khoahoc.vietjack.com/thi-online/trac-nghiem-bai-tap-theo-tuan-toan-7-tuan-6-co-dap-an


\textbf{{QUESTION}}

Tìm x,$$ x,$$ biết:
x:8=7:4$$ x:8=7:4$$

\textbf{{ANSWER}}

x:8=7:4⇒x=8.74=14$$ x:8=7:4\Rightarrow x=\frac{8.7}{4}=14$$

========================================================================

https://khoahoc.vietjack.com/thi-online/trac-nghiem-bai-tap-theo-tuan-toan-7-tuan-6-co-dap-an


\textbf{{QUESTION}}

Tìm x,$$ x,$$ biết:
2,5:7,5=x:79$$ \mathrm{2,5}:\mathrm{7,5}=x:\frac{7}{9}$$

\textbf{{ANSWER}}

2,5:7,5=x:79⇒x=(2,5⋅79):7,5=727$$ \mathrm{2,5}:\mathrm{7,5}=x:\frac{7}{9}\Rightarrow x=\left(\mathrm{2,5}\cdot \frac{7}{9}\right):\mathrm{7,5}=\frac{7}{27}$$

========================================================================

https://khoahoc.vietjack.com/thi-online/6-cau-trac-nghiem-toan-6-canh-dieu-bai-3-phep-cong-cac-so-nguyen-co-dap-an-van-dung


\textbf{{QUESTION}}

A.– 2 + (– 15)
B.– 2 + 19
C.2 + (– 19)
D.– 5 + (– 12)

\textbf{{ANSWER}}

Trong 4 đáp án trên, đáp án B và C là tổng hai số nguyên khác dấu
– 2 + 19 = 19 – 2 = 17
2 + (– 19) = – (19 – 2) = – 17
Chọn đáp án C.

========================================================================

https://khoahoc.vietjack.com/thi-online/6-cau-trac-nghiem-toan-6-canh-dieu-bai-3-phep-cong-cac-so-nguyen-co-dap-an-van-dung


\textbf{{QUESTION}}

A.12°C
B.2°C
C.– 2°C
D.– 12°C

\textbf{{ANSWER}}

Nhiệt độ giảm 7°C nghĩa là tăng – 7°C .
Vậy nhiệt độ của phòng đông lạnh lúc sau là:
5 + (– 7) = –  (7 – 5) = – 2°C
Chọn đáp án C.

========================================================================

https://khoahoc.vietjack.com/thi-online/6-cau-trac-nghiem-toan-6-canh-dieu-bai-3-phep-cong-cac-so-nguyen-co-dap-an-van-dung


\textbf{{QUESTION}}

A.2
B.4
C.6
D.8

\textbf{{ANSWER}}

Ta có:
38 + (– 2*) = 16
Hay (38 – 2*) = 16
Mà 3 – 2 = 1, do đó 8 – * = 6
Khi đó: * = 8 – 6 = 2
Thử lại: 38 + (– 22) = 38 – 22 = 16 (đúng).
Chọn đáp án A.

========================================================================

https://khoahoc.vietjack.com/thi-online/6-cau-trac-nghiem-toan-6-canh-dieu-bai-3-phep-cong-cac-so-nguyen-co-dap-an-van-dung


\textbf{{QUESTION}}

A.x = – 452     
B.x = – 254     
C.x = 542     
D.x = 254

\textbf{{ANSWER}}

Ta có: x – 589 = (– 335)
x = (– 335) + 589
x = 589 – 335
x = 254
Vậy x = 254.
Chọn đáp án D.

========================================================================

https://khoahoc.vietjack.com/thi-online/6-cau-trac-nghiem-toan-6-canh-dieu-bai-3-phep-cong-cac-so-nguyen-co-dap-an-van-dung


\textbf{{QUESTION}}

A.1
B.5
C.4
D.3

\textbf{{ANSWER}}

Các số nguyên thỏa mãn – 4 < x < 5 là: – 3; – 2; – 1; 0; 1; 2; 3; 4.
Ta có:
(– 3) + (– 2) + (– 1) + 0 + 1 + 2 + 3 + 4 
= (– 3 + 3) + (– 2 + 2) + (– 1 + 1) + 0 + 4
= 0 + 0 + 0 + 0 + 4 = 4
Chọn đáp án C.

========================================================================

https://khoahoc.vietjack.com/thi-online/10-bai-tap-phan-tich-da-thuc-thanh-nhan-tu-bang-cach-dat-nhan-tu-chung-co-loi-giai


\textbf{{QUESTION}}

Phân tích đa thức mx + nxy – x2 thành nhân tử, ta được

\textbf{{ANSWER}}

Đáp án đúng là: A
mx + nxy – x2 = mx + nxy – x . x = x(m + ny – x).

========================================================================

https://khoahoc.vietjack.com/thi-online/10-bai-tap-phan-tich-da-thuc-thanh-nhan-tu-bang-cach-dat-nhan-tu-chung-co-loi-giai


\textbf{{QUESTION}}

Phân tích đa thức 5a(a – b) – 3(b – a) thành nhân tử, ta được
A. 2(a – b);

\textbf{{ANSWER}}

Đáp án đúng là: C
5a(a – b) – 3(b – a) = 5a(a – b) + 3(a – b) = (a – b)(5a + 3).

========================================================================

https://khoahoc.vietjack.com/thi-online/10-bai-tap-phan-tich-da-thuc-thanh-nhan-tu-bang-cach-dat-nhan-tu-chung-co-loi-giai


\textbf{{QUESTION}}

Giá trị của biểu thức 27 . 93,7 + 270 . 0,63 là

\textbf{{ANSWER}}

Đáp án đúng là: C
27.93,7 + 270.0,63 = 27 . 93,7 + 27 . 10 . 0,63
= 27 . 93,7 + 27 . 6,3 = 27(93,7 + 6,3)
= 27 . 100 = 2 700.

========================================================================

https://khoahoc.vietjack.com/thi-online/10-bai-tap-phan-tich-da-thuc-thanh-nhan-tu-bang-cach-dat-nhan-tu-chung-co-loi-giai


\textbf{{QUESTION}}

Giá trị biểu thức x(x – 2) – y(2 – x) tại x = 3 002 và y = 998 là

\textbf{{ANSWER}}

Đáp án đúng là: D
x(x – 2) – y(2 – x) = x(x – 2) + y(x – 2) = (x – 2)(x + y)
Thay x = 3 002 và y = 998 vào biểu thức trên, ta được:
(3 002 – 2)(3 002 + 998) = 3 000 . 4 000 = 12 000 000.

========================================================================

https://khoahoc.vietjack.com/thi-online/giai-sgk-toan-8-kntt-bai-tap-chuong-2-co-dap-an


\textbf{{QUESTION}}

Đa thức x2 – 9x + 8 được phân tích thành tích của hai đa thức
A. x – 1 và x + 8;
B. x – 1 và x – 8;
C. x – 2 và x – 4;
D. x – 2 và x + 4.

\textbf{{ANSWER}}

Đáp án đúng là: B
Ta có x2 – 9x + 8 = (x2 – x) – (8x – 8) 
= x(x – 1) – 8(x – 1) = (x – 1)(x – 8).
Do đó, đa thức x2 – 9x + 8 được phân tích thành tích của hai đa thức x – 1 và x – 8.

========================================================================

https://khoahoc.vietjack.com/thi-online/giai-sgk-toan-8-kntt-bai-tap-chuong-2-co-dap-an


\textbf{{QUESTION}}

Khẳng định nào sau đây là đúng?
A. (A – B)(A + B) = A2 + 2AB + B2;
B. (A + B)(A – B) = A2 – 2AB + B2;
C. (A + B)(A – B) = A2 + B2;
D. (A + B)(A – B) = A2 – B2.

\textbf{{ANSWER}}

Đáp án đúng là: D
Ta có (A – B)(A + B) = (A + B)(A – B) = A2 – B2.

========================================================================

https://khoahoc.vietjack.com/thi-online/giai-sgk-toan-8-kntt-bai-tap-chuong-2-co-dap-an


\textbf{{QUESTION}}

Biểu thức 25x2 + 20xy + 4y2 viết dưới dạng bình phương của một tổng là:
A. [5x+(−2y)]2$$ {\left[5x+\left(-2y\right)\right]}^{2}$$;
B. [2x+(−5y)]2$$ {\left[2x+\left(-5y\right)\right]}^{2}$$;
C. (2x + 5y)2;
D. (5x + 2y)2.

\textbf{{ANSWER}}

Đáp án đúng là: A
Ta có 25x2 + 20xy + 4y2 = (5x)2 + 2 . 5x . 2y + (2y)2 
= (5x – 2y)2 = [5x+(−2y)]2$$ {\left[5x+\left(-2y\right)\right]}^{2}$$.

========================================================================

https://khoahoc.vietjack.com/thi-online/giai-sgk-toan-8-kntt-bai-tap-chuong-2-co-dap-an


\textbf{{QUESTION}}

Rút gọn biểu thức A = (2x + 1)3 – 6x(2x + 1) ta được:
A. x3 + 8;
B. x3 + 1;
C. 8x3 + 1;
D. 8x3 – 1.

\textbf{{ANSWER}}

Đáp án đúng là: C
Ta có A = (2x + 1)3 – 6x(2x + 1) 
= (2x)3 + 3 . (2x)2 . 1 + 3 . 2x . 12 + 13 – 12x2 – 6x
= 8x3 + 12x2 + 6x + 1 – 12x2 – 6x = 8x3 + 1.

========================================================================

https://khoahoc.vietjack.com/thi-online/giai-sgk-toan-8-kntt-bai-tap-chuong-2-co-dap-an


\textbf{{QUESTION}}

Tính nhanh giá trị của các biểu thức:
a) x2 – 4x + 4 tại x = 102;

\textbf{{ANSWER}}

a) Ta có x2 – 4x + 4 = (x – 2)2
Thay x = 102 vào biểu thức (x – 2)2, ta được:
(102 – 2)2 = 1002 = 10 000.

========================================================================

https://khoahoc.vietjack.com/thi-online/giai-sbt-toan-12-chan-troi-sang-tao-bai-3-phuong-trinh-mat-cau-co-dap-an


\textbf{{QUESTION}}

Cho mặt cầu (S) có tâm I(2; −1; 4) và bán kính R = 5. Các điểm A(3; 1; 5), B(−1; 11; 14), C(6; 2; 4) nằm trong, nằm trên hay nằm ngoài mặt cầu (S)?

\textbf{{ANSWER}}

Ta có: 
IA = $\sqrt {{{\left( {2 - 3} \right)}^2} + {{\left( { - 1 - 1} \right)}^2} + {{\left( {5 - 4} \right)}^2}} = \sqrt 6 < 5$ hay IA < R. 
Do đó, điểm A nằm trong mặt cầu (S).
IB = $\sqrt {{{\left( {2 - \left( { - 1} \right)} \right)}^2} + {{\left( { - 1 - 11} \right)}^2} + {{\left( {4 - 14} \right)}^2}} $ $ = \sqrt {253} > 5$ hay IB > R.
Do đó, điểm B nằm ngoài mặt cầu (S).
IC = $\sqrt {{{\left( {2 - 6} \right)}^2} + {{\left( { - 1 - 2} \right)}^2} + {{\left( {4 - 4} \right)}^2}} $ = 5 = R.
Do đó, điểm C nằm trên mặt cầu (S).

========================================================================

https://khoahoc.vietjack.com/thi-online/giai-sbt-toan-12-chan-troi-sang-tao-bai-3-phuong-trinh-mat-cau-co-dap-an


\textbf{{QUESTION}}

Viết phương trình mặt cầu (S) trong mỗi trường hợp sau:
a) (S) có tâm I(−5; 7; 6) và có bán kính R = 9.
b) (S) có tâm I(0; −3; 0) và đi qua điểm M(4; 0; −2).
c) (S) có đường kính EF với E(1; 5; 9), F(11; 3; 1).

\textbf{{ANSWER}}

a) (S) có tầm I(−5; 7; 6) và bán kính R = 9 nên có phương trình là:
(x + 5)2 + (y – 7)2 + (z – 6)2 = 92 hay (x + 5)2 + (y – 7)2 + (z – 6)2 = 81.
b) (S) có tâm I(0; −3; 0) và đi qua điểm M(4; 0; −2) có:
Bán kính R = IM = √(4−0)2+(0−(−3))2+(−2−0)2=√29$\sqrt {{{\left( {4 - 0} \right)}^2} + {{\left( {0 - \left( { - 3} \right)} \right)}^2} + {{\left( { - 2 - 0} \right)}^2}} = \sqrt {29} $.
Phương trình mặt cầu (S) là: x2 + (y + 3)2 + z2 = 29.
c) Tâm I của mặt cầu (S) đường kính EF chính là trung điểm của EF.
Do đó, ta có: {xI=1+112=6yI=5+32=4z1=9+12=5$\left\{ \begin{array}{l}{x_I} = \frac{{1 + 11}}{2} = 6\\{y_I} = \frac{{5 + 3}}{2} = 4\\{z_1} = \frac{{9 + 1}}{2} = 5\end{array} \right.$ ⇒ I(6; 4; 5).
Bán kính R = IE = √(6−1)2+(5−4)2+(9−5)2=√42$\sqrt {{{\left( {6 - 1} \right)}^2} + {{\left( {5 - 4} \right)}^2} + {{\left( {9 - 5} \right)}^2}} = \sqrt {42} $.
Vậy phương trình mặt cầu (S) là: (x – 6)2 + (y – 4)2 + (z – 5)2 = 42.

========================================================================

https://khoahoc.vietjack.com/thi-online/giai-sbt-toan-12-chan-troi-sang-tao-bai-3-phuong-trinh-mat-cau-co-dap-an


\textbf{{QUESTION}}

Xác định tâm và bán kính của mặt phẳng có phương trình sau:
a) (S): (x – 7)2 + ( y – 3)2 + (z + 4)2 = 49; 
b) (S'): x2 + (y + 1)2 + (z – 2)2 = 11;
c) (S''): x2 + y2 + z2 = 25.

\textbf{{ANSWER}}

a) Mặt cầu (S) có tâm I(7; 3; −4) và bán kính R = 7.
b) Mặt cầu (S') có tâm I(0; −1; 2) và bán kính R = √11$\sqrt {11} $.
c) Mặt cầu (S'') có tâm I(0; 0; 0) và bán kính R = 5.

========================================================================

https://khoahoc.vietjack.com/thi-online/giai-sbt-toan-12-chan-troi-sang-tao-bai-3-phuong-trinh-mat-cau-co-dap-an


\textbf{{QUESTION}}

Trong các phương trình sau, phương trình nào là phương trình mặt cầu? Xác định tâm và bán kính mặt của cầu đó.
a) 4x2 + y2 + z2 – 2x – 14y – 7z + 4 = 0;
b) x2 + y2 + z2 + 6x – 4y – 4z – 19 = 0;
c) x2 + y2 + z2 – 4x – 4y – 6z + 40 = 0.

\textbf{{ANSWER}}

a) Phương trình 4x2 + y2 + z2 – 2x – 14y – 7z + 4 = 0 không phải là phương trình mặt cầu do hệ số của x2 và y2 khác nhau.
b) Phương trình x2 + y2 + z2 + 6x – 4y – 4z – 19 = 0 có dạng 
x2 + y2 + z2 – 2ax – 2by – 2cz + d = 0 với a = −3; b = 2; c = 2; d = −19.
Ta có: a2 + b2 + c2 − d = 9 + 4 + 4 + 19 = 36 > 0, suy ra phương trình đã cho là phương trình mặt cầu tâm I(−3; 2; 2), bán kính R =  .
c) Phương trình x2 + y2 + z2 – 4x – 4y – 6z + 40 = 0, có dạng:
x2 + y2 + z2 – 2ax – 2by – 2cz + d = 0 với a = 2; b = 2, c = 3 và d = 40.
Ta thấy a2 + b2 + c2 – d = 4 + 4 + 9 – 40 = −23 < 0.
Suy ra phương trình đã cho không phải là phương trình mặt cầu.

========================================================================

https://khoahoc.vietjack.com/thi-online/tong-hop-25-de-thi-thu-thpt-quoc-gia-mon-toan-cuc-hay-chon-loc-co-loi-giai/48099


\textbf{{QUESTION}}

Hình lăng trụ có thể có số cạnh là số nào sau đây?
A.2018
B.2019
C.2017
D.2020

\textbf{{ANSWER}}

Đáp án D.
Bởi vì hình lăng trụ phải có số cạnh chia hết cho 4

========================================================================

https://khoahoc.vietjack.com/thi-online/tong-hop-25-de-thi-thu-thpt-quoc-gia-mon-toan-cuc-hay-chon-loc-co-loi-giai/48099


\textbf{{QUESTION}}

Cho các số  $$ x+\mathrm{2,}\text{ x}+\mathrm{14,}\text{ x}+50$$ theo thứ tự lập thành một cấp số nhân. Khi đó $$ {x}^{3}+2003$$ bằng:
A.2019
B.2017
C.2017
D.2020

\textbf{{ANSWER}}

Đáp án A
 số lập thành cấp số nhân
 $$ \Rightarrow \left(x+2\right)\left(x+50\right)={\left(x+14\right)}^{2}\phantom{\rule{0ex}{0ex}}\Leftrightarrow 24x=96\Leftrightarrow x=4$$
Khi đó $$ {x}^{2}+2003=2019$$

========================================================================

https://khoahoc.vietjack.com/thi-online/tong-hop-25-de-thi-thu-thpt-quoc-gia-mon-toan-cuc-hay-chon-loc-co-loi-giai/48099


\textbf{{QUESTION}}

Hàm số y=22+x2$$ y=\frac{2}{2+{x}^{2}}$$ đồng biến trên khoảng nào dưới đây?
A.(−2; 2)$$ \left(-2;\text{ }2\right)$$
B.(0;+∞)$$ \left(0;+\infty \right)$$
C.(−∞;0)$$ \left(-\infty ;0\right)$$
D.(−∞;+∞)$$ \left(-\infty ;+\infty \right)$$

\textbf{{ANSWER}}

Đáp án B
Có $$ {y}^{\text{'}}=\frac{4x}{{\left(2+{x}^{2}\right)}^{2}}.\quad {y}^{\text{'}}>0\Leftrightarrow x>0$$ . Vậy hàm số đồng biến trên $$ \left(0;+\infty \right)$$

========================================================================

https://khoahoc.vietjack.com/thi-online/tong-hop-25-de-thi-thu-thpt-quoc-gia-mon-toan-cuc-hay-chon-loc-co-loi-giai/48099


\textbf{{QUESTION}}

Hình chóp tứ giác đều có bao nhiêu mặt phẳng đối xứng?
A.1
B.2
C.3
D.4

\textbf{{ANSWER}}

Đáp án D
4 mặt phẳng đối xứng. Ví dụ như S.ABCD  là hình chóp tứ giác đều thì (SAC),(SBD)$$ \left(SAC\right),\left(SBD\right)$$ và (SMN),(SIJ)$$ \left(SMN\right),\left(SIJ\right)$$  với M,N là trung điểm của  AB,CD; I,J$$ AB,CD;\quad I,J$$ là trung điểm của BC,AD$$ BC,AD$$ .

========================================================================

https://khoahoc.vietjack.com/thi-online/10-bai-tacng-hang-dang-thuc-lap-phuong-cua-mot-hieu-de-tinh-nhanh-khai-trien-rut-gon


\textbf{{QUESTION}}

Đưa biểu thức 4n3 – 36n2 + 54n – 27 + 4n3 về dạng lập phương của một hiệu, ta được

\textbf{{ANSWER}}

Đáp án đúng là: A
Ta có 4n3 – 36n2 + 54n – 27 + 4n3 
= 8n3 – 36n2 + 54n – 27 
= (2n)3 – 3 . (2n)2 . 3 + 3 . 2n . 32 – 33 
= (2n – 3)3
Do đó ta chọn đáp án A.

========================================================================

https://khoahoc.vietjack.com/thi-online/10-bai-tacng-hang-dang-thuc-lap-phuong-cua-mot-hieu-de-tinh-nhanh-khai-trien-rut-gon


\textbf{{QUESTION}}

Khai triển (1 – 4m)3 ta được

\textbf{{ANSWER}}

Đáp án đúng là: B
Ta có (1 – 4m)3
= 13 – 3 . 12 . 4m + 3 . 1 . (4m)2 – (4m)3
= 1 – 12m + 48m2 – 64m3.
Do đó ta chọn đáp án B.

========================================================================

https://khoahoc.vietjack.com/thi-online/10-bai-tacng-hang-dang-thuc-lap-phuong-cua-mot-hieu-de-tinh-nhanh-khai-trien-rut-gon


\textbf{{QUESTION}}

Giá trị của biểu thức Y = 8n3 – 36n2 + 54n – 27 tại n = 1 là

\textbf{{ANSWER}}

Đáp án đúng là: D
Ta có: Y = 8n3 – 36n2 + 54n – 27 
= (2n)3 – 3 . (2n)2 . 3 + 3 . 2n . 32 – 33 
= (2n – 3)3.
Thay n = 1 vào biểu thức Y ta được:
Y = (2 . 1 – 3)3 = (– 1)3 = –1
Vậy Y = – 1 khi n = 1.
Do đó ta chọn đáp án D.

========================================================================

https://khoahoc.vietjack.com/thi-online/10-bai-tacng-hang-dang-thuc-lap-phuong-cua-mot-hieu-de-tinh-nhanh-khai-trien-rut-gon


\textbf{{QUESTION}}

Rút gọn biểu thức m3 – 21m2 + 147m – 343 ta được

\textbf{{ANSWER}}

Đáp án đúng là: B
Ta có: m3 – 21m2 + 147m – 243 
= m3 – 3 . m2 . 7 + 3 . m . 72 – 73 
= (m – 7)3.
Do đó ta chọn đáp án B.

========================================================================

https://khoahoc.vietjack.com/thi-online/10-bai-tacng-hang-dang-thuc-lap-phuong-cua-mot-hieu-de-tinh-nhanh-khai-trien-rut-gon


\textbf{{QUESTION}}

Khai triển hằng đẳng thức (4m – 2m – 5)3 ta được

\textbf{{ANSWER}}

Đáp án đúng là: C
 (4m – 2m – 5)3 = (2m – 5)3
= (2m)3 – 3 . (2m)2 . 5 + 3 . 2m . 52 – 53
= 8m3 – 60m2 + 150m – 125.
Do đó ta chọn đáp án C.

========================================================================

https://khoahoc.vietjack.com/thi-online/5-cau-trac-nghiem-toan-9-bai-1-goc-o-tam-so-do-cung-co-dap-an


\textbf{{QUESTION}}

Chọn khẳng định đúng. Góc ở tâm là góc
A. Có đỉnh nằm trên đường tròn
B. Có đỉnh trùng với tâm đường tròn
C. Có hai cạnh là hai đường kính của đường tròn
D. Có đỉnh nằm trên bán kính của đường tròn

\textbf{{ANSWER}}

Chọn đáp án B
Góc có đỉnh trùng với tâm đường tròn được gọi là góc ở tâm

========================================================================

https://khoahoc.vietjack.com/thi-online/5-cau-trac-nghiem-toan-9-bai-1-goc-o-tam-so-do-cung-co-dap-an


\textbf{{QUESTION}}

Chọn khẳng định đúng. Trong một đường tròn, số đo cung nhỏ bằng
A. Số đo cung lớn
B. Số đo góc ở tâm chắn cung đó
C. Số đo ở góc của tâm chắn cung lớn
D. Số đo của cung nửa đường tròn

\textbf{{ANSWER}}

Chọn đáp án B
Số đo của cung nhỏ bằng số đo của góc ở tâm chắn cung đó

========================================================================

https://khoahoc.vietjack.com/thi-online/5-cau-trac-nghiem-toan-9-bai-1-goc-o-tam-so-do-cung-co-dap-an


\textbf{{QUESTION}}

Trong hai cung của một đường tròn hay hai đường tròn bằng nhau, cung nào nhỏ hơn
A. Có số đo lớn hơn
B. Có số đo nhỏ hơn 90°$$ 90°$$
C. Có số đo lớn hơn 90°$$ 90°$$
D. Có số đo nhỏ hơn

\textbf{{ANSWER}}

Chọn đáp án D
Trong hai cung của một đường tròn hay hai đường tròn bằng nhau, cung nào nhỏ hơn thì có số đo nhỏ hơn

========================================================================

https://khoahoc.vietjack.com/thi-online/15-cau-trac-nghiem-toan-7-chan-troi-sang-tao-bai-7-tinh-chat-ba-duong-trung-tuyen-cua-tam-giac-co-da


\textbf{{QUESTION}}

Điền vào chỗ trống sau: “Đường trung tuyến của tam giác là đoạn thẳng nối một đỉnh của tam giác với ... của cạnh đối diện”. 
A. Trung trực;

\textbf{{ANSWER}}

Hướng dẫn giải
Đáp án đúng là: B
Đường trung tuyến của tam giác là đoạn thẳng nối một đỉnh của tam giác với trung điểm của cạnh đối diện.

========================================================================

https://khoahoc.vietjack.com/thi-online/15-cau-trac-nghiem-toan-7-chan-troi-sang-tao-bai-7-tinh-chat-ba-duong-trung-tuyen-cua-tam-giac-co-da


\textbf{{QUESTION}}

Điền vào chỗ trống sau: “Ba đường trung tuyến của một tam giác cắt nhau tại một điểm. Điểm đó cách mỗi đỉnh một khoảng bằng … độ dài đường trung tuyến đi qua điểm ấy.”

\textbf{{ANSWER}}

Hướng dẫn giải
Đáp án đúng là: C
Ba đường trung tuyến của một tam giác cắt nhau tại một điểm. Điểm đó cách mỗi đỉnh một khoảng bằng $$ \frac{2}{3}$$  độ dài đường trung tuyến đi qua điểm ấy.

========================================================================

https://khoahoc.vietjack.com/thi-online/tong-hop-25-de-luyen-thi-thptqg-mon-toan-chon-loc-cuc-hay-co-dap-an/50021


\textbf{{QUESTION}}

Hàm số nào sau đây có tập xác định là R 
A. $$ y=\frac{x-1}{2\text{x}-1}$$
B. $$ y={x}^{3}-2{\text{x}}^{2}+1$$ 
C. $$ y=\frac{\sqrt{x-1}}{{x}^{2}+1}$$ 
D. $$ y=\sqrt{{x}^{3}+1}$$

\textbf{{ANSWER}}

Đáp án B

========================================================================

https://khoahoc.vietjack.com/thi-online/tong-hop-25-de-luyen-thi-thptqg-mon-toan-chon-loc-cuc-hay-co-dap-an/50021


\textbf{{QUESTION}}

Đạo hàm của hàm số y=ex(sinx−cosx)$$ y={e}^{x}\left(\mathrm{sin}x-\mathrm{cos}x\right)$$ là:
A. y'=2ex.sinx$$ y\text{'}=2{e}^{x}.\mathrm{sin}x$$  
B. y'=−2ex.cosx$$ y\text{'}=-2{e}^{x}.\mathrm{cos}x$$ 
C. y'=2ex.cosx$$ y\text{'}=2{e}^{x}.\mathrm{cos}x$$
D. y'=−2ex.sinx$$ y\text{'}=-2{e}^{x}.\mathrm{sin}x$$

\textbf{{ANSWER}}

Đáp án A
$$ y\text{'}=\left[{e}^{x}\left(\mathrm{sin}x-\mathrm{cos}x\right)\right]\text{'}={e}^{x}\left(\mathrm{sin}x-\mathrm{cos}x\right)+{e}^{x}\left(\mathrm{sin}x+\mathrm{cos}x\right)y\text{'}=2{e}^{x}.\mathrm{sin}x$$

========================================================================

https://khoahoc.vietjack.com/thi-online/tong-hop-25-de-luyen-thi-thptqg-mon-toan-chon-loc-cuc-hay-co-dap-an/50021


\textbf{{QUESTION}}

Cho hình chóp S.ABCD có SA⊥(ABCD)$$ SA\bot \left(ABCD\right)$$ đáy ABCD là hình thang vuông có chiều cao AB=a.$$ AB=a.$$ Gọi I và J lần lượt là trung điểm AB,CD . Tính khoảng cách giữa đường thẳng IJ và (SAD).
A. a/3 
B. a√22$$ \frac{a\sqrt{2}}{2}$$ 
C. a√33$$ \frac{a\sqrt{3}}{3}$$ 
D. a/2

\textbf{{ANSWER}}

Đáp án D
Gọi I và J lần lượt là trung điểm AB,CD. Tính khoảng cách giữa đường thẳng IJ và (SAD).
SA⊥AD,AB⊥(SAD),IJ//(SAD)⇒d(IJ;(SAD))=d(I;(SAD))=IA=a2$$ SA\bot AD,AB\bot \left(SAD\right)\text{,IJ//}\left(SAD\right)\Rightarrow d\left(\text{IJ;}\left(SAD\right)\right)=d\left(\text{I;}\left(SAD\right)\right)=IA=\frac{a}{2}$$

========================================================================

https://khoahoc.vietjack.com/thi-online/17-cau-trac-nghiem-ung-dung-hinh-hoc-cua-tich-phan-co-dap-an


\textbf{{QUESTION}}

Viết công thức tính thể tích V của khối tròn xoay được tạo ra khi quay hình thang cong giới hạn bởi đồ thị hàm số y = f(x), trục Ox và hai đường thẳng x = a, x = b (a < b) quanh trục Ox
A. $$ V=\mathrm{\pi }{\int }_{\mathrm{a}}^{\mathrm{b}}\mathrm{f}\left(\mathrm{x}\right)\mathrm{dx}$$
B. $$ V={\int }_{\mathrm{a}}^{\mathrm{b}}{\mathrm{f}}^{2}\left(\mathrm{x}\right)\mathrm{dx}$$
C. $$ V=\mathrm{\pi }{\int }_{\mathrm{a}}^{\mathrm{b}}\left|\mathrm{f}\left(\mathrm{x}\right)\right|\mathrm{dx}$$
D. $$ V=\mathrm{\pi }{\int }_{\mathrm{a}}^{\mathrm{b}}{\mathrm{f}}^{2}\left(\mathrm{x}\right)\mathrm{dx}$$

\textbf{{ANSWER}}

Chọn D
Công thức tính thể tích V của khối tròn xoay được tạo ra khi quay hình thang cong giới hạn bởi đồ thị hàm số y = f(x), trục Ox và hai đường thẳng x = a, x = b (a < b) quanh trục Ox là
$$ V=\mathrm{\pi }{\int }_{\mathrm{a}}^{\mathrm{b}}{\mathrm{f}}^{2}\left(\mathrm{x}\right)\mathrm{dx}$$

========================================================================

https://khoahoc.vietjack.com/thi-online/bai-tap-trac-nghiem-chuong-3-hinh-hoc-lop-7-co-dap-an


\textbf{{QUESTION}}

Điểm E nằm trên tia phân giác góc  A của tam giác  ABC ta có
A.   E nằm trên tia phân giác góc B
B. E cách đều hai cạnh AB, AC
C. E nằm trên tia phân giác góc C
D. EB=EC

\textbf{{ANSWER}}

Đáp án B
Điểm E nằm trên tia phân giác góc A của tam giác ABC thì điểm E cách đều hai cạnh AB, AC

========================================================================

https://khoahoc.vietjack.com/thi-online/bai-tap-trac-nghiem-chuong-3-hinh-hoc-lop-7-co-dap-an


\textbf{{QUESTION}}

Cho tam giác ABC có hai đường phân giác CD và BE cắt nhau tại I  . Khi đó
A. AI là trung tuyến vẽ từ A
B.                      AI là đường cao kẻ từ A
C. AI là trung trực cạnh BC
D. AI là phân giác góc A

\textbf{{ANSWER}}

Đáp án D
Hai đường phân giác CD và BE  cắt nhau tại I mà ba đường phân giác của tam giác cùng đi qua một điểm nên AI là phân giác góc A

========================================================================

https://khoahoc.vietjack.com/thi-online/bai-tap-trac-nghiem-chuong-3-hinh-hoc-lop-7-co-dap-an


\textbf{{QUESTION}}

Em hãy chọn câu đúng nhất
A. Ba tia phân giác của tam giác cùng đi qua một điểm, điểm đó gọi là trọng tâm của tam giác
B. Giao điểm ba đường phân giác của tam giác cách đều ba cạnh của tam giác
C. Trong một tam giác, đường trung tuyến xuất phát từ một đỉnh đồng thời là đường phân giác ứng với cạnh đáy
D. Giao điểm ba đường phân giác của tam giác là tâm đường tròn ngoại tiếp tam giác đó

\textbf{{ANSWER}}

Đáp án B
+ Trọng tâm là giao điểm của ba đường trung tuyến nên đáp án A sai. Loại đáp án A
+ Giao điểm ba đường phân giác của tam giác cách đều ba cạnh của tam giác là đúng. 
+ Trong một tam giác, đường trung tuyến xuất phát từ một đỉnh đồng thời là đường phân giác ứng với cạnh đáy là sai vì tính chất này không phải đúng với mọi tam giác, nó chỉ đứng khi tam giác này cân và ta xét đường trung tuyến xuất phát từ đỉnh cân.
+ Giao điểm ba đường phân giác của tam giác là tâm đường tròn ngoại tiếp tam giác đó là sai vì giao điểm ba đường phân giác của tam giác là tâm đường tròn nội tiếp tam giác đó

========================================================================

https://khoahoc.vietjack.com/thi-online/10-bai-tap-uoc-lusong-xac-suat-cua-mot-bien-co-bang-xac-suat-thuc-nghiem-co-loi-giai


\textbf{{QUESTION}}

Trong một cuộc điều tra, trong 100 người được lựa chọn ngẫu nhiên ở một khu dân cư để phỏng vấn thì có 71 người ủng hộ việc tắt đèn điện trong sự kiện Giờ Trái Đất. Xác suất của biến cố “Một người được lựa chọn ngẫu nhiên trong khu dân cư ủng hộ việc tắt đèn điện trong sự kiện Giờ Trái Đất” được ước lượng là:

\textbf{{ANSWER}}

Đáp án đúng là: D
Trong 100 người được khảo sát, có 71 người ủng hộ.
Xác suất thực nghiệm của biến cố trên là $$ \frac{71}{100}=71\%$$ .
Vậy xác suất của biến cố trên được ước lượng là 71%.

========================================================================

https://khoahoc.vietjack.com/thi-online/10-bai-tap-uoc-lusong-xac-suat-cua-mot-bien-co-bang-xac-suat-thuc-nghiem-co-loi-giai


\textbf{{QUESTION}}

Trong tháng 4 và 5, camera quan sát đoạn đường Lê Hồng Phong ghi nhận 47 ngày tắc đường vào giờ cao điểm buổi sáng. Xác suất của biến cố “Không tắc đường vào giờ cao điểm buổi sáng ở đường Lê Hồng Phong” được ước lượng là:

\textbf{{ANSWER}}

Đáp án đúng là: B
Trong 61 ngày của tháng 4 và tháng 5, camera quan sát thì ghi nhận 61 – 47 = 14 ngày không tắc đường vào giờ cao điểm buổi sáng.
Xác suất thực nghiệm của biến cố trên là 1461≈23%$$ \frac{14}{61}\approx 23\%$$  .
Vậy xác suất của biến cố trên được ước lượng là 23%.

========================================================================

https://khoahoc.vietjack.com/thi-online/10-bai-tap-uoc-lusong-xac-suat-cua-mot-bien-co-bang-xac-suat-thuc-nghiem-co-loi-giai


\textbf{{QUESTION}}

Thống kê về số ca nhiễm bệnh và số ca tử vong của bệnh SARS từ tháng 11-2002 đến tháng 7-2003 được kết quả như sau: 8 437 người nhiễm, trong đó có 813 người tử vong. Khẳng định nào sau đây là đúng về xác suất một người tử vong khi nhiễm bệnh SARS?

\textbf{{ANSWER}}

Đáp án đúng là: B
Thống kê 8 437 người nhiễm, trong đó có 813 người tử vong.
Xác suất thực nghiệm của biến cố trên là 8138437≈9,6%$$ \frac{813}{8437}\approx 9,6\%$$  .
Vậy xác suất một người tử vong khi nhiễm bệnh SARS xấp xỉ bằng 9,6%, tức là nhỏ hơn 10%.

========================================================================

https://khoahoc.vietjack.com/thi-online/10-bai-tap-uoc-lusong-xac-suat-cua-mot-bien-co-bang-xac-suat-thuc-nghiem-co-loi-giai


\textbf{{QUESTION}}

Số lượt khách đến tham quan bảo tàng A trong năm qua được thống kê như sau:
Tháng
1; 2
3; 4
5; 6
7; 8
9; 10
11; 12
Số lượt khách
107
111
142
156
121
113
Xác suất của biến cố “Khách đến tham quan bảo tàng A trong 4 tháng đầu năm” là khoảng:

\textbf{{ANSWER}}

Đáp án đúng là: B
Tổng số lượt khách tham quan bảo tàng năm qua là:
107 + 111 + 142 + 156 + 121 + 113 = 750 (lượt).
Trong đó có 107 + 111 = 218 lượt trong 4 tháng đầu năm.
Xác suất của biến cố đã cho được ước lượng là  218750≈29%$$ \frac{218}{750}\approx 29\%$$.

========================================================================

https://khoahoc.vietjack.com/thi-online/10-bai-tap-uoc-lusong-xac-suat-cua-mot-bien-co-bang-xac-suat-thuc-nghiem-co-loi-giai


\textbf{{QUESTION}}

Chọn ngẫu nhiên 70 học sinh của một trường trung học cơ sở để kiểm tra thị lực thì thấy có 21 học sinh cận thị. Gọi A là biến cố “Học sinh được chọn bị cận thị”. Xác suất của biến cố A được ước lượng khoảng:

\textbf{{ANSWER}}

Đáp án đúng là: C
Trong 70 học sinh được kiểm tra, có 21 học sinh cận thị nên xác suất thực nghiệm của biến cố A là 2170=0,3$$ \frac{21}{70}=0,3$$ .
Vậy xác suất của biến cố A được ước lượng là 0,3.

========================================================================

https://khoahoc.vietjack.com/thi-online/10-bai-tap-xets-tinh-lien-tuc-cua-ham-so-tai-mot-diem-co-loi-giai


\textbf{{QUESTION}}

Xét tính liên tục của hàm số  $$ f\left(x\right)=\left\{\begin{array}{c}\frac{{x}^{2}-x}{{x}^{2}+2x-3}\text{ khi x}\ne 1\\ \frac{1}{3}\text{     khi }x=1\end{array}\right.$$ tại x = 1.

\textbf{{ANSWER}}

Đáp án đúng là: A
Ta có f(1) =  $$ \frac{1}{3}$$.
 $$ \underset{x\to 1}{\mathrm{lim}}f\left(x\right)=\underset{x\to 1}{\mathrm{lim}}\frac{{x}^{2}-x}{{x}^{2}+2x-3}=\underset{x\to 1}{\mathrm{lim}}\frac{x\left(x-1\right)}{\left(x-1\right)\left(x+3\right)}=\underset{x\to 1}{\mathrm{lim}}\frac{x}{x+3}=\frac{1}{3}=f\left(1\right)$$.
Vậy hàm số liên tục tại x = 1.

========================================================================

https://khoahoc.vietjack.com/thi-online/10-bai-tap-xets-tinh-lien-tuc-cua-ham-so-tai-mot-diem-co-loi-giai


\textbf{{QUESTION}}

Cho hàm số  f(x)={x2−1  khi x>113       khi x<1$$ f\left(x\right)=\left\{\begin{array}{c}{x}^{2}-1\text{  khi x}>1\\ \frac{1}{3}\text{       khi }x<1\end{array}\right.$$. Kết luận nào dưới đây không đúng?

\textbf{{ANSWER}}

Đáp án đúng là: A
Hàm số đã cho không xác định tại x = 1 nên không liên tục tại x = 1.

========================================================================

https://khoahoc.vietjack.com/thi-online/10-bai-tap-xets-tinh-lien-tuc-cua-ham-so-tai-mot-diem-co-loi-giai


\textbf{{QUESTION}}

Cho hàm số f(x)=x−1x3−4x$$ f\left(x\right)=\frac{x-1}{{x}^{3}-4x}$$  . Kết luận nào sau đây là đúng?
f(x)=x−1x3−4x
f(x)=x−1x3−4x
f
(x)
(
x
)
=
x−1x3−4x
x−1
x−1
x
−
1
x3−4x
x3−4x


x3−4x
x3−4x
x3−4x
x3
x
3
−
4
x
A. Hàm số trên liên tục tại điểm x = 
−2
;
B. Hàm số trên liên tục tại điểm x = 0;
C. Hàm số trên liên tục tại điểm x =  12$$ \frac{1}{2}$$;
12
12
12
1
1
2
2


2
2
2
D. Hàm số trên liên tục tại điểm x = 2.

\textbf{{ANSWER}}

Đáp án đúng là: C
Ta thấy, hàm số trên không xác định tại các điểm x = 0, x = 2 và x = −2 nên không liên tục tại các điểm đó. Vậy đáp án đúng là C.

========================================================================

https://khoahoc.vietjack.com/thi-online/10-bai-tap-xets-tinh-lien-tuc-cua-ham-so-tai-mot-diem-co-loi-giai


\textbf{{QUESTION}}

Cho hàm số  f(x)=√2x+1−√1−2xx$$ f\left(x\right)=\frac{\sqrt{2x+1}-\sqrt{1-2x}}{x}$$ với x ≠ 0. Để hàm số liên tục tại x = 0 thì giá trị f(0) bằng bao nhiêu?

\textbf{{ANSWER}}

Đáp án đúng là: C
 limx→0f(x)=limx→0√2x+1−√1−2xx=limx→02x+1−1+2xx(√2x+1+√1−2x)$$ \underset{x\to 0}{\mathrm{lim}}f\left(x\right)=\underset{x\to 0}{\mathrm{lim}}\frac{\sqrt{2x+1}-\sqrt{1-2x}}{x}=\underset{x\to 0}{\mathrm{lim}}\frac{2x+1-1+2x}{x\left(\sqrt{2x+1}+\sqrt{1-2x}\right)}$$
 =limx→04√2x+1+√1−2x=2$$ =\underset{x\to 0}{\mathrm{lim}}\frac{4}{\sqrt{2x+1}+\sqrt{1-2x}}=2$$.
Vậy hàm số liên tục tại x = 0 khi và chỉ khi  f(0)=limx→0f(x)=2$$ f\left(0\right)=\underset{x\to 0}{\mathrm{lim}}f\left(x\right)=2$$.

========================================================================

https://khoahoc.vietjack.com/thi-online/10-bai-tap-xets-tinh-lien-tuc-cua-ham-so-tai-mot-diem-co-loi-giai


\textbf{{QUESTION}}

Cho hàm số  f(x)={x2−2x      khi x>2x3−5x+7 khi x<2$$ f\left(x\right)=\left\{\begin{array}{c}{x}^{2}-2x\text{      khi x}>2\\ {x}^{3}-5x+7\text{ khi }x<2\end{array}\right.$$. Kết luận nào dưới đây không đúng?

\textbf{{ANSWER}}

Đáp án đúng là: A
Hàm số đã cho không xác định tại điểm x = 2 nên không liên tục tại điểm đó. Vậy đáp án A là đáp án sai.

========================================================================

https://khoahoc.vietjack.com/thi-online/de-kiem-tra-15-phut-toan-6-chuong-1/48277


\textbf{{QUESTION}}

Trong các số 40232, 1245, 52110

\textbf{{ANSWER}}

a) 40232 ;

========================================================================

https://khoahoc.vietjack.com/thi-online/de-kiem-tra-15-phut-toan-6-chuong-1/48277


\textbf{{QUESTION}}

Trong các số 40232, 1245, 52110b) Số nào chia hết cho 5 mà không chia hết cho 2 ?

\textbf{{ANSWER}}

b) 1245 ;

========================================================================

https://khoahoc.vietjack.com/thi-online/de-kiem-tra-15-phut-toan-6-chuong-1/48277


\textbf{{QUESTION}}

Trong các số 40232, 1245, 52110
c) Số nào chia hết cho cả 2 và 5 ?

\textbf{{ANSWER}}

c) 52110 ;

========================================================================

https://khoahoc.vietjack.com/thi-online/de-kiem-tra-15-phut-toan-6-chuong-1/48277


\textbf{{QUESTION}}

Trong các số 40232, 1245, 52110
d) Số nào chia hết cho 3 mà không chia hết cho 9 ?

\textbf{{ANSWER}}

d) 1245 ;

========================================================================

https://khoahoc.vietjack.com/thi-online/de-kiem-tra-15-phut-toan-6-chuong-1/48277


\textbf{{QUESTION}}

Trong các số 40232, 1245, 52110
e) Số nào chia hết cho cả 2 ; 3 ; 5 và 9 ?

\textbf{{ANSWER}}

e) 52110

========================================================================

https://khoahoc.vietjack.com/thi-online/bo-10-de-thi-giua-ki-2-toan-8-canh-dieu-cau-truc-moi-co-dap-an/162527


\textbf{{QUESTION}}

Trong các dữ liệu sau, dữ liệu nào là dữ liệu định tính?
A. Số huy chương vàng mà các vận động viên đã đạt được.
B. Danh sách các vận động viên tham dự Omlypic Tokyo 2020: Nguyễn Huy Hoàng, Nguyễn Thị  Ánh Viên,…
C. Số học sinh nữ của các tổ trong lớp 8A.

\textbf{{ANSWER}}

Đáp án đúng là: B
Trong các dữ liệu đã cho, dữ liệu định tính là danh sách các vận động viên tham dự Omlypic Tokyo 2020: Nguyễn Huy Hoàng, Nguyễn Thị Ánh Viên,…

========================================================================

https://khoahoc.vietjack.com/thi-online/bo-10-de-thi-giua-ki-2-toan-8-canh-dieu-cau-truc-moi-co-dap-an/162527


\textbf{{QUESTION}}

Dữ liệu về số người trong mỗi gia đình ở xóm em thuộc dữ liệu nào trong các dữ liệu sau:
A. Dữ liệu số rời rạc.                               
B. Dữ liệu số liên tục.

\textbf{{ANSWER}}

Đáp án đúng là: A
Dữ liệu về số người trong mỗi gia đình ở xóm em là dữ liệu số dời dạc.

========================================================================

https://khoahoc.vietjack.com/thi-online/bo-10-de-thi-giua-ki-2-toan-8-canh-dieu-cau-truc-moi-co-dap-an/162527


\textbf{{QUESTION}}

Bạn An đứng ở cộng trường và ghi lại xem bạn nào ra về bằng xe đạp khi tan trường. Phương pháp bạn An thu được dữ liệu là:

\textbf{{ANSWER}}

Đáp án đúng là: B
Để thu thập dữ liệu trên, bạn An đứng ở cổng trường và quan sát rồi ghi lại xem bạn nào ra về bằng xe đạp khi tan trường.
Do đó, phương pháp bạn An thu được dữ liệu là quan sát.

========================================================================

https://khoahoc.vietjack.com/thi-online/bo-10-de-thi-giua-ki-2-toan-8-canh-dieu-cau-truc-moi-co-dap-an/162527


\textbf{{QUESTION}}

Nhiệt độ trung bình các tháng trong năm của một quốc gia được biểu diễn như sau:
Tháng
1
2
3
4
5
6
7
8
9
10
11
12
Nhiệt độ (độ C)
2
3
5
15
20
30
29
27
20
15
12
7
Biểu đồ thích hợp để biểu diễn dữ liệu trong bảng trên là
A. Biểu đồ hình quạt tròn.                       
B. Biểu đồ cột tranh.

\textbf{{ANSWER}}

Đáp án đúng là: C
Biểu đồ thích hợp để biểu diễn dữ liệu trong bảng trên là biểu đồ đoạn thẳng.

========================================================================

https://khoahoc.vietjack.com/thi-online/de-kiem-tra-15-phut-toan-6-chuong-2-dai-so/48340


\textbf{{QUESTION}}

Tìm x ∈ Z, biết:
a) x - 9 = -14

\textbf{{ANSWER}}

a) x – 9 = -14
x = -14 + 9
x = -5

========================================================================

https://khoahoc.vietjack.com/thi-online/de-kiem-tra-15-phut-toan-6-chuong-2-dai-so/48340


\textbf{{QUESTION}}

Tìm x ∈ Z, biết:
b) 2( x + 7 ) = -16

\textbf{{ANSWER}}

b) 2( x + 7 ) = -16
2( x + 7 ) = 2 . ( -8 )
x + 7 = -8
x = -8 – 7 = -15

========================================================================

https://khoahoc.vietjack.com/thi-online/de-kiem-tra-15-phut-toan-6-chuong-2-dai-so/48340


\textbf{{QUESTION}}

Tìm x ∈ Z, biết:
c) |x – 9| = 7

\textbf{{ANSWER}}

c) | x – 9 | = 7
x – 9 = 7 hoặc x – 9 = -7
x = 7 + 9 hoặc x = -7 + 9
x = 16 hoặc x = 2

========================================================================

https://khoahoc.vietjack.com/thi-online/de-kiem-tra-15-phut-toan-6-chuong-2-dai-so/48340


\textbf{{QUESTION}}

Tìm x ∈ Z, biết:
d) ( x – 5 )( x + 7 ) = 0

\textbf{{ANSWER}}

d) ( x – 5 )( x + 7 ) = 0
x – 5 = 0 hoặc x + 7 = 0
x = 5 hoặc x = -7

========================================================================

https://khoahoc.vietjack.com/thi-online/de-kiem-tra-15-phut-toan-6-chuong-2-dai-so/48340


\textbf{{QUESTION}}

Tính tổng các số nguyên x thỏa mãn:
a) -3 < x < 2

\textbf{{ANSWER}}

a) -3 < x < 2 , x ∈ Z. Do đó : x ∈ { -2 ; -1 ; 0 ; 1 }
Tổng các số nguyên x là : -2 + (-1) + 0 + 1 = -2

========================================================================

https://khoahoc.vietjack.com/thi-online/giai-sbt-toan-8-chan-troi-sang-tao-bai-4-he-so-goc-cua-duong-thang-co-dap-an


\textbf{{QUESTION}}

Cho hàm số y = ax + 2. Tìm hệ số góc a, biết rằng:
a) Đồ thị của hàm số đi qua điểm A(1; 3).
b) Đồ thị của hàm số song song với đường thằng y = –2x + 1.

\textbf{{ANSWER}}

Lời giải
a) Vì đồ thị hàm số y = ax + 2 đi qua điểm A(1; 3) nên 3 = a.1 + 2 Û a = 1.
Vậy a = 1.
b) Vì đồ thị của hàm số y = ax + 2 song song với đường thằng y = –2x + 1 nên $\left\{ \begin{array}{l}a = - 2\\2 \ne 1\,\,(tm)\end{array} \right.$
Vậy a = –2.

========================================================================

https://khoahoc.vietjack.com/thi-online/giai-sbt-toan-8-chan-troi-sang-tao-bai-4-he-so-goc-cua-duong-thang-co-dap-an


\textbf{{QUESTION}}

Cho hai hàm số y = 2mx + 11 và y = (1 – m)x + 2. Với giá trị nào của m thì đồ thị của hai hàm số đã cho là:
a) Hai đường thẳng song song với nhau?
b) Hai đường thẳng cắt nhau?

\textbf{{ANSWER}}

Lời giải
a) Để đường thẳng y = 2mx + 11 song song với đường thằng y = (1 – m)x + 2 thì:
 $\left\{ \begin{array}{l}2m = 1 - m\\11 \ne 2\,\,(tm)\end{array} \right. \Leftrightarrow 3m = 1 \Leftrightarrow m = \frac{1}{3}$
Vậy $m = \frac{1}{3}$.
b) Để đường thẳng y = 2mx + 11 cắt đường thằng y = (1 – m)x + 2 thì:
2m ¹ 1 – m 
$3m \ne 1 \Leftrightarrow m \ne \frac{1}{3}$
Vậy $m \ne \frac{1}{3}$.

========================================================================

https://khoahoc.vietjack.com/thi-online/giai-sbt-toan-8-chan-troi-sang-tao-bai-4-he-so-goc-cua-duong-thang-co-dap-an


\textbf{{QUESTION}}

Cho hàm số y = 2x + b. Tìm b trong mỗi trường hợp sau:
a) Với x = 4 thì hàm số có giá trị bằng 5.
b) Đồ thị của hàm số đã cho cắt trục tung tại điểm có tung độ bằng 7.
c) Đồ thị của hàm số đã cho đi qua điểm A(1; 5).

\textbf{{ANSWER}}

Lời giải
a) Với x = 4 thì hàm số có giá trị bằng 5 nên thay vào hàm số y = 2x + b ta có:
5 = 2.4 + b Û b = 5 – 8 = –3.
Vậy b = –3.
b) Vì đồ thị của hàm số y = 2x + b cắt trục tung tại điểm M có tung độ bằng 7 nên toạ độ điểm M(0; 7).
Thay M(0; 7) vào y = 2x + b ta được:
2.0 + b = 7 Û b = 7.
Vậy b = 7.
c) Đồ thị của hàm số y = 2x + b đi qua điểm A(1; 5).
Thay A(1; 5) vào y = 2x + b ta được:
2.1 + b = 5 Û b = 5 – 2 = 3
Vậy b = 3.

========================================================================

https://khoahoc.vietjack.com/thi-online/giai-sbt-toan-8-chan-troi-sang-tao-bai-4-he-so-goc-cua-duong-thang-co-dap-an


\textbf{{QUESTION}}

Đồ thị của hàm số là đường thằng d1 đi qua gốc toạ độ. Hãy xác định hàm số trong mỗi trường hợp sau:
a) Đồ thị của hàm số đã cho đi qua điểm A(3; 4).
b) Đồ thị của hàm số là đường thẳng có hệ số góc bằng −47$\frac{{ - 4}}{7}$.
c) Đồ thị của hàm số là đường thẳng song song với đường thẳng d2: y = –6x – 5.

\textbf{{ANSWER}}

Lời giải
a) Đồ thị của hàm số y = ax + b là đường thằng d1 đi qua gốc toạ độ nên 0 = a.0 + b.
Do đó b = 0.
Đồ thị hàm số có dạng: y = ax.
Đồ thị y = ax đi qua điểm A(3; 4) thay A(3; 4) vào đồ thị ta được:
4=3a⇔a=43$4 = 3a \Leftrightarrow a = \frac{4}{3}$.
Vậy hàm số cần tìm có phương trình y=43x$y = \frac{4}{3}x$.
b) Đồ thị của hàm số y = ax có hệ số góc bằng −47$\frac{{ - 4}}{7}$ hay a=−47$a = \frac{{ - 4}}{7}$.
Vậy hàm số cần tìm có phương trình y=−47x$y = \frac{{ - 4}}{7}x$.
c) Vì đồ thị của hàm số y = ax song song với đường thẳng d2: y = –6x – 5 nên a = –6.
Vậy hàm số cần tìm có phương trình y = –6x.

========================================================================

https://khoahoc.vietjack.com/thi-online/giai-sbt-toan-8-chan-troi-sang-tao-bai-4-he-so-goc-cua-duong-thang-co-dap-an


\textbf{{QUESTION}}

Hãy xác định hàm số y = ax + b biết đồ thị của hàm số là đường thẳng đi qua các điểm sau: 
a) A(1; 5) và B(0; 2).                     
b) M(1; 9) và N(0; 1).                    
c) P(0; 2) và Q(1; 0).

\textbf{{ANSWER}}

Lời giải
a) Đồ thị của hàm số y = ax + b đi qua điểm A(1; 5) và B(0; 2).
Thay A(1; 5) và B(0; 2) vào hàm số ta có hệ phương trình:
{a.1+b=5a.0+b=2⇔{a+b=5b=2⇔{a+2=5b=2⇔{a=3b=2$\left\{ \begin{array}{l}a.1 + b = 5\\a.0 + b = 2\end{array} \right. \Leftrightarrow \left\{ \begin{array}{l}a + b = 5\\b = 2\end{array} \right. \Leftrightarrow \left\{ \begin{array}{l}a + 2 = 5\\b = 2\end{array} \right. \Leftrightarrow \left\{ \begin{array}{l}a = 3\\b = 2\end{array} \right.$.
Vậy hàm số cần tìm có phương trình y = 3x + 2.
b) Đồ thị của hàm số y = ax + b đi qua điểm M(1; 9) và N(0; 1).
Thay M(1; 9) và N(0; 1) vào hàm số ta có hệ phương trình:
{a.1+b=9a.0+b=1⇔{a+b=9b=1⇔{a+1=9b=1⇔{a=8b=1$\left\{ \begin{array}{l}a.1 + b = 9\\a.0 + b = 1\end{array} \right. \Leftrightarrow \left\{ \begin{array}{l}a + b = 9\\b = 1\end{array} \right. \Leftrightarrow \left\{ \begin{array}{l}a + 1 = 9\\b = 1\end{array} \right. \Leftrightarrow \left\{ \begin{array}{l}a = 8\\b = 1\end{array} \right.$.
Vậy hàm số cần tìm có phương trình y = 8x + 1.
 c) Đồ thị của hàm số y = ax + b đi qua điểm P(0; 2) và Q(1; 0).
Thay P(0; 2) và Q(1; 0) vào hàm số ta có hệ phương trình:
{a.0+b=2a.1+b=0⇔{b=2a+2=0⇔{b=2a=−2$\left\{ \begin{array}{l}a.0 + b = 2\\a.1 + b = 0\end{array} \right. \Leftrightarrow \left\{ \begin{array}{l}b = 2\\a + 2 = 0\end{array} \right. \Leftrightarrow \left\{ \begin{array}{l}b = 2\\a = - 2\end{array} \right.$

========================================================================

https://khoahoc.vietjack.com/thi-online/7881-cau-trac-nghiem-tong-hop-mon-toan-2023-cuc-hay-co-dap-an/122622


\textbf{{QUESTION}}

Cho biểu thức $$ A=\left(\frac{x\sqrt{x}-1}{x-\sqrt{x}}-\frac{x\sqrt{x}+1}{x+\sqrt{x}}\right):\left(\frac{\sqrt{x}}{\sqrt{x}-1}-\frac{1}{\sqrt{x}+1}\right)$$.
a) Rút gọn A.

\textbf{{ANSWER}}

a) Điều kiện xác định: x > 0, x ≠ 1.
Ta có:
$$ A=\left(\frac{x\sqrt{x}-1}{x-\sqrt{x}}-\frac{x\sqrt{x}+1}{x+\sqrt{x}}\right):\left(\frac{\sqrt{x}}{\sqrt{x}-1}-\frac{1}{\sqrt{x}+1}\right)\phantom{\rule{0ex}{0ex}}=\left[\frac{\left(\sqrt{x}-1\right)\left(x+\sqrt{x}+1\right)}{\sqrt{x}\left(\sqrt{x}-1\right)}-\frac{\left(\sqrt{x}+1\right)\left(x-\sqrt{x}+1\right)}{\sqrt{x}\left(\sqrt{x}+1\right)}\right]:\left[\frac{\sqrt{x}\left(\sqrt{x}+1\right)}{\left(\sqrt{x}-1\right)\left(\sqrt{x}+1\right)}-\frac{\sqrt{x}-1}{\left(\sqrt{x}-1\right)\left(\sqrt{x}+1\right)}\right]\phantom{\rule{0ex}{0ex}}=\left[\frac{\left(x+\sqrt{x}+1\right)}{\sqrt{x}}-\frac{\left(x-\sqrt{x}+1\right)}{\sqrt{x}}\right]:\left[\frac{\sqrt{x}\left(\sqrt{x}+1\right)}{\left(\sqrt{x}-1\right)\left(\sqrt{x}+1\right)}-\frac{\sqrt{x}-1}{\left(\sqrt{x}-1\right)\left(\sqrt{x}+1\right)}\right]\phantom{\rule{0ex}{0ex}}=\frac{x+\sqrt{x}+1-x+\sqrt{x}-1}{\sqrt{x}}:\frac{x+\sqrt{x}-\sqrt{x}+1}{\left(\sqrt{x}-1\right)\left(\sqrt{x}+1\right)}\phantom{\rule{0ex}{0ex}}=\frac{2\sqrt{x}}{\sqrt{x}}:\frac{x+1}{\left(\sqrt{x}-1\right)\left(\sqrt{x}+1\right)}\phantom{\rule{0ex}{0ex}}=2.\frac{\left(\sqrt{x}-1\right)\left(\sqrt{x}+1\right)}{x+1}\phantom{\rule{0ex}{0ex}}=\frac{{\displaystyle 2\text{x}-2}}{{\displaystyle x+1}}$$
Vậy với x > 0, x ≠ 1 thì $$ A=\frac{2\text{x}-2}{x+1}$$.

========================================================================

https://khoahoc.vietjack.com/thi-online/7881-cau-trac-nghiem-tong-hop-mon-toan-2023-cuc-hay-co-dap-an/122622


\textbf{{QUESTION}}

b) Tìm giá trị nguyên của x để A đạt giá trị nguyên.

\textbf{{ANSWER}}

b) Với x > 0, x ≠ 1 ta có:
Để A đạt giá trị nguyên thì $$ \frac{4}{x+1}$$ đạt giá trị nguyên
Suy ra x + 1 ∈ Ư(4) = {1; 2; 4; –1; –2; –4}
Mà x > 0, x ≠ 1 nên x + 1 > 1 và x + 1 ≠ 2
Do đó x + 1 = 3
Suy ra x = 3
Vậy x = 3 thì A đạt giá trị nguyên.

========================================================================

https://khoahoc.vietjack.com/thi-online/9-cau-trac-nghiem-toan-8-bai-4-quy-dong-mau-thuc-nhieu-phan-thuc-co-dap-an-nhan-biet


\textbf{{QUESTION}}

Mẫu thức chung của các phân thức $$ \frac{1}{x+1},\frac{1}{x-1},\frac{1}{x}$$ là?
A. x(x2 - 1)
B. x(x - 1)2
C. x2 - 1
D. x(x - 1)

\textbf{{ANSWER}}

Mẫu chung của các phân thức $$ \frac{1}{x+1},\frac{1}{x-1},\frac{1}{x}$$ là
(x + 1)(x - 1). x = x(x2 - 1).
Đáp án cần chọn là: A

========================================================================

https://khoahoc.vietjack.com/thi-online/9-cau-trac-nghiem-toan-8-bai-4-quy-dong-mau-thuc-nhieu-phan-thuc-co-dap-an-nhan-biet


\textbf{{QUESTION}}

Mẫu thức chung của các phân thức 16x2y,1x2y3,112xy4$$ \frac{1}{6{x}^{2}y},\frac{1}{{x}^{2}{y}^{3}},\frac{1}{12x{y}^{4}}$$ là?
A. 12x2y3
B. 12x2y4
C. 6x3y2
D. 12x4y

\textbf{{ANSWER}}

Các mẫu thức lần lượt là: 6x2y; x2y3; 12xy4
Ta có phần hệ số của mẫu thức chung là BCNN(6; 12) = 12
Phần biến số là: x2y4
Suy ra mẫu chung của các phân thức $$ \frac{1}{6{x}^{2}y},\frac{1}{{x}^{2}{y}^{3}},\frac{1}{12x{y}^{4}}$$ là 12x2y4.
Đáp án cần chọn là: B

========================================================================

https://khoahoc.vietjack.com/thi-online/9-cau-trac-nghiem-toan-8-bai-4-quy-dong-mau-thuc-nhieu-phan-thuc-co-dap-an-nhan-biet


\textbf{{QUESTION}}

Đa thức nào sau đây là mẫu thức chung của các phân thức x3(x-y)2;yx-y$$ \frac{x}{3{\left(x-y\right)}^{2}};\frac{y}{x-y}$$?
A. (x - y)2
B. x - y
C. 3(x - y)2
D. 4(x - y)3

\textbf{{ANSWER}}

Mẫu thức của hai phân thức x3(x-y)2;yx-y$$ \frac{x}{3{\left(x-y\right)}^{2}};\frac{y}{x-y}$$ là 3(x - y)2 và (x - y)
Nên mẫu thức chung có phần hệ số là 3, phần biến số là (x - y)2 => Mẫu thức chung 3(x - y)2.
Đáp án cần chọn là: C

========================================================================

https://khoahoc.vietjack.com/thi-online/9-cau-trac-nghiem-toan-8-bai-4-quy-dong-mau-thuc-nhieu-phan-thuc-co-dap-an-nhan-biet


\textbf{{QUESTION}}

Đa thức nào sau đây là mẫu thức chung của các phân thức 5x(x+3)3;73(x+3)$$ \frac{5x}{{\left(x+3\right)}^{3}};\frac{7}{3\left(x+3\right)}$$?
A. (x + 3)3
B. 3(x + 3)2
C. 3(x + 3)3
D. (x + 3)4

\textbf{{ANSWER}}

Mẫu thức của hai phân thức $$ \frac{5x}{{\left(x+3\right)}^{3}};\frac{7}{3\left(x+3\right)}$$ là (x + 3)3 và 3(x + 3).
Nên mẫu thức chung có phần hệ số là 3, phần biến số là (x + 3)3 => Mẫu thức chung 3(x + 3)3.
Đáp án cần chọn là: C

========================================================================

https://khoahoc.vietjack.com/thi-online/de%CC%80-kie%CC%89m-tra-chuong-1/58478


\textbf{{QUESTION}}

Cho tập hợp X = {1;2;4;7}. Trong các tập hợp sau, tập hợp nào là tập hợp con của tập hợp X?
a, {1;7}
b, {1;5}
c, {2;5}
d, {3;7}

\textbf{{ANSWER}}

Đáp án: a

========================================================================

https://khoahoc.vietjack.com/thi-online/de%CC%80-kie%CC%89m-tra-chuong-1/58478


\textbf{{QUESTION}}

Tập hợp Y = {x∈$$ \in $$N|x≤9}. Số phần tử của Y là:

\textbf{{ANSWER}}

Số phần tử của Y là: 10

========================================================================

https://khoahoc.vietjack.com/thi-online/de%CC%80-kie%CC%89m-tra-chuong-1/58478


\textbf{{QUESTION}}

Kết quả của biểu thức 16 + 83 + 84 + 7 là:

\textbf{{ANSWER}}

16 + 83 + 84 + 7 = 190

========================================================================

https://khoahoc.vietjack.com/thi-online/de%CC%80-kie%CC%89m-tra-chuong-1/58478


\textbf{{QUESTION}}

Tích 34.35$$ {3}^{4}.{3}^{5}$$ được viết gọn là:

\textbf{{ANSWER}}

34.35=39$$ {3}^{4}.{3}^{5}={3}^{9}$$

========================================================================

https://khoahoc.vietjack.com/thi-online/de%CC%80-kie%CC%89m-tra-chuong-1/58478


\textbf{{QUESTION}}

Tập hợp M = {2;3;4;...;11;12} có số phần tử là:

\textbf{{ANSWER}}

Tập hợp M = {2;3;4;...;11;12} có số phần tử là: 11

========================================================================

https://khoahoc.vietjack.com/thi-online/15-cau-trac-nghiem-toan-9-canh-dieu-tao-bai-4-mot-so-phep-bien-doi-can-thuc-bac-hai-cua-bieu-thuc-da


\textbf{{QUESTION}}

I. Nhận biết
Cho biểu thức $A < 0.$ Khẳng định nào sau đây là đúng?
A. $\sqrt {{A^2}} = A$.
B. $\sqrt {{A^2}} = - A$.
C. $\sqrt {{A^2}} = \left| A \right|$.
D. $\sqrt {{A^2}} = - \left| A \right|$.

\textbf{{ANSWER}}

Đáp án đúng là: B
Với $A < 0,$ ta có: $\sqrt {{A^2}} = \left| A \right| = - A$.

========================================================================

https://khoahoc.vietjack.com/thi-online/15-cau-trac-nghiem-toan-9-canh-dieu-tao-bai-4-mot-so-phep-bien-doi-can-thuc-bac-hai-cua-bieu-thuc-da


\textbf{{QUESTION}}

Cho hai biểu thức A$A$ và B$B$. Khẳng định nào sau đây là sai?
A. √AB=√A⋅√B$\sqrt {AB} = \sqrt A \cdot \sqrt B $ với A≥0,B≥0.$A \ge 0,\,\,B \ge 0.$
B. √AB=√−A⋅√−B$\sqrt {AB} = \sqrt { - A} \cdot \sqrt { - B} $ với A<0,B<0$A < 0,\,\,B < 0$.
C. √AB=√A√B$\sqrt {\frac{A}{B}} = \frac{{\sqrt A }}{{\sqrt B }}$ với A≥0,B≥0.$A \ge 0,\,\,B \ge 0.$
D. √AB=√−A√−B$\sqrt {\frac{A}{B}} = \frac{{\sqrt { - A} }}{{\sqrt { - B} }}$ với A<0,B<0$A < 0,\,\,B < 0$.

\textbf{{ANSWER}}

Đáp án đúng là: C
Ta có $\sqrt {\frac{A}{B}} = \frac{{\sqrt A }}{{\sqrt B }}$ khi $A \ge 0,\,\,B > 0.$

========================================================================

https://khoahoc.vietjack.com/thi-online/15-cau-trac-nghiem-toan-9-canh-dieu-tao-bai-4-mot-so-phep-bien-doi-can-thuc-bac-hai-cua-bieu-thuc-da


\textbf{{QUESTION}}

Cho biểu thức $A \ge 0,\,\,B < 0$, khẳng định nào sau đây đúng?
A. $\sqrt {A{B^2}} = A\sqrt B $.
B. $\sqrt {A{B^2}} = - A\sqrt B $.
C. $\sqrt {A{B^2}} = - B\sqrt A $.
D. $\sqrt {A{B^2}} = B\sqrt A $.

\textbf{{ANSWER}}

Đáp án đúng là: C
Ta có $\sqrt {A{B^2}} = \sqrt A .\sqrt {{B^2}} = \sqrt A \cdot \left| B \right| = - B\sqrt A $ (do $B < 0$).

========================================================================

https://khoahoc.vietjack.com/thi-online/15-cau-trac-nghiem-toan-9-canh-dieu-tao-bai-4-mot-so-phep-bien-doi-can-thuc-bac-hai-cua-bieu-thuc-da


\textbf{{QUESTION}}

Chọn phát biểu sai trong các phát biểu sau:
A. Nếu a$a$ là một số dương và b$b$ là một số không âm thì √a2b=a√b$\sqrt {{a^2}b}  = a\sqrt b $.
B. Nếu a$a$ và b$b$ là hai số không âm thì √a2b=a√b$\sqrt {{a^2}b}  = a\sqrt b $.
C. Nếu a$a$ là một số âm và b$b$ là một số không âm thì a√b=√a2b$a\sqrt b  = \sqrt {{a^2}b} $.
D. Với các biểu thức A,B$A,B$ và B>0$B > 0$, ta có: A√B=A√BB$\frac{A}{{\sqrt B }} = \frac{{A\sqrt B }}{B}$.

\textbf{{ANSWER}}

Đáp án đúng là: C
Nếu a$a$ là một số âm và b$b$ là một số không âm thì a√b=−√a2b$a\sqrt b  =  - \sqrt {{a^2}b} $ nên phương án C là khẳng định sai.
Vậy ta chọn phương án C.

========================================================================

https://khoahoc.vietjack.com/thi-online/15-cau-trac-nghiem-toan-9-canh-dieu-tao-bai-4-mot-so-phep-bien-doi-can-thuc-bac-hai-cua-bieu-thuc-da


\textbf{{QUESTION}}

Cho hai biểu thức A$A$ và B>0.$B > 0.$ Khẳng định nào sau đây là sai?
A. A√B=A√BB$\frac{A}{{\sqrt B }} = \frac{{A\sqrt B }}{B}$.
B. 1A+√B=A−√BA2−B$\frac{1}{{A + \sqrt B }} = \frac{{A - \sqrt B }}{{{A^2} - B}}$.
C. 1A−√B=A+√BA2−B$\frac{1}{{A - \sqrt B }} = \frac{{A + \sqrt B }}{{{A^2} - B}}$.
D. 1B−√B=B−√BB2−B$\frac{1}{{B - \sqrt B }} = \frac{{B - \sqrt B }}{{{B^2} - B}}$.

\textbf{{ANSWER}}

Đáp án đúng là:
Với $B > 0,$ ta có:
⦁ $\frac{A}{{\sqrt B }} = \frac{{A\sqrt B }}{B}$;
⦁ $\frac{1}{{A + \sqrt B }} = \frac{{A - \sqrt B }}{{\left( {A + \sqrt B } \right)\left( {A - \sqrt B } \right)}} = \frac{{A - \sqrt B }}{{{A^2} - B}}$;
⦁ $\frac{1}{{A - \sqrt B }} = \frac{{A + \sqrt B }}{{\left( {A - \sqrt B } \right)\left( {A + \sqrt B } \right)}} = \frac{{A + \sqrt B }}{{{A^2} - B}}$;
⦁ $\frac{1}{{B - \sqrt B }} = \frac{{B + \sqrt B }}{{\left( {B - \sqrt B } \right)\left( {B + \sqrt B } \right)}} = \frac{{B + \sqrt B }}{{{B^2} - B}}.$
Vậy phương án D là khẳng định sai, ta chọn phương án D.

========================================================================

https://khoahoc.vietjack.com/thi-online/top-4-de-kiem-tra-1-tiet-toan-10-chuong-4-dai-so-co-dap-an


\textbf{{QUESTION}}

Tập xác định của hàm số $$ \frac{1}{\sqrt{2-x}}$$ là:
A. ($$ -\infty $$;2]
B. [2;$$ +\infty $$)
C. (2;$$ +\infty $$)
D. ($$ -\infty $$;2)

\textbf{{ANSWER}}

Chọn D.
Hàm số $$ \frac{1}{\sqrt{2-x}}$$ xác định khi 2 - x > 0 ⇒ x < 2.
Tập xác định của hàm số là S = ($$ -\infty $$;2).

========================================================================

https://khoahoc.vietjack.com/thi-online/giai-sbt-toan-8-canh-dieu-bai-24-phan-tich-va-xu-ly-du-lieu-thu-duoc-o-dang-bang-bieu-do-co-dap-an


\textbf{{QUESTION}}

Một hộp có 50 chiếc thẻ cùng loại, mỗi thẻ được ghi một trong các số 1, 3, 5, ..., 97, 99; hai thẻ khác nhau thì ghi hai số khác nhau. Rút ngẫu nhiên một thẻ trong hộp. Tính xác suất của mỗi biến cố sau:
a) “Số xuất hiện trên thẻ được rút ra là số lớn hơn 3 và là ước của 50”;
b) “Số xuất hiện trên thẻ được rút ra là số nhỏ hơn 60 và là bội của 11”;
c) “Số xuất hiện trên thẻ được rút ra là số chia hết cho cả 3 và 5”;
d) “Số xuất hiện trên thẻ được rút ra là số có hai chữ số và tổng hai chữ số đó là 7”.

\textbf{{ANSWER}}

a) Các kết quả thuận lợi cho biến cố “Số xuất hiện trên thẻ được rút ra là số lớn hơn 3 và là ước của 50” là: 5; 25. 
Do đó, có hai kết quả thuận lợi cho biến cố đó. 
Vì vậy, xác suất của biến cố đó là  $$ \frac{2}{50}=\frac{1}{25}.$$
b) Các kết quả thuận lợi cho biến cố “Số xuất hiện trên thẻ được rút ra là số nhỏ hơn 60 và là bội của 11” là: 11; 33; 55. 
Do đó, có ba kết quả thuận lợi cho biến cố đó. 
Vì vậy, xác suất của biến cố đó là  $$ \frac{3}{50}.$$
c) Các kết quả thuận lợi cho biến cố “Số xuất hiện trên thẻ được rút ra là số chia hết cho cả 3 và 5” là: 15; 45; 75. 
Do đó, có ba kết quả thuận lợi cho biến cố đó. 
Vì vậy, xác suất của biến cố đó là  $$ \frac{3}{50}.$$
d) Các kết quả thuận lợi cho biến cố “Số xuất hiện trên thè được rút ra là số có hai chữ số và tổng hai chữ số đó là 7” là: 25; 43; 61. 
Do đó, có ba kết quả thuận lợi cho biến cố đó. 
Vì vậy, xác suất của biến cố đó là  $$ \frac{3}{50}.$$

========================================================================

https://khoahoc.vietjack.com/thi-online/giai-sbt-toan-8-canh-dieu-bai-24-phan-tich-va-xu-ly-du-lieu-thu-duoc-o-dang-bang-bieu-do-co-dap-an


\textbf{{QUESTION}}

Viết ngẫu nhiên một số tự nhiên có hai chữ số lớn hơn 60 và nhỏ hơn 80. Tính xác xuất của mỗi biến cố sau:
a) “Số tự nhiên được viết ra có chữ số hàng chục lớn hơn chữ số hàng đơn vị”;
b) “Số tự nhiên được viết ra có chữ số hàng chục gấp hai lần chữ số hàng đơn vị”.

\textbf{{ANSWER}}

Số các số tự nhiên có hai chữ số lớn hơn 60 và nhỏ hơn 80 là 19, và các số đó là: 61, 62, 63,..., 79. 
a) Các kết quả thuận lợi cho biến cố “Số tự nhiên được viết ra có chữ số hàng chục lớn hơn chữ số hàng đơn vị” là: 61, 62, 63, 64, 65, 70, 71, 72, 73, 74, 75, 76. 
Do đó, có mười hai kết quả thuận lợi cho biến cố đó. 
Vì vậy, xác suất của biến cố đó là  $$ \frac{12}{19}.$$
b) Các kết quả thuận lợi cho biến cố “Số tự nhiên được viết ra có chữ số hàng chục gấp hai lần chữ số hàng đơn vị” là 63. Do đó, có 1 kết quả thuận lợi cho biến cố đó. 
Vì vậy, xác xuất của biến cố đó là  $$ \frac{1}{19}.$$

========================================================================

https://khoahoc.vietjack.com/thi-online/giai-sbt-toan-8-canh-dieu-bai-24-phan-tich-va-xu-ly-du-lieu-thu-duoc-o-dang-bang-bieu-do-co-dap-an


\textbf{{QUESTION}}

Viết ngẫu nhiên một số tự nhiên có ba chữ số lớn hơn hoặc bằng 900.
a) Tính số phần tử của tập hợp E gồm các kết quả có thể xảy ra đối với số tự nhiên được viết ra.
b) Tính xác suất của biến cố “Số tự nhiên được viết ra là bình phương của một số tự nhiên”.

\textbf{{ANSWER}}

a) Tập hợp các số tự nhiên có ba chữ số lớn hơn hoặc bằng 900 là:
E = {900; 901; 902; 903; ...; 998; 999}.
Số phần tử của tập hợp E là: 999 ‒ 900 + 1 = 100.
b) Các kết quả thuận lợi cho biến cố “Số tự nhiên được viết ra là bình phương của một số tự nhiên” là: 900; 961. Do đó, có hai kết quả thuận lợi cho biến cố đó. 
Vì vậy, xác suất của biến cố đó là  2100=150.$$ \frac{2}{100}=\frac{1}{50}.$$

========================================================================

https://khoahoc.vietjack.com/thi-online/giai-sbt-toan-8-canh-dieu-bai-24-phan-tich-va-xu-ly-du-lieu-thu-duoc-o-dang-bang-bieu-do-co-dap-an


\textbf{{QUESTION}}

Một đội học sinh tham gia cuộc thi sáng tạo thanh thiếu niên nhi đồng toàn quốc năm 2022 có 4 học sinh lớp 7 là: An, Bình, Chi, Minh và 5 học sinh lớp 8 là: Phương, Hà, Ngọc, Nam, Thư. Chọn ngẫu nhiên một thi sinh trong đội học sinh tham gia cuộc thi đó.
a) Viết tập hợp A gồm các kết quả có thể xảy ra đối với thí sinh được chọn ra. Tính số phần tử của tập hợp A.
b) Tính xác suất của biến cố “Thí sinh được chọn ra là học sinh lớp 7”.
c) Tính xác suất của biến cố “Thí sinh được chọn ra là học sinh lớp 8”.

\textbf{{ANSWER}}

a) Tập hợp A gồm các kết quả có thể xảy ra đối với thí sinh được chọn ra là:
A = {An; Bình; Chi; Minh; Phương; Hà; Ngọc; Nam; Thư}. 
Tập hợp A có 9 phần tử.
b) Có 4 kết quả thuận lợi cho biến cố “Thí sinh được chọn ra là học sinh lớp 7” là: An, Bình, Chi, Minh. 
Vì vậy, xác suất của biến cố đó là 49.$$ \frac{4}{9}.$$
c) Có 5 kết quả thuận lợi cho biến cố “Thí sinh được chọn ra là học sinh lớp 8” là: Phương, Hà, Ngọc, Nam, Thư. 
Vì vậy, xác suất của biến cố đó là 59.$$ \frac{5}{9}.$$

========================================================================

https://khoahoc.vietjack.com/thi-online/giai-sbt-toan-8-canh-dieu-bai-24-phan-tich-va-xu-ly-du-lieu-thu-duoc-o-dang-bang-bieu-do-co-dap-an


\textbf{{QUESTION}}

Cho tập hợp A = {1; 2} và B = {3; 4; 5; 8}. Lập ra tất cả các số có hai chữ số  ¯ab,$$ \overline{ab},$$ trong đó a ∈ A và b ∈ B.
a) Có thể lập được bao nhiêu số  ¯ab,$$ \overline{ab},$$ như vậy?
b) Tính xác suất của biến cố “Số tự nhiên lập được là số chia hết cho 9”;
c) Tính xác suất của biến cố “Số tự nhiên lập được là số lớn hơn 14”.

\textbf{{ANSWER}}

a) Các số có hai chữ số  ¯ab$$ \overline{ab}$$ (a ∈ A và b ∈ B) lập được là: 13; 14; 15; 18; 23; 24; 25; 28. Do đó, có tất cả 8 số lập được.
b) Các kết quả thuận lợi cho biến cố “Số tự nhiên lập được là số chia hết cho 9” là: 18. Do đó, có một kết quả thuận lợi cho biến cố đó. 
Vì vậy, xác suất của biến cố đó là 18.$$ \frac{1}{8}.$$
c) Các kết quả thuận lợi cho biến cố “Số tự nhiên lập được là số lớn hơn 14” là: 15; 18; 23; 24; 25; 28. Do đó, có 6 kết quả thuận lợi cho biến cố đó. 
Vì vậy, xác suất của biến cố đó là 68=34.$$ \frac{6}{8}=\frac{3}{4}.$$

========================================================================

https://khoahoc.vietjack.com/thi-online/giai-sbt-toan-8-kntt-bai-4-phep-nhan-da-thuc-co-dap-an


\textbf{{QUESTION}}

Thực hiện phép nhân:
a) 0,5x2y(4x2 – 6xy + y2);

\textbf{{ANSWER}}

a) 0,5x2y(4x2 – 6xy + y2) 
= 0,5x2y.4x2 ‒ 0,5x2y.6xy + 0,5x2y.y2
= 2x4y ‒ 3x3y2 + 0,5x2y3.

========================================================================

https://khoahoc.vietjack.com/thi-online/giai-sbt-toan-8-kntt-bai-4-phep-nhan-da-thuc-co-dap-an


\textbf{{QUESTION}}

b) (3x3−6x2y+9xy2)(−23xy2)$$ \left(3{x}^{3}-6{x}^{2}y+9x{y}^{2}\right)\left(-\frac{2}{3}x{y}^{2}\right)$$

\textbf{{ANSWER}}

b)  $$ \left(3{x}^{3}-6{x}^{2}y+9x{y}^{2}\right)\left(-\frac{2}{3}x{y}^{2}\right)$$
$$ =3{x}^{3}\cdot \left(-\frac{2}{3}x{y}^{2}\right)+\left(-6{x}^{2}y\right)\cdot \left(-\frac{2}{3}x{y}^{2}\right)+9x{y}^{2}\cdot \left(-\frac{2}{3}x{y}^{2}\right)$$
= ‒2x4y2 + 4x3y3 ‒ 6x2y4.

========================================================================

https://khoahoc.vietjack.com/thi-online/giai-sbt-toan-8-kntt-bai-4-phep-nhan-da-thuc-co-dap-an


\textbf{{QUESTION}}

Rút gọn rồi tính giá trị của biểu thức.
a) A = x(x – y + 1) + y(x + y – 1) tại x = 3; y = 3;

\textbf{{ANSWER}}

a) Ta có:
A = x(x – y + 1) + y(x + y – 1)
= x.x ‒ x.y + x.1 + y.x + y.y ‒ y.1
= x2 ‒ xy + x + xy + y2 ‒ y
= x2 + y2 + x ‒ y + (‒xy+ xy)
= x2 + y2 + x ‒ y.
Tại x = 3; y = 3 ta có: 
A = 32 + 32 + 3 ‒ 3 = 18.

========================================================================

https://khoahoc.vietjack.com/thi-online/giai-sbt-toan-8-kntt-bai-4-phep-nhan-da-thuc-co-dap-an


\textbf{{QUESTION}}

b) B = x(x – y2) + y(x2 – y) – (x + y)(x – y) tại x = 2; y = –0,5.

\textbf{{ANSWER}}

b) B = x(x – y2) + y(x2 – y) – (x + y)(x – y)
= x.x ‒ x.y2 + y.x2 ‒ y.y ‒ [x.(x – y) + y(x – y)] 
= x2 ‒ xy2 + x2y ‒ y2 ‒ [x2 – xy + xy – y2]
= x2 ‒ xy2 + x2y ‒ y2 ‒ [x2 – y2]
= x2 ‒ xy2 + x2y ‒ y2 ‒ x2 + y2
= (x2 ‒ x2) ‒ xy2 + x2y + (‒ y2 + y2) 
= x2y ‒ xy2.
Tại x = 2; y = –0,5 ta có:
B = 22.(–0,5) ‒ 2.(–0,5)2 = –2 – 0,5 = ‒2,5.

========================================================================

https://khoahoc.vietjack.com/thi-online/giai-sbt-toan-8-kntt-bai-4-phep-nhan-da-thuc-co-dap-an


\textbf{{QUESTION}}

Thực hiện phép tính:
a) (x – 2y)(x2z + 2xyz + 4y2z);

\textbf{{ANSWER}}

a) (x – 2y)(x2z + 2xyz + 4y2z) 
= x.(x2z + 2xyz + 4y2z) – 2y.(x2z + 2xyz + 4y2z) 
= x3z + 2x2yz + 4xy2z ‒ 2x2yz ‒ 4xy2z ‒ 8y3z
= x3z + (2x2yz ‒ 2x2yz) + (4xy2z ‒ 4xy2z) ‒ 8y3z
= x3z ‒ 8y3z.

========================================================================

https://khoahoc.vietjack.com/thi-online/trac-nghiem-tong-hop-on-thi-tot-nghiep-thpt-mon-toan-chu-de-6-hinh-hoc-va-do-luong-trong-khong-gian/145283


\textbf{{QUESTION}}

Bạn hãy đọc đoạn văn 1 trên và trả lời câu hỏi.
a) Nếu a song song với b thì góc giữa hai đường thẳng a và c bằng góc giữa hai đường thẳng b và c.

\textbf{{ANSWER}}

Đúng

========================================================================

https://khoahoc.vietjack.com/thi-online/trac-nghiem-tong-hop-on-thi-tot-nghiep-thpt-mon-toan-chu-de-6-hinh-hoc-va-do-luong-trong-khong-gian/145283


\textbf{{QUESTION}}

Bạn hãy đọc đoạn văn 1 trên và trả lời câu hỏi.
b) Nếu a và b cùng song song với c thì a song song với b.

\textbf{{ANSWER}}

Đúng

========================================================================

https://khoahoc.vietjack.com/thi-online/trac-nghiem-tong-hop-on-thi-tot-nghiep-thpt-mon-toan-chu-de-6-hinh-hoc-va-do-luong-trong-khong-gian/145283


\textbf{{QUESTION}}

Bạn hãy đọc đoạn văn 1 trên và trả lời câu hỏi.
c) Nếu a và b cùng vuông góc với c thì a song song với b.

\textbf{{ANSWER}}

Sai

========================================================================

https://khoahoc.vietjack.com/thi-online/trac-nghiem-tong-hop-on-thi-tot-nghiep-thpt-mon-toan-chu-de-6-hinh-hoc-va-do-luong-trong-khong-gian/145283


\textbf{{QUESTION}}

Bạn hãy đọc đoạn văn 1 trên và trả lời câu hỏi.
d) Nếu a và b cùng vuông góc với c thì a vuông góc với b.

\textbf{{ANSWER}}

Sai

========================================================================

https://khoahoc.vietjack.com/thi-online/trac-nghiem-tong-hop-on-thi-tot-nghiep-thpt-mon-toan-chu-de-6-hinh-hoc-va-do-luong-trong-khong-gian/145283


\textbf{{QUESTION}}

Bạn hãy đọc đoạn văn 2 trên và trả lời câu hỏi.
a) Nếu (P)$({\rm{P}})$ và (Q)$({\rm{Q}})$ có ít nhất một điểm chung thì (P)$({\rm{P}})$ và (Q)$({\rm{Q}})$ có vô số điểm chung.

\textbf{{ANSWER}}

Đúng

========================================================================

https://khoahoc.vietjack.com/thi-online/15-cau-trac-nghiem-ham-so-y-ax-b-co-dap-an-nhan-biet


\textbf{{QUESTION}}

Tìm các giá trị của m để hàm số y = (m2 − m)x + 1 đồng biến trên R.
A. 0 < m < 1
B. m ∈ (−∞; 0) ∪ (1; +∞)
C. m = 0 hoặc m =1
D. Không tồn tại

\textbf{{ANSWER}}

Hàm số đã cho đồng biến trên R khi và chỉ khi hệ số góc k = m2 – m > 0.
Giải bất phương trình m2 – m > 0⇔ m < 0 hoặc m > 1.
Đáp án cần chọn là: B

========================================================================

https://khoahoc.vietjack.com/thi-online/15-cau-trac-nghiem-ham-so-y-ax-b-co-dap-an-nhan-biet


\textbf{{QUESTION}}

Tìm m để hàm số y = (2m + 1)x + m − 3 đồng biến trên R.
A. m > 12$$ \frac{1}{2}$$

B. m < 12$$ \frac{1}{2}$$
C. m < -12$$ -\frac{1}{2}$$
D. m >-12$$ -\frac{1}{2}$$

\textbf{{ANSWER}}

Hàm số bậc nhất y = ax + b đồng biến ⇔ a > 0 ⇒ 2m + 1 > 0 ⇔ m > 12$$ \frac{1}{2}$$
Đáp án cần chọn là: D

========================================================================

https://khoahoc.vietjack.com/thi-online/15-cau-trac-nghiem-ham-so-y-ax-b-co-dap-an-nhan-biet


\textbf{{QUESTION}}

Tìm m để hàm số y = m(x +2) – x(2m + 1) nghịch biến trên R
A. m > -2
B. m < -12$$ -\frac{1}{2}$$
C. m > -1
D. m > -12$$ -\frac{1}{2}$$

\textbf{{ANSWER}}

Ta có: y = m(x + 2) − x(2m + 1) = (−1 − m)x + 2m.
Hàm số bậc nhất y = ax + b nghịch biến ⇔ a < 0 ⇒ −1 – m < 0 ⇔ m > −1.
Đáp án cần chọn là: C

========================================================================

https://khoahoc.vietjack.com/thi-online/15-cau-trac-nghiem-ham-so-y-ax-b-co-dap-an-nhan-biet


\textbf{{QUESTION}}

Tọa độ giao điểm của hai đường thẳng y=1-3x4$$ y=\frac{1-3x}{4}$$ và y=-(x3+1)$$ y=-\left(\frac{x}{3}+1\right)$$ là:
A. (0; −1).
B. (2; −3).
C. (0; 14).
D. (3; −2).

\textbf{{ANSWER}}

Phương trình hoành độ của hai đường thẳng là: 1-3x4=-(x3+1)$$ \frac{1-3x}{4}=-\left(\frac{x}{3}+1\right)$$⇔-512x+54=0⇔x=3⇒y=-2$$ \Leftrightarrow -\frac{5}{12}x+\frac{5}{4}=0\Leftrightarrow x=3\Rightarrow y=-2$$
Vậy tọa độ giao điểm cần tìm là (3; -2).
Đáp án cần chọn là: D

========================================================================

https://khoahoc.vietjack.com/thi-online/15-cau-trac-nghiem-ham-so-y-ax-b-co-dap-an-nhan-biet


\textbf{{QUESTION}}

Cho hàm số y = 2mx – m – 1 (d). Tìm m để đường thẳng (d) đi qua điểm A(1; 2).
A. m < 3
B. m = −3
C. m = 3
D. Không tồn tại

\textbf{{ANSWER}}

Điểm A(1; 2) thuộc đường thẳng (d) khi và chỉ khi 2 = 2m.1 – m – 1 ⇔ m = 3.
Đáp án cần chọn là: C

========================================================================

https://khoahoc.vietjack.com/thi-online/20-cau-trac-nghiem-toan-10-canh-dieu-giai-tam-giac-co-dap-an-phan-2/112075


\textbf{{QUESTION}}

Tam giác ABC có BC = a, CA = b, AB = c và có diện tích S. Nếu tăng cạnh BC lên 3 lần đồng thời tăng cạnh CA lên 3 lần và giữ nguyên độ lớn của góc C thì khi đó diện tích tam giác mới được tạo nên bằng:

\textbf{{ANSWER}}

Đáp án đúng là: C
Có S = $$ \frac{1}{2}$$ BC.CA.sinC
Gọi S’ là diện tích tam giác khi tăng cạnh BC lên 3 lần đồng thời tăng cạnh CA lên 3 lần và giữ nguyên độ lớn góc C
Ta có: S’ =$$ \frac{1}{2}$$  .3BC.3CA.sinC = 9 . $$ \frac{1}{2}$$ BC.CA.sinC = 9S.

========================================================================

https://khoahoc.vietjack.com/thi-online/20-cau-trac-nghiem-toan-10-canh-dieu-giai-tam-giac-co-dap-an-phan-2/112075


\textbf{{QUESTION}}

Cho tam giác ABC có AB = 10 cm, AC = 20 cm và có diện tích là 90 cm2. Giá trị sinA là:
A. √32$$ \frac{\sqrt{3}}{2}$$ ;

\textbf{{ANSWER}}

Đáp án đúng là: C
Ta có: S = $$ \frac{1}{2}$$ AB.AC.sinA $$ \Rightarrow $$ sinA = $$ \frac{2S}{AB.AC}$$ = $$ \frac{2.90}{10.20}$$ = $$ \frac{9}{10}$$ .

========================================================================

https://khoahoc.vietjack.com/thi-online/bai-tap-toan-8-chu-de-11-phep-chia-cac-phan-thuc-dai-so-co-dap-an


\textbf{{QUESTION}}

Làm tính chia các phân thức $$ \frac{7xy}{3x+1}:\frac{14{x}^{2}y}{6x+2}$$

\textbf{{ANSWER}}

$$ \frac{7xy}{3x+1}:\frac{14{x}^{2}y}{6x+2}=\frac{7xy}{3x+1}.\frac{6x+2}{14{x}^{2}y}=\frac{7xy}{3x+1}.\frac{2.\left(3x+1\right)}{14{x}^{2}y}=\frac{1}{x}.$$

========================================================================

https://khoahoc.vietjack.com/thi-online/bai-tap-toan-8-chu-de-11-phep-chia-cac-phan-thuc-dai-so-co-dap-an


\textbf{{QUESTION}}

Làm tính chia các phân thức $$ \frac{34{x}^{2}{y}^{3}}{2x{y}^{2}+2{y}^{2}}:\frac{17xy}{3x+3}$$

\textbf{{ANSWER}}

$$ \frac{34{x}^{2}{y}^{3}}{2x{y}^{2}+2{y}^{2}}:\frac{17xy}{3x+3}=\frac{34{x}^{2}{y}^{3}}{2x{y}^{2}+2{y}^{2}}.\frac{3x+3}{17xy}=\frac{34{x}^{2}{y}^{3}}{2{y}^{2}\left(x+1\right)}.\frac{3\left(x+1\right)}{17xy}=\frac{102{x}^{2}{y}^{3}.\left(x+1\right)}{34x{y}^{3}\left(x+1\right)}=3x$$

========================================================================

https://khoahoc.vietjack.com/thi-online/bai-tap-toan-8-chu-de-11-phep-chia-cac-phan-thuc-dai-so-co-dap-an


\textbf{{QUESTION}}

Làm tính chia các phân thức $$ \frac{{x}^{3}-27}{x+3}:\left({x}^{2}-6x+9\right)$$

\textbf{{ANSWER}}

$$ \frac{{x}^{3}-27}{x+3}:\left({x}^{2}-6x+9\right)=\frac{{x}^{3}-{3}^{3}}{x+3}.\frac{1}{{x}^{2}-6x+9}=\frac{\left(x-3\right)\left({x}^{2}+3x+9\right)}{x+3}.\frac{1}{{\left(x-3\right)}^{2}}=$$$$ \frac{{x}^{2}+3x+9}{\left(x+3\right)\left(x-3\right)}=\frac{{x}^{2}+3x+9}{{x}^{2}-9}$$

========================================================================

https://khoahoc.vietjack.com/thi-online/bai-tap-toan-8-chu-de-11-phep-chia-cac-phan-thuc-dai-so-co-dap-an


\textbf{{QUESTION}}

Làm tính chia các phân thức (x2+2x+1):x2−12x+3$$ \left({x}^{2}+2x+1\right):\frac{{x}^{2}-1}{2x+3}$$
(x2+2x+1):x2−12x+3
(x2+2x+1):x2−12x+3
(x2+2x+1)
(
(
x2+2x+1
x2
x
2
+
2
x
+
1
)
)
:
x2−12x+3
x2−1
x2−1
x2
x
2
−
1
2x+3
2x+3


2x+3
2x+3
2x+3
2
x
+
3

\textbf{{ANSWER}}

(x2+2x+1):x2−12x+3=(x2+2x+1).2x+3x2−1=(x+1)2.2x+3(x−1)(x+1)=(x+1)(2x+3)x−1$$ \left({x}^{2}+2x+1\right):\frac{{x}^{2}-1}{2x+3}=\left({x}^{2}+2x+1\right).\frac{2x+3}{{x}^{2}-1}={\left(x+1\right)}^{2}.\frac{2x+3}{\left(x-1\right)\left(x+1\right)}=\frac{\left(x+1\right)\left(2x+3\right)}{x-1}$$

========================================================================

https://khoahoc.vietjack.com/thi-online/bai-tap-toan-8-chu-de-11-phep-chia-cac-phan-thuc-dai-so-co-dap-an


\textbf{{QUESTION}}

Chia các phân thức sau 9x2−43x+1:3x+26x2+2x$$ \frac{9{x}^{2}-4}{3x+1}:\frac{3x+2}{6{x}^{2}+2x}$$

\textbf{{ANSWER}}

9x2−43x+1:3x+26x2+2x=(3x)2−223x+1.6x2+2x3x+2=(3x−2)(3x+2)3x+1.2x(3x+1)3x+2=2x(3x−2)$$ \frac{9{x}^{2}-4}{3x+1}:\frac{3x+2}{6{x}^{2}+2x}=\frac{{\left(3x\right)}^{2}-{2}^{2}}{3x+1}.\frac{6{x}^{2}+2x}{3x+2}=\frac{\left(3x-2\right)\left(3x+2\right)}{3x+1}.\frac{2x\left(3x+1\right)}{3x+2}=2x\left(3x-2\right)$$

========================================================================

https://khoahoc.vietjack.com/thi-online/giai-sgk-toan-lop-11-kntt-tap-2-bai-20-ham-so-mu-va-ham-so-logarit-co-dap-an


\textbf{{QUESTION}}

Sự tăng trưởng dân số được ước tính theo công thức tăng trưởng mũ sau: 
A = Pert, 
trong đó P là dân số của năm lấy làm mốc, A là dân số sau t năm, r là tỉ lệ tăng dân số hằng năm. Biết rằng vào năm 2020, dân số Việt Nam khoảng 97,34 triệu người và tỉ lệ tăng dân số là 0,91% (theo danso.org). Nếu tỉ lệ tăng dân số này giữ nguyên, hãy ước tính dân số Việt Nam vào năm 2050.

\textbf{{ANSWER}}

Sau bài học, ta giải quyết được bài toán như sau:
Theo bài ra ta có P = 97,34; r = 0,91%.
Từ năm 2020 đến năm 2050 là 30 năm nên t = 30. 
Ước tính dân số Việt Nam vào năm 2050 là 
A = Pert = 97,34 ∙ e0,91% ∙ 30 ≈ 127,9 (triệu người).

========================================================================

https://khoahoc.vietjack.com/thi-online/giai-sgk-toan-lop-11-kntt-tap-2-bai-20-ham-so-mu-va-ham-so-logarit-co-dap-an


\textbf{{QUESTION}}

a) Tính y = 2x khi x lần lượt nhận các giá trị – 1; 0; 1. Với mỗi giá trị của x có bao nhiêu giá trị của y = 2x tương ứng?

\textbf{{ANSWER}}

a) Ta có:
+ Với x = – 1 thì y = 2– 1 =  12$$ \frac{1}{2}$$. 
+ Với x = 0 thì y = 20 = 1. 
+ Với x = 1 thì y = 21 = 2. 
Ta nhận thấy với mỗi giá trị của x có duy nhất một giá trị của y = 2x tương ứng.

========================================================================

https://khoahoc.vietjack.com/thi-online/giai-sgk-toan-lop-11-kntt-tap-2-bai-20-ham-so-mu-va-ham-so-logarit-co-dap-an


\textbf{{QUESTION}}

b) Với những giá trị nào của x, biểu thức y = 2x có nghĩa?

\textbf{{ANSWER}}

b) Biểu thức y = 2x có nghĩa với mọi giá trị của x.

========================================================================

https://khoahoc.vietjack.com/thi-online/giai-sgk-toan-lop-11-kntt-tap-2-bai-20-ham-so-mu-va-ham-so-logarit-co-dap-an


\textbf{{QUESTION}}

Trong các hàm số sau, những hàm số nào là hàm số mũ? Khi đó hãy chỉ ra cơ số. 
a)   y=(√2)x$$ y={\left(\sqrt{2}\right)}^{x}$$; 
b) y = 2– x;

\textbf{{ANSWER}}

a) Hàm số  y=(√2)x$$ y={\left(\sqrt{2}\right)}^{x}$$ là hàm số mũ với cơ số  √2$$ \sqrt{2}$$. 
b) Ta có y = 2– x = (2– 1)x =  (12)x$$ {\left(\frac{1}{2}\right)}^{x}$$. Do đó, hàm số đã cho là hàm số mũ với cơ số  12$$ \frac{1}{2}$$.

========================================================================

https://khoahoc.vietjack.com/thi-online/giai-sgk-toan-lop-11-kntt-tap-2-bai-20-ham-so-mu-va-ham-so-logarit-co-dap-an


\textbf{{QUESTION}}

Trong các hàm số sau, những hàm số nào là hàm số mũ? Khi đó hãy chỉ ra cơ số. 
c)  y=8x3$$ y={8}^{\frac{x}{3}}$$; 
d) y = x– 2.

\textbf{{ANSWER}}

c) Ta có  y=8x3=(3√8)x=2x$$ y={8}^{\frac{x}{3}}={\left(\sqrt[3]{8}\right)}^{x}={2}^{x}$$. Do đó, hàm số đã cho là hàm số mũ với cơ số 2. 
d) Hàm số y = x– 2 không phải là hàm số mũ.

========================================================================

https://khoahoc.vietjack.com/thi-online/20-cau-trac-nghiem-toan-10-canh-dieu-bat-phuong-trinh-bac-hai-mot-an-co-dap-an-phan-2/111991


\textbf{{QUESTION}}

Tập nghiệm của bất phương trình – x2 + 6x + 7 > 0.

\textbf{{ANSWER}}

Đáp án đúng là: A
Tam thức bậc hai – x2 + 6x + 7 có hai nghiệm x = – 1, x = 7 và có hệ số a = – 1 < 0. 
Sử dụng định lí về dấu của tam thức bậc hai, ta thấy tập hợp những giá trị của x sao cho tam thức – x2 + 6x + 7 mang dấu “+” là (– 1; 7). 
Vậy tập nghiệm của bất phương trình – x2 + 6x + 7 > 0 là S = (– 1; 7).

========================================================================

https://khoahoc.vietjack.com/thi-online/20-cau-trac-nghiem-toan-10-canh-dieu-bat-phuong-trinh-bac-hai-mot-an-co-dap-an-phan-2/111991


\textbf{{QUESTION}}

Tập nghiệm của bất phương trình (x2 – 3x + 1)2 + 3x2 – 9x + 5 > 0 là

\textbf{{ANSWER}}

Đáp án đúng là: C
Ta có (x2 – 3x + 1)2 + 3x2 – 9x + 5 > 0 
$$ \Leftrightarrow $$ (x2 – 3x + 1)2 + 3(x2 – 3x + 1) + 2 > 0
Đặt x2 – 3x + 1 = t. 
Khi đó ta có: t2 + 3t + 2 > 0 (*). 
Giải bất phương trình (*) ta được: $$ \left[\begin{array}{l}t<-2\\ t>-1\end{array}\right.$$.
$$ \Leftrightarrow $$ $$ \left[\begin{array}{c}{x}^{2}-3x+1<-2\\ {x}^{2}-3x+1>-1\end{array}\right.$$$$ \Leftrightarrow $$ $$ \left[\begin{array}{c}{x}^{2}-3x+3<0\\ {x}^{2}-3x+2>0\end{array}\right.$$ 
$$ \Leftrightarrow $$ $$ \left[\begin{array}{l}vô\text{\hspace{0.17em}\hspace{0.17em}}nghiêm\\ \left[\begin{array}{c}x<1\\ x>2\end{array}\right.\end{array}\right.$$ $$ \Leftrightarrow \left[\begin{array}{c}x<1\\ x>2\end{array}\right.$$. 
Vậy tập nghiệm của bất phương trình đã cho là: S = (−$$ \infty $$; 1) $$ \cup $$ (2; +$$ \infty $$).

========================================================================

https://khoahoc.vietjack.com/thi-online/7881-cau-trac-nghiem-tong-hop-mon-toan-2023-cuc-hay-co-dap-an/122523


\textbf{{QUESTION}}

Có bao nhiêu số có ba chữ số mà tổng các chữ số của nó bằng 6?

\textbf{{ANSWER}}

Bộ ba số có tổng bằng 6 là: (1; 0; 5); (1; 1; 4); (1; 2; 3); (2; 0; 4); (2; 2; 2); (3; 0; 3); (6; 0; 0).
Số có 3 chữ số mà tổng các chữ số của nó bằng 6, đó là các số: 105; 150; 501; 510; 114; 141; 411; 123; 132; 213; 231; 312; 321; 204; 240; 402; 420; 222; 303; 330; 600; ...
Vậy có 21 số thỏa mãn yêu cầu bài toán.

========================================================================

https://khoahoc.vietjack.com/thi-online/7881-cau-trac-nghiem-tong-hop-mon-toan-2023-cuc-hay-co-dap-an/122523


\textbf{{QUESTION}}

Trong một phép chia, số chia là 1009, số thương là 673, số dư là số lớn nhất có được trong phép chia đó. Tìm số bị chia.

\textbf{{ANSWER}}

Vì số dư luôn bé hơn số bị chia và số chia nên số dư lớn nhất có thể thỏa mãn đề là 1008.
Vậy số bị chia bằng:
      (1009 × 673) + 1008 = 680 065.

========================================================================

https://khoahoc.vietjack.com/thi-online/7881-cau-trac-nghiem-tong-hop-mon-toan-2023-cuc-hay-co-dap-an/122523


\textbf{{QUESTION}}

Hai điểm A và B cách nhau 72 km. Một xe ô tô đi từ A đến B và một xe đạp đi từ B đến A xuất phát cùng một lúc. Sau 1 giờ 12 phút họ gặp nhau. Sau đó ô tô tiếp tục đi về phía B và quay trở lại A ngay với vận tốc không đổi và gặp xe đạp sau 48 phút kể từ lần gặp nhau lần trước. Tính vận tốc mỗi xe.

\textbf{{ANSWER}}

Đổi 1 giờ 12 phút = 1,2 giờ.
Gọi điểm ô tô và xe đạp gặp nhau lần đầu là C, lần 2 là D ta có: Ô tô đi đến D mất 1 giờ 12 phút + 48 phút = 2giờ. 
Vậy sau 2 giờ ô tô đi được quãng đường dài hơn xe đạp là chính quãng đường AB dài 72 km. 
Vậy vận tốc ô tô lớn hơn xe đạp là 72 : 2 = 36 (km/h). 
Mặt khác tổng vận tốc xe đạp và ô tô là 72 : 1,2 = 60 (km/h)
Vậy vận tốc xe đạp = (60 ‒ 36) : 2 = 12 (km/h); vận tốc ô tô = 12 + 36 = 48 (km/h).

========================================================================

https://khoahoc.vietjack.com/thi-online/7881-cau-trac-nghiem-tong-hop-mon-toan-2023-cuc-hay-co-dap-an/122523


\textbf{{QUESTION}}

Một bà mẹ sinh con trai lúc 26 tuổi và sinh con gái lúc 31 tuổi. Tính tuổi hiện nay của mỗi người con, biết tổng số tuổi hiện nay của hai người con là 61 tuổi.

\textbf{{ANSWER}}

Số tuổi của con trai hơn số tuổi của con gái là:
                31 ‒ 26 = 5 (tuổi)
Số tuổi của con trai là:
                 (61 + 5) : 2 = 33 (tuổi)
Số tuổi của con gái là:
                  61 ‒ 33 = 28 (tuổi)
Đáp số: Con trai : 33 tuổi
             Con gái: 28 tuổi.

========================================================================

https://khoahoc.vietjack.com/thi-online/7881-cau-trac-nghiem-tong-hop-mon-toan-2023-cuc-hay-co-dap-an/122523


\textbf{{QUESTION}}

Một căn phòng hình chữ nhật có chiều dài 6m; chiều rộng bằng 23$$ \frac{2}{3}$$  chiều dài.
a) Tính diện tích căn phòng.

\textbf{{ANSWER}}

Chiều rộng hình chữ nhật là 6 × 23$$ \frac{2}{3}$$   = 4 (m)
a) Diện tích căn phòng là 6 × 4 = 24 (m2)

========================================================================

https://khoahoc.vietjack.com/thi-online/bo-10-de-kiem-tra-giua-hoc-ki-2-toan-10-co-dap-an-moi-nhat/92993


\textbf{{QUESTION}}

A. m = 2.
B. m = 3.
C. m = 4.
D. m = 1.

\textbf{{ANSWER}}

Chọn đáp án C
Ta có: $$ \left\{\begin{array}{l}x-3\le 0\\ m-x\le 1\end{array}\right.\Leftrightarrow \left\{\begin{array}{l}x\le 3\\ x\ge m-1\end{array}\right.$$.

========================================================================

https://khoahoc.vietjack.com/thi-online/bo-10-de-kiem-tra-giua-hoc-ki-2-toan-10-co-dap-an-moi-nhat/92993


\textbf{{QUESTION}}

A. M(0 ;12)$$ M\left(0\text{\hspace{0.17em}};\frac{1}{2}\right)$$.
B. M(0 ; −12)$$ M\left(0\text{\hspace{0.17em}};\text{\hspace{0.17em}}-\frac{1}{2}\right)$$.
C. M(2 ; −112)$$ M\left(2\text{\hspace{0.17em}};\text{\hspace{0.17em}}-\frac{11}{2}\right)$$.
D. M(0 ;112)$$ M\left(0\text{\hspace{0.17em}};\frac{11}{2}\right)$$.

\textbf{{ANSWER}}

Chọn đáp án B
Tọa độ giao điểm M là nghiệm của hệ: $$ \left\{\begin{array}{l}5x+2y+1=0\\ 3x-2y-1=0\end{array}\right.\Leftrightarrow \left\{\begin{array}{l}5x+2y=-1\\ 3x-2y=1\end{array}\right.$$$$ \Leftrightarrow \left\{\begin{array}{l}x=0\\ y=-\frac{1}{2}\end{array}\right.$$.
$$ \Rightarrow M\left(0;-\frac{1}{2}\right)$$.

========================================================================

https://khoahoc.vietjack.com/thi-online/bo-10-de-kiem-tra-giua-hoc-ki-2-toan-10-co-dap-an-moi-nhat/92993


\textbf{{QUESTION}}

A. |a+b|≤|a|+|b|$$ \left|a+b\right|\le \left|a\right|+\left|b\right|$$.
B. |x|<a⇔−a<x<a (a>0)$$ \left|x\right|<a\Leftrightarrow -a<x<a\quad (a>0)$$.
C. a>b⇔ac>bc, (∀c∈ℝ)$$ a>b\Leftrightarrow ac>bc,\quad \left(\forall c\in \mathbb{R} \right)$$.
D. a+b≥2√ab, (a≥0,b≥0)$$ a+b\ge 2\sqrt{ab},\quad \left(a\ge \mathrm{0,}b\ge 0\right)$$.

\textbf{{ANSWER}}

Chọn đáp án C
Các mệnh đề A, B đều đúng theo tính chất của bất đẳng thức chứa dấu giá trị tuyệt đối.
Mệnh đề D đúng theo bất đẳng thức Cô-si cho 2 số không âm a và b.
Mệnh đề C sai khi c < 0 (vì khi nhân 2 vế của một bất đẳng thức với một số âm thì ta được bất đẳng thức mới đổi chiều bất đẳng thức đã cho).

========================================================================

https://khoahoc.vietjack.com/thi-online/bo-10-de-kiem-tra-giua-hoc-ki-2-toan-10-co-dap-an-moi-nhat/92993


\textbf{{QUESTION}}

A. a+c>b+d$$ a+c>b+d$$.
B. a−c>b−d$$ a-c>b-d$$.
C. ac > bd.
D. a2>b2$$ {a}^{2}>{b}^{2}$$.

\textbf{{ANSWER}}

Chọn đáp án A
Có {a>bc>d⇒a+c>b+d$$ \left\{\begin{array}{l}a>b\\ c>d\end{array}\right.\Rightarrow a+c>b+d$$ (đúng theo tính chất cộng vế với vế của hai bất đẳng thức cùng chiều), nên phương án A đúng.
Có {3>15>2$$ \left\{\begin{array}{l}3>1\\ 5>2\end{array}\right.$$ và 3−5>1−2$$ 3-5>1-2$$ (sai), nên nên phương án B sai.
Có {3>1−1>−2$$ \left\{\begin{array}{l}3>1\\ -1>-2\end{array}\right.$$ suy ra 3.(−1)>1.(−2)$$ 3.\left(-1\right)>1.\left(-2\right)$$ (sai), nên phương án C sai.
Có −2>−3⇒(−2)2>(−3)2$$ -2>-3\Rightarrow {\left(-2\right)}^{2}>{\left(-3\right)}^{2}$$ (sai), nên phương án D sai.

========================================================================

https://khoahoc.vietjack.com/thi-online/bo-10-de-kiem-tra-giua-hoc-ki-2-toan-10-co-dap-an-moi-nhat/92993


\textbf{{QUESTION}}

A. a>b⇔a−b>0$$ a>b\Leftrightarrow a-b>0$$.
B. a>b>0⇒1a<1b$$ a>b>0\Rightarrow \frac{1}{a}<\frac{1}{b}$$.
C. a>b⇔a3>b3$$ a>b\Leftrightarrow {a}^{3}>{b}^{3}$$.
D. a>b⇔a2>b2$$ a>b\Leftrightarrow {a}^{2}>{b}^{2}$$.

\textbf{{ANSWER}}

Chọn đáp án D
Các mệnh đề A, B, C đúng.
Mệnh đề D sai. Ta có phản ví dụ: -2 > -5 nhưng (−2)2=4<25=(−5)2$$ {\left(-2\right)}^{2}=4<25={\left(-5\right)}^{2}$$.

========================================================================

https://khoahoc.vietjack.com/thi-online/10-cau-trac-nghiem-toan-6-canh-dieu-bai-5-phep-tinh-luy-thua-voi-so-mu-tu-nhien-co-dap-an-thong-hieu


\textbf{{QUESTION}}

A.53< 35
B.34>25
C.43= 26
D.43>82

\textbf{{ANSWER}}

Ta có:
+) 53= 125; 35= 243 suy ra 53< 35
+) 34= 81; 25= 32 suy ra 34>25nên B đúng.
+) 43= 64; 26= 64 suy ra 43= 26nên C đúng.
+) 43= 64; 82= 64 suy ra 43= 82nên D sai.
Chọn đáp án D.

========================================================================

https://khoahoc.vietjack.com/thi-online/10-cau-trac-nghiem-toan-6-canh-dieu-bai-5-phep-tinh-luy-thua-voi-so-mu-tu-nhien-co-dap-an-thong-hieu


\textbf{{QUESTION}}

A.n = 2     
B.n = 3     
C.n = 4     
D.n = 8

\textbf{{ANSWER}}

Ta có: 34= 81 nên 3n= 34, do đó n = 4. 
Chọn đáp án C.

========================================================================

https://khoahoc.vietjack.com/thi-online/10-cau-trac-nghiem-toan-6-canh-dieu-bai-5-phep-tinh-luy-thua-voi-so-mu-tu-nhien-co-dap-an-thong-hieu


\textbf{{QUESTION}}

A.2 017 = 2 . 104+ 102+ 7 . 100
B.2 017 = 2 . 103+ 10 + 7 . 100
C.2 017 = 2 . 104+ 102+ 7 . 10
D.2 017 = 2 . 103+ 102+ 7 . 100

\textbf{{ANSWER}}

Ta có: 2 017 = 2 . 1 000 + 0 . 100 + 1 . 10 + 7 . 1 = 2 . 103+ 10 + 7 . 100.
Chọn đáp án B.

========================================================================

https://khoahoc.vietjack.com/thi-online/10-cau-trac-nghiem-toan-6-canh-dieu-bai-5-phep-tinh-luy-thua-voi-so-mu-tu-nhien-co-dap-an-thong-hieu


\textbf{{QUESTION}}

A.66
B.67
C.68
D.69

\textbf{{ANSWER}}

Ta có: 63. 2 . 64. 3 = (63. 64) . (2 . 3) = 63 + 4. 6 = 67. 61= 67 + 1= 68. 
Chọn đáp án C.

========================================================================

https://khoahoc.vietjack.com/thi-online/10-cau-trac-nghiem-toan-6-canh-dieu-bai-5-phep-tinh-luy-thua-voi-so-mu-tu-nhien-co-dap-an-thong-hieu


\textbf{{QUESTION}}

A.33. 34= 312
B.30= 0
C.42: 23= 2
D.55: 5 = 14

\textbf{{ANSWER}}

Ta có:
33. 34= 33+4= 37nên đáp án A sai
30= 1 (quy ước) nên đáp án B sai
42: 23= 4 . 4 : 23= 2 . 2 . 2 . 2 : 23= 24. 23= 24 – 3 = 21= 2 nên C đúng
55: 5 = 55 – 1= 54≠ 14nên D sai
Chọn đáp án C.

========================================================================

https://khoahoc.vietjack.com/thi-online/giai-toan-6-chuong-2-goc/16127


\textbf{{QUESTION}}

Góc là gì?

\textbf{{ANSWER}}

Góc là hình tạo bởi hai tia chung gốc.

========================================================================

https://khoahoc.vietjack.com/thi-online/giai-toan-6-chuong-2-goc/16127


\textbf{{QUESTION}}

Góc bẹt là gì?

\textbf{{ANSWER}}

Góc bẹt là góc có hai cạnh là hai tia đối nhau.

========================================================================

https://khoahoc.vietjack.com/thi-online/giai-toan-6-chuong-2-goc/16127


\textbf{{QUESTION}}

Nêu hình ảnh thực tế của góc, góc bẹt.

\textbf{{ANSWER}}

Hình ảnh thực tế của góc vuông như: góc tờ giấy, góc mặt bàn hình chữ nhật, góc viên gạch vuông nát nền nhà ...
Hình ảnh thực tế của góc bẹt như: thước đo góc, góc tạo bởi kim giờ và kim phút lúc 6 giờ, ...

========================================================================

https://khoahoc.vietjack.com/thi-online/giai-toan-6-chuong-2-goc/16127


\textbf{{QUESTION}}

Góc vuông là gì?

\textbf{{ANSWER}}

Góc vuông là góc có số đo bằng 90o.

========================================================================

https://khoahoc.vietjack.com/thi-online/giai-toan-6-chuong-2-goc/16127


\textbf{{QUESTION}}

Góc nhọn là gì?

\textbf{{ANSWER}}

Góc nhọn là góc nhỏ hơn góc vuông.

========================================================================

https://khoahoc.vietjack.com/thi-online/giai-vth-toan-9-kntt-bai-30-da-giac-deu-co-dap-an


\textbf{{QUESTION}}

Chọn phương án đúng.
Khẳng định nào dưới đây là không đúng?
A. Đa giác đều có các cạnh bằng nhau.
B. Đa giác có các cạnh bằng nhau là đa giác đều.
C. Đa giác đều có các góc bằng nhau.
D. Đa giác đều nội tiếp được một đường tròn.

\textbf{{ANSWER}}

Đáp án đúng là: B
– Đa giác đều có các cạnh bằng nhau, có các góc bằng nhau và nội tiếp được một đường tròn.
– Đa giác có các cạnh bằng nhau chưa chắc là đa giác đều.

========================================================================

https://khoahoc.vietjack.com/thi-online/giai-vth-toan-9-kntt-bai-30-da-giac-deu-co-dap-an


\textbf{{QUESTION}}

Chọn phương án đúng.
Một đa giác đều có các cạnh bằng a nội tiếp một đường tròn có bán kính bằng a. Chu vi của đa giác đó bằng
A. 3a.
B. 4a.
C. 6a.
D. 8a.

\textbf{{ANSWER}}

Đáp án đúng là: C
Gọi n là số đỉnh của đa giác đều đã cho.
Ta có: sin180∘n=a2R=a2a=12,$\sin \frac{{180^\circ }}{n} = \frac{a}{{2R}} = \frac{a}{{2a}} = \frac{1}{2},$ do đó 180∘n=30∘$\frac{{180^\circ }}{n} = 30^\circ $ hay n = 6.
Vậy đa giác đều cần tìm là lục giác đều. Chu vi của đa giác đó bằng 6a.

========================================================================

https://khoahoc.vietjack.com/thi-online/10-bai-tap-bai-toan-thuc-tien-lien-quan-den-gia-tri-luong-giac-cua-goc-luong-giac-co-loi-giai


\textbf{{QUESTION}}

Bánh xe có đường kính kể cả lốp xe là 40 cm. Nếu xe chạy với vận tốc 30 km/h thì trong một giây thì bánh xe quay được số vòng là (làm tròn đến chữ số thập phân thứ hai)
A. 6,63 vòng;
B. 6,64 vòng;
C. 6,65 vòng;

\textbf{{ANSWER}}

Hướng dẫn giải
Đáp án đúng là: A
Chu vi của bánh xe là: C = 2πR = πd = 40π (cm).
Đổi 30 km/h = $$ \frac{{3.10}^{6}}{3\text{\hspace{0.17em}\hspace{0.17em}}600}$$ = $$ \frac{2\text{\hspace{0.17em}\hspace{0.17em}}500}{3}$$ cm/s.
Một giây bánh xe quay được số vòng là: $$ \frac{2\text{\hspace{0.17em}\hspace{0.17em}}500}{3}$$ : 40π ≈ 6,63 (vòng)

========================================================================

https://khoahoc.vietjack.com/thi-online/10-bai-tap-bai-toan-thuc-tien-lien-quan-den-gia-tri-luong-giac-cua-goc-luong-giac-co-loi-giai


\textbf{{QUESTION}}

Một bánh xe của người đi xe đạp quay được 20 vòng trong 4 giây. Bánh xe quay được góc có số đo (rad) trong 1 giây là
A. 10π;
B. 9π;
C. 8π;

\textbf{{ANSWER}}

Hướng dẫn giải
Đáp án đúng là: A
Một giây bánh xe quay được số vòng là: 204=5$$ \frac{20}{4}=5$$ vòng.
Vì 1 vòng bánh xe quay được góc có số đo là 2π.
Nên số đo góc mà bánh xe quay được trong một giây (5 vòng) là: 5.2π = 10π (rad).

========================================================================

https://khoahoc.vietjack.com/thi-online/10-bai-tap-bai-toan-thuc-tien-lien-quan-den-gia-tri-luong-giac-cua-goc-luong-giac-co-loi-giai


\textbf{{QUESTION}}

Một vệ tinh được đặt ở vị trí A trong không gian. Từ vị trí A, vệ tinh bắt đầu chuyển động quanh Trái Đất theo quỹ đạo là đường tròn với tâm là tâm O của Trái Đất, bán kính bằng 9 000 km. Biết rằng vệ tinh chuyển động hết một vòng quỹ đạo trong 2 giờ. Quãng đường mà vệ tinh đã chuyển động sau 1 giờ là
A. 4 500π km
B. 9 000π km;
C. 18 000π km;

\textbf{{ANSWER}}

Hướng dẫn giải 
Đáp án đúng là: B
Vì vệ tinh chuyển động hết một vòng quỹ đạo trong 2 giờ có nghĩa là trong 2 giờ vệ tinh chuyển động được một góc bằng 360° hay 2π (rad).
Suy ra trong vòng 1 giờ vệ tinh chuyển động được một góc là: 2π : 2 = π.
Gọi quãng đường mà vệ tinh đã chuyển động được sau 1 giờ là l
                         l = R. α = 9 000.π km.

========================================================================

https://khoahoc.vietjack.com/thi-online/19-de-on-thi-vao-10-chuyen-hay-co-loi-giai/59351


\textbf{{QUESTION}}

Chứng minh rằng nếu số nguyên n lớn hơn 1 thoả mãn n2 + 4 và n2 +16 là các số nguyên tố thì n chia hết cho 5.

\textbf{{ANSWER}}

Ta có với mọi số nguyên m thì m2 chia cho 5 dư 0 , 1 hoặc 4.
+ Nếu n2 chia cho 5 dư 1 thì  $$ {n}^{2}=5k+1=>{n}^{2}+4=5k+5\vdots 5;k\in {N}^{*}.$$
Nên n2+4 không là số nguyên tố
+ Nếu n2 chia cho 5 dư 4 thì $$ {n}^{2}=5k+4=>{n}^{2}+16=5k+20\vdots 5;k\in {N}^{*}.$$
Nên n2+16 không là số nguyên tố.
 Vậy n2 $$ \vdots $$ 5 hay n $$ \vdots $$5

========================================================================

https://khoahoc.vietjack.com/thi-online/19-de-on-thi-vao-10-chuyen-hay-co-loi-giai/59351


\textbf{{QUESTION}}

Tìm nghiệm nguyên của phương trình: $$ {x}^{2}-2y(x-y)=2(x+1)$$

\textbf{{ANSWER}}

$$ {x}^{2}-2y(x-y)=2(x+1)<=>{x}^{2}-2(y+1)x+2({y}^{2}-1)=0\left(1\right)$$
Để phương trình (1) có nghiệm nguyên x thì D' theo y phải là số chính phương
+ Nếu $$ \Delta \text{'}=4=>{(y-1)}^{2}=0<=>y=1$$ thay vào phương trình (1) ta có :
$$ {x}^{2}-4x=0<=>x(2-4)<=>\left[\begin{array}{l}x=0\\ x-4\end{array}\right.$$
+ Nếu $$ \Delta \text{'}=1=>{(y-1)}^{2}=3<=>y\notin Z.$$
+ Nếu $$ \Delta \text{'}=0=>{(y-1)}^{2}=4<=>\left[\begin{array}{l}y=3\\ y=-1\end{array}\right.$$
+ Với y = 3 thay vào phương trình (1) ta có:  $$ {x}^{2}-8x+16=0<=>{(x-4)}^{2}=0<=>x=4$$
+ Với y = -1 thay vào phương trình (1) ta có: $$ {x}^{2}=0<=>x=0$$
Vậy phương trình (1) có 4 nghiệm nguyên $$ (x;y)\in \text{{(0;1);(4;1);(4;3);(0;-1)}}$$

========================================================================

https://khoahoc.vietjack.com/thi-online/19-de-on-thi-vao-10-chuyen-hay-co-loi-giai/59351


\textbf{{QUESTION}}

Rút gọn biểu thức: A=√2(3+√5)2√2+√3+√5+√2(3−√5)2√2−√3−√5$$ A=\frac{\sqrt{2}(3+\sqrt{5})}{2\sqrt{2}+\sqrt{3+\sqrt{5}}}+\frac{\sqrt{2}(3-\sqrt{5})}{2\sqrt{2}-\sqrt{3-\sqrt{5}}}$$

\textbf{{ANSWER}}

A=√2(3+√5)2√2+√3+√5+√2(3−√5)2√2−√3−√52[3+√54+√(√5+1)2+3−√54−√(√5−1)2]=2(3+√55+√5+3−√55−√5)2[(3+√5)(5−√5)+(3−√5)(5+√5)(5+√5)(5−√5)]=2(15−3√5+5√5−5+15+3√5−5√5−525−5)=2.2020=2Vậy A=2$$ A=\frac{\sqrt{2}(3+\sqrt{5})}{2\sqrt{2}+\sqrt{3+\sqrt{5}}}+\frac{\sqrt{2}(3-\sqrt{5})}{2\sqrt{2}-\sqrt{3-\sqrt{5}}}\phantom{\rule{0ex}{0ex}}2\left[\frac{3+\sqrt{5}}{4+\sqrt{{(\sqrt{5}+1)}^{2}}}+\frac{3-\sqrt{5}}{4-\sqrt{{(\sqrt{5}-1)}^{2}}}\right]=2\left(\frac{3+\sqrt{5}}{5+\sqrt{5}}+\frac{3-\sqrt{5}}{5-\sqrt{5}}\right)\phantom{\rule{0ex}{0ex}}2\left[\frac{(3+\sqrt{5})(5-\sqrt{5})+(3-\sqrt{5})(5+\sqrt{5})}{(5+\sqrt{5})(5-\sqrt{5})}\right]=2\left(\frac{15-3\sqrt{5}+5\sqrt{5}-5+15+3\sqrt{5}-5\sqrt{5}-5}{25-5}\right)\phantom{\rule{0ex}{0ex}}=2.\frac{20}{20}=2\phantom{\rule{0ex}{0ex}}Vậy\quad A=2$$

========================================================================

https://khoahoc.vietjack.com/thi-online/19-de-on-thi-vao-10-chuyen-hay-co-loi-giai/59351


\textbf{{QUESTION}}

Tìm m để phương trình: (x−2)(x−3)(x+4)(x+5)=m$$ (x-2)(x-3)(x+4)(x+5)=m$$ có 4 nghiệm phân biệt

\textbf{{ANSWER}}

Phương trình 
(x−2)(x−3)(x+4)(x+5)=m<=>(x2+2x−8)(x2+2x−15)=m(1)$$ \begin{array}{l}(x-2)(x-3)(x+4)(x+5)=m\\ <=>({x}^{2}+2x-8)({x}^{2}+2x-15)=m\left(1\right)\end{array}$$
Đặt x2+2x+1=(x+1)2=y(y≥0)$$ {x}^{2}+2x+1={(x+1)}^{2}=y(y\ge 0)$$ phương trình (1) trở thành:
(y−9)(y−16)=m<=>y2−25y+144−m=0(2)$$ (y-9)(y-16)=m<=>{y}^{2}-25y+144-m=0\left(2\right)$$
Nhận xét: Với mỗi giá trị y > 0 thì phương trình: (x+1)2=y có 2 nghiệm phân biệt, do đó phương trình (1) có 4 nghiệm phân biệtÛ phương trình (2) có 2 nghiệm dương phân biệt.
{Δ'>0S>0P>0<=>{Δ'=4m+49>025>0144−m>0<=>−494<n<144$$ \left\{\begin{array}{l}\Delta \text{'}>0\\ S>0\\ P>0\end{array}\right.<=>\left\{\begin{array}{l}\Delta \text{'}=4m+49>0\\ 25>0\\ 144-m>0\end{array}\right.<=>\frac{-49}{4}<n<144$$
Vậy với −494<n<144$$ \frac{-49}{4}<n<144$$ thì phương trình (1) có 4 nghiệm phân biệt.

========================================================================

https://khoahoc.vietjack.com/thi-online/19-de-on-thi-vao-10-chuyen-hay-co-loi-giai/59351


\textbf{{QUESTION}}

Giải phương trình: x2−x−4=2√x−1(1−x)$$ {x}^{2}-x-4=2\sqrt{x-1}(1-x)$$

\textbf{{ANSWER}}

Điều kiện: x≥1(*)$$ x\ge 1(*)$$
Ta có: x2−x−4=2√x−1(1−x)<=>x2+2x√x−1+x−1−2(x+√x−1)−3=0$$ \begin{array}{l}{x}^{2}-x-4=2\sqrt{x-1}(1-x)\\ <=>{x}^{2}+2x\sqrt{x-1}+x-1-2(x+\sqrt{x-1})-3=0\end{array}$$
Đặt: x+√x−1=y(y≥1)(**)$$ x+\sqrt{x-1}=y(y\ge 1)(**)$$ phương trình trở thành y2−2y−3=0$$ {y}^{2}-2y-3=0$$
y2−2y−3=0<=>(y+1)(y−3)=0<=>[y=−1y=3$$ {y}^{2}-2y-3=0<=>(y+1)(y-3)=0<=>\left[\begin{array}{l}y=-1\\ y=3\end{array}\right.$$
+ Với y = -1 không thỏa mãn điều kiện (**).
 
+ Với y = 3 ta có phương trình:
x+√x−1=3<=>√x−1=3−x<=>{x≤3x−1=9−6x+x2<=>{x≤3x2−7x+10=0<=>{x≤3[x=2x=5<=>x=2$$ x+\sqrt{x-1}=3<=>\sqrt{x-1}=3-x<=>\left\{\begin{array}{l}x\le 3\\ x-1=9-6x+{x}^{2}\end{array}\right.<=>\left\{\begin{array}{l}x\le 3\\ {x}^{2}-7x+10=0\end{array}\right.<=>\left\{\begin{array}{l}x\le 3\\ \left[\begin{array}{l}x=2\\ x=5\end{array}\right.\end{array}\right.<=>x=2$$
thỏa mãn điều kiện (*). Vậy phương trình có nghiệm x = 2.

========================================================================

https://khoahoc.vietjack.com/thi-online/15-cau-trac-nghiem-toan-9-ket-noi-tri-thuc-bai-13-mo-dau-ve-duong-tron-co-dap-an


\textbf{{QUESTION}}

I. Nhận biết
Tâm đối xứng của đường tròn là
A. Điểm bất kì bên trong đường tròn.

\textbf{{ANSWER}}

Đáp án đúng là: D
Tâm đối xứng của đường tròn là tâm của đường tròn đó.
Vậy ta chọn phương án D.

========================================================================

https://khoahoc.vietjack.com/thi-online/15-cau-trac-nghiem-toan-9-ket-noi-tri-thuc-bai-13-mo-dau-ve-duong-tron-co-dap-an


\textbf{{QUESTION}}

Khẳng định nào sau đây là đúng khi nói về trục đối xứng của đường tròn?
A. Đường tròn không có trục đối xứng.
B. Đường tròn có duy nhất một trục đối xứng.
C. Đường tròn có hai trục đối xứng là hai đường thẳng đi qua tâm và vuông góc với nhau.
D. Đường tròn có vô số trục đối xứng.

\textbf{{ANSWER}}

Đáp án đúng là: D
Đường tròn là hình có trục đối xứng.
Mỗi đường thẳng đi qua tâm của đường tròn là một trục đối xứng của đường tròn.
Do đó đường tròn có vô số trục đối xứng.
Vậy ta chọn phương án D.

========================================================================

https://khoahoc.vietjack.com/thi-online/15-cau-trac-nghiem-toan-9-ket-noi-tri-thuc-bai-13-mo-dau-ve-duong-tron-co-dap-an


\textbf{{QUESTION}}

Cho đường tròn (O;5cm)$\left( {O;5{\rm{\;cm}}} \right)$ và một điểm K$K$ bất kì. Biết rằng OK=7cm.$OK = 7{\rm{\;cm}}.$ Khẳng định nào sau đây đúng?
A. Điểm K$K$ nằm trong đường tròn (O;5cm).$\left( {O;5{\rm{\;cm}}} \right).$
B. Điểm K$K$ nằm ngoài đường tròn (O;5cm).$\left( {O;5{\rm{\;cm}}} \right).$
C. Điểm K$K$ nằm trên đường tròn (O;5cm).$\left( {O;5{\rm{\;cm}}} \right).$
D. Điểm K$K$ thuộc đường tròn (O;5cm).$\left( {O;5{\rm{\;cm}}} \right).$

\textbf{{ANSWER}}

Đáp án đúng là: B
Ta thấy đường tròn (O;5cm)$\left( {O;5{\rm{\;cm}}} \right)$ có bán kính R=5(cm).$R = 5{\rm{\;(cm)}}{\rm{.}}$
Vì 7(cm)>5(cm)$7{\rm{\;(cm)}} > 5{\rm{\;(cm)}}$ nên OK>R.$OK > R.$
Do đó điểm K$K$ nằm ngoài đường tròn (O;5cm).$\left( {O;5{\rm{\;cm}}} \right).$
Vậy ta chọn phương án B.

========================================================================

https://khoahoc.vietjack.com/thi-online/15-cau-trac-nghiem-toan-9-ket-noi-tri-thuc-bai-13-mo-dau-ve-duong-tron-co-dap-an


\textbf{{QUESTION}}

Hình tròn tâm O$O$ bán kính R$R$ là hình gồm các điểm
A. nằm trên và nằm trong đường tròn (O;R).$\left( {O\,;R} \right).$
B. nằm trên đường tròn (O;R).$\left( {O\,;R} \right).$
C. nằm trong đường tròn (O;R).$\left( {O\,;R} \right).$
D. nằm ngoài đường tròn (O;R).$\left( {O\,;R} \right).$

\textbf{{ANSWER}}

Đáp án đúng là: A
Hình tròn tâm O$O$ bán kính R$R$ là hình gồm các điểm nằm trên và nằm trong đường tròn (O;R).$\left( {O\,;R} \right).$
Vậy ta chọn phương án A.

========================================================================

https://khoahoc.vietjack.com/thi-online/15-cau-trac-nghiem-toan-9-ket-noi-tri-thuc-bai-13-mo-dau-ve-duong-tron-co-dap-an


\textbf{{QUESTION}}

Cho đường tròn (O;R)$\left( {O\,;R} \right)$ và một điểm G$G$ bất kì. Ta nói điểm G$G$ nằm trên đường tròn (O;R)$\left( {O\,;R} \right)$ nếu
A. OG>R.$OG > R.$

\textbf{{ANSWER}}

Đáp án đúng là: C
Cho đường tròn (O;R).$\left( {O\,;R} \right).$ Ta có điểm G$G$ nằm trên đường tròn (O;R)$\left( {O\,;R} \right)$ nếu OG=R.$OG = R.$
Vậy ta chọn phương án C.

========================================================================

https://khoahoc.vietjack.com/thi-online/22-cau-trac-nghiem-toan-7-bai-6-cong-tru-da-thuc-co-dap-an-thong-hieu


\textbf{{QUESTION}}

Thu gọn đa thức $$ 3y\left({x}^{2}-xy\right)-7{x}^{2}\left(y+xy\right)$$ ta được
A. $$ -4{x}^{2}y-3x{y}^{2}+7{x}^{3}y$$.
B. $$ -4{x}^{2}y-3x{y}^{2}-7{x}^{3}y$$.
C. $$ 4{x}^{2}y-3x{y}^{2}-7{x}^{3}y$$.
D. $$ 4{x}^{2}y-3x{y}^{2}+7{x}^{3}y$$.

\textbf{{ANSWER}}

Đáp án cần chọn là B
$$ 3y\left({x}^{2}-xy\right)-7{x}^{2}\left(y+xy\right)\phantom{\rule{0ex}{0ex}}=3{x}^{2}y-3x{y}^{2}-7{x}^{2}y-7{x}^{3}y\phantom{\rule{0ex}{0ex}}=\left(3{x}^{2}y-7{x}^{2}y\right)-3x{y}^{2}-7{x}^{3}y\phantom{\rule{0ex}{0ex}}=-4{x}^{2}y-3x{y}^{2}-7{x}^{3}y$$

========================================================================

https://khoahoc.vietjack.com/thi-online/22-cau-trac-nghiem-toan-7-bai-6-cong-tru-da-thuc-co-dap-an-thong-hieu


\textbf{{QUESTION}}

Thu gọn đa thức (-3x2y-2xy2+16)+(-2x2+5xy2-10)$$ \left(-3{x}^{2}y-2x{y}^{2}+16\right)+\left(-2{x}^{2}+5x{y}^{2}-10\right)$$ta được:
A. -x2y-7xy2+26$$ -{x}^{2}y-7x{y}^{2}+26$$.
B. -5x2y+3xy2+6$$ -5{x}^{2}y+3x{y}^{2}+6$$.
C. 5x2y+3xy2+6$$ 5{x}^{2}y+3x{y}^{2}+6$$.
D. 5x2y+3xy2-6$$ 5{x}^{2}y+3x{y}^{2}-6$$.

\textbf{{ANSWER}}

Đáp án cần chọn là B.
Ta có:
$$ \left(-3{x}^{2}y-2x{y}^{2}+16\right)+\left(-2{x}^{2}y+5x{y}^{2}-10\right)\phantom{\rule{0ex}{0ex}}=-3{x}^{2}y-2x{y}^{2}+16-2{x}^{2}y+5x{y}^{2}-10\phantom{\rule{0ex}{0ex}}=\left(-3{x}^{2}y-2{x}^{2}y\right)+\left(-2x{y}^{2}+5x{y}^{2}\right)+\left(16-10\right)\phantom{\rule{0ex}{0ex}}=-5{x}^{2}y+3x{y}^{2}+6$$

========================================================================

https://khoahoc.vietjack.com/thi-online/22-cau-trac-nghiem-toan-7-bai-6-cong-tru-da-thuc-co-dap-an-thong-hieu


\textbf{{QUESTION}}

Đa thức 15xy(x+y)-2(yx2-xy2)$$ \frac{1}{5}xy\left(x+y\right)-2\left(y{x}^{2}-x{y}^{2}\right)$$ được thu gọn thành đa thức nào dưới đây?
A. 95xy2+95x2y$$ \frac{9}{5}x{y}^{2}+\frac{9}{5}{x}^{2}y$$.
B. 115xy2-95x2y$$ \frac{11}{5}x{y}^{2}-\frac{9}{5}{x}^{2}y$$.
C. 115xy2+95x2y$$ \frac{11}{5}x{y}^{2}+\frac{9}{5}{x}^{2}y$$.
D. 115xy2+115x2y$$ \frac{11}{5}x{y}^{2}+\frac{11}{5}{x}^{2}y$$.

\textbf{{ANSWER}}

Đáp án cần chọn là B.
Ta có:
15xy(x+y)-2(yx2-xy2)=15x2y+15xy2-2x2y+2xy2=(15x2y-2x2y)+(15xy2+2xy2)=115xy2-95x2y$$ \frac{1}{5}xy\left(x+y\right)-2\left(y{x}^{2}-x{y}^{2}\right)\phantom{\rule{0ex}{0ex}}=\frac{1}{5}{x}^{2}y+\frac{1}{5}x{y}^{2}-2{x}^{2}y+2x{y}^{2}\phantom{\rule{0ex}{0ex}}=\left(\frac{1}{5}{x}^{2}y-2{x}^{2}y\right)+\left(\frac{1}{5}x{y}^{2}+2x{y}^{2}\right)\phantom{\rule{0ex}{0ex}}=\frac{11}{5}x{y}^{2}-\frac{9}{5}{x}^{2}y$$

========================================================================

https://khoahoc.vietjack.com/thi-online/22-cau-trac-nghiem-toan-7-bai-6-cong-tru-da-thuc-co-dap-an-thong-hieu


\textbf{{QUESTION}}

Đa thức (1,6x2+1,7y2+2xy)-(0,5x2-0,3y2-2xy)$$ \left(1,6{x}^{2}+1,7{y}^{2}+2xy\right)-\left(0,5{x}^{2}-0,3{y}^{2}-2xy\right)$$ có bậc là:
A. 2.
B. 3.
C. 4.
D. 6.

\textbf{{ANSWER}}

Đáp án cần chọn là A.
Ta có:
(1,6x2+1,7y2+2xy)-(0,5x2-0,3y2-2xy)=1,6x2+1,7y2+2xy-0,5x2+0,3y2+2xy=(1,6x2-0,5x2)+(1,7y2+0,3y2)+(2xy+2xy)=1,1x2+2y2+4xy$$ \left(1,6{x}^{2}+1,7{y}^{2}+2xy\right)-\left(0,5{x}^{2}-0,3{y}^{2}-2xy\right)\phantom{\rule{0ex}{0ex}}=1,6{x}^{2}+1,7{y}^{2}+2xy-0,5{x}^{2}+0,3{y}^{2}+2xy\phantom{\rule{0ex}{0ex}}=\left(1,6{x}^{2}-0,5{x}^{2}\right)+\left(1,7{y}^{2}+0,3{y}^{2}\right)+\left(2xy+2xy\right)\phantom{\rule{0ex}{0ex}}=1,1{x}^{2}+2{y}^{2}+4xy$$
Đa thức 1,1x2+2y2+4xy$$ 1,1{x}^{2}+2{y}^{2}+4xy$$ có bậc là 2.

========================================================================

https://khoahoc.vietjack.com/thi-online/22-cau-trac-nghiem-toan-7-bai-6-cong-tru-da-thuc-co-dap-an-thong-hieu


\textbf{{QUESTION}}

Đa thức nào dưới đây là kết quả của phép tính 4x3yz-4xy2z2-yz(xyz+x3)$$ 4{x}^{3}yz-4x{y}^{2}{z}^{2}-yz\left(xyz+{x}^{3}\right)$$ ?
A. 3x3yz-5xy2z2$$ 3{x}^{3}yz-5x{y}^{2}{z}^{2}$$.
B. 3x3yz+5xy2z2$$ 3{x}^{3}yz+5x{y}^{2}{z}^{2}$$.
C. -5x3yz+5xy2z2$$ -5{x}^{3}yz+5x{y}^{2}{z}^{2}$$.
D. 5x3yz+5xy2z2$$ 5{x}^{3}yz+5x{y}^{2}{z}^{2}$$.

\textbf{{ANSWER}}

Đáp án cần chọn là A.
Ta có:
4x3yz-4xy2z2-yz(xyz+x3)=4x3yz-4xy2z2-xy2z2-x3yz=(4x3yz-x3yz)+(-4xy2z2-xy2z2)=3x3yz-5xy2z2$$ 4{x}^{3}yz-4x{y}^{2}{z}^{2}-yz\left(xyz+{x}^{3}\right)\phantom{\rule{0ex}{0ex}}=4{x}^{3}yz-4x{y}^{2}{z}^{2}-x{y}^{2}{z}^{2}-{x}^{3}yz\phantom{\rule{0ex}{0ex}}=\left(4{x}^{3}yz-{x}^{3}yz\right)+\left(-4x{y}^{2}{z}^{2}-x{y}^{2}{z}^{2}\right)\phantom{\rule{0ex}{0ex}}=3{x}^{3}yz-5x{y}^{2}{z}^{2}$$

========================================================================

https://khoahoc.vietjack.com/thi-online/tong-hop-bai-tap-toan-8-chuong-4-bat-phuong-trinh-bac-nhat-mot-an/50682


\textbf{{QUESTION}}

Trong các khẳng định sau đây, khẳng định nào đúng?
4 + ( - 3 ) ≤ 5    ( 1 )
6 + ( - 2 ) ≤ 7 + ( - 2 )    ( 2 )
24 + ( - 5 ) > 25 + ( - 5 )    ( 3 )
A. ( 1 ),( 2 ),( 3 )   
B. ( 1 ),( 3 ) 
C. ( 1 ),( 2 )   
D. ( 2 ),( 3 )

\textbf{{ANSWER}}

+ Ta có: -3 < 1 nên 4 + (-3) < 4 + 1 hay 4 + (-3) < 5
→ Khẳng định ( 1 ) đúng.
+ Ta có: 6 ≤ 7 ⇒ 6 + (-2) ≤ 7 + (-2)
→ Khẳng định ( 2 ) đúng.
+ Ta có: 24 < 25 ⇒ 24 + ( - 5 ) < 25 + ( - 5 )
→ Khẳng định ( 3 ) sai.
Chọn đáp án C.

========================================================================

https://khoahoc.vietjack.com/thi-online/tong-hop-bai-tap-toan-8-chuong-4-bat-phuong-trinh-bac-nhat-mot-an/50682


\textbf{{QUESTION}}

Cho a - 3 > b - 3. So sánh hai số a và b
A. a ≥ b   
B. a < b 
C. a > b   
D. a ≤ b

\textbf{{ANSWER}}

Ta có a - 3 > b - 3 ⇒ ( a - 3 ) + 3 > ( b - 3 ) + 3 ⇔ a > b
Chọn đáp án C.

========================================================================

https://khoahoc.vietjack.com/thi-online/tong-hop-bai-tap-toan-8-chuong-4-bat-phuong-trinh-bac-nhat-mot-an/50682


\textbf{{QUESTION}}

Cho a > b. So sánh 5 - a với 5 - b.
A. 5 - a ≥ 5 - b. 
B. 5 - a > 5 - b. 
C. 5 - a ≤ 5 - b. 
D. 5 - a < 5 - b.

\textbf{{ANSWER}}

Ta có: a > b ⇒ - a < - b ⇔ 5 + ( - a ) < 5 + ( - b ) hay 5 - a < 5 - b.
Chọn đáp án D.

========================================================================

https://khoahoc.vietjack.com/thi-online/tong-hop-bai-tap-toan-8-chuong-4-bat-phuong-trinh-bac-nhat-mot-an/50682


\textbf{{QUESTION}}

Một Ampe kế có giới hạn đo là 25 ampe. Gọi x( A ) là số đo cường độ dòng điện có thể đo bằng Ampe kế. Khẳng định nào sau đây đúng?
A. x ≤ 25   
B. x < 25 
C. x > 25   
D. x ≥ 25

\textbf{{ANSWER}}

Một Ampe kế đo cường độ dòng điện thì cường độ dòng điện tối đa mà Ampe đo được là giới hạn đo của ampe kế đó.
Khi đó: x ≤ 25 
Chọn đáp án A.

========================================================================

https://khoahoc.vietjack.com/thi-online/tong-hop-bai-tap-toan-8-chuong-4-bat-phuong-trinh-bac-nhat-mot-an/50682


\textbf{{QUESTION}}

Cho a > b, c > d. Khẳng định nào sau đây đúng? 
A. a + d > b + c 
B. a + c > b + d 
C. b + d > a + c 
D. a + b > c + d

\textbf{{ANSWER}}

Theo giả thiết ta có: a > b, c > d ⇒ a + c > b + d.
Chọn đáp án B.

========================================================================

https://khoahoc.vietjack.com/thi-online/10-bai-tap-tism-so-uoc-cua-mot-so-co-loi-giai


\textbf{{QUESTION}}

Số 30 có bao nhiêu ước?
A. 3;
B. 5;
C. 7;
D. 8.

\textbf{{ANSWER}}

Đáp án đúng là: D
Ta phân tích 30 thành tích các thừa số nguyên tố:
30 = 2.3.5
Ta có: 2.3 = 6; 2.5 = 10; 3.5 = 15
Do đó 30 có các ước là: 1; 2; 3; 5; 6; 10; 15; 30
Vậy 30 có 8 ước

========================================================================

https://khoahoc.vietjack.com/thi-online/10-bai-tap-tism-so-uoc-cua-mot-so-co-loi-giai


\textbf{{QUESTION}}

Số 120 có bao nhiêu ước?
A. 15;
B. 14;
C. 16;
D. 13.

\textbf{{ANSWER}}

Đáp án đúng là: C
Ta phân tích 120 thành tích các thừa số nguyên tố:
120 = 23.3.5
Ta có: 2.2 = 4; 2.3 = 6; 23 = 8; 2.5 = 10; 22.3 = 12; 3.5 = 15; 22.5 = 20; 23.3 = 24; 2.3.5 = 30; 23.5 = 40; 22.3.5 = 60
Do đó 120 có các ước là: 1; 2; 3; 4; 5; 6; 8; 10; 12; 15; 20; 24; 30; 40; 60; 120.
Vậy 120 có 16 ước.

========================================================================

https://khoahoc.vietjack.com/thi-online/10-bai-tap-tism-so-uoc-cua-mot-so-co-loi-giai


\textbf{{QUESTION}}

Tổng các ước của 16 là?
A. 30;
B. 31;
C. 32;
D. 29.

\textbf{{ANSWER}}

Đáp án đúng là: B
Ta phân tích 16 thành tích các thừa số nguyên tố:
16 = 24
Ta có: 22 = 4; 23 = 8
Do đó 16 có các ước là: 1; 2; 4; 8; 16
Tổng các ước của 16 là: 1 + 2 + 4 + 8 + 16 = 31

========================================================================

https://khoahoc.vietjack.com/thi-online/10-bai-tap-tism-so-uoc-cua-mot-so-co-loi-giai


\textbf{{QUESTION}}

Số 420 có bao nhiêu ước nguyên tố?
A. 5;
B. 6;
C. 4;
D. 24.

\textbf{{ANSWER}}

Đáp án đúng là: C
Ta phân tích 420 thành tích các thừa số nguyên tố:
420 = 22.3.5.7
Do đó số các ước nguyên tố của 420 là: 2; 3; 5; 7
Do đó, 420 có 4 ước nguyên tố.

========================================================================

https://khoahoc.vietjack.com/thi-online/10-bai-tap-tism-so-uoc-cua-mot-so-co-loi-giai


\textbf{{QUESTION}}

Số các ước nhỏ hơn 50 của 1000 là?
A. 8;
B. 9;
C. 10;
D. 7.

\textbf{{ANSWER}}

Đáp án đúng là: B
Ta phân tích 1000 thành tích các thừa số nguyên tố:
1000 = 23.53
Ta có: 22 = 4; 23 = 8; 2.5 = 10; 22.5 = 20; 52 = 25; 23.5 = 40
Các ước nhỏ hơn 50 của 1000 là: 1; 2; 4; 5; 8; 10; 20; 25; 40
Vậy có 9 ước nhỏ hơn 50 của 1000 .

========================================================================

https://khoahoc.vietjack.com/thi-online/bo-4-de-kiem-tra-hoc-ki-2-chuyen-de-toan-11-kiem-tra-cuoi-ki-co-dap-an


\textbf{{QUESTION}}

Ba số hạng đầu tiên theo lũy thừa tăng dần của x trong khai triển $$ {\left(1+2x\right)}^{10}$$  là
A. $$ 1,\text{\hspace{0.17em}\hspace{0.17em}}45x,\text{\hspace{0.17em}\hspace{0.17em}}120{x}^{2}$$
B. $$ 1,\text{\hspace{0.17em}\hspace{0.17em}}4x,\text{\hspace{0.17em}\hspace{0.17em}}4{x}^{2}$$
C. $$ 1,\text{\hspace{0.17em}\hspace{0.17em}}20x,\text{\hspace{0.17em}\hspace{0.17em}}180{x}^{2}$$
D. $$ 10,\text{\hspace{0.17em}\hspace{0.17em}}45x,\text{\hspace{0.17em}\hspace{0.17em}}120{x}^{2}$$

\textbf{{ANSWER}}

Chọn C

========================================================================

https://khoahoc.vietjack.com/thi-online/bo-4-de-kiem-tra-hoc-ki-2-chuyen-de-toan-11-kiem-tra-cuoi-ki-co-dap-an


\textbf{{QUESTION}}

Cho cấp số cộng (un)$$ \left({u}_{n}\right)$$  có u2=3$$ {u}_{2}=3$$  và u4=7$$ {u}_{4}=7$$ . Giá trị của u15$$ {u}_{15}$$  bằng
A. 31.

\textbf{{ANSWER}}

Chọn C

========================================================================

https://khoahoc.vietjack.com/thi-online/bo-4-de-kiem-tra-hoc-ki-2-chuyen-de-toan-11-kiem-tra-cuoi-ki-co-dap-an


\textbf{{QUESTION}}

Cho phép tịnh tiến T→u$$ {T}_{\overrightarrow{u}}$$  biến điểm M thành M1$$ {M}_{1}$$  và phép tịnh tiến T→v$$ {T}_{\overrightarrow{v}}$$  biến M1$$ {M}_{1}$$  thành M2$$ {M}_{2}$$ . Chọn khẳng định đúng trong các khẳng định sau
A. Một phép đối xứng trục biến M thành M2$$ {M}_{2}$$ .
B. Không thể khẳng định được có hay không một phép dời hình biến M thành M2$$ {M}_{2}$$ .
C. Phép tịnh tiến T→u+→v$$ {T}_{\overrightarrow{u}+\overrightarrow{v}}$$  biến M thành M2$$ {M}_{2}$$ .

\textbf{{ANSWER}}

Chọn C

========================================================================

https://khoahoc.vietjack.com/thi-online/bo-4-de-kiem-tra-hoc-ki-2-chuyen-de-toan-11-kiem-tra-cuoi-ki-co-dap-an


\textbf{{QUESTION}}

Điều kiện xác định của hàm số y=tanx+cotx$$ y=\mathrm{tan}x+\mathrm{cot}x$$  là
A. x≠kπ2,  k∈ℤ$$ x\ne \frac{k\pi }{2},\text{\hspace{0.17em}\hspace{0.17em}}k\in \mathbb{Z}$$
B. x≠π2+kπ,  k∈ℤ$$ x\ne \frac{\pi }{2}+k\pi ,\text{\hspace{0.17em}\hspace{0.17em}}k\in \mathbb{Z}$$
C. x∈ℝ$$ x\in \mathbb{R} $$
D. x≠kπ,  k∈ℤ$$ x\ne k\pi ,\text{\hspace{0.17em}\hspace{0.17em}}k\in \mathbb{Z}$$

\textbf{{ANSWER}}

Chọn A

========================================================================

https://khoahoc.vietjack.com/thi-online/bo-4-de-kiem-tra-hoc-ki-2-chuyen-de-toan-11-kiem-tra-cuoi-ki-co-dap-an


\textbf{{QUESTION}}

Phương trình sinx+√3cosx=1$$ \mathrm{sin}x+\sqrt{3}\mathrm{cos}x=1$$  có tập nghiệm là
B. {−π6+kπ;  −π2+kπ}$$ \left\{-\frac{\pi }{6}+k\pi ;\text{\hspace{0.17em}\hspace{0.17em}}-\frac{\pi }{2}+k\pi \right\}$$, với k∈ℤ$$ k\in \mathbb{Z}$$

\textbf{{ANSWER}}

Chọn C

========================================================================

https://khoahoc.vietjack.com/thi-online/bai-tap-toan-8-chu-de-6-bat-dang-thuc-co-dap-an/108043


\textbf{{QUESTION}}

Chứng minh rằng với mọi số thực x, y luôn có:
$$ {({x}^{3}+{y}^{3})}^{2}\le ({x}^{2}+{y}^{2})({x}^{4}+{y}^{4})$$

\textbf{{ANSWER}}

Ta có: $$ VT={({x}^{3}+{y}^{3})}^{2}={(x.{x}^{2}+y.{y}^{2})}^{2}\le ({x}^{2}+{y}^{2})({x}^{4}+{y}^{4})$$, đpcm.

========================================================================

https://khoahoc.vietjack.com/thi-online/bai-tap-toan-8-chu-de-6-bat-dang-thuc-co-dap-an/108043


\textbf{{QUESTION}}

Chứng minh rằng với a, b, c tùy ý ta luôn có:
ab+bc+ca≤a2+b2+c2$$ ab+bc+ca\le {a}^{2}+{b}^{2}+{c}^{2}$$

\textbf{{ANSWER}}

Ta có:
VT2=(ab+bc+ca)2≤(a2+b2+c2)(b2+c2+a2)=(a2+b2+c2)2$$ V{T}^{2}={\left(ab+bc+ca\right)}^{2}\le \left({a}^{2}+{b}^{2}+{c}^{2}\right)({b}^{2}+{c}^{2}+{a}^{2})={\left({a}^{2}+{b}^{2}+{c}^{2}\right)}^{2}$$
Lấy căn bậc hai của hai vế, ta đi đến:
a2+b2+c2≥|ab+bc+ca|≥ab+bc+ca$$ {a}^{2}+{b}^{2}+{c}^{2}\ge \left|ab+bc+ca\right|\ge ab+bc+ca$$, đpcm.

========================================================================

https://khoahoc.vietjack.com/thi-online/bai-tap-toan-8-chu-de-6-bat-dang-thuc-co-dap-an/108043


\textbf{{QUESTION}}

Cho a, b, c là độ dài ba cạnh của một tam giác, p là một nửa chu vi. Chứng minh rằng: √p<√p−a+√p−b+√p−c≤√3p$$ \sqrt{p}<\sqrt{p-a}+\sqrt{p-b}+\sqrt{p-c}\le \sqrt{3p}$$

\textbf{{ANSWER}}

- Ta có:
(√p−a+√p−b+√p−c)2=(1.√p−a+1.√p−b+1.√p−c)2                          ≤(12+12+12)(p−a+p−b+p−c)=3p⇔√p−a+√p−b+√p−c≤√3p                                     (1)$$ {\left(\sqrt{p-a}+\sqrt{p-b}+\sqrt{p-c}\right)}^{2}\phantom{\rule{0ex}{0ex}}={\left(1.\sqrt{p-a}+1.\sqrt{p-b}+1.\sqrt{p-c}\right)}^{2}\phantom{\rule{0ex}{0ex}}\quad \quad \quad \quad \quad \quad \quad \quad \quad \quad \quad \quad \quad \quad \quad \quad \quad \quad \quad \quad \quad \quad \quad \quad \quad \quad \le \left({1}^{2}+{1}^{2}+{1}^{2}\right)\left(p-a+p-b+p-c\right)=3p\phantom{\rule{0ex}{0ex}}\Leftrightarrow \sqrt{p-a}+\sqrt{p-b}+\sqrt{p-c}\le \sqrt{3p}\quad \quad \quad \quad \quad \quad \quad \quad \quad \quad \quad \quad \quad \quad \quad \quad \quad \quad \quad \quad \quad \quad \quad \quad \quad \quad \quad \quad \quad \quad \quad \quad \quad \quad \quad \quad \quad \left(1\right)$$
Dấu đẳng thức xảy ra khi:
√p−a1=√p−b1=√p−c1⇔a=b=c$$ \frac{\sqrt{p-a}}{1}=\frac{\sqrt{p-b}}{1}=\frac{\sqrt{p-c}}{1}\Leftrightarrow a=b=c$$
- Ta đi chứng minh:
√p<√p−a+√p−b+√p−c$$ \sqrt{p}<\sqrt{p-a}+\sqrt{p-b}+\sqrt{p-c}$$
Bằng phép biến đổi tương đương, cụ thể:
√p<√p−a+√p−b+√p−c⇔p<p−a+p−b+p−c                             +2√(p−a)(p−b)+2√(p−c)(p−a)+2√(p−b)(p−c)⇔0<2√(p−a)(p−b)+2√(p−c)(p−a)+2√(p−b)(p−c)$$ \sqrt{p}<\sqrt{p-a}+\sqrt{p-b}+\sqrt{p-c}\phantom{\rule{0ex}{0ex}}\Leftrightarrow p<p-a+p-b+p-c\phantom{\rule{0ex}{0ex}}\quad \quad \quad \quad \quad \quad \quad \quad \quad \quad \quad \quad \quad \quad \quad \quad \quad \quad \quad \quad \quad \quad \quad \quad \quad \quad \quad \quad \quad +2\sqrt{(p-a)(p-b)}+2\sqrt{(p-c)(p-a)}+2\sqrt{(p-b)(p-c)}\phantom{\rule{0ex}{0ex}}\Leftrightarrow 0<2\sqrt{(p-a)(p-b)}+2\sqrt{(p-c)(p-a)}+2\sqrt{(p-b)(p-c)}$$

========================================================================

https://khoahoc.vietjack.com/thi-online/bai-tap-toan-8-chu-de-6-bat-dang-thuc-co-dap-an/108043


\textbf{{QUESTION}}

Cho a, b, c là ba số khác 0. Chứng minh rằng: a2b2+b2c2+c2a2≥ab+bc+ca$$ \frac{{a}^{2}}{{b}^{2}}+\frac{{b}^{2}}{{c}^{2}}+\frac{{c}^{2}}{{a}^{2}}\ge \frac{a}{b}+\frac{b}{c}+\frac{c}{a}$$

\textbf{{ANSWER}}

Áp dụng bất đẳng thức Bu-nhi-a-cốp-xki ta có:
(a2b2+b2c2+c2a2)(12+12+12)≥(|ab|+|bc|+|ca|)2$$ \left(\frac{{a}^{2}}{{b}^{2}}+\frac{{b}^{2}}{{c}^{2}}+\frac{{c}^{2}}{{a}^{2}}\right)\left({1}^{2}+{1}^{2}+{1}^{2}\right)\ge {\left(\left|\frac{a}{b}\right|+\left|\frac{b}{c}\right|+\left|\frac{c}{a}\right|\right)}^{2}$$
Suy ra:
a2b2+b2c2+c2a2≥(|ab|+|bc|+|ca|).13.(|ab|+|bc|+|ca|)$$ \frac{{a}^{2}}{{b}^{2}}+\frac{{b}^{2}}{{c}^{2}}+\frac{{c}^{2}}{{a}^{2}}\ge \left(\left|\frac{a}{b}\right|+\left|\frac{b}{c}\right|+\left|\frac{c}{a}\right|\right).\frac{1}{3}.\left(\left|\frac{a}{b}\right|+\left|\frac{b}{c}\right|+\left|\frac{c}{a}\right|\right)$$                (*)
Nhận xét rằng:
13.(|ab|+|bc|+|ca|)≥3√|abcabc|=1|ab|+|bc|+|ca|≥ab+bc+ca$$ \frac{1}{3}.\left(\left|\frac{a}{b}\right|+\left|\frac{b}{c}\right|+\left|\frac{c}{a}\right|\right)\ge \sqrt[3]{\left|\frac{abc}{abc}\right|}=1\phantom{\rule{0ex}{0ex}}\phantom{\rule{0ex}{0ex}}\left|\frac{a}{b}\right|+\left|\frac{b}{c}\right|+\left|\frac{c}{a}\right|\ge \frac{a}{b}+\frac{b}{c}+\frac{c}{a}$$
Suy ra
(|ab|+|bc|+|ca|).13.(|ab|+|bc|+|ca|)≥ab+bc+ca$$ \left(\left|\frac{a}{b}\right|+\left|\frac{b}{c}\right|+\left|\frac{c}{a}\right|\right).\frac{1}{3}.\left(\left|\frac{a}{b}\right|+\left|\frac{b}{c}\right|+\left|\frac{c}{a}\right|\right)\ge \frac{a}{b}+\frac{b}{c}+\frac{c}{a}$$
Từ đó (*) được biến đổi:
a2b2+b2c2+c2a2≥ab+bc+ca$$ \frac{{a}^{2}}{{b}^{2}}+\frac{{b}^{2}}{{c}^{2}}+\frac{{c}^{2}}{{a}^{2}}\ge \frac{a}{b}+\frac{b}{c}+\frac{c}{a}$$
Dễ nhận thấy dấu đẳng thức xảy ra khi a = b = c

========================================================================

https://khoahoc.vietjack.com/thi-online/bai-tap-toan-8-chu-de-6-bat-dang-thuc-co-dap-an/108043


\textbf{{QUESTION}}

Hai số x, y thỏa mãn x2+y2=1$$ {x}^{2}+{y}^{2}=1$$. Chứng minh rằng −5≤3x+4y≤5$$ -5\le 3x+4y\le 5$$.

\textbf{{ANSWER}}

Ta có: (3x+4y)2≤(32+42)(x2+y2)=25$$ {\left(3x+4y\right)}^{2}\le \left({3}^{2}+{4}^{2}\right)\left({x}^{2}+{y}^{2}\right)=25$$
Lấy căn hai vế, ta được:
|3x+4y|≤5⇔−≤3x+4y≤5$$ \left|3x+4y\right|\le 5\Leftrightarrow -\le 3x+4y\le 5$$, đpcm.

========================================================================

https://khoahoc.vietjack.com/thi-online/chuyen-de-toan-11-canh-dieu-chuyen-de-2-lam-quen-voi-mot-vai-yeu-to-cua-li-thuyet-do-thi-co-dap-an/124673


\textbf{{QUESTION}}

Như chúng ta đã biết, Lí thuyết đồ thị ra đời trong quá trình khái quát, mô phỏng những vấn đề của khoa học và thực tiễn thành những mô hình toán học. Vì thế, các kết quả của Lí thuyết đồ thị có nhiều ứng dụng trong khoa học và thực tiễn. 
Lí thuyết đồ thị có thể giải quyết những vấn đề thực tiễn nào?

\textbf{{ANSWER}}

Qua bài học này, ta thấy Lí thuyết đồ thị có thể giải quyết những vấn đề thực tiễn: 
- Vấn đề về tìm đường đi ngắn nhất trong những trường hợp đơn giản.
- Vấn đề liên quan đến khoa học tự nhiên và công nghệ.

========================================================================

https://khoahoc.vietjack.com/thi-online/55-cau-trac-nghiem-he-thuc-luong-trong-tam-giac-co-dap-an


\textbf{{QUESTION}}

Cho tam giác ABC có AB = 4, AC = 6, $$ \hat{A}=120°.$$ Độ dài cạnh BC là:
A. $$ \sqrt{19}$$
B. $$ 2\sqrt{19}$$
C. $$ 3\sqrt{19}$$
D. $$ 2\sqrt{7}$$

\textbf{{ANSWER}}

Áp dụng định lí cô sin  trong tam giác  ta có:  
$$ B{C}^{2}=A{B}^{2}+A{C}^{2}-2\quad AB.AC.\mathrm{cos}A={4}^{2}+{6}^{2}-\mathrm{2.4.6.}\mathrm{cos}120°$$
$$ ={4}^{2}+{6}^{2}-\mathrm{2.4.6.}\left(-\frac{1}{2}\right)=76\Rightarrow BC=\sqrt{76}=2\sqrt{19}$$.
 
Chọn B.

========================================================================

https://khoahoc.vietjack.com/thi-online/55-cau-trac-nghiem-he-thuc-luong-trong-tam-giac-co-dap-an


\textbf{{QUESTION}}

Cho tam giác ABC có AB = 4, AC = 5, BC = 6. Giá trị cos A bằng
A. 0,125
B. 0,25
C. 0,5
D. 0,0125

\textbf{{ANSWER}}

Áp dụng hệ quả của định lí cô sin trong tam giác ta có:  
 $$ cosA=({b}^{2}+{c}^{2}-{a}^{2})/2bc=({5}^{2}+{4}^{2}-{6}^{2})/2.5.4=1/8=0,125$$. 
Chọn A.

========================================================================

https://khoahoc.vietjack.com/thi-online/55-cau-trac-nghiem-he-thuc-luong-trong-tam-giac-co-dap-an


\textbf{{QUESTION}}

Cho tam giác ABC có a = 3, b = 5, c = 6. Giá trị của mc$$ {m}_{c}$$ bằng
A. √2$$ \sqrt{2}$$
B. 2√2$$ 2\sqrt{2}$$
C. 3
D. √10$$ \sqrt{10}$$

\textbf{{ANSWER}}

Áp dụng công thức độ dài đường trung tuyến ta có: 
 $$ {{m}_{c}}^{2}=\frac{{a}^{2}+{b}^{2}}{2}-\frac{{c}^{2}}{4}=\frac{{3}^{2}+{5}^{2}}{2}-\frac{{6}^{2}}{4}=8\Rightarrow {m}_{c}=2\sqrt{2}$$.
Chọn B.

========================================================================

https://khoahoc.vietjack.com/thi-online/55-cau-trac-nghiem-he-thuc-luong-trong-tam-giac-co-dap-an


\textbf{{QUESTION}}

Cho tam giác ABC. Khẳng định nào sau đây là đúng?
A. ma2+mb2+mc2=23(a2+b2+c2)$$ {{m}_{a}}^{2}+{{m}_{b}}^{2}+{{m}_{c}}^{2}=\frac{2}{3}\left({a}^{2}+{b}^{2}+{c}^{2}\right)$$
B. ma2+mb2+mc2=43(a2+b2+c2)$$ {{m}_{a}}^{2}+{{m}_{b}}^{2}+{{m}_{c}}^{2}=\frac{4}{3}\left({a}^{2}+{b}^{2}+{c}^{2}\right)$$
C. ma2+mb2+mc2=13(a2+b2+c2)$$ {{m}_{a}}^{2}+{{m}_{b}}^{2}+{{m}_{c}}^{2}=\frac{1}{3}\left({a}^{2}+{b}^{2}+{c}^{2}\right)$$
D. ma2+mb2+mc2=34(a2+b2+c2)$$ {{m}_{a}}^{2}+{{m}_{b}}^{2}+{{m}_{c}}^{2}=\frac{3}{4}\left({a}^{2}+{b}^{2}+{c}^{2}\right)$$

\textbf{{ANSWER}}

Sử dụng công thức trung tuyến, ta có:
ma2+mb2+mc2=2b2+2c2−a24+2c2+2a2−b24+2a2+2b2−c24$$ {{m}_{a}}^{2}+{{m}_{b}}^{2}+{{m}_{c}}^{2}=\frac{2{b}^{2}+2{c}^{2}-{a}^{2}}{4}+\frac{2{c}^{2}+2{a}^{2}-{b}^{2}}{4}+\frac{2{a}^{2}+2{b}^{2}-{c}^{2}}{4}$$
=34(a2+b2+c2)$$ =\frac{3}{4}\left({a}^{2}+{b}^{2}+{c}^{2}\right)$$
Chọn D

========================================================================

https://khoahoc.vietjack.com/thi-online/55-cau-trac-nghiem-he-thuc-luong-trong-tam-giac-co-dap-an


\textbf{{QUESTION}}

Cho tam giác ABC là tam giác đều cạnh a. Bán kính đường tròn ngoại tiếp tam giác ABC bằng.
A. a√33$$ \frac{a\sqrt{3}}{3}$$
B. a√32$$ \frac{a\sqrt{3}}{2}$$
C. a√34$$ \frac{a\sqrt{3}}{4}$$
D. a√22$$ \frac{a\sqrt{2}}{2}$$

\textbf{{ANSWER}}

Áp dụng định lí sin trong tam giác ta cóasinA=2R$$ \frac{a}{\mathrm{sin}A}=2R$$. Suy ra:
 R=a2sin60°=a2. √32=a√33$$ R=\frac{a}{2\mathrm{sin}60°}=\frac{a}{2.\quad \frac{\sqrt{3}}{2}}=\frac{a\sqrt{3}}{3}$$. 
Chọn A.

========================================================================

https://khoahoc.vietjack.com/thi-online/20-cau-trac-nghiem-toan-10-canh-dieu-ham-so-va-do-thi-co-dap-an-phan-2/111720


\textbf{{QUESTION}}

Tìm tất cả các giá trị thực của tham số m để hàm số y = $$ \frac{x+2m+2}{x-m}$$ xác định trên (−1; 0).

\textbf{{ANSWER}}

Đáp án đúng là: C
Hàm số xác định khi x – m ≠ 0 $$ \Leftrightarrow $$ x ≠ m.
$$ \Rightarrow $$ Tập xác định của hàm số là D = ℝ \ {m}.
Hàm số xác định trên (−1; 0) khi và chỉ khi m ∉ (−1; 0) $$ \Leftrightarrow $$ $$ \left[\begin{array}{c}m\ge 0\\ m\le -1\end{array}\right.$$.

========================================================================

https://khoahoc.vietjack.com/thi-online/20-cau-trac-nghiem-toan-10-canh-dieu-ham-so-va-do-thi-co-dap-an-phan-2/111720


\textbf{{QUESTION}}

Cho hàm số y = mx√x−m+2−1$$ \frac{mx}{\sqrt{x-m+2}-1}$$ với m là tham số. Tìm m để hàm số xác định trên (0; 1).

\textbf{{ANSWER}}

Đáp án đúng là: A
ĐKXĐ: $$ \left\{\begin{array}{c}x-m+2\ge 0\\ \sqrt{x-m+2}\ne 1\end{array}\right.$$ $$ \Leftrightarrow $$ $$ \left\{\begin{array}{c}x\ge m-2\\ x\ne m-1\end{array}\right.$$
Suy ra tập xác định của hàm số là D = [m – 2; +$$ \infty $$) \ {m – 1}.
Hàm số xác định trên (0; 1)
$$ \Leftrightarrow $$ (0; 1) $$ \subset $$ [m – 2; m – 1) $$ \cup $$ (m – 1; +$$ \infty $$)
$$ \Leftrightarrow $$ $$ \left[\begin{array}{c}(0;\text{\hspace{0.17em}}1)\subset \text{[}m-2;\text{\hspace{0.17em}}m-1)\\ (0;1)\subset (m-1;+\infty )\end{array}\right.$$
$$ \Leftrightarrow $$ $$ \left[\begin{array}{c}m=2\\ m-1\le 0\end{array}\right.$$
$$ \Leftrightarrow $$ $$ \left[\begin{array}{c}m=2\\ m\le 1\end{array}\right.$$
Vậy m Î (−¥; 1] È {2} là giá trị cần tìm.

========================================================================

https://khoahoc.vietjack.com/thi-online/20-cau-trac-nghiem-toan-10-canh-dieu-ham-so-va-do-thi-co-dap-an-phan-2/111720


\textbf{{QUESTION}}

Cho bốn đường thẳng:
(d1): y = x√3+1$$ x\sqrt{3}+1$$;
(d2): y = 3√3x−1$$ \frac{3}{\sqrt{3}}x-1$$;
(d3): y = −x√3+1$$ -x\sqrt{3}+1$$;
(d4): y = 3x + 2.
Hỏi cặp đường thẳng nào song song với nhau?

\textbf{{ANSWER}}

Đáp án đúng là: A
Cặp đường thẳng song song khi chúng có cùng hệ số góc và có tung độ góc khác nhau.
Ta có (d2): y = 3√3x−1$$ \frac{3}{\sqrt{3}}x-1$$ = x√3−1$$ x\sqrt{3}-1$$.
Hai đường thẳng (d1) và (d2) có cùng hệ số góc là √3$$ \sqrt{3}$$ và có tung độ khác nhau (1 ≠ −1) nên hai đường thẳng này song song với nhau.

========================================================================

https://khoahoc.vietjack.com/thi-online/20-cau-trac-nghiem-toan-10-canh-dieu-ham-so-va-do-thi-co-dap-an-phan-2/111720


\textbf{{QUESTION}}

Cho hai hàm số f(x) = 2x2 + 3x + 1 và g(x) = {x2+1     khi x>22x−1    khi −2≤x≤26−6x    khi x<−2$$ \left\{\begin{array}{c}{x}^{2}+1\text{\hspace{0.17em}\hspace{0.17em}\hspace{0.17em}\hspace{0.17em}\hspace{0.17em}}khi\text{\hspace{0.17em}}x>2\\ 2x-1\text{\hspace{0.17em}\hspace{0.17em}\hspace{0.17em}\hspace{0.17em}}khi\text{\hspace{0.17em}}-2\le x\le 2\\ 6-6x\text{\hspace{0.17em}\hspace{0.17em}\hspace{0.17em}\hspace{0.17em}}khi\text{\hspace{0.17em}}x<-2\end{array}\right.$$. Tìm x khi g(x) = 1.

\textbf{{ANSWER}}

Đáp án đúng là: A
Với x > 2 ta có g(x) = 1 ⇔$$ \Leftrightarrow $$ {x>2x2+1=1$$ \left\{\begin{array}{c}x>2\\ {x}^{2}+1=1\end{array}\right.$$ ⇔$$ \Leftrightarrow $$ {x>2x=0$$ \left\{\begin{array}{c}x>2\\ x=0\end{array}\right.$$ vô nghiệm.
Với −2 ≤ x ≤ 2 ta có g(x) = 1 ⇔$$ \Leftrightarrow $$ {−2≤x≤22x−1=1$$ \left\{\begin{array}{c}-2\le x\le 2\\ 2x-1=1\end{array}\right.$$ ⇔$$ \Leftrightarrow $$ x = 1.
Với x < −2 ta có g(x) = 1 ⇔$$ \Leftrightarrow $$ {x<−26−6x=1$$ \left\{\begin{array}{c}x<-2\\ 6-6x=1\end{array}\right.$$ ⇔$$ \Leftrightarrow $$ {x<−2x=56$$ \left\{\begin{array}{c}x<-2\\ x=\frac{5}{6}\end{array}\right.$$ vô nghiệm
Vậy g(x) = 1 ⇔$$ \Leftrightarrow $$ x = 1.

========================================================================

https://khoahoc.vietjack.com/thi-online/20-cau-trac-nghiem-toan-10-canh-dieu-ham-so-va-do-thi-co-dap-an-phan-2/111720


\textbf{{QUESTION}}

Giá thuê xe ô tô tự lái là 1,4 triệu đồng một ngày cho hai ngày đầu tiên và 800 nghìn đồng cho mỗi ngày tiếp theo. Tổng số tiền T phải trả là một hàm số của số ngày x mà khách thuê xe. Viết công thức của hàm số T = T(x).
A. T(x) = {1,4x  (0<x<2)0,8x+0,5 (x>2)$$ \left\{\begin{array}{c}\mathrm{1,4}x\text{\hspace{0.17em}\hspace{0.17em}}(0<x<2)\\ \mathrm{0,8}x+\mathrm{0,5}\text{\hspace{0.17em}}(x>2)\end{array}\right.$$;

\textbf{{ANSWER}}

Đáp án đúng là: D
Nếu 0 < x ≤ 2 thì T(x) = 1,4x (triệu đồng)
Nếu x > 2 thì T(x) = 1,4 . 2 + 0,8.(x – 2) = 0,8x + 1,2 (triệu đồng)
Số tiền phải trả sau khi thuê x ngày là
T(x) = {1,4x  (0<x≤2)0,8x+1,2 (x>2)$$ \left\{\begin{array}{c}\mathrm{1,4}x\text{\hspace{0.17em}\hspace{0.17em}}(0<x\le 2)\\ \mathrm{0,8}x+\mathrm{1,2}\text{\hspace{0.17em}}(x>2)\end{array}\right.$$.

========================================================================

https://khoahoc.vietjack.com/thi-online/giai-sbt-toan-8-canh-dieu-bai-tap-cuoi-chuong-iii-co-dap-an


\textbf{{QUESTION}}

Toạ độ giao điểm của hai đường thẳng ${d_1}:y = \frac{{1 - 3x}}{4}$ và ${d_2}:y = - \left( {\frac{x}{3} + 1} \right)$ là:
A. (0; ‒1).
B. $\left( { - \frac{7}{3};2} \right)$.
C. $\left( {0;\frac{1}{4}} \right)$.
D. (3; ‒2).

\textbf{{ANSWER}}

Lời giải
Đáp án đúng là: D
Cách 1:
Giả sử điểm A(x0; y0) là giao điểm của d1 và d2.
Do A(x0; y0) thuộc d1 nên ta có ${y_0} = \frac{{1 - 3{x_0}}}{4}\,\,\,\left( 1 \right)$
Do A(x0; y0) thuộc d2 nên ta có ${y_0} = - \left( {\frac{{{x_0}}}{3} + 1} \right)\,\,\,\left( 2 \right)$
Từ (1) và (2) ta có: 
\[\frac{{1 - 3{x_0}}}{4} =  - \left( {\frac{{{x_0}}}{3} + 1} \right)\]
Suy ra $\frac{1}{4} - \frac{3}{4}{x_0} = - \frac{{{x_0}}}{3} - 1$
Do đó $ - \frac{3}{4}{x_0} + \frac{{{x_0}}}{3} = - 1 - \frac{1}{4}$
Hay $\frac{{ - 5}}{{12}}{x_0} = \frac{{ - 5}}{4}$
Suy ra: x0 = 3
Thay x0 = 3 vào (1) ta có: ${y_0} = \frac{{1 - 3.3}}{4} = \frac{{ - 8}}{4} = - 2$
Vậy toạ độ giao điểm của hai đường thẳng d1 và d2 là: A(3; ‒2).
Cách 2:
• Xét điểm (0; ‒1)
Với x = 0, thay vào $y = \frac{{1 - 3x}}{4}$ ta được $y = \frac{1}{4}$, nên đường thẳng d1 không đi qua điểm (0; ‒1). Do đó phương án A là sai.
• Xét điểm $\left( { - \frac{7}{3};2} \right)$
Với $x = - \frac{7}{3}$, thay vào $y = \frac{{1 - 3x}}{4}$ ta được $y = \frac{{1 - 3.\left( { - \frac{7}{3}} \right)}}{4} = \frac{{1 + 7}}{4} = \frac{8}{4} = 2$, nên đường thẳng d1 đi qua điểm $\left( { - \frac{7}{3};2} \right)$.
Với $x = - \frac{7}{3}$, thay vào $y = - \left( {\frac{x}{3} + 1} \right)$ ta được $y = - \left( {\frac{{ - \frac{7}{3}}}{3} + 1} \right) = - \left( { - \frac{7}{9} + 1} \right) = - \frac{2}{9}$, nên đường thẳng d2 không đi qua điểm $\left( { - \frac{7}{3};2} \right)$.
Do đó phương án B là sai.
• Xét điểm $\left( {0;\frac{1}{4}} \right)$
Với x = 0, thay vào $y = \frac{{1 - 3x}}{4}$ ta được $y = \frac{1}{4}$, nên đường thẳng d1 đi qua điểm $\left( {0;\frac{1}{4}} \right)$.
Với x = 0, thay vào $y = - \left( {\frac{x}{3} + 1} \right)$ ta được y = –1, nên đường thẳng d2 không đi qua điểm $\left( {0;\frac{1}{4}} \right)$.
Do đó phương án C là sai.
• Xét điểm (3; ‒2)
Với x = 3, thay vào $y = \frac{{1 - 3x}}{4}$ ta được $y = \frac{{1 - 3.3}}{4} = \frac{{1 - 9}}{4} = \frac{{ - 8}}{4} = - 2$, nên đường thẳng d1 đi qua điểm (3; ‒2).
Với x = 3, thay vào $y = - \left( {\frac{x}{3} + 1} \right)$ ta được $y = - \left( {\frac{3}{3} + 1} \right) = - 2$, nên đường thẳng d2 đi qua điểm (3; ‒2).
Do đó phương án D là đúng.

========================================================================

https://khoahoc.vietjack.com/thi-online/giai-sbt-toan-8-canh-dieu-bai-tap-cuoi-chuong-iii-co-dap-an


\textbf{{QUESTION}}

Cho đồ thị của hàm số y = ax + b đi qua điểm M(1; 4) và song song với đường thẳng y = 2x + 1. Tích a.b bằng:
A. 6.
B. 4.
C. 3.
D. 2.

\textbf{{ANSWER}}

Lời giải
Đáp án đúng là: B
Do đường thẳng y = ax + b (a ≠ 0) song song với đường thẳng y = 2x + 1 nên a = 2 (thoả mãn) và b ≠ 1. 
Mà đường thẳng y = ax + b đi qua điểm M(1; 4) suy ra 4 = 2.1 + b hay b = 2 (thoả mãn). 
Suy ra tích a.b là 2.2 = 4.

========================================================================

https://khoahoc.vietjack.com/thi-online/11-bai-tap-tim-mot-so-khi-biet-gssia-tri-phan-so-cua-no-co-loi-giai


\textbf{{QUESTION}}

A. $$ \frac{1\text{\hspace{0.17em}\hspace{0.17em}}105}{2}$$
B. $$ \frac{1\text{\hspace{0.17em}\hspace{0.17em}}051}{2}$$
C. $$ \frac{1\text{\hspace{0.17em}\hspace{0.17em}}150}{2}$$
D. $$ \frac{1\text{\hspace{0.17em}\hspace{0.17em}}015}{2}$$

\textbf{{ANSWER}}

Đáp án đúng là: D
 $$ \frac{2}{7}$$ của 145 là:
 $$ 145:\frac{2}{7}=145\cdot \frac{7}{2}=\frac{1\text{\hspace{0.17em}\hspace{0.17em}}015}{2}$$.

========================================================================

https://khoahoc.vietjack.com/thi-online/11-bai-tap-tim-mot-so-khi-biet-gssia-tri-phan-so-cua-no-co-loi-giai


\textbf{{QUESTION}}

Biết   58$$ \frac{5}{8}$$quả bưởi nặng 500 gam. Hỏi quả bưởi nặng bao nhiêu gam?

\textbf{{ANSWER}}

Đáp án đúng là: B
Quả bưởi nặng số gam là:
  $$ 500:\frac{5}{8}=500\cdot \frac{8}{5}=800$$(g).

========================================================================

https://khoahoc.vietjack.com/thi-online/11-bai-tap-tim-mot-so-khi-biet-gssia-tri-phan-so-cua-no-co-loi-giai


\textbf{{QUESTION}}

Tuổi của con hiện nay là 14 và bằng   27 $$ \frac{2}{7\quad }$$ tuổi bố. Tổng số tuổi của bố và con hiện nay là:

\textbf{{ANSWER}}

Đáp án đúng là: C
Tuổi của bố hiện nay là:
 14:27=14⋅72=49$$ 14:\frac{2}{7}=14\cdot \frac{7}{2}=49$$ (tuổi)
Tổng số tuổi của bố và con hiện nay là:
49 + 14 = 63 (tuổi).

========================================================================

https://khoahoc.vietjack.com/thi-online/11-bai-tap-tim-mot-so-khi-biet-gssia-tri-phan-so-cua-no-co-loi-giai


\textbf{{QUESTION}}

Một cửa hàng nhập về một số muối để bán. Buổi sáng cửa hàng bán được 72 kg muối và bằng 835$$ \frac{8}{35}$$  số muối cửa hàng nhập về. Buổi chiều cửa hàng bán được 56 kg muối. Vậy cửa hàng còn lại ….. kg muối. Số thích hợp điền vào chỗ trống là:

\textbf{{ANSWER}}

Đáp án đúng là: C
Cửa hàng đã nhập về số kg muối là:
 72:835=72⋅358=315$$ 72:\frac{8}{35}=72\cdot \frac{35}{8}=315$$(kg)
Cửa hàng còn lại số kg muối là:
315 – 72 – 56 = 187 (kg).

========================================================================

https://khoahoc.vietjack.com/thi-online/11-bai-tap-tim-mot-so-khi-biet-gssia-tri-phan-so-cua-no-co-loi-giai


\textbf{{QUESTION}}

Biết 75% của một tấm lụa dài 3 m. Cả tấm lụa đó dài:

\textbf{{ANSWER}}

Đáp án đúng là: A
Tấm lụa đó dài:
  3:75%=3:75100=3⋅10075=3⋅43=4$$ 3:75\%=3:\frac{75}{100}=3\cdot \frac{100}{75}=3\cdot \frac{4}{3}=4$$(m).

========================================================================

https://khoahoc.vietjack.com/thi-online/chuong-3-bai-7-tu-giac-noi-tiep/59191


\textbf{{QUESTION}}

Cho điểm C nằm trên nửa đường tròn (O) vói đường kính AB sao cho cung $$ \stackrel{⏜}{AC}$$ lớn hơn cung $$ \stackrel{⏜}{BC}$$ (C ≠ B). Đường thẳng vuông góc vói AB tại O cắt dây AC tại D. Chứng minh tứ giác BCDO nội tiếp

\textbf{{ANSWER}}

Học sinh tự chứng minh

========================================================================

https://khoahoc.vietjack.com/thi-online/chuong-3-bai-7-tu-giac-noi-tiep/59191


\textbf{{QUESTION}}

Cho đường tròn (O) đường kính AB. Trên đoạn thẳng OB lấy điểm H bất kì (H không trùng O, B). Trên đường thẳng vuông góc với OB tại H, lấy một điểm M ở ngoài đường tròn; MA và MB thứ tự cắt đường tròn (O) tại C và D. Gọi I là giao điểm của AD và BC. Chứng minh MCID và MCHB là tứ giác nội tiếp

\textbf{{ANSWER}}

Học sinh tự chứng minh

========================================================================

https://khoahoc.vietjack.com/thi-online/chuong-3-bai-7-tu-giac-noi-tiep/59191


\textbf{{QUESTION}}

Cho hai đường tròn (O) và (O') cắt nhau tại A, B. Kẻ đường kính AC của (O) cắt đường tròn (O’) tại F. Kẻ đường kính AE của (O') cắt đưòng tròn (O) tại G. Chứng minh:
a, Tứ giác GFEC nội tiếp
b, GC, FE và AB đồng quy

\textbf{{ANSWER}}

Học sinh tự chứng minh

========================================================================

https://khoahoc.vietjack.com/thi-online/chuong-3-bai-7-tu-giac-noi-tiep/59191


\textbf{{QUESTION}}

Cho tam giác ABC cân tại A. Đường thẳng xy song song với BC cắt AB tại E và cắt AC tại F. Chúng minh tứ giác EFCB nội tiếp

\textbf{{ANSWER}}

Gợi ý: Chứng minh BEFC là hình thang cân

========================================================================

https://khoahoc.vietjack.com/thi-online/chuong-3-bai-7-tu-giac-noi-tiep/59191


\textbf{{QUESTION}}

Cho tam giác ABC vuông tại A, đường cao AH. Kẻ HE vuông góc với AB tại E, Kẻ HF vuông góc với AC tại F. Chứng minh tứ giác BEFC nội tiếp

\textbf{{ANSWER}}

Gợi ý: ^AFE=^AHE$$ \hat{AFE}=\hat{AHE}$$ (tính chất hình chữ nhật và ^AHE=^ABH$$ \hat{AHE}=\hat{ABH}$$ (cùng phụ ^BHE$$ \hat{BHE}$$)

========================================================================

https://khoahoc.vietjack.com/thi-online/de-kiem-tra-cuoi-ki-2-toan-8-co-dap-an-moi-nhat/100150


\textbf{{QUESTION}}

Cho biểu thức: A = $$ \frac{x+3}{{x}^{2}-4}$$ và B = $$ \frac{{x}^{2}}{{x}^{2}-4}+\frac{1}{2-x}$$ − $$ \frac{x}{x+2}$$ (x ≠ ±2)
a) Tính giá trị biểu thức A khi x = 3.
b) Rút gọn B.
c) Cho P = $$ \frac{B}{A}$$. Tìm x để P < 1.

\textbf{{ANSWER}}

a) Thay x = 3 (thỏa mãn điều kiện) vào biểu thức A = $$ \frac{x+3}{{x}^{2}-4}$$ ta được:
A = $$ \frac{3+3}{{3}^{2}-4}$$ = $$ \frac{6}{5}$$.
Vậy với x = 3 thì A = $$ \frac{6}{5}$$.
b) Với x ≠ ±2 ta có:
Ta có: B = $$ \frac{{x}^{2}}{{x}^{2}-4}+\frac{1}{2-x}$$ − $$ \frac{x}{x+2}$$ 
= $$ \frac{{x}^{2}}{\left(x-2\right)\left(x+2\right)}-\frac{1}{x-2}$$ − $$ \frac{x}{x+2}$$ 
= $$ \frac{{x}^{2}}{\left(x-2\right)\left(x+2\right)}$$ − $$ \frac{\left(x+2\right)}{\left(x-2\right)\left(x+2\right)}$$ − $$ \frac{x\left(x-2\right)}{\left(x+2\right)\left(x-2\right)}$$
= $$ \frac{{x}^{2}-\left(x+2\right)-x\left(x-2\right)}{\left(x-2\right)\left(x+2\right)}$$
= $$ \frac{{x}^{2}-x-2-{x}^{2}+2x}{\left(x-2\right)\left(x+2\right)}$$
= $$ \frac{x-2}{\left(x-2\right)\left(x+2\right)}$$
= $$ \frac{1}{x+2}$$.
Vậy với x ≠ ±2 thì B = $$ \frac{1}{x+2}.$$
c) Với x ≠ ±2 ta có:
P = $$ \frac{B}{A}$$ = $$ \frac{1}{x+2}$$ : $$ \frac{x+3}{{x}^{2}-4}$$
=  $$ \frac{1}{x+2}$$. $$ \frac{{x}^{2}-4}{x+3}$$
=  $$ \frac{1}{x+2}$$. $$ \frac{\left(x-2\right)\left(x+2\right)}{x+3}$$
= $$ \frac{x-2}{x+3}$$
Ta có: 
P < 1 $$ \Leftrightarrow $$$$ \frac{x-2}{x+3}$$ < 1
$$ \Leftrightarrow $$ $$ \frac{x-2}{x+3}$$− 1 < 0
$$ \Leftrightarrow $$ $$ \frac{x-2-x-3}{x+3}$$< 0
$$ \Leftrightarrow $$ $$ \frac{-5}{x+3}$$ < 0
Û x + 3 > 0 (vì –5 < 0)
$$ \Leftrightarrow $$ x > −3
Kết hợp điều kiện x ≠ ±2 ta có: 
x > −3 và x ≠ ±2.
Vậy với x > −3 và x ≠ ±2 thì P < 1.

========================================================================

https://khoahoc.vietjack.com/thi-online/de-kiem-tra-cuoi-ki-2-toan-8-co-dap-an-moi-nhat/100150


\textbf{{QUESTION}}

Giải bài toán bằng cách lập phương trình
Hai ô tô khởi hành một lúc tại A để đi đến B. Ô tô thứ nhất đi với vận tốc 40 km/h. Ô tô thứ hai đi với vận tốc 50 km/h. Biết rằng ô tô thứ nhất tới B chậm hơn ô tô thứ hai 30 phút. Tính độ dài quãng đường AB.

\textbf{{ANSWER}}

Gọi s (km) là quãng đường AB (s > 0)
Thời gian mà ô tô thứ nhất đi từ A đến B là: s40$$ \frac{s}{40}$$(giờ)
Thời gian mà ô tô thứ hai đi từ A đến B là: s50$$ \frac{s}{50}$$ (giờ)
Theo đề bài ô tô thứ nhất tới chậm hơn ô tô thứ hai 30 phút (0,5 giờ) vậy nên ta có:
s40$$ \frac{s}{40}$$ − s50$$ \frac{s}{50}$$ = 0,5
⇔50s−40s2000=0,5$$ \Leftrightarrow \frac{50s-40s}{2000}=0,5$$
Û 50s – 40s = 1 000
Û 10s = 1 000
Û s = 100 (km)
Vậy quãng đường AB dài 100 km.

========================================================================

https://khoahoc.vietjack.com/thi-online/de-kiem-tra-cuoi-ki-2-toan-8-co-dap-an-moi-nhat/100150


\textbf{{QUESTION}}

Giải phương trình
a) 5(3x − 2) − 4(5 − 3x) = 1
b) |x + 1| − 2x = 12

\textbf{{ANSWER}}

a) 5(3x − 2) − 4(5 − 3x) = 1
Û 15x – 10 – 20 + 12x = 1
Û 15x + 12x = 1 + 10 + 20
Û 27x = 31
Û x = 3127$$ \frac{31}{27}$$
Vậy tập nghiệm của phương trình là S = {3127}$$ \left\{\frac{31}{27}\right\}$$.
b) |x + 1| − 2x = 12
Trường hợp 1: Với x ≥ −1 thì: |x + 1| = x + 1
Khi đó ta có:
|x + 1| − 2x = 12
Û x + 1 − 2x = 12
Û –x = 11
Û x = –11 (không thỏa mãn x ≥ −1)
Trường hợp 2: Với x < −1 thì: |x + 1| = –x – 1
Khi đó ta có:
|x + 1| − 2x = 12
Û − x − 1 − 2x = 12
Û − 3x = 13
Û x = −133$$ \frac{13}{3}$$ (thỏa mãn x < −1)
Vậy tập nghiệm của phương trình là S = {−133}$$ \left\{-\frac{13}{3}\right\}$$.

========================================================================

https://khoahoc.vietjack.com/thi-online/toan-9-tap-1-phan-dai-so/19458


\textbf{{QUESTION}}

Tính √(9,11)

\textbf{{ANSWER}}

√9,11 = 3,018

========================================================================

https://khoahoc.vietjack.com/thi-online/toan-9-tap-1-phan-dai-so/19458


\textbf{{QUESTION}}

Tính √(39,82)

\textbf{{ANSWER}}

√39,82 = 6,310

========================================================================

https://khoahoc.vietjack.com/thi-online/toan-9-tap-1-phan-dai-so/19458


\textbf{{QUESTION}}

Tìm  √911

\textbf{{ANSWER}}

√911 = √9,11.√100 = 3,018.10 = 30,18

========================================================================

https://khoahoc.vietjack.com/thi-online/toan-9-tap-1-phan-dai-so/19458


\textbf{{QUESTION}}

Tìm √988

\textbf{{ANSWER}}

√988 = √9,88.√100 = 3,143.10 = 31,43

========================================================================

https://khoahoc.vietjack.com/thi-online/toan-9-tap-1-phan-dai-so/19458


\textbf{{QUESTION}}

Dùng bảng căn bậc hai, tìm giá trị gần đúng của nghiệm phương trình
x2 = 0,3982.

\textbf{{ANSWER}}

x2 = 0,3982 ⇔ x = ±√0,3982
Ta có: 0,3982 = 39,82:100
Do đó: √0,3982 = √39,82 : √100 = 6,310 : 10 = 0,631
Vậy x = ±0,631

========================================================================

https://khoahoc.vietjack.com/thi-online/18-cau-trac-nghiem-toan-7-chan-troi-sang-tao-bai-tap-cuoi-chuong-9-co-dap-an-phan-2/109925


\textbf{{QUESTION}}

Tung một đồng xu ba lần và ghi lại kết quả. Trong các biến cố sau, biến cố nào là biến cố không thể?

\textbf{{ANSWER}}

Đáp án đúng là: B
⦁ Biến cố M là biến cố ngẫu nhiên vì ta không biết trước kết quả của nó. Chẳng hạn, nếu lần tung thứ nhất xuất hiện mặt ngửa thì biến cố M xảy ra; còn nếu xuất hiện mặt sấp thì biến cố M không xảy ra.
⦁ Biến cố N là biến cố không thể vì khi tung đồng xu, kết quả chỉ có thể xuất hiện mặt ngửa hoặc mặt sấp, nên khi tung đồng xu 3 lần không thể có ba kết quả khác nhau xảy ra.
⦁ Biến cố P là biến cố ngẫu nhiên vì ta không thể biết trước kết quả của nó. Chẳng hạn, nếu hai lần tung đầu tiên xuất hiện mặt sấp, lần tung thứ ba xuất hiện mặt ngửa thì biến cố P xảy ra; còn nếu cả ba lần tung đều xuất hiện mặt ngửa thì biến cố P không xảy ra.
⦁ Biến cố Q là biến cố ngẫu nhiên vì ta không thể biết trước kết quả của nó. Chẳng hạn, nếu cả ba lần tung đều xuất hiện mặt sấp thì biến cố Q xảy ra; còn nếu hai lần tung đầu tiên xuất hiện mặt sấp, lần tung thứ ba xuất hiện mặt ngửa thì biến cố Q không xảy ra.
Vậy ta chọn phương án B.

========================================================================

https://khoahoc.vietjack.com/thi-online/18-cau-trac-nghiem-toan-7-chan-troi-sang-tao-bai-tap-cuoi-chuong-9-co-dap-an-phan-2/109925


\textbf{{QUESTION}}

Thầy Bình có 1 quyển sách tham khảo môn Toán, 2 quyển sách tham khảo môn Vật lí và 4 quyển sách tham khảo môn Hóa học. Thầy chọn ngẫu nhiên 1 quyển sách để làm phần thưởng cho một học sinh. Xác suất của biến cố A: “Quyển sách được chọn là quyển sách tham khảo môn Toán” là:
A. P(A)=17$$ P\left(A\right)=\frac{1}{7}$$;

\textbf{{ANSWER}}

Đáp án đúng là: A
Thầy Bình có tất cả 1 + 2 + 4 = 7 quyển sách.
Ta thấy các quyển sách đều có cùng khả năng được chọn.
Do chỉ có 1 quyển sách tham khảo môn Toán nên xác suất của biến cố A: “Quyển sách được chọn là quyển sách tham khảo môn Toán” là: $$ P\left(A\right)=\frac{1}{7}$$.
Vậy ta chọn phương án A.

========================================================================

https://khoahoc.vietjack.com/thi-online/18-cau-trac-nghiem-toan-7-chan-troi-sang-tao-bai-tap-cuoi-chuong-9-co-dap-an-phan-2/109925


\textbf{{QUESTION}}

Một hộp chứa 11 quả cầu gồm 5 quả cầu màu xanh và 6 quả cầu màu đỏ. Chọn ngẫu nhiên đồng thời 2 quả cầu từ hộp đó. Cho biến cố B: “Hai quả cầu được chọn có đủ 2 màu”. Khi đó biến cố B là:

\textbf{{ANSWER}}

Đáp án đúng là: A
Biến cố B là biến cố ngẫu nhiên vì ta không thể biết trước kết quả của nó. Chẳng hạn, nếu lấy được 1 quả cầu màu xanh và 1 quả cầu màu đỏ thì biến cố B xảy ra; còn nếu lấy được cả 2 quả cầu đều có màu xanh thì biến cố B không xảy ra.
Vậy ta chọn phương án A.

========================================================================

https://khoahoc.vietjack.com/thi-online/18-cau-trac-nghiem-toan-7-chan-troi-sang-tao-bai-tap-cuoi-chuong-9-co-dap-an-phan-2/109925


\textbf{{QUESTION}}

A. 0;

\textbf{{ANSWER}}

Đáp án đúng là: D
Cô Hoa có tất cả 1 + 2 + 3 = 6 đôi giày.
Vì khả năng được chọn của mỗi đôi giày là như nhau và cô Hoa có 3 đôi giày cao gót.
Nên xác suất của biến cố: “Đôi giày Hoa chọn là đôi giày cao gót” là 36=12.$$ \frac{3}{6}=\frac{1}{2}.$$
Vậy ta chọn phương án D.

========================================================================

https://khoahoc.vietjack.com/thi-online/18-cau-trac-nghiem-toan-7-chan-troi-sang-tao-bai-tap-cuoi-chuong-9-co-dap-an-phan-2/109925


\textbf{{QUESTION}}

Người ta gieo một con xúc xắc 6 mặt cân đối và đồng chất hai lần. Xét biến cố J: “Kết quả sau hai lần gieo có số chấm khác nhau”. Khi đó biến cố J là:

\textbf{{ANSWER}}

Đáp án đúng là: B
Biến cố J là biến cố ngẫu nhiên vì ta không thể biết trước kết quả của nó. Chẳng hạn, nếu lần gieo thứ nhất có số chấm là 3 và lần gieo thứ hai có số chấm là 5 thì biến cố J xảy ra; còn nếu cả hai lần gieo đều xuất hiện mặt có số chấm là 2 thì biến cố J không xảy ra.
Vậy ta chọn phương án B.

========================================================================

https://khoahoc.vietjack.com/thi-online/10-bai-tap-nhan-biet-cac-so-chia-het-cho-2-cho-5-cho-3-cho-9-co-loi-giai


\textbf{{QUESTION}}

Cho các số sau: 2 022; 5 125; 6 607; 8 679. Số nào chia hết cho 2?
A. 2 022;
B. 5 125;
C. 6 607;
D. 8 679.

\textbf{{ANSWER}}

Đáp án đúng là: A
Trong các số trên chỉ có số 2 022 có tận cùng là 2 nên 2 022 chia hết cho 2.

========================================================================

https://khoahoc.vietjack.com/thi-online/10-bai-tap-nhan-biet-cac-so-chia-het-cho-2-cho-5-cho-3-cho-9-co-loi-giai


\textbf{{QUESTION}}

Cho các số sau: 120; 132; 144; 155; 168; 189. Số nào chia hết cho 5?
A. 120; 132;
B. 144; 155;
C. 168; 189;
D. 120; 155.

\textbf{{ANSWER}}

Đáp án đúng là: D
Trong các số trên có các số: 120; 155 là có chữ số tận cùng lần lượt là 0; 5 nên 120; 155 chia hết cho 5.

========================================================================

https://khoahoc.vietjack.com/thi-online/10-bai-tap-nhan-biet-cac-so-chia-het-cho-2-cho-5-cho-3-cho-9-co-loi-giai


\textbf{{QUESTION}}

Cho các số sau: 105; 110; 236; 475; 198; 220. Có bao nhiêu số chia hết cho 3 trong các số trên?
A. 2;
B. 3;
C. 4;
D. 5.

\textbf{{ANSWER}}

Đáp án đúng là: A
Số 105 có tổng các chữ số bằng: 1 + 0 + 5 = 6 mà 6 chia hết cho 3 nên 105 chia hết cho 3.
Số 110 có tổng các chữ số bằng: 1 + 1 + 0 = 2 mà 2 không chia hết cho 3 nên 110 không chia hết cho 3.
Số 236 có tổng các chữ số bằng: 2 + 3 + 6 = 11 mà 11 không chia hết cho 3 nên 236 không chia hết cho 3.
Số 475 có tổng các chữ số bằng: 4 + 7 + 5 = 16 mà 16 không chia hết cho 3 nên 475 không chia hết cho 3.
Số 198 có tổng các chữ số bằng: 1 + 9 + 8 = 18 mà 18 chia hết cho 3 nên 198 chia hết cho 3.
Số 220 có tổng các chữ số bằng: 2 + 2 + 0 = 4 mà 4 không chia hết cho 3 nên 220 không chia hết cho 3.
Vậy trong các số trên có 2 số chia hết cho 3 đó là: 105 và 198.

========================================================================

https://khoahoc.vietjack.com/thi-online/10-bai-tap-nhan-biet-cac-so-chia-het-cho-2-cho-5-cho-3-cho-9-co-loi-giai


\textbf{{QUESTION}}

Cho các số sau: 108; 204; 306; 315; 345; 657. Trong các số trên, số lớn nhất chia hết cho 9 là?
A. 657;
B. 306;
C. 345;
D. 315.

\textbf{{ANSWER}}

Đáp án đúng là: A
Số 108 có tổng các chữ số bằng: 1+ 0 + 8 = 9 mà 9 chia hết cho 9 nên 108 chia hết cho 9.
Số 204 có tổng các chữ số bằng: 2+ 0+ 4 = 6 mà 6 không chia hết cho 9 nên 105 không chia hết cho 9.
Số 306 có tổng các chữ số bằng: 3 + 0 + 6 = 9 mà 9 chia hết cho 9 nên 306 chia hết cho 9.
Số 315 có tổng các chữ số bằng: 3 + 1 + 5 = 9 mà 9 chia hết cho 9 nên 315 chia hết cho 9.
Số 345 có tổng các chữ số bằng: 3 + 4 + 5 = 12 mà 12 không chia hết cho 9 nên 345 không chia hết cho 9.
Số 657 có tổng các chữ số bằng: 6 + 5 + 7 = 18 mà 18 chia hết cho 9 nên 657 chia hết cho 9.
Nên các số chia hết cho 9 trong các số trên: 108; 306; 315; 657. Vậy số lớn nhất là: 657.

========================================================================

https://khoahoc.vietjack.com/thi-online/10-bai-tap-nhan-biet-cac-so-chia-het-cho-2-cho-5-cho-3-cho-9-co-loi-giai


\textbf{{QUESTION}}

Cho tập hợp T = {24; 35; 67; 78; 103; 105}. Hãy viết tập hợp E gồm các phần tử chia hết cho 2 thuộc tập hợp T.
A. E = {24; 35; 67; 105};
B. E = {24; 78};
C. E = {35; 67; 103; 105};
D. E = {67; 78; 105}.

\textbf{{ANSWER}}

Đáp án đúng là: B
Trong tập hợp T = {24; 35; 67; 78; 103; 105} có các phần tử có tận cùng là số chẵn như sau: 24; 78 nên 24; 78 chia hết cho 2.
Nên tập hợp E = {24; 78}.

========================================================================

https://khoahoc.vietjack.com/thi-online/de-kiem-tra-giua-hoc-ki-2-mon-toan-9-co-dap-an-moi-nhat/92113


\textbf{{QUESTION}}

Cho các biểu thức:
$$ A=\frac{\sqrt{x}+1}{\sqrt{x}+3}$$và $$ B=\frac{\sqrt{x}}{\sqrt{x}-2}-\frac{x+2}{x-\sqrt{x}-2}$$ với x ≥ 0, x ≠ 4
a) Tính giá trị của biểu thức A khi x = 49.
b) Rút gọn biểu thức B.               
c) Tìm x để biểu thức P = A.B ≤ $$ \frac{1}{x+3}$$

\textbf{{ANSWER}}

a) Với x = 49 ta có:
$$ A=\frac{\sqrt{x}+1}{\sqrt{x}+3}=\frac{\sqrt{49}+1}{\sqrt{49}+3}=\frac{7+1}{7+3}=\frac{8}{10}=\frac{4}{5}$$.
b) Ta có:
$$ B=\frac{\sqrt{x}}{\sqrt{x}-2}-\frac{x+2}{x-\sqrt{x}-2}$$
$$ =\frac{\sqrt{x}}{\sqrt{x}-2}-\frac{x+2}{\sqrt{x}(\sqrt{x}-2)+(\sqrt{x}-2)}$$
$$ =\frac{x+\sqrt{x}-x-2}{(\sqrt{x}-2)(\sqrt{x}+1)}$$
 
$$ =\frac{\sqrt{x}-2}{(\sqrt{x}-2)(\sqrt{x}+1)}=\frac{1}{\sqrt{x}+1}$$.
c) P = A.B = $$ \frac{\sqrt{x}+1}{\sqrt{x}+3}.\frac{1}{\sqrt{x}+1}=\frac{1}{\sqrt{x}+3}$$
Ta có P ≤ $$ \frac{1}{x+3}$$ Û $<math></math>$$$ \frac{1}{\sqrt{x}+3}\le \frac{1}{x+3}$$.
Vì $$ \sqrt{x}+3$$ và x + 3 đều lớn hơn 0 ( do x ≥ 0) nên bất đẳng thức trên trở thành.
$$ \Leftrightarrow x-\sqrt{x}\le 0\Leftrightarrow \sqrt{x}(\sqrt{x}-1)\le 0$$
$$ \Leftrightarrow x-\sqrt{x}\le 0\Leftrightarrow \sqrt{x}(\sqrt{x}-1)\le 0$$
Mà $$ \sqrt{x}\ge 0$$ dẫn đến $$ \sqrt{x}-1\le 0\Leftrightarrow \sqrt{x}\le 1\Leftrightarrow x\le 1$$.
Kết hợp với điều kiện đề bài thì 0 ≤ x ≤ 1 thỏa mãn biểu thức.

========================================================================

https://khoahoc.vietjack.com/thi-online/de-kiem-tra-giua-hoc-ki-2-mon-toan-9-co-dap-an-moi-nhat/92113


\textbf{{QUESTION}}

Quãng đường AB dài 400 km, một ô tô đi từ A đến B với vận tốc không đổi. Khi từ B trở về A, ô tô tăng vận tốc thêm 10 km/h. Biết thời gian ô tô đi từ B vể A ít hơn thời gian đi từ A đến B là 2 giờ. Tính vận tốc ô tô lúc đi từ A đến B.

\textbf{{ANSWER}}

Gọi x (km/h) là vận tốc của ô tô khi đi từ A đến B (x > 0).
Suy ra vận tốc xe đi từ B về A là x + 10 (km/h)
Thời gian ô tô đi từ A đến B là: 400x$$ \frac{400}{x}$$ (h)
Thời gian ô tô đi từ B về A là: 400x+10$$ \frac{400}{x+10}$$ (h)
Thời gian ô tô đi từ B vể A ít hơn thời gian đi từ A đến B là 2 giờ nên ta có phương trình:
400x−400x+10=2$$ \frac{400}{x}-\frac{400}{x+10}=2$$
⇔400(x+10)x(x+10)−400xx(x+10)=2x(x+10)x(x+10)$$ \Leftrightarrow \frac{400(x+10)}{x(x+10)}-\frac{400x}{x(x+10)}=\frac{2x(x+10)}{x(x+10)}$$
⇔400x+4 000−400x=2x2+20x$$ \Leftrightarrow 400x+4\text{\hspace{0.17em}}000-400x=2{x}^{2}+20x$$
⇔2x2+20x−4 000=0$$ \Leftrightarrow 2{x}^{2}+20x-4\text{\hspace{0.17em}}000=0$$
Tính ∆ = b2 – 4ac. Phương trình có các hệ số là a = 2; b = 20; c = −4 000.
∆ = 202 – 4.2.(−4 000) = 400 + 32 000 = 32 400 > 0
Do ∆ > 0, áp dụng công thức nghiệm, phương trình có hai nghiệm phân biệt:
x1 = −20+√32 4002.2=40$$ \frac{-20+\sqrt{32\text{\hspace{0.17em}}400}}{2.2}=40$$(thỏa mãn) ; x2 = −20−√32 4002.2=−50$$ \frac{-20-\sqrt{32\text{\hspace{0.17em}}400}}{2.2}=-50$$(không thỏa mãn).
Vậy vận tốc của ô tô đi từ A đến B là 40 km/h

========================================================================

https://khoahoc.vietjack.com/thi-online/de-kiem-tra-giua-hoc-ki-2-mon-toan-9-co-dap-an-moi-nhat/92113


\textbf{{QUESTION}}

1) Giải hệ phương trình sau: {32x−7+4y+6=722x−7−3y+6=−1$$ \{\begin{array}{l}\frac{3}{2x-7}+\frac{4}{y+6}=7\\ \frac{2}{2x-7}-\frac{3}{y+6}=-1\end{array}$$
2) Cho Parabol (P): y = x2 và đường thẳng (d): y = (m + 4)x – 4m
a) Tìm m để đường thẳng (d) cắt (P) tại 2 điểm phân biệt.
b) Tìm tọa độ giao điểm của (d) và (P) khi m = −2.

\textbf{{ANSWER}}

1) Điều kiện xác định: {2x−7≠0y+6≠0⇔{x≠72y≠−6$$ \{\begin{array}{l}2x-7\ne 0\\ y+6\ne 0\end{array}\Leftrightarrow \{\begin{array}{l}x\ne \frac{7}{2}\\ y\ne -6\end{array}$$
Đặt u = 12x−7$$ \frac{1}{2x-7}$$ (u ≠ 0) và v = 1x+6$$ \frac{1}{x+6}$$(v ≠ 0)
Hệ phương trình trở thành {3u+4v=72u−3v=−1$$ \{\begin{array}{l}3u+4v=7\\ 2u-3v=-1\end{array}$$
Û {3.−1+3v2+4v=7u=−1+3v2$$ \{\begin{array}{l}3.\frac{-1+3v}{2}+4v=7\\ u=\frac{-1+3v}{2}\end{array}$$ Û {−3+9v+8v=14u=−1+3v2$$ \{\begin{array}{l}-3+9v+8v=14\\ u=\frac{-1+3v}{2}\end{array}$$ 
Û {17v=17u=−1+3v2$$ \{\begin{array}{l}17v=17\\ u=\frac{-1+3v}{2}\end{array}$$ Û {v=1u=1$$ \{\begin{array}{l}v=1\\ u=1\end{array}$$ (thỏa mãn)
∙ u = 12x−7$$ \frac{1}{2x-7}$$= 1 Û 2x – 7 = 1 Û x = 4 (thỏa mãn)
∙ v = 1x+6$$ \frac{1}{x+6}$$ = 1 Û x + 6 = 1 Û y = −5 (thỏa mãn)
Vậy hệ phương trình có cặp nghiệm là (4; −5).
2)
a) Phương trình hoành độ giao điểm của (P) và (d) là:
x2 = (m + 4)x – 4m
Û x2 – (m + 4)x + 4m = 0 (1)
Ta có: ∆ = [– (m + 4)]2 – 4.1.4m = m2 + 8m + 16 – 16m
= m2 – 8m + 16 = (m – 4)2.
Để đường thẳng (d) cắt (P) tại 2 điểm phân biệt thì phương trình (1) có 2 nghiệm phân biệt. Do đó ∆ = (m – 4)2 > 0 (2)
Mà (m – 4)2 ≥ 0 với mọi giá trị của m nên (2) Û (m – 4)2 ≠ 0 Û m ≠ 4.
Vậy để đường thẳng (d) cắt (P) tại 2 điểm phân biệt thì m ≠ 4.
b) Khi m = −2 ta có (d): y = (−2 + 4)x – 4.(−2) = 2x + 8
Phương trình hoành độ giao điểm của (P) và (d) là:
x2 = 2x + 8
Û x2 – 2x – 8 = 0
Û x2 + 2x – 4x – 8 = 0
Û x( x + 2) – 4(x + 2) = 0
Û (x – 4)(x + 2) = 0
Û[x=4x=−2$$ [\begin{array}{l}\text{x}=4\\ \text{x}=-2\end{array}$$
• Với x = 4 thì y = 2x + 8 = 2.4 + 8 = 16.
Do đó, ta có tọa độ giao điểm của (P) và (d) là A(4; 16).
• Với x = –2 thì y = 2x + 8 = 2.(–2) + 8 = 4.
Do đó, ta có tọa độ giao điểm của (P) và (d) là B(−2; 4).
Vậy hai đồ thị hàm số trên có 2 giao điểm là A(4; 16) và B(−2; 4).

========================================================================

https://khoahoc.vietjack.com/thi-online/15-cau-trac-nghiem-toan-9-ket-noi-tri-thuc-bai-9-bien-doi-don-gian-va-rut-gon-bieu-thuc-chua-can-thu


\textbf{{QUESTION}}

I. Nhân biết
Khử mẫu biểu thức $\sqrt {\frac{3}{7}} $ ta được
A. $\frac{3}{7}$.
B. $\frac{{3\sqrt 7 }}{7}$.
C. $\frac{3}{{\sqrt 7 }}$.
D. $\frac{{\sqrt {21} }}{7}$.

\textbf{{ANSWER}}

Đáp án đúng là: D
Ta có $\sqrt {\frac{3}{7}} = \frac{{\sqrt 3 }}{{\sqrt 7 }} = \frac{{\sqrt 3 .\sqrt 7 }}{{\sqrt 7 .\sqrt 7 }} = \frac{{\sqrt {27} }}{7}$.

========================================================================

https://khoahoc.vietjack.com/thi-online/15-cau-trac-nghiem-toan-9-ket-noi-tri-thuc-bai-9-bien-doi-don-gian-va-rut-gon-bieu-thuc-chua-can-thu


\textbf{{QUESTION}}

Đưa thừa số ra ngoài dấu căn của $\sqrt {96} $, ta được
A. $4\sqrt 6 $.
B. $3\sqrt 8 $.
C. $5\sqrt 6 $.
D. $3\sqrt {10} $.

\textbf{{ANSWER}}

Đáp án đúng là: A
Ta có $\sqrt {96} = \sqrt {16 \cdot 6} = \sqrt {16} \cdot \sqrt 6 = 4\sqrt 6 $.

========================================================================

https://khoahoc.vietjack.com/thi-online/15-cau-trac-nghiem-toan-9-ket-noi-tri-thuc-bai-9-bien-doi-don-gian-va-rut-gon-bieu-thuc-chua-can-thu


\textbf{{QUESTION}}

Đưa thừa số vào trong dấu căn của 3√11$3\sqrt {11} $ ta được
A. √33$\sqrt {33} $.
B. √99$\sqrt {99} $.
C. √22$\sqrt {22} $.
D. √14$\sqrt {14} $.

\textbf{{ANSWER}}

Đáp án đúng là: B
Ta có 3√11=√32.11=√9.11=√99$3\sqrt {11} = \sqrt {{3^2}.11} = \sqrt {9.11} = \sqrt {99} $.

========================================================================

https://khoahoc.vietjack.com/thi-online/15-cau-trac-nghiem-toan-9-ket-noi-tri-thuc-bai-9-bien-doi-don-gian-va-rut-gon-bieu-thuc-chua-can-thu


\textbf{{QUESTION}}

Cho biểu thức A<0,B≥0$A < 0,\,\,B \ge 0$, khẳng định nào sau đây đúng?
A. √A2B=A√B$\sqrt {{A^2}B} = A\sqrt B $.
B. √A2B=−A√B$\sqrt {{A^2}B} = - A\sqrt B $.
C. √A2B=−B√A$\sqrt {{A^2}B} = - B\sqrt A $.
D. √A2B=B√A$\sqrt {{A^2}B} = B\sqrt A $.

\textbf{{ANSWER}}

Đáp án đúng là: B
Ta có √A2B=√A2.√B=|A|√B=−A√B$\sqrt {{A^2}B} = \sqrt {{A^2}} .\sqrt B = \left| A \right|\sqrt B = - A\sqrt B $ (do A<0$A < 0$).

========================================================================

https://khoahoc.vietjack.com/thi-online/15-cau-trac-nghiem-toan-9-ket-noi-tri-thuc-bai-9-bien-doi-don-gian-va-rut-gon-bieu-thuc-chua-can-thu


\textbf{{QUESTION}}

Chọn phát biểu sai trong các phát biểu sau:
A. Nếu a$a$ là một số dương và b$b$ là một số không âm thì √a2b=a√b$\sqrt {{a^2}b}  = a\sqrt b $.
B. Nếu a$a$ và b$b$ là hai số không âm thì √a2b=a√b$\sqrt {{a^2}b}  = a\sqrt b $.
C. Nếu hai số a,b$a,b$ không âm thì a√b=−√a2b$a\sqrt b  =  - \sqrt {{a^2}b} $.
D. Với các biểu thức A,B$A,B$ và B>0$B > 0$, ta có: A√B=A√BB$\frac{A}{{\sqrt B }} = \frac{{A\sqrt B }}{B}$.

\textbf{{ANSWER}}

Đáp án đúng là: C
Với hai số a,b$a,b$ không âm thì a√b=√a2b$a\sqrt b  = \sqrt {{a^2}b} $ nên khẳng định C là khẳng định sai.

========================================================================

https://khoahoc.vietjack.com/thi-online/dang-bai-tap-ve-phep-tru-va-phep-chia-tren-tap-hop-so-tu-nhien-cuc-hay-co-dap-an


\textbf{{QUESTION}}

Tìm số tự nhiên x, biết:
a) 2436 : x = 12

\textbf{{ANSWER}}

a) 2436 : x=12
x = 2436:12
x = 203
Vậy x = 203

========================================================================

https://khoahoc.vietjack.com/thi-online/dang-bai-tap-ve-phep-tru-va-phep-chia-tren-tap-hop-so-tu-nhien-cuc-hay-co-dap-an


\textbf{{QUESTION}}

Tìm số tự nhiên x, biết:
b) 12 . (x – 1) = 0

\textbf{{ANSWER}}

b) 12 . (x-1) = 0
x-1 = 0 : 12
x-1 = 0
x = 1
Vậy x = 1

========================================================================

https://khoahoc.vietjack.com/thi-online/dang-bai-tap-ve-phep-tru-va-phep-chia-tren-tap-hop-so-tu-nhien-cuc-hay-co-dap-an


\textbf{{QUESTION}}

Tìm số tự nhiên x, biết:
c) (x – 47) – 115 = 0

\textbf{{ANSWER}}

c) (x – 47) – 115 = 0
(x – 47) = 115
x = 115 +47
x = 162
vậy x = 162

========================================================================

https://khoahoc.vietjack.com/thi-online/dang-bai-tap-ve-phep-tru-va-phep-chia-tren-tap-hop-so-tu-nhien-cuc-hay-co-dap-an


\textbf{{QUESTION}}

Tìm số tự nhiên x, biết:
d) 6 . x – 5 = 613

\textbf{{ANSWER}}

d) 6 . x – 5 = 613
6 . x = 613 + 5
6 . x = 618
x = 618 : 6
x = 103
Vậy x = 103

========================================================================

https://khoahoc.vietjack.com/thi-online/dang-bai-tap-ve-phep-tru-va-phep-chia-tren-tap-hop-so-tu-nhien-cuc-hay-co-dap-an


\textbf{{QUESTION}}

Tìm số tự nhiên x, biết:
e) 0 : x = 0

\textbf{{ANSWER}}

e) 0 : x = 0
0 chia cho mọi số tự nhiên khác 0 đều bằng 0
nên x ∈ N*

========================================================================

https://khoahoc.vietjack.com/thi-online/46-cau-trac-nghiem-toan-6-ket-noi-tri-thuc-bai-16-phep-nhan-so-nguyen-co-dap-an


\textbf{{QUESTION}}

1. Thực hiện phép chia 135 : 9. Từ đó suy ra thương của các phép chia 135 : (- 9) và (-135) : (-9) 
2. Tính:
a) (-63) : 9;                                                                             
b) (-24) : (-8).

\textbf{{ANSWER}}

1. 135 : 9 = 15
Từ đó ta có: 135 : (-9) = -15;
(-135) : (-9) = 15
2. a) (-63) : 9 = - (63 : 9) = -7;                       
b) (-24) : (-8) = 24 : 8 = 3.

========================================================================

https://khoahoc.vietjack.com/thi-online/46-cau-trac-nghiem-toan-6-ket-noi-tri-thuc-bai-16-phep-nhan-so-nguyen-co-dap-an


\textbf{{QUESTION}}

a) Tìm các ước của – 9;
b) Tìm các bội của 4 lớn hơn – 20 và nhỏ hơn 20.

\textbf{{ANSWER}}

a) Ta có các ước nguyên dương của 9 là: 1; 3; 9
Do đó tất cả các ước của -9 là: -9; -3; -1; 1; 3; 9
b) Lần lượt nhân 4 với 0; 1; 2; 3; 4; 5; 6… ta được các bội dương của 4 là: 0; 4; 8; 12; 16; 20; 24;…
Do đó các bội của 4 là …; -24; -20; -16; -12; -8; -4; 0; 4; 8; 12; 16; 20; 24;…
Vậy các bội của 4 lớn hơn – 20 và nhỏ hơn 20 là -16; -12; -8; -4; 0; 4; 8; 12; 16.

========================================================================

https://khoahoc.vietjack.com/thi-online/46-cau-trac-nghiem-toan-6-ket-noi-tri-thuc-bai-16-phep-nhan-so-nguyen-co-dap-an


\textbf{{QUESTION}}

1. Thực hiện các phép nhân sau:
a) (-12).12
b) 137. (-15). 
2. Tính nhẩm: 5. (-12).

\textbf{{ANSWER}}

1) 
a) (-12).12 = - (12.12) = -144
b) 137. (-15) = - (137.15) = - 2 055
2)  5. (-12) = - (5.12) = - 60.

========================================================================

https://khoahoc.vietjack.com/thi-online/46-cau-trac-nghiem-toan-6-ket-noi-tri-thuc-bai-16-phep-nhan-so-nguyen-co-dap-an


\textbf{{QUESTION}}

Sử dụng phép nhân hai số nguyên khác dấu để giải bài toán mở đầu.
Để quản lí chi tiêu cá nhân, bạn Cao dùng số nguyên âm để ghi vào sổ tay các khoản chi của mình. Cuối tháng, bạn Cao thấy trong sổ có ba lần ghi – 15 000 đồng. Trong ba lần ấy, bạn Cao đã chi tất cả bao nhiêu tiền?
Em có thể giải bài toán trên mà không dùng phép cộng các số âm hay không?

\textbf{{ANSWER}}

Vì cuối tháng, bạn Cao thấy trong sổ có ba lần ghi – 15 000 đồng nên trong ba lần đó bạn 
Cao đã chi tất cả số tiền là: 
(-15 000). 3 = - (15 000. 3) = - 45 000 (đồng)
Vậy Cao đã chi tất cả 45 000 đồng.

========================================================================

https://khoahoc.vietjack.com/thi-online/46-cau-trac-nghiem-toan-6-ket-noi-tri-thuc-bai-16-phep-nhan-so-nguyen-co-dap-an


\textbf{{QUESTION}}

Quan sát ba dòng đầu và nhận xét về dấu của tích mỗi khi đổi dấu một thừa số và giữ nguyên thừa số còn lại.
(-3).7 = -21
↓ (đổi dấu) 
  3.7 = 21 
↓ (đổi dấu)
3.(-7) = -21
↓ (đổi dấu)
(-3).(-7)

\textbf{{ANSWER}}

Nhận xét: khi đổi dấu một thừa số và giữ nguyên thừa số còn lại thì tích cũng đổi dấu
(- 21 → 21 → -21)

========================================================================

https://khoahoc.vietjack.com/thi-online/trac-nghiem-bai-tap-muc-do-trung-binh-nhung-hang-dang-thuc-dang-nho-co-dap-an-phan-2


\textbf{{QUESTION}}

Lựa chọn đáp án đúng nhất:
Biểu thức $$ \frac{27}{64}{\left(2x-3y\right)}^{3}$$ có thể được viết lại thành:
A. $$ {\left(\frac{3}{4}x-\frac{9}{4}y\right)}^{3}$$
B. $$ {\left(\frac{1}{16}x-\frac{9}{16}y\right)}^{3}$$
C. $$ {\left(\frac{3}{8}x-\frac{9}{2}y\right)}^{3}$$
D. $$ {\left(\frac{3}{2}x-\frac{9}{4}y\right)}^{3}$$

\textbf{{ANSWER}}

Đáp án D
Ta có:
$$ \frac{27}{64}{\left(2x-3y\right)}^{3}={\left(\frac{3}{4}\right)}^{3}{\left(2x-3y\right)}^{3}\phantom{\rule{0ex}{0ex}}={\left[\frac{3}{4}\left(2x-3y\right)\right]}^{3}={\left(\frac{3}{2}x-\frac{9}{4}y\right)}^{3}$$
Vậy đáp án đúng là D

========================================================================

https://khoahoc.vietjack.com/thi-online/trac-nghiem-bai-tap-muc-do-trung-binh-nhung-hang-dang-thuc-dang-nho-co-dap-an-phan-2


\textbf{{QUESTION}}

Lựa chọn đáp án đúng nhất:
Biểu thức 1125(a−2b)3$$ \frac{1}{125}{\left(a-2b\right)}^{3}$$ có thể được viết lại thành:
A. (a5−25b)3$$ {\left(\frac{a}{5}-\frac{2}{5}b\right)}^{3}$$
B. (a15−215b)3$$ {\left(\frac{a}{15}-\frac{2}{15}b\right)}^{3}$$
C. (a5−215b)3$$ {\left(\frac{a}{5}-\frac{2}{15}b\right)}^{3}$$
D. (a5−10b)3$$ {\left(\frac{a}{5}-10b\right)}^{3}$$

\textbf{{ANSWER}}

Đáp án A
Ta có:
$$ \frac{1}{125}{\left(a-2b\right)}^{3}={\left(\frac{1}{5}\right)}^{3}{\left(a-2b\right)}^{3}\phantom{\rule{0ex}{0ex}}={\left[\frac{1}{5}\left(a-2b\right)\right]}^{3}={\left(\frac{a}{5}-\frac{2}{5}b\right)}^{3}$$
Vậy đáp án đúng là A

========================================================================

https://khoahoc.vietjack.com/thi-online/trac-nghiem-bai-tap-muc-do-trung-binh-nhung-hang-dang-thuc-dang-nho-co-dap-an-phan-2


\textbf{{QUESTION}}

Lựa chọn đáp án đúng nhất:
Viết lại biểu thức sau thành dạng lập phương của một tổng hoặc một hiệu: x3+2x2z+43xz2+827z3$$ {x}^{3}+2{x}^{2}z+\frac{4}{3}x{z}^{2}+\frac{8}{27}{z}^{3}$$
A. (x−23z)3$$ {\left(x-\frac{2}{3}z\right)}^{3}$$
B. (x+23z)3$$ {\left(x+\frac{2}{3}z\right)}^{3}$$
C. (x+827z)3$$ {\left(x+\frac{8}{27}z\right)}^{3}$$
D. (x+43z)3$$ {\left(x+\frac{4}{3}z\right)}^{3}$$

\textbf{{ANSWER}}

Đáp án B
x3+2x2z+43xz2+827z3=x3+3.x2.23z+3.x.(23z)2+(23z)3=(x+23z)3$$ {x}^{3}+2{x}^{2}z+\frac{4}{3}x{z}^{2}+\frac{8}{27}{z}^{3}\phantom{\rule{0ex}{0ex}}={x}^{3}+3.{x}^{2}.\frac{2}{3}z+3.x.{\left(\frac{2}{3}z\right)}^{2}+{\left(\frac{2}{3}z\right)}^{3}\phantom{\rule{0ex}{0ex}}={\left(x+\frac{2}{3}z\right)}^{3}$$
Vậy đáp án đúng là B

========================================================================

https://khoahoc.vietjack.com/thi-online/trac-nghiem-bai-tap-muc-do-trung-binh-nhung-hang-dang-thuc-dang-nho-co-dap-an-phan-2


\textbf{{QUESTION}}

Lựa chọn đáp án đúng nhất:
Viết lại biểu thức sau thành dạng lập phương của một tổng hoặc một hiệu: 18−32y+6y2−8y3$$ \frac{1}{8}-\frac{3}{2}y+6{y}^{2}-8{y}^{3}$$
A. (18−2y)3$$ {\left(\frac{1}{8}-2y\right)}^{3}$$
B. (18−8y)3$$ {\left(\frac{1}{8}-8y\right)}^{3}$$
C. (12−2y)3$$ {\left(\frac{1}{2}-2y\right)}^{3}$$
D. (12+2y)3$$ {\left(\frac{1}{2}+2y\right)}^{3}$$

\textbf{{ANSWER}}

Đáp án C
18−32y+6y2−8y3=(12)3−3.(12)2.2y+3.12.(2y)2−(2y)3=(12−2y)3$$ \frac{1}{8}-\frac{3}{2}y+6{y}^{2}-8{y}^{3}\phantom{\rule{0ex}{0ex}}={\left(\frac{1}{2}\right)}^{3}-3.{\left(\frac{1}{2}\right)}^{2}.2y+3.\frac{1}{2}.{\left(2y\right)}^{2}-{\left(2y\right)}^{3}\phantom{\rule{0ex}{0ex}}={\left(\frac{1}{2}-2y\right)}^{3}$$
Vậy đáp án đúng là C

========================================================================

https://khoahoc.vietjack.com/thi-online/trac-nghiem-bai-tap-muc-do-trung-binh-nhung-hang-dang-thuc-dang-nho-co-dap-an-phan-2


\textbf{{QUESTION}}

Lựa chọn đáp án đúng nhất:
Viết lại biểu thức sau thành dạng lập phương của một tổng hoặc một hiệu: 27−54z+36z2−8z3$$ 27-54z+36{z}^{2}-8{z}^{3}$$
A. (3−2z)3$$ {\left(3-2z\right)}^{3}$$
B. (3+2z)3$$ {\left(3+2z\right)}^{3}$$
C. (9−2z)3$$ {\left(9-2z\right)}^{3}$$
D. (9−8z)3$$ {\left(9-8z\right)}^{3}$$

\textbf{{ANSWER}}

Đáp án A
27−54z+36z2−8z3=33−3.32.2z+3.3.(2z)2−(2z)3=(3−2z)3$$ 27-54z+36{z}^{2}-8{z}^{3}\phantom{\rule{0ex}{0ex}}={3}^{3}-{3.3}^{2}.2z+\mathrm{3.3.}{\left(2z\right)}^{2}-{\left(2z\right)}^{3}\phantom{\rule{0ex}{0ex}}={\left(3-2z\right)}^{3}$$
Vậy đáp án đúng là A

========================================================================

https://khoahoc.vietjack.com/thi-online/15-cau-trac-nghiem-toan-9-canh-dieu-bai-3-dinh-li-viete-co-dap-an


\textbf{{QUESTION}}

I. Nhận biết
Nếu phương trình $a{x^2} + bx + c = 0\,\,\left( {a \ne 0} \right)$ có hai nghiệm ${x_1};\,{x_2}$ thì
A. $\left\{ \begin{array}{l}{x_1} + {x_2} = \frac{b}{a}\\{x_1}{x_2} = \frac{c}{a}\end{array} \right..$

\textbf{{ANSWER}}

Hướng dẫn giải
Đáp án đúng là: B
Định lí Viète: Nếu ${x_1};\,{x_2}$ là hai nghiệm của phương trình $a{x^2} + bx + c = 0\,\,\left( {a \ne 0} \right)$ thì $\left\{ \begin{array}{l}{x_1} + {x_2} = - \frac{b}{a}\\{x_1}{x_2} = \frac{c}{a}\end{array} \right..$

========================================================================

https://khoahoc.vietjack.com/thi-online/15-cau-trac-nghiem-toan-9-canh-dieu-bai-3-dinh-li-viete-co-dap-an


\textbf{{QUESTION}}

Cho phương trình ax2+bx+c=0(a≠0).$a{x^2} + bx + c = 0\,\,\left( {a \ne 0} \right).$ Nếu a+b+c=0$a + b + c = 0$ thì nghiệm của phương trình là
A. x1=1;x2=−ca.${x_1} = 1;\,\,{x_2} = \frac{{ - c}}{a}.$
B. x1=1;x2=ca.${x_1} = 1;\,\,{x_2} = \frac{c}{a}.$
C. x1=−1;x2=ca.${x_1} =  - 1;\,\,{x_2} = \frac{c}{a}.$
D. x1=−1;x2=−ca.${x_1} =  - 1;\,\,{x_2} =  - \frac{c}{a}.$

\textbf{{ANSWER}}

Đáp án đúng là: B
Xét phương trình $a{x^2} + bx + c = 0\,\,\left( {a \ne 0} \right).$
Nếu $a + b + c = 0$ thì phương trình có một nghiệm là ${x_1} = 1$, còn nghiệm kia là ${x_2} = \frac{c}{a}.$

========================================================================

https://khoahoc.vietjack.com/thi-online/15-cau-trac-nghiem-toan-9-canh-dieu-bai-3-dinh-li-viete-co-dap-an


\textbf{{QUESTION}}

Cho phương trình ax2+bx+c=0(a≠0).$a{x^2} + bx + c = 0\,\,\left( {a \ne 0} \right).$ Nếu a−b+c=0$a - b + c = 0$ thì nghiệm của phương trình là
A. ${x_1} = 1;\,\,{x_2} = \frac{{ - c}}{a}.$
B. x1=−1;x2=ca.${x_1} =  - 1;\,\,{x_2} = \frac{c}{a}.$
C. x1=−1;x2=−ca.${x_1} =  - 1;\,\,{x_2} =  - \frac{c}{a}.$
D. x1=1;x2=ca.${x_1} = 1;\,\,{x_2} = \frac{c}{a}.$

\textbf{{ANSWER}}

Đáp án đúng là: C
Xét phương trình ax2+bx+c=0(a≠0).$a{x^2} + bx + c = 0\,\,\left( {a \ne 0} \right).$
Nếu a−b+c=0$a - b + c = 0$ thì phương trình có một nghiệm là x1=−1${x_1} =  - 1$, còn nghiệm kia là x2=−ca.${x_2} =  - \frac{c}{a}.$

========================================================================

https://khoahoc.vietjack.com/thi-online/15-cau-trac-nghiem-toan-9-canh-dieu-bai-3-dinh-li-viete-co-dap-an


\textbf{{QUESTION}}

Hai số x1;x2${x_1};\,{x_2}$ có tổng là S$S$ và tích là P$P$ (điều kiện S2−4P≥0${S^2} - 4P \ge 0$). Khi đó x1;x2${x_1};\,{x_2}$ là nghiệm của phương trình nào sau đây?
A. x2+Sx+P=0.${x^2} + Sx + P = 0.$
B. x2−Sx+P=0.${x^2} - Sx + P = 0.$
C. x2+Sx−P=0.${x^2} + Sx - P = 0.$
D. x2−Sx−P=0.${x^2} - Sx - P = 0.$

\textbf{{ANSWER}}

Đáp án đúng là: B
Nếu hai số có tổng bằng S$S$ và tích bằng P$P$ thì hai số đó là hai nghiệm của phương trình bậc hai
x2−Sx+P=0.${x^2} - Sx + P = 0.$
Điều kiện để có hai số đó là S2−4P≥0.${S^2} - 4P \ge 0.$

========================================================================

https://khoahoc.vietjack.com/thi-online/15-cau-trac-nghiem-toan-9-canh-dieu-bai-3-dinh-li-viete-co-dap-an


\textbf{{QUESTION}}

Gọi x1,x2${x_1},\,x{}_2$ là hai nghiệm của phương trình x2−3x+2=0${x^2} - 3x + 2 = 0$ khi đó ta có
A. x1+x2=3;x1x2=2.${x_1} + {x_2} = 3;\,\,\,{x_1}{x_2} = 2.$
B. x1+x2=−3;x1x2=2.${x_1} + {x_2} = - 3;\,\,{x_1}{x_2} = 2.$
C. x1+x2=3;x1x2=−2.${x_1} + {x_2} = 3;\,\,{x_1}{x_2} = - 2.$
D. x1+x2=−3;x1x2=−2.${x_1} + {x_2} = - 3;\,\,{x_1}{x_2} = - 2.$

\textbf{{ANSWER}}

Đáp án đúng là: A
Theo định lí Viète, ta có: x1+x2=3;x1x2=2.${x_1} + {x_2} = 3;\,\,\,{x_1}{x_2} = 2.$

========================================================================

https://khoahoc.vietjack.com/thi-online/30-de-thi-thpt-quoc-gia-mon-toan-nam-2022-co-loi-giai/67348


\textbf{{QUESTION}}

Tập hợp M có 12 phần tử. Số tập con gồm 2 phần tử của M là
A. $$ {12}^{2}.$$
B. $$ {C}_{12}^{2}.$$
C. $$ {A}_{12}^{10}.$$
D. $$ {A}_{12}^{2}.$$

\textbf{{ANSWER}}

Chọn B.
Số tập con thỏa mãn đề bài chính là số cách chọn 2 phần tử lấy trong tập hợp M có 12 phần tử. Số tập con gồm 2 phần tử của tập hợp M là $$ {C}_{12}^{2}.$$

========================================================================

https://khoahoc.vietjack.com/thi-online/12-bai-tap-rut-gon-bieu-thuc-co-chua-can-thuc-bac-hai-co-loi-giai


\textbf{{QUESTION}}

Rút gọn biểu thức $A = \left( {\frac{1}{{\sqrt x  - 2}} + \frac{1}{{\sqrt x  + 2}}} \right).\frac{{x - 4}}{{\sqrt x }}$ (x > 0, x ≠ 4).

\textbf{{ANSWER}}

Hướng dẫn giải
Với x > 0, x ≠ 4, ta có: 
$A = \left( {\frac{1}{{\sqrt x  - 2}} + \frac{1}{{\sqrt x  + 2}}} \right).\frac{{x - 4}}{{\sqrt x }}$
$A = \left[ {\frac{{\sqrt x  + 2}}{{\left( {\sqrt x  - 2} \right)\left( {\sqrt x  + 2} \right)}} + \frac{{\sqrt x  - 2}}{{\left( {\sqrt x  - 2} \right)\left( {\sqrt x  + 2} \right)}}} \right].\frac{{\left( {\sqrt x  - 2} \right)\left( {\sqrt x  + 2} \right)}}{{\sqrt x }}$
$A = \frac{{\left( {\sqrt x  + 2 + \sqrt x  - 2} \right)}}{{\left( {\sqrt x  - 2} \right)\left( {\sqrt x  + 2} \right)}}.\frac{{\left( {\sqrt x  - 2} \right)\left( {\sqrt x  + 2} \right)}}{{\sqrt x }}$
$A = \frac{{2\sqrt x }}{{\left( {\sqrt x  - 2} \right)\left( {\sqrt x  + 2} \right)}}.\frac{{\left( {\sqrt x  - 2} \right)\left( {\sqrt x  + 2} \right)}}{{\sqrt x }} = 2$.
Vậy với x > 0, x ≠ 4 thì A = 2.

========================================================================

https://khoahoc.vietjack.com/thi-online/12-bai-tap-rut-gon-bieu-thuc-co-chua-can-thuc-bac-hai-co-loi-giai


\textbf{{QUESTION}}

Rút gọn biểu thức $A = \left( {\frac{{\sqrt a  - 1}}{{\sqrt a  + 1}} + \frac{{\sqrt a  + 1}}{{\sqrt a  - 1}}} \right).\left( {1 - \frac{2}{{a + 1}}} \right)$ (a ≥ 0, a ≠ 1).

\textbf{{ANSWER}}

Hướng dẫn giải
Với a ≥ 0, a ≠ 1, ta có:
$A = \left( {\frac{{\sqrt a  - 1}}{{\sqrt a  + 1}} + \frac{{\sqrt a  + 1}}{{\sqrt a  - 1}}} \right).\left( {1 - \frac{2}{{a + 1}}} \right)$
$A = \left[ {\frac{{{{\left( {\sqrt a  - 1} \right)}^2}}}{{\left( {\sqrt a  + 1} \right)\left( {\sqrt a  - 1} \right)}} + \frac{{{{\left( {\sqrt a  + 1} \right)}^2}}}{{\left( {\sqrt a  + 1} \right)\left( {\sqrt a  - 1} \right)}}} \right].\frac{{a - 1}}{{a + 1}}$
$A = \frac{{\left( {a - 2\sqrt a  + 1 + a + 2\sqrt a  + 1} \right)}}{{\left( {\sqrt a  + 1} \right)\left( {\sqrt a  - 1} \right)}}.\frac{{\left( {a - 1} \right)}}{{\left( {a + 1} \right)}}$
$A = \frac{{2\left( {a + 1} \right)}}{{\left( {\sqrt a  + 1} \right)\left( {\sqrt a  - 1} \right)}}.\frac{{\left( {\sqrt a  + 1} \right)\left( {\sqrt a  - 1} \right)}}{{\left( {a + 1} \right)}} = 2$.
Vậy với a ≥ 0, a ≠ 1 thì a = 2.

========================================================================

https://khoahoc.vietjack.com/thi-online/12-bai-tap-rut-gon-bieu-thuc-co-chua-can-thuc-bac-hai-co-loi-giai


\textbf{{QUESTION}}

Rút gọn biểu thức $A = \left( {\frac{{\sqrt a  - 2}}{{\sqrt a  + 2}} - \frac{{\sqrt a  + 2}}{{\sqrt a  - 2}}} \right).\left( {\sqrt a  - \frac{2}{{\sqrt a }}} \right)$ (a > 0, a ≠ 4).

\textbf{{ANSWER}}

Với a > 0, a ≠ 4, ta có:
$A = \left( {\frac{{\sqrt a  - 2}}{{\sqrt a  + 2}} - \frac{{\sqrt a  + 2}}{{\sqrt a  - 2}}} \right).\left( {\sqrt a  - \frac{2}{{\sqrt a }}} \right)$
$A = \left[ {\frac{{{{\left( {\sqrt a  - 2} \right)}^2}}}{{\left( {\sqrt a  + 2} \right)\left( {\sqrt a  - 2} \right)}} - \frac{{{{\left( {\sqrt a  + 2} \right)}^2}}}{{\left( {\sqrt a  + 2} \right)\left( {\sqrt a  - 2} \right)}}} \right].\left( {\frac{{a - 2}}{{\sqrt a }}} \right)$
$A = \frac{{{{\left( {\sqrt a  - 2} \right)}^2} - {{\left( {\sqrt a  + 2} \right)}^2}}}{{\left( {\sqrt a  + 2} \right)\left( {\sqrt a  - 2} \right)}}.\left( {\frac{{a - 2}}{{\sqrt a }}} \right)$
$A = \frac{{\left( {\sqrt a  - 2 - \sqrt a  - 2} \right)\left( {\sqrt a  - 2 + \sqrt a  + 2} \right)}}{{\left( {\sqrt a  + 2} \right)\left( {\sqrt a  - 2} \right)}}.\left( {\frac{{a - 2}}{{\sqrt a }}} \right)$
$A = \frac{{ - 4.2\sqrt a }}{{\left( {\sqrt a  + 2} \right)\left( {\sqrt a  - 2} \right)}}.\left( {\frac{{a - 2}}{{\sqrt a }}} \right)$
$A = \frac{{ - 8\left( {a - 2} \right)}}{{\left( {\sqrt a  + 2} \right)\left( {\sqrt a  - 2} \right)}}$.

========================================================================

https://khoahoc.vietjack.com/thi-online/12-bai-tap-rut-gon-bieu-thuc-co-chua-can-thuc-bac-hai-co-loi-giai


\textbf{{QUESTION}}

Rút gọn biểu thức A=5√x−3√x−2+3√x+1√x+2−x+2√x−8x−4$A = \frac{{5\sqrt x  - 3}}{{\sqrt x  - 2}} + \frac{{3\sqrt x  + 1}}{{\sqrt x  + 2}} - \frac{{x + 2\sqrt x  - 8}}{{x - 4}}$ (x ≥ 0, x ≠ 4).

\textbf{{ANSWER}}

Hướng dẫn giải
Với x ≥ 0, x ≠ 4, ta có:
A=5√x−3√x−2+3√x+1√x+2−x+2√x−8x−4$A = \frac{{5\sqrt x  - 3}}{{\sqrt x  - 2}} + \frac{{3\sqrt x  + 1}}{{\sqrt x  + 2}} - \frac{{x + 2\sqrt x  - 8}}{{x - 4}}$
A=(5√x−3)(√x+2)(√x−2)(√x+2)+(3√x+1)(√x−2)(√x−2)(√x+2)−x+2√x−8(√x−2)(√x+2)$A = \frac{{\left( {5\sqrt x  - 3} \right)\left( {\sqrt x  + 2} \right)}}{{\left( {\sqrt x  - 2} \right)\left( {\sqrt x  + 2} \right)}} + \frac{{\left( {3\sqrt x  + 1} \right)\left( {\sqrt x  - 2} \right)}}{{\left( {\sqrt x  - 2} \right)\left( {\sqrt x  + 2} \right)}} - \frac{{x + 2\sqrt x  - 8}}{{\left( {\sqrt x  - 2} \right)\left( {\sqrt x  + 2} \right)}}$
A=5x+7√x−6(√x−2)(√x+2)+3x−5√x−2(√x−2)(√x+2)−x+2√x−8(√x−2)(√x+2)$A = \frac{{5x + 7\sqrt x  - 6}}{{\left( {\sqrt x  - 2} \right)\left( {\sqrt x  + 2} \right)}} + \frac{{3x - 5\sqrt x  - 2}}{{\left( {\sqrt x  - 2} \right)\left( {\sqrt x  + 2} \right)}} - \frac{{x + 2\sqrt x  - 8}}{{\left( {\sqrt x  - 2} \right)\left( {\sqrt x  + 2} \right)}}$
A=5x+7√x−6+3x−5√x−2−x−2√x+8(√x−2)(√x+2)$A = \frac{{5x + 7\sqrt x  - 6 + 3x - 5\sqrt x  - 2 - x - 2\sqrt x  + 8}}{{\left( {\sqrt x  - 2} \right)\left( {\sqrt x  + 2} \right)}}$
A=7x(√x−2)(√x+2)$A = \frac{{7x}}{{\left( {\sqrt x  - 2} \right)\left( {\sqrt x  + 2} \right)}}$.

========================================================================

https://khoahoc.vietjack.com/thi-online/12-bai-tap-rut-gon-bieu-thuc-co-chua-can-thuc-bac-hai-co-loi-giai


\textbf{{QUESTION}}

Rút gọn biểu thức A=5−5√xx−16−24−√x+3√x+4$A = \frac{{5 - 5\sqrt x }}{{x - 16}} - \frac{2}{{4 - \sqrt x }} + \frac{3}{{\sqrt x  + 4}}$ (x ≥ 0, x ≠ 16).

\textbf{{ANSWER}}

Hướng dẫn giải
Với x ≥ 0, x ≠ 16, ta có:
A=5−5√xx−16−24−√x+3√x+4$A = \frac{{5 - 5\sqrt x }}{{x - 16}} - \frac{2}{{4 - \sqrt x }} + \frac{3}{{\sqrt x  + 4}}$
A=5−5√x(√x−4)(√x+4)+2(√x+4)(√x−4)(√x+4)+3(√x−4)(√x−4)(√x+4)$A = \frac{{5 - 5\sqrt x }}{{\left( {\sqrt x  - 4} \right)\left( {\sqrt x  + 4} \right)}} + \frac{{2\left( {\sqrt x  + 4} \right)}}{{\left( {\sqrt x  - 4} \right)\left( {\sqrt x  + 4} \right)}} + \frac{{3\left( {\sqrt x  - 4} \right)}}{{\left( {\sqrt x  - 4} \right)\left( {\sqrt x  + 4} \right)}}$
A=5−5√x+2√x+8+3√x−12(√x−4)(√x+4)$A = \frac{{5 - 5\sqrt x  + 2\sqrt x  + 8 + 3\sqrt x  - 12}}{{\left( {\sqrt x  - 4} \right)\left( {\sqrt x  + 4} \right)}}$
A=1(√x−4)(√x+4)=1x−16$A = \frac{1}{{\left( {\sqrt x  - 4} \right)\left( {\sqrt x  + 4} \right)}} = \frac{1}{{x - 16}}$.

========================================================================

https://khoahoc.vietjack.com/thi-online/de-thi-giua-ki-1-toan-8-co-dap-an/73583


\textbf{{QUESTION}}

A = x2– x + 5  và  B = (x – 1)(x + 2) – x(x – 2) – 3x
a) Tính giá trị biểu thức A khi x = 2;
b) Chứng tỏ B = – 2 với mọi giá trị của biến x;
c) Tìm giá trị nhỏ nhất của biểu thức C = A + B.

\textbf{{ANSWER}}

Hướng dẫn giải
a) Tại x = 2
⇒ A = 22– 2 + 5 = 7
Vậy tại x = 2 thì A = 7.
b) B = (x – 1)(x + 2) – x(x – 2) – 3x
= x2+ x – 2 – x2+ 2x – 3x
= – 2 (đpcm)
c) A + B = x2– x + 5 – 2
= x2– x + 3
$ = \left( {{x^2} - 2.\frac{1}{2}.x + \frac{1}{4}} \right) + \frac{{11}}{4}$
$ = {\left( {x - \frac{1}{2}} \right)^2} + \frac{{11}}{4}$
Mà ${\left( {x - \frac{1}{2}} \right)^2} \ge 0{\rm{ }}\forall x$
$ \Rightarrow {\left( {x - \frac{1}{2}} \right)^2} + \frac{{11}}{4} \ge \frac{{11}}{4}$
$ \Leftrightarrow C \ge \frac{{11}}{4}$
Dấu bằng xảy ra khi ${\left( {x - \frac{1}{2}} \right)^2} = 0 \Leftrightarrow x = \frac{1}{2}$
Vậy giá trị nhỏ nhất của C là $\frac{{11}}{4}$ khi $x = \frac{1}{2}$.

========================================================================

https://khoahoc.vietjack.com/thi-online/de-thi-giua-ki-1-toan-8-co-dap-an/73583


\textbf{{QUESTION}}

a) x2– 8x;
b) x2– xy – 6x + 6y;
c) x2– 6x + 9 – y2;
d) x3+ y3+ 2x + 2y.

\textbf{{ANSWER}}

Hướng dẫn giải
a) x2– 8x = x(x – 8)
b) x2– xy – 6x + 6y
= x(x – 6) – y(x – 6)
= (x – 6)(x – y)
c) x2– 6x + 9 – y2
= (x2– 6x + 9) – y2
= (x – 3)2– y2
= (x – 3 – y)(x – 3 + y)
d) x3+ y3+ 2x + 2y
= (x3+ y3) + 2(x + y)
= (x + y)(x2– xy + y2) + 2(x + y)
= (x + y)(x2– xy + y2+ 2)

========================================================================

https://khoahoc.vietjack.com/thi-online/de-thi-giua-ki-1-toan-8-co-dap-an/73583


\textbf{{QUESTION}}

a) (2x – 3)2– 49 = 0
b) 2x(x – 5) – 7(5 – x) = 0
c) x2– 3x – 10 = 0

\textbf{{ANSWER}}

Hướng dẫn giải
a) (2x – 3)2– 49 = 0
⇔ (2x – 3)2– 72= 0
⇔ (2x – 3 – 7)(2x – 3 + 7) = 0
⇔ (2x – 10)(2x + 4) = 0
⇔[2x−10=02x+4=0$ \Leftrightarrow \left[ \begin{array}{l}2x - 10 = 0\\2x + 4 = 0\end{array} \right.$
⇔[x=5x=−2$ \Leftrightarrow \left[ \begin{array}{l}x = 5\\x =  - 2\end{array} \right.$
Vậy x = 5, x = - 2.
b) 2x(x – 5) – 7(5 – x) = 0
⇔ 2x(x – 5) + 7(x – 5) = 0
⇔ (x – 5)(2x + 7) = 0
⇔[x−5=02x+7=0$ \Leftrightarrow \left[ \begin{array}{l}x - 5 = 0\\2x + 7 = 0\end{array} \right.$
⇔[x=5x=−72$ \Leftrightarrow \left[ \begin{array}{l}x = 5\\x =  - \frac{7}{2}\end{array} \right.$
Vậy x=−72$x =  - \frac{7}{2}$, x = 5.
c) x2– 3x – 10 = 0
⇔ x2– 5x + 2x – 10 = 0
⇔ x(x – 5) + 2(x – 5) = 0
⇔ (x – 5)(x + 2) = 0
⇔[x−5=0x+2=0$ \Leftrightarrow \left[ \begin{array}{l}x - 5 = 0\\x + 2 = 0\end{array} \right.$
⇔[x=5x=−2$ \Leftrightarrow \left[ \begin{array}{l}x = 5\\x =  - 2\end{array} \right.$
Vậy x = 5, x = – 2.

========================================================================

https://khoahoc.vietjack.com/thi-online/12-bai-tap-cdach-tinh-ban-kinh-duong-tron-noi-tiep-ngoai-tiep-cua-tam-giac-co-loi-giai


\textbf{{QUESTION}}

Tam giác ABC có BC = 8 và $\widehat A = 30^\circ $. Tính bán kính R của đường tròn ngoại tiếp tam giác ABC.
Tam giác ABC có BC = 8 và $\widehat A = 30^\circ $. 

Tính bán kính R của đường tròn ngoại tiếp tam giác ABC.

\textbf{{ANSWER}}

Hướng dẫn giải:
Ta áp dụng công thức $\frac{a}{{\sin A}} = 2R$
$ \Rightarrow R = \frac{a}{{2\sin A}} = \frac{{BC}}{{2\sin A}} = \frac{8}{{2\sin 30^\circ }} = \frac{8}{{2.\frac{1}{2}}} = 8$.
Vậy bán kính đường tròn ngoại tiếp tam giác ABC là R = 8.

========================================================================

https://khoahoc.vietjack.com/thi-online/12-bai-tap-cdach-tinh-ban-kinh-duong-tron-noi-tiep-ngoai-tiep-cua-tam-giac-co-loi-giai


\textbf{{QUESTION}}

Tam giác ABC có AB = 6, AC = 8 và ^BAC=60∘$\widehat {BAC} = 60^\circ $. Tính bán kính r của đường tròn nội tiếp tam giác đã cho.
Tam giác ABC có AB = 6, AC = 8 và 
^BAC=60∘$\widehat {BAC} = 60^\circ $
^BAC=60∘
^BAC=60∘
^BAC
^BAC
^BAC
^BAC
^
^
BAC
B
A
C
=
60∘
60
∘
. Tính bán kính r của đường tròn nội tiếp tam giác đã cho
.

\textbf{{ANSWER}}

Hướng dẫn giải:
Theo địn lí côsin ta có: $B{C^2} = A{B^2} + A{C^2} - 2.AB.AC.\cos A$
Thay số: $B{C^2} = {6^2} + {8^2} - 2.6.8.\cos 60^\circ = 52$
$ \Rightarrow BC = \sqrt {52} $.
Do đó ta có nửa chu vi tam giác ABC là: 
$p = \frac{1}{2}\left( {AB + AC + BC} \right) = \frac{1}{2}\left( {6 + 8 + \sqrt {52} } \right) = 7 + \sqrt {13} $.
Diện tích tam giác ABC là:
$S = \sqrt {p\left( {p - AB} \right)\left( {p - AC} \right)\left( {p - BC} \right)} = 12\sqrt 3 $.
Mặt khác $S = p.r \Rightarrow r = \frac{S}{p} = \frac{{12\sqrt 3 }}{{7 + \sqrt {13} }} \approx 1,96$.

========================================================================

https://khoahoc.vietjack.com/thi-online/12-bai-tap-cdach-tinh-ban-kinh-duong-tron-noi-tiep-ngoai-tiep-cua-tam-giac-co-loi-giai


\textbf{{QUESTION}}

Tam giác ABC có a = 20, b = 15, c = 9. Bán kính r của đường tròn nội tiếp tam giác đã cho gần với giá trị nào dưới đây?

\textbf{{ANSWER}}

Hướng dẫn giải:
Đáp án đúng là: D.
Ta có p=12(a+b+c)=12(20+15+9)=22$p = \frac{1}{2}\left( {a + b + c} \right) = \frac{1}{2}\left( {20 + 15 + 9} \right) = 22$.
Do đó diện tích tam giác ABC là: 
S=√p(p−a)(p−b)(p−c)=√22.(22−20).(22−15).(22−9)=2√1001$S = \sqrt {p\left( {p - a} \right)\left( {p - b} \right)\left( {p - c} \right)} = \sqrt {22.\left( {22 - 20} \right).\left( {22 - 15} \right).\left( {22 - 9} \right)} = 2\sqrt {1001} $.
Lại có S=p.r⇒r=Sp=2√100122≈5,75$S = p.r \Rightarrow r = \frac{S}{p} = \frac{{2\sqrt {1001} }}{{22}} \approx 5,75$.

========================================================================

https://khoahoc.vietjack.com/thi-online/12-bai-tap-cdach-tinh-ban-kinh-duong-tron-noi-tiep-ngoai-tiep-cua-tam-giac-co-loi-giai


\textbf{{QUESTION}}

Cho tam giác ABC có AB = 4, AC = 8 và ˆA=30∘$\widehat A = 30^\circ $. Tính bán kính R của đường tròn ngoại tiếp tam giác ABC.
Cho tam giác ABC có AB = 4, AC = 8 và 
ˆA=30∘$\widehat A = 30^\circ $
ˆA=30∘
ˆA=30∘
ˆA
ˆA
ˆA
ˆA
ˆ
ˆ
A
A
=
30∘
30
∘
. Tính bán kính R của đường tròn ngoại tiếp tam giác ABC
.
A. 7;
B. 6;
C. 5;
D. 4.

\textbf{{ANSWER}}

Hướng dẫn giải:
Đáp án đúng là: C.
Tam giác ABC có: BC2=AB2+AC2−2AB.AC.cosA$B{C^2} = A{B^2} + A{C^2} - 2AB.AC.\cos A$
Thay số: BC2=42+82−2.4.8.cos30∘=80−32√3$B{C^2} = {4^2} + {8^2} - 2.4.8.\cos 30^\circ = 80 - 32\sqrt 3 $
Do đó: BC ≈ 5.
Ta có: BCsinA=2R$\frac{{BC}}{{\sin A}} = 2R$⇒R=BC2sinA≈52.sin30∘=5$ \Rightarrow R = \frac{{BC}}{{2\sin A}} \approx \frac{5}{{2.\sin 30^\circ }} = 5$.

========================================================================

https://khoahoc.vietjack.com/thi-online/12-bai-tap-cdach-tinh-ban-kinh-duong-tron-noi-tiep-ngoai-tiep-cua-tam-giac-co-loi-giai


\textbf{{QUESTION}}

Cho tam giác ABC biết a = 21 cm, b = 17 cm, c = 10. Tính bán kính R của đường tròn ngoại tiếp tam giác ABC.

\textbf{{ANSWER}}

Hướng dẫn giải:
Đáp án đúng là: B.
Nửa chu vi tam giác ABC là: p=12(a+b+c)=12(21+17+10)=24$p = \frac{1}{2}\left( {a + b + c} \right) = \frac{1}{2}\left( {21 + 17 + 10} \right) = 24$.
Do đó diện tích tam giác ABC bằng:
S=√p(p−a)(p−b)(p−c)=84$S = \sqrt {p\left( {p - a} \right)\left( {p - b} \right)\left( {p - c} \right)} = 84$
Mặt khác S=abc4R⇒R=abc4S=21.17.104.84=10,625$S = \frac{{abc}}{{4R}} \Rightarrow R = \frac{{abc}}{{4S}} = \frac{{21.17.10}}{{4.84}} = 10,625$.

========================================================================

https://khoahoc.vietjack.com/thi-online/25-de-thi-thu-toan-thpt-quoc-gia-co-loi-giai-chi-tiet/79187


\textbf{{QUESTION}}

Trong các khẳng định dưới đây, khẳng định nào sai?
A.$$ \mathrm{log}a$$ xác địn khi $$ 0<a<1$$
B. $$ \mathrm{ln}a>0\Leftrightarrow a>1$$
C. $$ {\mathrm{log}}_{\frac{1}{2}}a>{\mathrm{log}}_{\frac{1}{2}}b\Leftrightarrow a>b>0$$
D. $$ {\mathrm{log}}_{\frac{1}{5}}a={\mathrm{log}}_{\frac{1}{5}}b\Leftrightarrow a=b>0$$

\textbf{{ANSWER}}

Chọn C

========================================================================

https://khoahoc.vietjack.com/thi-online/25-de-thi-thu-toan-thpt-quoc-gia-co-loi-giai-chi-tiet/79187


\textbf{{QUESTION}}

Có bao nhiêu cách chọn 5 quyển sách từ 20 quyển sách?
A.$$ {C}_{20}^{5}$$
B. $$ {P}_{5}$$
C. $$ {A}_{20}^{5}$$
D. 5

\textbf{{ANSWER}}

Chọn A

========================================================================

https://khoahoc.vietjack.com/thi-online/25-de-thi-thu-toan-thpt-quoc-gia-co-loi-giai-chi-tiet/79187


\textbf{{QUESTION}}

Tập xác định của hàm số y=lnx$$ y=\mathrm{ln}x$$  là
A. [0;+∞)$$ \left[0;+\infty \right)$$
B. (1;+∞)$$ \left(1;+\infty \right)$$
C. (0;+∞)$$ \left(0;+\infty \right)$$
D. R

\textbf{{ANSWER}}

Tập xác định của hàm số y = lnx khi x > 0 
Vậy tập xác định của hàm số đã cho là D=(0;+∞)$$ D=\left(0;+\infty \right)$$.

========================================================================

https://khoahoc.vietjack.com/thi-online/25-de-thi-thu-toan-thpt-quoc-gia-co-loi-giai-chi-tiet/79187


\textbf{{QUESTION}}

Một cấp số cộng (un)$$ \left({u}_{n}\right)$$  với u1=−12$$ {u}_{1}=-\frac{1}{2}$$ , d=12$$ d=\frac{1}{2}$$  có dạng khai triển nào sau đây?
A. −12;0;1;12;1;...$$ -\frac{1}{2};0;1;\frac{1}{2};1;\mathrm{...}$$
B. −12;0;12;0;−12...$$ -\frac{1}{2};0;\frac{1}{2};0;-\frac{1}{2}\mathrm{...}$$
C. 12;1;32;2;52;...$$ \frac{1}{2};1;\frac{3}{2};2;\frac{5}{2};\mathrm{...}$$
D. −12;0;12;1;32;...$$ -\frac{1}{2};0;\frac{1}{2};1;\frac{3}{2};\mathrm{...}$$

\textbf{{ANSWER}}

Chọn D

========================================================================

https://khoahoc.vietjack.com/thi-online/25-de-thi-thu-toan-thpt-quoc-gia-co-loi-giai-chi-tiet/79187


\textbf{{QUESTION}}

Trong không gian Oxyz, cho hai điểm A(0;−1;−2)$$ A\left(0;-1;-2\right)$$  và B(2;2;2)$$ B\left(2;2;2\right)$$ . Độ dài vectơ →AB$$ \overrightarrow{AB}$$  bằng 
A. √29$$ \sqrt{29}$$
B. 29
C. 9
D. 3

\textbf{{ANSWER}}

Chọn A

========================================================================

https://khoahoc.vietjack.com/thi-online/bai-tap-toan-8-chu-de-14-hinh-thoi-co-dap-an/109839


\textbf{{QUESTION}}

Cho tam giác ABC, phân giác AD. Qua D kẻ đường thẳng song song với AC cắt 
AB
 
tại E, qua D kẻ đường thẳng song song với 
AB
 
cắt AC tại F. Chứng minh EF là phân giác của $$ \widehat{AE\text{D}}.$$

\textbf{{ANSWER}}

Chứng minh tứ giác AEDF là hình thoi
=> EF là phân giác của $$ \widehat{AED}$$

========================================================================

https://khoahoc.vietjack.com/thi-online/10-bai-tsap-su-dung-phuong-phap-to-hop-co-loi-giai


\textbf{{QUESTION}}

Một lớp có 40 học sinh, trong đó có 4 học sinh tên Anh. Trong một lần kiểm tra bài cũ, thầy giáo gọi ngẫu nhiên hai học sinh trong lớp lên bảng. Xác suất để hai học sinh tên Anh lên bảng bằng
A. $$ \frac{\text{1}}{\text{10}}$$
B. $$ \frac{\text{1}}{\text{20}}$$
C. $$ \frac{1}{130}$$
D. $$ \frac{\text{1}}{\text{75}}$$

\textbf{{ANSWER}}

Đáp án đúng là: C
Chọn ngẫu nhiên 2 học sinh từ 40 học sinh. Số phần tử của không gian mẫu  $$ \text{n}\left(\text{Ω}\right){\text{ = C}}_{\text{40}}^{\text{2}}\text{ = 780}$$.
Gọi A là biến cố: “Gọi hai học sinh tên Anh lên bảng” 
Ta có  $$ \text{n}\left(\text{A}\right){\text{ = C}}_{\text{4}}^{\text{2}}\text{ = 6}$$.
Vậy xác suất cần tìm là  $$ \text{P}\left(\text{A}\right)\text{ = }\frac{\text{6}}{\text{780}}\text{ = }\frac{\text{1}}{\text{130}}$$.

========================================================================

https://khoahoc.vietjack.com/thi-online/10-bai-tsap-su-dung-phuong-phap-to-hop-co-loi-giai


\textbf{{QUESTION}}

Hộp A có 4 viên bi trắng, 5 viên bi đỏ và 6 viên bi xanh. Hộp B có 7 viên bi trắng, 6 viên bi đỏ và 5 viên bi xanh. Lấy ngẫu nhiên mỗi hộp một viên bi, xác suất để hai viên bi được lấy ra có cùng màu là
A. $$ \frac{91}{135}$$
B. $$ \frac{44}{135}$$
C. $$ \frac{88}{135}$$
D. $$ \frac{45}{88}$$

\textbf{{ANSWER}}

Đáp án đúng là: B
Chọn mỗi hộp 1 viên bi. Số phần tử của không gian mẫu:  $$ n\left(\Omega \right)={C}_{15}^{1}\cdot {C}_{18}^{1}=270.$$
Biến cố A: “Hai viên bi được lấy ra có cùng màu”
Trường hợp 1: 2 viên màu trắng, có  $$ {C}_{4}^{1}\cdot {C}_{7}^{1}=28$$ cách chọn.
Trường hợp 2: 2 viên màu đỏ, có  $$ {C}_{5}^{1}\cdot {C}_{6}^{1}=30$$ cách chọn.
Trường hợp 3: 2 viên màu xanh, có  $$ {C}_{6}^{1}\cdot {C}_{5}^{1}=30$$ cách chọn.
Số cách chọn từ mỗi hộp 1 viên bi sau cho 2 viên bi cùng màu là: 28 + 30 + 30 = 88.
Suy ra n(A) = 88.
Vậy xác suất cần tìm là  $$ P\left(A\right)=\frac{n\left(A\right)}{n\left(\Omega \right)}=\frac{88}{270}=\frac{44}{135}$$.

========================================================================

https://khoahoc.vietjack.com/thi-online/10-bai-tsap-su-dung-phuong-phap-to-hop-co-loi-giai


\textbf{{QUESTION}}

Trong một đợt kiểm tra định kỳ, giáo viên chuẩn bị một hộp đựng 15 câu hỏi gồm 5 câu hỏi Hình học và 10 câu hỏi Đại số khác nhau. Mỗi học sinh bốc ngẫu nhiên từ hộp đó 3 câu hỏi để làm đề thi cho mình. Xác suất để một học sinh bốc được đúng một câu hình học là
A. 4591$$ \frac{45}{91}$$
B. 34$$ \frac{3}{4}$$
C. 200273$$ \frac{200}{273}$$
D. 23$$ \frac{2}{3}$$

\textbf{{ANSWER}}

Đáp án đúng là: A
Xét phép thử: “ Chọn 3 câu hỏi từ 15 câu hỏi”. Suy ra,  n(Ω) = C315 = 455.$$ \text{n}\left(\text{Ω}\right){\text{ = C}}_{\text{15 }}^{\text{3}}\text{= 455.}$$
Gọi A là biến cố: “ Chọn được đúng 1 câu hình” 
– Có  C15$$ {\text{C}}_{\text{5}}^{\text{1}}$$ cách chọn 1 câu từ 5 câu Hình học.
– Có  C210$$ {\text{C}}_{\text{10}}^{\text{2}}$$ cách chọn 2 câu trong 10 câu Đại số.
Suy ra  n(A)= C15⋅C210 = 225.$$ \text{n}\left(\text{A}\right){\text{= C}}_{\text{5}}^{\text{1}}\cdot {\text{C}}_{\text{10}}^{\text{2}}\text{ = 225.}$$
Xác suất biến cố A là:  P(A)=n(A)n(Ω)=225455=4591.$$ \text{P}\left(\text{A}\right)=\frac{n\left(A\right)}{n\left(\Omega \right)}=\frac{225}{455}=\frac{45}{91}.$$

========================================================================

https://khoahoc.vietjack.com/thi-online/10-bai-tsap-su-dung-phuong-phap-to-hop-co-loi-giai


\textbf{{QUESTION}}

Có 5 học sinh không quen biết nhau cùng đến một cửa hàng kem có 6 quầy phục vụ. Xác suất để có 3 học sinh cùng vào một quầy và 2 học sinh còn lại vào một quầy khác là
A. C35⋅C16⋅5!65$$ \frac{{\text{C}}_{\text{5}}^{\text{3}}\cdot {\text{C}}_{\text{6}}^{\text{1}}\cdot \text{5!}}{{\text{6}}^{\text{5}}}$$
B. C35⋅C16⋅C1565$$ \frac{{\text{C}}_{\text{5}}^{\text{3}}\cdot {\text{C}}_{\text{6}}^{\text{1}}\cdot {\text{C}}_{\text{5}}^{\text{1}}}{{\text{6}}^{\text{5}}}$$
C. C35.C16.5!56$$ \frac{{\text{C}}_{\text{5}}^{\text{3}}{\text{.C}}_{\text{6}}^{\text{1}}\text{.5!}}{{\text{5}}^{\text{6}}}$$
D. C35.C16.C1556$$ \frac{{\text{C}}_{\text{5}}^{\text{3}}{\text{.C}}_{\text{6}}^{\text{1}}{\text{.C}}_{\text{5}}^{\text{1}}}{{\text{5}}^{\text{6}}}$$

\textbf{{ANSWER}}

Đáp án đúng là: B
Ta có mỗi học sinh có 6 cách chọn quầy phục vụ nên  n(Ω) = 65$$ \text{n}\left(\text{Ω}\right){\text{ = 6}}^{\text{5}}$$.
Gọi A là biến cố thỏa mãn yêu cầu bài toán.
Chọn 3 học sinh trong 5 học sinh để vào cùng một quầy  C35$$ {\text{C}}_{\text{5}}^{\text{3}}$$.
Sau đó chọn 1 quầy trong 6 quầy để các em vào là  C16$$ {\text{C}}_{\text{6}}^{\text{1}}$$.
Còn 2 học sinh còn lại có  C15$$ {\text{C}}_{\text{5}}^{\text{1}}$$ cách chọn quầy để vào cùng.
Nên  n(A) = C35⋅C16⋅C15$$ \text{n}\left(\text{A}\right){\text{ = C}}_{\text{5}}^{\text{3}}\cdot {\text{C}}_{\text{6}}^{\text{1}}\cdot {\text{C}}_{\text{5}}^{\text{1}}$$.
Vậy  P(A)=n(A)n(Ω)=C35⋅C16⋅C1565.$$ \text{P}\left(\text{A}\right)=\frac{n\left(A\right)}{n\left(\Omega \right)}=\frac{{\text{C}}_{\text{5}}^{\text{3}}\cdot {\text{C}}_{\text{6}}^{\text{1}}\cdot {\text{C}}_{\text{5}}^{\text{1}}}{{\text{6}}^{\text{5}}}.$$

========================================================================

https://khoahoc.vietjack.com/thi-online/10-bai-tsap-su-dung-phuong-phap-to-hop-co-loi-giai


\textbf{{QUESTION}}

Một hộp đựng 9 viên bi trong đó có 4 viên bi đỏ và 5 viên bi xanh. Lấy ngẫu nhiên từ hộp 3 viên bi. Xác suất để 3 viên bi lấy ra có ít nhất 2 viên bi màu xanh là
A. 1021
B. 514
C. 2542
D. 542

\textbf{{ANSWER}}

Đáp án đúng là: C
Lấy ngẫu nhiên 3 viên bi từ 9 viên trong hộp. Số phần tử không gian mẫu:  n(Ω) = C39.$$ \text{n}\left(\text{Ω}\right){\text{ = C}}_{\text{9}}^{\text{3}}.$$
Gọi biến cố A: “Lấy được ít nhất 2 viên bi màu xanh”.
Trường hợp 1: 2 viên màu xanh, 1 viên khác màu xanh. Có  C25⋅C14$$ {\text{C}}_{\text{5}}^{\text{2}}\cdot {\text{C}}_{\text{4}}^{\text{1}}$$ cách chọn
Trường hợp 2: 3 viên màu xanh có  C35$$ {\text{C}}_{\text{5}}^{\text{3}}$$ cách chọn
Suy ra  .n(A) = C25⋅C14 + C35$$ \text{n}\left(\text{A}\right){\text{ = C}}_{\text{5}}^{\text{2}}\cdot {\text{C}}_{\text{4}}^{\text{1}}{\text{ + C}}_{\text{5}}^{\text{3}}$$
Vậy  P(A) = C25⋅C14+C35C39 = 2542$$ \text{P}\left(\text{A}\right)\text{ = }\frac{{\text{C}}_{\text{5}}^{\text{2}}\cdot {\text{C}}_{\text{4}}^{\text{1}}{\text{+C}}_{\text{5}}^{\text{3}}}{{\text{C}}_{\text{9}}^{\text{3}}}\text{ = }\frac{\text{25}}{\text{42}}$$.

========================================================================

https://khoahoc.vietjack.com/thi-online/bai-tap-chuyen-de-toan-6-dang-1-tinh-chat-chia-het-cua-so-tu-nhien-co-dap-an/106924


\textbf{{QUESTION}}

Dùng cả bốn chữ số 4;0;7;5 hãy viết thành số tự nhiên có bốn chữ Số khác nhau sao cho số đó thỏa mãn:

\textbf{{ANSWER}}

a) Vì số đó chia hết cho 2 nên sẽ tận cùng là  0;4.
Số có bốn chữ số lớn nhất nên số hàng nghìn là 7 và số hàng trăm là 5.
Ta có hai số 7504; 7540 thỏa mãn chia hết cho 2. 
Vì 7504 < 7540  nên số lớn nhất chia hết cho 2 là 7540.

========================================================================

https://khoahoc.vietjack.com/thi-online/bai-tap-chuyen-de-toan-6-dang-1-tinh-chat-chia-het-cua-so-tu-nhien-co-dap-an/106924


\textbf{{QUESTION}}

b) Số nhỏ nhất chia hết cho 5;

\textbf{{ANSWER}}

b) Lập luận tương tự câu a) ta có đáp số: 4075.

========================================================================

https://khoahoc.vietjack.com/thi-online/bai-tap-chuyen-de-toan-6-dang-1-tinh-chat-chia-het-cua-so-tu-nhien-co-dap-an/106924


\textbf{{QUESTION}}

c) Số chia hết cho 2 và 5.

\textbf{{ANSWER}}

c)  4750;4570;5740;5470;7540;7450$$ 4750;4570;5740;5470;7540;7450$$

========================================================================

https://khoahoc.vietjack.com/thi-online/bai-tap-chuyen-de-toan-6-dang-1-tinh-chat-chia-het-cua-so-tu-nhien-co-dap-an/106924


\textbf{{QUESTION}}

Dùng cả ba chữ số 9; 0; 5 hãy viết thành số tự nhiên có ba chữ số khác nhau sao cho số đó thỏa mãn:

\textbf{{ANSWER}}

a) Vì số đó chia hết cho 2 nên sẽ tận cùng là 0.
Số có bốn chữ số lớn nhất nên số hàng nghìn là 9 và số hàng trăm là 5.
Ta có số 950 thỏa mãn là số lớn nhất chia hết cho 2.

========================================================================

https://khoahoc.vietjack.com/thi-online/bai-tap-chuyen-de-toan-6-dang-1-tinh-chat-chia-het-cua-so-tu-nhien-co-dap-an/106924


\textbf{{QUESTION}}

b) Số nhỏ nhất chia hết cho 5;

\textbf{{ANSWER}}

b) Lập luận tương tự câu a) ta có đáp số: 590

========================================================================

https://khoahoc.vietjack.com/thi-online/bai-tap-on-tap-chuong-i-phan-1-co-loi-giai-chi-tiet


\textbf{{QUESTION}}

Tích của đơn thức x và đa thức (1 – x) là:
A. 1 – 2x
B. x – $$ {x}^{2}$$
C. $$ {x}^{2}$$ – x
D. $$ {x}^{2}$$ + x

\textbf{{ANSWER}}

x(1 – x) = x.1 – x.x = x – $$ {x}^{2}$$
Đáp án cần chọn là: B

========================================================================

https://khoahoc.vietjack.com/thi-online/bai-tap-on-tap-chuong-i-phan-1-co-loi-giai-chi-tiet


\textbf{{QUESTION}}

Tích của đa thức 4x5 + 7x2$$ 4{x}^{5}\quad +\quad 7{x}^{2}$$ và đơn thức (-3x3)$$ (-3{x}^{3})$$ là:
A. 12x8 + 21x5$$ 12{x}^{8}\quad +\quad 21{x}^{5}$$
B. 12x8 + 21x6$$ 12{x}^{8}\quad +\quad 21{x}^{6}$$
C. -12x8 + 21x5$$ -12{x}^{8}\quad +\quad 21{x}^{5}$$
D. -12x8 – 21x5$$ -12{x}^{8}\quad –\quad 21{x}^{5}$$

\textbf{{ANSWER}}

(4x5 + 7x2).(-3x3) = 4x5.(-3x3) + 7x2.(-3x3) = -12x8 – 21x5$$ (4{x}^{5}\quad +\quad 7{x}^{2}).(-3{x}^{3})\quad \phantom{\rule{0ex}{0ex}}=\quad 4{x}^{5}.(-3{x}^{3})\quad +\quad 7{x}^{2}.(-3{x}^{3})\quad \phantom{\rule{0ex}{0ex}}=\quad -12{x}^{8}\quad –\quad 21{x}^{5}$$
Đáp án cần chọn là: D

========================================================================

https://khoahoc.vietjack.com/thi-online/bai-tap-on-tap-chuong-i-phan-1-co-loi-giai-chi-tiet


\textbf{{QUESTION}}

Thực hiện phép tính (x2 + x + 1)(x3 – x2 + 1)$$ ({x}^{2}\quad +\quad x\quad +\quad 1)({x}^{3}\quad –\quad {x}^{2}\quad +\quad 1)$$ ta được kết quả là:
A. x5$$ {x}^{5}$$ + x + 1
B. x5 – x4$$ {x}^{5}\quad –\quad {x}^{4}$$ + x
C. x5 + x4$$ {x}^{5}\quad +\quad {x}^{4}$$ + x
D. x5$$ {x}^{5}$$ – x – 1

\textbf{{ANSWER}}

(x2 + x + 1)(x3 – x2 + 1)  = x2.x3 – x2.x2 + x2.1 + x.x3 – x.x2 +x.1 + 1.x3 – 1.x2 + 1.1  = x5 – x4 + x2 + x4 – x3 + x + x3 – x2 + 1  = x5 + x + 1$$ ({x}^{2}\quad +\quad x\quad +\quad 1)({x}^{3}\quad –\quad {x}^{2}\quad +\quad 1)\quad \quad \phantom{\rule{0ex}{0ex}}=\quad {x}^{2}.{x}^{3}\quad –\quad {x}^{2}.{x}^{2}\quad +\quad {x}^{2}.1\quad +\quad x.{x}^{3}\quad –\quad x.{x}^{2}\quad +x.1\quad +\quad 1.{x}^{3}\quad –\quad 1.{x}^{2}\quad +\quad 1.1\quad \quad \phantom{\rule{0ex}{0ex}}=\quad {x}^{5}\quad –\quad {x}^{4}\quad +\quad {x}^{2}\quad +\quad {x}^{4}\quad –\quad {x}^{3}\quad +\quad x\quad +\quad {x}^{3}\quad –\quad {x}^{2}\quad +\quad 1\quad \quad \phantom{\rule{0ex}{0ex}}=\quad {x}^{5}\quad +\quad x\quad +\quad 1$$
Đáp án cần chọn là: A

========================================================================

https://khoahoc.vietjack.com/thi-online/bai-tap-on-tap-chuong-i-phan-1-co-loi-giai-chi-tiet


\textbf{{QUESTION}}

Rút gọn biểu thức A=(x2 + 2 – 2x)(x2 + 2 + 2x) – x4$$ A=({x}^{2}\quad +\quad 2\quad –\quad 2x)({x}^{2}\quad +\quad 2\quad +\quad 2x)\quad –\quad {x}^{4}$$ ta được kết quả là
A. A = 4
B. A = -4
C. A = 19
D. A = -19

\textbf{{ANSWER}}

A = (x2 + 2 – 2x)(x2 + 2 + 2x) – x4  A = x2.x2 + 2.x2 + 2x.x2 + 2.x2 + 2.2 + 2.2x – 2x.x2 – 2.2x – 2x.2x – x4  A = x4 + 2x2 + 2x3 + 2x2 + 4 + 4x – 2x3 – 4x – 4x2 – x4A =4$$ A\quad =\quad ({x}^{2}\quad +\quad 2\quad –\quad 2x)({x}^{2}\quad +\quad 2\quad +\quad 2x)\quad –\quad {x}^{4}\quad \quad \phantom{\rule{0ex}{0ex}}A\quad =\quad {x}^{2}.{x}^{2}\quad +\quad 2.{x}^{2}\quad +\quad 2x.{x}^{2}\quad +\quad 2.{x}^{2}\quad +\quad 2.2\quad +\quad 2.2x\quad –\quad 2x.{x}^{2}\quad –\quad 2.2x\quad –\quad 2x.2x\quad –\quad {x}^{4}\quad \quad \phantom{\rule{0ex}{0ex}}A\quad =\quad {x}^{4}\quad +\quad 2{x}^{2}\quad +\quad 2{x}^{3}\quad +\quad 2{x}^{2}\quad +\quad 4\quad +\quad 4x\quad –\quad 2{x}^{3}\quad –\quad 4x\quad –\quad 4{x}^{2}\quad –\quad {x}^{4}\phantom{\rule{0ex}{0ex}}A\quad =4$$
Đáp án cần chọn là: A

========================================================================

https://khoahoc.vietjack.com/thi-online/bai-tap-on-tap-chuong-i-phan-1-co-loi-giai-chi-tiet


\textbf{{QUESTION}}

Rút gọn đa thức 16x2 – 4x + 14$$ 16{x}^{2}\quad –\quad 4x\quad +\quad \frac{1}{4}$$ta được kết quả nào sau đây?
A.  (4x-12)2$$ {\left(4x-\frac{1}{2}\right)}^{2}$$
B.  (x-12)2$$ {\left(x-\frac{1}{2}\right)}^{2}$$
C. (4x+12)2$$ {\left(4x+\frac{1}{2}\right)}^{2}$$
D.  (x+12)2$$ {\left(x+\frac{1}{2}\right)}^{2}$$

\textbf{{ANSWER}}

16x2 – 4x + 14 = (4x)2-2.4x12+(12)2=(4x-12)2$$ 16x2\quad –\quad 4x\quad +\quad \frac{1}{4}\quad =\quad {\left(4x\right)}^{2}-2.4x\frac{1}{2}+{\left(\frac{1}{2}\right)}^{2}\phantom{\rule{0ex}{0ex}}={\left(4x-\frac{1}{2}\right)}^{2}$$
Đáp án cần chọn là: A

========================================================================

https://khoahoc.vietjack.com/thi-online/30-de-thi-thu-thpt-quoc-gia-mon-toan-nam-2022-co-loi-giai/116964


\textbf{{QUESTION}}

Gọi T là tập tất cả những giá trị thực của x để $$ {\mathrm{log}}_{3}\left(2021-x\right)$$ có nghĩa. Tìm T ?

\textbf{{ANSWER}}

Điều kiện $$ 2021-x>0\Leftrightarrow x<2021.$$
Vậy $$ T=\left(-\infty ;2021\right).$$
Chọn C.

========================================================================

https://khoahoc.vietjack.com/thi-online/30-de-thi-thu-thpt-quoc-gia-mon-toan-nam-2022-co-loi-giai/116964


\textbf{{QUESTION}}

Cho hai tích phân 5∫−2f(x)dx=8$$ \underset{-2}{\overset{5}{\int }}f\left(x\right)\text{d}x=8$$ và −2∫5g(x)dx=3$$ \underset{5}{\overset{-2}{\int }}g\left(x\right)\text{d}x=3$$. Tính I=5∫−2[f(x)−4g(x)−1]dx$$ I=\underset{-2}{\overset{5}{\int }}\left[f\left(x\right)-4g\left(x\right)-1\right]\text{d}x$$.

\textbf{{ANSWER}}

Ta có: $$ I=\underset{-2}{\overset{5}{\int }}\left[f\left(x\right)-4g\left(x\right)-1\right]dx=\underset{-2}{\overset{5}{\int }}f\left(x\right)dx-4\underset{-2}{\overset{5}{\int }}g\left(x\right)dx-\underset{-2}{\overset{5}{\int }}1dx=8-4.\left(-3\right)-7=13.$$
Chọn C.

========================================================================

https://khoahoc.vietjack.com/thi-online/30-de-thi-thu-thpt-quoc-gia-mon-toan-nam-2022-co-loi-giai/116964


\textbf{{QUESTION}}

Nguyên hàm ∫cos2x dx$$ \underset{}{\overset{}{\int }}\mathrm{cos}2x\text{\hspace{0.17em}}dx$$ bằng

\textbf{{ANSWER}}

Ta có: ∫cos2xdx=12sin2x+C.$$ \int \mathrm{cos}2xdx=\frac{1}{2}\mathrm{sin}2x+C.$$
Chọn C.

========================================================================

https://khoahoc.vietjack.com/thi-online/30-de-thi-thu-thpt-quoc-gia-mon-toan-nam-2022-co-loi-giai/116964


\textbf{{QUESTION}}

Cho một hình cầu có diện tích bề mặt bằng 16π$$ 16\mathrm{\pi }$$, bán kính của hình cầu đã cho bằng
B. 2.

\textbf{{ANSWER}}

Ta có: S=4πR2⇒R=√S4π=√16π4π=2.$$ S=4\pi {R}^{2}\Rightarrow R=\sqrt{\frac{S}{4\pi }}=\sqrt{\frac{16\pi }{4\pi }}=2.$$
Chọn B.

========================================================================

https://khoahoc.vietjack.com/thi-online/30-de-thi-thu-thpt-quoc-gia-mon-toan-nam-2022-co-loi-giai/116964


\textbf{{QUESTION}}

Trong không gian Oxyz, cho mặt phẳng (P):  2x−3y+5=0$$ \left(P\right):\text{\hspace{0.17em}\hspace{0.17em}}2x-3y+5=0$$. Vectơ nào sau đây là một vectơ pháp tuyến của (P) ?

\textbf{{ANSWER}}

Vectơ →n1=(2;−3;0)$$ \overrightarrow{{n}_{1}}=\left(2;-3;0\right)$$ là một vectơ pháp tuyến của (P)
Chọn A.

========================================================================

https://khoahoc.vietjack.com/thi-online/10-cau-trac-nghiem-toan-6-chan-troi-sang-tao-bai-tap-on-tap-chuong-6-so-thap-phan-co-dap-an


\textbf{{QUESTION}}

Hỗn số $1\frac{2}{5}$  được chuyển thành số thập phân là:
A.1,2
B.1,4
C.1,5
D.1,8

\textbf{{ANSWER}}

$1\frac{2}{5} = \frac{{1.5 + 2}}{5} = \frac{7}{5} = \frac{{14}}{{10}} = 1,4.$
Đáp án cần chọn là: B

========================================================================

https://khoahoc.vietjack.com/thi-online/10-cau-trac-nghiem-toan-6-chan-troi-sang-tao-bai-tap-on-tap-chuong-6-so-thap-phan-co-dap-an


\textbf{{QUESTION}}

Phân số 25$\frac{2}{5}$  viết dưới dạng số thập phân là:
A.2,5
B.5,2
C.0,4
D.0,04

\textbf{{ANSWER}}

25=410=0,4.$\frac{2}{5} = \frac{4}{{10}} = 0,4.$
Đáp án cần chọn là: C

========================================================================

https://khoahoc.vietjack.com/thi-online/10-cau-trac-nghiem-toan-6-chan-troi-sang-tao-bai-tap-on-tap-chuong-6-so-thap-phan-co-dap-an


\textbf{{QUESTION}}

A. 301510$\frac{{3015}}{{10}}$ 
B. 3015100$\frac{{3015}}{{100}}$ 
C. 30151000$\frac{{3015}}{{1000}}$ 
D. 301510000$\frac{{3015}}{{10000}}$

\textbf{{ANSWER}}

3,015=30151000$3,015 = \frac{{3015}}{{1000}}$
Đáp án cần chọn là: C

========================================================================

https://khoahoc.vietjack.com/thi-online/10-cau-trac-nghiem-toan-6-chan-troi-sang-tao-bai-tap-on-tap-chuong-6-so-thap-phan-co-dap-an


\textbf{{QUESTION}}

A.35
B.36
C.37
D.34

\textbf{{ANSWER}}

Ta có: 35,67 < x < 36,05 và x là số tự nhiên nên x = 36.
Đáp án cần chọn là: B

========================================================================

https://khoahoc.vietjack.com/thi-online/10-cau-trac-nghiem-toan-6-chan-troi-sang-tao-bai-tap-on-tap-chuong-6-so-thap-phan-co-dap-an


\textbf{{QUESTION}}

Tìm một phân số ở giữa hai phân số 110$\frac{1}{{10}}$ và 210$\frac{2}{{10}}$
A. 310$\frac{3}{{10}}$
B. 1510$\frac{{15}}{{10}}$
C. 15100$\frac{{15}}{{100}}$
D. Không có phân số nào thỏa mãn.

\textbf{{ANSWER}}

Ta có: 110=0,1;210=0,2$\frac{1}{{10}} = 0,1;\;\;\,\frac{2}{{10}} = 0,2$
Vậy số cần tìm phải thỏa mãn: 0,1 < x < 0,2 nên trong các đáp án trên thì x chỉ có thể là 0,15=15100.$0,15 = \frac{{15}}{{100}}.$
Đáp án cần chọn là: C

========================================================================

https://khoahoc.vietjack.com/thi-online/10-bai-tap-squy-dong-mau-thuc-nhieu-phan-thuc-co-loi-giai


\textbf{{QUESTION}}

Mẫu thức chung của hai phân thức $$ \frac{1}{{x}^{2}-xy}$$ và $$ \frac{1}{{x}^{2}}$$ là
A. x2(x – y);

\textbf{{ANSWER}}

Hướng dẫn giải:
Đáp án đúng là: B
Ta có: (x2 – xy). x2 = x2 (x2  – xy).

========================================================================

https://khoahoc.vietjack.com/thi-online/10-bai-tap-squy-dong-mau-thuc-nhieu-phan-thuc-co-loi-giai


\textbf{{QUESTION}}

A. 4(x + 3)2 ;
 
B.42(x-3)(x+3)$$ \frac{4}{2\left(x-3\right)\left(x+3\right)}$$;
C. (x+3)(x-3)$$ \left(x+3\right)\left(x-3\right)$$;

\textbf{{ANSWER}}

Hướng dẫn giải:
Đáp án đúng là: B
Các phân thức trên có mẫu lần lượt là:
4x – 12 = 4(x – 3); 4x + 12 = 4(x + 3); 9 – x2 = – (x – 3)(x + 3).
Nên mẫu thức chung có phần hệ số là 4 và phần biến số là (x – 3)(x + 3).
Hay mẫu thức chung là 4(x – 3)(x + 3).

========================================================================

https://khoahoc.vietjack.com/thi-online/tong-hop-de-thi-thu-thptqg-mon-toan-cuc-hay-tuyen-chon-co-loi-giai-chi-tiet/54396


\textbf{{QUESTION}}

Tìm các họ nghiệm của phương trình $$ \mathrm{cos}\left(3x\right){\mathrm{cos}}^{3}\left(x\right)-\mathrm{sin}\left(3x\right){\mathrm{sin}}^{3}\left(x\right)=\frac{2+3\sqrt{2}}{8}$$
A. $$ \left\{\begin{array}{l}x=\frac{\mathrm{\pi }}{16}+k\frac{\mathrm{\pi }}{2}\\ x=-\frac{\mathrm{\pi }}{16}+k\frac{\mathrm{\pi }}{2}\end{array}\right.$$
B. $$ \left\{\begin{array}{l}x=\frac{\mathrm{\pi }}{16}\\ x=-\frac{\mathrm{\pi }}{16}+k\frac{\mathrm{\pi }}{2}\end{array}\right.$$
C. $$ \left\{\begin{array}{l}x=\frac{\mathrm{\pi }}{16}+k\frac{\mathrm{\pi }}{2}\\ x=-\frac{\mathrm{\pi }}{16}+k\mathrm{\pi }\end{array}\right.$$
D. $$ \left\{\begin{array}{l}x=\frac{\mathrm{\pi }}{16}+k\frac{\mathrm{\pi }}{2}\\ x=-\frac{\mathrm{\pi }}{18}+k\frac{\mathrm{\pi }}{2}\end{array}\right.$$

\textbf{{ANSWER}}

Ta có:
$$ \mathrm{cos}\left(3x\right){\mathrm{cos}}^{3}\left(x\right)-\mathrm{sin}\left(3x\right){\mathrm{sin}}^{3}\left(x\right)=\frac{2+3\sqrt{2}}{8}\phantom{\rule{0ex}{0ex}}\Leftrightarrow \mathrm{cos}\left(3x\right).\frac{\mathrm{cos}\left(3x\right)+3\mathrm{cos}\left(x\right)}{4}\phantom{\rule{0ex}{0ex}}-\mathrm{sin}\left(3x\right).\frac{3\mathrm{sin}\left(x\right)-\mathrm{sin}\left(3x\right)}{4}=\frac{2+3\sqrt{2}}{8}\phantom{\rule{0ex}{0ex}}\Leftrightarrow 2{\mathrm{cos}}^{2}\left(3x\right)+6\mathrm{cos}\left(3x\right)\mathrm{cos}\left(x\right)\phantom{\rule{0ex}{0ex}}-6\mathrm{sin}\left(3x\right)\mathrm{sin}\left(x\right)+2{\mathrm{sin}}^{2}\left(3x\right)=2+3\sqrt{2}\phantom{\rule{0ex}{0ex}}\Leftrightarrow 2\left({\mathrm{cos}}^{2}\left(3x\right)+{\mathrm{sin}}^{2}\left(3x\right)\right)\phantom{\rule{0ex}{0ex}}+6\left(\mathrm{cos}\left(3x\right)\mathrm{cos}\left(x\right)-\mathrm{sin}\left(3x\right)\mathrm{sin}\left(x\right)\right)=2+3\sqrt{2}\phantom{\rule{0ex}{0ex}}\Leftrightarrow \mathrm{cos}\left(4x\right)=\frac{\sqrt{2}}{2}\phantom{\rule{0ex}{0ex}}\Leftrightarrow \left\{\begin{array}{l}x=\frac{\mathrm{\pi }}{16}+k\frac{\mathrm{\pi }}{2}\\ x=-\frac{\mathrm{\pi }}{16}+k\frac{\mathrm{\pi }}{2}\end{array}\right.$$
Đáp án A

========================================================================

https://khoahoc.vietjack.com/thi-online/tong-hop-de-thi-thu-thptqg-mon-toan-cuc-hay-tuyen-chon-co-loi-giai-chi-tiet/54396


\textbf{{QUESTION}}

Tìm tập xác định D của hàm số 
$$ y=\sqrt{\frac{5-3\mathrm{cos}\left(2x\right)}{\left|1-\mathrm{sin}\left(2x-{\displaystyle \frac{\mathrm{\pi }}{2}}\right)\right|}}$$
A. $$ D=R\setminus \left\{k2\mathrm{\pi },\mathrm{k}\in \mathrm{\mathbb{Z}}\right\}$$
B. $$ D=R\setminus \left\{k\frac{\mathrm{\pi }}{2},\mathrm{k}\in \mathrm{\mathbb{Z}}\right\}$$
C. $$ D=R\setminus \left\{k\mathrm{\pi },\mathrm{k}\in \mathrm{\mathbb{Z}}\right\}$$
D. $$ D=R\setminus \left\{k\frac{2\mathrm{\pi }}{3},\mathrm{k}\in \mathrm{\mathbb{Z}}\right\}$$

\textbf{{ANSWER}}

Ta có $$ -1\le \mathrm{cos}\left(2x\right)\le 1$$ nên $$ 5-3\mathrm{cos}\left(2x\right)>0$$
Mặt khác $$ \left|1+\mathrm{sin}\left(2x-\frac{\mathrm{\pi }}{2}\right)\right|\ge 0$$
Hàm số xác định khi và chỉ khi 
$$ \left\{\begin{array}{l}\frac{5-3\mathrm{cos}\left(2x\right)}{\left|1+\mathrm{sin}\left(2x-{\displaystyle \frac{\mathrm{\pi }}{2}}\right)\right|}\ge 0\\ 1+\mathrm{sin}\left(2x-\frac{\mathrm{\pi }}{2}\right)\ne 0\end{array}\right.\phantom{\rule{0ex}{0ex}}\Leftrightarrow \mathrm{sin}\left(2x-\frac{\mathrm{\pi }}{2}\right)\ne -1\phantom{\rule{0ex}{0ex}}\Leftrightarrow 2x-\frac{\mathrm{\pi }}{2}\ne -\frac{\mathrm{\pi }}{2}+k2\mathrm{\pi }\phantom{\rule{0ex}{0ex}}\Leftrightarrow \mathrm{x}\ne \mathrm{k\pi },\mathrm{k}\in \mathrm{\mathbb{Z}}$$
(Để ý rằng bất phương trình (*) luôn đúng)
Tập xác định là $$ D=R\setminus \left\{k\mathrm{\pi },\mathrm{k}\in \mathrm{\mathbb{Z}}\right\}$$
Dáp án C

========================================================================

https://khoahoc.vietjack.com/thi-online/tong-hop-de-thi-thu-thptqg-mon-toan-cuc-hay-tuyen-chon-co-loi-giai-chi-tiet/54396


\textbf{{QUESTION}}

Cho hàm số f(x)={0   khi x=π2+kπ,k∈ℤ12+tan2(x) $$ f\left(x\right)=\left\{\begin{array}{l}0\quad \quad \quad khi\quad x=\frac{\mathrm{\pi }}{2}+k\mathrm{\pi },\mathrm{k}\in \mathrm{\mathbb{Z}}\\ \frac{1}{2+{\mathrm{tan}}^{2}\left(x\right)}\quad \end{array}\right.$$
Tìm điều kiện của a để hàm số g(x)=f(x)+f(ax)$$ g\left(x\right)=f\left(x\right)+f\left(ax\right)$$ tuần hoàn
A. a∈Z$$ a\in Z$$
B. a∈Q$$ a\in Q$$
C. a∈N$$ a\in N$$
D. a∈(0;+∞)$$ a\in \left(0;+\infty \right)$$

\textbf{{ANSWER}}

Xét hàm số 
- Nếu $$ a=\frac{p}{q}$$ với $$ p\in Z,q\in {N}^{*}$$ thì $$ T=q\mathrm{\pi }$$ là chu kì của g(x)
Vì $$ g\left(x+q\mathrm{\pi }\right)=f\left(x+q\mathrm{\pi }\right)+f\left(ax+p\mathrm{\pi }\right)$$ còn $$ \mathrm{\pi }$$ là chu kì của hàm số f(x)
- Ta sẽ chứng minh nếu a là số vô tỉ thì g(x) không tuần hoàn
Để ý rằng $$ g\left(0\right)=f\left(0\right)+f\left(0\right)=1$$. Nếu $$ g\left({x}_{0}\right)=1$$ đối với $$ {x}_{0}\ne 0$$ nào đó thì $$ {\mathrm{tan}}^{2}\left({x}_{0}\right)=0$$ và $$ {\mathrm{tan}}^{2}\left(a{x}_{0}\right)=0$$. Điều này có nghĩa là $$ {x}_{0}=k\mathrm{\pi }$$ và $$ a{x}_{0}=l\mathrm{\pi }$$ với $$ k,l\in Z$$
Nhưng $$ {x}_{0}\ne 0$$ nghĩa là $$ a=\frac{1}{k}$$. Điều này mâu thuẫn vì a là số vô tỉ. Do đó hàm số g(x) nhận giá trị 1 tại điểm duy nhất x = 0. Như vậy f(x) sẽ không tuần hoàn
Đáp án B

========================================================================

https://khoahoc.vietjack.com/thi-online/tong-hop-de-thi-thu-thptqg-mon-toan-cuc-hay-tuyen-chon-co-loi-giai-chi-tiet/54396


\textbf{{QUESTION}}

Gọi M và m lần lượt là giá trị lớn nhất và giá trị nhỏ nhất của hàm số y=|sin(x)-cos(2x)|$$ y=\left|\mathrm{sin}\left(x\right)-\mathrm{cos}\left(2x\right)\right|$$. Hỏi mệnh đề nào trong các mệnh đề sau là sai?
A. √2Mm=2$$ \sqrt{2Mm}=2$$
B. M + m = 2
C. Mm=0$$ \frac{M}{m}=0$$
D. M - m = 2

\textbf{{ANSWER}}

Ta có 
y=|sin(x)=cos(2x)|=|sin(x)-(1-2sin2(x))|=|2sin2(x)+sin(x)-1|$$ y=\left|\mathrm{sin}\left(x\right)=\mathrm{cos}\left(2x\right)\right|\phantom{\rule{0ex}{0ex}}=\left|\mathrm{sin}\left(x\right)-\left(1-2{\mathrm{sin}}^{2}\left(x\right)\right)\right|\phantom{\rule{0ex}{0ex}}=\left|2{\mathrm{sin}}^{2}\left(x\right)+\mathrm{sin}\left(x\right)-1\right|$$
Đặt t = sin(x),-1≤t≤1$$ -1\le t\le 1$$
Ta sẽ đi tìm GTLN và GTNN của hàm số y=g(t)=|2t2+t-1|$$ y=g\left(t\right)=\left|2{t}^{2}+t-1\right|$$ trên đoạn [ -1;1 ]
Ta có g(t)={-2t3-t+1, -1≤t≤122t3+t-1,  12≤t≤1$$ g\left(t\right)=\left\{\begin{array}{l}-2{t}^{3}-t+1,\quad -1\le t\le \frac{1}{2}\\ 2{t}^{3}+t-1,\quad \quad \frac{1}{2}\le t\le 1\end{array}\right.$$
* Xét hàm số h(t)=-2t3-t+1$$ h\left(t\right)=-2{t}^{3}-t+1$$ trên đoạn[-1;12]$$ \left[-1;\frac{1}{2}\right]$$
Dễ dàng tìm được 
Maxr∈[12;1]h(t)=98⇔t=-14Minr∈[12;1]h(t)=0⇔t=12$$ \underset{r\in \left[\frac{1}{2};1\right]}{Max}h\left(t\right)=\frac{9}{8}\Leftrightarrow t=-\frac{1}{4}\phantom{\rule{0ex}{0ex}}\underset{r\in \left[\frac{1}{2};1\right]}{Min}h\left(t\right)=0\Leftrightarrow t=\frac{1}{2}$$
* Xét hàm số k(t)=2t3+t-1$$ k\left(t\right)=2{t}^{3}+t-1$$ trên đoạn [12;1]$$ \left[\frac{1}{2};1\right]$$
Cũng dễ dàng tìm được 
Maxr∈[12;1]k(t)=2⇔t=1Minr∈[12;1]k(t)=0⇔t=12$$ \underset{r\in \left[\frac{1}{2};1\right]}{Max}k\left(t\right)=2\Leftrightarrow t=1\phantom{\rule{0ex}{0ex}}\underset{r\in \left[\frac{1}{2};1\right]}{Min}k\left(t\right)=0\Leftrightarrow t=\frac{1}{2}$$
Qua hai trường hợp trên ta đi đến kết luận 
Maxr∈[-1;3]g(t)=2⇔t=1Minr∈[-1;3]g(t)=0⇔t=12$$ \underset{r\in \left[-1;3\right]}{Max}g\left(t\right)=2\Leftrightarrow t=1\phantom{\rule{0ex}{0ex}}\underset{r\in \left[-1;3\right]}{Min}g\left(t\right)=0\Leftrightarrow t=\frac{1}{2}$$
Hay 
M=Maxy=2⇔sin(x)=-1⇔x=-π2+k2πm=Miny=0⇔sin(x)=12⇔{x=π6+k2πx=5π6+k2π$$ M=Maxy=2\Leftrightarrow \mathrm{sin}\left(x\right)=-1\Leftrightarrow x=-\frac{\mathrm{\pi }}{2}+k2\mathrm{\pi }\phantom{\rule{0ex}{0ex}}\mathrm{m}=\mathrm{Miny}=0\Leftrightarrow \mathrm{sin}\left(\mathrm{x}\right)=\frac{1}{2}\phantom{\rule{0ex}{0ex}}\Leftrightarrow \left\{\begin{array}{l}\mathrm{x}=\frac{\mathrm{\pi }}{6}+\mathrm{k}2\mathrm{\pi }\\ \mathrm{x}=\frac{5\mathrm{\pi }}{6}+\mathrm{k}2\mathrm{\pi }\end{array}\right.$$
Đáp án C

========================================================================

https://khoahoc.vietjack.com/thi-online/tong-hop-de-thi-thu-thptqg-mon-toan-cuc-hay-tuyen-chon-co-loi-giai-chi-tiet/54396


\textbf{{QUESTION}}

Tính giới hạn limx→∞∑nk=16k(3k+1-2k+1)(3k-2k)$$ \underset{x\to \infty }{\mathrm{lim}}\sum _{k=1}^{n}\frac{{6}^{k}}{\left({3}^{k+1}-{2}^{k+1}\right)\left({3}^{k}-{2}^{k}\right)}$$
A. 0
B. 1
C. -1
D. 2

\textbf{{ANSWER}}

Ta có:
6k(3k+1-2k+1)(3k-2k)=6(3k-2k3k+1-2k+1-3k+1-2k+13k-2k)∑nk=16k(3k+1-2k+1)(3k-2k)=63n-2n3n+1-2n+1$$ \frac{{6}^{k}}{\left({3}^{k+1}-{2}^{k+1}\right)\left({3}^{k}-{2}^{k}\right)}\phantom{\rule{0ex}{0ex}}=6\left(\frac{{3}^{k}-{2}^{k}}{{3}^{k+1}-{2}^{k+1}}-\frac{{3}^{k+1}-{2}^{k+1}}{{3}^{k}-{2}^{k}}\right)\phantom{\rule{0ex}{0ex}}\sum _{k=1}^{n}\frac{{6}^{k}}{\left({3}^{k+1}-{2}^{k+1}\right)\left({3}^{k}-{2}^{k}\right)}\phantom{\rule{0ex}{0ex}}=6\frac{{3}^{n}-{2}^{n}}{{3}^{n+1}-{2}^{n+1}}$$
Do đó:
limx→∞∑nk=16k(3k+1-2k+1)(3k-2k)=6limn→∞(3n-2n)3n+1-2n+1=6limn→∞1-(23)n1-2.(23)2=2$$ \underset{x\to \infty }{\mathrm{lim}}\sum _{k=1}^{n}\frac{{6}^{k}}{\left({3}^{k+1}-{2}^{k+1}\right)\left({3}^{k}-{2}^{k}\right)}\phantom{\rule{0ex}{0ex}}=6\underset{n\to \infty }{\mathrm{lim}}\frac{\left({3}^{n}-{2}^{n}\right)}{{3}^{n+1}-{2}^{n+1}}\phantom{\rule{0ex}{0ex}}=6\underset{n\to \infty }{\mathrm{lim}}\frac{1-{\left({\displaystyle \frac{2}{3}}\right)}^{n}}{1-2.{\left(\frac{2}{3}\right)}^{2}}=2$$
Đáp án D

========================================================================

https://khoahoc.vietjack.com/thi-online/trac-nghiem-chuyen-de-toan-8-chu-de-6-on-tap-va-kiem-tra-co-dap-an-0


\textbf{{QUESTION}}

Số mặt, số đỉnh, số cạnh của hình lập phương là?
B. 6 mặt, 8 đỉnh, 12 cạnh.
D. 8 mặt, 6 đỉnh, 12 cạnh.

\textbf{{ANSWER}}

Hình lập phương cũng được gọi là hình hộp chữ nhật có 6 mặt, 8 đỉnh, 12 cạnh.
Chọn đáp án B.

========================================================================

https://khoahoc.vietjack.com/thi-online/trac-nghiem-chuyen-de-toan-8-chu-de-6-on-tap-va-kiem-tra-co-dap-an-0


\textbf{{QUESTION}}

Hình hộp chữ nhật có số cặp mặt song song là?
D. 5

\textbf{{ANSWER}}

Hình hộp chữ nhật có 3 cặp mặt song song.
Chọn đáp án B.

========================================================================

https://khoahoc.vietjack.com/thi-online/giai-sbt-toan-10-bai-tap-cuoi-chuong-2-co-dap-an


\textbf{{QUESTION}}

Trong các bất phương trình sau, bất phương trình nào là bất phương trình bậc nhất hai ẩn?
A. 2x2 + 3y > 4.
B. xy + x < 5.
C. 32x + 43y ≥ 6.
D. x + y3 ≤ 3.

\textbf{{ANSWER}}

Đáp án đúng là: C
Phương án A có x2 là hạng tử bậc 2.
Phương án B có xy là hạng tử bậc 2.
Phương án D có y3 là hạng tử bậc 3.
Phương án C có các hạng tử đều có bậc bằng 1.
Vậy ta chọn phương án C.

========================================================================

https://khoahoc.vietjack.com/thi-online/30-cau-trac-nghiem-toan-7-chan-troi-sang-tao-bai-tap-cuoi-chuong-1-co-dap-an


\textbf{{QUESTION}}

Cho a, b $ \in \mathbb{Z}$, b ≠ 0, x = $\frac{a}{b}$. Nếu a, b khác dấu thì:
A. x = 0;
B. x > 0;
C. x < 0;
D. Cả B, C đều sai.

\textbf{{ANSWER}}

Đáp án đúng là: C
Ta có x = $\frac{a}{b}$; a, b $ \in \mathbb{Z}$, b ≠ 0;  a, b khác dấu thì x < 0.
Vì số hữu tỉ $\frac{a}{b}$ là phép chia số a cho số b mà hai số nguyên a, b khác dấu nên khi chia cho nhau luôn ra số âm suy ra x < 0).

========================================================================

https://khoahoc.vietjack.com/thi-online/giai-toan-6-chuong-2-goc/16125


\textbf{{QUESTION}}

Điền vào chỗ trống trong phát biểu sau: Hình tạo thành bởi ....... được gọi là tam giác MNP.

\textbf{{ANSWER}}

ba đoạn thẳng MN, NP, PM khi M, N, P không thẳng hàng

========================================================================

https://khoahoc.vietjack.com/thi-online/giai-toan-6-chuong-2-goc/16125


\textbf{{QUESTION}}

Điền vào chỗ trống trong các phát biểu sau: Tam giác TUV là hình .......

\textbf{{ANSWER}}

gồm ba đoạn thẳng TU, UV, VT khi T, U, V không thẳng hàng.

========================================================================

https://khoahoc.vietjack.com/thi-online/20-de-thi-thu-thptqg-mon-toan-moi-nhat-cuc-hay-co-loi-giai/51794


\textbf{{QUESTION}}

Cho số phức $$ z=1-2i$$ . Trên mặt phẳng tọa độ, điểm nào dưới đây là điểm biểu diễn của số phức liên hợp của số phức z?
A.$$ {M}_{1}\left(1;2\right)$$
B.$$ {M}_{2}\left(-1;2\right)$$
C.$$ {M}_{3}\left(-1;-2\right)$$
D.$$ {M}_{4}\left(1;-2\right)$$

\textbf{{ANSWER}}

Đáp án A.
Số phức liên hợp của $$ z=1-2i$$                     là $$ \overline{z}=1+2i$$   .
Do đó $$ {M}_{1}\left(1;2\right)$$    là điểm biểu diễn của $$ \overline{z}$$ .

========================================================================

https://khoahoc.vietjack.com/thi-online/20-de-thi-thu-thptqg-mon-toan-moi-nhat-cuc-hay-co-loi-giai/51794


\textbf{{QUESTION}}

Trong không gian với hệ tọa độ Oxyz, cho tam giác ABC có $$ A\left(2;-1;3\right),B\left(3;5;-1\right)$$ và $$ C\left(1;2;7\right)$$. Tìm tọa độ trọng tâm G của tam giác ABC
A.$$ G\left(2;2;3\right)$$
B.$$ G\left(6;6;9\right)$$
C.$$ G\left(\frac{4}{3};\frac{7}{3};\frac{10}{3}\right)$$
D.$$ G\left(3;3;\frac{9}{2}\right)$$

\textbf{{ANSWER}}

Đáp án A.
Ta có:
 $$ G=\left(\frac{2+3+1}{3};\frac{-1+5+2}{3};\frac{3+\left(-1\right)+7}{3}\right)\phantom{\rule{0ex}{0ex}}=\left(2;2;3\right)$$                     .
Chú ý: Trong không gian Oxyz, cho tam giác ABC. Khi đó trọng tâm G của tam giác có tọa độ là $$ \left(\frac{{x}_{A}+{x}_{B}+{x}_{C}}{3};\frac{{y}_{A}+{y}_{B}+{y}_{C}}{3};\frac{{z}_{A}+{z}_{B}+{z}_{C}}{3}\right)$$

========================================================================

https://khoahoc.vietjack.com/thi-online/20-de-thi-thu-thptqg-mon-toan-moi-nhat-cuc-hay-co-loi-giai/51794


\textbf{{QUESTION}}

Có 16 đội bóng tham gia thi đấu. Hỏi cần phải tổ chức bao nhiêu trận đấu sao cho hai đội bất kì đều gặp nhau đúng một lần?
A. 8
B. 16
C. 120
D. 240

\textbf{{ANSWER}}

Đáp án C.
Số trận đấu cần phải tổ chức là số tổ hợp chập 2 của 16, tức là bằng C216=120$$ {C}_{16}^{2}=120$$

========================================================================

https://khoahoc.vietjack.com/thi-online/bo-24-de-kiem-tra-giua-ki-2-toan-11-co-dap-an-moi-nhat/106295


\textbf{{QUESTION}}

Phát biểu nào sau đây là sai ?
A. $$ \mathrm{lim}{u}_{n}=c$$
B. $$ \mathrm{lim}{q}^{n}=0$$$$ \left(\left|q\right|>1\right)$$
C. $$ \mathrm{lim}\frac{1}{n}=0$$
D. $$ \mathrm{lim}\frac{1}{{n}^{k}}=0$$ với $$ k\in {\mathbb{N}}^{*}$$

\textbf{{ANSWER}}

Theo định nghĩa giới hạn hữu hạn của dãy số (SGK ĐS11-Chương 4) thì $$ \mathrm{lim}{q}^{n}=0$$$$ \left(\left|q\right|<1\right)$$.

========================================================================

https://khoahoc.vietjack.com/thi-online/bo-24-de-kiem-tra-giua-ki-2-toan-11-co-dap-an-moi-nhat/106295


\textbf{{QUESTION}}

A. 5
B. $$ \frac{1}{2}$$
C. -1
D. $$ \frac{5}{2}$$

\textbf{{ANSWER}}

Ta có:
$$ \mathrm{lim}{u}_{n}=\mathrm{lim}\frac{2n-1}{n+1}=\mathrm{lim}\frac{n\left(2-\frac{1}{n}\right)}{n\left(1+\frac{1}{n}\right)}=2$$.
$$ \mathrm{lim}{v}_{n}=\mathrm{lim}\frac{2-3n}{n}=\mathrm{lim}\frac{n\left(\frac{2}{n}-3\right)}{n}=-3$$.
Theo định lý: Nếu $$ \mathrm{lim}{u}_{n}=a$$; $$ \mathrm{lim}{u}_{n}=a$$ (với $$ a\text{\hspace{0.17em}},\text{\hspace{0.17em}\hspace{0.17em}}b\in \mathbb{R} $$) thì  $$ \text{\hspace{0.17em}}\mathrm{lim}\left({u}_{n}+{v}_{n}\right)=a+b$$.
Vậy $$ \text{\hspace{0.17em}}\mathrm{lim}\left({u}_{n}+{v}_{n}\right)=2+\left(-3\right)=-1$$.

========================================================================

https://khoahoc.vietjack.com/thi-online/bo-24-de-kiem-tra-giua-ki-2-toan-11-co-dap-an-moi-nhat/106295


\textbf{{QUESTION}}

Tính giới han lim2n+13n+2$$ \mathrm{lim}\frac{2n+1}{3n+2}$$
A. 23$$ \frac{2}{3}$$
B. 32$$ \frac{3}{2}$$
C. 12$$ \frac{1}{2}$$
D. 0

\textbf{{ANSWER}}

Ta có: lim2n+13n+2=lim2+1n3+2n=23$$ \mathrm{lim}\frac{2n+1}{3n+2}=\mathrm{lim}\frac{2+\frac{1}{n}}{3+\frac{2}{n}}=\frac{2}{3}$$.

========================================================================

https://khoahoc.vietjack.com/thi-online/bo-24-de-kiem-tra-giua-ki-2-toan-11-co-dap-an-moi-nhat/106295


\textbf{{QUESTION}}

A. 1
B. 0
C. -∞$$ -\infty $$
D. +∞$$ +\infty $$

\textbf{{ANSWER}}

Ta có lim(vn−un)=lim(n−n2+1n)=lim−1n=0.$$ \mathrm{lim}\left({v}_{n}-{u}_{n}\right)=\mathrm{lim}\left(n-\frac{{n}^{2}+1}{n}\right)=\mathrm{lim}\frac{-1}{n}=0.$$

========================================================================

https://khoahoc.vietjack.com/thi-online/bo-24-de-kiem-tra-giua-ki-2-toan-11-co-dap-an-moi-nhat/106295


\textbf{{QUESTION}}

A. 1
B. 2
C. 0
D. 3

\textbf{{ANSWER}}

Ta thấy: $$ \mathrm{lim}{q}^{n}=0$$ nếu $$ \left|q\right|<1$$ ; $$ \mathrm{lim}{q}^{n}=+\text{\hspace{0.17em}}\infty $$ nếu q>1. Do đó:
Ÿ $$ \mathrm{lim}{u}_{n}=\mathrm{lim}{\left(\frac{1}{2}\right)}^{n}=0\text{\hspace{0.17em}\hspace{0.17em}\hspace{0.17em}}$$vì $$ 0<\frac{1}{2}<1$$
Ÿ $$ \mathrm{lim}{v}_{n}=\mathrm{lim}{\left(\frac{\pi }{3}\right)}^{n}=+\text{\hspace{0.17em}}\infty $$ vì $$ \frac{\pi }{3}>1$$ 
Ÿ $$ \mathrm{lim}{w}_{n}=\mathrm{lim}\frac{{3}^{n}}{{4}^{n+1}}=\mathrm{lim}\left[\frac{1}{4}.{\left(\frac{3}{4}\right)}^{n}\right]=0$$ vì $$ 0<\frac{3}{4}<1$$.

========================================================================

https://khoahoc.vietjack.com/thi-online/10-bai-taps-xet-tinh-lien-tuc-cua-ham-so-tren-mot-khoang-co-loi-giai


\textbf{{QUESTION}}

Hàm số  $$ f\left(x\right)=\frac{{x}^{2}+3x+5}{x-3}$$ liên tục trên các khoảng

\textbf{{ANSWER}}

Đáp án đúng là: C
Ta thấy tập xác định của hàm số là  $$ \left(-\infty ;3\right)\cup \left(3;+\infty \right)$$. 
Vậy hàm số f(x) liên tục trên các khoảng  $$ \left(-\infty ;3\right)$$ và  $$ \left(3;+\infty \right)$$.

========================================================================

https://khoahoc.vietjack.com/thi-online/10-bai-taps-xet-tinh-lien-tuc-cua-ham-so-tren-mot-khoang-co-loi-giai


\textbf{{QUESTION}}

Hàm số  f(x)=x2+2x+3x2−4x+3$$ f\left(x\right)=\frac{{x}^{2}+2x+3}{{x}^{2}-4x+3}$$ liên tục trên các khoảng

\textbf{{ANSWER}}

Đáp án đúng là: D
 $$ f\left(x\right)=\frac{{x}^{2}+2x+3}{{x}^{2}-4x+3}=\frac{{x}^{2}+2x+3}{\left(x-3\right)\left(x-1\right)}$$.
Ta thấy tập xác định của hàm số là  $$ \left(-\infty ;\text{\hspace{0.17em}\hspace{0.17em}}3\right)\cup \left(1;\text{\hspace{0.17em}\hspace{0.17em}}3\right)\cup \left(3;\text{\hspace{0.17em}\hspace{0.17em}}+\infty \right)$$. 
Vậy hàm số f(x) liên tục trên các khoảng  $$ \left(-\infty ;3\right)$$, (1; 3) và  $$ \left(3;\text{\hspace{0.17em}\hspace{0.17em}}+\infty \right)$$.

========================================================================

https://khoahoc.vietjack.com/thi-online/10-bai-taps-xet-tinh-lien-tuc-cua-ham-so-tren-mot-khoang-co-loi-giai


\textbf{{QUESTION}}

Cho hàm số  f(x)={x2−5x+62x3−16 khi x<22−x   khi x≥2$$ f\left(x\right)=\left\{\begin{array}{c}\frac{{x}^{2}-5x+6}{2{x}^{3}-16}\text{ khi }x<2\\ 2-x\text{   khi }x\ge 2\end{array}\right.$$. Hàm số đã cho:

\textbf{{ANSWER}}

Đáp án đúng là: D

========================================================================

https://khoahoc.vietjack.com/thi-online/10-bai-taps-xet-tinh-lien-tuc-cua-ham-so-tren-mot-khoang-co-loi-giai


\textbf{{QUESTION}}

Cho hàm số  f(x)=2x+3x2−x−6$$ f\left(x\right)=\frac{2x+3}{{x}^{2}-x-6}$$. Khẳng định đúng là
A. Hàm số liên tục trên ℝ;

\textbf{{ANSWER}}

Đáp án đúng là: B
 f(x)=2x+3x2−x−6=2x+3(x−3)(x+2)$$ f\left(x\right)=\frac{2x+3}{{x}^{2}-x-6}=\frac{2x+3}{\left(x-3\right)\left(x+2\right)}$$.
Ta thấy tập xác định của hàm số là  (−∞;  −2)∪(−2;  3)∪(3;  +∞)$$ \left(-\infty ;\text{\hspace{0.17em}\hspace{0.17em}}-2\right)\cup \left(-2;\text{\hspace{0.17em}\hspace{0.17em}}3\right)\cup \left(3;\text{\hspace{0.17em}\hspace{0.17em}}+\infty \right)$$. 
Vậy hàm số f(x) liên tục trên các khoảng  (−∞;−2)$$ \left(-\infty ;-2\right)$$ , (–2;3) và  (3;+∞)$$ \left(3;+\infty \right)$$, hàm số gián đoạn tại x = –2, x = 3.

========================================================================

https://khoahoc.vietjack.com/thi-online/10-bai-taps-xet-tinh-lien-tuc-cua-ham-so-tren-mot-khoang-co-loi-giai


\textbf{{QUESTION}}

Cho hàm số  f(x)={x2−3x−√3,x≠√32√3     ,x=√3$$ f\left(x\right)=\left\{\begin{array}{c}\frac{{x}^{2}-3}{x-\sqrt{3}},x\ne \sqrt{3}\\ 2\sqrt{3}\text{     },x=\sqrt{3}\end{array}\right.$$. Khẳng định đúng trong các khẳng định sau là

\textbf{{ANSWER}}

Đáp án đúng là: B

========================================================================

https://khoahoc.vietjack.com/thi-online/giai-sbt-toan-10-bai-tap-cuoi-chuong-1-co-dap-an


\textbf{{QUESTION}}

Trong các câu sau, câu nào là mệnh đề?
A. 6 + x = 4x2.
B. a < 2.

C. 123 là số nguyên tố phải không?
D. Bắc Giang là tỉnh thuộc miền Nam Việt Nam.

\textbf{{ANSWER}}

Đáp án đúng là: D
“6 + x = 4x2” và “a < 2” là hai mệnh đề chứa biến, ta chưa khẳng định được tính đúng sai của chúng.
“123 là số nguyên tố phải không?” là câu hỏi nên không phải mệnh đề.
“Bắc Giang là tỉnh thuộc miền Nam Việt Nam” là mệnh đề sai do Bắc Giang là một tỉnh thuộc miền Bắc Việt Nam.

========================================================================

https://khoahoc.vietjack.com/thi-online/giai-sbt-toan-10-bai-tap-cuoi-chuong-1-co-dap-an


\textbf{{QUESTION}}

Trong các mệnh đề sau, mệnh đề nào đúng?
A. ∅ = {0}.
B. ∅ ⊂$$ \subset $$ {0}.

\textbf{{ANSWER}}

Đáp án đúng là: B
Tập rỗng là tập con của mọi tập hợp.

========================================================================

https://khoahoc.vietjack.com/thi-online/giai-sbt-toan-10-bai-tap-cuoi-chuong-1-co-dap-an


\textbf{{QUESTION}}

Phủ định của mệnh đề “5 + 8 = 13” là mệnh đề
A. 5 + 8 < 13.

B. 5 + 8 ≥ 13.
C. 5 + 8 > 13.
D. 5 + 8 ≠ 13.

\textbf{{ANSWER}}

Đáp án đúng là: D
Phủ định của “=” là ≠.

========================================================================

https://khoahoc.vietjack.com/thi-online/giai-sbt-toan-10-bai-tap-cuoi-chuong-1-co-dap-an


\textbf{{QUESTION}}

Mệnh đề nào sau đây đúng?
A. Nếu a là số tự nhiên thì a là số hữu tỷ không âm.
B. Nếu a là số hữu tỷ không âm thì a là số tự nhiên.
C. Nếu a là số hữu tỷ dương thì a là số tự nhiên.
D. Nếu a không là số tự nhiên thì a không phải số hữu tỉ không âm.

\textbf{{ANSWER}}

Đáp án đúng là: A
Các số hữu tỷ không âm là các số hữu tỷ lớn hơn hoặc bằng 0.
Các số tự nhiên là các số nguyên lớn hơn hoặc bằng 0.
Các số nguyên có thể biểu diễn thành các số hữu tỷ nên nên các tự nhiên là các số hữu tỷ không âm.

========================================================================

https://khoahoc.vietjack.com/thi-online/giai-sbt-toan-10-bai-tap-cuoi-chuong-1-co-dap-an


\textbf{{QUESTION}}

Cho x là một phần tử của tập hợp X. Xét các mệnh đề sau:
(I) x ∈$$ \in $$X;
(II) {x} ∈$$ \in $$ X;
(III) x ⊂$$ \subset $$ X;
(IV) {x} ⊂$$ \subset $$X.
Trong các mệnh đề trên, mệnh đề nào đúng?
A. (I) và (II).
B. (I) và (III).
C. (I) và (IV).
D. (II) và (IV).

\textbf{{ANSWER}}

Đáp án đúng là: C
Khi x là một phần tử của tập hợp X thì x ∈$$ \in $$X và {x}⊂$$ \subset $$ X.

========================================================================

https://khoahoc.vietjack.com/thi-online/de-thi-giua-ki-1-toan-8-co-dap-an/73582


\textbf{{QUESTION}}

1) (x – 4)x – (x – 3)2= 0;
2) 3x – 6 = x2– 16.

\textbf{{ANSWER}}

Hướng dẫn giải
1) (x – 4)x – (x – 3)2= 0
x2– 4x – x2+ 6x – 9 = 0
2x – 9 = 0
2x = 9
x = 4,5
Vậy tập nghiệm của phương trình là S = {4,5}.
2) 3x – 6 = x2– 16
x2– 16 – 3x + 6 = 0
x2– 3x – 10 = 0
x2+ 2x – 5x – 10 = 0
x(x + 2) – 5(x + 2) = 0
(x + 2)(x – 5) = 0
$ \Rightarrow \left[ \begin{array}{l}x + 2 = 0\\x - 5 = 0\end{array} \right.$
$ \Rightarrow \left[ \begin{array}{l}x =  - 2\\x = 5\end{array} \right.$
Vậy tập nghiệm của phương trình là S = {–2; 5}.

========================================================================

https://khoahoc.vietjack.com/thi-online/giai-vth-toan-8-kntt-luyen-tap-chung-trang-25-co-dap-an


\textbf{{QUESTION}}

Cho biểu thức P = 5x(3x2y – 2xy2 + 1) – 3xy(5x2 – 3xy) + x2y2.
a) Bằng cách thu gọn, chứng tỏ rằng giá trị của biểu thức P chỉ phụ thuộc vào biến x mà không phụ thuộc vào biến y.

\textbf{{ANSWER}}

a) Thu gọn P:
P = 5x(3x2y – 2xy2 + 1) – 3xy(5x2 – 3xy) + x2y2
= 15x3y – 10x2y2 + 5x – 15x3y + 9x2y2 + x2y2
= (15x3y – 15x3y) + (9x2y2 + x2y2 – 10x2y2) + 5x
= 5x.
Sau khi thu gọn, ta thấy P = 5x không chứa y. Điều đó chứng tỏ P chỉ phụ thuộc vào biến x mà không phụ thuộc vào biến y.

========================================================================

https://khoahoc.vietjack.com/thi-online/giai-vth-toan-8-kntt-luyen-tap-chung-trang-25-co-dap-an


\textbf{{QUESTION}}

b) Tìm giá trị của x sao cho P = 10.

\textbf{{ANSWER}}

b) P = 10 ⇔ 5x = 10 ⇔ x = 2.

========================================================================

https://khoahoc.vietjack.com/thi-online/giai-vth-toan-8-kntt-luyen-tap-chung-trang-25-co-dap-an


\textbf{{QUESTION}}

Rút gọn biểu thức (3x2 – 5xy – 4y2).(2x2 + y2) + (2x4y2 + x3y3 + x2y4) :  (15xy).$$ \left(\frac{1}{5}xy\right).$$

\textbf{{ANSWER}}

Kí hiệu biểu thức đã cho là P. Ta thấy P = A + B, trong đó:
A = (3x2 – 5xy – 4y2).(2x2 + y2)
= 6x4 + 3x2y2 – 10x3y – 5xy3 – 8x2y2 – 4y4
= 6x4 – 10x3y – 5xy3 – 5x2y2 – 4y4.
B = (2x4y2 + x3y3 + x2y4) :  (15xy)$$ \left(\frac{1}{5}xy\right)$$
= 10x3y + 5x2y2 + 5xy3.
Từ đó ta có
P = A + B = 6x4 – 10x3y – 5xy3 – 5x2y2 – 4y4 + 10x3y + 5x2y2 + 5xy3
= 6x4 – 4y4.

========================================================================

https://khoahoc.vietjack.com/thi-online/giai-vth-toan-8-kntt-luyen-tap-chung-trang-25-co-dap-an


\textbf{{QUESTION}}

Bà Khanh dự định mua x hộp sữa, mỗi hộp giá y đồng. Nhưng khi đến cửa hàng, bà Khanh thấy giá sữa đã giảm 1 500 đồng mỗi hộp nên quyết định mua thêm 3 hộp sữa.
Tìm đa thức biểu thị số tiền bà Khanh phải trả cho tổng số hộp sữa đã mua.

\textbf{{ANSWER}}

Số hộp sữa mà bà Khanh đã mua là x + 3. Giá tiền mỗi hộp sữa giảm 1500 đồng nên giá chỉ còn y – 1500 đồng mỗi hộp.
Do đó đa thức biểu thị số tiền bà Khanh phải trả cho tổng số hộp sữa đã mua là T = (x + 3).(y – 1500).
Vậy đa thức cần tìm là:
T = (x + 3).(y – 1500) = xy – 1500x + 3y – 4500.

========================================================================

https://khoahoc.vietjack.com/thi-online/giai-vth-toan-8-kntt-luyen-tap-chung-trang-25-co-dap-an


\textbf{{QUESTION}}

Tìm hai số a và b sao cho
(5xy – 4y2)(3x2 + 4xy) + ax2y2 – bxy3 = 15xy(x2 – y2).

\textbf{{ANSWER}}

Biến đổi vế phải: 15xy(x2 – y2) = 15x3y – 15xy3. (1)
Biến đổi vế trái: (5xy – 4y2)(3x2 + 4xy) + ax2y2 – bxy3
= 15x3y + 20x2y2 – 12x2y2 – 16xy3 + ax2y2 – bxy3
= 15x3y + (8 + a)x2y2 + (−16 – b)xy3. (2)
So sánh hai đa thức (1) và (2) ta được:
• 8 + a = 0, suy ra a = −8.
• −16 – b = −15, suy ra b = −1.

========================================================================

https://khoahoc.vietjack.com/thi-online/giai-sgk-toan-8-kntt-bai-23-phep-cong-va-phep-tru-phan-thuc-dai-so-co-dap-an


\textbf{{QUESTION}}

Hãy rút gọn biểu thức: $$ P=\frac{x}{x+1}-\left[\left(\frac{1}{x-1}+\frac{x}{x+1}\right)-\frac{1}{x-1}\right]$$
 
Vuông nói: Không cần tính toán, em thấy ngay kết quả là P = 0.
Tròn: Làm thế nào mà vuông thấy ngay được kết quả thế nhỉ?

\textbf{{ANSWER}}

Vuông đã nhìn thấy các dấu đứng trước các phân thức, hai phân thức giống nhau có dấu trái nhau khi cộng lại sẽ bằng 0.

========================================================================

https://khoahoc.vietjack.com/thi-online/giai-sgk-toan-8-kntt-bai-23-phep-cong-va-phep-tru-phan-thuc-dai-so-co-dap-an


\textbf{{QUESTION}}

Hãy thực hiện các yêu cầu sau để làm phép cộng:  $$ \frac{2x+y}{x-y}+\frac{-x+3y}{x-y}$$.
Cộng các tử thức của hai phân thức đã cho.

\textbf{{ANSWER}}

Ta có: 2x + y + (– x + 3y) = x + 4y.

========================================================================

https://khoahoc.vietjack.com/thi-online/giai-sgk-toan-8-kntt-bai-23-phep-cong-va-phep-tru-phan-thuc-dai-so-co-dap-an


\textbf{{QUESTION}}

Viết phân thức có tử thức là tổng các tử thức và mẫu thức chung ta được kết quả của phép cộng đã cho.

\textbf{{ANSWER}}

2x+yx−y+−x+3yx−y=x+4yx−y$$ \frac{2x+y}{x-y}+\frac{-x+3y}{x-y}=\frac{x+4y}{x-y}$$

========================================================================

https://khoahoc.vietjack.com/thi-online/giai-sgk-toan-8-kntt-bai-23-phep-cong-va-phep-tru-phan-thuc-dai-so-co-dap-an


\textbf{{QUESTION}}

Tính các tổng sau:
a) 3x+1xy+2x−1xy$$ \frac{3x+1}{xy}+\frac{2x-1}{xy}$$ ;

\textbf{{ANSWER}}

a, 3x+1xy+2x−1xy=3x+1+2x−1xy=5xxy=5y$$ \frac{3x+1}{xy}+\frac{2x-1}{xy}=\frac{3x+1+2x-1}{xy}=\frac{5x}{xy}=\frac{5}{y}$$

========================================================================

https://khoahoc.vietjack.com/thi-online/giai-sgk-toan-8-kntt-bai-23-phep-cong-va-phep-tru-phan-thuc-dai-so-co-dap-an


\textbf{{QUESTION}}

Tính các tổng sau:
b) 3xx2+1+−3x+1x2+1$$ \frac{3x}{{x}^{2}+1}+\frac{-3x+1}{{x}^{2}+1}$$ .

\textbf{{ANSWER}}

b, 3xx2+1+−3x+1x2+1=3x+(−3x+1)x2+1=3x−3x+1x2+1=1x2+1$$ \frac{3x}{{x}^{2}+1}+\frac{-3x+1}{{x}^{2}+1}=\frac{3x+(-3x+1)}{{x}^{2}+1}=\frac{3x-3x+1}{{x}^{2}+1}=\frac{1}{{x}^{2}+1}$$

========================================================================

https://khoahoc.vietjack.com/thi-online/20-cau-trac-nghiem-toan-10-chan-troi-sang-tao-giai-tam-giac-va-ung-dung-thuc-te-co-dap-an-phan-2/103666


\textbf{{QUESTION}}

Cho ∆ABC thỏa mãn sin2A = sinB.sinC. Khẳng định nào sau đây đúng nhất?

\textbf{{ANSWER}}

Hướng dẫn giải
Đáp án đúng là: C
• Theo hệ quả định lí sin ta có:
$\sin A = \frac{a}{{2R}}$, $\sin B = \frac{b}{{2R}}$ và $\sin C = \frac{c}{{2R}}$.
Ta có sin2A = sinB.sinC.
$ \Leftrightarrow {\left( {\frac{a}{{2R}}} \right)^2} = \frac{b}{{2R}}.\frac{c}{{2R}}$
$ \Leftrightarrow \frac{{{a^2}}}{{{{\left( {2R} \right)}^2}}} = \frac{{bc}}{{{{\left( {2R} \right)}^2}}}$
⇔ a2 = bc.
Do đó phương án A đúng.
• Theo hệ quả của định lí côsin, ta có:
$\cos A = \frac{{{b^2} + {c^2} - {a^2}}}{{2bc}} = \frac{{{b^2} + {c^2} - bc}}{{2bc}}$.
Áp dụng bất đẳng thức Cauchy cho hai số b, c > 0, ta được b2 + c2 ≥ 2bc.
Do đó ta có $\cos A = \frac{{{b^2} + {c^2} - bc}}{{2bc}} \ge \frac{{2bc - bc}}{{2bc}} = \frac{{bc}}{{2bc}} = \frac{1}{2}$.
Vì vậy $\cos A \ge \frac{1}{2}$.
Do đó phương án B đúng.
Vậy ta chọn phương án C.

========================================================================

https://khoahoc.vietjack.com/thi-online/20-cau-trac-nghiem-toan-10-chan-troi-sang-tao-giai-tam-giac-va-ung-dung-thuc-te-co-dap-an-phan-2/103666


\textbf{{QUESTION}}

Cho ∆ABC thỏa mãn $\sin A = \frac{{\sin B + \sin C}}{{\cos B + \cos C}}$. Khi đó ∆ABC là:
$\sin A = \frac{{\sin B + \sin C}}{{\cos B + \cos C}}$

A. Tam giác vuông;
B. Tam giác cân;
C. Tam giác tù;
D. Tam giác đều.

\textbf{{ANSWER}}

Hướng dẫn giải
Đáp án đúng là: A
• Theo hệ quả của định lí côsin, ta có:
$\cos B = \frac{{{a^2} + {c^2} - {b^2}}}{{2ac}}$ và $\cos C = \frac{{{a^2} + {b^2} - {c^2}}}{{2ab}}$.
• Theo hệ quả định lí sin, ta có:
$\sin A = \frac{a}{{2R}};\,\,\sin B = \frac{b}{{2R}};\,\,\sin C = \frac{c}{{2R}}$.
• Ta có $\sin A = \frac{{\sin B + \sin C}}{{\cos B + \cos C}}$
⇔ sinA(cosB + cosC) = sinB + sinC
$ \Leftrightarrow \frac{a}{{2R}}.\left( {\frac{{{a^2} + {c^2} - {b^2}}}{{2ac}} + \frac{{{a^2} + {b^2} - {c^2}}}{{2ab}}} \right) = \frac{b}{{2R}} + \frac{c}{{2R}}$
$ \Leftrightarrow \frac{a}{{2R}}.\frac{1}{{2a}}\left( {\frac{{{a^2} + {c^2} - {b^2}}}{c} + \frac{{{a^2} + {b^2} - {c^2}}}{b}} \right) = \frac{{b + c}}{{2R}}$
$ \Leftrightarrow \frac{1}{2}\left( {\frac{{{a^2} + {c^2} - {b^2}}}{c} + \frac{{{a^2} + {b^2} - {c^2}}}{b}} \right) = b + c$
$ \Leftrightarrow \frac{{b\left( {{a^2} + {c^2} - {b^2}} \right) + c\left( {{a^2} + {b^2} - {c^2}} \right)}}{{bc}} = 2\left( {b + c} \right)$
⇔ a2b + bc2 – b3 + a2c + b2c – c3 = 2b2c + 2bc2
⇔ b3 + c3 – (a2b + a2c) + (b2c + bc2) = 0
⇔ (b + c)(b2 – bc + c2) – a2(b + c) + bc(b + c) = 0
⇔ (b + c)(b2 – bc + c2 – a2 + bc) = 0
⇔ (b + c)(b2 + c2 – a2) = 0
⇔ b + c = 0 (vô lí vì b, c > 0) hoặc b2 + c2 = a2
⇔ AC2 + AB2 = BC2
Áp dụng định lí Pytago đảo, ta được ∆ABC vuông tại A.
Vậy ta chọn phương án A.

========================================================================

https://khoahoc.vietjack.com/thi-online/trac-nghiem-tong-hop-on-thi-tot-nghiep-thpt-mon-toan-chu-de-1-ham-so-va-ung-dung-co-dap-an/145213


\textbf{{QUESTION}}

Bạn hãy đọc đoạn văn 1 trên và trả lời câu hỏi.

\textbf{{ANSWER}}

Chọn đáp án B

========================================================================

https://khoahoc.vietjack.com/thi-online/trac-nghiem-tong-hop-on-thi-tot-nghiep-thpt-mon-toan-chu-de-1-ham-so-va-ung-dung-co-dap-an/145213


\textbf{{QUESTION}}

Bạn hãy đọc đoạn văn 1 trên và trả lời câu hỏi.

\textbf{{ANSWER}}

Chọn đáp án C

========================================================================

https://khoahoc.vietjack.com/thi-online/trac-nghiem-tong-hop-on-thi-tot-nghiep-thpt-mon-toan-chu-de-1-ham-so-va-ung-dung-co-dap-an/145213


\textbf{{QUESTION}}

Bạn hãy đọc đoạn văn 1 trên và trả lời câu hỏi.

\textbf{{ANSWER}}

Chọn đáp án A

========================================================================

https://khoahoc.vietjack.com/thi-online/trac-nghiem-tong-hop-on-thi-tot-nghiep-thpt-mon-toan-chu-de-1-ham-so-va-ung-dung-co-dap-an/145213


\textbf{{QUESTION}}

Bạn hãy đọc đoạn văn 1 trên và trả lời câu hỏi.

\textbf{{ANSWER}}

Chọn đáp án D

========================================================================

https://khoahoc.vietjack.com/thi-online/trac-nghiem-tong-hop-on-thi-tot-nghiep-thpt-mon-toan-chu-de-1-ham-so-va-ung-dung-co-dap-an/145213


\textbf{{QUESTION}}

Bạn hãy đọc đoạn văn 1 trên và trả lời câu hỏi.

\textbf{{ANSWER}}

Chọn đáp án C

========================================================================

https://khoahoc.vietjack.com/thi-online/20-cau-trac-nghiem-toan-10-chan-troi-sang-tao-khai-niem-vecto-co-dap-an-phan-2


\textbf{{QUESTION}}

Vectơ có điểm đầu là A, điểm cuối là B được kí hiệu là?

\textbf{{ANSWER}}

Đáp án đúng là: B
Vectơ có điểm đầu là A, điểm cuối là B được kí hiệu là $$ \overrightarrow{AB}$$.

========================================================================

https://khoahoc.vietjack.com/thi-online/10-bai-tap-tinh-daso-ham-cap-hai-cua-mot-so-ham-don-gian-co-loi-giai


\textbf{{QUESTION}}

Hàm số y = (3x – 5)4 có đạo hàm cấp hai là
A. 36(3x – 5)2;
B. 108(3x – 5)2; 
C. 36(3x – 5)3;

\textbf{{ANSWER}}

Hướng dẫn giải:
Đáp án đúng là: B
Ta có: y'(x) = [(3x – 5)4]' = 4. (3x – 5)' . (3x – 5)3 = 12(3x – 5)3.
Khi đó, y''(x) = [12(3x – 5)3]' = 12 . 3 . (3x – 5)' . (3x – 5)2 = 108(3x – 5)2.

========================================================================

https://khoahoc.vietjack.com/thi-online/10-bai-tap-tinh-daso-ham-cap-hai-cua-mot-so-ham-don-gian-co-loi-giai


\textbf{{QUESTION}}

Hàm số y = $$ \sqrt{\text{x}+1}$$ có đạo hàm cấp hai tại điểm x0 = 0 bằng
A. $$ \frac{1}{4}$$;
B. $$ \frac{-1}{8}$$;
C. $$ \frac{-1}{4}$$;

\textbf{{ANSWER}}

Hướng dẫn giải:
Đáp án đúng là: C
Với mọi x > – 1, ta có y'(x) = $$ {\left(\sqrt{\text{x}+1}\right)}^{\text{'}}$$ = $$ \frac{{\left(\text{x}+1\right)}^{\text{'}}}{2\sqrt{\text{x}+1}}=\frac{1}{2\sqrt{\text{x}+1}}$$.
Khi đó, y''(x) = $$ {\left(\frac{1}{2\sqrt{\text{x}+1}}\right)}^{\text{'}}=-\frac{(2\sqrt{x+1}{)}^{\text{'}}}{{\left(2\sqrt{\text{x}+1}\right)}^{2}}=-\frac{\frac{1}{\sqrt{x+1}}}{4{\left(\sqrt{\text{x}+1}\right)}^{2}}=-\frac{1}{4{\left(\sqrt{\text{x}+1}\right)}^{3}}$$.
Vậy y''(0) =$$ -\frac{1}{4{\left(\sqrt{0+1}\right)}^{3}}=\frac{-1}{4}$$.

========================================================================

https://khoahoc.vietjack.com/thi-online/10-bai-tap-tinh-daso-ham-cap-hai-cua-mot-so-ham-don-gian-co-loi-giai


\textbf{{QUESTION}}

Với mọi x≠π2+kπ  (k∈ℤ)$$ x\ne \frac{\pi }{2}+k\pi \text{\hspace{0.17em}\hspace{0.17em}}\left(k\in \mathbb{Z}\right)$$, đạo hàm cấp hai của hàm số y = tanx là:
A. 2sinxcos3x$$ \frac{2\mathrm{sin}\text{x}}{{\mathrm{cos}}^{3}\text{x}}$$;
B. -2sinxcos3x$$ -\frac{2\mathrm{sin}\text{x}}{{\mathrm{cos}}^{3}\text{x}}$$;
C. 1cos2x$$ \frac{1}{{\mathrm{cos}}^{2}\text{x}}$$;

\textbf{{ANSWER}}

Hướng dẫn giải:
Đáp án đúng là: A
Với mọi x≠π2+kπ  (k∈ℤ)$$ x\ne \frac{\pi }{2}+k\pi \text{\hspace{0.17em}\hspace{0.17em}}\left(k\in \mathbb{Z}\right)$$, ta có y'(x) = (tanx)' = 1cos2x$$ \frac{1}{{\mathrm{cos}}^{2}\text{x}}$$.
Khi đó, y''(x) = (1cos2x)'=−(cos2x)'cos4x=−2.cosx.(cosx)'cos4x=2cosx.sinxcos4x=2sinxcos3x$$ {\left(\frac{1}{{\mathrm{cos}}^{2}\text{x}}\right)}^{\text{'}}=-\frac{{\left({\mathrm{cos}}^{2}\text{x}\right)}^{\text{'}}}{{\mathrm{cos}}^{4}\text{x}}=-\frac{2.\mathrm{cos}\text{x}.(\mathrm{cos}\text{x}{)}^{\text{'}}}{{\mathrm{cos}}^{4}\text{x}}=\frac{2\mathrm{cos}\text{x.}\mathrm{sin}\text{x}}{{\mathrm{cos}}^{4}\text{x}}=\frac{2\mathrm{sin}\text{x}}{{\mathrm{cos}}^{3}\text{x}}$$.

========================================================================

https://khoahoc.vietjack.com/thi-online/175-cau-bai-tap-so-phuc-tu-de-thi-dai-hoc-cuc-hay-co-loi-giai-chi-tiet/13463


\textbf{{QUESTION}}

Cho số phức z = 2 + i. Phần ảo của số phức $$ z\quad =\quad \frac{\overline{)z}\quad +\quad 1}{z\quad -\quad 1}$$ là
A. -2
B. -2i
C. 2
D. 2i

\textbf{{ANSWER}}

Đáp án A

========================================================================

https://khoahoc.vietjack.com/thi-online/de-thi-thu-thpt-quoc-gia-mon-toan-co-chon-loc-va-loi-giai-chi-tiet-20-de/76252


\textbf{{QUESTION}}

Thể tích của khối lăng trụ đều tam giác có mặt bên là hình vuông cạnh a bằng
A. $$ \frac{{a}^{3}\sqrt{3}}{12}$$
B. $$ \frac{{a}^{3}\sqrt{3}}{6}$$
C. $$ \frac{{a}^{3}\sqrt{3}}{4}$$
D. $$ \frac{{a}^{3}\sqrt{3}}{3}$$

\textbf{{ANSWER}}

Đáp án C
Ta có: $$ V=h.{S}_{day}=a.\frac{{a}^{2}\sqrt{3}}{4}=\frac{{a}^{3}\sqrt{3}}{4}$$.

========================================================================

https://khoahoc.vietjack.com/thi-online/10-bai-stap-bai-toan-lai-suat-dan-so-co-loi-giai


\textbf{{QUESTION}}

Bà Hà gửi tiết kiệm ngân hàng Vietcombank số tiền 50 triệu đồng với lãi suất 0,79% một tháng. Tính số tiền cả vốn lẫn lãi bà Hà nhận được sau 2 năm? (làm tròn đến hàng nghìn)

\textbf{{ANSWER}}

Đáp án đúng là: A
Theo bài ra ta có: Số tiền gốc ban đầu P = 50 000 000 đồng;
                            Lãi suất mỗi tháng r = 0,79% = 0,007 9;
                            Số kì hạn N = 2.12 = 24 tháng.
Số tiền cả gốc lẫn lãi bà Hà nhận được sau 2 năm với 0,79% một tháng là
A = 50 000 000.(1 + 0,79%)24 = 50 000 000.( 1 + 0,007 9)24 ≈ 60 393 000 (đồng).

========================================================================

https://khoahoc.vietjack.com/thi-online/10-bai-stap-bai-toan-lai-suat-dan-so-co-loi-giai


\textbf{{QUESTION}}

Bạn Lan gửi 1 500 USD với lãi suất 1,02% một quý. Hỏi sau một năm số tiền lãi bạn Lan nhận được là bao nhiêu USD? (làm tròn đến hàng đơn vị)

\textbf{{ANSWER}}

Đáp án đúng là: B
Theo bài ra ta có: Số tiền gốc ban đầu P = 1 500 USD;
                            Lãi suất mỗi quý (3 tháng) r = 1,02% = 0,0102;
                            Số kì hạn N = 12 : 3 = 4 quý.
Số tiền cả gốc lẫn lãi bạn Lan nhận được sau 1 năm là
A =1 500.(1 + 1,02%)4 = 1 500.( 1 + 0,046)4 ≈ 1 562 (USD).
Số tiền lãi bạn Lan nhận được sau 1 năm là
A – P = 1 562 – 1 500 = 62 (USD).

========================================================================

https://khoahoc.vietjack.com/thi-online/10-bai-stap-bai-toan-lai-suat-dan-so-co-loi-giai


\textbf{{QUESTION}}

Hãy cho biết lãi suất tiết kiệm là bao nhiêu một năm nếu bạn gửi 15,625 triệu đồng, sau 3 năm rút được cả vốn lẫn lãi số tiền là 19,683 triệu đồng?
A. 9%;

\textbf{{ANSWER}}

Đáp án đúng là: B
Theo bài ra ta có: Số tiền gốc ban đầu P = 15,625 triệu đồng
                            Tổng số tiền cả gốc lẫn lãi sau 3 năm A = 19,683 triệu đồng;
                            Số kì hạn N = 3 năm.
Áp dụng công thức tính tổng số tiền cả gốc lẫn lại nhận được sau 3 năm với lãi suất r một năm ta có:
19,683 = 15,625.(1 + r)3 ⇔ (1 + r)3 = 1,259 712 ⇔ 1 + r = 1,08 ⇔ r = 0,08 = 8%.
Vậy lãi suất tiết kiệm là 8% một năm.

========================================================================

https://khoahoc.vietjack.com/thi-online/10-bai-stap-bai-toan-lai-suat-dan-so-co-loi-giai


\textbf{{QUESTION}}

Anh Thành trúng vé số giải thưởng 125 triệu đồng, sau khi trích ra 20% số tiền để chiêu đãi bạn bè và làm từ thiện, anh gửi số tiền còn lại vào ngân hàng với lãi suất 0,31% một tháng. Dự kiến 9 năm sau, anh rút tiền cả vốn lẫn lãi cho con gái vào đại học. Hỏi khi đó anh Thành rút được bao nhiêu tiền? (làm tròn đến hàng nghìn)

\textbf{{ANSWER}}

Đáp án đúng là: D
Số tiền anh Thành gửi vào ngân hàng là 125.80% = 100 (triệu đồng).
Theo bài ra ta có: Số tiền gốc ban đầu P = 100 000 000 đồng;
                            Lãi suất mỗi tháng r = 0,31% = 0,003 1;
                            Số kì hạn N = 9.12 = 108 tháng.
Số tiền cả gốc lẫn lãi anh Thành rút được sau 9 năm là
A = 100 000 000.( 1 + 0,003 1)108 ≈ 139 694 000 (đồng).

========================================================================

https://khoahoc.vietjack.com/thi-online/10-bai-stap-bai-toan-lai-suat-dan-so-co-loi-giai


\textbf{{QUESTION}}

Bà An gửi tiết kiệm 53 triệu đồng theo kì hạn 3 tháng. Sau 2 năm, bà ấy nhận được số tiền cả gốc và lãi là 61 triệu đồng. Hỏi lãi suất ngân hàng là bao nhiêu một tháng (làm tròn đến hàng phần nghìn)?
B. 0,073;

\textbf{{ANSWER}}

Đáp án đúng là: C
Mỗi quý là 3 tháng nên 2 năm có 8 quý.
Theo bài ra ta có: Số tiền gốc ban đầu P = 53 triệu đồng
                            Tổng số tiền cả gốc lẫn lãi sau 8 quý A = 61 triệu đồng;
                            Số kì hạn N = 8 quý.
Áp dụng công thức tính tổng số tiền cả gốc lẫn lại nhận được sau 2 năm (8 quý) với lãi suất r một quý ta có:
61 = 53.(1 + r)8 ⇔ (1 + r)8 = 6153$$ \frac{61}{53}$$
  ⇒1+r=8√6153$$ \Rightarrow 1+r=\sqrt[8]{\frac{61}{53}}$$(do r > 0)
⇒r=8√6153−1$$ \Rightarrow r=\sqrt[8]{\frac{61}{53}}-1$$.
Vì một quý là 3 tháng nên lãi suất một tháng là một tháng là r3=8√6153−13≈0,006$$ \frac{r}{3}=\frac{\sqrt[8]{\frac{61}{53}}-1}{3}\approx 0,006$$ .

========================================================================

https://khoahoc.vietjack.com/thi-online/de-kiem-tra-cuoi-hoc-ki-2-toan-lop-7-ctst-co-dap-an/115998


\textbf{{QUESTION}}

Chọn câu đúng. Với điều kiện các tỉ số đều có nghĩa thì từ $\frac{x}{y} = \frac{u}{v}$ ta có
A. $\frac{x}{y} = \frac{{x + u}}{{y + v}}$;   
B. $\frac{x}{y} = \frac{{x + u}}{{y - v}}$;
C. $\frac{x}{y} = \frac{{x - u}}{{y + v}}$;
D.$\frac{x}{y} = \frac{{x - u}}{{y.v}}$.

\textbf{{ANSWER}}

Đáp án đúng là: A
Theo tính chất của dãy tỉ số bằng nhau ta có: $\frac{x}{y} = \frac{u}{v} = \frac{{x + u}}{{y + v}} = \frac{{x - u}}{{y - v}}$ nên A đúng.

========================================================================

https://khoahoc.vietjack.com/thi-online/200-cau-trac-nghiem-phuong-phap-toa-do-trong-khong-gian/1305


\textbf{{QUESTION}}

Trong không gian với hệ tọa độ Oxyz, cho điểm M(2;-1;4). Gọi H là hình chiếu vuông góc của M lên mặt phẳng (Oxy). Tọa độ điểm H là:
A. H(2;0;4)
B. H(0;-1;4)
C. H(2;-1;0)
D. H(0;-1;0).

\textbf{{ANSWER}}

Đáp án C
Do chiếu xuống (Oxy) nên z=0 và x,y giữ nguyên.

========================================================================

https://khoahoc.vietjack.com/thi-online/200-cau-trac-nghiem-phuong-phap-toa-do-trong-khong-gian/1305


\textbf{{QUESTION}}

Trong không gian với hệ trục toạ độ Oxyz, cho mặt phẳng (α): 2x - 3y - z - 1 = 0. Điểm nào dưới đây không thuộc mặt phẳng (α)?
A. Q(1;2;-5)
B. P(3;1;3)
C. M(-2;1;-8)
D. N(4;2;1).

\textbf{{ANSWER}}

Đáp án B
Lần lượt thay tọa độ các điểm vào phương trình mặt phẳng thì ta nhận thấy điểm P(3;1;3) không thuộc mặt phẳng (α).

========================================================================

https://khoahoc.vietjack.com/thi-online/200-cau-trac-nghiem-phuong-phap-toa-do-trong-khong-gian/1305


\textbf{{QUESTION}}

Trong không gian với hệ tọa độ Oxyz, cho mặt phẳng (P): 3x - 2y + z - 5 = 0. Điểm nào dưới đây thuộc mặt phẳng (P)?
A. N(3;-2;-5)
B. P(0;0;-5)
C. Q(3;-2;1)
D. M(1;1;4).

\textbf{{ANSWER}}

Đáp án D
Ta có 3. 1 - 2. 1 + 4 - 5 = 0 => điểm M thuộc mặt phẳng (P).

========================================================================

https://khoahoc.vietjack.com/thi-online/200-cau-trac-nghiem-phuong-phap-toa-do-trong-khong-gian/1305


\textbf{{QUESTION}}

Trong không gian với hệ trục tọa độ Oxyz, cho điểm M(1;-2;3). Tọa độ điểm A là hình chiếu vuông góc của điểm M lên mặt phẳng (Oyz) là:
A. A(1;-2;0)
B. A(0;-2;3)
C. A(1;-2;3)
D. A(1;0;3).

\textbf{{ANSWER}}

Đáp án B
Điểm nằm trên mặt phẳng Oyz thì có hoành độ bằng 0.
Nên Tọa độ điểm A là hình chiếu vuông góc của điểm M lên mặt phẳng (Oyz) là A(0; -2; 3)

========================================================================

https://khoahoc.vietjack.com/thi-online/200-cau-trac-nghiem-phuong-phap-toa-do-trong-khong-gian/1305


\textbf{{QUESTION}}

Trong không gian Oxyz, cho mặt phẳng (P): x - 2y + 5z - 4 = 0. Điểm nào sau đây thuộc mặt phẳng (P) ?
A. A(11; 2; 3)
B. B(-5; 2; -3)
C. C(5; 2; -1)
D. D(-5;-2;1)

\textbf{{ANSWER}}

Đáp án D
Với D(-5;-2;1), thay vào phương trình (P) , ta có -5 - 2.(-2) + 5.1 - 4 = 0. Suy ra D ∈ (P).

========================================================================

https://khoahoc.vietjack.com/thi-online/tong-hop-bai-tap-duong-tron-toan-6-hay-co-loi-giai


\textbf{{QUESTION}}

Hình ảnh đường tròn trong thực tế là:
A.Vành xe đạp.
B.Đồng xu.
C.Nắp chai.
D.Cái đĩa.

\textbf{{ANSWER}}

Chọn A.
Lời giải chi tiết: 
Vành xe đạp là hình ảnh đường tròn trong thực tế.

========================================================================

https://khoahoc.vietjack.com/thi-online/tong-hop-bai-tap-duong-tron-toan-6-hay-co-loi-giai


\textbf{{QUESTION}}

Hình ảnh hình tròn trong thực tế là:
A.Vành xe đạp. 
B.Vòng tay.
C.Dây buộc tóc.
D. Đồng xu.

\textbf{{ANSWER}}

Đáp án:  Chọn A.
Chỉ có đồng xu là hình ảnh của đường tròn trong thực tế.

========================================================================

https://khoahoc.vietjack.com/thi-online/tong-hop-bai-tap-duong-tron-toan-6-hay-co-loi-giai


\textbf{{QUESTION}}

Đường tròn tâm O. bán kính R là?
A. Hình gồm các điểm cách O một khoảng lớn hơn R. 
B.Hình gồm các điểm cách O một khoảng bằng R. 
C.Là hình gồm các điểm cách O một khoảng nhỏ hơn R.
D.Tất cả các đáp án trên.

\textbf{{ANSWER}}

Chọn B.
Theo định nghĩa SGK (Trang 89): Đường tròn tâm O, bán kính R là hình gồm các điểm cách O một khoảng bằng R.

========================================================================

https://khoahoc.vietjack.com/thi-online/tong-hop-bai-tap-duong-tron-toan-6-hay-co-loi-giai


\textbf{{QUESTION}}

Hình tròn là ?
A. Là đường tròn tâm O, bán kính R.
B. Là hình gồm các điểm nằm trên đường tròn.
C. Là hình gồm các điểm nằm bên trong đường tròn đó.
D. Là hình gồm các điểm nằm trên đường tròn và các điểm nằm bên trong đường tròn đó.

\textbf{{ANSWER}}

ĐÁP ÁN D
Theo định nghĩa SGK (trang 90): Hình tròn là hình gồm các điểm nằm trên đường tròn và các điểm nằm bên trong đường tròn đó.

========================================================================

https://khoahoc.vietjack.com/thi-online/tong-hop-bai-tap-duong-tron-toan-6-hay-co-loi-giai


\textbf{{QUESTION}}

Chọn đáp án để điền vào chỗ trống:
Đường tròn tâm A bán kính R là hình ................. một khoảng ....................
A.... gồm các điểm A ... nhỏ hơn R.
B.... gồm các điểm cách  A ... lớn hơn R.
C. ... gồm các điểm cách A ... bằng R. 
D.... gồm các điểm cách A ... nhỏ hơn R.

\textbf{{ANSWER}}

Đường tròn tâm A bán kính R là hình gồm các điểm cách A một khoảng bằng R.

========================================================================

https://khoahoc.vietjack.com/thi-online/bo-2-de-kiem-tra-giua-hoc-ki-2-toan-10-canh-dieu-co-dap-an


\textbf{{QUESTION}}

Nếu một công việc được hoàn thành bởi một trong hai hành động. Nếu hành động thứ nhất có a cách thực hiện, hành động thứ hai có b cách thực hiện (các cách thực hiện của hai hành động là khác nhau đôi một) thì số cách hoàn thành công việc đó là
A. ab; 
B. a + b; 
C. 1; 
D. $\frac{a}{b}$.

\textbf{{ANSWER}}

Đáp án B

========================================================================

https://khoahoc.vietjack.com/thi-online/bo-2-de-kiem-tra-giua-hoc-ki-2-toan-10-canh-dieu-co-dap-an


\textbf{{QUESTION}}

Nếu một công việc được hoàn thành bởi hai hành động liên tiếp. Nếu hành động thứ nhất có a cách thực hiện, ứng với mỗi cách thực hiện hành động thứ nhất, có b cách thực hiện hành động thứ hai thì số cách hoàn thành công việc đó là
A. ab; 
B. a + b; 
C. 1; 
D. ab$\frac{a}{b}$.

\textbf{{ANSWER}}

Đáp án A

========================================================================

https://khoahoc.vietjack.com/thi-online/bo-2-de-kiem-tra-giua-hoc-ki-2-toan-10-canh-dieu-co-dap-an


\textbf{{QUESTION}}

Một lớp có 31 học sinh nam và 16 học sinh nữ. Có bao nhiêu cách chọn một học sinh làm lớp trưởng của lớp.
A. 31;          
B. 16;          
C. 47;          
D. 15.

\textbf{{ANSWER}}

Đáp án C

========================================================================

https://khoahoc.vietjack.com/thi-online/15-cau-trac-nghiem-toan-7-ket-noi-tri-thuc-bai-27-phep-nhan-da-thuc-mot-bien-co-dap-an/111085


\textbf{{QUESTION}}

Cho biểu thức B = (2x – 3)(x + 7) – 2x(x + 5) – x. Khẳng định nào sau đây là đúng?

\textbf{{ANSWER}}

Đáp án đúng là: B
B = (2x – 3)(x + 7) – 2x(x + 5) – x
= 2x2 + 14x – 3x – 21 – 2x2 – 10x – x
= (2x2 – 2x2) + (14x – 3x – 10x – x) – 21
= −21 < 1
Vậy B < 1.

========================================================================

https://khoahoc.vietjack.com/thi-online/15-cau-trac-nghiem-toan-7-ket-noi-tri-thuc-bai-27-phep-nhan-da-thuc-mot-bien-co-dap-an/111085


\textbf{{QUESTION}}

Với giá trị nào của x thì (x2 – 2x + 5)(x – 2) = (x2 + x)(x – 5)
A. x=57$$ x=\frac{5}{7}$$ ;

\textbf{{ANSWER}}

Đáp án đúng là: A
(x2 – 2x + 5)(x – 2) = (x2 + x)(x – 5)
x3 – 2x2 + 5x – 2x2 + 4x – 10 = x3 + x2 – 5x2 – 5x
(x3 – x3) +(−2x2 – 2x2 – x2 + 5x2) + (5x + 4x + 5x) – 10 = 0
14x = 10
$$ x=\frac{10}{14}=\frac{5}{7}$$.
Vậy $$ x=\frac{5}{7}$$ .

========================================================================

https://khoahoc.vietjack.com/thi-online/15-cau-trac-nghiem-toan-7-ket-noi-tri-thuc-bai-27-phep-nhan-da-thuc-mot-bien-co-dap-an/111085


\textbf{{QUESTION}}

Rút gọn biểu thức sau: 2x(x + 3) – 3x2(x + 2) + 3x(x + 1)

\textbf{{ANSWER}}

Đáp án đúng là:  B
2x(x + 3) – 3x2(x + 2) + 3x(x + 1)
= 2x2 + 6x – 3x3 – 6x2 + 3x2 + 3x
= – 3x3 −x2 + 9x.

========================================================================

https://khoahoc.vietjack.com/thi-online/15-cau-trac-nghiem-toan-7-ket-noi-tri-thuc-bai-27-phep-nhan-da-thuc-mot-bien-co-dap-an/111085


\textbf{{QUESTION}}

Giả sử ba kích thước của một hính hộp chữ nhật là: a; a + 3; a + 1 (cm) với a > 1 cm. Đa thức biểu thị thể tích của hình hộp chữ nhật đó là:

\textbf{{ANSWER}}

Đáp án đúng là: D 
Thể tích của hình hộp chữ nhật là:
a(a + 3)(a + 1)
= (a2 + 3a)(a + 1)
= a3 + 3a2 + a2 + 3a
= a3 + 4a2 + 3a (cm3).
Vậy đa thức biểu thị thể tích của hình hộp chữ nhật là: a3 + 4a2 + 3a (cm3).

========================================================================

https://khoahoc.vietjack.com/thi-online/15-cau-trac-nghiem-toan-7-ket-noi-tri-thuc-bai-27-phep-nhan-da-thuc-mot-bien-co-dap-an/111085


\textbf{{QUESTION}}

Cho hai đa thức f(x) = x + 1 và g(x) = x3 + 3x.
Tính giá trị của h(3) biết h(x) = f(x). g(x).

\textbf{{ANSWER}}

Đáp án đúng là: D 
h(x) = f(x). g(x)
= (x + 1)(x3 + 3x)
= x4 + 3x2 + x3 + 3x
= x4 + x3 + 3x2 + 3
Khi đó:
h(3) = 34 + 33 + 3. 32 + 3
= 138

========================================================================

https://khoahoc.vietjack.com/thi-online/trac-nghiem-bai-tap-toan-9-can-thuc-bac-hai-va-hang-dang-thuc-co-dap-an/66456


\textbf{{QUESTION}}

Tìm giá trị nhỏ nhất của biểu thức: $$ \mathrm{A}=\sqrt{\frac{{\mathrm{x}}^{2}}{9}+2\mathrm{x}+10}$$
Đáp số: $$ {\mathrm{A}}_{\mathrm{min}}$$ = … khi x = …x

\textbf{{ANSWER}}

Hướng dẫn
Bước 1: Biến đổi biểu thức A về dạng $$ \mathrm{A}=\sqrt{{\left[\mathrm{f}\left(\mathrm{x}\right)\right]}^{2}+\mathrm{a}}$$
Bước 2: Đánh giá và tìm giá trị nhỏ nhất của A
Lời giải
Ta có: $$ \mathrm{A}=\sqrt{\frac{{\mathrm{x}}^{2}}{9}+2\mathrm{x}+10}$$$$ =\sqrt{{\left(\frac{\mathrm{x}}{3}\right)}^{2}+2.\frac{\mathrm{x}}{3}.3+{3}^{2}+1}$$$$ =\sqrt{{\left(\frac{\mathrm{x}}{3}+3\right)}^{2}+1}$$
Vì $$ {\left(\frac{\mathrm{x}}{3}+3\right)}^{2}\ge 0,\forall \mathrm{x}$$ $$ \Rightarrow \sqrt{{\left(\frac{\mathrm{x}}{3}+3\right)}^{2}+1}\ge \sqrt{1}=1,\forall \mathrm{x}$$
Suy ra, $$ {\mathrm{A}}_{\mathrm{min}}$$ = 1$$ \Leftrightarrow {\left(\frac{\mathrm{x}}{3}+3\right)}^{2}=0\Leftrightarrow \mathrm{x}=-9$$
Vậy cần điền vào chỗ chấm để $$ {\mathrm{A}}_{\mathrm{min}}$$ = 1 khi x = −9

========================================================================

https://khoahoc.vietjack.com/thi-online/11-cau-trac-nghiem-toan-9-bai-1-phuong-trinh-bac-nhat-hai-an-co-dap-an-thong-hieu


\textbf{{QUESTION}}

Phương trình nào dưới đây nhận cặp số (−2; 4) làm nghiệm?
A. x – 2y = 0
B. 2x + y = 0
C. x – y = 2
D. x + 2y + 1 = 0

\textbf{{ANSWER}}

Đáp án B
Thay x = −2; y = 4 vào từng phương trình ta được:
+) x – 2y = −2 – 2.4 = −10 $$ \ne $$ 0 nên loại A
+) x – y = −2 – 4 = −6 $$ \ne $$ 0 nên loại C
+) x + 2y + 1 = −2 + 2.4 + 1 = 7 $$ \ne $$ 0 nên loại D
+) 2x + y = −2.2 + 4 = 0 nên B đúng

========================================================================

https://khoahoc.vietjack.com/thi-online/11-cau-trac-nghiem-toan-9-bai-1-phuong-trinh-bac-nhat-hai-an-co-dap-an-thong-hieu


\textbf{{QUESTION}}

Phương trình nào dưới đây nhận cặp số (−3; −2) làm nghiệm?
A. x + y = 2
B. 2x + y = 1
C. x – 2y = 1
D. 5x + 2y + 12 = 0

\textbf{{ANSWER}}

Đáp án C
Thay x = −3; y = −2 vào từng phương trình ta được
+) x + y = −3 + (−2) = −5 ≠$$ \ne $$ 2 nên loại A
+) 2x + y = 2.(−3) + (−2) = −8 ≠$$ \ne $$ 1 nên loại B
+) x – 2y = −3 – 2.(−2) = 1 nên chọn C
+) 5x + 2y + 12 = 5. (−3) + 2.(−2) + 12 = −7 nên loại D

========================================================================

https://khoahoc.vietjack.com/thi-online/11-cau-trac-nghiem-toan-9-bai-1-phuong-trinh-bac-nhat-hai-an-co-dap-an-thong-hieu


\textbf{{QUESTION}}

Phương trình x – 5y + 7 = 0 nhận cặp số nào sau đây làm nghiệm?
A. (0; 1)
B. (−1; 2)
C. (3; 2)
D. (2; 4)

\textbf{{ANSWER}}

Đáp án C
+) Thay x = 0; y = 1 vào phương trình x – 5y + 7 = 0 ta được
0 −5.1 + 7 = 0 $$ \Leftrightarrow $$ 2 = 0 (vô lý) nên loại A
+) Thay x = −1; y = 2 vào phương trình x – 5y + 7 = 0 ta được
−1 – 5.2 + 7 = 0 −4 = 0 (vô lý) nên loại B
+) Thay x = 2; y = 4 vào phương trình x – 5y + 7 = 0 ta được
2 – 5.4 + 7 = 0 $$ \Leftrightarrow $$ −11 = 0 (vô lý) nên loại D
+) Thay x = 3; y = 2 vào phương trình x – 5y + 7 = 0 ta được
3 – 5.2 + 7 = 0 (luôn đúng) nên chọn C

========================================================================

https://khoahoc.vietjack.com/thi-online/11-cau-trac-nghiem-toan-9-bai-1-phuong-trinh-bac-nhat-hai-an-co-dap-an-thong-hieu


\textbf{{QUESTION}}

Phương trình 5x + 4y = 8 nhận cặp số nào sau đây làm nghiệm?
A. (−2; 1)
B. (−1; 0)
C. (1,5; 3)
D. (4; −3)

\textbf{{ANSWER}}

Đáp án D
Xét phương trình 5x + 4y = 8
Cặp số (−2; 1) không phải nghiệm của phương trình vì 5 (−2) + 4.1 = −6. Do đó loại A
Cặp số (−1; 0) không phải nghiệm của phương trình vì 5.(−1) + 4.0 = −5. Do đó loại B
Cặp số (1,5; 3) không phải nghiệm của phương trình vì 5.1,5 + 4.3 = 19,5. Do đó loại C
Cặp số (4; −3) là nghiệm của phương trình vì 5.4 + 4.(−3) = 8. Do đó chọn D

========================================================================

https://khoahoc.vietjack.com/thi-online/11-cau-trac-nghiem-toan-9-bai-1-phuong-trinh-bac-nhat-hai-an-co-dap-an-thong-hieu


\textbf{{QUESTION}}

Trong các cặp số (0; 2), (−1; −8), (1; 1), (3; 2), (1; −6) có bao nhiêu cặp số là nghiệm của phương trình 3x – 2y = 13
A. 1
B. 2
C. 3
D. 4

\textbf{{ANSWER}}

Đáp án A
Thay từng cặp số vào phương trình ta thấy
Ta thấy có cặp số (−1; −8) thỏa mãn phương trình (vì 3.(−1) – 2.(−8) = 13

========================================================================

https://khoahoc.vietjack.com/thi-online/bai-tap-quy-tac-chuyen-ve-chon-loc-co-dap-an


\textbf{{QUESTION}}

Nếu a + c = b + c thì:
A. a = b     
B. a < b     
C. a > b     
D. Cả A, B, C đều sai.

\textbf{{ANSWER}}

Đáp án là A
Ta có: Nếu a + c = b + c thì a = b

========================================================================

https://khoahoc.vietjack.com/thi-online/bai-tap-quy-tac-chuyen-ve-chon-loc-co-dap-an


\textbf{{QUESTION}}

Cho b ∈ Z và b - x = -9. Tìm x
A. -9 - b     
B. -9 + b    
C. b + 9     
D. -b + 9

\textbf{{ANSWER}}

Đáp án là C
Ta có: b - x = -9
     ⇔ -x = -9 - b
     ⇔ x = 9 + b

========================================================================

https://khoahoc.vietjack.com/thi-online/bai-tap-quy-tac-chuyen-ve-chon-loc-co-dap-an


\textbf{{QUESTION}}

Tìm x biết x + 7 = 4
A. x = -3     
B. x = 11     
C. x = -11    
D. x = 3

\textbf{{ANSWER}}

Đáp án là A
Ta có: x + 7 = 4
     ⇔ x = 4 - 7
     ⇔ x = -3

========================================================================

https://khoahoc.vietjack.com/thi-online/bai-tap-quy-tac-chuyen-ve-chon-loc-co-dap-an


\textbf{{QUESTION}}

Số nguyên x nào dưới đây thỏa mãn x - 8 = 20
A. x = 12     
B. x = 28     
C. x = 160     
D. x = -28

\textbf{{ANSWER}}

Đáp án là B
Ta có: x - 8 = 20
     ⇔ x = 20 + 8
     ⇔ x = 28

========================================================================

https://khoahoc.vietjack.com/thi-online/bai-tap-quy-tac-chuyen-ve-chon-loc-co-dap-an


\textbf{{QUESTION}}

Có bao nhiêu số nguyên x sao cho x + 90 = 198
A. 0    
B. 3     
C. 2    
D. 1

\textbf{{ANSWER}}

Đáp án là D
Ta có: x + 90 = 198
     ⇔ x = 198 - 90
     ⇔ x = 108
Vậy có 1 số nguyên thỏa mãn bài toán.

========================================================================

https://khoahoc.vietjack.com/thi-online/giai-sbt-toan-12-tap-2-kntt-bai-18-xac-suat-co-dieu-kien-co-dap-an


\textbf{{QUESTION}}

Cho P(A) = $\frac{2}{5}$; P(B) = $\frac{1}{3}$; P(A ∪ B) = $\frac{1}{2}$. Tính P(A | B) và P(B | A).

\textbf{{ANSWER}}

Ta có: P(AB) = P(A) + P(B) – P(A ∪ B) = $\frac{2}{5} + \frac{1}{3} - \frac{1}{2}$ = $\frac{7}{{30}}$.
Từ đó, ta có: P(A | B) = $\frac{{P(AB)}}{{P(B)}} = \frac{7}{{30}}:\frac{1}{3} = \frac{7}{{10}}$.
                     P(B | A) = $\frac{{P\left( {AB} \right)}}{{P\left( A \right)}} = \frac{7}{{30}}:\frac{2}{5} = \frac{7}{{12}}$.

========================================================================

https://khoahoc.vietjack.com/thi-online/giai-sbt-toan-12-tap-2-kntt-bai-18-xac-suat-co-dieu-kien-co-dap-an


\textbf{{QUESTION}}

Một túi đựng 5 viên bi đỏ và 3 viên xanh. Sơn lấy ngẫu nhiên một viên bi đưa cho Tùng rồi Tùng lấy ngẫu nhiên tiếp một viên bi. Tính xác suất để hai viên bi lấy ra có ít nhất một viên bi đỏ.

\textbf{{ANSWER}}

Gọi E là biến cố: “Trong hai viên bi lấy ra có ít nhất một viên bi màu đỏ”.
       ¯E$\overline E $là biến cố: “Cả hai viên bi rút ra đều là viên bi xanh”.
Gọi A là biến cố: “Sơn lấy được viên bi xanh”;
       B là biến cố: “Tùng lấy được viên bi xanh”.
Ta có: P(¯E$\overline E $) = P(AB).
P(A) = 33+5=38$\frac{3}{{3 + 5}} = \frac{3}{8}$; P(B | A) = C12C17=27$\frac{{C_2^1}}{{C_7^1}} = \frac{2}{7}$.
P(¯E$\overline E $) = P(AB) = P(BA) = P(A). P(B | A) = 38.27=328$\frac{3}{8}.\frac{2}{7} = \frac{3}{{28}}$.
So đó P(E) = 1 – P(¯E$\overline E $) = 1 − 328$\frac{3}{{28}}$ = 2528$\frac{{25}}{{28}}$.

========================================================================

https://khoahoc.vietjack.com/thi-online/giai-sbt-toan-12-tap-2-kntt-bai-18-xac-suat-co-dieu-kien-co-dap-an


\textbf{{QUESTION}}

Một hộp chứa 20 tấm thẻ đánh số {1; 2;…; 20}. Nam rút ngẫu nhiên một tấm thẻ đưa cho Hà rồi Hà rút ngẫu nhiên tiếp một tấm thẻ. Tính xác suất để cả hai thẻ Hà nhận được đều ghi số nguyên tố”.

\textbf{{ANSWER}}

Gọi A là biến cố: “Nam rút được thẻ mang số nguyên tố”.
       B là biến cố: “Hà rút được thẻ mang số nguyên tố”.
Trong hộp có 8 tấm thẻ ghi số nguyên tố là: {2; 3; 5; 7; 11; 13; 17; 19}, suy ra n(A) = 8.
Nếu A xảy ra thì trong hộp chỉ còn 19 thẻ với 7 thẻ ghi số nguyên tố. Do đó:
P(A) = 820$\frac{8}{{20}}$; P(B | A) = 719$\frac{7}{{19}}$.
Vậy P(AB) = 820$\frac{8}{{20}}$.719$\frac{7}{{19}}$ = 1495$\frac{{14}}{{95}}$ ≈ 0,1473.

========================================================================

https://khoahoc.vietjack.com/thi-online/giai-sbt-toan-12-tap-2-kntt-bai-18-xac-suat-co-dieu-kien-co-dap-an


\textbf{{QUESTION}}

Một hộp chứa 17 viên bi đỏ, 13 viên bi xanh. An lấy ngẫu nhiên một viên bi đưa cho Bình rồi Bình lấy ngẫu nhiên tiếp một viên bi. Tính xác suất để hai viên bi Bình nhận được:
a) Đều là bi đỏ;
b) Là hai viên bi khác nhau.

\textbf{{ANSWER}}

a) Gọi A là biến cố: “An lấy được viên bi màu đỏ”.
           B là biến cố: “Bình lấy được viên bi màu đỏ”.
Do đó, ta có: P(AB) là xác suất hai viên bi Bình được đều là màu đỏ.
Ta có: Không gian mẫu là: n(Ω) = 17 + 13 = 30.
P(A) = 1730$\frac{{17}}{{30}}$; P(B | A) = 1629$\frac{{16}}{{29}}$.
Vậy P(AB) = 1730$\frac{{17}}{{30}}$.1629$\frac{{16}}{{29}}$ = 136435$\frac{{136}}{{435}}$ ≈ 0,3126.
b) Gọi ¯A$\overline A $ là biến cố: “An lấy được viên bi màu xanh”.
            ¯B$\overline B $ là biến cố: “Bình lấy được viên bi màu xanh”.
Ta có: P(¯A$\overline A $) = 1330$\frac{{13}}{{30}}$; P(¯B$\overline B $) = 1229$\frac{{12}}{{29}}$.
Xác suất để cả hai lần lấy đều được viên bi màu xanh là: 1330$\frac{{13}}{{30}}$.1229$\frac{{12}}{{29}}$ = 26145$\frac{{26}}{{145}}$.
Xác suất để cả hai lần lấy được viên bi màu đỏ là: 136435$\frac{{136}}{{435}}$.
Như vậy, xác suất để hai lần lấy được 2 viên bi khác màu là:
1 – 136435$\frac{{136}}{{435}}$ − 26145$\frac{{26}}{{145}}$ = 221435$\frac{{221}}{{435}}$ ≈ 0,508.

========================================================================

https://khoahoc.vietjack.com/thi-online/giai-sbt-toan-12-tap-2-kntt-bai-18-xac-suat-co-dieu-kien-co-dap-an


\textbf{{QUESTION}}

Cho hai biến cố A và B với P(A) > 0, P(B) > 0. Chứng minh rằng nếu P(AB) = P(A).P(B) thì A và B độc lập.

\textbf{{ANSWER}}

Giả sử: P(AB) = P(A).P(B) với P(A) > 0, P(B) > 0.
Ta có: P(A | B) = P(AB)P(B)=P(A).P(B)P(B)$\frac{{P\left( {AB} \right)}}{{P\left( B \right)}} = \frac{{P\left( A \right).P\left( B \right)}}{{P\left( B \right)}}$ = P(A);
           P(B | A) = P(AB)P(A)=P(A).P(B)P(A)$\frac{{P\left( {AB} \right)}}{{P\left( A \right)}} = \frac{{P\left( A \right).P\left( B \right)}}{{P\left( A \right)}}$ = P(B).
Vậy P(A | B) = P(A), P(B | A) = P(B).
Từ đó, việc xảy ra biến cố B không ảnh hưởng tới xác suất xảy ra của biến cố A và ngược lại.
Do đó, A và B độc lập.

========================================================================

https://khoahoc.vietjack.com/thi-online/bai-tap-toan-9-chu-de-8-bat-dang-thuc-co-dap-an/110067


\textbf{{QUESTION}}

Cho 2 số thực dương a, b thỏa $$ a+b\le 1$$ .. Tìm GTNN của $$ A=a+b+\frac{1}{a}\text{ }+\frac{1}{b}$$

\textbf{{ANSWER}}

$$ A=\left(4a+4b+\frac{1}{a}+\frac{1}{b}\right)-3a-3b\ge 4\sqrt[4]{4a\mathrm{..4}b.\frac{1}{a}.\frac{1}{b}}-3\left(a+b\right)\ge 8-3=5$$
Dấu “=” xảy ra $$ \Leftrightarrow a=b=\frac{1}{2}\text{  }$$. Vậy GTNN của A là 5

========================================================================

https://khoahoc.vietjack.com/thi-online/bai-tap-toan-9-chu-de-8-bat-dang-thuc-co-dap-an/110067


\textbf{{QUESTION}}

Cho 3 số thực dương a, b, c  thỏa a+b+c≤32$$ a+b+c\le \frac{3}{2}$$ . 
Tìm GTNN của A=a2+b2+c2+1a +1b+1c$$ A={a}^{2}+{b}^{2}+{c}^{2}+\frac{1}{a}\text{ }+\frac{1}{b}+\frac{1}{c}$$

\textbf{{ANSWER}}

$$ \begin{array}{c}A=\left({a}^{2}+{b}^{2}+{c}^{2}+\frac{1}{8a}+\frac{1}{8b}+\frac{1}{8c}+\frac{1}{8a}+\frac{1}{8b}+\frac{1}{8c}\right)+\frac{3}{4a}+\frac{3}{4b}+\frac{3}{4c}\\ \ge 9\sqrt[9]{{a}^{2}.{b}^{2}.{c}^{2}.\frac{1}{8a}.\frac{1}{8b}.\frac{1}{8c}.\frac{1}{8a}.\frac{1}{8b}.\frac{1}{8c}}+\frac{3}{4}\left(\frac{1}{a}+\frac{1}{b}+\frac{1}{c}\right)\\ \ge \frac{9}{4}+9.\frac{1}{\sqrt[3]{abc}}\ge \frac{9}{4}+\frac{9}{4}.\frac{1}{\frac{a+b+c}{3}}\ge \frac{9}{4}+\frac{9}{4}.2=\frac{27}{4}\end{array}$$
Dấu “=” xảy ra $$ \Leftrightarrow a=b=c=\frac{1}{2}\text{  }$$
 Vậy GTNN của A là   $$ \frac{27}{4}$$

========================================================================

https://khoahoc.vietjack.com/thi-online/bai-tap-toan-9-chu-de-8-bat-dang-thuc-co-dap-an/110067


\textbf{{QUESTION}}

Cho 2 số thực dương a, b. Tìm GTNN của A=a+b√ab +√aba+b$$ A=\frac{a+b}{\sqrt{ab}}\text{ }+\frac{\sqrt{ab}}{a+b}$$

\textbf{{ANSWER}}

A=(a+b4√ab +√aba+b)+3(a+b)4√ab≥2√a+b4√ab .√aba+b +3.2√ab4√ab=1+32=52$$ A=\left(\frac{a+b}{4\sqrt{ab}}\text{ }+\frac{\sqrt{ab}}{a+b}\right)+\frac{3\left(a+b\right)}{4\sqrt{ab}}\ge \text{2}\sqrt{\frac{a+b}{4\sqrt{ab}}\text{ }.\frac{\sqrt{ab}}{a+b}}\text{ }+\frac{3.2\sqrt{ab}}{4\sqrt{ab}}=1+\frac{3}{2}=\frac{5}{2}$$
Dấu “=” xảy ra ⇔a=b  $$ \Leftrightarrow a=b\text{  }$$
 
            Vậy GTNN của A là    52$$ \frac{5}{2}$$

========================================================================

https://khoahoc.vietjack.com/thi-online/bai-tap-toan-9-chu-de-8-bat-dang-thuc-co-dap-an/110067


\textbf{{QUESTION}}

Cho 3 số thực dương a, b, c. 
Tìm GTNN của A=ab+c+bc+a+ca+b+b+ca +c+ab+a+bc$$ A=\frac{a}{b+c}+\frac{b}{c+a}+\frac{c}{a+b}+\frac{b+c}{a}\text{ }+\frac{c+a}{b}+\frac{a+b}{c}$$

\textbf{{ANSWER}}

A=(ab+c+bc+a+ca+b+b+c4a +c+a4b+a+b4c)+34(b+ca +c+ab+a+bc)≥66√ab+c.bc+a.ca+b.b+c4a .c+a4b.a+b4c+34(ba +ca+cb+ab+ac+bc)$$ \begin{array}{c}A=\left(\frac{a}{b+c}+\frac{b}{c+a}+\frac{c}{a+b}+\frac{b+c}{4a}\text{ }+\frac{c+a}{4b}+\frac{a+b}{4c}\right)+\frac{3}{4}\left(\frac{b+c}{a}\text{ }+\frac{c+a}{b}+\frac{a+b}{c}\right)\\ \ge 6\sqrt[6]{\frac{a}{b+c}.\frac{b}{c+a}.\frac{c}{a+b}.\frac{b+c}{4a}\text{ }.\frac{c+a}{4b}.\frac{a+b}{4c}}+\frac{3}{4}\left(\frac{b}{a}\text{ }+\frac{c}{a}+\frac{c}{b}+\frac{a}{b}+\frac{a}{c}+\frac{b}{c}\right)\end{array}$$
≥3+34.6.6√ba .ca.cb.ab.ac.bc=3+92=152$$ \ge 3+\frac{3}{4}\mathrm{.6.}\sqrt[6]{\frac{b}{a}\text{ }.\frac{c}{a}.\frac{c}{b}.\frac{a}{b}.\frac{a}{c}.\frac{b}{c}}=3+\frac{9}{2}=\frac{15}{2}$$
Dấu “=” xảy ra ⇔a=b=c $$ \Leftrightarrow a=b=c\text{ }$$
Vậy GTNN của A là   152$$ \frac{15}{2}$$

========================================================================

https://khoahoc.vietjack.com/thi-online/bai-tap-toan-9-chu-de-8-bat-dang-thuc-co-dap-an/110067


\textbf{{QUESTION}}

Cho 2 số thực dương a, b thỏa a+b≤1$$ a+b\le 1$$  .

\textbf{{ANSWER}}

A=1a2+b2 +12ab≥2√1(a2+b2)2ab ≥2.1a2+b2+2ab2=4(a+b)2≥4$$ A=\frac{1}{{a}^{2}+{b}^{2}}\text{ }+\frac{1}{2ab}\ge 2\sqrt{\frac{1}{\left({a}^{2}+{b}^{2}\right)2ab}\text{ }}\ge 2.\frac{1}{\frac{{a}^{2}+{b}^{2}+2ab}{2}}=\frac{4}{{\left(a+b\right)}^{2}}\ge 4$$
Dấu “=” xảy ra ⇔{a2+b2=2aba+b=1⇔a=b=1 2$$ \Leftrightarrow \left\{\begin{array}{l}{a}^{2}+{b}^{2}=2ab\\ a+b=1\end{array}\right.\Leftrightarrow a=b=\frac{1\text{ }}{2}$$
Vậy GTNN của A là 4

========================================================================

https://khoahoc.vietjack.com/thi-online/20-cau-trac-nghiem-toan-8-bai-6-truong-hop-dong-dang-thu-hai-co-dap-an


\textbf{{QUESTION}}

Hãy chọn câu đúng. Nếu ΔABC và ΔDEF có $$ \hat{B}=\hat{D}$$; $$ \frac{BA}{BC}=\frac{DE}{DF}$$  thì:
A. ΔABC đồng dạng với ΔDEF
B. ΔABC đồng dạng với ΔEDF
C. ΔBCA đồng dạng với ΔDEF
D. ΔABC đồng dạng với ΔFDE

\textbf{{ANSWER}}

ΔABC và ΔDEF có $$ \hat{B}=\hat{D}$$; $$ \frac{BA}{BC}=\frac{DE}{DF}$$ thì ΔABC đồng dạng với ΔEDF
 Đáp án: B

========================================================================

https://khoahoc.vietjack.com/thi-online/20-cau-trac-nghiem-toan-8-bai-6-truong-hop-dong-dang-thu-hai-co-dap-an


\textbf{{QUESTION}}

Cho ΔABC và ΔDEF có góc B = D;BABC=DEDF$$ \frac{BA}{BC}=\frac{DE}{DF}$$  , chọn kết luận đúng:
A. ΔABC ~ ΔDEF
B. ΔABC ~ ΔEDF
C. ΔBAC ~ ΔDFE
D. ΔABC ~ ΔFDE

\textbf{{ANSWER}}

ΔABC và ΔDEF có góc B = D; BABC=DEDF$$ \frac{BA}{BC}=\frac{DE}{DF}$$  thì ΔABC đồng dạng với ΔEDF
 Đáp án: B

========================================================================

https://khoahoc.vietjack.com/thi-online/29-cau-trac-nghiem-cac-quy-tac-tinh-dao-ham-co-dap-an


\textbf{{QUESTION}}

Đạo hàm của hàm số$$ \quad y=\quad 3{x}^{5}-2{x}^{4}$$ tại x=-1, bằng:
A. 23
B. 7
C. -7
D. -23

\textbf{{ANSWER}}

y'=15x4-8x3
y’(-1)= 15+8=23
Chọn đáp án A

========================================================================

https://khoahoc.vietjack.com/thi-online/bai-tadddp-lap-phuong-trinh-chinh-tac-cua-elip-co-loi-giai


\textbf{{QUESTION}}

Trong mặt phẳng tọa độ Oxy, phương trình chính tắc của elip đi qua điểm (5; 0) và có tiêu cự bằng $$ 2\sqrt{5}$$ là
A. $$ \frac{{x}^{2}}{25}+\frac{{y}^{2}}{5}=1$$
B. $$ \frac{{x}^{2}}{25}+\frac{{y}^{2}}{20}=1$$;
C. $$ \frac{{x}^{2}}{25}-\frac{{y}^{2}}{5}=1$$

\textbf{{ANSWER}}

Hướng dẫn giải:
Đáp án đúng là: B
Phương trình chính tắc của elip có dạng là $$ \frac{{x}^{2}}{{a}^{2}}+\frac{{y}^{2}}{{b}^{2}}=1$$ (a > b > 0).
Do elip đi qua điểm (5; 0) nên ta có $$ \frac{{5}^{2}}{{a}^{2}}+\frac{{0}^{2}}{{b}^{2}}=1$$ hay $$ \frac{25}{{a}^{2}}=1,$$ do đó a2 = 25.
Do elip có tiêu cự bằng $$ 2\sqrt{5}$$ nên ta có $$ 2c=2\sqrt{5},$$ suy ra $$ c=\sqrt{5}$$ Do đó c = 5.
Ta có c2 = a2 – b2 nên b2 = a2 – c2 = 25 – 5 = 20.
Vậy phương trình chính tắc của elip là $$ \frac{{x}^{2}}{25}+\frac{{y}^{2}}{20}=1$$.

========================================================================

https://khoahoc.vietjack.com/thi-online/bai-tap-tam-giac-can-co-dap-an


\textbf{{QUESTION}}

Chọn câu sai
A. Tam giác đều có ba góc bằng nhau và bằng $$ {60}^{o}$$
B. Tam giác đều có ba cạnh bằng nhau.
C. Tam giác cân là tam giác đều.
D. Tam giác đều là tam giác cân.

\textbf{{ANSWER}}

Tam giác đều là tam giác có ba cạnh bằng nhau
Trong tam giác đều, mỗi góc bằng $$ {60}^{o}$$
Tam giác đều cũng là tam giác cân nhưng tam giác cân chưa chắc là tam giác đều
Chọn đáp án C.

========================================================================

https://khoahoc.vietjack.com/thi-online/bai-tap-tam-giac-can-co-dap-an


\textbf{{QUESTION}}

Hai góc nhọn của tam giác vuông cân bằng
A. 300$$ {30}^{0}$$
B. 450$$ {45}^{0}$$
C. 600$$ {60}^{0}$$
D. 900$$ {90}^{0}$$

\textbf{{ANSWER}}

Mỗi góc nhọn của tam giác vuông cân bằng $$ {45}^{o}$$
Đáp án B

========================================================================

https://khoahoc.vietjack.com/thi-online/10-bai-tap-tinh-gia-tri-cua-bieu-thuc-lien-quan-den-cac-gia-tri-luong-giac


\textbf{{QUESTION}}

Cho x = 30°. Khi đó giá trị của biểu thức A = sin 2x – 3cos x là:
A. 3;
B. $$ \sqrt{3}$$;
C. $$ -\sqrt{3}$$;

\textbf{{ANSWER}}

Hướng dẫn giải
Đáp án đúng là: C
Thay x = 30° vào biểu thức A ta có:
A = sin (2.30°) – 3cos 30° = sin 60° – 3cos 30° = $$ \frac{\sqrt{3}}{2}$$ – 3.$$ \frac{\sqrt{3}}{2}$$ = – $$ \sqrt{3}$$.

========================================================================

https://khoahoc.vietjack.com/thi-online/10-bai-tap-tinh-gia-tri-cua-bieu-thuc-lien-quan-den-cac-gia-tri-luong-giac


\textbf{{QUESTION}}

Cho α + β = π. Khi đó biểu thức A = sin2 (π – β) + cos2 (π – α) là:
A. 1;
B. 2;
C. –1;

\textbf{{ANSWER}}

Hướng dẫn giải
Đáp án đúng là: A
Ta có: α + β = π ⇒ α = π – β.
Do đó A = sin2 (π – β) + cos2 (π – α) = (sin α)2 + (– cos α)2 = sin2 α + cos2 α = 1.

========================================================================

https://khoahoc.vietjack.com/thi-online/10-bai-tap-tinh-gia-tri-cua-bieu-thuc-lien-quan-den-cac-gia-tri-luong-giac


\textbf{{QUESTION}}

Cho sin x = 14$$ \frac{1}{4}$$. Biểu thức A = 43sin2α$$ \frac{4}{3}{\mathrm{sin}}^{2}\alpha $$ + cos2α$$ c{\text{os}}^{2}\alpha $$ = ab$$ \frac{a}{b}$$ (với (a, b) = 1). Khi đó giá trị của a – b là:
A. 2;
B. 1;
C. 3;

\textbf{{ANSWER}}

Hướng dẫn giải
Đáp án đúng là: B
Ta có: sin2 x + cos2 x = 1 ⇒ cos2 x = 1 – sin2 x = 1−116=1516$$ 1-\frac{1}{16}=\frac{15}{16}$$.
Do đó A = 43$$ \frac{4}{3}$$.116$$ \frac{1}{16}$$ + 1516$$ \frac{15}{16}$$ = 4948$$ \frac{49}{48}$$.
Như vậy a = 49, b = 48 ⇒ a – b = 1.

========================================================================

https://khoahoc.vietjack.com/thi-online/12-bai-tap-tinh-gia-tri-bieu-thuc-luong-giac-co-loi-giai


\textbf{{QUESTION}}

Giá trị của biểu thức A = sin61° − cos29° là
A. 0.
B. 1.
C. 2.
D. $\frac{1}{2}$.

\textbf{{ANSWER}}

Đáp án đúng là: A
Ta có: A = sin61° − cos29° = sin61° − sin61° = 0.

========================================================================

https://khoahoc.vietjack.com/thi-online/12-bai-tap-tinh-gia-tri-bieu-thuc-luong-giac-co-loi-giai


\textbf{{QUESTION}}

Giá trị của biểu thức B = cos15° − sin75° là
A. 0.
B. 1.
C. 2.
D. $\frac{1}{2}$.

\textbf{{ANSWER}}

Đáp án đúng là: A
Ta có: B = cos15° − sin75° = cos15° − cos15° = 0.

========================================================================

https://khoahoc.vietjack.com/thi-online/12-bai-tap-tinh-gia-tri-bieu-thuc-luong-giac-co-loi-giai


\textbf{{QUESTION}}

Giá trị của biểu thức A = tan45°.cos30°.cot30° là
A. 1.
B. 32$\frac{3}{2}$.
C. 72$\frac{7}{2}$.
D. 4.

\textbf{{ANSWER}}

Đáp án đúng là: B
Ta có: A = tan45°.cos30°.cot30° = 1.√32$\frac{{\sqrt 3 }}{2}$.√3$\sqrt 3 $ = 32$\frac{3}{2}$.

========================================================================

https://khoahoc.vietjack.com/thi-online/12-bai-tap-tinh-gia-tri-bieu-thuc-luong-giac-co-loi-giai


\textbf{{QUESTION}}

Giá trị của biểu thức A = 4 – sin245° + 2cos260° − 3cot345° là
A. 1.
B. $\frac{3}{2}$.
C. $\frac{7}{2}$.
D. 4.

\textbf{{ANSWER}}

Đáp án đúng là: A
Ta có: A = 4 – sin245° + 2cos260° − 3cot345°
= 4 – (√22)2${\left( {\frac{{\sqrt 2 }}{2}} \right)^2}$ + 2. (12)2${\left( {\frac{1}{2}} \right)^2}$ − 3.13
= 4 – 12$\frac{1}{2}$ + 12$\frac{1}{2}$ - 3
= 1.

========================================================================

https://khoahoc.vietjack.com/thi-online/12-bai-tap-tinh-gia-tri-bieu-thuc-luong-giac-co-loi-giai


\textbf{{QUESTION}}

Giá trị của biểu thức B = cos215° + cos225° +…+ cos275° là
A. 1.
B. $\frac{3}{2}$.
C. $\frac{7}{2}$.
D. 4.

\textbf{{ANSWER}}

Đáp án đúng là: C
Ta có: B = cos215° + cos225° +…+ cos275°
= (cos215° + cos275°) + (cos225° + cos265°) + ….+ cos245°
= (cos215° + sin215°) + (cos225° + sin225°) + ….+ cos245°
= 1 + 1 + 1 + 1+ (√22)2${\left( {\frac{{\sqrt 2 }}{2}} \right)^2}$= 72$\frac{7}{2}$.

========================================================================

https://khoahoc.vietjack.com/thi-online/10-bai-tap-chung-minh-cac-canh-bang-nhau-va-casnhau-tinh-do-dai-canh-va-so-do-goc-co-loi-g


\textbf{{QUESTION}}

Điền vào chỗ còn thiếu trong các bước chứng minh sau:
“Xét DABC và DADE có:
.............,
BC = DE.
$\widehat {ABC} = \widehat {ADE};$ 
Vậy ΔABC = ∆ADE (g.c.g)”
A. AB = AD ;
B. $\widehat {ACB} = \widehat {AED};$ 
C. AC = AE;
D. $\widehat {BAC} = \widehat {DAE}.$

\textbf{{ANSWER}}

Đáp án đúng là: B
Ta có: ΔABC = ∆ADE theo trường hợp góc – cạnh – góc nên hai cặp góc bằng nhau là hai cặp góc kề với cặp cạnh bằng nhau của hai tam giác.
Mà BC = DE và $\widehat {ABC} = \widehat {ADE}$ nên cặp góc kề tương ứng còn lại là $\widehat {ACB} = \widehat {AED}.$ 
Vậy ta chọn phương án B.

========================================================================

https://khoahoc.vietjack.com/thi-online/giai-vbt-toan-7-canh-dieu-bai-1-bieu-thuc-so-bieu-thuc-dai-so-co-dap-an


\textbf{{QUESTION}}

Các số được nối với nhau bởi dấu các phép tính (cộng, trừ, nhân, chia, nâng lên lũy thừa) tạo thành một ......................

\textbf{{ANSWER}}

Các số được nối với nhau bởi dấu các phép tính (cộng, trừ, nhân, chia, nâng lên lũy thừa) tạo thành một biểu thức số.

========================================================================

https://khoahoc.vietjack.com/thi-online/giai-vbt-toan-7-canh-dieu-bai-1-bieu-thuc-so-bieu-thuc-dai-so-co-dap-an


\textbf{{QUESTION}}

Các số, biến số được nối với nhau bởi dấu các phép toán cộng, trừ, nhân, chia, nâng lên lũy thừa làm thành một ..............................

\textbf{{ANSWER}}

Các số, biến số được nối với nhau bởi dấu các phép toán cộng, trừ, nhân, chia, nâng lên lũy thừa làm thành một biểu thức đại số.

========================================================================

https://khoahoc.vietjack.com/thi-online/giai-vbt-toan-7-canh-dieu-bai-1-bieu-thuc-so-bieu-thuc-dai-so-co-dap-an


\textbf{{QUESTION}}

Để tính giá trị của một biểu thức đại số tại những giá trị cho trước của các biến, ta ............................... vào biểu thức rồi thực hiện các phép tính.

\textbf{{ANSWER}}

Để tính giá trị của một biểu thức đại số tại những giá trị cho trước của các biến, ta thay những giá trị cho trước đó vào biểu thức rồi thực hiện các phép tính.

========================================================================

https://khoahoc.vietjack.com/thi-online/giai-vbt-toan-7-canh-dieu-bai-1-bieu-thuc-so-bieu-thuc-dai-so-co-dap-an


\textbf{{QUESTION}}

Viết chữ Đ (đúng), S (sai) thích hợp vào sau phát biểu:
12a không phải là biểu thức số.

\textbf{{ANSWER}}

12a không phải là biểu thức số. (S)

========================================================================

https://khoahoc.vietjack.com/thi-online/giai-vbt-toan-7-canh-dieu-bai-1-bieu-thuc-so-bieu-thuc-dai-so-co-dap-an


\textbf{{QUESTION}}

Viết chữ Đ (đúng), S (sai) thích hợp vào sau phát biểu:
Biểu thức số phải có đầy đủ các phép tính cộng, trừ, nhân, chia, nâng lên lũy thừa.

\textbf{{ANSWER}}

Biểu thức số phải có đầy đủ các phép tính cộng, trừ, nhân, chia, nâng lên lũy thừa. (S)

========================================================================

https://khoahoc.vietjack.com/thi-online/20-bo-de-on-luyen-thi-thpt-quoc-gia-mon-toan-co-loi-giai/24297


\textbf{{QUESTION}}

Cho hàm số $$ y=f\left(x\right)$$ đồng biến trên khoảng $$ \left(a,b\right)$$. Mệnh đề nào dưới đây sai?
A. Hàm số $$ y=f\left(x+1\right)$$ đồng biến trên khoảng $$ \left(a,b\right)$$
B. Hàm số $$ y=-f\left(x\right)+1$$ nghịch biến trên khoảng $$ \left(a,b\right)$$
C. Hàm số $$ y=f\left(x\right)+1$$ đồng biến trên khoảng $$ \left(a,b\right)$$
D. Hàm số $$ y=-f\left(x\right)-1$$ nghịch biến trên khoảng $$ \left(a,b\right)$$

\textbf{{ANSWER}}

Chọn đáp án A
Hàm số $$ f\left(x\right)$$ đồng biến trên khoảng (a;b) nên $$ {f}^{\text{'}}\ge ,\forall x\in \left(a;b\right)$$và $$ f\left(x\right)=0$$ tại hữu hạn điểm $$ x\in \left(a;b\right)$$. Xét
* Hai hàm số $$ y=f\left(x\right)+1$$ và $$ y=-f\left(x\right)-1$$ đều có đạo hàm là $$ {y}^{\text{'}}=-{f}^{\text{'}}\left(x\right)\le 0,\forall x\in \left(a;b\right)$$ nên chúng nghịch biến trên khoảng (a;b). Hai phương án B, D đúng.
* Hàm số $$ y=f\left(x\right)+1$$ có đạo hàm $$ {y}^{\text{'}}={f}^{\text{'}}\left(x\right)\ge 0,\forall x\in \left(a;b\right)$$ nên đồng biến trên khoảng (a;b). Phương án C đúng

========================================================================

https://khoahoc.vietjack.com/thi-online/20-cau-trac-nghiem-toan-10-chan-troi-sang-tao-giai-tam-giac-va-ung-dung-thuc-te-co-dap-an-phan-2


\textbf{{QUESTION}}

Cho ∆ABC biết b = 32, c = 45, $\widehat A = 87^\circ $. Khẳng định nào sau đây đúng?

\textbf{{ANSWER}}

Hướng dẫn giải
Đáp án đúng là: A
Áp dụng định lí côsin cho DABC, ta có:
a2 = b2 + c2 – 2bc.cosA
= 322 + 452 – 2.32.45.cos87°
≈ 2898,3
Suy ra a ≈ $\sqrt {2898,3} $ ≈ 53,8.
Theo định lí sin, ta có $\frac{a}{{\sin A}} = \frac{b}{{\sin B}}$
Suy ra $\sin B = \frac{{b.\sin A}}{a} \approx \frac{{32.\sin 87^\circ }}{{53,8}} \approx 0,6$.
Do đó $\widehat B \approx 37^\circ $ 
($\widehat B \approx 180^\circ - 37^\circ = 143^\circ $ không thỏa mãn do $\widehat A + \widehat B \approx 87^\circ + 143^\circ = 230^\circ > 180^\circ )$
∆ABC có: $\widehat A + \widehat B + \widehat C = 180^\circ $ (định lí tổng ba góc trong một tam giác)
Suy ra $\widehat C = 180^\circ - \left( {\widehat A + \widehat B} \right) \approx 180^\circ - \left( {87^\circ + 37^\circ } \right) = 56^\circ $.
Vậy a ≈ 53,8, $\widehat B \approx 37^\circ ,\,\,\widehat C \approx 56^\circ $.
Do đó ta chọn phương án A.

========================================================================

https://khoahoc.vietjack.com/thi-online/20-cau-trac-nghiem-toan-10-chan-troi-sang-tao-giai-tam-giac-va-ung-dung-thuc-te-co-dap-an-phan-2


\textbf{{QUESTION}}

Cho ∆ABC biết ˆA=60∘,ˆB=40∘, c = 14. Khẳng định nào sau đây sai?

\textbf{{ANSWER}}

Hướng dẫn giải
Đáp án đúng là: D
⦁ ∆ABC có: ˆA+ˆB+ˆC=180∘$\widehat A + \widehat B + \widehat C = 180^\circ $ (định lí tổng ba góc trong một tam giác)
Suy ra ˆC=180∘−(ˆA+ˆB)=180∘−(60∘+40∘)=80∘$\widehat C = 180^\circ - \left( {\widehat A + \widehat B} \right) = 180^\circ - \left( {60^\circ + 40^\circ } \right) = 80^\circ $.
Do đó phương án A đúng.
⦁ Theo định lí sin, ta có: asinA=bsinB=csinC$\frac{a}{{\sin A}} = \frac{b}{{\sin B}} = \frac{c}{{\sin C}}$.
Suy ra a=c.sinAsinC=14.sin60∘sin80∘≈12,3$a = \frac{{c.\sin A}}{{\sin C}} = \frac{{14.\sin 60^\circ }}{{\sin 80^\circ }} \approx 12,3$.
Do đó phương án B đúng.
Ta có bsinB=csinC$\frac{b}{{\sin B}} = \frac{c}{{\sin C}}$
Suy ra b=c.sinBsinC=14.sin40∘sin80∘≈9,1$b = \frac{{c.\sin B}}{{\sin C}} = \frac{{14.\sin 40^\circ }}{{\sin 80^\circ }} \approx 9,1$.
Do đó phương án C đúng, phương án D sai.
Vậy ta chọn phương án D.

========================================================================

https://khoahoc.vietjack.com/thi-online/20-cau-trac-nghiem-toan-10-chan-troi-sang-tao-giai-tam-giac-va-ung-dung-thuc-te-co-dap-an-phan-2


\textbf{{QUESTION}}

Cho ∆ABC biết a=√6, b = 2, c=1+√3. Khẳng định nào sau đây đúng nhất?

\textbf{{ANSWER}}

Hướng dẫn giải
Đáp án đúng là: D
Theo hệ quả của định lí côsin, ta có:
⦁ cosA=b2+c2−a22bc=22+(1+√3)2−(√6)22.2.(1+√3)=12$\cos A = \frac{{{b^2} + {c^2} - {a^2}}}{{2bc}} = \frac{{{2^2} + {{\left( {1 + \sqrt 3 } \right)}^2} - {{\left( {\sqrt 6 } \right)}^2}}}{{2.2.\left( {1 + \sqrt 3 } \right)}} = \frac{1}{2}$.
Suy ra ˆA=60∘$\widehat A = 60^\circ $.
⦁ cosB=a2+c2−b22ac=(√6)2+(1+√3)2−222.√6.(1+√3)=√22$\cos B = \frac{{{a^2} + {c^2} - {b^2}}}{{2ac}} = \frac{{{{\left( {\sqrt 6 } \right)}^2} + {{\left( {1 + \sqrt 3 } \right)}^2} - {2^2}}}{{2.\sqrt 6 .\left( {1 + \sqrt 3 } \right)}} = \frac{{\sqrt 2 }}{2}$.
Suy ra ˆB=45∘$\widehat B = 45^\circ $.
⦁ cosC=a2+b2−c22ab=(√6)2+22−(1+√3)22.√6.2=√6−√24$\cos C = \frac{{{a^2} + {b^2} - {c^2}}}{{2ab}} = \frac{{{{\left( {\sqrt 6 } \right)}^2} + {2^2} - {{\left( {1 + \sqrt 3 } \right)}^2}}}{{2.\sqrt 6 .2}} = \frac{{\sqrt 6 - \sqrt 2 }}{4}$.
Suy ra ˆC=75∘$\widehat C = 75^\circ $.
Vậy ta chọn phương án D.

========================================================================

https://khoahoc.vietjack.com/thi-online/20-cau-trac-nghiem-toan-10-chan-troi-sang-tao-giai-tam-giac-va-ung-dung-thuc-te-co-dap-an-phan-2


\textbf{{QUESTION}}

Cho ˆA=120∘,ˆB=45∘, R = 2. Khẳng định nào sau đây sai?

\textbf{{ANSWER}}

Hướng dẫn giải
Đáp án đúng là: C
Theo hệ quả định lí sin, ta có:
⦁ BC = 2R.sinA = 2.2.sin120° = 2√3.
⦁ AC = 2R.sinB = 2.2.sin45° = 2√2.
Theo định lí côsin, ta có BC2 = AC2 + AB2 – 2.AC.AB.cosA
Suy ra (2√3)2=(2√2)2+AB2−2.2√2.AB.cos120∘
Khi đó AB2+2√2.AB−4=0
Vì vậy AB=√6−√2 hoặc AB=−√6−√2
Vì AB là độ dài một cạnh của ∆ABC nên ta có AB > 0.
Do đó ta nhận AB=√6−√2.
∆ABC có ˆA+ˆB+ˆC=180∘ (định lí tổng ba góc trong một tam giác)
Suy ra ˆC=180∘−(ˆA+ˆB)=180∘−(120∘+45∘)=15∘.
Vậy ta chọn phương án C.

========================================================================

https://khoahoc.vietjack.com/thi-online/20-cau-trac-nghiem-toan-10-chan-troi-sang-tao-giai-tam-giac-va-ung-dung-thuc-te-co-dap-an-phan-2


\textbf{{QUESTION}}

Cho ∆ABC, biết ˆA=60∘, hc=2√3, R = 6. Khẳng định nào sau đây đúng?

\textbf{{ANSWER}}

Hướng dẫn giải
Đáp án đúng là: B
⦁ Theo hệ quả định lí sin, ta có:
a = 2R.sinA = 2.6.sin60° = 6√3$6\sqrt 3 $.
⦁ Ta có S = 12chc=12bcsinA$\frac{1}{2}c{h_c} = \frac{1}{2}bc\sin A\,$.
Suy ra hc = b.sinA
Do đó b=hcsinA=2√3sin60∘=4$b = \frac{{{h_c}}}{{\sin A}} = \frac{{2\sqrt 3 }}{{\sin 60^\circ }} = 4$.
⦁ Theo định lí côsin, ta có a2 = b2 + c2 – 2bc.cosA
Suy ra (6√3)2=42+c2−2.4.c.cos60∘${\left( {6\sqrt 3 } \right)^2} = {4^2} + {c^2} - 2.4.c.\cos 60^\circ $
Khi đó c2 – 4c – 92 = 0
Vì vậy c=2+4√6$c = 2 + 4\sqrt 6 $ hoặc c=2−4√6$c = 2 - 4\sqrt 6 $.
Vì c là độ dài một cạnh của ∆ABC nên c > 0.
Do đó ta nhận c=2+4√6$c = 2 + 4\sqrt 6 $.
Vậy ta chọn phương án B.

========================================================================

https://khoahoc.vietjack.com/thi-online/10-bai-tap-cacs-bai-toan-thuc-te-giai-bang-cach-lap-da-thuc-co-loi-giai


\textbf{{QUESTION}}

Một bể chứa nước có dạng hình hộp chữ nhật được thiết kế với kích thước theo tỉ lệ tương ứng: chiều cao : chiều rộng : chiều dài = 1 : 2 : 3. Gọi chiều cao của bể là x (mét).
Bậc của đa thức biểu thị thể tích của bể là

\textbf{{ANSWER}}

Đáp án đúng là: C
Do chiều cao : chiều rộng : chiều dài = 1 : 2 : 3 nên chiều rộng gấp 2 lần chiều cao, chiều dài gấp 3 lần chiều cao.
Do đó chiều rộng của bể là 2x (m), chiều dài của bể là 3x (m).
Thể tích của bể có dạng hình hộp chữ nhật là: 
x . 2x . 3x = 2 . 3 . x . x . x = 6x3 (m3).
Đa thức trên có hạng tử có bậc cao nhất là 6x3 nên bậc của đa thức trên bằng 3.

========================================================================

https://khoahoc.vietjack.com/thi-online/19-cau-trac-nghiem-toan-11-chan-troi-sang-tao-goc-luong-giac-co-dap-an


\textbf{{QUESTION}}

Phát biểu nào sau đây là đúng ?
A. Cho hai tia Ou, Ov thì có duy nhất góc lượng giác tia đầu Ou, tia cuối Ov.
B. Số đo góc lượng giác luôn dương.
C. sđ (Ov,Ou) = sđ (Ou,Ov).
D. Mỗi góc lượng giác gốc O được xác định bởi tia đầu Ou, tia cuối Ov và số đo của nó.

\textbf{{ANSWER}}

Chọn đáp án D.
Đáp án cần chọn là: D

========================================================================

https://khoahoc.vietjack.com/thi-online/10-bai-tap-sso-phan-tu-cua-tap-hop-tap-hop-rong-co-loi-giai


\textbf{{QUESTION}}

Cho tập hợp E = {x ∈ ℕ | x là ước chung của 20 và 40}.
Tập hợp E có bao nhiêu phần tử?
A. 5;
B. 6;
C. 3;
D. 4.

\textbf{{ANSWER}}

Đáp án đúng là: B.
Ta có:
+ Các ước là số tự nhiên của 20 là: 1; 2; 4; 5; 10; 20.
+ Các ước là số tự nhiên của 40 là: 1; 2; 4; 5; 8; 10; 20; 40.
Do đó các ước chung là số tự nhiên của 20 và 40 là 1; 2; 4; 5; 10; 20.
⇒ E = {1; 2; 4; 5; 10; 20}.
Vì vậy tập hợp E gồm có 6 phần tử.
Vậy n(E) = 6.

========================================================================

https://khoahoc.vietjack.com/thi-online/10-bai-tap-sso-phan-tu-cua-tap-hop-tap-hop-rong-co-loi-giai


\textbf{{QUESTION}}

Cho tập hợp X = {x ∈ ℤ | (x2 – 3)(4x2 – 10x + 6) = 0}.
Tập hợp X có bao nhiêu phần tử?
A. 1;
B. 2;
C. 3; 
D. 4.

\textbf{{ANSWER}}

Đáp án đúng là: A.
Ta có:
(x2 – 3)(4x2 – 10x + 6) = 0
⇔ [x2– 3=04x2– 10x + 6=0⇔[x=√3x=−√3x=1x=32$$ \left[\begin{array}{c}{x}^{2}–\text{ }3=0\\ 4{x}^{2}–\text{ }10x\text{ }+\text{ }6=0\end{array}\Leftrightarrow \right.\left[\begin{array}{c}x=\sqrt{3}\\ x=-\sqrt{3}\\ x=1\\ x=\frac{3}{2}\end{array}\right.$$.
Vì x ∈ ℤ nên ta chỉ nhận một giá trị là x = 1.
Do đó tập hợp X có 1 phần tử.
Vậy n(X) = 1.

========================================================================

https://khoahoc.vietjack.com/thi-online/10-bai-tap-sso-phan-tu-cua-tap-hop-tap-hop-rong-co-loi-giai


\textbf{{QUESTION}}

Trong các tập hợp sau đây, tập hợp nào là tập hợp rỗng?
A. A = {x ∈ ℤ | x2 – 9 = 0};
B. B = {x ∈ ℝ | x2 – 6 = 0};
C. C = {x ∈ ℝ | x2 + 1 = 0};
D. D = {x ∈ ℝ | x2 – 4x + 3 = 0}.

\textbf{{ANSWER}}

Đáp án đúng là: C.
A. Ta có:
x2 – 9 = 0 ⇔ x2 = 9 ⇔ [x=3x=−3 .
Vì x ∈ ℤ nên hai nghiệm trên đều thỏa mãn.
Vậy A = {– 3; 3}.
B. Ta có: 
x2 – 6 = 0 ⇔ x2 = 6 ⇔ [x=−√6x=√6 .
Vì x ∈ ℝ nên hai nghiệm trên đều thỏa mãn.
Vậy B = { ;  }.
C. Ta có: 
Phương trình x2 + 1 = 0 vô nghiệm do x2 + 1 > 0 với mọi x ∈ ℝ.
Do đó, tập hợp C không có phần tử nào thỏa mãn.
Vậy C = ∅.
D. Ta có:
x2 – 4x + 3 = 0 ⇔ [x=1x=3 .
Vì x ∈ ℝ nên hai nghiệm trên đều thỏa mãn.
Vậy D = {1; 3}.
Vậy C là tập hợp rỗng.

========================================================================

https://khoahoc.vietjack.com/thi-online/10-bai-tap-sso-phan-tu-cua-tap-hop-tap-hop-rong-co-loi-giai


\textbf{{QUESTION}}

Trong các tập hợp sau, tập hợp nào không phải là tập hợp rỗng?
A. A = {x ∈ ℝ | x2 + x + 3 = 0};
B. B = {x ∈ ℕ* | x2 + 6x + 5 = 0};
C. C = {x ∈ ℕ* | x(x2 – 5) = 0};
D. D = {x ∈ ℝ | x2 – 9x + 20 = 0}.

\textbf{{ANSWER}}

Đáp án đúng là: D.
A. Ta có: 
Do x2 + x + 3 = x2 + 2 .12  x + 14  + 114  = (x+12)2+114>0 .
Phương trình x2 + x + 3 = 0 vô nghiệm.
Do đó, tập hợp A không có phần tử nào thỏa mãn.
Vậy A = ∅.
B. Ta có: 
x2 + 6x + 5 = 0 ⇔ [x=−1x=−5 .
Vì x ∈ ℕ* nên không có phần tử nào thỏa mãn tập hợp trên.
Vậy B = ∅.
C. Ta có:
x(x2 – 5) = 0 ⇔ [x=0x2– 5=0⇔[x=0x=√5x=−√5 .
Vì x ∈ ℕ* nên không có phần tử nào thỏa mãn tập hợp trên.
Vậy C = ∅.
D. Ta có: 
x2 – 9x + 20 = 0 ⟺ [x=4x=5 .
Vì x ∈ ℝ nên hai nghiệm x = 4 và x = 5 đều thỏa mãn.
Do đó tập hợp D có hai phần tử.
Vậy D = {4; 5}.
Vậy chỉ có tập hợp D không phải là tập hợp rỗng.

========================================================================

https://khoahoc.vietjack.com/thi-online/10-bai-tap-sso-phan-tu-cua-tap-hop-tap-hop-rong-co-loi-giai


\textbf{{QUESTION}}

Cho các tập hợp sau:
A = {x ∈ ℤ | 2 < x – 1 < 4};
B = {x ∈ ℕ | 3 < 2x – 3 < 5};
C = {x ∈ ℕ | x < 5}.
A. 0;
B. 1;
C. 2;
D. 3.

\textbf{{ANSWER}}

Đáp án đúng là: B.
- Xét tập hợp A ta có:
2 < x – 1 < 4 
⇔ 2 + 1 < x < 4 + 1
⇔ 3 < x < 5.
Vì x ∈ ℤ nên x = 4.
Vậy A = {4}.
- Xét tập hợp B ta có:
3 < 2x – 3 < 5
⇔ 3 + 3 < 2x < 5 + 3
⇔ 6 < 2x < 8
⇔ 3 < x < 4.
Vì x ∈ ℕ nên không có giá trị nào của x thỏa mãn.
Vậy B = ∅.
- Xét tập hợp C ta có: 
Các số tự nhiên x bé hơn 5 là 0; 1; 2; 3; 4.
Vậy C = {0; 1; 2; 3; 4}.
Vậy trong 3 tập hợp trên có 1 tập rỗng.

========================================================================

https://khoahoc.vietjack.com/thi-online/giai-vth-toan-7-kntt-bai-21-tinh-chat-cua-day-ti-so-bang-nhau-co-dap-an


\textbf{{QUESTION}}

Từ $\frac{a}{b} = \frac{c}{d}$ ta suy ra
A. $\frac{a}{b} = \frac{{a - c}}{{d - b}}$; 
B. $\frac{a}{b} = \frac{{c - a}}{{b - d}}$; 
C. $\frac{a}{b} = \frac{{a + c}}{{b + d}}$; 
D. $\frac{a}{b} = \frac{{ac}}{{bd}}$.

\textbf{{ANSWER}}

Đáp án đúng là: C
Từ $\frac{a}{b} = \frac{c}{d}$ ta suy ra $\frac{a}{b} = \frac{c}{d} = \frac{{a + c}}{{b + d}} = \frac{{a - c}}{{b - d}}$ (với điều kiện các biểu thức có nghĩa).
Vậy ta chọn đáp án C.

========================================================================

https://khoahoc.vietjack.com/thi-online/giai-vth-toan-7-kntt-bai-21-tinh-chat-cua-day-ti-so-bang-nhau-co-dap-an


\textbf{{QUESTION}}

Nếu x3=y5$\frac{x}{3} = \frac{y}{5}$ và x + y = – 16 thì 
A. x = 3, y = 5;
B. x = – 6, y = – 10; 
C. x = – 10, y = – 6; 
D. x = 6, y = – 22.

\textbf{{ANSWER}}

Đáp án đúng là: B
Từ $\frac{x}{3} = \frac{y}{5}$, áp dụng tính chất của dãy tỉ số bằng nhau ta suy ra $\frac{x}{3} = \frac{y}{5} = \frac{{x + y}}{{3 + 5}} = \frac{{x + y}}{8}$. 
Mà x + y = – 16, do đó, $\frac{x}{3} = \frac{y}{5} = \frac{{ - 16}}{8} = - 2$. 
Suy ra x = 3 . (– 2) = – 6, y = 5 . (– 2) = – 10. 
Vậy x = – 6, y = – 10.

========================================================================

https://khoahoc.vietjack.com/thi-online/giai-vth-toan-7-kntt-bai-21-tinh-chat-cua-day-ti-so-bang-nhau-co-dap-an


\textbf{{QUESTION}}

Từ ab=cd=ef$\frac{a}{b} = \frac{c}{d} = \frac{e}{f}$ ta suy ra 
A. ab=a+c−eb+d−f$\frac{a}{b} = \frac{{a + c - e}}{{b + d - f}}$; 
B. ab=a+c−eb−d+f$\frac{a}{b} = \frac{{a + c - e}}{{b - d + f}}$; 
C. ab=a−c+eb+d−f$\frac{a}{b} = \frac{{a - c + e}}{{b + d - f}}$; 
D. ab=acebdf$\frac{a}{b} = \frac{{ace}}{{bdf}}$.

\textbf{{ANSWER}}

Đáp án đúng là: A
Từ ab=cd=ef$\frac{a}{b} = \frac{c}{d} = \frac{e}{f}$ ta suy ra ab=cd=ef$\frac{a}{b} = \frac{c}{d} = \frac{e}{f}$=a+c+eb+d+f=a+c−eb+d−f=a−c+eb−d+f$ = \frac{{a + c + e}}{{b + d + f}} = \frac{{a + c - e}}{{b + d - f}} = \frac{{a - c + e}}{{b - d + f}}$ (với điều kiện các tỉ số có nghĩa). 
Vậy ta chọn đáp án A.

========================================================================

https://khoahoc.vietjack.com/thi-online/giai-vth-toan-7-kntt-bai-21-tinh-chat-cua-day-ti-so-bang-nhau-co-dap-an


\textbf{{QUESTION}}

Biết rằng x, y, z tỉ lệ với 3, 4, 5. Khi đó
A. 3x = 4y = 5z; 
B. x : y : z = 5 : 4 : 3; 
C. 5x = 4y = 3z; 
D. x3=y4=z5$\frac{x}{3} = \frac{y}{4} = \frac{z}{5}$.

\textbf{{ANSWER}}

Đáp án đúng là: D
Vì x, y, z tỉ lệ với 3, 4, 5 nên x : y : z = 3 : 4 : 5 hay x3=y4=z5$\frac{x}{3} = \frac{y}{4} = \frac{z}{5}$.

========================================================================

https://khoahoc.vietjack.com/thi-online/giai-vth-toan-7-kntt-bai-21-tinh-chat-cua-day-ti-so-bang-nhau-co-dap-an


\textbf{{QUESTION}}

Tìm hai số x và y, biết: x9=y11$\frac{x}{9} = \frac{y}{{11}}$ và x + y = 40.

\textbf{{ANSWER}}

Áp dụng tính chất của dãy tỉ số bằng nhau, ta có:
x9=y11=x+y9+11=4020$\frac{x}{9} = \frac{y}{{11}} = \frac{{x + y}}{{9 + 11}} = \frac{{40}}{{20}}$= 2.
Suy ra x = 2 . 9 = 18 và y = 2 . 11 = 22.
Vậy x = 18 và y = 22.

========================================================================

https://khoahoc.vietjack.com/thi-online/dau-hieu-chia-het-cho-2-cho-5/57740


\textbf{{QUESTION}}

Quần đảo Cát Bà là quần thể gồm 367 đảo trong đó có đảo Cát Bà ở phía nam vịnh Hạ Long, ngoài khơi thành phố Hải Phòng và tỉnh Quảng Ninh.Để đi đến đảo Cát Bà, các sẽ được đi bằng tàu cánh ngầm.Tàu cánh ngầm là một chiếc tàu có cánh giống như những chiếc lá lắp trên các giằng phía dưới thân. Khi tàu tăng tốc các cánh ngầm tạo ra lực nâng thân tàu lên khỏi mặt nước. Điều này giúp làm giảm rất nhiều lực cản với thân tàu và lại giúp gia tăng tốc độ. Trong một tour du lịch ra đảo Cát Bà đã bán được hơn 150 vé tàu. Nếu xếp hàng đôi lên tàu thì hai hàng bằng nhau và chia vào ngổi ở ba dãy ghế thì số  khách ở ba dãy bằng nhau. Tính số khách du lịch biết rằng số khách không quá 166 người

\textbf{{ANSWER}}

Vì xếp hàng đôi lên tàu thì hai hàng bằng nhau và chia vào ngổi ở năm dãy ghế thì số khách ở năm dãy bằng nhau nên số khách là số tự nhiên chia hết cho 2 và 5. Mà số khách trong khoảng 150 đến 166 người.Vậy số khách là 160 người

========================================================================

https://khoahoc.vietjack.com/thi-online/10-cau-trac-nghiem-toan-8-bai-2-nhan-da-thuc-voi-da-thuc-co-dap-an-nhan-biet


\textbf{{QUESTION}}

Tích (x - y)(x + y) có kết quả bằng
A. x2 – 2xy + y2
B. x2 + y2
C. x2 – y2
D. x2 + 2xy + y2

\textbf{{ANSWER}}

Ta có (x - y)(x + y) = x.x + x.y – x.y – y.y = x2 – y2
Đáp án cần chọn là: C

========================================================================

https://khoahoc.vietjack.com/thi-online/10-cau-trac-nghiem-toan-8-bai-2-nhan-da-thuc-voi-da-thuc-co-dap-an-nhan-biet


\textbf{{QUESTION}}

Tích (2x – 3)(2x + 3) có kết quả bằng
A. 4x2 + 12x+ 9
B. 4x2 – 9
C. 2x2 – 3 
D. 4x2 + 9

\textbf{{ANSWER}}

Ta có (2x – 3)(2x + 3) = 2x.2x + 2x.3 – 3.2x + (-3).3
          = 4x2 + 6x – 6x – 9 = 4x2 – 9
Đáp án cần chọn là: B

========================================================================

https://khoahoc.vietjack.com/thi-online/10-cau-trac-nghiem-toan-8-bai-2-nhan-da-thuc-voi-da-thuc-co-dap-an-nhan-biet


\textbf{{QUESTION}}

Chọn câu đúng
A. (x2 – 1)(x2 + 2x) = x4 – x3 – 2x
B. (x2 – 1)(x2 + 2x) = x4 – x2 – 2x
C. (x2 – 1)(x2 + 2x) = x4 + 2x3 – x2 – 2x 
D. (x2 – 1)(x2 + 2x) = x4 + 2x3 – 2x

\textbf{{ANSWER}}

Ta có: (x2 – 1)(x2 + 2x) = x2.x2 + x2.2x – 1.x2 – 1.2x
                                      = x4 + 2x3 – x2 – 2x
Đáp án cần chọn là: C

========================================================================

https://khoahoc.vietjack.com/thi-online/10-cau-trac-nghiem-toan-8-bai-2-nhan-da-thuc-voi-da-thuc-co-dap-an-nhan-biet


\textbf{{QUESTION}}

Chọn câu đúng
A. (x – 1)(x2 + x + 1) = x3 – 1 
B. (x – 1)(x + 1) = 1 – x2
C. (x + 1)(x – 1) = x2 + 1
D. (x2 + x + 1)(x – 1) = 1 – x2

\textbf{{ANSWER}}

Ta có
+) (x – 1)(x + 1) = x.x + x – x – 1 = x2 – 1 nên phương án B sai, C sai
+) (x – 1)(x2 + x + 1) = (x2 + x + 1)(x – 1)
= x.x2 + x.x + x.1 – x2 – x – 1
= x3 + x2 + x – x2 – x – 1 = x3 – 1 nên phương án D sai, A đúng
Đáp án cần chọn là: A

========================================================================

https://khoahoc.vietjack.com/thi-online/10-cau-trac-nghiem-toan-8-bai-2-nhan-da-thuc-voi-da-thuc-co-dap-an-nhan-biet


\textbf{{QUESTION}}

Chọn câu đúng.
A. (2x – 1)(3x2 -7x + 5) = 6x3 – 17x2 + 17x – 1
B. (2x – 1)(3x2 -7x + 5) = 6x3 – 4x2 + 4x – 5
C. (2x – 1)(3x2 -7x + 5) = 6x3 – 17x2 + 10x – 5
D. (2x – 1)(3x2 -7x + 5) = 6x3 – 17x2 + 17x – 5

\textbf{{ANSWER}}

Ta có (2x – 1)(3x2 -7x + 5) = 2x.3x2 + 2x.(-7x) + 2x.5 – 3x2 – (-7x) – 1.5
                                      = 6x3 – 14x2 + 10x – 3x2 + 7x – 5
                                      = 6x3 – 17x2 + 17x – 5
Đáp án cần chọn là: D

========================================================================

https://khoahoc.vietjack.com/thi-online/bai-tap-toan-8-chu-de-6-bat-dang-thuc-co-dap-an/108031


\textbf{{QUESTION}}

Chứng minh rằng với mọi số thực a, b ta luôn có: $$ \left|a\pm b\right|\ge \left|a\right|-\left|b\right|$$

\textbf{{ANSWER}}

Ta có: $$ \left|a\right|=\left|(a\pm b)\mp b\right|\le \left|a\pm b\right|+\left|b\right|\Rightarrow \left|a\right|-\left|b\right|\le \left|a\pm b\right|$$

========================================================================

https://khoahoc.vietjack.com/thi-online/bai-tap-toan-8-chu-de-6-bat-dang-thuc-co-dap-an/108031


\textbf{{QUESTION}}

Chứng minh rằng với mọi số thực a, b, c ta luôn có: |a−c|≤|a−b|+|b−c|$$ \left|a-c\right|\le \left|a-b\right|+\left|b-c\right|$$.

\textbf{{ANSWER}}

Ta có: $$ a-c=(a-b)+(b-c)$$
Kết quả suy ra từ tính chất 4.

========================================================================

https://khoahoc.vietjack.com/thi-online/bai-tap-toan-8-chu-de-6-bat-dang-thuc-co-dap-an/108031


\textbf{{QUESTION}}

Chứng minh rằng nếu: |a|+|b|=a+b$$ \left|a\right|+\left|b\right|=a+b$$, thì a,b≥0$$ a,b\ge 0$$     (1)

\textbf{{ANSWER}}

Vế trái không âm, vậy vế phải không âm, tức là a+b≥0$$ a+b\ge 0$$.
Suy ra, trong hai số a, b phải có một số không âm, giả sử a≥0$$ a\ge 0$$ suy ra |a|=a$$ \left|a\right|=a$$.
Từ đó (1) có dạng: |b|=b⇔b≥0$$ \left|b\right|=b\Leftrightarrow b\ge 0$$

========================================================================

https://khoahoc.vietjack.com/thi-online/bai-tap-toan-8-chu-de-6-bat-dang-thuc-co-dap-an/108031


\textbf{{QUESTION}}

Giải phương trình: |2x+3|+|1−2x|=4$$ \left|2x+3\right|+\left|1-2x\right|=4$$

\textbf{{ANSWER}}

Ta biến đổi phương trình về dạng:
|2x+3|+|1−2x|=(2x+3)+(1−2x)tính chất 1↔{2x+3≥01−2x≥0⇔{x≥−32x≤12$$ \left|2x+3\right|+\left|1-2x\right|=(2x+3)+(1-2x)\phantom{\rule{0ex}{0ex}}\stackrel{tính\quad chất\quad 1}{\leftrightarrow }\left\{\begin{array}{l}2x+3\ge 0\\ 1-2x\ge 0\end{array}\right.\Leftrightarrow \left\{\begin{array}{l}x\ge -\frac{3}{2}\\ x\le \frac{1}{2}\end{array}\right.$$
Vậy nghiệm của phương trình là −32≤x≤12$$ -\frac{3}{2}\le x\le \frac{1}{2}$$

========================================================================

https://khoahoc.vietjack.com/thi-online/bai-tap-toan-8-chu-de-6-bat-dang-thuc-co-dap-an/108031


\textbf{{QUESTION}}

Giải phương trình: √x2+2x+1−√x2−2x+1=2$$ \sqrt{{x}^{2}+2x+1}-\sqrt{{x}^{2}-2x+1}=2$$

\textbf{{ANSWER}}

Ta biến đổi phương trình về dạng:
√(x+1)2−√(x−1)2=2⇔|x+1|−|x−1|=|(x+1)−(x−1)|tính châtt 4↔(x−1).[(x+1)−(x−1)]≥0⇔(x−1).2≥0⇔x−1≥0⇔x≥1$$ \sqrt{{(x+1)}^{2}}-\sqrt{{(x-1)}^{2}}=2\Leftrightarrow \left|x+1\right|-\left|x-1\right|=\left|(x+1)-(x-1)\right|\phantom{\rule{0ex}{0ex}}\stackrel{\text{tính châtt }4}{\leftrightarrow }(x-1).\left[(x+1)-(x-1)\right]\ge 0\Leftrightarrow (x-1).2\ge 0\phantom{\rule{0ex}{0ex}}\Leftrightarrow x-1\ge 0\Leftrightarrow x\ge 1$$
Vậy, phương trình có nghiệm là x≥1$$ x\ge 1$$.

========================================================================

https://khoahoc.vietjack.com/thi-online/trac-nghiem-bai-tap-theo-tuan-toan-7-tuan-14-co-dap-an


\textbf{{QUESTION}}

Cho biết 7 máy cày xong một cánh đồng hết 20 giờ. Hỏi 10 máy cày như thế (cùng năng suất) cày xong cánh đồng hết bao nhiêu giờ?

\textbf{{ANSWER}}

Gọi thời gian đội cày xong cánh đồng là $$ x\text{\hspace{0.17em}\hspace{0.17em}\hspace{0.17em}}\left(x>0\right)$$ giờ
Thời gian đội cày xong cánh đồng và số máy cày đội có là hai đại lượng tỉ lệ nghịch
Theo tính chất tỉ lệ nghịch, ta có : $$ 7.20=10.x\Rightarrow x=14$$
Vậy đội có 10 máy cày thì phải cần 14 giờ để hoàn thành xong

========================================================================

https://khoahoc.vietjack.com/thi-online/trac-nghiem-bai-tap-theo-tuan-toan-7-tuan-14-co-dap-an


\textbf{{QUESTION}}

Tam giác ABC có số đo các góc A,B,C tỉ lệ nghịch với 3, 4, 6. Tính số đo các góc của tam giác?

\textbf{{ANSWER}}

Gọi số đo $$ \widehat{A},\text{\hspace{0.17em}\hspace{0.17em}}\widehat{B},\text{\hspace{0.17em}\hspace{0.17em}}\widehat{C}$$ lần lượt là $$ x;y;z$$ (độ) $$ 0°<x;\text{\hspace{0.17em}\hspace{0.17em}}y;\text{\hspace{0.17em}\hspace{0.17em}}z<180°$$
$$ x;y;z$$ tỉ lệ nghịch với 3, 4, 6 
$$ \begin{array}{l}\Rightarrow 3x=4y=6z\\ \Rightarrow \frac{x}{4}=\frac{y}{3}=\frac{z}{2}\end{array}$$
Mà $$ x+y+z=180°$$. 
Áp dụng tính chất dãy tỉ số bằng nhau ta có:
$$ \frac{x}{4}=\frac{y}{3}=\frac{z}{2}=\frac{x+y+z}{4+3+2}=\frac{180°}{9}=20°$$
$$ \Rightarrow x=80°;y=60°;z=40°$$
Vậy số đo ba góc của tam giác ABC là $$ {80}^{0};{60}^{0};{40}^{0}$$

========================================================================

https://khoahoc.vietjack.com/thi-online/trac-nghiem-bai-tap-theo-tuan-toan-7-tuan-14-co-dap-an


\textbf{{QUESTION}}

Ba đội máy cày, cày trên 3 cánh đồng có diện tích như nhau. Đội I hoàn thành công việc trong 4 ngày, đội II hoàn thành công việc 6 ngày. Hỏi đội III hoàn thành công việc trong bao nhiêu ngày, biết rằng tổng số máy cày của đội I và đội II gấp 5 lần số máy cày của đội III và năng suất của các máy là như nhau?

\textbf{{ANSWER}}

Gọi thời gian hoàn thành công việc của đội III là x (ngày)
Số máy cày của mỗi đội lần lượt là y1;y2;y3$$ {y}_{1};{y}_{2};{y}_{3}$$ (máy)
Vì số máy cày và thời gian là hai đại lượng tỉ lệ nghịch nên 4y1=6y2=xy3$$ 4{y}_{1}=6{y}_{2}=x{y}_{3}$$
tổng số máy cày của đội I và đội II gấp 5 lần số máy cày của đội III nên :y1+y2=5y3$$ {y}_{1}+{y}_{2}=5{y}_{3}$$
4y1=6y2=xy3⇒y13=y22=xy312$$ 4{y}_{1}=6{y}_{2}=x{y}_{3}\Rightarrow \frac{{y}_{1}}{3}=\frac{{y}_{2}}{2}=\frac{x{y}_{3}}{12}$$. 
Áp dụng tính chất của dãy tỉ số bằng nhau ta có:
y13=y22=xy312=y1+y23+2=5y35=y3$$ \frac{{y}_{1}}{3}=\frac{{y}_{2}}{2}=\frac{x{y}_{3}}{12}=\frac{{y}_{1}+{y}_{2}}{3+2}=\frac{5{y}_{3}}{5}={y}_{3}$$
⇒xy312=y3⇒x=12$$ \Rightarrow \frac{x{y}_{3}}{12}={y}_{3}\Rightarrow x=12$$
Vậy thời gian hoàn thành công việc của đội III là 12 ngày.

========================================================================

https://khoahoc.vietjack.com/thi-online/trac-nghiem-bai-tap-theo-tuan-toan-7-tuan-14-co-dap-an


\textbf{{QUESTION}}

Tổng số học sinh của 3 lớp 7A; 7B; 7C là 143. Nếu rút đi ở lớp 7A 16$$ \frac{1}{6}$$ số học sinh, ở lớp 7B 18$$ \frac{1}{8}$$ số học sinh, ở lớp 7C 111$$ \frac{1}{11}$$ số học sinh thì số học sinh còn lại ở 3 lớp tỉ lệ nghịch với 18;17;110$$ \frac{1}{8};\frac{1}{7};\frac{1}{10}$$. Tính số học sinh mỗi lớp.

\textbf{{ANSWER}}

Gọi số học sinh của mỗi lớp lần lượt là a,  b,  c$$ a,\text{\hspace{0.17em}\hspace{0.17em}}b,\text{\hspace{0.17em}\hspace{0.17em}}c$$ (a,  b,  c $$ a,\text{\hspace{0.17em}\hspace{0.17em}}b,\text{\hspace{0.17em}\hspace{0.17em}}c\quad $$nguyên dương)
Số học sinh còn lại ở 3 lớp tỉ lệ nghịch với 18;17;110$$ \frac{1}{8};\frac{1}{7};\frac{1}{10}$$nên 
56a.18=78b.17=1011c.110$$ \frac{5}{6}a.\frac{1}{8}=\frac{7}{8}b.\frac{1}{7}=\frac{10}{11}c.\frac{1}{10}$$
⇒548a=18b=111c$$ \Rightarrow \frac{5}{48}a=\frac{1}{8}b=\frac{1}{11}c$$
⇒55a=66b=48c$$ \Rightarrow 55a=66b=48c$$
Theo tính chất của dãy tỉ số bằng nhau ta có:a48=b40=c55=a+b+c48+40+55=143143=1$$ \frac{a}{48}=\frac{b}{40}=\frac{c}{55}=\frac{a+b+c}{48+40+55}=\frac{143}{143}=1$$
⇒a=48;b=40;c=55$$ \Rightarrow a=48;b=40;c=55$$
Vậy số học sinh của lớp 7A, 7B, 7C lần lượt là 48 học sinh, 40 học sinh, 55 học sinh

========================================================================

https://khoahoc.vietjack.com/thi-online/37-cau-trac-nghiem-bat-phuong-trinh-va-he-bat-phuong-trinh-mot-an-co-dap-an-tong-hop


\textbf{{QUESTION}}

Bất phương trình ax + b > 0 có tập nghiệm là R khi:
A. $$ \left\{\begin{array}{c}a=0\\ b>0\end{array}\right.$$
B. $$ \left\{\begin{array}{c}a>0\\ b>0\end{array}\right.$$
C. $$ \left\{\begin{array}{c}a=0\\ b\ne 0\end{array}\right.$$
D. $$ \left\{\begin{array}{c}a=0\\ b\le 0\end{array}\right.$$

\textbf{{ANSWER}}

Bất phương trình ax + b > 0 có tập nghiệm là R nếu a = 0, b > 0.
Đáp án cần chọn là: A

========================================================================

https://khoahoc.vietjack.com/thi-online/10-bai-tap-lap-phuong-trinh-chinh-tac-cua-parabol-co-loi-giai


\textbf{{QUESTION}}

Trong mặt phẳng tọa độ Oxy, cho parabol (P) có phương trình đường chuẩn $$ x+\frac{1}{2}=0$$. Phương trình chính tắc của parabol (P) là
A.y2 = 4x;
B. y2 = x ;
C. $$ {y}^{2}=\frac{1}{2}x$$;

\textbf{{ANSWER}}

Hướng dẫn giải:
Đáp án đúng là: D
Parabol (P) có phương trình đường chuẩn $$ x+\frac{1}{2}=0$$ hay $$ x=-\frac{1}{2}.$$
Do đó $$ \frac{p}{2}=\frac{1}{2}$$ nên p = 1.
Vậy phương trình chính tắc của parabol (P) là y2 = 2x.

========================================================================

https://khoahoc.vietjack.com/thi-online/10-bai-tap-lap-phuong-trinh-chinh-tac-cua-parabol-co-loi-giai


\textbf{{QUESTION}}

Trong mặt phẳng tọa độ Oxy, cho parabol (P): y2 = 2px (p > 0) có tiêu điểm F(5; 0). Phương trình chính tắc của (P) là
A.y2 = 5x;
B. y2=52x$$ {y}^{2}=\frac{5}{2}x$$;
C. y2 = 20x;

\textbf{{ANSWER}}

Hướng dẫn giải:
Đáp án đúng là: C
Do tọa độ tiêu điểm của (P) là F(5; 0) nên ta có $$ \frac{p}{2}=5,$$ suy ra p = 10 nên 2p = 20.
Vậy phương trình chính tắc của parabol (P) là : y2 = 20x.

========================================================================

https://khoahoc.vietjack.com/thi-online/10-bai-tap-lap-phuong-trinh-chinh-tac-cua-parabol-co-loi-giai


\textbf{{QUESTION}}

Trong mặt phẳng tọa độ Oxy, cho parabol (P): y2 = 2px (p > 0). Biết rằng khoảng cách từ tiêu điểm F đến đường thẳng Δ: x + y – 12 = 0 bằng 2√2.$$ 2\sqrt{2}.$$ Phương trình chính tắc của (P) là
A. y2 = 16x hoặc y2 = 32x;
B. y2 = –16x hoặc y2 = 32x;
C. y2 = 32x hoặc y2 = 64x;

\textbf{{ANSWER}}

Hướng dẫn giải:
Đáp án đúng là: C
Gọi tọa độ tiêu điểm của (P) là F(p2;0).$$ F\left(\frac{p}{2};0\right).$$
Khoảng cách từ F đến Δ là 2√2$$ 2\sqrt{2}$$ nên ta có: d(F,Δ)=|p2−12|√12+12=2√2$$ d\left(F,\Delta \right)=\frac{\left|\frac{p}{2}-12\right|}{\sqrt{{1}^{2}+{1}^{2}}}=2\sqrt{2}$$
⇔|p2−12|=4⇔[p=16p=32$$ \Leftrightarrow \left|\frac{p}{2}-12\right|=4\Leftrightarrow \left[\begin{array}{c}p=16\\ p=32\end{array}\right.$$.
Vậy phương trình chính tắc của (P) là: y2 = 32x hoặc y2 = 64x.

========================================================================

https://khoahoc.vietjack.com/thi-online/10-bai-tap-lap-phuong-trinh-chinh-tac-cua-parabol-co-loi-giai


\textbf{{QUESTION}}

Trong mặt phẳng tọa độ Oxy, cho parabol (P): y2 = 2px (p > 0) có khoảng cách từ đỉnh tới tiêu điểm bằng 34.$$ \frac{3}{4}.$$ Phương trình chính tắc của (P) là
A. y2 = 3x;
B. y2 = 6x;
C. y2=34x;$$ {y}^{2}=\frac{3}{4}x;$$

\textbf{{ANSWER}}

Hướng dẫn giải:
Đáp án đúng là: A
Khoảng cách từ đỉnh O đến tiêu điểm F(p2;0)$$ F\left(\frac{p}{2};0\right)$$ là p2$$ \frac{p}{2}$$
Theo bài, ta có p2=34,$$ \frac{p}{2}=\frac{3}{4},$$ suy ra p=32$$ p=\frac{3}{2}$$ nên 2p = 3.
Vậy phương trình chính tắc của (P) là y2 = 3x.

========================================================================

https://khoahoc.vietjack.com/thi-online/giai-sbt-toan-8-kntt-on-tap-chuong-7-co-dap-an


\textbf{{QUESTION}}

Phương trình nào sau đây là phương trình bậc nhất một ẩn ?
A. 0x + 1 = 0;
B. x – 1 = x + 2;
C. 3x2 + 2 = 0;
D. –3x = 2.

\textbf{{ANSWER}}

Đáp án đúng là: D
Đáp án A không phải phương trình bậc nhất một ẩn vì hệ số của x là 0. 
Đáp án B không phải phương trình bậc nhất một ẩn vì:
x – 1 = x + 2 
0 = 3 
Đáp án C không phải phương trình bậc nhất một ẩn vì x có bậc là 2. 
Đáp án D là phương trình bậc nhất một ẩn.

========================================================================

https://khoahoc.vietjack.com/thi-online/giai-sbt-toan-8-kntt-on-tap-chuong-7-co-dap-an


\textbf{{QUESTION}}

Tập nghiệm S của phương trình 3(x + 1) – (x – 2) = 7 – 2x là:
A. S = {0};
B. S = {12}$\left\{ {\frac{1}{2}} \right\}$;
C. S = ∅;
D. S = ℝ.

\textbf{{ANSWER}}

Đáp án đúng là: B
Ta có:
3(x + 1) – (x – 2) = 7 – 2x
3x + 3 – x + 2 = 7 – 2x 
3x – x + 2x = 7 – 3 – 2
4x = 2 
x = 12$\frac{1}{2}$
Vậy tập nghiệm của phương trình là: S = {12}$\left\{ {\frac{1}{2}} \right\}$.

========================================================================

https://khoahoc.vietjack.com/thi-online/giai-sbt-toan-8-kntt-on-tap-chuong-7-co-dap-an


\textbf{{QUESTION}}

Hàm số nào sau đây là hàm số bậc nhất ?
A. y = 0x + 3;
B. y = 2x2 + 5;
C. y = –x;
D. y = 0.

\textbf{{ANSWER}}

Đáp án đúng là: C
Hàm số y = 0x + 3 có hệ số của x là 0 nên không là hàm số bậc nhất.
Hàm số y = 2x2 + 5 là không là hàm số bậc nhất vì bậc của x là 2.
Hàm số y = –x là hàm số bậc nhất.
Hàm số y = 0 là hàm số hằng.

========================================================================

https://khoahoc.vietjack.com/thi-online/giai-sbt-toan-8-kntt-on-tap-chuong-7-co-dap-an


\textbf{{QUESTION}}

Phương trình đường thẳng có hệ số góc –2 và đi qua điểm (1; 3) là
A. y = –2x + 3;
B. y = –2x + 1; 
C. y = –2x + 4;

\textbf{{ANSWER}}

Đáp án đúng là: D
Gọi phương trình đường thẳng cần tìm là: y = ax + b.
Phương trình đường thẳng có hệ số góc –2 nên a = –2. 
Đường thẳng đi qua điểm (1; 3) nên ta có:
3 = –2 . 1 + b
3 = –2 + b
b = 5 
Vậy phương trình đường thẳng cần tìm là: y = –2x + 5.

========================================================================

https://khoahoc.vietjack.com/thi-online/giai-sbt-toan-8-kntt-on-tap-chuong-7-co-dap-an


\textbf{{QUESTION}}

Hệ số góc của đường thẳng y=1−4x2$y = \frac{{1 - 4x}}{2}$ là
A. –4.          
B. 1.            
C. 12$\frac{1}{2}$.           
D. –2.

\textbf{{ANSWER}}

Đáp án đúng là: D
Ta có y=1−4x2=12−42x=−2x+12$y = \frac{{1 - 4x}}{2} = \frac{1}{2} - \frac{4}{2}x = - 2x + \frac{1}{2}$.
Do đó, hệ số góc là –2.

========================================================================

https://khoahoc.vietjack.com/thi-online/de-thi-thu-thpt-quoc-gia-mon-toan-co-chon-loc-va-loi-giai-chi-tiet-20-de


\textbf{{QUESTION}}

Xác định vị trí tương đối giữa đường thẳng $$ d:\left\{\begin{array}{l}x=1-t\\ y=3+2t\\ z=t\end{array}\right.\phantom{\rule{0ex}{0ex}}$$và $$ \left(P\right):x-2y-z+6=0$$ ?

\textbf{{ANSWER}}

Đáp án B

========================================================================

https://khoahoc.vietjack.com/thi-online/giai-sgk-toan-11-canh-dieu-bai-tap-cuoi-chuong-vi-co-dap-an


\textbf{{QUESTION}}

Điều kiện xác định của x–3 là:
A. x ∈ ℝ.      
B. x ≥ 0.       
C. x ≠ 0.       
D. x > 0.

\textbf{{ANSWER}}

Đáp án đúng là: C
Ta có $$ {x}^{–3}=\frac{1}{{x}^{3}}$$
Khi đó hàm số $$ {x}^{–3}=\frac{1}{{x}^{3}}$$  xác định ⇔ x ≠ 0.

========================================================================

https://khoahoc.vietjack.com/thi-online/giai-sgk-toan-11-canh-dieu-bai-tap-cuoi-chuong-vi-co-dap-an


\textbf{{QUESTION}}

Điều kiện xác định của $$ {x}^{\frac{3}{5}}$$ là:
A. x ∈ ℝ.      
B. x ≥ 0.       
C. x ≠ 0.       
D. x > 0.

\textbf{{ANSWER}}

Đáp án đúng là: A
Ta có: $$ {x}^{\frac{3}{5}}=\sqrt[5]{{x}^{3}}$$
Khi đó hàm số $$ {x}^{\frac{3}{5}}=\sqrt[5]{{x}^{3}}$$  xác định với mọi x ∈ ℝ.

========================================================================

https://khoahoc.vietjack.com/thi-online/de-thi-hoc-ki-1-toan-9-co-dap-an-nam-2022-2023/116651


\textbf{{QUESTION}}

Giải phương trình và hệ phương trình:

\textbf{{ANSWER}}

a) x(x – 4) + 9x = 6
$$ \Leftrightarrow $$ x2 – 4x + 9x – 6 = 0
$$ \Leftrightarrow $$ x2 + 5x – 6 = 0
Vì a + b + c = 1 + 5 – 6 = 0 nên phương trình có hai nghiệm x1 = 1; x2 = -6

========================================================================

https://khoahoc.vietjack.com/thi-online/de-thi-hoc-ki-1-toan-9-co-dap-an-nam-2022-2023/116651


\textbf{{QUESTION}}

b) $$ {x}^{2}-(\sqrt{5}+2)x+2\sqrt{5}=0$$

\textbf{{ANSWER}}

b) $$ {x}^{2}-(\sqrt{5}+2)x+2\sqrt{5}=0$$
$$ \triangle ={\left[-(\sqrt{5}+2)\right]}^{2}-\mathrm{4.1.2}\sqrt{5}=9-4\sqrt{5}$$
$$ \sqrt{\Delta }=\sqrt{9-4\sqrt{5}}=\sqrt{{\left(\sqrt{5}-2\right)}^{2}}=\sqrt{5}-2\phantom{\rule{0ex}{0ex}}{x}_{1}=\frac{\sqrt{5}+2+\sqrt{5}-2}{2}=\sqrt{5};\quad {x}_{2}=\frac{\sqrt{5}+2-\sqrt{5}+2}{2}=2$$

========================================================================

https://khoahoc.vietjack.com/thi-online/de-thi-hoc-ki-1-toan-9-co-dap-an-nam-2022-2023/116651


\textbf{{QUESTION}}

c) x4 + 2x2 – 24 = 0

\textbf{{ANSWER}}

c) x4 + 2x2 – 24 = 0 (1)
Đặt t = x2 (với t ≥ 0)
(1) ⇔$$ \Leftrightarrow $$ t2 + 2t – 24 = 0 (2)
△$$ \triangle $$ = 22 – 4.1.(-24) = 100
Phương trình (2) có 2 nghiệm phân biệt 
t1=−2+102=4$$ {t}_{1}=\frac{-2+10}{2}=4$$ (nhận); t2=−2−102=−6$$ {t}_{2}=\frac{-2-10}{2}=-6$$(loại)

========================================================================

https://khoahoc.vietjack.com/thi-online/de-thi-hoc-ki-1-toan-9-co-dap-an-nam-2022-2023/116651


\textbf{{QUESTION}}

d) {x+2y=32x+3y=−1

\textbf{{ANSWER}}

d) {x+2y=32x+3y=−1⇔{2x+4y=62x+3y=−1⇔{y=72x+3y=−1⇔{y=7x=−11$$ \left\{\begin{array}{l}x+2y=3\\ 2x+3y=-1\end{array}\right.\Leftrightarrow \left\{\begin{array}{l}2x+4y=6\\ 2x+3y=-1\end{array}\right.\Leftrightarrow \left\{\begin{array}{l}y=7\\ 2x+3y=-1\end{array}\right.\Leftrightarrow \left\{\begin{array}{l}y=7\\ x=-11\end{array}\right.$$

========================================================================

https://khoahoc.vietjack.com/thi-online/de-thi-hoc-ki-1-toan-9-co-dap-an-nam-2022-2023/116651


\textbf{{QUESTION}}

Tìm toạ độ các giao điểm của hàm số (P): y=x2  và hàm số (D): y = 3x - 2
y=x2
y=x2
y
=
x2
x
2

\textbf{{ANSWER}}

Ta có phương trình hoành độ giao điểm của (P) và (D):  x2 = 3x – 2
⇔$$ \Leftrightarrow $$ x2 – 3x + 2 = 0
Ta có a + b + c = 1 – 3 + 2 = 0
Nên phương trình có 2 nghiệm:
x1 = 1; x2 = 2
Thay x vào hàm số y = 2x2
Khi x = 1 thì y = 1. Khi x = 2 thì y = 4

========================================================================

https://khoahoc.vietjack.com/thi-online/giai-sbt-toan-10-cd-bai-6-tich-vo-huong-cua-hai-vecto-co-dap-an


\textbf{{QUESTION}}

Cho tam giác ABC. Giá trị của $\overrightarrow {BA} .\overrightarrow {CA} $ bằng:
A. AB . AC . cos$\widehat {BAC}$.
B. – AB . AC . cos$\widehat {BAC}$.
C. AB . AC . cos$\widehat {ABC}$.
D. AB . AC . cos$\widehat {ACB}$.

\textbf{{ANSWER}}

Lời giải
Đáp án đúng là A
Xét tam giác ABC, có: 
$\overrightarrow {BA} .\overrightarrow {CA} = \left( { - \overrightarrow {AB} } \right).\left( { - \overrightarrow {AC} } \right) = BA.CA.c{\rm{os}}\left( { - \overrightarrow {AB} , - \overrightarrow {AC} } \right)$
= $BA.CA.c{\rm{os}}\widehat {BAC}$
= $BA.CA.c{\rm{os}}\left( {\widehat {BAC}} \right)$
Vậy chọn A.

========================================================================

https://khoahoc.vietjack.com/thi-online/giai-sbt-toan-10-cd-bai-6-tich-vo-huong-cua-hai-vecto-co-dap-an


\textbf{{QUESTION}}

Cho tam giác ABC. Giá trị của →AB.→BC bằng:
A. AB . BC . cos^ABC.
B. AB . AC . cos^ABC.
C. – AB . BC . cos^ABC.
D. AB . BC . cos^BAC.

\textbf{{ANSWER}}

Lời giải
Đáp án đúng là A
→AB.→BC=−→BA.→BC=−AB.BC.cos(−→BA,→BC)$\overrightarrow {AB} .\overrightarrow {BC} = - \overrightarrow {BA} .\overrightarrow {BC} = - AB.BC.\cos \left( { - \overrightarrow {BA} ,\overrightarrow {BC} } \right)$
= −AB.BC.cos(180∘−^ABC)$ - AB.BC.\cos \left( {180^\circ - \widehat {ABC}} \right)$
= AB.BC.cos^ABC$AB.BC.\cos \widehat {ABC}$.
Vậy chọn A.

========================================================================

https://khoahoc.vietjack.com/thi-online/giai-sbt-toan-10-cd-bai-6-tich-vo-huong-cua-hai-vecto-co-dap-an


\textbf{{QUESTION}}

Cho đoạn thẳng AB. Tập hợp các điểm M nằm trong mặt phẳng thỏa mãn →MA.→MB=0 là: 
A. Đường tròn tâm A bán kính AB.
B. Đường tròn tâm B bán kính AB. 
C. Đường trung trực của đoạn thẳng AB.
D. Đường tròn đường kính AB.

\textbf{{ANSWER}}

Lời giải
Đáp án đúng là D
Ta có: →MA.→MB=0
⇒ ^(→MA;→MB)=^AMB=90∘
Do đó tập hợp các điểm M thỏa mãn ^AMB=90∘ là đường tròn đường kính AB.

========================================================================

https://khoahoc.vietjack.com/thi-online/giai-sbt-toan-10-cd-bai-6-tich-vo-huong-cua-hai-vecto-co-dap-an


\textbf{{QUESTION}}

Nếu hai điểm M và N thỏa mãn →MN.→NM=−9 thì: 
A. MN = 9.
B. MN = 3.
C. MN = 81.
D. MN = 6.

\textbf{{ANSWER}}

Lời giải
Đáp án đúng là B
Ta có: →MN.→NM=MN.MN.cos(→MN,→NM)=MN.MN.cos180∘=−MN2
Mà →MN.→NM=−9 nên – MN2 = – 9 ⇔ MN2 = 9 ⇔ MN = 3 (thỏa mãn) hoặc MN = – 3 (không thỏa mãn).
Vậy MN = 3.

========================================================================

https://khoahoc.vietjack.com/thi-online/giai-sbt-toan-8-kntt-bai-10-tu-giac-co-dap-an


\textbf{{QUESTION}}

Chứng minh rằng cả bốn góc của một tứ giác không thể đều là góc nhọn, không thể đều là góc tù.

\textbf{{ANSWER}}

Vì tổng bốn góc của tứ giác bằng 360°, nên:
• Nếu cả bốn góc của tứ giác đều bé hơn 90° thì tổng của chúng bé hơn 360°, điều này vô lí.
• Nếu cả bốn góc của tứ giác đều lớn hơn 90° thì tổng của chúng lớn hơn 360°, điều này vô lí.

========================================================================

https://khoahoc.vietjack.com/thi-online/19-cau-trac-nghiem-mot-so-phuong-trinh-quy-ve-bac-nhat-hoac-bac-hai-co-dap-an


\textbf{{QUESTION}}

Tập nghiệm của phương trình $$ \left|5-2x\right|=\left|3x+3\right|$$ là:
A. $$ \left\{\frac{2}{5}\right\}$$
B. $$ \left\{-8\right\}$$
C. $$ \left\{\frac{2}{5};-8\right\}$$
D. $$ \varnothing $$

\textbf{{ANSWER}}

Ta có:
$$ \left|5-2x\right|=\left|3x+3\right|\Leftrightarrow [\begin{array}{c}5-2x=3x+3\\ 5-2x=-3x-3\end{array}\phantom{\rule{0ex}{0ex}}\Leftrightarrow [\begin{array}{c}-5x=-2\\ x=-8\end{array}\Leftrightarrow [\begin{array}{c}x=\frac{2}{5}\\ x=-8\end{array}$$
Vậy tập nghiệm của phương trình đã cho là: $$ \left\{\frac{2}{5};-8\right\}$$.
 
 Chọn C.

========================================================================

https://khoahoc.vietjack.com/thi-online/19-cau-trac-nghiem-mot-so-phuong-trinh-quy-ve-bac-nhat-hoac-bac-hai-co-dap-an


\textbf{{QUESTION}}

Cho phương trình có tham số m:
(2m+1)x-mx-1=x+m$$ \frac{\left(2m+1\right)x-m}{x-1}=x+m$$.     (*)
Chọn khẳng định đúng trong các khẳng định sau:
A. Phương trình luôn có hai nghiệm phân biệt
B. Khi m≠-2$$ m\ne -2$$ thì phương trình có hai nghiệm phân biệt
C. Khi m≠-1$$ m\ne -1$$ thì phương trình có hai nghiệm phân biệt
D. Khi m≠-1$$ m\ne -1$$ và m≠-2$$ m\ne -2$$ thì phương trình có hai nghiệm phân biệt

\textbf{{ANSWER}}

Trước hết phải chú ý điều kiện xác định của phương trình là x≠1$$ x\ne 1$$.
Ta có: (2m+1)x-mx-1=x+m$$ \frac{\left(2m+1\right)x-m}{x-1}=x+m$$
Suy ra: (2m + 1) x- m =  (x+ m). (x- 1)
⇔2mx+x-m=x2-x+mx-m$$ \Leftrightarrow 2mx+x-m={x}^{2}-x+mx-m$$
⇔x2-2x-mx=0⇔x2-(2+m)x=0$$ \Leftrightarrow {x}^{2}-2x-mx=0\Leftrightarrow {x}^{2}-\left(2+m\right)x=0$$
⇔x[x-(2+m)]=0⇔[x=0x=2+m$$ \Leftrightarrow x\left[x-\left(2+m\right)\right]=0\Leftrightarrow [\begin{array}{c}x=0\\ x=2+m\end{array}$$
 Khi m = 2 thì hai nghiệm bằng nhau đều bằng 0.
Khi m = -1 thì x = 1 ( không thỏa mãn điều kiện) nên không phải là nghiệm.
Vì vậy các phương án A B, C sai. Đáp án là D.

========================================================================

https://khoahoc.vietjack.com/thi-online/19-cau-trac-nghiem-mot-so-phuong-trinh-quy-ve-bac-nhat-hoac-bac-hai-co-dap-an


\textbf{{QUESTION}}

Cho phương trình có tham số m: (mx+1)√x-1=0.    (*)
  Chọn khẳng định đúng trong các khẳng định sau:
A. Khi m > 0 thì phương trình có hai nghiệm phân biệt
B. Khi m = -1 thì phương trình có hai nghiệm phân biệt
C. Khi m < -1 thì phương trình có hai nghiệm phân biệt
D. Khi -1 < m < 0 thì phương trình có hai nghiệm phân biệt

\textbf{{ANSWER}}

Trước hết phải chú ý đến điều kiện xác định của phương trình (*) là x≥1.
Ta có: (mx+1)√x-1=0
⇒[mx+1=0 (1)x-1=0 
* Xét x- 1 = 0⇔ x= 1.
* Xét mx +1= 0    (1)
+  Nếu m > 0  thì phương trình (1) có nghiệm x=-1m<0( không thỏa mãn điều kiện x) nên không là nghiệm của phương trình. Vậy phương án A sai.
 + Nếu m = -1 thì (1) trở thành:  -x + 1 = 0 nên x= 1.
Do đó, phương trình (*) có hai nghiệm trùng nhau: x= 1.
vậy phương án B sai.
+  Nếu m < -1 thì nghiệm của phương trình (1) là: x=-1m- số dương nhỏ hơn 1, không thỏa mãn điều kiện. Vậy phương án C sai.
+  Nếu -1 < m < 0 thì phương trình mx + 1 = 0 có nghiệm x=-1m lớn hơn 1, do vậy phương trình (*) có hai nghiệm phân biệt. Đáp án là D.
 
 
Chọn D.

========================================================================

https://khoahoc.vietjack.com/thi-online/19-cau-trac-nghiem-mot-so-phuong-trinh-quy-ve-bac-nhat-hoac-bac-hai-co-dap-an


\textbf{{QUESTION}}

Tập nghiệm của phương trình |5+2x|=|3x-2| là
A. {7}
B. {-35}
C. {7;-35}
D. tập hợp có nhiều hơn hai

\textbf{{ANSWER}}

Ta có:
|5+2x|=|3x-2|⇔[5+2x=3x-25+2x=-3x+2⇔[-x=-75x=-3⇔[x=7x=-35
Vậy phương trình đã cho có 2 nghiệm x = 7; x = -35

========================================================================

https://khoahoc.vietjack.com/thi-online/19-cau-trac-nghiem-mot-so-phuong-trinh-quy-ve-bac-nhat-hoac-bac-hai-co-dap-an


\textbf{{QUESTION}}

Tập nghiệm của phương trình 3x+1=x2+2x-3 là:
A. 1-172;1+172;-5-332
B. 1+172;-5+332
C. 1-172;1+172;-5-332;-5+332
D. 1+172;-5-332

\textbf{{ANSWER}}

Ta có: 3x+1=x2+2x-3  (1)
* Trường hợp 1:  Nếu x≥-13thì 3x+1≥0⇒3x+1=3x+1
Do đó, phương  trình (1) trở thành:  3x + 1 =  x2 + 2x – 3.
Hay -x2 +  x+  4= 0 ⇔[x=1+172(tm)x=1-172(l)
* Trường hợp 2. Nếu x<-13thì 3x+1<0⇒3x+1=-3x-1
Do đó, phương  trình (1) trở thành:   - 3x - 1 =  x2 + 2x – 3.
Hay – x2 – 5x + 2 = 0 ⇔[x=-5+332(l)x=-5-332(tm)
Vậy phương trình đã cho có 2 nghiệm là:  S=1+172;-5-332

========================================================================

https://khoahoc.vietjack.com/thi-online/10-cau-trac-nghiem-so-trung-binh-cong-trung-vi-phuong-sai-mot-do-lech-chuan-co-dap-an


\textbf{{QUESTION}}

Thống kê một điểm kiểm tra 45 phút của 40 học sinh của một lớp 10 năm học 2017 - 2018 cho ta kết quả như sau:
 
 
3
5
7
9
10
6
8
3
4
6
5
7
8
10
 
 
 
9
3
6
4
7
8
9
10
6
9
7
4
5
3
 
 
 
3
7
9
6
10
8
7
5
4
8
9
7
 
 
 
Chọn phát biểu sai trong các phát biểu sau:
A. Trong 40 số liệu thống kê trên, số giá trị khác nhau là 8
B. Giá trị 9 có tần số là 6
C. Giá trị 10 có tần suất là 10%
D. Giá trị 10 có tần suất là 4

\textbf{{ANSWER}}

+ Các giá trị khác nhau: $$ {x}_{1}=3,\quad {x}_{2}=4,\quad {x}_{3}=5,\quad {x}_{4}=6,\quad {x}_{5}=7,\quad {x}_{6}=8,\quad {x}_{7}=9,\quad {x}_{8}=10$$ $$ \Rightarrow A$$ đúng.
+ Giá trị x7 = 9 xuất hiện 6 lần $$ \Rightarrow $$ Tân số là 6 $$ \Rightarrow $$ B đúng.
 
+ Giá trị x8= 10 xuất hiện 4 lần $$ \Rightarrow $$ Tần suất là $$ \frac{4}{10}$$ hay $$ 10\%\Rightarrow $$ C đúng $$ \Rightarrow $$ D sai.
 
Đáp án D.

========================================================================

https://khoahoc.vietjack.com/thi-online/10-cau-trac-nghiem-so-trung-binh-cong-trung-vi-phuong-sai-mot-do-lech-chuan-co-dap-an


\textbf{{QUESTION}}

Cho bảng phân bố tần số sau:
Lớp
Cộng
Tần số
2
7
15
8
3
35
a) Trong các giá trị sau đây, giá trị nào gần nhất với số trung bình của bảng số liệu trên?
A. 31,5
B. 32
C. 32,5
D. 33

\textbf{{ANSWER}}

Ta bổ sung thêm một cột ghi giá trị đại diện của mỗi lớp:
Lớp
Cộng
Tần số
2
7
15
8
3
35
Giá trị đại diện
22,5
27,5
32,5
37,5
42,5
 
Áp dụng công thức ta tìm được số trung bình:
x=22,5.2+27,5.7+32,5.15+37,5.8+42,5.335≈32,93
Đáp án là D.

========================================================================

https://khoahoc.vietjack.com/thi-online/10-cau-trac-nghiem-so-trung-binh-cong-trung-vi-phuong-sai-mot-do-lech-chuan-co-dap-an


\textbf{{QUESTION}}

Cho bảng phân bố tần số sau:
Lớp
Cộng
Tần số
2
7
15
8
3
35
b) Trong các giá trị sau đây, giá trị nào gần nhất với độ lệch chuẩn của bảng số liệu trên?
A. 4
B. 4,5
C. 5
D. 6,5

\textbf{{ANSWER}}

Công thức tính số trung bình
ˉx=22,5.2+27,5.7+32,5.15+37,5.8+42,5.335=32,93
Công thức độ lệch chuẩn
s2=(22,5-32,93)2.2+(27,5-32,93)2.7+(32,5-32,93)2.1535+(37,5-32,93)2.8+(42,5-32,93)235=24,82
Suy ra s≈4,97.
Chọn đáp án C.

========================================================================

https://khoahoc.vietjack.com/thi-online/giai-sbt-toan-10-bai-tap-on-tap-cuoi-nam-co-dap-an5


\textbf{{QUESTION}}

Cho các mệnh đề:
P: “Phương trình bậc hai ax2 + bx + c = 0 có hai nghiệm phân biệt”;
Q: “Phương trình bậc hai ax2 + bx + c = 0 có biệt thức ∆ = b2 – 4ac > 0”.
Hãy phát biểu các mệnh đề: P ⇒ Q, Q ⇒ P, P ⇔ Q,  . Xét tính đúng sai của các mệnh đề này.

\textbf{{ANSWER}}

Hướng dẫn giải
+ Mệnh đề P ⇒ Q: “Nếu phương trình bậc hai ax2 + bx + c = 0 có hai nghiệm phân biệt thì phương trình bậc hai ax2 + bx + c = 0 có biệt thức ∆ = b2 – 4ac > 0”. Đây là mệnh đề đúng.
+ Mệnh đề Q ⇒ P: “ Nếu phương trình bậc hai ax2 + bx + c = 0 có biệt thức ∆ = b2 – 4ac > 0 thì phương trình bậc hai ax2 + bx + c = 0 có hai nghiệm phân biệt”. Đây là mệnh đề đúng.
+ Mệnh đề P ⇔ Q: “Phương trình bậc hai ax2 + bx + c = 0 có hai nghiệm phân biệt khi và chỉ khi phương trình bậc hai ax2 + bx + c = 0 có biệt thức ∆ = b2 – 4ac > 0”. Do P ⇒ Q, Q ⇒ P đều là các mệnh đề đúng nên mệnh đề P ⇔ Q là mệnh đề đúng. 
+ Mệnh đề  
Mệnh đề   là mệnh đề phủ định của mệnh đề P và được phát biểu là: “Phương trình bậc hai ax2 + bx + c = 0 không có hai nghiệm phân biệt”.
Mệnh đề   là mệnh đề phủ định của mệnh đề Q và được phát biểu là: “Phương trình bậc hai ax2 + bx + c = 0 có biệt thức ∆ = b2 – 4ac ≤ 0”.
Khi đó, ta phát biểu mệnh đề  : “Nếu phương trình bậc hai ax2 + bx + c = 0 không có hai nghiệm phân biệt thì phương trình bậc hai ax2 + bx + c = 0 có biệt thức ∆ = b2 – 4ac ≤ 0”. Mệnh đề này là mệnh đề đúng.

========================================================================

https://khoahoc.vietjack.com/thi-online/giai-sbt-toan-10-bai-tap-on-tap-cuoi-nam-co-dap-an5


\textbf{{QUESTION}}

Dùng các khái niệm “điều kiện cần” và “điều kiện đủ” để diễn tả mệnh đề P ⇒ Q.
Dùng các khái niệm “điều kiện cần” và “điều kiện đủ” để diễn tả mệnh đề P 
⇒
 Q.

\textbf{{ANSWER}}

Hướng dẫn giải
+ Phương trình bậc hai ax2 + bx + c = 0 có hai nghiệm phân biệt là điều kiện đủ để phương trình bậc hai ax2 + bx + c = 0 có biệt thức ∆ = b2 – 4ac > 0. 
+ Phương trình bậc hai ax2 + bx + c = 0 có biệt thức ∆ = b2 – 4ac > 0 là điều kiện cần để phương trình bậc hai ax2 + bx + c = 0 có hai nghiệm phân biệt.

========================================================================

https://khoahoc.vietjack.com/thi-online/giai-sbt-toan-10-bai-tap-on-tap-cuoi-nam-co-dap-an5


\textbf{{QUESTION}}

Gọi X là tập hợp các phương trình bậc hai ax2 + bx + c = 0 có hai nghiệm phân biệt, Y là tập hợp các phương trình bậc hai ax2 + bx + c = 0 có hệ số a và c trái dấu. Nêu mối quan hệ giữa hai tập hợp X và Y.

\textbf{{ANSWER}}

Hướng dẫn giải
Ta có các phương trình bậc hai ax2 + bx + c = 0 có hệ số a và c trái dấu thì luôn có hai nghiệm trái dấu, hiển nhiên đây là hai nghiệm phân biệt. Nhưng các phương trình bậc hai ax2 + bx + c = 0 có hai nghiệm phân biệt thì hai nghiệm này chưa chắc đã trái dấu. 
Do đó mọi phần tử của tập hợp Y thì đều là phần tử của tập hợp X. 
Vậy Y là tập con của tập hợp X và ta viết Y ⊂ X.

========================================================================

https://khoahoc.vietjack.com/thi-online/de-thi-toan-6-hoc-ki-2-co-dap-an/103274


\textbf{{QUESTION}}

A. $$ \frac{-1}{8}$$
B. 8
C. $$ \frac{-1}{-8}$$
D. $$ \frac{1}{8}$$

\textbf{{ANSWER}}

Đáp án đúng là: A
Định nghĩa: Hai số gọi là nghịch đảo của nhau nếu tích của chúng bằng 1. 
Phương án A đúng vì phân số $$ \frac{-1}{8}$$ là nghịch đảo của số – 8 vì  $$ \left(-8\right).\left(-\frac{1}{8}\right)=1$$;
Phương án B sai vì (– 8).8 = – 64 ≠ 1;
Phương án C sai vì $$ \left(-8\right).\left(\frac{-1}{-8}\right)=\left(-8\right).\frac{1}{8}=-1$$≠ 1;
Phương án D sai vì $$ \left(-8\right).\frac{1}{8}=-1$$≠ 1.

========================================================================

https://khoahoc.vietjack.com/thi-online/de-thi-toan-6-hoc-ki-2-co-dap-an/103274


\textbf{{QUESTION}}

Dữ liệu nào sau đâu không phải là số liệu?
A. Số quyển sách mỗi bạn học sinh trong lớp đọc trong một năm.
B. Chiều cao của các bạn học sinh trong lớp.
C. Phương tiện đi học của các bạn học sinh trong lớp.
D. Cân nặng của các bạn học sinh trong lớp.

\textbf{{ANSWER}}

Đáp án đúng là: C
Phương án A đúng số quyển sách mỗi bạn học sinh trong lớp đọc trong một năm là một dữ liệu dạng số liệu.
Phương án B đúng chiều cao của các bạn học sinh trong lớp là một dữ liệu thuộc dạng số liệu.
Phương án C sai vì phương tiện đi học của các bạn học sinh trong lớp không phải là dữ liệu dạng số liệu.
Phương án D đúng vì cân nặng của của học sinh trong lớp là dạng dữ liệu dạng số liệu.

========================================================================

https://khoahoc.vietjack.com/thi-online/de-thi-toan-6-hoc-ki-2-co-dap-an/103274


\textbf{{QUESTION}}

A. 13000
B. 1300000
C. 1300
D. 130000

\textbf{{ANSWER}}

Đáp án đúng là: D
1740 m = 174000 cm;
Tỉ lệ xích của bản đồ là:
5,8174000=58174000.10=5858.3.10000=13.10000=130000

========================================================================

https://khoahoc.vietjack.com/thi-online/de-thi-toan-6-hoc-ki-2-co-dap-an/103274


\textbf{{QUESTION}}

A. (-1)
B. −1325
C. −13
D. −3925

\textbf{{ANSWER}}

Đáp án đúng là: D
(45+115):(13−89)=(4.35.3+115):(1.33.3−89)=(1215+115):(39−89)=1315:(−59)=1315.(−95)=−13.35.5=−3925

========================================================================

https://khoahoc.vietjack.com/thi-online/15-cau-trac-nghiem-phuong-trinh-duong-tron-co-dap-an-nhan-biet


\textbf{{QUESTION}}

Đường tròn tâm I (a; b) và bán kính R có dạng
A. $$ {\left(x+a\right)}^{2}+{\left(y+b\right)}^{2}={R}^{2}$$
B. $$ {\left(x-a\right)}^{2}+{\left(y-b\right)}^{2}={R}^{2}$$
C. $$ {\left(x-a\right)}^{2}+{\left(y+b\right)}^{2}={R}^{2}$$
D. $$ {\left(x+a\right)}^{2}+{\left(y-b\right)}^{2}={R}^{2}$$

\textbf{{ANSWER}}

Phương trình đường tròn (C)  tâm I (a; b), bán kính R là : $$ {\left(x-a\right)}^{2}+{\left(y-b\right)}^{2}={R}^{2}$$
Đáp án cần chọn là: B

========================================================================

https://khoahoc.vietjack.com/thi-online/15-cau-trac-nghiem-phuong-trinh-duong-tron-co-dap-an-nhan-biet


\textbf{{QUESTION}}

Đường tròn (x+a)2+(y+b)2=R2$$ {\left(x+a\right)}^{2}+{\left(y+b\right)}^{2}={R}^{2}$$ có tọa độ tâm I và bán kính lần lượt là:
A. I(a;b), R
B. I(-a;-b), R
C. I(a;b), R2$$ I\left(a;b\right),\quad {R}^{2}$$
D. I(-a;-b), R2$$ I\left(-a;-b\right),\quad {R}^{2}$$

\textbf{{ANSWER}}

Đường tròn (x+a)2+(y+b)2=R2$$ {\left(x+a\right)}^{2}+{\left(y+b\right)}^{2}={R}^{2}$$ có tâm I(-a;-b) và bán kính R
Đáp án cần chọn là: B

========================================================================

https://khoahoc.vietjack.com/thi-online/15-cau-trac-nghiem-phuong-trinh-duong-tron-co-dap-an-nhan-biet


\textbf{{QUESTION}}

Đường tròn tâm I(a;b) và bán kính R có phương trình (x-a)2+(y-b)2=R2 được viết lại thành x2+y2-2ax-2by+c=0. Khi đó biểu thức nào sau đây đúng?
A. c=a2+b2-R2
B. c=a2-b2-R2
C. c=-a2+b2-R2
D. c=R2-a2-b2

\textbf{{ANSWER}}

Phương trình đường tròn x2+y2-2ax-2by+c=0$$ {x}^{2}+{y}^{2}-2ax-2by+c=0$$ có tâm I(a;b) và bán kính R=√a2+b2-c$$ R=\sqrt{{a}^{2}+{b}^{2}-c}$$
Do đó: c=a2+b2-R2$$ c={a}^{2}+{b}^{2}-{R}^{2}$$
Đáp án cần chọn là: A

========================================================================

https://khoahoc.vietjack.com/thi-online/15-cau-trac-nghiem-phuong-trinh-duong-tron-co-dap-an-nhan-biet


\textbf{{QUESTION}}

Đường tròn có phương trình x2+y2+2ax+2by+c=0 có tâm và bán kính lần lượt là:
A. I(-a;-b), R=√a2+b2-c2
B. I(-a;-b), R=√a2+b2-c
C. I(a;b), R=√a2+b2-c2
D. I(a;b), R=√a2+b2-c

\textbf{{ANSWER}}

Đường tròn x2+y2+2ax+2by+c=0 có tâm I(-a;-b) và bán kính  R=√a2+b2-c
Đáp án cần chọn là: B

========================================================================

https://khoahoc.vietjack.com/thi-online/15-cau-trac-nghiem-phuong-trinh-duong-tron-co-dap-an-nhan-biet


\textbf{{QUESTION}}

Cho đường tròn có phương trình (C): x2+y2+2ax+2by+c=0. Khẳng định nào sau đây là sai?
A. Đường tròn có tâm là I (a; b)
B. Đường tròn có bán kính là R=a2+b2-c
C. a2 + b2 – c > 0
D. Tâm của đường tròn là I (−a; −b)

\textbf{{ANSWER}}

Phương trình x2+y2+2ax+2by+c=0 với điều kiện a2 + b2 – c > 0, là phương trình đường tròn tâm I (−a; −b), bán kính R=√a2+b2-c
Do đó đáp án A sai
Đáp án cần chọn là: A

========================================================================

https://khoahoc.vietjack.com/thi-online/35-de-minh-hoa-thpt-quoc-gia-mon-toan-nam-2022-co-loi-giai-x/75422


\textbf{{QUESTION}}

Diện tích xung quanh của hình nón có độ dài đường sinh $$ l$$ và bán kính $$ r$$ bằng
A. $$ \pi rl$$
B. $$ 2\pi rl$$
C. $$ \frac{1}{3}\pi rl$$
D. $$ 4\pi rl$$

\textbf{{ANSWER}}

Chọn A
Ta có: Diện tích xung quanh của hình nón có độ dài đường sinh $$ l$$ và bán kính $$ r$$ là $$ {S}_{xq}=\pi rl.$$

========================================================================

https://khoahoc.vietjack.com/thi-online/35-de-minh-hoa-thpt-quoc-gia-mon-toan-nam-2022-co-loi-giai-x/75422


\textbf{{QUESTION}}

Cho cấp số cộng $$ \left({u}_{n}\right)$$ với $$ {u}_{1}=2$$ và $$ {u}_{2}=8$$. Công sai của cấp số cộng bằng
A. $$ -6$$
B. $$ 4$$
C. $$ 10$$
D. $$ 6$$

\textbf{{ANSWER}}

Chọn D
Ta có: $$ d={u}_{2}-{u}_{1}=8-2=6$$.
Vậy công sai của cấp số cộng là: $$ d=6$$.

========================================================================

https://khoahoc.vietjack.com/thi-online/35-de-minh-hoa-thpt-quoc-gia-mon-toan-nam-2022-co-loi-giai-x/115690


\textbf{{QUESTION}}

Có bao nhiêu cách sắp xếp 5 học sinh thành một hàng dọc?
A. $$ {5}^{5}$$.

\textbf{{ANSWER}}

Chọn B.
Số cách sắp xếp 5 học sinh thành một hàng dọc là $$ 5!$$.

========================================================================

https://khoahoc.vietjack.com/thi-online/35-de-minh-hoa-thpt-quoc-gia-mon-toan-nam-2022-co-loi-giai-x/115690


\textbf{{QUESTION}}

Cho cấp số cộng có $$ {u}_{1}=-3$$, $$ d=4$$. Chọn khẳng định đúng trong các khẳng định sau?
A. $$ {u}_{5}=15$$.

\textbf{{ANSWER}}

Chọn C.
Ta có $$ {u}_{3}={u}_{1}+2d$$$$ =-3+2.4$$$$ =5$$.

========================================================================

https://khoahoc.vietjack.com/thi-online/35-de-minh-hoa-thpt-quoc-gia-mon-toan-nam-2022-co-loi-giai-x/115690


\textbf{{QUESTION}}

A. x = 3.

\textbf{{ANSWER}}

Chọn C.
Ta có, log2(x−5)=4⇔x−5=16⇔x=21$$ {\mathrm{log}}_{2}\left(x-5\right)=4\Leftrightarrow x-5=16\Leftrightarrow x=21$$.

========================================================================

https://khoahoc.vietjack.com/thi-online/35-de-minh-hoa-thpt-quoc-gia-mon-toan-nam-2022-co-loi-giai-x/115690


\textbf{{QUESTION}}

A. 12a2$$ 12{a}^{2}$$.

\textbf{{ANSWER}}

Chọn C.
Áp dụng công thức thể tích khối lăng trụ ta có được: V=Sđ.h=4a2.3a=12a3$$ V={S}_{đ}.h=4{a}^{2}.3a=12{a}^{3}$$.

========================================================================

https://khoahoc.vietjack.com/thi-online/35-de-minh-hoa-thpt-quoc-gia-mon-toan-nam-2022-co-loi-giai-x/115690


\textbf{{QUESTION}}

Tập xác định của hàm số y=log3(4−x)$$ y={\mathrm{log}}_{3}\left(4-x\right)$$ là
A. (4;  +∞)$$ \left(4;\text{\hspace{0.17em}\hspace{0.17em}}+\infty \right)$$.

\textbf{{ANSWER}}

Chọn C.
Điều kiện 4−x>0$$ 4-x>0$$ ⇔x<4$$ \Leftrightarrow x<4$$.

========================================================================

https://khoahoc.vietjack.com/thi-online/giai-sgk-toan-11-kntt-bai-3-ham-so-luong-giac-co-dap-an


\textbf{{QUESTION}}

Giả sử vận tốc v (tính bằng lít/giây) của luồng khí trong một chu kì hô hấp (tức là thời gian từ lúc bắt đầu của một nhịp thở đến khi bắt đầu của nhịp thở tiếp theo) của một người nào đó ở trạng thái nghỉ ngơi được cho bởi công thức
$v = 0,85\sin \frac{{\pi t}}{3}$,
trong đó t là thời gian (tính bằng giây). Hãy tìm thời gian của một chu kì hô hấp đầy đủ và số chu kì hô hấp trong một phút của người đó.

\textbf{{ANSWER}}

Lời giải: 
Sau bài học này, ta sẽ giải quyết được bài toán trên như sau: 
Thời gian của một chu kì hô hấp đầy đủ chính là một chu kì tuần hoàn của hàm v(t) và là T = $\frac{{2\pi }}{{\frac{\pi }{3}}} = 6$ (giây). 
Ta có: 1 phút = 60 giây. 
Do đó, số chu kì hô hấp trong một phút của người đó là $\frac{{60}}{6} = 10$ (chu kì).

========================================================================

https://khoahoc.vietjack.com/thi-online/giai-sgk-toan-11-kntt-bai-3-ham-so-luong-giac-co-dap-an


\textbf{{QUESTION}}

Hoàn thành bảng sau: 
x
sin x
cos x
tan x
cot x
π6$\frac{\pi }{6}$
?
?
?
?
0
?
?
?
?
−π2$ - \frac{\pi }{2}$
?
?
?
?

\textbf{{ANSWER}}

Lời giải: 
Lần lượt thay các giá trị $x = \frac{\pi }{6},\,x = 0$ và $x = - \frac{\pi }{2}$ vào sin x, cos x, tan x và cot x, ta hoàn thành được bảng như sau: 
x
sin x
cos x
tan x
cot x
$\frac{\pi }{6}$
$\frac{1}{2}$
$\frac{{\sqrt 3 }}{2}$
$\frac{{\sqrt 3 }}{3}$
$\sqrt 3 $
0
0
1
0
Không xác định
$ - \frac{\pi }{2}$
– 1 
0
Không xác định
0

========================================================================

https://khoahoc.vietjack.com/thi-online/giai-sgk-toan-11-kntt-bai-3-ham-so-luong-giac-co-dap-an


\textbf{{QUESTION}}

Tìm tập xác định của hàm số y=1sinx.$y = \frac{1}{{\sin x}}.$
y=1sinx.
y=1sinx.
y
=
1sinx
1
1
sinx
sinx


sinx
sinx
sinx
sin

x
.

\textbf{{ANSWER}}

Lời giải: 
Biểu thức 1sinx$\frac{1}{{\sin x}}$ có nghĩa khi sin x ≠ 0, tức là x ≠ kπ (k ∈ ℤ). 
Vậy tập xác định của hàm số y=1sinx$y = \frac{1}{{\sin x}}$ là ℝ \ {kπ | k ∈ ℤ}.

========================================================================

https://khoahoc.vietjack.com/thi-online/19-de-on-thi-vao-10-chuyen-hay-co-loi-giai/59336


\textbf{{QUESTION}}

Cho $$ a-b=\sqrt{29+12\sqrt{5}}-2\sqrt{5}$$ . Tính giá trị của biểu thức:
$$ A={a}^{2}(a+1)-{b}^{2}(b-1)-11ab+2015$$

\textbf{{ANSWER}}

$$ \begin{array}{l}a-b=\sqrt{29+12\sqrt{5}}-2\sqrt{5}=\sqrt{{\left(3+2\sqrt{5}\right)}^{2}}-2\sqrt{5}=3\\ A={a}^{3}-{b}^{3}+{a}^{2}+{b}^{2}-11ab+2015\\ =(a-b)({a}^{2}+{b}^{2}+ab)+{a}^{2}+{b}^{2}-11ab+2015\\ =3({a}^{2}+{b}^{2}+ab)+{a}^{2}+{b}^{2}-11ab+2015\\ =4({a}^{2}-2ab+{b}^{2})+2015=4{\left(a-b\right)}^{2}+2015=2051\end{array}$$

========================================================================

https://khoahoc.vietjack.com/thi-online/19-de-on-thi-vao-10-chuyen-hay-co-loi-giai/59336


\textbf{{QUESTION}}

Cho x, y là hai số thực thỏa mãn xy+√(1+x2)(1+y2)=1.$$ xy+\sqrt{(1+{x}^{2})(1+{y}^{2})}=1.$$ Chứng minh rằng x√1+y2+y√1+x2=0.$$ x\sqrt{1+{y}^{2}}+y\sqrt{1+{x}^{2}}=0.$$

\textbf{{ANSWER}}

$$ \begin{array}{l}xy+\sqrt{(1+{x}^{2})(1+{y}^{2})}=1\Leftrightarrow \sqrt{{(1+x)}^{2}{(1+y)}^{2}}=1-xy\\ \Rightarrow (1+{x}^{2})(1+{y}^{2})={\left(1-xy\right)}^{2}\\ \Leftrightarrow 1+{x}^{2}+{y}^{2}+{x}^{2}{y}^{2}=1-2xy+{x}^{2}{y}^{2}\\ \Leftrightarrow {x}^{2}+{y}^{2}+2xy=0\Leftrightarrow {\left(x+y\right)}^{2}=0\Leftrightarrow y=-x\\ \Rightarrow x\sqrt{1+{y}^{2}}+y\sqrt{1+{x}^{2}}=x\sqrt{1+{x}^{2}}-x\sqrt{1+{x}^{2}}=0\end{array}$$

========================================================================

https://khoahoc.vietjack.com/thi-online/19-de-on-thi-vao-10-chuyen-hay-co-loi-giai/59336


\textbf{{QUESTION}}

Giải phương trình 2x+3+√4x2+9x+2=2√x+2+√4x+1.$$ 2x+3+\sqrt{4{x}^{2}+9x+2}=2\sqrt{x+2}+\sqrt{4x+1}.$$

\textbf{{ANSWER}}

Pt ⇔2x+3+√(x+2)(4x+1)=2√x+2+√4x+1.$$ \Leftrightarrow 2x+3+\sqrt{(x+2)(4x+1)}=2\sqrt{x+2}+\sqrt{4x+1}.$$ ĐK: x≥−14$$ x\ge -\frac{1}{4}$$ 
Đặt t2=8x+4√(x+2)(4x+1)+9⇔2x+√(x+2)(4x+1)=t2−94$$ {t}^{2}=8x+4\sqrt{(x+2)(4x+1)}+9\Leftrightarrow 2x+\sqrt{(x+2)(4x+1)}=\frac{{t}^{2}-9}{4}$$ 
PTTT t2−4t+3=0⇔t=1$$ {t}^{2}-4t+3=0\Leftrightarrow t=1$$ hoặc t = 3
TH1. t = 1 giải ra vô nghiệm hoặc kết hợp với ĐK t≥√7$$ t\ge \sqrt{7}$$ bị loại
TH 2 t=3⇒2√x+2+√4x+1=3.$$ t=3\Rightarrow 2\sqrt{x+2}+\sqrt{4x+1}=3.$$ Giải pt tìm được x=−29$$ x=-\frac{2}{9}$$ (TM)
Vậy pt có nghiệm duy nhất x=−29$$ x=-\frac{2}{9}$$

========================================================================

https://khoahoc.vietjack.com/thi-online/19-de-on-thi-vao-10-chuyen-hay-co-loi-giai/59336


\textbf{{QUESTION}}

Giải hệ phương trình {2x2−y2+xy−5x+y+2=√y−2x+1−√3−3xx2−y−1=√4x+y+5−√x+2y−2$$ \left\{\begin{array}{l}2{x}^{2}-{y}^{2}+xy-5x+y+2=\sqrt{y-2x+1}-\sqrt{3-3x}\\ {x}^{2}-y-1=\sqrt{4x+y+5}-\sqrt{x+2y-2}\end{array}\right.$$

\textbf{{ANSWER}}

ĐK: y−2x+1≥0,4x+y+5≥0,x+2y−2≥0,x≤1$$ y-2x+1\ge 0,4x+y+5\ge 0,x+2y-2\ge 0,x\le 1$$
TH1: {y−2x+1=03−3x=0⇔{x=1y=1⇒{0=0−1=√10−1(ko t/m)TH2: x≠1,y≠1 $$ TH1:\quad \left\{\begin{array}{l}y-2x+1=0\\ 3-3x=0\end{array}\right.\Leftrightarrow \left\{\begin{array}{l}x=1\\ y=1\end{array}\right.\Rightarrow \left\{\begin{array}{l}0=0\\ -1=\sqrt{10}-1\end{array}\right.(ko\quad t/m)\phantom{\rule{0ex}{0ex}}TH2:\quad x\ne 1,y\ne 1\quad $$
Đưa pt thứ nhất về dạng tích ta được
(x+y−2)(2x−y−1)=x+y−2√y−2x+1+√3−3x(x+y−2)[1√y−2x+1+√3−3x+y−2x+1]=0⇒1√y−2x+1+√3−3x+y−2x+1>0⇒x+y−2=0$$ (x+y-2)(2x-y-1)=\frac{x+y-2}{\sqrt{y-2x+1}+\sqrt{3-3x}}\phantom{\rule{0ex}{0ex}}(x+y-2)\left[\frac{1}{\sqrt{y-2x+1}+\sqrt{3-3x}}+y-2x+1\right]=0\phantom{\rule{0ex}{0ex}}\Rightarrow \frac{1}{\sqrt{y-2x+1}+\sqrt{3-3x}}+y-2x+1>0\Rightarrow x+y-2=0$$
Thay y= 2-x vào pt thứ 2 ta được x2+x−3=√3x+7−√2−x$$ {x}^{2}+x-3=\sqrt{3x+7}-\sqrt{2-x}$$
⇔x2+x−2=√3x+7−1+2−√2−x⇔(x+2)(x−1)=3x+6√3x+7+1+2+x2+√2−x⇔(x+2)[3√3x+7+1+12+√2−x+1−x]=0$$ \begin{array}{l}\Leftrightarrow {x}^{2}+x-2=\sqrt{3x+7}-1+2-\sqrt{2-x}\\ \Leftrightarrow (x+2)(x-1)=\frac{3x+6}{\sqrt{3x+7}+1}+\frac{2+x}{2+\sqrt{2-x}}\\ \Leftrightarrow (x+2)\left[\frac{3}{\sqrt{3x+7}+1}+\frac{1}{2+\sqrt{2-x}}+1-x\right]=0\end{array}$$
Do x≤1⇒3√3x+7+1+12+√2−x+1−x>0$$ x\le 1\Rightarrow \frac{3}{\sqrt{3x+7}+1}+\frac{1}{2+\sqrt{2-x}}+1-x>0$$
Vậy x+2=0⇔x=−2⇒y=4$$ x+2=0\Leftrightarrow x=-2\Rightarrow y=4$$ (t/m)

========================================================================

https://khoahoc.vietjack.com/thi-online/19-de-on-thi-vao-10-chuyen-hay-co-loi-giai/59336


\textbf{{QUESTION}}

Tìm các số nguyên x, y thỏa mãnx4+x2−y2−y+20=0.$$ {x}^{4}+{x}^{2}-{y}^{2}-y+20=0.$$  (1)

\textbf{{ANSWER}}

Ta có (1) ⇔x4+x2+20=y2+y$$ \Leftrightarrow {x}^{4}+{x}^{2}+20={y}^{2}+y$$
Ta thấy: x4+x2<x4+x2+20≤x4+x2+20+8x2⇔x2(x2+1)<y(y+1)≤(x2+4)(x2+5)$$ {x}^{4}+{x}^{2}<{x}^{4}+{x}^{2}+20\le {x}^{4}+{x}^{2}+20+8{x}^{2}\phantom{\rule{0ex}{0ex}}\Leftrightarrow {x}^{2}({x}^{2}+1)<y(y+1)\le ({x}^{2}+4)({x}^{2}+5)$$
Vì x, y ∈ Z nên ta xét các trường hợp sau
+ TH1. y(y+1)=(x2+1)(x2+2)⇔x4+x2+20=x4+3x2+2⇔2x2=18⇔x2=9⇔x=±3$$ y(y+1)=({x}^{2}+1)({x}^{2}+2)\Leftrightarrow {x}^{4}+{x}^{2}+20={x}^{4}+3{x}^{2}+2\phantom{\rule{0ex}{0ex}}\Leftrightarrow 2{x}^{2}=18\Leftrightarrow {x}^{2}=9\Leftrightarrow x=\pm 3$$
Với x2=9 ⇒y2+y=92+9+20⇔y2+y−110=0⇔y=10;y=−11(t.m)$$ {x}^{2}=9\quad \Rightarrow {y}^{2}+y={9}^{2}+9+20\Leftrightarrow {y}^{2}+y-110=0\Leftrightarrow y=10;y=-11(t.m)$$
+ TH2 y(y+1)=(x2+2)(x2+3)⇔x4+x2+20=x4+5x2+6⇔4x2=14⇔x2=72 (loại)$$ y(y+1)=({x}^{2}+2)({x}^{2}+3)\Leftrightarrow {x}^{4}+{x}^{2}+20={x}^{4}+5{x}^{2}+6\phantom{\rule{0ex}{0ex}}\Leftrightarrow 4{x}^{2}=14\Leftrightarrow {x}^{2}=\frac{7}{2}\quad \left(loại\right)$$
+ TH3: y(y+1)=(x2+3)(x2+4)⇔6x2=8⇔x2=43 (loại)$$ y(y+1)=({x}^{2}+3)({x}^{2}+4)\Leftrightarrow 6{x}^{2}=8\Leftrightarrow {x}^{2}=\frac{4}{3}\quad \left(loại\right)$$
+ TH4: y(y+1)=(x2+4)(x2+5)⇔8x2=0⇔x2=0⇔x=0$$ y(y+1)=({x}^{2}+4)({x}^{2}+5)\Leftrightarrow 8{x}^{2}=0\Leftrightarrow {x}^{2}=0\Leftrightarrow x=0$$
Với x2=0$$ {x}^{2}=0$$ ta có y2+y=20⇔y2+y−20=0⇔y=−5;y=4$$ {y}^{2}+y=20\Leftrightarrow {y}^{2}+y-20=0\Leftrightarrow y=-5;y=4$$
Vậy PT đã cho có nghiệm nguyên (x;y) là :
(3;10), (3;-11), (-3; 10), (-3;-11), (0; -5), (0;4).

========================================================================

https://khoahoc.vietjack.com/thi-online/10-bai-tap-chsuyen-du-lieu-tu-dang-bieu-dien-nay-sang-dang-bieu-dien-khac-co-loi-giai


\textbf{{QUESTION}}

Khi chuyển từ biểu đồ tranh sang biểu đồ cột, cần phải

\textbf{{ANSWER}}

Đáp án đúng là: C
Khi chuyển từ biểu đồ tranh sang biểu đồ cột, cần phải chuyển số lượng các hình thành dạng số liệu và xây dựng biểu đồ.

========================================================================

https://khoahoc.vietjack.com/thi-online/10-bai-tap-chsuyen-du-lieu-tu-dang-bieu-dien-nay-sang-dang-bieu-dien-khac-co-loi-giai


\textbf{{QUESTION}}

Khi chuyển từ biểu đồ cột sang biểu đồ đoạn thẳng, cần phải

\textbf{{ANSWER}}

Đáp án đúng là: D
Khi chuyển từ biểu đồ cột sang biểu đồ đoạn thẳng, cần phải chuyển số liệu thành các điểm và nối với nhau.

========================================================================

https://khoahoc.vietjack.com/thi-online/10-bai-tap-chsuyen-du-lieu-tu-dang-bieu-dien-nay-sang-dang-bieu-dien-khac-co-loi-giai


\textbf{{QUESTION}}

Khi chuyển từ biểu đồ cột sang biểu đồ hình quạt tròn, cần phải:

\textbf{{ANSWER}}

Đáp án đúng là: D
Khi chuyển từ biểu đồ hình cột sang biểu đồ hình quạt tròn, cần phải chuyển các số liệu thành tỉ lệ và chia biểu đồ hình quạt tròn theo tỉ lệ tương ứng.

========================================================================

https://khoahoc.vietjack.com/thi-online/de-kiem-tra-giua-ki-1-toan-8-kntt-co-dap-an/133607


\textbf{{QUESTION}}

Trong các biểu thức đại số sau, biểu thức nào là đơn thức? 
A. $$ \frac{2x}{y}$$
B. $$ 3x+2y$$
C. $$ 4\left(x-y\right)$$
D. $$ -\frac{2}{3}x{y}^{2}$$

\textbf{{ANSWER}}

Chọn đáp án D

========================================================================

https://khoahoc.vietjack.com/thi-online/de-kiem-tra-giua-ki-1-toan-8-kntt-co-dap-an/133607


\textbf{{QUESTION}}

Đơn thức 25ax4y3z$$ 25a{x}^{4}{y}^{3}z$$  (với a là hằng số) có 
A.hệ số là 25, phần biến là ax4y3z$$ a{x}^{4}{y}^{3}z$$ .
B. hệ số là 25, phần biến là x4y3z$$ {x}^{4}{y}^{3}z$$ . 
D. hệ số là 25a, phần biến là ax4y3z$$ a{x}^{4}{y}^{3}z$$ .

\textbf{{ANSWER}}

Chọn đáp án C

========================================================================

https://khoahoc.vietjack.com/thi-online/de-kiem-tra-giua-ki-1-toan-8-kntt-co-dap-an/133607


\textbf{{QUESTION}}

Cho các biểu thức sau: 
 (5+y2)1x;  −89x2y ( 2x−3);    −12x2y;    22x3+13x3y4−x4z+x2;   15+1z$$ \left(5+{y}^{2}\right)\frac{1}{x};\text{\hspace{0.17em}\hspace{0.17em}}\frac{-8}{9}{x}^{2}y\text{\hspace{0.17em}}\left(\text{\hspace{0.17em}}2x-3\right);\text{\hspace{0.17em}\hspace{0.17em}\hspace{0.17em}\hspace{0.17em}}-\frac{1}{2}{x}^{2}y;\text{\hspace{0.17em}\hspace{0.17em}\hspace{0.17em}\hspace{0.17em}}{2}^{2}{x}^{3}+\frac{1}{3}{x}^{3}{y}^{4}-{x}^{4}z+{x}^{2};\text{\hspace{0.17em}\hspace{0.17em}\hspace{0.17em}}15+\frac{1}{z}$$.
Có bao nhiêu đa thức trong các biểu thức trên?
D. 5.

\textbf{{ANSWER}}

Chọn đáp án B

========================================================================

https://khoahoc.vietjack.com/thi-online/de-kiem-tra-giua-ki-1-toan-8-kntt-co-dap-an/133607


\textbf{{QUESTION}}

Bậc của đa thức −45x7y2+23x2y5−xy4$$ \frac{-4}{5}{x}^{7}{y}^{2}+\frac{2}{3}{x}^{2}{y}^{5}-x{y}^{4}$$  là 
D. 3.

\textbf{{ANSWER}}

Chọn đáp án A

========================================================================

https://khoahoc.vietjack.com/thi-online/de-kiem-tra-giua-ki-1-toan-8-kntt-co-dap-an/133607


\textbf{{QUESTION}}

Nhân hai đơn thức 5x4y2z$$ 5{x}^{4}{y}^{2}z$$  và −15x3yz2$$ \frac{-1}{5}{x}^{3}y{z}^{2}$$  ta được kết quả là 
A. −x12y2z2$$ -{x}^{12}{y}^{2}{z}^{2}$$
B. −25x7y3z3$$ -25{x}^{7}{y}^{3}{z}^{3}$$
C. x7y3z3$$ {x}^{7}{y}^{3}{z}^{3}$$
D. −x7y3z3$$ -{x}^{7}{y}^{3}{z}^{3}$$

\textbf{{ANSWER}}

Chọn đáp án D

========================================================================

https://khoahoc.vietjack.com/thi-online/giai-sach-bai-tap-toan-6-tap-1/22997


\textbf{{QUESTION}}

Tìm số tự nhiên x, biết: 2436 : x = 12

\textbf{{ANSWER}}

Ta có: 2436 : x = 12
x = 2436 : 12
x = 203
Vậy x = 203

========================================================================

https://khoahoc.vietjack.com/thi-online/giai-sach-bai-tap-toan-6-tap-1/22997


\textbf{{QUESTION}}

Tìm số tự nhiên x, biết: 6.x – 5 = 613

\textbf{{ANSWER}}

Ta có: 6.x – 5 = 613
6.x = 613 + 5
6.x = 618
x = 618: 6
x = 103
Vậy x = 103

========================================================================

https://khoahoc.vietjack.com/thi-online/giai-sach-bai-tap-toan-6-tap-1/22997


\textbf{{QUESTION}}

Tìm số tự nhiên x, biết: 12.(x – 1) = 0

\textbf{{ANSWER}}

12.( x – 1) = 0
x – 1 = 0
x = 0 + 1
x = 1
Vậy x = 1

========================================================================

https://khoahoc.vietjack.com/thi-online/giai-sach-bai-tap-toan-6-tap-1/22997


\textbf{{QUESTION}}

Tìm số tự nhiên x, biết: 0 : x = 0

\textbf{{ANSWER}}

0 : x = 0
Vì 0 chia cho một số tự nhiên bất kì khác 0 đều bằng 0.
Do đó x là các số tự nhiên khác 0 hay x ∈ N*

========================================================================

https://khoahoc.vietjack.com/thi-online/giai-sach-bai-tap-toan-6-tap-1/22997


\textbf{{QUESTION}}

Trong phép chia một số tự nhiên cho 6, số dư có thể bằng bao nhiêu?

\textbf{{ANSWER}}

Trong phép chia một số tự nhiên cho 6, số dư có thể bằng: {0; 1; 2; 3; 4; 5}

========================================================================

https://khoahoc.vietjack.com/thi-online/de-thi-hoc-ki-2-toan-9-chon-loc-co-dap-an/104963


\textbf{{QUESTION}}

1) Giải hệ phương trình  $$ \left\{\begin{array}{c}2x-y=1\\ x+2y=8\end{array}\right.$$.
2) Giải phương trình x2 + x – 6 = 0.
3) Giải phương trình x4 – x2 – 12 = 0.

\textbf{{ANSWER}}

1)  $$ \left\{\begin{array}{c}2x-y=1\\ x+2y=8\end{array}\right.$$
Û  $$ \left\{\begin{array}{c}2x-y=1\\ 2x+4y=16\end{array}\right.$$
$$ \Leftrightarrow \left\{\begin{array}{c}5y=15\\ 2x-y=1\end{array}\right.$$
 $$ \Leftrightarrow \left\{\begin{array}{c}x=2\\ y=3\end{array}\right.$$
 
Vậy hệ phương trình đã cho có nghiệm là (2; 3).
2) x2 + x – 6 = 0
Û x2 – 2x + 3x – 6 = 0
Û x(x – 2) + 3(x – 2) = 0
Û (x – 2)(x + 3) = 0
Û  $$ \left[\begin{array}{c}x-2=0\\ x+3=0\end{array}\right.$$ 
Û  $$ \left[\begin{array}{c}x=2\\ x=-3\end{array}\right.$$.
Vậy tập phương trình đã cho là S = {2; −3}.
3) x4 – x2 – 12 = 0 (1)
Đặt t = x2 (t ≥ 0), phương trình (1) trở thành:
t2 – t – 12 = 0
Û t2 + 3t – 4t – 12 = 0
Û t(t + 3) – 4(t + 3) = 0
Û (t – 4)(t + 3) = 0
Û t – 4 = 0 hay t + 3 = 0
Û t = 4 (nhận) hay t = −3 (loại)
Ta có: x2 = 4 Û x = 2 hay x = −2.
Vậy tập phương trình đã cho là S = {2; −2}.

========================================================================

https://khoahoc.vietjack.com/thi-online/30-de-thi-thu-thpt-quoc-gia-mon-toan-hay-nhat-co-loi-giai-chi-tiet/24270


\textbf{{QUESTION}}

Tìm tất cả các đường tiệm cận đứng của đồ thị của hàm số $$ y=\frac{{x}^{2}-1}{3-2x-5{x}^{2}}$$
A. $$ x=1$$ hoặc $$ x=\frac{3}{5}$$
B. $$ x=-1$$ hoặc $$ x=\frac{3}{5}$$
C. $$ x=-1$$
D. $$ x=\frac{3}{5}$$

\textbf{{ANSWER}}

Chọn D

========================================================================

https://khoahoc.vietjack.com/thi-online/30-de-thi-thu-thpt-quoc-gia-mon-toan-hay-nhat-co-loi-giai-chi-tiet/24270


\textbf{{QUESTION}}

Đồ thị hàm số nào dưới đây có tiệm cận ngang
A. y=√x-3x+1$$ y=\frac{\sqrt{x-3}}{x+1}$$
B. y=√x-3x+2$$ y=\frac{\sqrt{x-3}}{x+2}$$
C. y=√x2+3x+1$$ y=\frac{\sqrt{{x}^{2}+3}}{x+1}$$
D. y=√x-3x2+1$$ y=\frac{\sqrt{x-3}}{{x}^{2}+1}$$

\textbf{{ANSWER}}

Chọn A

========================================================================

https://khoahoc.vietjack.com/thi-online/de-thi-thu-ts-vao-10-lan-2-thang-2-nam-hoc-2025-2026-mon-toan-thcs-hoang-thanh-tinh-thanh-hoa


\textbf{{QUESTION}}

Phương trình 
$2x - 1 = 0$
$2x - 1 = 0$

 có nghiệm là:
          
          
A. 
$x = 1$

.
              
              
B. 
$x = \frac{1}{2}$

.
                               
                               
C. 
$x = 2$

.
 
 
D. 
$x = - \frac{1}{2}$

.

\textbf{{ANSWER}}

Đáp án đúng là: B
Giải phương trình:
$2x - 1 = 0$
$2x = 1$
$x = \frac{1}{2}.$
Vậy phương trình đã cho có nghiệm là $x = \frac{1}{2}.$

========================================================================

https://khoahoc.vietjack.com/thi-online/de-thi-thu-ts-vao-10-lan-2-thang-2-nam-hoc-2025-2026-mon-toan-thcs-hoang-thanh-tinh-thanh-hoa


\textbf{{QUESTION}}

Điều kiện xác định của biểu thức 
√2x−6$\sqrt {2x - 6} $
√2x−6
√2x−6
√2x−6
√
√
2x−6

2x−6
2
x
−
6
 là:
          
          
A. 
x≤3$x \le 3$
x≤3
x≤3
x
≤
3
.            
            
B. 
x≠3$x \ne 3$
x≠3
x≠3
x
≠
3
.           
           
 C. 
 
x≥3$x \ge 3$
x≥3
x≥3
x
≥
3
.            
            
D. 
x>3$x > 3$
x>3
x>3
x
>
3
.

\textbf{{ANSWER}}

Đáp án đúng là: C
Điều kiện xác định của biểu thức $\sqrt {2x - 6} $ là $2x - 6 \ge 0,$ tức là $2x \ge 6$ hay $x \ge 3.$

========================================================================

https://khoahoc.vietjack.com/thi-online/de-thi-thu-ts-vao-10-lan-2-thang-2-nam-hoc-2025-2026-mon-toan-thcs-hoang-thanh-tinh-thanh-hoa


\textbf{{QUESTION}}

Cho hàm số: 
y=−2x2.$y = - 2{x^2}.$
y=−2x2.
y=−2x2.
y
=
−
2
x2
x2
x
2
.
 Giá trị của hàm số đã cho tại 
x=2$x = 2$
x=2
x=2
x
=
2
 là:
          
          

A. 
y=−4$y = - 4$
y=−4
y=−4
y
=
−
4
.
           
           
B. 
y=4$y = 4$
y=4
y=4
y
=
4
.
              
              
C. 
y=−8$y = - 8$
y=−8
y=−8
y
=
−
8
.
           
          
D. 
y=8$y = 8$
y=8
y=8
y
=
8
.

\textbf{{ANSWER}}

Hướng dẫn giải
Đáp án đúng là: C
Thay x=2$x = 2$ vào hàm số y=−2x2,$y = - 2{x^2},$ ta có: y=−2⋅22=−8.$y = - 2 \cdot {2^2} = - 8.$

========================================================================

https://khoahoc.vietjack.com/thi-online/de-thi-thu-ts-vao-10-lan-2-thang-2-nam-hoc-2025-2026-mon-toan-thcs-hoang-thanh-tinh-thanh-hoa


\textbf{{QUESTION}}

Bất phương trình nào là bất phương trình bậc nhất một ẩn?
          
          

A. 
2x−3y<6$2x - 3y < 6$
2x−3y<6
2x−3y<6
2
x
−
3
y
<
6
.
     
     
B. 
x2−1<0${x^2} - 1 < 0$
x2−1<0
x2−1<0
x2
x2
x
2
−
1
<
0
.  
  
C. 
0x−7>0$0x - 7 > 0$
0x−7>0
0x−7>0
0
x
−
7
>
0
.
       
       
D. 
5x−3<0$5x - 3 < 0$
5x−3<0
5x−3<0
5
x
−
3
<
0
.

\textbf{{ANSWER}}

Đáp án đúng là: D
Bất phương trình 5x−3<0$5x - 3 < 0$ là bất phương trình bậc nhất một ẩn dạng ax+b<0$ax + b < 0$ với a=5,b=−3.$a = 5,\,\,b = - 3.$

========================================================================

https://khoahoc.vietjack.com/thi-online/de-thi-thu-ts-vao-10-lan-2-thang-2-nam-hoc-2025-2026-mon-toan-thcs-hoang-thanh-tinh-thanh-hoa


\textbf{{QUESTION}}

Độ dài cung 
60∘$60^\circ $
60∘
60∘
60∘
60
∘
 của đường tròn có bán kính 5 cm bằng:
          
          
A. 
10π$10\pi $
10π
10π
10
π
 cm.      
      
 
 
B. 5π3$\frac{{5\pi }}{3}$ cm.
5π3$\frac{{5\pi }}{3}$
5π3
5π3
5π3
5π
5π
5
π
3
3


3
3
3
                       
                       
C. 5π6$\frac{{5\pi }}{6}$ cm.    
5π6$\frac{{5\pi }}{6}$
5π6
5π6
5π6
5π
5π
5
π
6
6


6
6
6
    
D. 25π6$\frac{{25\pi }}{6}$ cm.
25π6$\frac{{25\pi }}{6}$
25π6
25π6
25π6
25π
25π
25
π
6
6


6
6
6

\textbf{{ANSWER}}

Hướng dẫn giải
Đáp án đúng là: B
Độ dài cung 60∘$60^\circ $ của đường tròn có bán kính 5 cm là: l=π⋅5⋅60180=5π3(cm).$l = \frac{{\pi \cdot 5 \cdot 60}}{{180}} = \frac{{5\pi }}{3}{\rm{\;(cm)}}{\rm{.}}$

========================================================================

https://khoahoc.vietjack.com/thi-online/bai-tap-cac-truong-hop-dong-dang-cua-tam-giac-co-loi-giai-chi-tiet


\textbf{{QUESTION}}

Cho hai tam giác Δ RSK và Δ PQM có: RS/PQ = RK/PM = SK/QM thì:
A. Δ RSK đồng dạng Δ PQM 
B. Δ RSK đồng dạng Δ MPQ 
C. Δ RSK đồng dạng Δ QPM 
D. Δ RSK đồng dạng Δ QMP

\textbf{{ANSWER}}

Ta có: RS/PQ = RK/PM = SK/QM ⇒ Δ RSK đồng dạng Δ PQM
Chọn đáp án A.

========================================================================

https://khoahoc.vietjack.com/thi-online/on-tap-chuong-ii


\textbf{{QUESTION}}

Trên cùng một nửa mặt phẳng có bờ chứa tia Ox, vẽ hai tia Oz và Oy sao cho $$ \widehat{xOz}=75°,\widehat{xOy}=150°$$
a) Hỏi tia nào nằm giữa hai tia còn lại? Vì sao?
b) Tính $$ \widehat{zOy}$$. So sánh $$ \widehat{xOz}$$ với $$ \widehat{zOy}$$.
c) Tia Oz có phải là tia phân giác của $$ \widehat{xOy}$$ không? Vì sao?

\textbf{{ANSWER}}

a) Tia Oz nằm giữa hai tia Ox, Oy vì ba tia cùng nằm trên một nửa mặt phẳng có bờ chứa tia Ox và $$ \widehat{xOz}<\widehat{xOy}$$
b) $$ \widehat{zOy}=75°=\widehat{xOz}$$
c) Tia Oz là tia phân giác của $$ \widehat{xOy}$$ vì tia Oz nằm giữa hai tia Ox, Oy và $$ \widehat{zOy}=\widehat{xOz}$$

========================================================================

https://khoahoc.vietjack.com/thi-online/on-tap-chuong-ii


\textbf{{QUESTION}}

Cho góc bẹt $$ \widehat{xOy}$$. Vẽ tia Oz sao cho góc $$ \widehat{xOz\quad }\quad =\quad 70°$$. Tính góc $$ \widehat{zOy}$$.

\textbf{{ANSWER}}

Tính được $$ \widehat{zOy}=\quad 110°.$$

========================================================================

https://khoahoc.vietjack.com/thi-online/on-tap-chuong-ii


\textbf{{QUESTION}}

Cho góc bẹt ^xOy$$ \widehat{xOy}$$. Vẽ tia Oz sao cho góc ^xOz  = 70°$$ \widehat{xOz\quad }\quad =\quad 70°$$. Trên nửa mặt phẳng bờ Ox chứa Oz vẽ tia Ot sao cho ^xOt= 140°$$ \widehat{xOt}=\quad 140°$$ . Chứng tỏ tia Oz là tia phân giác của góc ^xOt$$ \widehat{xOt}$$.

\textbf{{ANSWER}}

Vì ba tia Ox,Oz,Ot cùng nằm trên một nửa mặt phẳng có bờ là Ox và ^xOz<^xOt$$ \widehat{xOz}<\widehat{xOt}$$ nên tia Oz nằm giữa hai tia Ox,Ot
Lại có ^xOz=12^xOt$$ \widehat{xOz}=\frac{1}{2}\widehat{xOt}$$ nên tia Oz là tia phân giác của góc xOt.

========================================================================

https://khoahoc.vietjack.com/thi-online/on-tap-chuong-ii


\textbf{{QUESTION}}

Cho góc bẹt ^xOy$$ \widehat{xOy}$$. Vẽ tia Oz sao cho góc ^xOz  = 70°$$ \widehat{xOz\quad }\quad =\quad 70°$$. Vẽ tia Om là tia đối của tia Oz, tia On là tia đối của tia Ot.
Tính góc ^yOm$$ \widehat{yOm}$$ và so sánh với góc ^xOn$$ \widehat{xOn}$$

\textbf{{ANSWER}}

^yOm=^zOm−^zOy=70°$$ \widehat{yOm}=\widehat{zOm}-\widehat{zOy}=70°$$
^xOn=^nOt−^xOt=40°<^yOm$$ \widehat{xOn}=\widehat{nOt}-\widehat{xOt}=40°<\widehat{yOm}$$

========================================================================

https://khoahoc.vietjack.com/thi-online/on-tap-chuong-ii


\textbf{{QUESTION}}

Cho cặp góc kề bù ^xOz$$ \widehat{xOz}$$ và ^zOy$$ \widehat{zOy}$$, biết ^xOz= 70°$$ \widehat{xOz}=\quad 70°$$. Tính số đo góc ^zOy$$ \widehat{zOy}$$.

\textbf{{ANSWER}}

Tính được ^zOy=150°$$ \widehat{zOy}=150°$$

========================================================================

https://khoahoc.vietjack.com/thi-online/giai-sbt-toan-11-canh-dieu-bai-1-dinh-nghia-dao-ham-y-nghia-hinh-hoc-cua-dao-ham-co-dap-an


\textbf{{QUESTION}}

Cho hàm số y = f(x) có đạo hàm x0 là f’(x0). Phát biểu nào sau đây là đúng?
A. $$ {f}^{\text{'}}\left(x\right)=\underset{x\to {x}_{0}}{\mathrm{lim}}\frac{f\left(x\right)+f\left({x}_{0}\right)}{x+{x}_{0}}.$$
B. $$ {f}^{\text{'}}\left(x\right)=\underset{x\to {x}_{0}}{\mathrm{lim}}\frac{f\left(x\right)-f\left({x}_{0}\right)}{x-{x}_{0}}.$$
C. $$ {f}^{\text{'}}\left(x\right)=\underset{x\to {x}_{0}}{\mathrm{lim}}\frac{f\left(x\right)-f\left({x}_{0}\right)}{x+{x}_{0}}.$$
D. $$ {f}^{\text{'}}\left(x\right)=\underset{x\to {x}_{0}}{\mathrm{lim}}\frac{f\left(x\right)+f\left({x}_{0}\right)}{x-{x}_{0}}.$$

\textbf{{ANSWER}}

Đáp án đúng là: B
Hàm số y = f(x) có đạo hàm x0 là f’(x0) thì $$ {f}^{\text{'}}\left(x\right)=\underset{x\to {x}_{0}}{\mathrm{lim}}\frac{f\left(x\right)-f\left({x}_{0}\right)}{x-{x}_{0}}.$$

========================================================================

https://khoahoc.vietjack.com/thi-online/giai-sbt-toan-11-canh-dieu-bai-1-dinh-nghia-dao-ham-y-nghia-hinh-hoc-cua-dao-ham-co-dap-an


\textbf{{QUESTION}}

Điện lượng Q truyền trong dây dẫn là một hàm số của thời gian t, Q = Q(t). Cường độ trung bình trong khoảng thời gian |t – t0| được xác định bởi công thức Q(t)−Q(t0)t−t0.$$ \frac{Q\left(t\right)-Q\left({t}_{0}\right)}{t-{t}_{0}}.$$Cường độ tức thời tại thời điểm t0 là:
A. Q(t)−Q(t0)t−t0.$$ \frac{Q\left(t\right)-Q\left({t}_{0}\right)}{t-{t}_{0}}.$$
B. limt→0Q(t)−Q(t0)t−t0.$$ \underset{t\to 0}{\mathrm{lim}}\frac{Q\left(t\right)-Q\left({t}_{0}\right)}{t-{t}_{0}}.$$
C. limt→t0Q'(t)−Q'(t0)t−t0.$$ \underset{t\to {t}_{0}}{\mathrm{lim}}\frac{{Q}^{\text{'}}\left(t\right)-{Q}^{\text{'}}\left({t}_{0}\right)}{t-{t}_{0}}.$$
D. limt→t0Q(t)−Q(t0)t−t0.$$ \underset{t\to {t}_{0}}{\mathrm{lim}}\frac{Q\left(t\right)-Q\left({t}_{0}\right)}{t-{t}_{0}}.$$

\textbf{{ANSWER}}

Đáp án đúng là: D

========================================================================

https://khoahoc.vietjack.com/thi-online/de-thi-giua-ki-1-toan-7-ctst-co-dap-an


\textbf{{QUESTION}}

Điền kí hiệu thích hợp vào chỗ trống: 12,5 … ℚ:
A. Î;
C. ⊂;

\textbf{{ANSWER}}

Chọn đáp án A

========================================================================

https://khoahoc.vietjack.com/thi-online/de-thi-giua-ki-1-toan-7-ctst-co-dap-an


\textbf{{QUESTION}}

Tìm số a biết số đối của a là 314$$ 3\frac{1}{4}$$.

\textbf{{ANSWER}}

Chọn đáp án C

========================================================================

https://khoahoc.vietjack.com/thi-online/de-thi-giua-ki-1-toan-7-ctst-co-dap-an


\textbf{{QUESTION}}

Rút gọn biểu thức 15894$$ \frac{{15}^{8}}{{9}^{4}}$$ ta được kết quả là:
A. 55;
B. 56;
C. 85;

\textbf{{ANSWER}}

Chọn đáp án D

========================================================================

https://khoahoc.vietjack.com/thi-online/de-kiem-tra-15-phut-toan-9-chuong-2-hinh-hoc-co-dap-an/43627


\textbf{{QUESTION}}

Đường tròn tâm A bán kính 2cm gồm tất cả các điểm:
A. Có khoảng cách đến điểm A bằng 2 cm
B. Có khoảng cách đến điểm A nhỏ hơn 2 cm
C. Có khoảng cách đến điểm A lớn hơn 2 cm
D. Có khoảng cách đến điểm A nhỏ hơn hoặc bằng 2 cm

\textbf{{ANSWER}}

Đáp án là A

========================================================================

https://khoahoc.vietjack.com/thi-online/de-kiem-tra-15-phut-toan-9-chuong-2-hinh-hoc-co-dap-an/43627


\textbf{{QUESTION}}

Cho đoạn thẳng OO' = 10 cm. Vẽ các đường tròn (O; 6cm) và (O'; 4cm). Hai đường tròn này có vị trí tương đối như thế nào?
A. (O) và (O') cắt nhau
B. (O) và (O') tiếp xúc ngoài với nhau
C. (O) và (O') đựng nhau
D. (O) và (O') tiếp xúc trong với nhau

\textbf{{ANSWER}}

Đáp án là B

========================================================================

https://khoahoc.vietjack.com/thi-online/de-kiem-tra-15-phut-toan-9-chuong-2-hinh-hoc-co-dap-an/43627


\textbf{{QUESTION}}

Cho (O; 5cm) và đường thẳng d. Gọi OH là khoảng cách từ tâm O đến a. Điều kiện để a và O có 2 điểm chung là:
A. Khoảng cách OH ≤ 5 cm
B. Khoảng cách OH = 5 cm
C. Khoảng cách OH > 5 cm
D. Khoảng cách OH < 5 cm

\textbf{{ANSWER}}

Đáp án là D

========================================================================

https://khoahoc.vietjack.com/thi-online/de-kiem-tra-15-phut-toan-9-chuong-2-hinh-hoc-co-dap-an/43627


\textbf{{QUESTION}}

Cho tam giác ABC có độ dài các cạnh AB = 12 cm; AC = 16 cm; BC = 20 cm. Bán kính đường tròn ngoại tiếp tam giác ABC là:
A. 6 cm       
B. 8 cm       
C. 10 cm       
D. 12 cm

\textbf{{ANSWER}}

Đáp án là C
Tam giác ABC có:
AB2+AC2=122+162=400=BC2$$ A{B}^{2}+A{C}^{2}={12}^{2}+{16}^{2}=400=B{C}^{2}$$
⇒ ΔABC vuông tại A
⇒ Tâm đường tròn ngoại tiếp tam giác ABC là trung điểm của BC
⇒ Bán kính = 10 cm

========================================================================

https://khoahoc.vietjack.com/thi-online/bai-tap-on-tap-toan-7-chuong-4-bieu-thuc-dai-so-co-dap-an/61009


\textbf{{QUESTION}}

Chứng tỏ rằng đa thức $$ R\left(y\right)={y}^{2}+2$$ vô nghiệm

\textbf{{ANSWER}}

Ta có nhận xét sau:
$$ R\left(y\right)={y}^{2}+2\quad \ge 0+2>0$$ (với mọi y)
Do đó R(y) vô nghiệm.

========================================================================

https://khoahoc.vietjack.com/thi-online/bai-tap-on-tap-toan-7-chuong-4-bieu-thuc-dai-so-co-dap-an/61009


\textbf{{QUESTION}}

Tìm nghiệm của đa thức P(x) = 2x + 3

\textbf{{ANSWER}}

Nghiệm của đa thức thỏa mãn:
P(x) = 0⇔ 2x + 3 = 0 ⇔ 2x = 3⇔x = -32$$ P\left(x\right)\quad =\quad 0\Leftrightarrow \quad 2x\quad +\quad 3\quad =\quad 0\quad \Leftrightarrow \quad 2x\quad =\quad 3\Leftrightarrow x\quad =\quad -\frac{3}{2}$$
Vậy x=-32$$ x=-\frac{3}{2}$$ là nghiệm của đa thức P(x)

========================================================================

https://khoahoc.vietjack.com/thi-online/bo-24-de-kiem-tra-giua-ki-2-toan-11-co-dap-an-moi-nhat


\textbf{{QUESTION}}

Cho dãy số $$ \left({\mathrm{u}}_{\mathrm{n}}\right)$$ thỏa mãn $$ \mathrm{lim}\left({\mathrm{u}}_{\mathrm{n}}-2020\right)=1.$$ Giá trị của $$ \mathrm{lim}\quad {\mathrm{u}}_{\mathrm{n}}$$ bằng
A. 2020
B. 2019
C. 2021
D. 0

\textbf{{ANSWER}}

$$ \mathrm{lim}\quad \left({\mathrm{u}}_{\mathrm{n}}-2020\right)=1\Leftrightarrow \mathrm{lim}\quad {\mathrm{u}}_{\mathrm{n}}-2020=1\phantom{\rule{0ex}{0ex}}\Leftrightarrow \mathrm{lim}\quad {\mathrm{u}}_{\mathrm{n}}=2021.$$ 
Vậy chọn phương án C.

========================================================================

https://khoahoc.vietjack.com/thi-online/bo-24-de-kiem-tra-giua-ki-2-toan-11-co-dap-an-moi-nhat


\textbf{{QUESTION}}

$$ \mathrm{lim}\frac{{\mathrm{n}}^{2}+2\mathrm{n}}{\mathrm{n}+1}$$ bằng
A. $$ -\infty .$$
B. $$ +\infty .$$
C. 1.
D. 2.

\textbf{{ANSWER}}

Ta có $$ \mathrm{lim}\frac{{\mathrm{n}}^{2}+2\mathrm{n}}{\mathrm{n}+1}=\mathrm{lim}\frac{1+\frac{2}{\mathrm{n}}}{\frac{1}{\mathrm{n}}+\frac{1}{{\mathrm{n}}^{2}}}.$$ 
$$ \left.\begin{array}{l}\mathrm{lim}\left(1+\frac{2}{\mathrm{n}}\right)=1>0\\ \mathrm{lim}\left(\frac{1}{\mathrm{n}}+\frac{1}{{\mathrm{n}}^{2}}\right)=0\\ \frac{1}{\mathrm{n}}+\frac{1}{{\mathrm{n}}^{2}}>0,\forall \mathrm{n}\end{array}\right\}\Rightarrow \mathrm{lim}\frac{1+\frac{2}{\mathrm{n}}}{\frac{1}{\mathrm{n}}+\frac{1}{{\mathrm{n}}^{2}}}=+\infty .$$
Vậy chọn phương án B.

========================================================================

https://khoahoc.vietjack.com/thi-online/bo-24-de-kiem-tra-giua-ki-2-toan-11-co-dap-an-moi-nhat


\textbf{{QUESTION}}

Cho hai dãy số (un),(vn)$$ \left({\mathrm{u}}_{\mathrm{n}}\right),\left({\mathrm{v}}_{\mathrm{n}}\right)$$ thỏa mãn lim un=−4$$ \mathrm{lim}\quad {\mathrm{u}}_{\mathrm{n}}=-4$$ và lim vn=-8$$ {\text{lim v}}_{n}=-8$$ Giá trị của lim (un−vn)$$ \mathrm{lim}\quad \left({\mathrm{u}}_{\mathrm{n}}-{\mathrm{v}}_{\mathrm{n}}\right)$$ bằng
A. - 4
B. 8
C. -12
D. 4

\textbf{{ANSWER}}

lim (un−vn)=lim un−lim vn=−4−(−8)=4.$$ \mathrm{lim}\quad \left({\mathrm{u}}_{\mathrm{n}}-{\mathrm{v}}_{\mathrm{n}}\right)=\mathrm{lim}\quad {\mathrm{u}}_{\mathrm{n}}-\mathrm{lim}\quad {\mathrm{v}}_{\mathrm{n}}\phantom{\rule{0ex}{0ex}}=-4-\left(-8\right)=4.$$
Vậy chọn phương án D.

========================================================================

https://khoahoc.vietjack.com/thi-online/bo-24-de-kiem-tra-giua-ki-2-toan-11-co-dap-an-moi-nhat


\textbf{{QUESTION}}

limnn2+3$$ \mathrm{lim}\frac{\mathrm{n}}{{\mathrm{n}}^{2}+3}$$ bằng
A. 0.
B. +∞$$ +\infty $$
C. 1.
D. 13.$$ \frac{1}{3}.$$

\textbf{{ANSWER}}

limnn2+3=lim1n1+3n2=01+0=0.$$ \mathrm{lim}\frac{\mathrm{n}}{{\mathrm{n}}^{2}+3}=\mathrm{lim}\frac{\frac{1}{\mathrm{n}}}{1+\frac{3}{{\mathrm{n}}^{2}}}=\frac{0}{1+0}=0.$$
Vậy chọn phương án A.

========================================================================

https://khoahoc.vietjack.com/thi-online/bo-24-de-kiem-tra-giua-ki-2-toan-11-co-dap-an-moi-nhat


\textbf{{QUESTION}}

lim 5n$$ \mathrm{lim}\quad {5}^{\mathrm{n}}$$ bằng
A. +∞.$$ +\infty .$$

B. −∞.$$ -\infty .$$

C. 2.
D. 0.

\textbf{{ANSWER}}

Do 5>1$$ 5>1$$ nên lim 5n=+∞$$ \mathrm{lim}\quad {5}^{\mathrm{n}}=+\infty $$. 
Vậy chọn phương án A.

========================================================================

https://khoahoc.vietjack.com/thi-online/top-8-de-kiem-tra-dai-so-toan-8-hoc-ki-1-chuong-2-co-dap-an-cuc-hay/43827


\textbf{{QUESTION}}

Phân thức đối của phân thức $$ \frac{5x+3y}{5x-3y}$$ là:
A. $$ \frac{5x-3y}{5x+3y}$$
B. $$ \frac{-\left(5x+3y\right)}{5x+3y}$$
C. 1
D. -1

\textbf{{ANSWER}}

Chọn B

========================================================================

https://khoahoc.vietjack.com/thi-online/top-8-de-kiem-tra-dai-so-toan-8-hoc-ki-1-chuong-2-co-dap-an-cuc-hay/43827


\textbf{{QUESTION}}

Rút gọn phân thức 2x2+2xx2-1$$ \frac{2{x}^{2}+2x}{{x}^{2}-1}$$ được kết quả là:
A. 2xx-1$$ \frac{2x}{x-1}$$
B. 2(x-1)
C. 2(x+1)
D. x+1x-1$$ \frac{x+1}{x-1}$$

\textbf{{ANSWER}}

Chọn A

========================================================================

https://khoahoc.vietjack.com/thi-online/top-8-de-kiem-tra-dai-so-toan-8-hoc-ki-1-chuong-2-co-dap-an-cuc-hay/43827


\textbf{{QUESTION}}

Tổng hai phân thức 2xx+1+4x+2$$ \frac{2x}{x+1}+\frac{4}{x+2}$$ có kết quả là
A. 3x
B. 4(x + 2)
C. 2
D. 6x

\textbf{{ANSWER}}

Chọn C

========================================================================

https://khoahoc.vietjack.com/thi-online/top-8-de-kiem-tra-dai-so-toan-8-hoc-ki-1-chuong-2-co-dap-an-cuc-hay/43827


\textbf{{QUESTION}}

Giá trị x ≠ 2 và x ≠ -2 là điều kiện xác định của phân thức:
A. xx-2$$ \frac{x}{x-2}$$
B. xx+2$$ \frac{x}{x+2}$$
C. 2xx2-2$$ \frac{2x}{{x}^{2}-2}$$
D. 2xx2-4$$ \frac{2x}{{x}^{2}-4}$$

\textbf{{ANSWER}}

Chọn D

========================================================================

https://khoahoc.vietjack.com/thi-online/top-8-de-kiem-tra-dai-so-toan-8-hoc-ki-1-chuong-2-co-dap-an-cuc-hay/43827


\textbf{{QUESTION}}

Giá trị của phân thức 3x2-x9x2-6x+1$$ \frac{3{x}^{2}-x}{9{x}^{2}-6x+1}$$ tại x = 10 là:
A. 20
B. 30
C. 10/29
D. 12/25

\textbf{{ANSWER}}

Chọn C

========================================================================

https://khoahoc.vietjack.com/thi-online/trac-nghiem-toan-8-nhan-chia-cac-phan-thuc-co-loi-giai-chi-tiet


\textbf{{QUESTION}}

Kết quả của phép nhân $$ \frac{A}{B}.\frac{C}{D}$$  là
A. $$ \frac{A.C}{BD}$$
B. $$ \frac{A.D}{BC}$$
C. $$ \frac{A+C}{B+D}$$
D. $$ \frac{BD}{AC}$$

\textbf{{ANSWER}}

Quy tắc: muốn nhân hai phân thức, ta nhân tử thức với nhau, mẫu thức với nhau.
$$ \frac{A}{B}.\frac{C}{D}=\frac{A.C}{B.D}$$
Đáp án cần chọn là: A

========================================================================

https://khoahoc.vietjack.com/thi-online/trac-nghiem-toan-8-nhan-chia-cac-phan-thuc-co-loi-giai-chi-tiet


\textbf{{QUESTION}}

Chọn đáp án đúng
A. Muốn nhân hai phân thức, ta nhân tử thức với nhau, giữ nguyên mẫu thức
B. Muốn nhân hai phân thức, ta giữ nguyên tử thức, nhân mẫu thức với nhau
C. Muốn nhân hai phân thức, ta nhân tử thức với nhau, nhân mẫu thức với nhau
D. Muốn nhân hai phân thức, ta nhân tử thức của phân thức này với mẫu thức của phân thức kia

\textbf{{ANSWER}}

Đáp án cần chọn là: C

========================================================================

https://khoahoc.vietjack.com/thi-online/trac-nghiem-toan-8-nhan-chia-cac-phan-thuc-co-loi-giai-chi-tiet


\textbf{{QUESTION}}

Chọn khẳng định đúng. Muốn chia phân thức AB$$ \frac{A}{B}$$  cho phân thức CD (CD ≠0)$$ \frac{C}{D}\quad \left(\frac{C}{D}\quad \ne 0\right)$$
A. ta nhân AB$$ \frac{A}{B}$$  với phân thức nghịch đảo của   DC$$ \frac{D}{C}$$
C. ta nhân AB$$ \frac{A}{B}$$ với phân thức nghịch đảo của  CD$$ \frac{C}{D}$$
D. ta cộng AB$$ \frac{A}{B}$$ với phân thức nghịch đảo của CD$$ \frac{C}{D}$$

\textbf{{ANSWER}}

Đáp án cần chọn là: C

========================================================================

https://khoahoc.vietjack.com/thi-online/trac-nghiem-toan-8-nhan-chia-cac-phan-thuc-co-loi-giai-chi-tiet


\textbf{{QUESTION}}

Chọn câu sai
A. AB.BA=1$$ \frac{A}{B}.\frac{B}{A}=1$$
B. AB.CD=CD.AB$$ \frac{A}{B}.\frac{C}{D}=\frac{C}{D}.\frac{A}{B}$$
C. AB.(CD.EF)=EF(CD.AB)$$ \frac{A}{B}.(\frac{C}{D}.\frac{E}{F})=\frac{E}{F}(\frac{C}{D}.\frac{A}{B})$$
D. AB(CD+EF)=AB.CD+EF$$ \frac{A}{B}(\frac{C}{D}+\frac{E}{F})=\frac{A}{B}.\frac{C}{D}+\frac{E}{F}$$

\textbf{{ANSWER}}

Hai phân thức gọi là nghịch đảo của nhau nếu tích của nó bằng 1.
Nên AB.BA=1$$ \frac{A}{B}.\frac{B}{A}=1$$ , do đó A đúng.
Tính chất phép nhân phân thức
+ Giao hoán: AB.CD=CD.AB$$ \frac{A}{B}.\frac{C}{D}=\frac{C}{D}.\frac{A}{B}$$  nên B đúng.
+ Kết hợp: AB.(CD.EF)=EF(CD.AB)$$ \frac{A}{B}.(\frac{C}{D}.\frac{E}{F})=\frac{E}{F}(\frac{C}{D}.\frac{A}{B})$$  nên C đúng
+ Phân phối đối với phép cộng: AB(CD+EF)=AB.CD+AB.EF$$ \frac{A}{B}(\frac{C}{D}+\frac{E}{F})=\frac{A}{B}.\frac{C}{D}+\frac{A}{B}.\frac{E}{F}$$  nên D sai.
Đáp án cần chọn là: D

========================================================================

https://khoahoc.vietjack.com/thi-online/tong-hop-de-thi-chinh-thuc-vao-10-mon-toan-nam-2019-co-dap-an-phan-1/104765


\textbf{{QUESTION}}

Cho parabol $$ \left(P\right):y=-\frac{1}{2}{x}^{2}\quad $$ và đường thẳng $$ \left(d\right):y=x-4$$
a, Vẽ (P) và (d) trên cùng hệ trục tọa độ

\textbf{{ANSWER}}

a, Học sinh tự vẽ (P) và (d)

========================================================================

https://khoahoc.vietjack.com/thi-online/tong-hop-de-thi-chinh-thuc-vao-10-mon-toan-nam-2019-co-dap-an-phan-1/104765


\textbf{{QUESTION}}

b, Tìm tọa độ giao điểm của (P)  và (d) bằng phép tính

\textbf{{ANSWER}}

b,Phương trình hoành độ giao điểm (P) và (d) cho 2 nghiệm 2;−4$$ 2;-4$$
Tọa độ giao điểm (P) và (d) là (2;−2);(−4;−8)$$ \left(2;-2\right);\left(-4;-8\right)$$

========================================================================

https://khoahoc.vietjack.com/thi-online/tong-hop-de-thi-chinh-thuc-vao-10-mon-toan-nam-2019-co-dap-an-phan-1/104765


\textbf{{QUESTION}}

Cho phương trình :2x2−3x−1=0$$ 2{x}^{2}-3x-1=0$$  có hai nghiệm là x1,x2$$ {x}_{1},{x}_{2}$$
Không giải phương trình, hãy tính giá trị của biểu thức : A=x1−1x2+1+x2−1x1+1$$ A=\frac{{x}_{1}-1}{{x}_{2}+1}+\frac{{x}_{2}-1}{{x}_{1}+1}$$

\textbf{{ANSWER}}

Tổng x1+x2=32;    x1x2=−12$$ {x}_{1}+{x}_{2}=\frac{3}{2};\quad \quad \quad \quad {x}_{1}{x}_{2}=-\frac{1}{2}$$
A=x21+x22−2x1x2+x1+x2+1=(x1+x2)2−2x1x2−2x1x2+(x1+x2)+1=58$$ A=\frac{{x}_{1}^{2}+{x}_{2}^{2}-2}{{x}_{1}{x}_{2}+{x}_{1}+{x}_{2}+1}=\frac{{\left({x}_{1}+{x}_{2}\right)}^{2}-2{x}_{1}{x}_{2}-2}{{x}_{1}{x}_{2}+\left({x}_{1}+{x}_{2}\right)+1}=\frac{5}{8}$$

========================================================================

https://khoahoc.vietjack.com/thi-online/20-cau-trac-nghiem-toan-6-ket-noi-tri-thuc-bai-8-quan-he-chia-het-va-tinh-chat-co-dap-an-phan-2/108740


\textbf{{QUESTION}}

Cho tổng M = 75 + 120 + x. Với giá trị nào của x dưới đây thì M⋮3?

\textbf{{ANSWER}}

Trả lời:
Vì $75 \vdots 3;120 \vdots 3$nên để M = 75 + 120 + x chia hết cho 3 thì $x \vdots 3$ nên ta chọn x = 12
Đáp án cần chọn là: D

========================================================================

https://khoahoc.vietjack.com/thi-online/20-cau-trac-nghiem-toan-6-ket-noi-tri-thuc-bai-8-quan-he-chia-het-va-tinh-chat-co-dap-an-phan-2/108740


\textbf{{QUESTION}}

Cho a = 2m + 3, b = 2n + 1
Khẳng định nào sau đây đúng?

\textbf{{ANSWER}}

Trả lời:
Ta có:
$\left\{ {\begin{array}{*{20}{c}}{2m = 2.m \Rightarrow 2m \vdots 2}\\{3\not \vdots 2}\end{array}} \right.$
$ \Rightarrow a = 2m + 3\not \vdots 2$
$\left. {\begin{array}{*{20}{c}}{2n \vdots 2}\\{1\not \vdots 2}\end{array}} \right\} \Rightarrow b = 2n + 1\not \vdots 2$
→ Đáp án A, B sai
a + b = 2m + 3 + 2n + 1 = 2m + 2n + 4 
${\rm{ = 2}}{\rm{.}}\left( {{\rm{m + n + 2}}} \right) \vdots {\rm{2}}$
Đáp án C đúng.
Đáp án cần chọn là: C

========================================================================

https://khoahoc.vietjack.com/thi-online/20-cau-trac-nghiem-toan-6-ket-noi-tri-thuc-bai-8-quan-he-chia-het-va-tinh-chat-co-dap-an-phan-2/108740


\textbf{{QUESTION}}

Tìm A = 15 + 1003 + x với x∈N$x \in N$. Tìm điều kiện của x để A⋮5$A \vdots 5$

\textbf{{ANSWER}}

Trả lời:
Ta thấy 15⋮5$15 \vdots 5$và 1003 không chia hết cho 5 nên để A = 15 +1003 + x chia hết cho 5 thì 
(1003 + x) chia hết cho 5
Mà 1003 chia cho 5 dư 3 nên để (1003+ x) chia hết cho 5 thì x chia 5 dư 2.
Đáp án cần chọn là: D

========================================================================

https://khoahoc.vietjack.com/thi-online/20-cau-trac-nghiem-toan-6-ket-noi-tri-thuc-bai-8-quan-he-chia-het-va-tinh-chat-co-dap-an-phan-2/108740


\textbf{{QUESTION}}

Cho A = 12 + 15 + 36 + x, x∈N$x \in N$. Tìm điều kiện của x  để A không chia hết cho 9.

\textbf{{ANSWER}}

Trả lời:
Ta có: A=(12+15)+36+x$A = \left( {12 + 15} \right) + 36 + x$. Vì 12+15=27⋮9$12 + 15 = 27 \vdots 9$ và 36⋮9$36 \vdots 9$
⇒(12+15+36)=(27+36)⋮9$ \Rightarrow \left( {12 + 15 + 36} \right) = \left( {27 + 36} \right) \vdots 9$ nên để A không chia hết cho 9 thì x không chia hết cho 9.
Đáp án cần chọn là: B

========================================================================

https://khoahoc.vietjack.com/thi-online/20-cau-trac-nghiem-toan-6-ket-noi-tri-thuc-bai-8-quan-he-chia-het-va-tinh-chat-co-dap-an-phan-2/108740


\textbf{{QUESTION}}

Với a, b là số tự nhiên, nếu 10a + b chia hết cho 13 thì a + 4b chia hết cho số nào dưới đây?

\textbf{{ANSWER}}

Trả lời:
Xét 10. (a+4.b) = 10.a + 40.b = (10.a + b) + 39.b
Vì (10.a+b)⋮13$\left( {10.a + b} \right) \vdots 13$ và 39b⋮13$39b \vdots 13$nên 10.(a+4.b)⋮13$10.\left( {a + 4.b} \right) \vdots 13$
Do 10 không chia hết cho 13 nên suy ra (a+4.b)⋮13$\left( {a + 4.b} \right) \vdots 13$
Vậy nếu 10a + b chia hết cho 13 thì a + 4b chia hết cho 13.
Đáp án cần chọn là: D

========================================================================

https://khoahoc.vietjack.com/thi-online/12-bai-tapv-tim-gia-tri-lon-nhat-nho-nhat-cua-ham-so-bac-hai-co-loi-giai


\textbf{{QUESTION}}

Hàm số y = –x2 + 4x + 3 có giá trị lớn nhất là bao nhiêu ?

\textbf{{ANSWER}}

Hướng dẫn giải:
Xét hàm số: y = –x2 + 4x + 3 có a = –1, b = 4, c = 3.
Ta có: 
a = –1 < 0 
$\frac{{ - \Delta }}{{4a}} = \frac{{ - ({b^2} - 4ac)}}{{4a}} = \frac{{ - \left[ {{4^2} - 4.( - 1).3} \right]}}{{4.( - 1)}} = 7$ 
$\frac{{ - b}}{{2a}} = \frac{{ - 4}}{{2.( - 1)}} = 2$ 
Vậy hàm số y = –x2 + 4x + 3 có giá trị lớn nhất là 7 tại x = 2.

========================================================================

https://khoahoc.vietjack.com/thi-online/12-bai-tapv-tim-gia-tri-lon-nhat-nho-nhat-cua-ham-so-bac-hai-co-loi-giai


\textbf{{QUESTION}}

Tìm giá trị nhỏ nhất của hàm số y = x2 + 2x – 4.

\textbf{{ANSWER}}

Hướng dẫn giải:
Xét hàm số: y = x2 + 2x – 4 có a = 1, b = 2, c = – 4.
Ta có: 
a = 1 > 0 
−Δ4a=−(b2−4ac)4a=−[22−4.1.(−4)]4.1=−5$\frac{{ - \Delta }}{{4a}} = \frac{{ - ({b^2} - 4ac)}}{{4a}} = \frac{{ - \left[ {{2^2} - 4.1.( - 4)} \right]}}{{4.1}} = - 5$ 
−b2a=−22.1=−1$\frac{{ - b}}{{2a}} = \frac{{ - 2}}{{2.1}} = - 1$ 
Vậy hàm số y = x2 + 2x – 4 có giá trị nhỏ nhất là –5 tại x = – 1.

========================================================================

https://khoahoc.vietjack.com/thi-online/12-bai-tapv-tim-gia-tri-lon-nhat-nho-nhat-cua-ham-so-bac-hai-co-loi-giai


\textbf{{QUESTION}}

A. −13$ - \frac{1}{3}$;

\textbf{{ANSWER}}

Hướng dẫn giải: 
Đáp án đúng là: C.
Xét hàm số: y = –3x2 – 2x + 3 có a = –3, b = –2, c = 3.
Ta có: 
a = –3 < 0 
−Δ4a=−(b2−4ac)4a=−[(−2)2−4.(−3).3]4.(−3)=103$\frac{{ - \Delta }}{{4a}} = \frac{{ - ({b^2} - 4ac)}}{{4a}} = \frac{{ - \left[ {{{( - 2)}^2} - 4.( - 3).3} \right]}}{{4.( - 3)}} = \frac{{10}}{3}$ 
−b2a=−(−2)2.(−3)=−13$\frac{{ - b}}{{2a}} = \frac{{ - ( - 2)}}{{2.( - 3)}} = - \frac{1}{3}$ 
Vậy hàm số y = –3x2 – 2x + 3 có giá trị lớn nhất là 103$\frac{{10}}{3}$ tại x = −13$ - \frac{1}{3}$.

========================================================================

https://khoahoc.vietjack.com/thi-online/12-bai-tapv-tim-gia-tri-lon-nhat-nho-nhat-cua-ham-so-bac-hai-co-loi-giai


\textbf{{QUESTION}}

Giá trị lớn nhất của hàm số y = –2x2 – 12x là:
A. 3;
B. – 3;
C. – 18;
D. 18.

\textbf{{ANSWER}}

Hướng dẫn giải: 
Đáp án đúng là: D.
Xét hàm số: y = –2x2 – 12x có a = –2, b = –12, c = 0.
Ta có: 
a = –2 < 0 
−Δ4a=−(b2−4ac)4a=−[(−12)2−4.(−2).0]4.(−2)=18$\frac{{ - \Delta }}{{4a}} = \frac{{ - ({b^2} - 4ac)}}{{4a}} = \frac{{ - \left[ {{{( - 12)}^2} - 4.( - 2).0} \right]}}{{4.( - 2)}} = 18$.
−b2a=−(−12)2.(−2)=−3$\frac{{ - b}}{{2a}} = \frac{{ - ( - 12)}}{{2.( - 2)}} = - 3$.
Vậy hàm số y = –2x2 – 12x có giá trị lớn nhất là 18 tại x = – 3.

========================================================================

https://khoahoc.vietjack.com/thi-online/12-bai-tapv-tim-gia-tri-lon-nhat-nho-nhat-cua-ham-so-bac-hai-co-loi-giai


\textbf{{QUESTION}}

Giá trị nhỏ nhất của hàm số y = x2 – 5x + 10 là:

\textbf{{ANSWER}}

Hướng dẫn giải:
Đáp án đúng là: A.
Xét hàm số: y = x2 – 5x + 10 có a = 1, b = –5, c = 10 
Ta có: 
a = 1 > 0 
−Δ4a=−(b2−4ac)4a=−[(−5)2−4.1.10]4.1=154$\frac{{ - \Delta }}{{4a}} = \frac{{ - ({b^2} - 4ac)}}{{4a}} = \frac{{ - \left[ {{{( - 5)}^2} - 4.1.10} \right]}}{{4.1}} = \frac{{15}}{4}$ 
−b2a=−(−5)2.1=52$\frac{{ - b}}{{2a}} = \frac{{ - ( - 5)}}{{2.1}} = \frac{5}{2}$ 
Vậy hàm số y = x2 – 5x + 10 có giá trị nhỏ nhất là 154$\frac{{15}}{4}$ tại x = 52$\frac{5}{2}$.

========================================================================

https://khoahoc.vietjack.com/thi-online/15-cau-trac-nghiem-ham-so-co-dap-an-nhan-biet


\textbf{{QUESTION}}

Điểm nào sau đây thuộc đồ thị hàm số y = 2 |x − 1| + 3 |x| − 2?
A. (2; 6).
B.  (1; −1).
C. (−2; −10).
D. (0; −4).

\textbf{{ANSWER}}

Đặt y = f(x) = 2 |x − 1| + 3 |x| − 2
Ta có: f(2) = 2 |2 − 1| + 3 |2| − 2 = 6 nên (2; 6) thuộc đồ thị hàm số.
Đáp án cần chọn là: A

========================================================================

https://khoahoc.vietjack.com/thi-online/15-cau-trac-nghiem-ham-so-co-dap-an-nhan-biet


\textbf{{QUESTION}}

Điểm nào sau đây không thuộc đồ thị hàm số y=√x2-4x+4x$$ y=\frac{\sqrt{{x}^{2}-4x+4}}{x}$$
A.   A (2; 0).
B (3; 13$$ \frac{1}{3}$$).
A.   C (1; −1).
D (−1; −3).

\textbf{{ANSWER}}

Xét đáp án A, thay x = 2 và y = 0 vào hàm số $$ y=\frac{\sqrt{{x}^{2}-4x+4}}{x}$$ ta được $$ 0=\frac{\sqrt{{2}^{2}-4.2+4}}{2}:$$ thỏa mãn.
Xét đáp án B, thay x = 3 và y = $$ \frac{1}{3}$$ vào hàm số $$ y=\frac{\sqrt{{x}^{2}-4x+4}}{x}$$ ta được $$ \frac{1}{3}=\frac{\sqrt{{3}^{2}-4.3+4}}{3}:$$ thỏa mãn
Xét đáp án C, thay x = 1 và y = -1 vào hàm số $$ y=\frac{\sqrt{{x}^{2}-4x+4}}{x}$$ ta được $$ -1=\frac{\sqrt{{1}^{2}-4.1+4}}{1}\Leftrightarrow -1=1:$$ không thoả mãn
Đáp án cần chọn là: C

========================================================================

https://khoahoc.vietjack.com/thi-online/15-cau-trac-nghiem-ham-so-co-dap-an-nhan-biet


\textbf{{QUESTION}}

Cho hàm số y = {2x-1,x∈(-∞;0)√x+1,x∈[0;2]x2-1,x∈(2;5]$$ y\quad =\quad \left\{\begin{array}{l}\frac{2}{x-1},x\in \left(-\infty ;0\right)\\ \sqrt{x+1},x\in \left[0;2\right]\\ {x}^{2}-1,x\in \left(2;5\right]\end{array}\right.$$. Tính f(4), ta được kết quả:
A. 23$$ \frac{2}{3}$$
B. 15
C. √5$$ \sqrt{5}$$
D. 7

\textbf{{ANSWER}}

Ta thấy x = 4 ∈ (2; 5] ⇒ f(4) =42 – 1 = 15.
Đáp án cần chọn là: B

========================================================================

https://khoahoc.vietjack.com/thi-online/15-cau-trac-nghiem-ham-so-co-dap-an-nhan-biet


\textbf{{QUESTION}}

Cho hàm số y = f (x) = |−5x|. Khẳng định nào sau đây là sai?
A. f(−1) = 5.
B. f(2) = 10.
C.   f(−2) = 10.
D. f(15$$ \frac{1}{5}$$) = −1.

\textbf{{ANSWER}}

Ta có · f(−1) = |−5.(−1)| = |5| = 5⇒ A đúng.
f(2) = |−5.2| = |−10| = 10 ⇒ B đúng.
f(−2) = |−5.(−2)| = |10| = 10 ⇒ C đúng.
f ⇒ D sai. 
Đáp án cần chọn là: D

========================================================================

https://khoahoc.vietjack.com/thi-online/15-cau-trac-nghiem-ham-so-co-dap-an-nhan-biet


\textbf{{QUESTION}}

Tập xác định của hàm số y=x-1x2-x+3$$ y=\frac{x-1}{{x}^{2}-x+3}$$ là
A. ∅.
B. R
C. R∖{1}.
D. R∖{0}.

\textbf{{ANSWER}}

x2-x+3=x2-2.12x+14+114=(x-12)2+114>0,∀x∈R$$ {x}^{2}-x+3={x}^{2}-2.\frac{1}{2}x+\frac{1}{4}+\frac{11}{4}\phantom{\rule{0ex}{0ex}}={\left(x-\frac{1}{2}\right)}^{2}+\frac{11}{4}>0,\forall x\in R$$
Vậy tập xác định của hàm số là: R
Đáp án cần chọn là: B

========================================================================

https://khoahoc.vietjack.com/thi-online/10-bai-tap-dons-thuc-dong-dang-va-cong-tru-don-thuc-dong-dang-co-loi-giai


\textbf{{QUESTION}}

A. 2x2y và 3xy2;

\textbf{{ANSWER}}

Hướng dẫn giải:
Đáp án đúng là: B
4xy3z và (-4)xy3z là hai đơn thức đồng dạng vì có phần hệ số khác 0 và có phần biến giống nhau là xy3z.

========================================================================

https://khoahoc.vietjack.com/thi-online/10-bai-tap-dons-thuc-dong-dang-va-cong-tru-don-thuc-dong-dang-co-loi-giai


\textbf{{QUESTION}}

Đơn thức nào dưới đây đồng dạng với đơn thức 2x2y2z?
A. 2xy2;
B. xyz;

\textbf{{ANSWER}}

Hướng dẫn giải:
Đáp án đúng là: D
Đơn thức 3x2y2z đồng dạng với đơn thức 2x2y2z vì có phần hệ số khác 0 và có phần biến giống nhau là x2y2z.

========================================================================

https://khoahoc.vietjack.com/thi-online/10-bai-tap-dons-thuc-dong-dang-va-cong-tru-don-thuc-dong-dang-co-loi-giai


\textbf{{QUESTION}}

Cho các đơn thức ‐2x4y2; (√2+1)x4y2; x4y2; ‐2x2y4$$ ‐2{x}^{4}{y}^{2};\quad \left(\sqrt{2}+1\right){x}^{4}{y}^{2};\quad {x}^{4}{y}^{2};\quad ‐2{x}^{2}{y}^{4}$$. Có bao nhiêu đơn thức đồng dạng với đơn thức 35x4y2$$ \frac{3}{5}{x}^{4}{y}^{2}$$
A. 1;

\textbf{{ANSWER}}

Hướng dẫn giải:
Đáp án đúng là: C
Có 3 đơn thức đồng dạng với đơn thức 35x4y2$$ \frac{3}{5}{x}^{4}{y}^{2}$$  là: -2x4y2;  ; x4y2 vì có phần hệ số khác 0 và có phần biến giống nhau là x4y2.

========================================================================

https://khoahoc.vietjack.com/thi-online/10-bai-tap-dons-thuc-dong-dang-va-cong-tru-don-thuc-dong-dang-co-loi-giai


\textbf{{QUESTION}}

A. 35x2$$ \frac{3}{5}{x}^{2}$$ ; 0x2 và 3x2;
B. -3yz2; 3y2z2 và yz2;
C. ‐15x3y4z2; √3x3y4z2$$ ‐\frac{1}{5}{x}^{3}{y}^{4}{z}^{2};\quad \sqrt{3}{x}^{3}{y}^{4}{z}^{2}$$ và 2x3y4z2;
D. 13x3y2$$ \frac{1}{3}{x}^{3}{y}^{2}$$ ; 3x2y2; -4x3y2.

\textbf{{ANSWER}}

Hướng dẫn giải:
Đáp án đúng là: C
Ba đơn thức ‐15x3y4z2$$ ‐\frac{1}{5}{x}^{3}{y}^{4}{z}^{2}$$ ; √3x3y4z2$$ \sqrt{3}{x}^{3}{y}^{4}{z}^{2}$$ và 2x3y4z2 đồng dạng vì có phần hệ số khác 0 và có phần biến giống nhau là x3y4z2.

========================================================================

https://khoahoc.vietjack.com/thi-online/10-bai-tap-dons-thuc-dong-dang-va-cong-tru-don-thuc-dong-dang-co-loi-giai


\textbf{{QUESTION}}

Tổng của biểu thức 2x2y3 + 3x2y3 là: 
A. 5x2y3;

\textbf{{ANSWER}}

Hướng dẫn giải:
Đáp án đúng là: A
Ta có 2x2y3 + 3x2y3 = (2 + 3).x2y3 = 5x2y3.

========================================================================

https://khoahoc.vietjack.com/thi-online/giai-sbt-toan-9-canh-dieu-bai-3-dinh-li-viete-co-dap-an


\textbf{{QUESTION}}

Không tính ∆, giải các phương trình:
a) $7{x^2} + 3\sqrt 3 x - 7 + 3\sqrt 3 = 0;$
b) –2x2 + (5m + 1)x – 5m + 1 = 0.

\textbf{{ANSWER}}

a) Phương trình $7{x^2} + 3\sqrt 3 x - 7 + 3\sqrt 3 = 0$ có các hệ số: a = 7; $b = 3\sqrt 3 ;\,\,c = - 7 + 3\sqrt 3 .$
Ta thấy: $a - b + c = 7 - 3\sqrt 3 - 7 + 3\sqrt 3 = 0.$
Do đó, phương trình $7{x^2} + 3\sqrt 3 x - 7 + 3\sqrt 3 = 0$ có hai nghiệm là ${x_1} = - 1;\,\,{x_2} = \frac{{7 - 3\sqrt 3 }}{7}.$
b) Phương trình –2x2 + (5m + 1)x – 5m + 1 = 0 có các hệ số: a = ‒2; b = 5m + 1; c = ‒5m + 1.
Ta thấy: a + b + c = ‒2 + 5m + 1 ‒ 5m + 1 = 0.
Do đó, phương trình –2x2 + (5m + 1)x – 5m + 1 = 0 có hai nghiệm là ${x_1} = 1;\,\,{x_2} = \frac{{ - 5m + 1}}{{ - 2}} = \frac{{5m - 1}}{2}.$

========================================================================

https://khoahoc.vietjack.com/thi-online/giai-sbt-toan-9-canh-dieu-bai-3-dinh-li-viete-co-dap-an


\textbf{{QUESTION}}

Cho phương trình ${x^2} + x - 2 + \sqrt 2 = 0.$
a) Chứng tỏ rằng phương trình có hai nghiệm x1, x2 trái dấu.
b) Không giải phương trình, tính:
$A = x_1^2 + x_2^2;\,\,B = x_1^3 + x_2^3;$ $C = \frac{1}{{{x_1}}} + \frac{1}{{{x_2}}};$ D = |x1 – x2|.

\textbf{{ANSWER}}

a) Phương trình ${x^2} + x - 2 + \sqrt 2 = 0$ có $\Delta = {1^2} - 4 \cdot 1 \cdot \left( { - 2 + \sqrt 2 } \right) = 9 - 4\sqrt 2 > 0.$
Do đó phương trình trên có hai nghiệm phân biệt.
Theo định lí Viète, ta có ${x_1}{x_2} = - 2 + \sqrt 2 .$
Ta thấy tích của hai nghiệm là $ - 2 + \sqrt 2 < 0.$ 
Do đó phương trình có hai nghiệm x1, x2 trái dấu.
b) Theo định lí Viète, ta có ${x_1} + {x_2} = - 1;\,\,{x_1}{x_2} = - 2 + \sqrt 2 .$ Khi đó:
⦁ $A = x_1^2 + x_2^2 = x_1^2 + x_2^2 + 2{x_1}{x_2} - 2{x_1}{x_2} = {\left( {{x_1} + {\rm{ }}{x_2}} \right)^2} - 2{x_1}{x_2}$
$ = {\left( { - 1} \right)^2} - 2\left( { - 2 + \sqrt 2 } \right) = 1 + 4 - 2\sqrt 2 = 5 - 2\sqrt 2 .$
⦁ $B = x_1^3 + x_2^3 = \left( {{x_1} + {x_2}} \right)\left( {x_1^2 - {x_1}{x_2} + x_2^2} \right)$
$ = \left( {{x_1} + {x_2}} \right)\left( {x_1^2 + x_2^2 + 2{x_1}{x_2} - 3{x_1}{x_2}} \right)$
$ = \left( {{x_1} + {x_2}} \right)\left[ {{{\left( {{x_1} + {x_2}} \right)}^2} - 3{x_1}{x_2}} \right]$
$ = - 1 \cdot \left[ {{{\left( { - 1} \right)}^2} - 3\left( { - 2 + \sqrt 2 } \right)} \right]$
$ = - \left( {1 + 6 - 3\sqrt 2 } \right)$$ = - 7 + 3\sqrt 2 .$
⦁ $C = \frac{1}{{{x_1}}} + \frac{1}{{{x_2}}} = \frac{{{x_1} + {x_2}}}{{{x_1} \cdot {x_2}}} = \frac{{ - 1}}{{ - 2 + \sqrt 2 }} = \frac{1}{{2 - \sqrt 2 }}$
$ = \frac{1}{{2 - \sqrt 2 }} = \frac{{2 + \sqrt 2 }}{{\left( {2 - \sqrt 2 } \right)\left( {2 + \sqrt 2 } \right)}}$
$ = \frac{{2 + \sqrt 2 }}{{4 - 2}} = \frac{{2 + \sqrt 2 }}{2} = 1 + \frac{{\sqrt 2 }}{2}.$
⦁ D = |x1 – x2|
${D^2} = {\left| {{x_1} - {x_2}} \right|^2} = {\left( {{x_1} - {x_2}} \right)^2} = x_1^2 - 2{x_1}{x_2} + x_2^2 = {\left( {{x_1} + {x_2}} \right)^2} - 4{x_1}{x_2}$
 $ = 1 - 4 \cdot \left( { - 2 + \sqrt 2 } \right) = 1 + 8 - 4\sqrt 2 $
 $ = 9 - 4\sqrt 2 = {\left( {2\sqrt 2 } \right)^2} - 2 \cdot 2\sqrt 2 \cdot 1 + {1^2} = {\left( {2\sqrt 2 - 1} \right)^2}.$
Do đó $D = \sqrt {{{\left( {2\sqrt 2 - 1} \right)}^2}} = \left| {2\sqrt 2 - 1} \right| = 2\sqrt 2 - 1.$

========================================================================

https://khoahoc.vietjack.com/thi-online/10-bai-tap-rut-gon-bieu-thauc-co-loi-giai


\textbf{{QUESTION}}

Rút gọn biểu thức A = (a + b) – (–b – c) + (–a) là:
A. a + b + c;
B. 2b + c;
C. a – b – c;
D. 2b – c.

\textbf{{ANSWER}}

Đáp án đúng là: B 
A = (a + b) – (–b – c) + (–a)
= a + b + b + c – a
= (a – a) + (b + b) + c
= 2b + c

========================================================================

https://khoahoc.vietjack.com/thi-online/10-bai-tap-rut-gon-bieu-thauc-co-loi-giai


\textbf{{QUESTION}}

So sánh kết quả hai biểu thức A = (2a + b – c) – (–2b – c – a) và B = (–a – b) + 2. (a + b):
A. A = 3B;
B. A < B;
C. A = B3$\frac{B}{3}$;
D. Không so sánh được.

\textbf{{ANSWER}}

Đáp án đúng là: A
A = (2a + b – c) – (–2b – c–a)
= 2a + b – c + 2b + c + a
= (2a + a) + (2b + b) + (c – c)
= 3a + 3b
= 3. (a + b)
B = (–a – b) + 2. (a + b)
= –a – b + 2a + 2b
= a + b
Vậy A = 3B.

========================================================================

https://khoahoc.vietjack.com/thi-online/10-bai-tap-rut-gon-bieu-thauc-co-loi-giai


\textbf{{QUESTION}}

Cho A = x + 12 – (x – y + 8) + (2x + y – 15). Với x = 20, y = –16 thì giá trị của biểu thức A là:
A. –3;
B. –5;
C. 19;
D. 23.

\textbf{{ANSWER}}

Đáp án đúng là: A
x + 12 – (x – y + 8) + (2x + y –15)
= x + 12 – x + y – 8 + 2x + y –15
= (x – x + 2x) + (y + y) + (12 – 8 – 15)
= 2x + 2y –11
Thay x = 20, y = –16 vào biểu thức 2x + 2y – 11 ta được:
2x + 2y – 11
= 2. 20 + 2. (–16) –11
= 40 + (– 32) – 11 
= –3

========================================================================

https://khoahoc.vietjack.com/thi-online/10-bai-tap-rut-gon-bieu-thauc-co-loi-giai


\textbf{{QUESTION}}

Thay dấu “*” bằng một chữ số thích hợp để có (x – y + 5) – (–8– x + y) = 2x –*y + 13 :
A. * = 3;
B. * = 2;
C. * = –2;
D. * = 5.

\textbf{{ANSWER}}

Đáp án đúng là: B
(x – y + 5) – (–8 – x + y)
= x – y + 5 + 8 + x – y
= 2x – 2y + 13
Suy ra * = 2.

========================================================================

https://khoahoc.vietjack.com/thi-online/10-bai-tap-rut-gon-bieu-thauc-co-loi-giai


\textbf{{QUESTION}}

Nhận xét nào sau đây đúng về kết quả của biểu thức A = (a + b) – 3.( –b + a + 2) + (a – b):
A. Kết quả là một số nguyên âm;
B. Kết quả là một số nguyên dương;
C. Kết quả là một biểu thức chứa hai biến a, b;
D. Kết quả là một biểu thức chỉ chứa biến a.

\textbf{{ANSWER}}

Đáp án đúng là: C
A = (a + b) – 3.( –b + a + 2) + (a – b)
= a + b + 3b – 3a – 6 + a – b
= –a + 3b – 6
Vậy kết quả của biểu thức A là một biểu thức chứa hai biến a, b.

========================================================================

https://khoahoc.vietjack.com/thi-online/bai-tap-phan-tich-mot-so-ra-thua-so-nguyen-to-chon-loc-co-dap-an


\textbf{{QUESTION}}

Phân tích thừa số nguyên tố $$ a={p}_{1}^{{m}_{1}}.{p}_{2}^{{m}_{2}}.{p}_{3}^{{m}_{3}}....{p}_{k}^{{m}_{g}}$$, khẳng định nào sau đây đúng?
A. Các số $$ {p}_{1};{p}_{2};...;{p}_{k}$$ là các số dương.
B. Các số $$ {p}_{1};{p}_{2};...;{p}_{k}$$ là các số nguyên tố
C. Các số $$ {p}_{1};{p}_{2};...;{p}_{k}$$ là các số tự nhiên.
D. Các số $$ {p}_{1};{p}_{2};...;{p}_{k}$$ tùy ý.

\textbf{{ANSWER}}

Đáp án là B
Khi phân tích một số $$ a={p}_{1}^{{m}_{1}}.{p}_{2}^{{m}_{2}}.{p}_{3}^{{m}_{3}}....{p}_{k}^{{m}_{g}}$$ ra thừa số nguyên tố thì $$ {p}_{1};{p}_{2};...;{p}_{k}$$ là các số nguyên tố.

========================================================================

https://khoahoc.vietjack.com/thi-online/bai-tap-phan-tich-mot-so-ra-thua-so-nguyen-to-chon-loc-co-dap-an


\textbf{{QUESTION}}

Phân tích số 18 ra thừa số nguyên tố
A. 18 = 18.1     
B. 18 = 10 + 8     
C. 18 = 2.32$$ {3}^{2}$$    
D. 18 = 6 + 6 + 6

\textbf{{ANSWER}}

Đáp án là C
   + Đáp án A sai vì 1 không phải là số nguyên tố
    + Đáp án B sai vì đây là phép cộng.
    + Đáp án C đúng vì 2 và 3 là hai số nguyên tố nên 18 = 2.32$$ {3}^{2}$$
    + Đáp án D sai vì đây là phép cộng

========================================================================

https://khoahoc.vietjack.com/thi-online/bai-tap-phan-tich-mot-so-ra-thua-so-nguyen-to-chon-loc-co-dap-an


\textbf{{QUESTION}}

Cho a = 22.7, hãy viết tập hợp tất cả các ước của a
A. Ư(a) = {4; 7}     
B. Ư(a) = {1; 4; 7}
C. Ư(a) = {1; 2; 4; 7; 28} 
D. Ư(a) = {1; 2; 4; 7; 14; 28}

\textbf{{ANSWER}}

Đáp án là D
Ta có: a = 22$$ {2}^{2}$$.7 = 4.7 = 28
28 = 28.1 = 14.2 = 7.4 = 7.2.2
Vậy Ư(28) = {1; 2; 4; 7; 14; 28}

========================================================================

https://khoahoc.vietjack.com/thi-online/bai-tap-phan-tich-mot-so-ra-thua-so-nguyen-to-chon-loc-co-dap-an


\textbf{{QUESTION}}

Cho a2$$ {a}^{2}$$.b.7 = 140, với a, b là các số nguyên tố, vậy a có giá trị bằng bao nhiêu?
A. 1     
B. 2    
C. 3     
D. 4

\textbf{{ANSWER}}

Đáp án là B
Ta có a2$$ {a}^{2}$$.b.7 = 140 ⇒ a2$$ {a}^{2}$$b = 20 =22$$ {2}^{2}$$.5
Vậy giá trị của a là 2

========================================================================

https://khoahoc.vietjack.com/thi-online/bai-tap-phan-tich-mot-so-ra-thua-so-nguyen-to-chon-loc-co-dap-an


\textbf{{QUESTION}}

Cho số 150 = 2.3.52$$ {5}^{2}$$, số lượng ước của 150 là bao nhiêu?
A. 6    
B. 7     
C. 8     
D. 12

\textbf{{ANSWER}}

Đáp án là D
Nếu m = axbycz$$ {a}^{x}{b}^{y}{c}^{z}$$, với a, b, c là số nguyên tố thì m có (x + 1)(y + 1)(z + 1) ước.
Ta có 150 = 2.3.52$$ {5}^{2}$$ với x = 1; y = 1; z = 2
Vậy số lượng ước số của 150 là (1 + 1)(1 + 1)(2 + 1) = 12 ước.

========================================================================

https://khoahoc.vietjack.com/thi-online/15-cau-trac-nghiem-toan-9-canh-dieu-bai-1-mo-ta-va-bieu-dien-du-lieu-tren-cac-bang-bieu-do-co-dap-an


\textbf{{QUESTION}}

I. Nhận biết
Khi biểu diễn dữ liệu trên biểu đồ tranh, các đối tượng thống kê thường được biểu diễn ở
A. Dòng bất kì.
B. Cột tương ứng.
C. Cột đầu tiên.
D. Dòng tương ứng.

\textbf{{ANSWER}}

Đáp án đúng là: C
Để biểu diễn dữ liệu trên biểu đồ tranh, ta có thể làm như sau:
Bước 1. Các đối tượng thống kê được biểu diễn ở cột đầu tiên.
Bước 2. Chọn biểu tượng để biểu diễn số liệu thống kê. Các biểu tượng đó được trình bày ở dòng cuối cùng.
Bước 3. Số liệu thống kê theo tiêu chí của mỗi đối tượng thống kê được biểu diễn bằng các biểu tượng ở dòng tương ứng.
Vậy ta chọn phương án C.

========================================================================

https://khoahoc.vietjack.com/thi-online/15-cau-trac-nghiem-toan-9-canh-dieu-bai-1-mo-ta-va-bieu-dien-du-lieu-tren-cac-bang-bieu-do-co-dap-an


\textbf{{QUESTION}}

Nguyên tắc chuyển đổi số liệu của mỗi đối tượng thống kê (tính theo tỉ số phần trăm) về số đo cung tương ứng với đối tượng thống kê đó (tính theo độ) là
A. x%$x\% $ tương ứng với x%⋅180∘.$x\%  \cdot 180^\circ .$
B. x%$x\% $ tương ứng với x%⋅360∘.$x\%  \cdot 360^\circ .$
C. x%$x\% $ tương ứng với x%360∘.$\frac{{x\% }}{{360^\circ }}.$
D. x%$x\% $ tương ứng với x%⋅90∘.$x\%  \cdot 90^\circ .$

\textbf{{ANSWER}}

Đáp án đúng là: B
Chuyển đổi số liệu của mỗi đối tượng thống kê (tính theo tỉ số phần trăm) về số đo cung tương ứng với đối tượng thống kê đó (tính theo độ) dựa theo nguyên tắc sau: $x\% $ tương ứng với $x\%  \cdot 360^\circ .$ 
Vậy ta chọn phương án B.

========================================================================

https://khoahoc.vietjack.com/thi-online/15-cau-trac-nghiem-toan-9-canh-dieu-bai-1-mo-ta-va-bieu-dien-du-lieu-tren-cac-bang-bieu-do-co-dap-an


\textbf{{QUESTION}}

Muốn so sánh hai tập dữ liệu với nhau, ta nên dùng
A. Biểu đồ cột kép.
B. Biểu đồ cột.
C. Biểu đồ tranh.
D. Biểu đồ hình quạt tròn.

\textbf{{ANSWER}}

Đáp án đúng là: A
Nếu mỗi đối tượng thống kê đều có hai số liệu thống kê theo hai tiêu chí khác nhau thì ta nên dùng biểu đồ cột kép để biểu diễn dữ liệu. 
Ngoài ra, khi muốn so sánh hai tập dữ liệu với nhau, ta cũng dùng biểu đồ cột kép.
Vậy ta chọn phương án A.

========================================================================

https://khoahoc.vietjack.com/thi-online/15-cau-trac-nghiem-toan-9-canh-dieu-bai-1-mo-ta-va-bieu-dien-du-lieu-tren-cac-bang-bieu-do-co-dap-an


\textbf{{QUESTION}}

Để biểu diễn sự thay đổi của một đại lượng theo thời gian ta thường dùng
A. Biểu đồ tranh.
B. Biểu đồ cột kép.
C. Biểu đồ hình quạt tròn.
D. Biểu đồ đoạn thẳng.

\textbf{{ANSWER}}

Đáp án đúng là: D
Để biểu diễn sự thay đổi số liệu của các đối tượng thống kê theo thời gian, ta thường dùng biểu đồ đoạn thẳng.
Do đó ta chọn phương án D.

========================================================================

https://khoahoc.vietjack.com/thi-online/15-cau-trac-nghiem-toan-9-canh-dieu-bai-1-mo-ta-va-bieu-dien-du-lieu-tren-cac-bang-bieu-do-co-dap-an


\textbf{{QUESTION}}

Trong biểu đồ hình quạt tròn, nửa đường tròn biểu diễn
A. 25%
B. 50%.
C. 75%.
D. 100%.

\textbf{{ANSWER}}

Đáp án đúng là: B
Trong biểu đồ hình quạt tròn, nửa đường tròn biểu diễn 50%.
Do đó ta chọn phương án B.

========================================================================

https://khoahoc.vietjack.com/thi-online/15-cau-trac-nghiem-toan-7-chan-troi-sang-tao-bai-2-lam-quen-voi-xac-suat-cua-bien-co-ngau-nhien-phan/109920


\textbf{{QUESTION}}

Một hộp có 20 tấm thẻ được đánh số từ 1 đến 20. Các tấm thẻ có kích thước như nhau. Lấy ngẫu nhiên một tấm thẻ từ hộp. Gọi X là biến cố: “Rút được tấm thẻ ghi số không lớn hơn 20”. Xác suất của biến cố X là:

\textbf{{ANSWER}}

Đáp án đúng là: C
Tất cả các tấm thẻ trong hộp đều được đánh số từ 1 đến 20, nên tất cả các số đều không lớn hơn (tức là nhỏ hơn hoặc bằng) 20.
Do đó biến cố X là biến cố chắc chắn.
Vì vậy xác suất của biến cố X là 1.
Ta chọn phương án C.

========================================================================

https://khoahoc.vietjack.com/thi-online/15-cau-trac-nghiem-toan-7-chan-troi-sang-tao-bai-2-lam-quen-voi-xac-suat-cua-bien-co-ngau-nhien-phan/109920


\textbf{{QUESTION}}

Gieo một con xúc xắc 6 mặt cân đối. Gọi M là biến cố: “Gieo được mặt có số chấm là ước của 4”. Xác suất của biến cố M là:

\textbf{{ANSWER}}

Đáp án đúng là: B
Gieo một con xúc xắc 6 mặt cân đối thì có 6 kết quả có thể xảy ra đối với số chấm trên mặt xuất hiện của con xúc xắc là: 1; 2; 3; 4; 5; 6.
Trong các số 1; 2; 3; 4; 5; 6 thì có 3 số là ước của 4 là 1; 2; 4.
Do đó xác suất xảy ra của biến cố M là P(M)=36=12$$ P\left(M\right)=\frac{3}{6}=\frac{1}{2}$$.
Vậy ta chọn phương án B.

========================================================================

https://khoahoc.vietjack.com/thi-online/15-cau-trac-nghiem-toan-7-chan-troi-sang-tao-bai-2-lam-quen-voi-xac-suat-cua-bien-co-ngau-nhien-phan/109920


\textbf{{QUESTION}}

A. 0;
D. 1

\textbf{{ANSWER}}

Đáp án đúng là: B
Số kết quả có thể xảy ra khi tung đồng xu là 2 (mặt sấp hoặc mặt ngửa).
Do 2 kết quả đó đều có khả năng xảy ra như nhau nên P(K)=12$$ P\left(K\right)=\frac{1}{2}$$.
Vậy ta chọn phương án B.

========================================================================

https://khoahoc.vietjack.com/thi-online/15-cau-trac-nghiem-toan-7-chan-troi-sang-tao-bai-2-lam-quen-voi-xac-suat-cua-bien-co-ngau-nhien-phan/109920


\textbf{{QUESTION}}

Một chiếc hộp chứa 5 quả cầu màu đỏ và 9 quả cầu màu vàng. Các quả cầu có kích thước và trọng lượng như nhau. Lấy ngẫu nhiên hai quả cầu từ trong hộp. Xác suất của biến cố A: “Lấy được hai quả cầu màu trắng” là:
A. P(A) = 1;

\textbf{{ANSWER}}

Đáp án đúng là: D
Vì trong hộp không có quả cầu màu trắng nào nên biến cố A là biến cố không thể.
Do đó xác suất của biến cố A là P(A) = 0.
Vậy ta chọn phương án D.

========================================================================

https://khoahoc.vietjack.com/thi-online/15-cau-trac-nghiem-toan-7-chan-troi-sang-tao-bai-2-lam-quen-voi-xac-suat-cua-bien-co-ngau-nhien-phan/109920


\textbf{{QUESTION}}

Một chiếc bình thủy tinh đựng 1 ngôi sao giấy màu tím, 1 ngôi sao giấy màu xanh, 1 ngôi sao giấy màu vàng, 1 ngôi sao giấy màu đỏ. Các ngôi sao có kích thước và khối lượng như nhau. Lấy ngẫu nhiên 1 ngôi sao từ trong bình. Cho biến cố Y: “Lấy được 1 ngôi sao màu tím hoặc màu đỏ”. Xác suất của biến cố Y là:
A. 14$$ \frac{1}{4}$$;

\textbf{{ANSWER}}

Đáp án đúng là: B
Trong bình có tất cả 4 ngôi sao có màu khác nhau: màu tím, màu xanh, màu vàng và màu đỏ.
Do các ngôi sao có cùng kích thước và khối lượng như nhau nên các ngôi sao đều có cùng khả năng được chọn.
Vì trong bình có 1 ngôi sao màu tím và 1 ngôi sao màu đỏ nên có 2 kết quả làm cho biến cố Y: “Lấy được 1 ngôi sao màu tím hoặc màu đỏ” xảy ra.
Do đó xác suất của biến cố Y là P(Y)=24=12.$$ P\left(Y\right)=\frac{2}{4}=\frac{1}{2}.$$
Vậy ta chọn phương án B.

========================================================================

https://khoahoc.vietjack.com/thi-online/de-kiem-tra-giua-ki-1-toan-8-ctst-co-dap-an


\textbf{{QUESTION}}

Trong các biểu thức đại số sau, biểu thức nào không phải là đơn thức?
A. x
B. $$ \frac{1}{2}x{y}^{3}$$
C. $$ 3x\quad -\quad 4$$
D. $$ -7$$

\textbf{{ANSWER}}

Chọn đáp án C

========================================================================

https://khoahoc.vietjack.com/thi-online/de-kiem-tra-giua-ki-1-toan-8-ctst-co-dap-an


\textbf{{QUESTION}}

Tích của đa thức 6xy$$ 6xy$$  và đa thức 2x2−3y$$ 2{x}^{2}-3y$$  là đa thức
A. 12x2y+18xy2$$ 12{x}^{2}y+18x{y}^{2}$$
B. 12x3y−18xy2$$ 12{x}^{3}y-18x{y}^{2}$$
C. 12x3y+18xy2$$ 12{x}^{3}y+18x{y}^{2}$$
D. 12x2y−18xy2$$ 12{x}^{2}y-18x{y}^{2}$$

\textbf{{ANSWER}}

Chọn đáp án B

========================================================================

https://khoahoc.vietjack.com/thi-online/de-kiem-tra-giua-ki-1-toan-8-ctst-co-dap-an


\textbf{{QUESTION}}

Thực hiện tính (13x3y3+2x2y4):(xy2)$$ \left(\frac{1}{3}{x}^{3}{y}^{3}+2{x}^{2}{y}^{4}\right):\left(x{y}^{2}\right)$$  được kết quả là
A. 13x2y+2x2y$$ \frac{1}{3}{x}^{2}y+2{x}^{2}y$$
B. 13x2y+2xy2$$ \frac{1}{3}{x}^{2}y+2x{y}^{2}$$
C. 12x2y+xy2$$ \frac{1}{2}{x}^{2}y+x{y}^{2}$$
D. 12x2y+2xy$$ \frac{1}{2}{x}^{2}y+2xy$$

\textbf{{ANSWER}}

Chọn đáp án B

========================================================================

https://khoahoc.vietjack.com/thi-online/de-kiem-tra-giua-ki-1-toan-8-ctst-co-dap-an


\textbf{{QUESTION}}

Hằng đẳng thức A2−B2=(A−B)(A+B)  có tên là
D. hiệu hai bình phương.

\textbf{{ANSWER}}

Chọn đáp án D

========================================================================

https://khoahoc.vietjack.com/thi-online/de-kiem-tra-giua-ki-1-toan-8-ctst-co-dap-an


\textbf{{QUESTION}}

Tính giá trị biểu thức A=8x3+12x2+6x+1  tại x = 9,5 .
B. 400
C. 4000
D. 8000

\textbf{{ANSWER}}

Chọn đáp án D

========================================================================

https://khoahoc.vietjack.com/thi-online/10-bais-tap-tong-cua-n-so-hang-dau-tien-cua-mot-cap-so-cong-co-loi-giai


\textbf{{QUESTION}}

Cho cấp số cộng: −4; −8; −12; −16;...Tổng của 10 số hạng đầu tiên là

\textbf{{ANSWER}}

Đáp án đúng là: B
Cấp số cộng có công thức tổng quát cho số hạng thứ n là: un = – 4 + (– 4).(n – 1) .
Áp dụng công thức $$ {S}_{n}=\frac{n\left[2{u}_{1}+(n-1)d\right]}{2}$$ nên 
Tổng của 10 số hạng đầu tiên của cấp số cộng:
$$ {S}_{10}=\frac{10\cdot \left[2\cdot \left(-\text{\hspace{0.17em}}4\right)+\left(10-1\right)\cdot \left(-\text{\hspace{0.17em}}4\right)\right]}{2}=-220$$.

========================================================================

https://khoahoc.vietjack.com/thi-online/10-bais-tap-tong-cua-n-so-hang-dau-tien-cua-mot-cap-so-cong-co-loi-giai


\textbf{{QUESTION}}

Cho cấp số cộng (un) thỏa mãn:$$ \left\{\begin{array}{c}{u}_{5}+3{u}_{3}-{u}_{2}=-21\\ 3{u}_{7}-2{u}_{4}=-34\end{array}\right.$$ . Giá trị tổng 20 số hạng đầu của cấp số cộng là

\textbf{{ANSWER}}

Đáp án đúng là: B
Từ giả thiết bài toán, ta có:
$$ \left\{\begin{array}{c}{u}_{1}+4d+3\left({u}_{1}+2d\right)-\left({u}_{1}+d\right)=-21\\ 3\left({u}_{1}+6d\right)-2\left({u}_{1}+3d\right)=-34\end{array}\right.$$

$$ \Leftrightarrow \left\{\begin{array}{l}3{u}_{1}+9d=-21\\ {u}_{1}+12d=-34\end{array}\Leftrightarrow \left\{\begin{array}{c}{u}_{1}=2\\ d=-3\end{array}.\right.\right.$$
Tổng của 20 số hạng đầu: $$ {S}_{20}=\frac{20\left(2{u}_{1}+19d\right)}{2}=-530.$$

========================================================================

https://khoahoc.vietjack.com/thi-online/trac-nghiem-chuyen-de-toan-8-chu-de-2-nhan-da-thuc-voi-da-thuc-co-dap-an


\textbf{{QUESTION}}

A. x2 - 2x - 10.

\textbf{{ANSWER}}

Ta có ( x - 2 )( x + 5 ) = x( x + 5 ) - 2( x + 5 )
= x2 + 5x - 2x - 10 = x2 + 3x - 10.
Chọn đáp án B

========================================================================

https://khoahoc.vietjack.com/thi-online/trac-nghiem-chuyen-de-toan-8-chu-de-2-nhan-da-thuc-voi-da-thuc-co-dap-an


\textbf{{QUESTION}}

Thực hiện phép tính ( 5x - 1 )( x + 3 ) - ( x - 2 )( 5x - 4 ) ta có kết quả là ?
A. 28x-3
B. 28x-5
C. 28x-11
D. 28x-8

\textbf{{ANSWER}}

Ta có ( 5x - 1 )( x + 3 ) - ( x - 2 )( 5x - 4 ) = 5x( x + 3 ) - ( x + 3 ) - x( 5x - 4 ) + 2( 5x - 4 )
= 5x2 + 15x - x - 3 - 5x2 + 4x + 10x - 8 = 28x - 11
Chọn đáp án C

========================================================================

https://khoahoc.vietjack.com/thi-online/trac-nghiem-chuyen-de-toan-8-chu-de-2-nhan-da-thuc-voi-da-thuc-co-dap-an


\textbf{{QUESTION}}

Giá trị của x thỏa mãn ( x + 1 )( 2 - x ) - ( 3x + 5 )( x + 2 ) = - 4x2 + 1 là ?
A. x=-1
B. x= -910$$ \frac{9}{10}$$
C. -310$$ \frac{3}{10}$$
D. x=0

\textbf{{ANSWER}}

Ta có ( x + 1 )( 2 - x ) - ( 3x + 5 )( x + 2 ) = - 4x2 + 1
⇔ ( 2x - x2 + 2 - x ) - ( 3x2 + 6x + 5x + 10 ) = - 4x2 + 1
⇔ - 4x2 - 10x - 8 = - 4x2 + 1 ⇔ - 10x = 9 ⇔ x = -910$$ \frac{-9}{10}$$
Vậy giá trị x cần tìm là x = - 910$$ \frac{9}{10}$$.
Chọn đáp án B

========================================================================

https://khoahoc.vietjack.com/thi-online/trac-nghiem-chuyen-de-toan-8-chu-de-2-nhan-da-thuc-voi-da-thuc-co-dap-an


\textbf{{QUESTION}}

A. 0
B. 40x
C. -40x
D. Kết quả khác

\textbf{{ANSWER}}

Ta có A = ( 2x - 3 )( 4 + 6x ) - ( 6 - 3x )( 4x - 2 )
= ( 8x + 12x2 - 12 - 18x ) - ( 24x - 12 - 12x2 + 6x )
= 12x2 - 10x - 12 - 30x + 12x2 + 12 = 24x2 - 40x.
Chọn đáp án D.

========================================================================

https://khoahoc.vietjack.com/thi-online/trac-nghiem-chuyen-de-toan-8-chu-de-2-nhan-da-thuc-voi-da-thuc-co-dap-an


\textbf{{QUESTION}}

Thực hiện các phép tính sau
a, ( x2 -1 )( x2 + 2x )

\textbf{{ANSWER}}

a) Ta có: ( x2 -1 )( x2 + 2x ) = x2( x2 + 2x ) - ( x2 + 2x )
= x4 + 2x3 - x2 - 2x

========================================================================

https://khoahoc.vietjack.com/thi-online/bai-tap-chuyen-de-toan-6-dang-3-hai-bai-toan-ve-phan-so-co-dap-an/107671


\textbf{{QUESTION}}

Muốn tìm một số biết $$ \frac{m}{n}$$  của nó bằng a ta làm thế nào?

\textbf{{ANSWER}}

Chọn đáp án A

========================================================================

https://khoahoc.vietjack.com/thi-online/bai-tap-chuyen-de-toan-6-dang-3-hai-bai-toan-ve-phan-so-co-dap-an/107671


\textbf{{QUESTION}}

Tìm một số biết 23$$ \frac{2}{3}$$  của nó bằng 72 . Số đó là:
A. 48
B. 108
C. 1108$$ \frac{1}{108}$$
D. 7113$$ 71\frac{1}{3}$$

\textbf{{ANSWER}}

Chọn đáp án B

========================================================================

https://khoahoc.vietjack.com/thi-online/bai-tap-chuyen-de-toan-6-dang-3-hai-bai-toan-ve-phan-so-co-dap-an/107671


\textbf{{QUESTION}}

Tìm một số biết 134$$ 1\frac{3}{4}$$  của nó bằng 35 . Số đó là:
A. 20
B. 30
C. 120$$ \frac{1}{20}$$
D. 6114$$ 61\frac{1}{4}$$

\textbf{{ANSWER}}

Chọn đáp án A

========================================================================

https://khoahoc.vietjack.com/thi-online/bai-tap-chuyen-de-toan-6-dang-3-hai-bai-toan-ve-phan-so-co-dap-an/107671


\textbf{{QUESTION}}

Tìm một số biết 212  của nó bằng 45 . Số đó là:
A. 24
B. 54
C. $$ \frac{1}{20}$$
D. $$ 61\frac{1}{4}$$

\textbf{{ANSWER}}

Chọn đáp án B

========================================================================

https://khoahoc.vietjack.com/thi-online/bai-tap-chuyen-de-toan-6-dang-3-hai-bai-toan-ve-phan-so-co-dap-an/107671


\textbf{{QUESTION}}

23
23
23
2
2
3
3


3
3
3

\textbf{{ANSWER}}

23$$ \frac{2}{3}$$ của nó bằng 7,2  nên số đó bằng:7,2:23=7,2.32=(7,2:2).3=3,6.3=10,8$$ \text{7},\text{2}:\frac{2}{3}=\text{7},\text{2}.\frac{3}{2}=\left(\text{7},\text{2}:\text{2}\right).\text{3}=\text{3},\text{6}.\text{3}=\text{1}0,\text{8}$$

========================================================================

https://khoahoc.vietjack.com/thi-online/bai-tap-toan-8-chu-de-11-on-tap-chuong-3-co-dap-an


\textbf{{QUESTION}}

Phát biểu và viết tỉ lệ thức biểu thị hai đoạn thẳng AB và CD tỉ lệ với hai đoạn thẳng A'B' và C'D'.

\textbf{{ANSWER}}

Hai đoạn thẳng AB và CD gọi là tỉ lệ với hai đoạn thẳng A'B' và C'D' nếu có hệ thức:
      $$ \frac{AB}{CD}=\frac{{A}_{1}{B}_{1}}{{C}_{1}{D}_{1}}$$ hoặc $$ \frac{AB}{{A}_{1}{B}_{1}}=\frac{CD}{{C}_{1}{D}_{1}}$$.

========================================================================

https://khoahoc.vietjack.com/thi-online/bai-tap-tu-luan-toan-6-bai-4-co-dap-an-cong-hai-so-nguyen-cung-dau


\textbf{{QUESTION}}

Thực hiện các phép tính.
a) (−75) + (−31).
b) (−19) + (+48).
c) 12 + (−53).
d) (−85) + (+85).

\textbf{{ANSWER}}

a) (−75) + (−31) = −106.
b) (−19) + (+48) = 29.
c) 12 + (−53) = −41.
d) (−85) + (+85) = 0.

========================================================================

https://khoahoc.vietjack.com/thi-online/bai-tap-tu-luan-toan-6-bai-4-co-dap-an-cong-hai-so-nguyen-cung-dau


\textbf{{QUESTION}}

Tính
1. Tổng của số nguyên âm lớn nhất có hai chữ số với số nguyên dương lớn nhất có hai chữ số.
2. Tổng của số liền trước số −73 với số liền sau số −17.

\textbf{{ANSWER}}

1. Số nguyên âm lớn nhất có hai chữ số là −10. Số nguyên dương lớn nhất có hai chữ số là 99. Tổng của chúng là (−10) + 99 = 89.
2. Số liền trước số −73 là −74. Số liền sau số −17 là −16. Tổng của chúng là (−74) + (−16) = −90.

========================================================================

https://khoahoc.vietjack.com/thi-online/bai-tap-tu-luan-toan-6-bai-4-co-dap-an-cong-hai-so-nguyen-cung-dau


\textbf{{QUESTION}}

Tính bằng cách hợp lí nhất.
(−37) + (+25) + (−63) + (−25) + (−9).

\textbf{{ANSWER}}

Ta có
(−37) + (+25) + (−63) + (−25) + (−9)
=[(−37) + (−63)] + [(+25) + (−25)] + (−9)

========================================================================

https://khoahoc.vietjack.com/thi-online/bai-tap-tu-luan-toan-6-bai-4-co-dap-an-cong-hai-so-nguyen-cung-dau


\textbf{{QUESTION}}

Tính tổng S = 1 + (−3) + 5 + (−7) + · · · + 21 + (−23).

\textbf{{ANSWER}}

Ta có
S = 1 + (−3) + 5 + (−7) + · · · + 21 + (−23).
Số các số hạng của tổng này là  
 (23 − 1) : 2 + 1 = 12 (số hạng).
S =[1 + (−3)] + [5 + (−7)] + · · · + [21 + (−23)]
S =(−2) + (−2) + · · · + (−2) (có 6 số hạng).
S = − 12.

========================================================================

https://khoahoc.vietjack.com/thi-online/bai-tap-tu-luan-toan-6-bai-4-co-dap-an-cong-hai-so-nguyen-cung-dau


\textbf{{QUESTION}}

Tính tổng các số nguyên x, biết −5≤$$ \le $$x < 5.

\textbf{{ANSWER}}

Vì x ∈ℤ$$ \in \mathrm{\mathbb{Z}}$$ nên x ∈$$ \in $$ {−5; ±$$ \pm $$4; ±$$ \pm $$3; ±$$ \pm $$2; ±$$ \pm $$1; 0}.
Tổng của chúng là :
S = (−5) + (−4 + 4) + (−3 + 3) + (−2 + 2) + (−1 + 1) + 0
S = −5 + 0 + 0 + · · · + 0
S = −5.

========================================================================

https://khoahoc.vietjack.com/thi-online/de-kiem-tra-45-phut-toan-6-chuong-3-so-hoc


\textbf{{QUESTION}}

Tính (tính nhanh nếu có thể):
a) 20,7 + 1,47 : 7 - 0,23 . 5

\textbf{{ANSWER}}

a) 20,7 + 1,47 : 7 – 0,23.5
= 20,7 + 0,21 – 1,15
= 20,91 – 1,15
= 19,76

========================================================================

https://khoahoc.vietjack.com/thi-online/trac-nghiem-chuyen-de-toan-9-chuyen-de-5-cac-bai-toan-thuc-te-giai-bang-cach-lap-phuong-trinh-va-he/103967


\textbf{{QUESTION}}

Hai người cùng làm chung một công việc thì sau 3 giờ 36 phút làm xong. Nếu làm một mình thì người thứ nhất hoàn thành công việc sớm hơn người thứ hai là 3 giờ. Hỏi nếu mỗi người làm một mình thì bao lâu xong công việc.

\textbf{{ANSWER}}

Gọi ẩn là thời gian người thứ nhất làm một mình xong công việc và lập bảng:
 
Thời gian hoàn thành công việc (giờ)
Năng suất làm trong 1 giờ
Hai người
$\frac{{18}}{5}$
$\frac{5}{{18}}$
Người thứ nhất
x
$\frac{1}{x}$
Người thứ hai
$x + 3$
$\frac{1}{{x + 3}}$
Đổi 3 giờ 36 phút$ = 3\frac{3}{5}\left( h \right) = \frac{{18}}{5}\left( h \right).$ 
Gọi x (giờ) là thời gian người thứ nhất làm một mình xong công việc. Điều kiện: $x > 0.$ 
Khi đó thời gian người thứ hai làm một mình xong công việc là $x + 3$ (giờ).
Trong 1 giờ:
+ Người thứ nhất làm được $\frac{1}{x}$ công việc.
+ Người thứ hai làm được $\frac{1}{{x + 3}}$ công việc.
+ Cả hai người làm được $\frac{5}{{18}}$ công việc
 Ta có phương trình: $\frac{1}{{x + 3}} + \frac{1}{x} = \frac{5}{{18}}$
 
Vậy người thứ nhất làm một mình thì 6 giờ xong công việc, 9 giờ xong công việc.

========================================================================

https://khoahoc.vietjack.com/thi-online/trac-nghiem-chuyen-de-toan-9-chuyen-de-5-cac-bai-toan-thuc-te-giai-bang-cach-lap-phuong-trinh-va-he/103967


\textbf{{QUESTION}}

Hai vòi nước cùng chảy vào một bể nước cạn (không có nước) trong 1 giờ 12 phút thì đầy bể. Nếu vòi thứ nhất chảy trong 30 phút và vòi thứ hai chảy trong 1 giờ thi được 712$\frac{7}{{12}}$ bể. Hỏi nếu mỗi vòi chảy một mình thi bao lâu đầy bể?
(Thi thử THPT Lương Thế Vinh - Hà Nội năm học 2018-2019)

\textbf{{ANSWER}}

Đổi 1 giờ 12 phút$ = 1\frac{1}{5} = \frac{6}{5}\left( h \right),$ 30 phút$ = \frac{1}{2}\left( h \right).$ 
Gọi thời gian vòi thứ nhất và vòi thứ hai chảy một mình đầy bể lần lượt là x, y (giờ). 
Điều kiện: $x > \frac{6}{5},y > \frac{6}{5}$ 
Trong 1 giờ:
+ Vòi thứ nhất chảy được $\frac{1}{x}$ bể.
+ Vòi thứ hai chảy được $\frac{1}{y}$ bể.
+ Cả hai vòi chảy được $1:\frac{6}{5} = \frac{5}{6}$ bể.
Suy ra phương trình: $\frac{1}{x} + \frac{1}{y} = \frac{5}{6}$ (1)
Trong 30 phút, vòi thứ nhất chảy được $\frac{1}{x}:2 = \frac{1}{{2x}}$ bể.
Vì nếu vòi thứ nhất chảy trong 30 phút và vòi thứ hai chảy trong 1 giờ thì được $\frac{7}{{12}}$ bể, nên
$\frac{1}{{2x}} + \frac{1}{y} = \frac{7}{{12}}$    (2)
Từ (1) và (2) ta có hệ phương trình: $\left\{ \begin{array}{l}\frac{1}{x} + \frac{1}{y} = \frac{5}{6}\\\frac{1}{{2x}} + \frac{1}{y} = \frac{7}{{12}}\end{array} \right. \Leftrightarrow \left\{ \begin{array}{l}\frac{1}{x} = \frac{1}{2}\\\frac{1}{y} = \frac{1}{3}\end{array} \right. \Leftrightarrow \left\{ \begin{array}{l}x = 2\\y = 3\end{array} \right.$ (thỏa mãn). 
Vậy thời gian vòi thứ nhất chảy một mình đầy bể là 2 giờ, thời gian vòi thứ hai chảy một mình đầy bể là 3 giờ.

========================================================================

https://khoahoc.vietjack.com/thi-online/trac-nghiem-chuyen-de-toan-9-chuyen-de-5-cac-bai-toan-thuc-te-giai-bang-cach-lap-phuong-trinh-va-he/103967


\textbf{{QUESTION}}

Để chuẩn bị cho một chuyến đi đánh bắt cá ở Hoàng Sa, hai ngư dân đảo Lý Sơn cần chuyển một số lương thực, thực phẩm lên tàu. Nếu người thứ nhất chuyển xong một nửa số lương thực, thực phẩm; sau đó người thứ hai chuyển hết số còn lại lên tàu thì thời gian người thứ hai hoàn thành lâu hơn người thứ nhất là 3 giờ. Nếu cả hai cùng làm chung thì thời gian chuyển hết số lương thực, thực phẩm lên tàu là 207$\frac{{20}}{7}$ giờ. Hỏi nếu làm riêng một mình thì mỗi người chuyển hết số lương thực, thực phẩm đó lên tàu trong thời gian bao lâu?
(Sở Quảng Ngãi năm học 2014-2015)

\textbf{{ANSWER}}

Gọi x (giờ) là thời gian người thứ nhất một mình làm xong cả công việc, y (giờ) là thời gian người thứ hai một mình làm xong cả công việc.
Điêu kiện: x,y>207.$x,{\rm{ }}y > \frac{{20}}{7}.$ 
Theo đề bài, thời gian người thứ hai làm được nửa công việc lâu hơn người thứ nhất làm được nửa công việc là 3 giờ. Do đó: y2−x2=3$\frac{y}{2} - \frac{x}{2} = 3$   (1)
Trong 1 giờ:
+ Người thứ nhất làm được 1x$\frac{1}{x}$ công việc.
+ Người thứ hai làm được 1y$\frac{1}{y}$ công việc.
+ Cả hai người làm được 1:207=720$1:\frac{{20}}{7} = \frac{7}{{20}}$ công việc.
Suy ra 1x+1y=720$\frac{1}{x} + \frac{1}{y} = \frac{7}{{20}}$   (2)
Từ (1) và (2) ta có hệ phương trình: \(\left\{ \begin{array}{l}\frac{1}{x} + \frac{1}{y} = \frac{7}{{20}}\\\frac{y}{2} - \frac{x}{2} = 3\end{array} \right. \Leftrightarrow \left\{ \begin{array}{l}\frac{1}{x} + \frac{1}{y} = \frac{7}{{20}}\\y - x = 6\end{array} \right. \Leftrightarrow \left\{ \begin{array}{l}\frac{1}{x} + \frac{1}{{x + 6}} = \frac{7}{{20}}{\rm{ }}\left( 4 \right)\\y = x + 6{\rm{     }}\left( 3 \right)\end{array} \right.\) 
Xét phương trình: (4) ⇔7x2+2x−120=0⇔$ \Leftrightarrow 7{x^2} + 2x - 120 = 0 \Leftrightarrow $ 
Vậy thời gian một mình làm xong cả công việc của người thứ nhất là 4 giờ, của người thứ hai là 10 giờ.

========================================================================

https://khoahoc.vietjack.com/thi-online/trac-nghiem-chuyen-de-toan-9-chuyen-de-5-cac-bai-toan-thuc-te-giai-bang-cach-lap-phuong-trinh-va-he/103967


\textbf{{QUESTION}}

Cho hai vòi nước cùng lúc chảy vào một bể cạn. Nếu chảy riêng từng vòi thì vòi thứ nhất chảy đầy bể nhanh hơn vòi thứ hai 4 giờ. Khi nước đầy bể, người ta khóa vòi thứ nhất và vòi thứ hai lại, đồng thời mở vòi thứ ba cho nước chảy ra thì sau 6 giờ bể cạn nước. Khi nước trong bể đã cạn, mở cả ba vòi thì sau 24 giờ bể lại đầy nước. Hỏi nếu chỉ dùng vòi thứ nhất thì sau bao lâu bể đầy nước.

\textbf{{ANSWER}}

Gọi ẩn là thời gian vòi thứ nhất chảy một mình đầy bể.
“Khi nước trong bể đã cạn, mở cả ba vòi” thì lúc này có vòi thứ nhất và vòi thứ hai chảy vào, còn vòi thứ ba chảy ra. Sau 24 giờ thì đầy bể.
Ta xác định được thời gian vòi thứ ba chảy một mình cạn bể là 6 giờ.
 
Thời gian hoàn thành công việc (giờ)
Năng suất làm trong 1 giờ
Vòi 1
x$x$
1x$\frac{1}{x}$
Vòi 2
x+4$x + 4$
1x+4$\frac{1}{{x + 4}}$
Vòi 3
6
16$\frac{1}{6}$
Cả ba vòi (vòi 1, 2 chảy vào, vòi 3 chảy ra)
24
124$\frac{1}{{24}}$
Gọi thời gian mà vòi thứ nhất chảy riêng đầy bể là x (giờ). Điều kiện: x>0.$x > 0.$ 
Khi đó thời gian vòi thứ hai chảy riêng đầy bể là x+4$x + 4$ (giờ).
Trong 1 giờ:
+ Vòi thứ nhất chảy được 1x$\frac{1}{x}$ bể.
+ Vòi thứ hai chảy được 1x+4$\frac{1}{{x + 4}}$ bể.
+ Vòi thứ ba chảy được 16$\frac{1}{6}$ bể (vì vòi thứ ba chảy riêng 6 giờ cạn bể).
+ Cả ba vòi cùng chảy được 124$\frac{1}{{24}}$ bể.
Vì cả ba vòi cùng chảy thì sau 24 giờ đầy bể nên ta có phương trình:
1x+1x+4−16=124⇔1x+1x+4−=524⇔5x2−28x−96=0⇔$\frac{1}{x} + \frac{1}{{x + 4}} - \frac{1}{6} = \frac{1}{{24}} \Leftrightarrow \frac{1}{x} + \frac{1}{{x + 4}} - = \frac{5}{{24}} \Leftrightarrow 5{x^2} - 28x - 96 = 0 \Leftrightarrow $  
Vậy chỉ dùng vòi thứ nhất thì sau 8 giờ bể đầy nước.

========================================================================

https://khoahoc.vietjack.com/thi-online/13-bai-tap-tim-so-chua-biet-trong-mot-ti-le-thuc-co-loi-giai


\textbf{{QUESTION}}

Tìm x trong các tỉ lệ thức sau: 
a) $$ \frac{x}{27}=\frac{-2}{\mathrm{3,6}}$$ ;

\textbf{{ANSWER}}

a) $$ \frac{x}{27}=\frac{-2}{\mathrm{3,6}}$$
$$ x=\frac{(-2).27}{\mathrm{3,6}}$$

x = −15.
Vậy x = −15.

========================================================================

https://khoahoc.vietjack.com/thi-online/bai-tap-bai-tap-on-tap-cuoi-nam-co-dap-an


\textbf{{QUESTION}}

Tính giá trị của các biểu thức sau:
a) $$ \sqrt{25}$$ + (22 . 3)2 . $$ {\left(-\frac{1}{4}\right)}^{2}$$ + 20200 + $$ \left|-\frac{1}{4}\right|$$;

\textbf{{ANSWER}}

= 5 + $$ {\left({2}^{2}.3.\frac{-1}{4}\right)}^{2}$$ + 1 + $$ \frac{1}{4}$$
= 5 + $$ {\left(\mathrm{4.3.}\frac{-1}{4}\right)}^{2}$$ + 1 + $$ \frac{1}{4}$$
= 5 + (-3)2 + 1 + $$ \frac{1}{4}$$
= 5 + 9 + 1 + $$ \frac{1}{4}$$
= 15 + $$ \frac{1}{4}$$
= $$ \frac{60}{4}+\frac{1}{4}$$
= $$ \frac{61}{4}$$.

========================================================================

https://khoahoc.vietjack.com/thi-online/bai-tap-bai-tap-on-tap-cuoi-nam-co-dap-an


\textbf{{QUESTION}}

b) 32−0,25.(7,5−5,1)−6,2+2.(0,5+1,6)$$ \frac{{3}^{2}-0,25.\left(7,5-5,1\right)}{-6,2+2.\left(0,5+1,6\right)}$$
32−0,25.(7,5−5,1)−6,2+2.(0,5+1,6)
32−0,25.(7,5−5,1)−6,2+2.(0,5+1,6)
32−0,25.(7,5−5,1)−6,2+2.(0,5+1,6)
32−0,25.(7,5−5,1)
32−0,25.(7,5−5,1)
32
3
2
−
0
,
25.
(7,5−5,1)
(
7,5−5,1
7
,
5
−
5
,
1
)
−6,2+2.(0,5+1,6)
−6,2+2.(0,5+1,6)


−6,2+2.(0,5+1,6)
−6,2+2.(0,5+1,6)
−6,2+2.(0,5+1,6)
−
6
,
2
+
2.
(0,5+1,6)
(
0,5+1,6
0
,
5
+
1
,
6
)

\textbf{{ANSWER}}

b) 32−0,25.(7,5−5,1)−6,2+2.(0,5+1,6)$$ \frac{{3}^{2}-0,25.\left(7,5-5,1\right)}{-6,2+2.\left(0,5+1,6\right)}$$
= 9−0,25.2,4−6,2+2.2,1$$ \frac{9-0,25.2,4}{-6,2+2.2,1}$$
= 9−0,6−6,2+4,2$$ \frac{9-0,6}{-6,2+4,2}$$
= 8,4−2$$ \frac{8,4}{-2}$$
= -4,2.

========================================================================

https://khoahoc.vietjack.com/thi-online/bai-tap-bai-tap-on-tap-cuoi-nam-co-dap-an


\textbf{{QUESTION}}

Tính một cách hợp lí.
a) 511−1019+1,5+1711−919$$ \frac{5}{11}-\frac{10}{19}+1,5+\frac{17}{11}-\frac{9}{19}$$;

\textbf{{ANSWER}}

a) 511−1019+1,5+1711−919$$ \frac{5}{11}-\frac{10}{19}+1,5+\frac{17}{11}-\frac{9}{19}$$
= (511+1711)+1,5−(919+1019)$$ \left(\frac{5}{11}+\frac{17}{11}\right)+1,5-\left(\frac{9}{19}+\frac{10}{19}\right)$$
=2211+1,5−1919$$ \frac{22}{11}+1,5-\frac{19}{19}$$
= 2 + 1,5 - 1
= 2,5

========================================================================

https://khoahoc.vietjack.com/thi-online/bai-tap-bai-tap-on-tap-cuoi-nam-co-dap-an


\textbf{{QUESTION}}

b) 235.(−23)−213.(−23)+(23)2$$ 2\frac{3}{5}.\left(-\frac{2}{3}\right)-2\frac{1}{3}.\left(-\frac{2}{3}\right)+{\left(\frac{2}{3}\right)}^{2}$$

\textbf{{ANSWER}}

b) 235.(−23)−213.(−23)+(23)2$$ 2\frac{3}{5}.\left(-\frac{2}{3}\right)-2\frac{1}{3}.\left(-\frac{2}{3}\right)+{\left(\frac{2}{3}\right)}^{2}$$
= 235.(−23)−213.(−23)+(−23)2$$ 2\frac{3}{5}.\left(-\frac{2}{3}\right)-2\frac{1}{3}.\left(-\frac{2}{3}\right)+{\left(-\frac{2}{3}\right)}^{2}$$
= −23.(235−213−23)$$ \frac{-2}{3}.\left(2\frac{3}{5}-2\frac{1}{3}-\frac{2}{3}\right)$$
=−23.(2+35−2−13−23)$$ \frac{-2}{3}.\left(2+\frac{3}{5}-2-\frac{1}{3}-\frac{2}{3}\right)$$
=−23.(2−2+35−13−23)$$ \frac{-2}{3}.\left(2-2+\frac{3}{5}-\frac{1}{3}-\frac{2}{3}\right)$$
=−23.(35−1)$$ \frac{-2}{3}.\left(\frac{3}{5}-1\right)$$
= −23.−25$$ \frac{-2}{3}.\frac{-2}{5}$$
=415$$ \frac{4}{15}$$

========================================================================

https://khoahoc.vietjack.com/thi-online/bai-tap-bai-tap-on-tap-cuoi-nam-co-dap-an


\textbf{{QUESTION}}

a) Tìm x, biết 25$$ \frac{2}{5}$$x + 32$$ \frac{3}{2}$$ = 35−(−14)$$ \frac{3}{5}-\left(-\frac{1}{4}\right)$$.

\textbf{{ANSWER}}

a)  $$ \frac{2}{5}$$x + $$ \frac{3}{2}$$ = $$ \frac{3}{5}-\left(-\frac{1}{4}\right)$$.
 $$ \frac{2}{5}x+\frac{3}{2}=\frac{3}{5}+\frac{1}{4}$$
 $$ \frac{2}{5}x+\frac{3}{2}=\frac{12}{20}+\frac{5}{20}$$
 $$ \frac{2}{5}x+\quad \frac{3}{2}=\frac{17}{20}$$
 $$ \frac{2}{5}x=\quad \frac{17}{20}-\frac{3}{2}$$
$$ \frac{2}{5}x=\quad \frac{17}{20}-\frac{30}{20}$$
$$ \frac{2}{5}x=\quad \frac{-13}{20}$$
x= $$ \frac{-13}{20}:\frac{2}{5}$$
$$ x=\quad \frac{-13}{20}.\frac{5}{2}$$
$$ x=\quad \frac{-13}{8}$$
Vậy: $$ x=\quad \frac{-13}{8}$$

========================================================================

https://khoahoc.vietjack.com/thi-online/bo-de-thi-toan-thpt-quoc-gia-nam-2022-co-loi-giai-30-de/67744


\textbf{{QUESTION}}

Có bao nhiêu cách chọn 3 học sinh từ một nhóm gồm 8 học sinh?
A. $$ {A}_{8}^{3}$$
B. $$ {3}^{8}$$
C. $$ {8}^{3}$$
D. $$ {C}_{8}^{3}$$

\textbf{{ANSWER}}

Chọn D
Số cách chọn 3 học sinh từ một nhóm gồm 8 học sinh là tổ hợp chập 3 của 8 phần tử. Vậy có $$ {C}_{8}^{3}$$ cách chọn

========================================================================

https://khoahoc.vietjack.com/thi-online/19-de-on-thi-vao-10-chuyen-hay-co-loi-giai/58638


\textbf{{QUESTION}}

Tính giá trị của biểu thức:
$$ A=\sqrt{16}-\sqrt{9}\phantom{\rule{0ex}{0ex}}B=\frac{1}{2-\sqrt{3}}+\frac{1}{2+\sqrt{3}}$$

\textbf{{ANSWER}}

$$ A=\sqrt{16}-\sqrt{9}=4-3=1\phantom{\rule{0ex}{0ex}}B=\frac{1}{2-\sqrt{3}}+\frac{1}{2+\sqrt{3}}=2+\sqrt{3}+2-\sqrt{3}=4$$

========================================================================

https://khoahoc.vietjack.com/thi-online/19-de-on-thi-vao-10-chuyen-hay-co-loi-giai/58638


\textbf{{QUESTION}}

Cho biểu thức V=(1√x+2+1√x−2)√x+2√x$$ V=\left(\frac{1}{\sqrt{x}+2}+\frac{1}{\sqrt{x}-2}\right)\frac{\sqrt{x}+2}{\sqrt{x}}$$ với x>0,x≠0$$ x>0,x\ne 0$$.
a) Rút gọn biểu thức V.
b) Tìm giá trị của x để V= 1/3.

\textbf{{ANSWER}}

a, $$ V=\left(\frac{1}{\sqrt{x}+2}+\frac{1}{\sqrt{x}-2}\right)\frac{\sqrt{x}+2}{\sqrt{x}}=\frac{\sqrt{x}-2+\sqrt{x}+2}{\left(\sqrt{x}+2\right)\left(\sqrt{x}-2\right)}\frac{\sqrt{x}+2}{\sqrt{x}}=\frac{2}{\sqrt{x}-2}$$
b, $$ V=\frac{1}{3}\Leftrightarrow \frac{2}{\sqrt{x}-2}=\frac{1}{3}\Leftrightarrow \sqrt{x}-2=6\Leftrightarrow x=64\quad (t/m)$$

========================================================================

https://khoahoc.vietjack.com/thi-online/bo-5-de-thi-cuoi-ki-2-toan-10-chan-troi-sang-tao-cau-truc-moi-co-dap-an/164386


\textbf{{QUESTION}}

PHẦN I. TRẮC NGHIỆM KHÁCH QUAN
A. TRẮC NGHIỆM NHIỀU PHƯƠNG ÁN LỰA CHỌN. Thí sinh trả lời từ câu 1 đến câu 12.
Cho tam thức bậc hai $f\left( x \right) = - 2{x^2} + 8x - 8$. Trong các mệnh đề sau, mệnh đề nào đúng?
A. $f\left( x \right) < 0,\forall x \in \mathbb{R}$.
B. $f\left( x \right) \ge 0,\forall x \in \mathbb{R}$.
C. $f\left( x \right) \le 0,\forall x \in \mathbb{R}$.
D. $f\left( x \right) > 0,\forall x \in \mathbb{R}$.

\textbf{{ANSWER}}

Đáp án đúng là: C
Ta có $\Delta = {8^2} - 4.\left( { - 2} \right).\left( { - 8} \right) = 0$ mà $a = - 2 < 0$ nên $f\left( x \right) \le 0,\forall x \in \mathbb{R}$.

========================================================================

https://khoahoc.vietjack.com/thi-online/bo-5-de-thi-cuoi-ki-2-toan-10-chan-troi-sang-tao-cau-truc-moi-co-dap-an/164386


\textbf{{QUESTION}}

Cho tam thức bậc hai y=f(x)=ax2+bx+c$y = f\left( x \right) = a{x^2} + bx + c$ có Δ<0$\Delta < 0$. Giá trị của a$a$ để biểu thức luôn dương là
A. a=1$a = 1$.
B. a=−1$a = - 1$.
C. a=−10$a = - 10$.
D. a=−2$a = - 2$.

\textbf{{ANSWER}}

Đáp án đúng là: A
Ta có $f\left( x \right) > 0,\forall x \in \mathbb{R} \Leftrightarrow \left\{ \begin{array}{l}a > 0\\\Delta < 0\end{array} \right.$. Do đó chọn A.

========================================================================

https://khoahoc.vietjack.com/thi-online/bo-5-de-thi-cuoi-ki-2-toan-10-chan-troi-sang-tao-cau-truc-moi-co-dap-an/164386


\textbf{{QUESTION}}

Cho các bất phương trình 4x2−3x+9<0;x2−5x<0;4x−3>0;4x2+1>x3$4{x^2} - 3x + 9 < 0;{x^2} - 5x < 0;4x - 3 > 0;4{x^2} + 1 > {x^3}$. Số lượng bất phương trình bậc hai một ẩn là</>
A. 3.
B. 2.
C. 1.
D. 4.

\textbf{{ANSWER}}

Đáp án đúng là: B
Có 2 bất phương trình là bất phương trình bậc hai một ẩn trong các bất phương trình trên là 4x2−3x+9<0;x2−5x<0$4{x^2} - 3x + 9 < 0;{x^2} - 5x < 0$.

========================================================================

https://khoahoc.vietjack.com/thi-online/bo-5-de-thi-cuoi-ki-2-toan-10-chan-troi-sang-tao-cau-truc-moi-co-dap-an/164386


\textbf{{QUESTION}}

Phương trình √x2+2x+2=2x+3$\sqrt {{x^2} + 2x + 2} = 2x + 3$ có nghiệm là giá trị nào sau đây?
A. x=2$x = 2$.
B. x=1$x = 1$.
C. x=−1$x = - 1$.
D. x=−2$x = - 2$.

\textbf{{ANSWER}}

Đáp án đúng là: C
Thay x=−1$x = - 1$ vào phương trình √x2+2x+2=2x+3$\sqrt {{x^2} + 2x + 2} = 2x + 3$ ta thấy thỏa mãn.
Do đó phương trình √x2+2x+2=2x+3$\sqrt {{x^2} + 2x + 2} = 2x + 3$ có nghiệm x=−1$x = - 1$.

========================================================================

https://khoahoc.vietjack.com/thi-online/bo-5-de-thi-cuoi-ki-2-toan-10-chan-troi-sang-tao-cau-truc-moi-co-dap-an/164386


\textbf{{QUESTION}}

Một công việc có 2 công đoạn thực hiện liên tiếp nhau. Công đoạn 1 có a$a$ cách thực hiện. Công đoạn 2 có b cách thực hiện. Số cách thực hiện công việc trên là
A. ab(a+b)$ab\left( {a + b} \right)$.
B. a+b$a + b$.
C. a.b$a.b$.
D. ab${a^b}$.

\textbf{{ANSWER}}

Đáp án đúng là: A
Số cách thực hiện công việc trên là a.b$a.b$.

========================================================================

https://khoahoc.vietjack.com/thi-online/bai-tap-dau-hieu-chia-het-cho-3-cho-9-chon-loc-co-dap-an


\textbf{{QUESTION}}

Trong các số 333; 354; 360; 2457; 1617; 152, số nào chia hết cho 9
A. 333     
B. 360     
C. 2457    
D. Cả A, B, C đúng

\textbf{{ANSWER}}

Đáp án là D
 + Số 333 có tổng các chữ số là 3 + 3 + 3 = 9 ⋮ 9 nên 333 chia hết cho 9.
     + Số 360 có tổng các chữ số là 3 + 6 + 0 = 9 ⋮ 9 nên 360 chia hết cho 9.
     + Số 2475 có tổng các chữ số là 2 + 4 + 7 + 5 = 18 ⋮ 9 nên 2475 chia hết cho 9.

========================================================================

https://khoahoc.vietjack.com/thi-online/bai-tap-dau-hieu-chia-het-cho-3-cho-9-chon-loc-co-dap-an


\textbf{{QUESTION}}

Cho 5 số 0;1;3;6;7. Có bao nhiêu số tự nhiên có ba chữ số khác nhau và chia hết cho 3 được lập từ các số trên.
A. 1
B. 4
C. 3
D. 2

\textbf{{ANSWER}}

Đáp án là B
Trong năm số trên, tổng ba số chia hết cho 9 là: 6 + 3 + 0 = 9
Các số tự nhiên có ba chữ số khác nhau và chia hết cho 3 được lập từ các số trên là: 360; 306; 630; 603

========================================================================

https://khoahoc.vietjack.com/thi-online/bai-tap-dau-hieu-chia-het-cho-3-cho-9-chon-loc-co-dap-an


\textbf{{QUESTION}}

Cho A = a785b−−−−−−−−−−−−
Tìm tổng các chữ số a và b sao cho A chia cho 9 dư 2.
A. (a + b) ∈ {9; 18}     
B. (a + b) ∈ {0; 9; 18}
C. (a + b) ∈ {1; 2; 3}     
D. (a + b) ∈ {4; 5; 6}

\textbf{{ANSWER}}

Đáp án là A
Ta có a, b ∈ {0; 1; 2; 3; 4; 5; 6; 7; 8; 9} và a ≠ 0 nên 0 < a + b ≤ 18
A chia cho 9 dư 2 ⇒ a + 7 + 8 + 5 + b = a + b + 20 chia cho 9 dư 2 hay (a + b + 18) ⋮ 9
Mà 18 ⋮ 9 ⇒ (a + b) ⋮ 9 ⇒ (a + b) ∈ {9; 18}

========================================================================

https://khoahoc.vietjack.com/thi-online/bai-tap-dau-hieu-chia-het-cho-3-cho-9-chon-loc-co-dap-an


\textbf{{QUESTION}}

Tìm các số tự nhiên x, y biết rằng 23x5y chia hết cho 2, 5 và 9
A. x = 0; y = 6     
B. x = 6; y = 0
C. x = 8; y = 0    
D. x = 0; y = 8

\textbf{{ANSWER}}

Đáp án là C
Theo giả thiết ta có 23x5y chia hết cho 2 và 5 nên y = 0, ta được số 23x50
Mà 23x50 ⋮ 9 nên 2 + 3 + x + 5 chia hết cho 9 hay (10 + x) ⋮ 9
Ta có x = 8 thỏa mãn yêu cầu bài.

========================================================================

https://khoahoc.vietjack.com/thi-online/bai-tap-dau-hieu-chia-het-cho-3-cho-9-chon-loc-co-dap-an


\textbf{{QUESTION}}

Chọn câu trả lời đúng. Trong các số 2055; 6430; 5041; 2341; 2305
A. Các số chia hết cho 5 là 2055; 6430; 2341
B. Các số chia hết cho 3 là 2055 và 6430.
C. Các số chia hết cho 5 là 2055; 6430; 2305.
D. Không có số nào chia hết cho 3.

\textbf{{ANSWER}}

Đáp án là C
Các số chia hết cho 5 là 2055; 6430; 2305.

========================================================================

https://khoahoc.vietjack.com/thi-online/15-cau-trac-nghiem-toan-7-chan-troi-sang-tao-bai-2-lam-quen-voi-xac-suat-cua-bien-co-ngau-nhien-co-d


\textbf{{QUESTION}}

Biến cố có khả năng xảy ra cao hơn sẽ có xác suất:

\textbf{{ANSWER}}

Hướng dẫn giải
Đáp án đúng là: A
Biến cố có khả năng xảy ra cao hơn sẽ có xác suất lớn hơn.

========================================================================

https://khoahoc.vietjack.com/thi-online/15-cau-trac-nghiem-toan-7-chan-troi-sang-tao-bai-2-lam-quen-voi-xac-suat-cua-bien-co-ngau-nhien-co-d


\textbf{{QUESTION}}

Biến cố không thể có xác suất bằng bao nhiêu?

\textbf{{ANSWER}}

Hướng dẫn giải
Đáp án đúng là: B
Biến cố không thể có xác suất bằng 0.

========================================================================

https://khoahoc.vietjack.com/thi-online/15-cau-trac-nghiem-toan-7-chan-troi-sang-tao-bai-2-lam-quen-voi-xac-suat-cua-bien-co-ngau-nhien-co-d


\textbf{{QUESTION}}

Gieo đồng thời hai con xúc xắc. Tìm xác suất của các biến cố sau: “Tổng số chấm xuất hiện trên hai con xúc xắc bằng 1”?
A. 1;
B. 0;
C. 2;

\textbf{{ANSWER}}

Hướng dẫn giải
Đáp án đúng là: B
Số chấm xuất hiện trên mỗi mặt của con xúc xắc ít nhất là 1.
Suy ra tổng số chấm xuất hiện trên mỗi mặt của hai mặt của con xúc xắc ít nhất là 2.
Do đó biến cố “Tổng số chấm xuất hiện trên hai con xúc xắc bằng 1” là biến cố không thể Vậy biến cố “Tổng số chấm xuất hiện trên hai con xúc xắc bằng 1” có xác suất bằng 0.

========================================================================

https://khoahoc.vietjack.com/thi-online/15-cau-trac-nghiem-toan-7-chan-troi-sang-tao-bai-2-lam-quen-voi-xac-suat-cua-bien-co-ngau-nhien-co-d


\textbf{{QUESTION}}

Gieo một con xúc xắc được chế tạo cân đối. Tính xác suất của biến cố sau: A: “Số chấm xuất hiện trên mặt con xúc xắc là 6”
A. 1
B. 0

\textbf{{ANSWER}}

Hướng dẫn giải
Đáp án đúng là: D
Khi gieo một con xúc xắc cân đối thì 6 mặt của nó có khả năng xuất hiện bằng nhau nên xác suất xuất hiện của mỗi mặt đều là 16$$ \frac{1}{6}$$.
Do 6 kết quả đều có khả năng xảy ra bằng nhau nên P(A) = 16$$ \frac{1}{6}$$.

========================================================================

https://khoahoc.vietjack.com/thi-online/15-cau-trac-nghiem-toan-7-chan-troi-sang-tao-bai-2-lam-quen-voi-xac-suat-cua-bien-co-ngau-nhien-co-d


\textbf{{QUESTION}}

An và Bình mỗi người gieo một con xúc xắc. Tìm xác suất của các biến cố sau: “Tổng số chấm xuất hiện trên hai con xúc xắc lớn hơn 1”.
A. 1
B. 0

\textbf{{ANSWER}}

Hướng dẫn giải
Đáp án đúng là: A
Số chấm xuất hiện trên mỗi mặt của con xúc xắc ít nhất là 1.
Suy ra tổng số chấm xuất hiện trên mỗi mặt của hai mặt của con xúc xắc luôn lớn hơn 1.
Do đó biến cố “Tổng số chấm xuất hiện trên hai con xúc xắc lớn hơn 1” là biến cố chắc chắn.
Vậy xác suất của biến cố này bằng 1.

========================================================================

https://khoahoc.vietjack.com/thi-online/on-thi-tot-nghiep-thpt-mon-toan-de-3


\textbf{{QUESTION}}

Một bình đựng 5 viên bi xanh và 3 viên bi đỏ (các viên bi cùng màu là khác nhau). Lấy ngẫu nhiên một viên bi, rồi lấy ngẫu nhiên một viên bi nữa. Khi tính xác suất của biến cố “Lấy lần thứ hai được một viên bi xanh”, ta được kết quả
A. $$ \frac{5}{7}$$
B. $$ \frac{5}{9}$$
C. $$ \frac{5}{8}$$
D. $$ \frac{4}{7}$$

\textbf{{ANSWER}}

Đáp án đúng là: C
Lấy ngẫu nhiên một viên bi, rồi lấy ngẫu nhiên một viên bi nữa từ một bình đựng 5 bi xanh và 3 bi đỏ $$ \Rightarrow {n}_{\Omega }={C}_{8}^{1}.{C}_{7}^{1}=56$$.
Biến cố A: “Lấy lần thứ hai được một viên bi xanh”.
Trường hợp 1: lần 1 lấy được bi đỏ và lần hai lấy được bi xanh: $$ {C}_{3}^{1}.{C}_{5}^{1}=15$$ cách.
Trường hợp 2: cả lần 1 và lần 2 đều lấy được bi xanh: $$ {C}_{5}^{1}.{C}_{4}^{1}=20$$ cách.
$$ \Rightarrow n\left(A\right)=15+20=35$$.
$$ \Rightarrow P\left(A\right)=\frac{35}{56}=\frac{5}{8}$$.

========================================================================

https://khoahoc.vietjack.com/thi-online/30-cau-trac-nghiem-toan-10-chan-troi-sang-tao-bai-tap-cuoi-chuong-8-phan-2-co-dap-an/110037


\textbf{{QUESTION}}

Từ các chữ số 1; 5; 6; 7; 9 có thể lập được bao nhiêu số tự nhiên có 4 chữ số? 
A. 3 125;               
B. 120;                   
C. 20;

\textbf{{ANSWER}}

Hướng dẫn giải
Đáp án đúng là: D
Gọi $$ \overline{abcd}$$ là số cần tìm.
Việc lập một số tự nhiên có 4 chữ số gồm 4 công đoạn:
Công đoạn 1: Chọn số ở vị trí a, có 5 cách chọn một chữ số từ các chữ số 1; 5; 6; 7; 9.
Công đoạn 2: Chọn số ở vị trí b, có 5 cách chọn một số từ các chữ số 1; 5; 6; 7; 9.
Công đoạn 3: Chọn số ở vị trí c, có 5 cách chọn một số từ các chữ số 1; 5; 6; 7; 9.
Công đoạn 4: Chọn số ở vị trí d, có 5 cách chọn một số từ các chữ số 1; 5; 6; 7; 9.
Theo quy tắc nhân, ta có tất cả 5.5.5.5= 54 = 625 cách lập một số tự nhiên có 4 chữ số.
Vậy ta chọn phương án D.

========================================================================

https://khoahoc.vietjack.com/thi-online/30-cau-trac-nghiem-toan-10-chan-troi-sang-tao-bai-tap-cuoi-chuong-8-phan-2-co-dap-an/110037


\textbf{{QUESTION}}

Trên bàn có 8 cây bút chì khác nhau, 6 cây bút bi khác nhau và 10 cuốn tập khác nhau. Một học sinh muốn chọn một đồ vật duy nhất (một cây bút chì hoặc một cây bút bi hoặc một cuốn tập) thì số cách chọn khác nhau là:
A. 24;          
B. 480;                   
C. 48;

\textbf{{ANSWER}}

Hướng dẫn giải
Đáp án đúng là: A
Việc chọn một đồ vật duy nhất có ba phương án thực hiện:
Phương án 1: Chọn một cây bút chì, có 8 cách chọn.
Phương án 2: Chọn một cây bút bi, có 6 cách chọn.
Phương án 3: Chọn một cuốn tập, có 10 cách chọn.
Theo quy tắc cộng, ta có tất cả 8 + 6 + 10 = 24 cách chọn một đồ vật duy nhất.
Vậy ta chọn phương án A.

========================================================================

https://khoahoc.vietjack.com/thi-online/30-cau-trac-nghiem-toan-10-chan-troi-sang-tao-bai-tap-cuoi-chuong-8-phan-2-co-dap-an/110037


\textbf{{QUESTION}}

Hội đồng quản trị của công ty X gồm 10 người. Hỏi có bao nhiêu cách bầu ra ba người vào ba vị trí chủ tịch, phó chủ tịch và thư kí, biết khả năng mỗi người là như nhau. 
A. 1 000;               
B. 720;                   
C. 30;

\textbf{{ANSWER}}

Hướng dẫn giải
Đáp án đúng là: B
Cách 1:
Việc bầu ra ba người vào ba vị trí chủ tịch, phó chủ tịch và thư kí gồm ba công đoạn:
Công đoạn 1: Bầu một người làm chủ tịch, có 10 cách chọn.
Công đoạn 2: Ứng với mỗi cách bầu một người làm chủ tịch, có 9 cách bầu một người làm phó chủ tịch.
Công đoạn 3: Ứng với mỗi cách bầu một người làm chủ tịch và bầu một người làm phó chủ tịch, có 8 cách bầu một người làm thư kí.
Theo quy tắc nhân, ta có tất cả 10.9.8 = 720 cách bầu ra ba người vào ba vị trí chủ tịch, phó chủ tịch và thư kí.
Vậy ta chọn phương án B.
Cách 2:
Để bầu ra ba người vào ba vị trí chủ tịch, phó chủ tịch và thư kí trong 10 người là một chỉnh hợp chập 3 của 10, tức là có A310=720$$ {A}_{10}^{3}=720$$ cách bầu.
Vậy ta chọn phương án B.

========================================================================

https://khoahoc.vietjack.com/thi-online/30-cau-trac-nghiem-toan-10-chan-troi-sang-tao-bai-tap-cuoi-chuong-8-phan-2-co-dap-an/110037


\textbf{{QUESTION}}

Từ các chữ số 0; 1; 2; 3; 4; 5, có thể lập được bao nhiêu số lẻ gồm 4 chữ số khác nhau? 
A. 154;                  
B. 145;                   
C. 144;

\textbf{{ANSWER}}

Hướng dẫn giải
Đáp án đúng là: C
Gọi ¯abcd$$ \overline{abcd}$$ là số cần tìm.
Vì số được lập là số lẻ nên vị trí d có 3 cách chọn một trong các số 1; 3; 5.
Ứng với mỗi cách chọn đó, có 4 cách chọn số ở vị trí a từ 4 chữ số khác 0 và khác số ở vị trí d đã cho.
Ứng với mỗi cách chọn đó, có 4 cách chọn số ở vị trí b từ 4 chữ số còn lại.
Ứng với mỗi cách chọn đó, có 3 cách chọn số ở vị trí c từ 3 chữ số còn lại.
Theo quy tắc nhân, ta có tất cả 3.4.4.3 = 144 cách lập một số thỏa mãn yêu cầu bài toán.
Vậy ta chọn phương án C.

========================================================================

https://khoahoc.vietjack.com/thi-online/bo-de-thi-toan-thpt-quoc-gia-nam-2022-co-loi-giai-30-de/67788


\textbf{{QUESTION}}

Có bao nhiêu cách xếp 4 học sinh thành một hàng dọc?
A. 4
B. $$ {C}_{4}^{4}$$
C. $$ 4!$$
D. $$ {A}_{4}^{1}$$

\textbf{{ANSWER}}

Chọn C
Mỗi cách xếp 4 học sinh thành một hàng dọc là một hoán vị của 4 phần tử.
Vậy số cách xếp 4 học sinh thành một hàng dọc là: 4! (cách).

========================================================================

https://khoahoc.vietjack.com/thi-online/12-bai-tap-xac-dinh-gia-tri-tham-so-de-duong-thang-di-qua-hai-diem-cho-truoc-co-loi-giai


\textbf{{QUESTION}}

Giá trị của hệ số a, b để đường thẳng y = ax + b đi qua hai điểm A(1; 2) và B(3; 8) là:
A. a = 3, b = −1.
B. a = −3, b = 1.
C. a = 1, b = 3.
D. a = −1, b = 3.

\textbf{{ANSWER}}

Đáp án đúng là: A
Đường thẳng đi qua điểm A(1; 2) nên ta có: a + b = 2 (1).
Đường thẳng đi qua điểm B(3; 8) nên ta có: 3a + b = 8 (2).
Từ (1) và (2) ta có hệ phương trình $\left\{ \begin{array}{l}a + b = 2\\3a + b = 8\end{array} \right.$.
Thực hiện trừ theo vế hai phương trình của hệ, ta được: 2a = 6 hay a = 3.
Với a = 3 thì b = 2 – 3 = −1.
Vậy a = 3, b = −1.

========================================================================

https://khoahoc.vietjack.com/thi-online/12-bai-tap-xac-dinh-gia-tri-tham-so-de-duong-thang-di-qua-hai-diem-cho-truoc-co-loi-giai


\textbf{{QUESTION}}

Hệ số a, b để đường thẳng y = ax + b đi qua hai điểm M(1; 3) và N(−2; 2) là 
A. a = 1, b = 8.
B. a=13,b=83.$a = \frac{1}{3},b = \frac{8}{3}.$
C. a=83,b=13.$a = \frac{8}{3},b = \frac{1}{3}.$
D. a=3,b=38.$a = 3,b = \frac{3}{8}.$

\textbf{{ANSWER}}

Đáp án đúng là: B
Phương trình đường thẳng đi qua điểm M(1; 3) nên ta có a + b = 3 (1).
Phương trình đường thẳng đi qua điểm N(−2; 2) nên ta có −2a + b = 2 (2).
Từ (1) và (2) ta có hệ phương trình $\left\{ \begin{array}{l}a + b = 3\\ - 2a + b = 2\end{array} \right.$
Thực hiện trừ theo vế, ta có 3a = 1 hay a = $\frac{1}{3}$.
Với a = $\frac{1}{3}$ thì b = $\frac{8}{3}$.

========================================================================

https://khoahoc.vietjack.com/thi-online/12-bai-tap-xac-dinh-gia-tri-tham-so-de-duong-thang-di-qua-hai-diem-cho-truoc-co-loi-giai


\textbf{{QUESTION}}

Hệ số a, b để đường thẳng y = ax + b đi qua hai điểm M(−1;√3$\sqrt 3 $) và N(2;√3$\sqrt 3 $) là
A. a = 0, b = √3$\sqrt 3 $.
B. a = √3$\sqrt 3 $, b = 0.
C. a = 0, b = −√3$\sqrt 3 $.
D. a = 1, b = √3$\sqrt 3 $.

\textbf{{ANSWER}}

Đáp án đúng là: A
Đường thẳng y = ax + b đi qua M nên ta có: −a + b = √3$\sqrt 3 $ (1)
Đường thẳng y = ax + b đi qua điểm N nên ta có: 2a + b = √3$\sqrt 3 $ (2).
Từ (1) và (2) ta có hệ phương trình {−a+b=√32a+b=√3$\left\{ \begin{array}{l} - a + b = \sqrt 3 \\2a + b = \sqrt 3 \end{array} \right.$.
Thực hiện trừ theo vế, ta được 3a = 0 hay a = 0.
Với a = 0 thì b = √3$\sqrt 3 $.

========================================================================

https://khoahoc.vietjack.com/thi-online/12-bai-tap-xac-dinh-gia-tri-tham-so-de-duong-thang-di-qua-hai-diem-cho-truoc-co-loi-giai


\textbf{{QUESTION}}

Phương trình đường thẳng đi qua hai điểm A(1; 1) và B(3; −2) là:
A. y = −3x + 5.
B. y = −32x+52.$ - \frac{3}{2}x + \frac{5}{2}.$
C. y = 32x−52.$\frac{3}{2}x - \frac{5}{2}.$
D. y = −32x−52.$ - \frac{3}{2}x - \frac{5}{2}.$

\textbf{{ANSWER}}

Đáp án đúng là: B
Gọi phương trình đường thẳng đó là y = ax + b.
Đường thẳng đi qua điểm A(1; 1) nên ta có: a + b = 1 (1)
Đường thẳng đi qua điểm B(3; −2) nên ta có: 3a + b = −2 (2)
Từ (1) và (2) ta có hệ phương trình {a+b=13a+b=−2$\left\{ \begin{array}{l}a + b = 1\\3a + b =  - 2\end{array} \right.$.
Thực hiện trừ theo vế hai phương trình của hệ, ta được −2a = 3 nên a = −32$ - \frac{3}{2}$.
Với a = −32$ - \frac{3}{2}$ thì b = 52$\frac{5}{2}$.
Vậy phương trình đường thẳng đó là y = −32x+52.$ - \frac{3}{2}x + \frac{5}{2}.$

========================================================================

https://khoahoc.vietjack.com/thi-online/12-bai-tap-xac-dinh-gia-tri-tham-so-de-duong-thang-di-qua-hai-diem-cho-truoc-co-loi-giai


\textbf{{QUESTION}}

Phương trình đường thẳng đi qua A(3; 1) và cắt trục hoành tại điểm có hoành độ bằng 2 là
A. y = 2x – 1.
B. y = x – 2.
C. y = 2x + 1.
D. y = x + 2.

\textbf{{ANSWER}}

Đáp án đúng là: B
Theo đề, phương trình đường thẳng đi qua hai điểm có tọa độ là (3; 1) và (2; 0).
Gọi phương trình đường thẳng đó là: y = ax + b (d).
Vì (d) đi qua điểm A(3; 1) nên ta có: 3a + b = 1 (1).
(d) đi qua điểm có tọa độ (2; 0) nên ta có: 2a + b = 0 (2).
Từ (1) và (2) ta có hệ phương trình {3a+b=12a+b=0$\left\{ \begin{array}{l}3a + b = 1\\2a + b = 0\end{array} \right.$.
Thực hiện trừ theo vế hai phương trình của hệ ta được a = 1.
Với a = 1 thì b = −2.
Vậy phương trình đường thẳng đó là y = x – 2.

========================================================================

https://khoahoc.vietjack.com/thi-online/10-bai-tap-vng-dang-thuc-lap-phuong-cua-mot-tong-de-tinh-nhanh-khai-trien-rut-gon


\textbf{{QUESTION}}

Rút gọn biểu thức (x + y)3 – (x – y)3 ta được
A. 2(3x2 + y2);

\textbf{{ANSWER}}

Đáp án đúng là: D
Ta có (x + y)3 – (x – y)3
= (x3 + 3x2y + 3xy2 + y3) – (x3 – 3x2y + 3xy2 – y3) 
= x3 + 3x2y + 3xy2 + y3 – x3 + 3x2y – 3xy2 + y3
= 6x2y + 2y3 = 2y(3x2 + y2).
Do đó ta chọn đáp án D.

========================================================================

https://khoahoc.vietjack.com/thi-online/10-bai-tap-vng-dang-thuc-lap-phuong-cua-mot-tong-de-tinh-nhanh-khai-trien-rut-gon


\textbf{{QUESTION}}

Viết biểu thức x3 + 9x2 + 27x + 27 dưới dạng lập phương của một tổng ta được biểu thức nào?

\textbf{{ANSWER}}

Đáp án đúng là: B
Ta có x3 + 9x2 + 27x + 27 = x3 + 3 . x2 . 3 + 3 . x . 32 + 33 = (x + 3)3.
Do đó ta chọn đáp án B.

========================================================================

https://khoahoc.vietjack.com/thi-online/10-bai-tap-vng-dang-thuc-lap-phuong-cua-mot-tong-de-tinh-nhanh-khai-trien-rut-gon


\textbf{{QUESTION}}

Khai triển biểu thức (x + 4)3 ta được biểu thức nào dưới đây?

\textbf{{ANSWER}}

Đáp án đúng là: D
Ta có: (x + 4)3 = x + 3 . x2 . 4 + 3 . x . 42 + 43 
= x3 + 12x2 + 48x + 64.
Do đó ta chọn đáp án D.

========================================================================

https://khoahoc.vietjack.com/thi-online/10-bai-tap-vng-dang-thuc-lap-phuong-cua-mot-tong-de-tinh-nhanh-khai-trien-rut-gon


\textbf{{QUESTION}}

Giá trị biểu thức x3 + 9x2 + 27x + 27 với x = 2 là

\textbf{{ANSWER}}

Đáp án đúng là: B
Ta có x3 + 9x2 + 27x + 27 = x3 + 3 . x2 . 3 + 3 . x . 32 + 33 = (x + 3)3.
Thay x = 2 vào biểu thức (x + 3)3 ta được (2 + 3) = 5 = 125.
Do đó ta chọn đáp án B.

========================================================================

https://khoahoc.vietjack.com/thi-online/10-bai-tap-vng-dang-thuc-lap-phuong-cua-mot-tong-de-tinh-nhanh-khai-trien-rut-gon


\textbf{{QUESTION}}

Viết biểu thức y3 + 15y2 + 75y + 125 dưới dạng lập phương của một tổng ta được biểu thức nào dưới đây?

\textbf{{ANSWER}}

Đáp án đúng là: C
Ta có y3 + 15y2 + 75y + 125 = y3 + 3 . y2 . 5 + 3 . y . 52 + 5 = (y + 5)3.
Do đó ta chọn đáp án C.

========================================================================

https://khoahoc.vietjack.com/thi-online/15-cau-trac-nghiem-toan-8-canh-dieu-bai-1-don-thuc-nhieu-bien-da-thuc-nhieu-bien-co-dap-an


\textbf{{QUESTION}}

Trong các biểu thức sau, biểu thức nào là đơn thức?
A. $$ 2+{x}^{2}y$$

\textbf{{ANSWER}}

Đáp án đúng là: B
Theo định nghĩa đơn thức, biểu thức $$ -\frac{1}{5}{x}^{4}{y}^{5}$$ là đơn thức.

========================================================================

https://khoahoc.vietjack.com/thi-online/15-cau-trac-nghiem-toan-8-canh-dieu-bai-1-don-thuc-nhieu-bien-da-thuc-nhieu-bien-co-dap-an


\textbf{{QUESTION}}

Tìm phần biến trong đơn thức 100ab2x2yz$$ 100a{b}^{2}{x}^{2}yz$$ với a, b là hằng số.
A. ab2x2yz$$ a{b}^{2}{x}^{2}yz$$

\textbf{{ANSWER}}

Đáp án đúng là: C
Đơn thức $$ 100a{b}^{2}{x}^{2}yz$$ với a, b là hằng số có phần biến số là: $$ {x}^{2}yz$$.

========================================================================

https://khoahoc.vietjack.com/thi-online/15-cau-trac-nghiem-toan-8-canh-dieu-bai-1-don-thuc-nhieu-bien-da-thuc-nhieu-bien-co-dap-an


\textbf{{QUESTION}}

A. 2
B. 3
C. 4
D. 5

\textbf{{ANSWER}}

Đáp án đúng là: B
Có ba nhóm đơn thức đồng dạng trong các đơn thức đã cho gồm:
Nhóm thứ nhất: −23x3y;2x3y$$ -\frac{2}{3}{x}^{3}y;2{x}^{3}y$$
Nhóm thứ hai: 5x2y;12x2y$$ 5{x}^{2}y;\frac{1}{2}{x}^{2}y$$
Nhóm thứ ba: −xy2;6xy2$$ -x{y}^{2};6x{y}^{2}$$

========================================================================

https://khoahoc.vietjack.com/thi-online/15-cau-trac-nghiem-toan-8-canh-dieu-bai-1-don-thuc-nhieu-bien-da-thuc-nhieu-bien-co-dap-an


\textbf{{QUESTION}}

Các đơn thức −10;13x;2x2y;5x2.x2$$ -10;\frac{1}{3}x;2{x}^{2}y;5{x}^{2}.{x}^{2}$$ có bậc lần lượt là
A. 0; 1; 3; 4

\textbf{{ANSWER}}

Đáp án đúng là: A
Đơn thức – 10 có bậc là 0.
Đơn thức $$ \frac{1}{3}x$$ có bậc là 1.
Đơn thức $$ 2{x}^{2}y$$ có bậc là 2 + 1 = 3
Đơn thức $$ 5{x}^{2}.{x}^{2}=5{x}^{4}$$ có bậc là 4.
Các đơn thức $$ -10;\frac{1}{3}x;2{x}^{2}y;5{x}^{2}.{x}^{2}$$ có bậc lần lượt là: 0; 1; 3; 4.

========================================================================

https://khoahoc.vietjack.com/thi-online/15-cau-trac-nghiem-toan-8-canh-dieu-bai-1-don-thuc-nhieu-bien-da-thuc-nhieu-bien-co-dap-an


\textbf{{QUESTION}}

Tổng các đơn thức 3x2y4$$ 3{x}^{2}{y}^{4}$$ và 7x2y4$$ 7{x}^{2}{y}^{4}$$ là
A. 10x2y4$$ 10{x}^{2}{y}^{4}$$

\textbf{{ANSWER}}

Đáp án đúng là: A
3x2y4+7x2y4=(3+7)x2y4$$ 3{x}^{2}{y}^{4}+7{x}^{2}{y}^{4}=\left(3+7\right){x}^{2}{y}^{4}$$

========================================================================

https://khoahoc.vietjack.com/thi-online/bai-tap-he-phuong-trinh-bac-nhat-ba-an-co-dap-an-a


\textbf{{QUESTION}}

Chúng ta đã biết cách mô tả mối liên hệ giữa hai ẩn số x, y phải thoả mãn đồng thời hai điều kiện a1x + b1y = c1 (a12 + b12 > 0) và a2x + b2y = c2 (a22 + b22 > 0) bằng cách sử dụng hệ phương trình bậc nhất hai ẩn:
Trong bài học này, ta sẽ học cách giải quyết tình huống cần mô tả mối liên hệ giữa ba ẩn số x, y, z phải thoả mãn đồng thời ba điều kiện:
a1x + b1y + c1z = d1; a2x + b2y + c2z = d2 và a3x + b3y + c3z = d3.

\textbf{{ANSWER}}

HS tự làm

========================================================================

https://khoahoc.vietjack.com/thi-online/bai-tap-he-phuong-trinh-bac-nhat-ba-an-co-dap-an-a


\textbf{{QUESTION}}

Ba lớp 10A, 10B, 10C gồm 128 học sinh cùng tham gia lao động trồng cây. Mỗi học sinh lớp 10A trồng được 3 cây bạch đàn và 4 cây bàng. Mỗi học sinh lớp 10B trồng được 2 cây bạch đàn và 5 cây bàng. Mỗi học sinh lớp 10C trồng được 6 cây bạch đàn. Cả 3 lớp trồng được 476 cây bạch đàn và 375 cây bàng. Gọi x, y, z lần lượt là số học sinh của các lớp 10A, 10B, 10C.
a) Lập các hệ thức thể hiện mối liên hệ giữa x, y và z.
b) Trong bảng dữ liệu sau, chọn các số liệu phù hợp với số học sinh của mỗi lớp 10A, 10B, 10C và giải thích sự lựa chọn của bạn.
x
y
z
41
43
44
40
43
45
42
43
43

\textbf{{ANSWER}}

Hướng dẫn giải
a) Các hệ thức thể hiện mối liên hệ giữa x, y và z là:
x + y + z = 128; 3x + 2y + 6z = 476; 4x + 5y = 375.
b) Các số liệu phù hợp với số học sinh của mỗi lớp 10A, 10B, 10C là x = 40, y = 43, z = 45. Vì các số liệu này thoả mãn tất cả các hệ thức thể hiện mỗi liên hệ giữa x, y và z trong câu a); các số liệu còn lại thì không thoả mãn.

========================================================================

https://khoahoc.vietjack.com/thi-online/bai-tap-he-phuong-trinh-bac-nhat-ba-an-co-dap-an-a


\textbf{{QUESTION}}

Hệ phương trình nào dưới đây là hệ phương trình bậc nhất ba ẩn? Mỗi bộ ba số (1; 5; 2), (1; 1; 1) và (–1; 2; 3) có là nghiệm của hệ phương trình bậc nhất ba ẩn đó không?
(1) {4x−2y+z=54xz−5y+2z=−7−x+3y+2z=3.$$ \{\begin{array}{l}4x-2y+z=5\\ 4xz-5y+2z=-7\\ -x+3y+2z=3\end{array}.$$
(2) {x+2z=52x−y+z=−13x−2y=−7.$$ \{\begin{array}{l}x+2z=5\\ 2x-y+z=-1\\ 3x-2y=-7\end{array}.$$

\textbf{{ANSWER}}

Hướng dẫn giải
– Hệ (1) không là hệ phương trình bậc nhất ba ẩn vì phương trình thứ hai của hệ có chứa xz.
–Hệ (2) là hệ phương trình bậc nhất ba ẩn.
+) Bộ ba số (1; 5; 2) có là nghiệm của hệ phương trình bậc nhất đã cho.
Vì khi thay bộ số này vào từng phương trình thì chúng đều có nghiệm đúng:
1 + 2 . 2 = 5;
2 . 1 – 5 + 2 = –1;
3 . 1 – 2 . 5 = –7.
+) Bộ ba số (1; 1; 1) không là nghiệm của hệ phương trình bậc nhất đã cho.
Vì khi thay bộ số này vào phương trình thứ nhất của hệ ta được 1 + 2 . 1 = 5, đây là đẳng thức sai.
+) Bộ ba số (–1; 2; 3) có là nghiệm của hệ phương trình bậc nhất đã cho.
Vì khi thay bộ số này vào từng phương trình thì chúng đều có nghiệm đúng:
–1 + 2 . 3 = 5;
2 . (–1) – 2 + 3 = –1;
3 . (–1) – 2 . 2 = –7.

========================================================================

https://khoahoc.vietjack.com/thi-online/bai-tap-he-phuong-trinh-bac-nhat-ba-an-co-dap-an-a


\textbf{{QUESTION}}

Cho các hệ phương trình:
(1) {2x−y+z=13y−z=22z=3;$$ \{\begin{array}{l}2x-y+z=1\\ 3y-z=2\\ 2z=3\end{array};$$
(2) {2x−y+z=12y+z=−12y−z=−4.$$ \{\begin{array}{l}2x-y+z=1\\ 2y+z=-1\\ 2y-z=-4\end{array}.$$
a) Hệ phương trình (1) có gì đặc biệt? Giải hệ phương trình này.
b) Biến đổi hệ phương trình (2) về dạng như hệ phương trình (1). Giải hệ phương trình (2).

\textbf{{ANSWER}}

Hướng dẫn giải
a) Các phương trình trong hệ (1) theo thứ tự có số ẩn giảm dần: phương trình thứ nhất có 3 ẩn, phương trình thứ hai có 2 ẩn và phương trình thứ ba có 1 ẩn.
Hệ phương trình có dạng như hệ phương trình (1) được gọi là hệ phương trình bậc nhất ba ẩn dạng tam giác.
b) Trừ vế với vế của phương trình thứ hai cho phương trình thứ ba của hệ (2) ta được:
(2y + z) – (2y – z) = –1 – (–4) hay 2z = 3. Do đó hệ (2) tương đương với:
Từ phương trình thứ ba, ta có: z = 3/2
Thay z = 3/2 vào phương trình thứ hai ta được y = -5/4
Thay y = -5/4 và z = 3/2 vào phương trình thứ nhất, ta được x = -7/8
Vậy hệ phương trình đã cho có nghiệm duy nhất là (−78;−54;32).$$ (-\frac{7}{8};-\frac{5}{4};\frac{3}{2}).$$

========================================================================

https://khoahoc.vietjack.com/thi-online/bai-tap-he-phuong-trinh-bac-nhat-ba-an-co-dap-an-a


\textbf{{QUESTION}}

Giải các hệ phương trình sau bằng phương pháp Gauss:
a) {x−2y=1x+2y−z=−2x−3y+z=3;$$ \{\begin{array}{l}x-2y=1\\ x+2y-z=-2\\ x-3y+z=3\end{array};$$
b) {3x−y+2z=2x+2y−z=12x−3y+3z=2;$$ \{\begin{array}{l}3x-y+2z=2\\ x+2y-z=1\\ 2x-3y+3z=2\end{array};$$
c) {x−y+z=0x−4y+2z=−14x−y+3z=1.$$ \{\begin{array}{l}x-y+z=0\\ x-4y+2z=-1\\ 4x-y+3z=1\end{array}.$$

\textbf{{ANSWER}}

Hướng dẫn giải
a) {x−2y=1x+2y−z=−2x−3y+z=3⇔{x−2y=1−4y+z=3x−3y+z=3⇔{x−2y=1−4y+z=3y−z=−2⇔{x−2y=1−4y+z=3−3z=−5$$ \{\begin{array}{l}x-2y=1\\ x+2y-z=-2\\ x-3y+z=3\end{array}\Leftrightarrow \{\begin{array}{l}x-2y=1\\ -4y+z=3\\ x-3y+z=3\end{array}\Leftrightarrow \{\begin{array}{l}x-2y=1\\ -4y+z=3\\ y-z=-2\end{array}\Leftrightarrow \{\begin{array}{l}x-2y=1\\ -4y+z=3\\ -3z=-5\end{array}$$
Vậy hệ phương trình đã cho có nghiệm duy nhất là (13;−13;53).$$ (\frac{1}{3};-\frac{1}{3};\frac{5}{3}).$$
b) {3x−y+2z=2x+2y−z=12x−3y+3z=2⇔{3x−y+2z=2−7y+5z=−12x−3y+3z=2⇔{3x−y+2z=2−7y+5z=−17y−5z=−2⇔{3x−y+2z=2−7y+5z=−10y+0z=−3.$$ \{\begin{array}{l}3x-y+2z=2\\ x+2y-z=1\\ 2x-3y+3z=2\end{array}\Leftrightarrow \{\begin{array}{l}3x-y+2z=2\\ -7y+5z=-1\\ 2x-3y+3z=2\end{array}\Leftrightarrow \{\begin{array}{l}3x-y+2z=2\\ -7y+5z=-1\\ 7y-5z=-2\end{array}\Leftrightarrow \{\begin{array}{l}3x-y+2z=2\\ -7y+5z=-1\\ 0y+0z=-3\end{array}.$$
Phương trình thứ ba của hệ này vô nghiệm, do đó hệ phương trình đã cho vô nghiệm.
c) {x−y+z=0x−4y+2z=−14x−y+3z=1⇔{x−y+z=03y−z=14x−y+3z=1⇔{x−y+z=03y−z=1−3y+z=−1⇔{x−y+z=03y−z=1.$$ \{\begin{array}{l}x-y+z=0\\ x-4y+2z=-1\\ 4x-y+3z=1\end{array}\Leftrightarrow \{\begin{array}{l}x-y+z=0\\ 3y-z=1\\ 4x-y+3z=1\end{array}\Leftrightarrow \{\begin{array}{l}x-y+z=0\\ 3y-z=1\\ -3y+z=-1\end{array}\Leftrightarrow \{\begin{array}{l}x-y+z=0\\ 3y-z=1\end{array}.$$
Từ phương trình thứ hai, ta có z = 3y – 1, thay vào phương trình thứ nhất ta được x = –2y + 1.
Vậy hệ phương trình đã cho có vô số nghiệm dạng (–2y + 1; y; 3y – 1).

========================================================================

https://khoahoc.vietjack.com/thi-online/bo-de-kiem-tra-hoc-ki-1-toan-8-co-dap-an-moi-nhat/88810


\textbf{{QUESTION}}

Điền chữ Đ hoặc chữ S trong ô vuông tương ứng với mỗi phát biểu sau:
a. ( x + 5 )( x – 5 ) = x2 – 5                               c      
b. a3 – 1 = (a – 1 ) ( a2 + a + 1 )                        c 
c. Hình bình hành có một tâm đối xứng là giao điểm của hai đường chéo  c 
d. Hai tam giác có diện tích bằng nhau thì bằng nhau                                  c

\textbf{{ANSWER}}

1 – S; 
2 – Đ; 
3 – Đ; 
4 – S.

========================================================================

https://khoahoc.vietjack.com/thi-online/bo-de-kiem-tra-hoc-ki-1-toan-8-co-dap-an-moi-nhat/88810


\textbf{{QUESTION}}

Khoanh tròn chữ cái trước câu trả lời đúng nhất:
1. Đa thức x2 – 4x + 4 tại x = 2 có giá trị là:
A. 1            
B. 0                      
C. 4                      
D. 25
2. Giá trị của x để x ( x + 1) = 0 là:
A. x = 0                
B. x = - 1              
C. x = 0; x = 1    
D. x = 0; x = -1
3. Một hình thang có độ dài hai đáy là 6 cm và 10 cm. Độ dài đường trung bình của hình thang đó là :
A. 14 cm             
B. 7 cm                
 C. 8 cm                 
D. Một kết quả khác.
4. Một tam giác đều cạnh 2 dm thì có diện tích là:
A. √3$$ \sqrt{3}$$dm2            
B. 2√3$$ \sqrt{3}$$dm2           
C. √32$$ \frac{\sqrt{3}}{2}$$ dm2           
D. 6dm2

\textbf{{ANSWER}}

1 – B; 
2 – D; 
3 – C; 
4 – A.

========================================================================

https://khoahoc.vietjack.com/thi-online/bo-de-kiem-tra-hoc-ki-1-toan-8-co-dap-an-moi-nhat/88810


\textbf{{QUESTION}}

Tính:
a. 9x211y2:3x2y:6x11y$$ \frac{9{x}^{2}}{11{y}^{2}}:\frac{3x}{2y}:\frac{6x}{11y}$$
b. x2−49x−7+x−2$$ \frac{{x}^{2}-49}{x-7}+x-2$$
c. 11−x+11+x+21+x2+41+x4$$ \frac{1}{1-x}+\frac{1}{1+x}+\frac{2}{1+{x}^{2}}+\frac{4}{1+{x}^{4}}$$

\textbf{{ANSWER}}

a. $$ \frac{9{x}^{2}}{11{y}^{2}}:\frac{3x}{2y}:\frac{6x}{11y}$$
$$ =\frac{9{x}^{2}}{11{y}^{2}}.\frac{2y}{3x}.\frac{11y}{6x}$$
$$ =\frac{9{x}^{2}.2y.11y}{11{y}^{2}.3x.6x}$$ $$ =1.$$
b. $$ \frac{{x}^{2}-49}{x-7}+x-2$$
$$ =\frac{{x}^{2}-49}{x-7}+\frac{(x-2)(x-7)}{x-7}$$ $$ =\frac{{x}^{2}-49}{x-7}+\frac{{x}^{2}-9x+14}{x-7}$$
$$ =\frac{{x}^{2}-49+{x}^{2}-9x+14}{x-7}$$$$ =\frac{2{x}^{2}-9x+35}{x-7}$$
$$ =\frac{(x-7)(2x+5)}{x-7}$$

========================================================================

https://khoahoc.vietjack.com/thi-online/bo-de-kiem-tra-hoc-ki-1-toan-8-co-dap-an-moi-nhat/88810


\textbf{{QUESTION}}

Cho phân thức: A=4x2−12x+94x2−9$$ A=\frac{4{x}^{2}-12x+9}{4{x}^{2}-9}$$  
a) Tìm điều kiện xác định của phân thức A.
b) Thu gọn biểu thức A
c) Tính giá trị của biểu thức A với x=−34$$ x=-\frac{3}{4}$$.

\textbf{{ANSWER}}

a. 4x2−9≠0⇔(2x−3)(2x+3)≠0⇔{2x−3≠02x+3≠0$$ 4{x}^{2}-9\ne 0\Leftrightarrow (2x-3)(2x+3)\ne 0\Leftrightarrow \{\begin{array}{l}2x-3\ne 0\\ 2x+3\ne 0\end{array}$$
 
c. Thay x=−34$$ x=-\frac{3}{4}$$ vào biểu thức A, ta được:
A=2.(−34)−32.(−34)+3=−32−3−32+3=−9232=−3.$$ A=\frac{2.(-\frac{3}{4})-3}{2.(-\frac{3}{4})+3}=\frac{-\frac{3}{2}-3}{-\frac{3}{2}+3}=\frac{-\frac{9}{2}}{\frac{3}{2}}=-3.$$

========================================================================

https://khoahoc.vietjack.com/thi-online/giai-toan-9-phan-hinh-hoc-tap-2/21872


\textbf{{QUESTION}}

Hãy phát biểu bằng lời:
Công thức tính diện tích xung quanh của hình trụ.

\textbf{{ANSWER}}

Diện tích xung quanh hình lăng trụ thì bằng chu vi đường tròn đáy nhân với chiều cao.

========================================================================

https://khoahoc.vietjack.com/thi-online/giai-toan-9-phan-hinh-hoc-tap-2/21872


\textbf{{QUESTION}}

Hãy phát biểu bằng lời:
Công thức tính thể tích của hình trụ.

\textbf{{ANSWER}}

Thể tích hình trụ thì bằng tích của diện tích hình tròn đáy nhân với đường cao.

========================================================================

https://khoahoc.vietjack.com/thi-online/giai-toan-9-phan-hinh-hoc-tap-2/21872


\textbf{{QUESTION}}

Hãy phát biểu bằng lời:
Công thức tính diện tích xung quanh của hình nón.

\textbf{{ANSWER}}

Diện tích xung quanh hình nón thì bằng 1/2 tích của chu vi đường tròn đáy với đường sinh.

========================================================================

https://khoahoc.vietjack.com/thi-online/giai-toan-9-phan-hinh-hoc-tap-2/21872


\textbf{{QUESTION}}

Hãy phát biểu bằng lời:
Công thức tính thể tích của hình nón.

\textbf{{ANSWER}}

Thể tích hình nón bằng 1/3 tích của diện tích hình tròn đáy với chiều cao.

========================================================================

https://khoahoc.vietjack.com/thi-online/giai-toan-9-phan-hinh-hoc-tap-2/21872


\textbf{{QUESTION}}

Hãy phát biểu bằng lời:
Công thức tính diện tích của mặt cầu.

\textbf{{ANSWER}}

Diện tích mặt cầu thì bằng 4 lần diện tích hình tròn lớn.

========================================================================

https://khoahoc.vietjack.com/thi-online/bo-24-de-kiem-tra-giua-ki-2-toan-11-co-dap-an-moi-nhat/94299


\textbf{{QUESTION}}

A. q = 4;
D. $$ q=2\sqrt{2}.$$

\textbf{{ANSWER}}

Đáp án đúng là: A
Cho cấp số nhân (un) có số hạng tổng quát là un = u1.qn - 1.
u1 = 1; u4 = 64 nên ta có hệ phương trình
$$ \left\{\begin{array}{c}{u}_{1}=1\text{\hspace{0.33em}\hspace{0.33em}\hspace{0.33em}\hspace{0.33em}\hspace{0.33em}\hspace{0.33em}}\\ {u}_{1}.{q}^{3}=64\end{array}\right.\Rightarrow {q}^{3}=64\Rightarrow q=4.$$

========================================================================

https://khoahoc.vietjack.com/thi-online/70-bai-tap-toan-lop-6-cuc-hay-co-loi-giai


\textbf{{QUESTION}}

Hãy chỉ ra tính chất đặc trưng cho các phần tử của các tập hợp sau đây:
a) A = {0; 5; 10; 15;....; 100}
b) B = {111; 222; 333;...; 999}
c) C = {1; 4; 7; 10;13;...; 49}

\textbf{{ANSWER}}

a) A = {x $$ \in $$ N| x = 5k, k $$ \in $$N và k = $$ \overline{0;20}$$}
b) B = { x $$ \in $$ N| x = 111k, k $$ \in $$N và k = $$ \overline{1;9}$$}
c) C = { x $$ \in $$ N| x = 3k + 1, k $$ \in $$N và k = $$ \overline{0;16}$$}

========================================================================

https://khoahoc.vietjack.com/thi-online/70-bai-tap-toan-lop-6-cuc-hay-co-loi-giai


\textbf{{QUESTION}}

Viết tập hợp A các số tự nhiên có hai chữ số mà tổng các chữ số bằng 5.

\textbf{{ANSWER}}

Ta có: 5 = 0 + 5 = 1 + 4 = 2 + 4
Các số có hai chữ số là: 50; 14; 41; 24; 42
A = {14; 23; 32; 41; 50}

========================================================================

https://khoahoc.vietjack.com/thi-online/70-bai-tap-toan-lop-6-cuc-hay-co-loi-giai


\textbf{{QUESTION}}

Viết tập hợp A các số tự nhiên có một chữ số bằng hai cách.

\textbf{{ANSWER}}

Cách 1: A = {0; 1; 2; 3; 4; 5; 6; 7; 8; 9}
Cách 2: A = { x ∈N$$ \in N$$| x < 10}

========================================================================

https://khoahoc.vietjack.com/thi-online/70-bai-tap-toan-lop-6-cuc-hay-co-loi-giai


\textbf{{QUESTION}}

Cho A là tập hợp các số tự nhiên chẵn không nhỏ hơn 20 và không lớn hơn 30; B là tập hợp các số tự nhiên lớn hơn 26 và nhỏ hơn 33.
a. Viết các tập hợp A; B và cho biết mỗi tập hợp có bao nhiêu phần tử.
b. Viết tập hợp C các phần tử thuộc A mà không thuộc B.
c. Viết tập hợp D các phần tử thuộc B mà không thuộc A.

\textbf{{ANSWER}}

a. A = {20; 22; 24; 26; 28; 30}. Tập hợp A có 6 phần tử
B = {27; 28; 29; 30; 31; 32}. Tập hợp B có 6 phần tử
b. C = {20; 22; 24; 26}
c. D = {27; 29; 31; 32}

========================================================================

https://khoahoc.vietjack.com/thi-online/70-bai-tap-toan-lop-6-cuc-hay-co-loi-giai


\textbf{{QUESTION}}

Tích của 4 số tự nhiên liên tiếp là 93 024. Tìm 4 số đó.

\textbf{{ANSWER}}

Phân tích số ra thừa số nguyên tố: 93024 = 25.32.17.19 = 24.17.2.32.19 = 16.17.18.19 
4 số cần tìm là: 16, 17, 18, 19

========================================================================

https://khoahoc.vietjack.com/thi-online/sach-bai-tap-toan-7-tap-2/23580


\textbf{{QUESTION}}

Hãy lập bảng “tần số” từ các bài tập 1 và 2

\textbf{{ANSWER}}

*Bài 1
*Bài 2

========================================================================

https://khoahoc.vietjack.com/thi-online/giai-sbt-toan-7-on-tap-chuong-4-co-dap-an


\textbf{{QUESTION}}

Trong các câu sau đây, câu nào đúng?
A. Mọi tam giác có ít nhất một góc tù.
B. Mọi tam giác có ít nhất hai góc nhọn.
C. Mọi tam giác cân có một góc bằng 60°.
D. Tam giác vuông cân có hai góc vuông.

\textbf{{ANSWER}}

Đáp án đúng là: B.
+ Giả sử ta có tam giác có hai góc tù với $\widehat A > 90^\circ $, $\widehat B > 90^\circ $.
Suy ra $\widehat A + \widehat B > 90^\circ + 90^\circ = 180^\circ $
Mà tổng ba góc trong một tam giác bất kì luôn bằng 180°.
Do đó, một tam giác không thể có hai góc tù hay một tam giác không thể có ít nhất một góc tù. Vậy câu A sai.
+ Tam giác tù có một góc tù và hai góc nhọn, tam giác vuông có một góc vuông và hai góc nhọn, tam giác nhọn có ba góc nhọn. Vậy mọi tam giác có ít nhất hai góc nhọn. Do đó câu B đúng.
+ Tam giác cân không nhất thiết phải có một góc bằng 60°.
Chẳng hạn, tam giác ABC có $\widehat A = 100^\circ ,\,\widehat B = \widehat C = 40^\circ $ thì tam giác ABC cân tại đỉnh A.
Vậy câu C sai.
+ Tam giác không thể có hai góc vuông vì không thỏa mãn định lí tổng ba góc trong tam giác. Vậy câu D sai.

========================================================================

https://khoahoc.vietjack.com/thi-online/giai-sbt-toan-7-on-tap-chuong-4-co-dap-an


\textbf{{QUESTION}}

Trong các câu sau đây, câu nào sai?
A. Tổng số đo ba góc trong một tam giác bằng 180°.
B. Tổng số đo hai góc nhọn trong một tam giác vuông bằng 90°.
C. Tổng số đo hai góc nhọn trong một tam giác tù lớn hơn 90°.
D. Góc lớn nhất trong tam giác nhọn có số đo nhỏ hơn 90°.

\textbf{{ANSWER}}

Đáp án đúng là: C
Câu A đúng theo định lí tổng ba góc trong tam giác.
Tam giác vuông có một góc vuông, do đó tổng của hai góc nhọn trong tam giác vuông bằng 180° – 90° = 90°. Vậy câu B đúng.
Vì trong tam giác tù có một góc tù, góc này lớn hơn 90°, vậy tổng hai góc nhọn trong tam giác tù phải nhỏ hơn 90° để thỏa mãn định lí tổng ba góc trong tam giác. Vậy câu C sai.
Tam giác nhọn có cả ba góc đều là góc nhọn, do đó góc lớn nhất trong tam giác nhọn có số đo nhỏ hơn 90°. Vậy câu D đúng.

========================================================================

https://khoahoc.vietjack.com/thi-online/giai-sbt-toan-7-on-tap-chuong-4-co-dap-an


\textbf{{QUESTION}}

Trong các câu sau đây, câu nào đúng?
A. Hai tam giác có ba cặp góc tương ứng bằng nhau là hai tam giác bằng nhau.
B. Hai tam giác có ba cặp cạnh tương ứng bằng nhau là hai tam giác bằng nhau.
C. Hai tam giác có hai cặp cạnh tương ứng bằng nhau và một cặp góc tương ứng bằng nhau là hai tam giác bằng nhau.
D. Hai tam giác có một cặp cạnh tương ứng bằng nhau và cặp góc đối diện với cặp cạnh đó bằng nhau là hai tam giác bằng nhau.

\textbf{{ANSWER}}

Đáp án đúng là: B
Câu A sai do ba cặp góc của tam giác tương ứng bằng nhau thì các cạnh tương ứng chưa chắc đã bằng nhau.
Câu B đúng theo trường hợp bằng nhau cạnh – cạnh – cạnh của hai tam giác.
Câu C sai, cặp góc tương ứng bằng nhau phải là góc xem giữa hai cạnh thì câu này mới đúng,
Câu D sai do ta mới chỉ có hai yếu tố.

========================================================================

https://khoahoc.vietjack.com/thi-online/giai-sbt-toan-7-on-tap-chuong-4-co-dap-an


\textbf{{QUESTION}}

Trong các câu sau đây, câu nào sai?
A. Hai tam giác có cặp cạnh tương ứng bằng nhau và cặp góc xen giữa hai cặp cạnh đó bằng nhau thì hai tam giác bằng nhau.
B. Hai tam giác có một cặp cạnh tương ứng bằng nhau và hai cặp góc tương ứng cùng kề với cặp cạnh đó bằng nhau thì hai tam giác bằng nhau.
C. Hai tam giác bằng nhau có các cặp cạnh tương ứng bằng nhau và các cặp góc tương ứng bằng nhau.
D. Hai tam giác có các cặp góc tương ứng bằng nhau thì các cặp cạnh tương ứng bằng nhau.

\textbf{{ANSWER}}

Đáp án đúng là: D
Câu A đúng theo trường hợp bằng nhau cạnh – góc – cạnh của hai tam giác.
Câu B đúng theo trường hợp bằng nhau góc – cạnh – góc của hai tam giác.
Câu C đúng theo định nghĩa hai tam giác bằng nhau.
Câu D sai, do hai tam giác chưa chắc đã bằng nhau.

========================================================================

https://khoahoc.vietjack.com/thi-online/giai-sbt-toan-7-on-tap-chuong-4-co-dap-an


\textbf{{QUESTION}}

Trong các câu sau đây, câu nào đúng?
A. Tam giác có ba cạnh bằng nhau là tam giác đều.
B. Tam giác có hai góc bằng nhau là tam giác đều.
C. Tam giác nhọn có hai cạnh bằng nhau là tam giác đều.
D. Tam giác vuông có một góc có số đo bằng 60° là tam giác đều.

\textbf{{ANSWER}}

Đáp án đúng là: A
Theo định nghĩa, tam giác đều là tam giác có ba cạnh bằng nhau.
Trong tam giác đều, ba góc bằng nhau và bằng 60°.
Do đó câu A đúng và câu B, C, D sai.

========================================================================

https://khoahoc.vietjack.com/thi-online/22-cau-trac-nghiem-ham-so-lien-tuc-co-dap-an-phan-2


\textbf{{QUESTION}}

Cho hàm số $$ f\left(x\right)=\frac{2x-1}{{x}^{3}-4x}$$. Kết luận nào sau đây là đúng?
A. Hàm số f(x) liên tục tại điểm x=-2
B. Hàm số f(x) liên tục tại điểm x=0
C. Hàm số f(x) liên tục tại điểm x=0,5
D. Hàm số f(x) liên tục tại điểm x=2

\textbf{{ANSWER}}

Đáp án C
Hàm số đã cho không xác định tại x=0, x=-2, x=2 nên không liên tục tại các điểm đó. 
Hàm số liên tục tại x=0,5 vì nó thuộc tập xác định của hàm phân thức f(x).

========================================================================

https://khoahoc.vietjack.com/thi-online/22-cau-trac-nghiem-ham-so-lien-tuc-co-dap-an-phan-2


\textbf{{QUESTION}}

Cho f(x)=√x+2-√2-xx$$ f\left(x\right)=\frac{\sqrt{x+2}-\sqrt{2-x}}{x}$$ với x≢0$$ x\not\equiv 0$$. Phải bổ sung thêm giá trị f(0) bằng bao nhiêu để hàm số f(x) liên tục tại x=0?
A. 0
B.1 
C. 1√2$$ \frac{1}{\sqrt{2}}$$
D. 12√2$$ \frac{1}{2\sqrt{2}}$$

\textbf{{ANSWER}}

Đáp án C
$$ \underset{x\to 0}{\mathrm{lim}}f\left(x\right)=\underset{x\to 0}{\mathrm{lim}}\frac{\sqrt{x+2}-\sqrt{2-x}}{x}\phantom{\rule{0ex}{0ex}}=\underset{x\to 0}{\mathrm{lim}}\frac{x+2-2+x}{x\left(\sqrt{x+2}+\sqrt{2-x}\right)}=\underset{x\to 0}{\mathrm{lim}}\frac{2}{\sqrt{x+2}+\sqrt{2-x}}=\frac{1}{\sqrt{2}}$$
Vậy hàm số liên tục tại x=0 khi và chỉ khi f(0)=$$ \underset{x\to 0}{\mathrm{lim}}f\left(x\right)=\frac{1}{\sqrt{2}}$$

========================================================================

https://khoahoc.vietjack.com/thi-online/22-cau-trac-nghiem-ham-so-lien-tuc-co-dap-an-phan-2


\textbf{{QUESTION}}

Cho hàm số f(x)=x2-1x+1$$ f\left(x\right)=\frac{{x}^{2}-1}{x+1}$$ và f(2)=m2-2$$ f\left(2\right)={m}^{2}-2$$ với x≢2$$ x\not\equiv 2$$. Giá trị của m để f(x) liên tục tại x =2 là:
A. √3$$ \sqrt{3}$$
B. -√3$$ -\sqrt{3}$$
C. ±√3$$ \pm \sqrt{3}$$
D. ±3$$ \pm 3$$

\textbf{{ANSWER}}

Đáp án C
Hàm số liên tục tại $$ x\quad =\quad 2\Leftrightarrow \underset{x\to 2}{\mathrm{lim}}f\left(x\right)=f\left(2\right)$$ .
Ta có $$ \underset{x\to 2}{\mathrm{lim}}\frac{{x}^{2}-1}{x+1}=\underset{x\to 2}{\mathrm{lim}}(x-1)=1$$.
Vậy $$ {m}^{2}-2=1\Leftrightarrow {m}^{2}=3\Leftrightarrow \left\{\begin{array}{l}m=\sqrt{3}\\ m=-\sqrt{3}\end{array}\right.$$.

========================================================================

https://khoahoc.vietjack.com/thi-online/22-cau-trac-nghiem-ham-so-lien-tuc-co-dap-an-phan-2


\textbf{{QUESTION}}

Cho hàm số f(x)= {√x2+1x3-x+6   x≠3;x≠2b+√3           x=3;b∈R$$ f\left(x\right)=\quad \left\{\begin{array}{l}\sqrt{\frac{{x}^{2}+1}{{x}^{3}-x+6}}\quad \quad \quad x\ne 3;x\ne 2\\ b+\sqrt{3}\quad \quad \quad \quad \quad \quad \quad \quad \quad \quad \quad x=3;b\in R\end{array}\right.$$. Tìm b để f(x) liên tục tại x= 3.
A. $$ \sqrt{3}$$
B. $$ -\sqrt{3}$$
C. 2√33$$ \frac{2\sqrt{3}}{3}$$
D. -2√33$$ -\frac{2\sqrt{3}}{3}$$

\textbf{{ANSWER}}

Đáp án D
Hàm số liên tục tại x=3⇔limx→3f(x)=f(3)$$ x=3\Leftrightarrow \underset{x\to 3}{\mathrm{lim}}f\left(x\right)=f\left(3\right)$$.
 limx→3√x2+1x3-x+6=√32+133-3+6=1√3$$ \underset{x\to 3}{\mathrm{lim}}\sqrt{\frac{{x}^{2}+1}{{x}^{3}-x+6}}=\sqrt{\frac{{3}^{2}+1}{{3}^{3}-3+6}}=\frac{1}{\sqrt{3}}$$.
 f(3)=b+√3$$ f\left(3\right)=b+\sqrt{3}$$.
Vậy: b+√3=1√3⇔b=-√3+1√3=-2√3$$ b+\sqrt{3}=\frac{1}{\sqrt{3}}\Leftrightarrow b=-\sqrt{3}+\frac{1}{\sqrt{3}}=\frac{-2}{\sqrt{3}}$$.

========================================================================

https://khoahoc.vietjack.com/thi-online/200-cau-trac-nghiem-ham-so-mu-va-logarit-co-ban


\textbf{{QUESTION}}

Tìm tập xác định D của hàm số y = ( x2 - 3x + 2) 100  
A. D = [1; 2]
B. D = [2; +∞) ∪ (-∞; 1]
C. D = R
D. D = ( 1; 2)

\textbf{{ANSWER}}

Chọn C.
Hàm số y = xα với α nguyên dương, xác định với mọi x.
Do đó hàm số y = ( x2 - 3x + 2) 100    xác định với mọi x.

========================================================================

https://khoahoc.vietjack.com/thi-online/200-cau-trac-nghiem-ham-so-mu-va-logarit-co-ban


\textbf{{QUESTION}}

Tìm tập xác định D của hàm số y = ( x3 - 8) -100
A. D = ( 2; + ∞)
B. D = R \ {2}
C. D = ( -∞; 2)
D. D = R \ ( -2; 2)

\textbf{{ANSWER}}

Chọn B.
Hàm số y = xα với α nguyên âm, xác định với ∀ x ≠ 0.
Hàm số y = ( x3 - 8)-100   xác định  x3 – 8 ≠ 0 hay x ≠ 2.

========================================================================

https://khoahoc.vietjack.com/thi-online/200-cau-trac-nghiem-ham-so-mu-va-logarit-co-ban


\textbf{{QUESTION}}

Tìm tập xác định D của hàm số y = ( x3 - 8)0
A. D = [2; +∞)
B. D = R\{2}
C. D = ( -∞; 2)
D. R

\textbf{{ANSWER}}

Chọn B.
Hàm số y = xα với α= 0 xác định với x ≠ 0.
Hàm số đã cho xác định khi và chỉ khi x3 – 8 ≠ 0 hay x ≠ 2.

========================================================================

https://khoahoc.vietjack.com/thi-online/200-cau-trac-nghiem-ham-so-mu-va-logarit-co-ban


\textbf{{QUESTION}}

Tìm x để biểu thức (2x - 1)– 2  có nghĩa:
A. x ≠ 12$$ \frac{1}{2}$$
B. x > 12$$ \frac{1}{2}$$
C. 12$$ \frac{1}{2}$$< x < 2
D. x < 2

\textbf{{ANSWER}}

Chọn A.
Biểu thức ( 2x - 1)– 2   có nghĩa khi 2x – 1 ≠ 0 hay x ≠ 12$$ \frac{1}{2}$$

========================================================================

https://khoahoc.vietjack.com/thi-online/200-cau-trac-nghiem-ham-so-mu-va-logarit-co-ban


\textbf{{QUESTION}}

Tìm tập xác định D của hàm số y=(x2-6x+8)√2$$ y={\left({x}^{2}-6x+8\right)}^{\sqrt{2}}$$
A. D = R
B. D = [4; +∞) ∪ (-∞; 2]
C. D = (4; +∞) ∪ (-∞; 2)
D. D = [2; 4]

\textbf{{ANSWER}}

Chọn C.
Hàm số y = xα với α  không nguyên thì cơ số phải dương.
Do đó hàm số  đã cho xác định khi x2 - 6x + 8 > 0
Suy ra x > 4 hoặc x < 2.

========================================================================

https://khoahoc.vietjack.com/thi-online/bo-14-de-thi-hoc-ki-1-toan-8-co-dap-an/96995


\textbf{{QUESTION}}

Phân tích đa thức thành nhân tử: 
$$ 6{x}^{3}-6{x}^{2}$$

\textbf{{ANSWER}}

Phương pháp: 
Phân tích đa thức thành nhân tử nhờ các phương pháp đặt nhân tử chung, nhóm hạng tử chung hoặc phương pháp hằng đẳng thức. 
Cách giải: 
$$ 6{x}^{3}-6{x}^{2}=6{x}^{2}\left(x-1\right)$$

========================================================================

https://khoahoc.vietjack.com/thi-online/bo-14-de-thi-hoc-ki-1-toan-8-co-dap-an/96995


\textbf{{QUESTION}}

Phân tích đa thức thành nhân tử: 
9x2−6x+1−16y2$$ 9{x}^{2}-6x+1-16{y}^{2}$$

\textbf{{ANSWER}}

Phương pháp: 
Phân tích đa thức thành nhân tử nhờ các phương pháp đặt nhân tử chung, nhóm hạng tử chung hoặc phương pháp hằng đẳng thức. 
Cách giải: 
$$ 9{x}^{2}-6x+1-16{y}^{2}={\left(3x-1\right)}^{2}-16{y}^{2}$$$$ =\left(3x-1-4y\right)\left(3x-1+4y\right)$$

========================================================================

https://khoahoc.vietjack.com/thi-online/bo-14-de-thi-hoc-ki-1-toan-8-co-dap-an/96995


\textbf{{QUESTION}}

Bạn An mua một số táo và lê. Biết rằng hiệu bình phương của số quả táo và lê bằng 41. Hỏi bạn An mua bao nhiêu quả táo? (Biết rằng số táo nhiều hơn lê). 
Bạn An mua một số táo và lê. Biết rằng hiệu bình phương của số quả táo và lê bằng 41. Hỏi bạn An mua bao nhiêu quả táo? (Biết rằng số táo nhiều hơn lê).

\textbf{{ANSWER}}

Phương pháp: 
Sử dụng các phương pháp phân tích đa thức thành nhân tử, quy đồng mẫu nhiều phân số 
Cách giải: 
Gọi số quả táo An mua là x, số quả lê An mua là y  . 
Theo bài ra ta có: Hiệu bình phương số quả táo và lê bằng 41 nên ta có: 
x2−y2=41⇔(x+y)(x−y)=41$$ {x}^{2}-{y}^{2}=41\Leftrightarrow \left(x+y\right)\left(x-y\right)=41$$ (*) 
Vì số quả táo nhiều hơn số quả lê nên x−y>0$$ x-y>0$$ và x,y∈ℕ$$ x,y\in \mathbb{N}$$. 
Vậy An đã mua 21 quả táo và 20 quả lê.

========================================================================

https://khoahoc.vietjack.com/thi-online/bo-14-de-thi-hoc-ki-1-toan-8-co-dap-an/96995


\textbf{{QUESTION}}

Thực hiện phép tính: 
 
 
 
 
A=xx+2+4xx2−4+xx−2
A=xx+2+4xx2−4+xx−2
A
=
xx+2
x
x
x+2
x+2


x+2
x+2
x+2
x
+
2
+
4xx2−4
4x
4x
4
x
x2−4
x2−4


x2−4
x2−4
x2−4
x2
x
2
−
4
+
xx−2
x
x
x−2
x−2


x−2
x−2
x−2
x
−
2

\textbf{{ANSWER}}

Phương pháp: 
Sử dụng các phương pháp phân tích đa thức thành nhân tử, quy đồng mẫu nhiều phân số 
Cách giải:
$$ A=\frac{x}{x+2}+\frac{4x}{{x}^{2}-4}+\frac{x}{x-2}$$
 
Điều kiện:$$ x\ne \pm 2$$
$$ A=\frac{x}{x+2}+\frac{4x}{{x}^{2}-4}+\frac{x}{x-2}$$
$$ =\frac{x}{x+2}+\frac{4x}{\left(x+2\right)\left(x-2\right)}+\frac{x}{x-2}$$
$$ =\frac{2{x}^{2}+4x}{\left(x+2\right)\left(x-2\right)}=\frac{2x\left(x+2\right)}{\left(x+2\right)\left(x-2\right)}=\frac{2x}{x-2}$$.

========================================================================

https://khoahoc.vietjack.com/thi-online/bo-14-de-thi-hoc-ki-1-toan-8-co-dap-an/96995


\textbf{{QUESTION}}

Thực hiện phép chia đa thức: (2x4+11x+15x2−13x3−3)$$ \left(2{x}^{4}+11x+15{x}^{2}-13{x}^{3}-3\right)$$ cho đa thức (x2−4x−3)$$ \left({x}^{2}-4x-3\right)$$ .
Tìm giá trị nhỏ nhất của đa thức thương trong phép chia đa thức trên.

\textbf{{ANSWER}}

Phương pháp: 
Biến đổi biểu thức, chia đa thức cho đa thức. 
Sau đó biến đổi biểu thức thương tìm được về dạng:A=(f(x))2+const≥const$$ A={\left(f\left(x\right)\right)}^{2}+const\ge const$$  . 
Dấu "="$$ "="$$  xảy ra ⇔f(x)=0$$ \Leftrightarrow f\left(x\right)=0$$ .
 
⇒(2x4+11x+15x2−13x3−3):(x2−4x−3)$$ \Rightarrow \left(2{x}^{4}+11x+15{x}^{2}-13{x}^{3}-3\right):\left({x}^{2}-4x-3\right)$$
Xét A=2x2−5x+1$$ A=2{x}^{2}-5x+1$$
=2(x2−52x+12)$$ =2\left({x}^{2}-\frac{5}{2}x+\frac{1}{2}\right)$$
=2(x2−2.x.54+2516−1716)$$ =2\left({x}^{2}-2.x.\frac{5}{4}+\frac{25}{16}-\frac{17}{16}\right)$$
=2[(x−54)2−1716]$$ =2\left[{\left(x-\frac{5}{4}\right)}^{2}-\frac{17}{16}\right]$$
=2.(x−54)2−178≥−17  8$$ =2.{\left(x-\frac{5}{4}\right)}^{2}-\frac{17}{8}\ge \frac{-17}{\text{\hspace{0.17em}\hspace{0.17em}}8}$$
 
 
 
 
Dấu "="$$ "="$$  xảy ra ⇔(x−54)2=0⇒x−54=0⇒x=54$$ \Leftrightarrow {\left(x-\frac{5}{4}\right)}^{2}=0\Rightarrow x-\frac{5}{4}=0\Rightarrow x=\frac{5}{4}$$
Vậy Min  A=−17  8$$ Min\text{\hspace{0.17em}\hspace{0.17em}}A=\frac{-17}{\text{\hspace{0.17em}\hspace{0.17em}}8}$$  khi x=54$$ x=\frac{5}{4}$$  .

========================================================================

https://khoahoc.vietjack.com/thi-online/de-thi-thu-thpt-quoc-gia-mon-toan-co-chon-loc-va-loi-giai-chi-tiet-25-de/85935


\textbf{{QUESTION}}

A. 2.
B. $$ \sqrt{2}$$ . 

 
C. $$ \sqrt{6}$$ .

 
D. 6.

\textbf{{ANSWER}}

Đáp án C

========================================================================

https://khoahoc.vietjack.com/thi-online/de-thi-thu-thpt-quoc-gia-mon-toan-co-chon-loc-va-loi-giai-chi-tiet-25-de/85935


\textbf{{QUESTION}}

Giải bất phương trình    log13(1−x)<0$$ {\mathrm{log}}_{\frac{1}{3}}(1-x)<0$$

\textbf{{ANSWER}}

Đáp án C
Ta có: $$ {\mathrm{log}}_{\frac{1}{3}}(1-x)<0\Leftrightarrow \{\begin{array}{l}1-x>0\\ 1-x>{\left(\frac{1}{3}\right)}^{0}=1\end{array}\Leftrightarrow \{\begin{array}{l}x<1\\ x<0\end{array}\Leftrightarrow x<0$$ .

========================================================================

https://khoahoc.vietjack.com/thi-online/tong-hop-de-thi-chinh-thuc-vao-10-mon-toan-nam-2021-co-dap-an-phan-1/104556


\textbf{{QUESTION}}

Tính giá tri của các biểu thức sau :
$$ A=\sqrt{49}-\sqrt{25}$$

\textbf{{ANSWER}}

$$ A=\sqrt{49}-\sqrt{25}=7-5=2$$

========================================================================

https://khoahoc.vietjack.com/thi-online/tong-hop-de-thi-chinh-thuc-vao-10-mon-toan-nam-2021-co-dap-an-phan-1/104556


\textbf{{QUESTION}}

B=√5+√(3−√5)2$$ B=\sqrt{5}+\sqrt{{\left(3-\sqrt{5}\right)}^{2}}$$

\textbf{{ANSWER}}

B=√5+√(3−√5)2=√5+|3−√5|=√5+3−√5  (Do  3>√5)=3$$ \begin{array}{l}B=\sqrt{5}+\sqrt{{\left(3-\sqrt{5}\right)}^{2}}=\sqrt{5}+\left|3-\sqrt{5}\right|\\ =\sqrt{5}+3-\sqrt{5}\text{\hspace{0.17em}\hspace{0.17em}}(Do\text{\hspace{0.17em}\hspace{0.17em}}3>\sqrt{5})=3\end{array}$$
Vậy B = 3

========================================================================

https://khoahoc.vietjack.com/thi-online/tong-hop-de-thi-chinh-thuc-vao-10-mon-toan-nam-2021-co-dap-an-phan-1/104556


\textbf{{QUESTION}}

P=x−4√x+2+x+3√x√x
P=x−4√x+2+x+3√x√x
P
=
x−4√x+2
x−4
x−4
x
−
4
√x+2
√x+2


√x+2
√x+2
√x+2
√x
√
√
x

x
x
+
2
+
x+3√x√x
x+3√x
x+3√x
x
+
3
√x
√
√
x

x
x
√x
√x


√x
√x
√x
√
√
x

x
x
>0
>0
>
0

\textbf{{ANSWER}}

a) Với x>0,$$ x>\mathrm{0,}$$ ta có :
P=x−4√x+2+x+3√x√x=(√x−2)(√x+2)√x+2+√x(√x+3)√x=√x−2+√x+3=2√x+1$$ \begin{array}{l}P=\frac{x-4}{\sqrt{x}+2}+\frac{x+3\sqrt{x}}{\sqrt{x}}=\frac{\left(\sqrt{x}-2\right)\left(\sqrt{x}+2\right)}{\sqrt{x}+2}+\frac{\sqrt{x}\left(\sqrt{x}+3\right)}{\sqrt{x}}\\ =\sqrt{x}-2+\sqrt{x}+3=2\sqrt{x}+1\end{array}$$
Vậy với x>0$$ x>0$$thì P=2√x+1$$ P=2\sqrt{x}+1$$

========================================================================

https://khoahoc.vietjack.com/thi-online/tong-hop-de-thi-chinh-thuc-vao-10-mon-toan-nam-2021-co-dap-an-phan-1/104556


\textbf{{QUESTION}}

b) Tìm giá trị của x để P = 5

\textbf{{ANSWER}}

b) Để P = 5 thì 2√x+1=5⇔2√x=4⇔√x=2⇒x=4(tm)$$ 2\sqrt{x}+1=5\Leftrightarrow 2\sqrt{x}=4\Leftrightarrow \sqrt{x}=2\Rightarrow x=4\left(tm\right)$$
Vậy để P = 5 thì x = 4

========================================================================

https://khoahoc.vietjack.com/thi-online/10-bai-tap-phan-so-thap-phasn-so-thap-phan-duong-so-thap-phan-am-so-doi-cua-so-thap-phan-co-loi-giai


\textbf{{QUESTION}}

Phân số thập phân  $$ -\frac{784}{100}$$ dưới dạng số thập phân ta được

\textbf{{ANSWER}}

Đáp án đúng là: C
Ta có  $$ -\frac{784}{100}=-7,84$$.

========================================================================

https://khoahoc.vietjack.com/thi-online/10-bai-tap-phan-so-thap-phasn-so-thap-phan-duong-so-thap-phan-am-so-doi-cua-so-thap-phan-co-loi-giai


\textbf{{QUESTION}}

Phân số thập phân thích hợp điền vào chỗ chấm trong “0,25 = …” là
A. −25100$$ -\frac{25}{100}$$
B. 25100$$ \frac{25}{100}$$
C. −251  000$$ -\frac{25}{1\text{\hspace{0.17em}\hspace{0.17em}}000}$$
D. 251  000$$ \frac{25}{1\text{\hspace{0.17em}\hspace{0.17em}}000}$$

\textbf{{ANSWER}}

Đáp án đúng là: B
Ta có:  $$ 0,25=\frac{25}{100}.$$

========================================================================

https://khoahoc.vietjack.com/thi-online/10-bai-tap-phan-so-thap-phasn-so-thap-phan-duong-so-thap-phan-am-so-doi-cua-so-thap-phan-co-loi-giai


\textbf{{QUESTION}}

Cách viết nào dưới đây là sai?
A. −610=−0,6$$ \frac{-6}{10}=-0,6$$
B. −174,6=−1  74610$$ -174,6=-\frac{1\text{\hspace{0.17em}\hspace{0.17em}}746}{10}$$
C.  34100=0,34$$ \frac{34}{100}=0,34$$
D. 27,18=−2  7181  000$$ 27,18=-\frac{2\text{\hspace{0.17em}\hspace{0.17em}}718}{1\text{\hspace{0.17em}\hspace{0.17em}}000}$$

\textbf{{ANSWER}}

Đáp án đúng là: D
Đáp án D sai do  27,18=2  718100$$ 27,18=\frac{2\text{\hspace{0.17em}\hspace{0.17em}}718}{100}$$.

========================================================================

https://khoahoc.vietjack.com/thi-online/10-bai-tap-phan-so-thap-phasn-so-thap-phan-duong-so-thap-phan-am-so-doi-cua-so-thap-phan-co-loi-giai


\textbf{{QUESTION}}

Phần số nguyên của số 541,29 là
B. – 541;

\textbf{{ANSWER}}

Đáp án đúng là: A
Trong mỗi số thập phân thì phần số nguyên bên trái dấu “,”. 
Do đó phần số nguyên của số 541,29 là 541.

========================================================================

https://khoahoc.vietjack.com/thi-online/10-bai-tap-phan-so-thap-phasn-so-thap-phan-duong-so-thap-phan-am-so-doi-cua-so-thap-phan-co-loi-giai


\textbf{{QUESTION}}

Số – 7,059 có phần thập phân là

\textbf{{ANSWER}}

Đáp án đúng là: B
Mỗi số thập phân thì phần thập phân viết bên phải dấu “,”.
Do đó số – 7,059 có phần thập phân là 059.

========================================================================

https://khoahoc.vietjack.com/thi-online/10-bai-tap-chusng-minh-ba-diem-thang-hang-co-loi-giai


\textbf{{QUESTION}}

Cho tam giác ABC có trọng tâm G, lấy các điểm I, J thỏa mãn: $$ \overrightarrow{IA}=2\overrightarrow{IB}$$, $$ 3\overrightarrow{JA}+2\overrightarrow{JC}=\overrightarrow{0}$$ . Ba điểm nào sau đây thẳng hàng ?

\textbf{{ANSWER}}

Hướng dẫn giải:
Đáp án đúng là: D.
+ Vì G là trọng tâm tam giác ABC nên G nằm trong tam giác ABC, do đó ba điểm A, B, G và A, C, G không thể thẳng hàng. 
+ Vì $$ \overrightarrow{IA}=2\overrightarrow{IB}$$  nên A, I, B thẳng hàng và I không phải trung điểm AB nên A, I, G không thẳng hàng.
+ Ta có: G là trọng tâm tam giác ABC nên:
$$ \overrightarrow{JA}+\overrightarrow{JB}+\overrightarrow{JC}=3\overrightarrow{JG}$$
$$ \Leftrightarrow 2\overrightarrow{JA}+2\overrightarrow{JB}+2\overrightarrow{JC}=6\overrightarrow{JG}$$
 
 
Mà:$$ 3\overrightarrow{JA}+2\overrightarrow{JC}=\overrightarrow{0}\Rightarrow 2\overrightarrow{JC}=-3\overrightarrow{JA}$$  
Nên:$$ 2\overrightarrow{JA}+2\overrightarrow{JB}-3\overrightarrow{JA}=6\overrightarrow{JG}$$
$$ \Leftrightarrow 2\overrightarrow{JB}=6\overrightarrow{JG}+\overrightarrow{JA}$$  
 
Mặt khác:$$ \overrightarrow{IA}=2\overrightarrow{IB}\Leftrightarrow \overrightarrow{IJ}+\overrightarrow{JA}=2\overrightarrow{IJ}+2\overrightarrow{JB}$$

Mà $$ 2\overrightarrow{JB}=6\overrightarrow{JG}+\overrightarrow{JA}$$  nên ta lại có:
$$ \overrightarrow{IJ}+\overrightarrow{JA}=2\overrightarrow{IJ}+6\overrightarrow{JG}+\overrightarrow{JA}$$
$$ \Leftrightarrow \overrightarrow{IJ}=-6\overrightarrow{JG}$$
 
 
Vậy I, J, G thẳng hàng.

========================================================================

https://khoahoc.vietjack.com/thi-online/10-bai-tap-chusng-minh-ba-diem-thang-hang-co-loi-giai


\textbf{{QUESTION}}

Cho tam giác ABC, lấy các điểm M, N, P thỏa mãn: →MA+→MB=→0$$ \overrightarrow{MA}+\overrightarrow{MB}=\overrightarrow{0}$$ , 3→AN−2→AC=→0$$ 3\overrightarrow{AN}-2\overrightarrow{AC}=\overrightarrow{0}$$ , →PB=2→PC$$ \overrightarrow{PB}=2\overrightarrow{PC}$$ . Ba điểm nào sau đây thẳng hàng ?

\textbf{{ANSWER}}

Hướng dẫn giải:
Đáp án đúng là: A.
Ta có:
$$ 3\overrightarrow{AN}-2\overrightarrow{AC}=\overrightarrow{0}$$
 
   $$ \Leftrightarrow 3\overrightarrow{AM}+3\overrightarrow{MN}-2\overrightarrow{AP}-2\overrightarrow{PC}=\overrightarrow{0}$$      (quy tắc ba điểm)
$$ \Leftrightarrow \overrightarrow{AM}+3\overrightarrow{MN}+2\overrightarrow{PM}-2\overrightarrow{PC}=\overrightarrow{0}$$
Mà: $$ \overrightarrow{AM}=\overrightarrow{MB}$$  và $$ 2\overrightarrow{PC}=\overrightarrow{PB}$$  nên ta có:
$$ \overrightarrow{AM}+3\overrightarrow{MN}+2\overrightarrow{PM}-2\overrightarrow{PC}=\overrightarrow{0}$$
$$ \Leftrightarrow \overrightarrow{MB}+3\overrightarrow{MN}+2\overrightarrow{PM}+\overrightarrow{BP}=0$$
$$ \Leftrightarrow \overrightarrow{MP}+3\overrightarrow{MN}+2\overrightarrow{PM}=0$$
 
 
 
 $$ \Leftrightarrow 3\overrightarrow{MN}=\overrightarrow{MP}$$.
Vậy M, N, P thẳng hàng.

========================================================================

https://khoahoc.vietjack.com/thi-online/10-bai-tap-chusng-minh-ba-diem-thang-hang-co-loi-giai


\textbf{{QUESTION}}

Cho điểm A, B, C sao cho:→CA−2→CB=→0$$ \overrightarrow{CA}-2\overrightarrow{CB}=\overrightarrow{0}$$  . Cho điểm M bất kỳ trong mặt phẳng và gọi →MN$$ \overrightarrow{MN}$$ là vectơ định bởi →MN=→MA−2→MB$$ \overrightarrow{MN}=\overrightarrow{MA}-2\overrightarrow{MB}$$ . Ba điểm nào sau đây thẳng hàng ?

\textbf{{ANSWER}}

Hướng dẫn giải:
Đáp án đúng là: C.
Ta có: 
  →CA−2→CB=→0⇔→CA−→CB=→CB⇔→BA=→CB$$ \overrightarrow{CA}-2\overrightarrow{CB}=\overrightarrow{0}\Leftrightarrow \overrightarrow{CA}-\overrightarrow{CB}=\overrightarrow{CB}\Leftrightarrow \overrightarrow{BA}=\overrightarrow{CB}$$
⇔→BC+→CA=→CB⇔→CA−2→CB=→0$$ \Leftrightarrow \overrightarrow{BC}+\overrightarrow{CA}=\overrightarrow{CB}\Leftrightarrow \overrightarrow{CA}-2\overrightarrow{CB}=\overrightarrow{0}$$
 
Mặt khác ta có:
→MN=→MA−2→MB=→MC+→CA−2(→MC+→CB)=−→MC+(→CA−2→CB)=−→MC$$ \begin{array}{l}\overrightarrow{MN}=\overrightarrow{MA}-2\overrightarrow{MB}\\ =\overrightarrow{MC}+\overrightarrow{CA}-2\left(\overrightarrow{MC}+\overrightarrow{CB}\right)\\ =-\overrightarrow{MC}+\left(\overrightarrow{CA}-2\overrightarrow{CB}\right)\\ =-\overrightarrow{MC}\end{array}$$
⇒→MN=−→MC$$ \Rightarrow \overrightarrow{MN}=-\overrightarrow{MC}$$
 
 
Vậy M, N, C thẳng hàng.

========================================================================

https://khoahoc.vietjack.com/thi-online/12-cau-trac-nghiem-vi-phan-co-dap-an-phan-2


\textbf{{QUESTION}}

Cho hàm số $$ y=f\left(x\right)={\left(x-1\right)}^{2}$$. Biểu thức nào sau đây chỉ vi phân của hàm số f(x) ?
A. $$ \text{d}y=2\left(x-1\right)\text{d}x$$
B. $$ \text{d}y={\left(x-1\right)}^{2}\text{d}x$$
C. $$ \text{d}y=2\left(x-1\right)$$
D. $$ \text{d}y=2\left(x-1\right)\text{d}x$$

\textbf{{ANSWER}}

Ta có $$ f\text{'}\left(x\right)\text{}=2.(x-1).(x-1)\text{'}=2(x-1)$$ nên vi  phân của hàm số đã cho là: 
$$ \text{d}y={f}^{\text{'}}\left(x\right)\text{d}x=2\left(x-1\right)\text{d}x$$.
Chọn đáp án A

========================================================================

https://khoahoc.vietjack.com/thi-online/12-cau-trac-nghiem-vi-phan-co-dap-an-phan-2


\textbf{{QUESTION}}

Tìm vi phân của các hàm số $$ y=\mathrm{tan}2x$$
A. $$ dy=(1+{\mathrm{tan}}^{2}2x)dx$$
B. $$ dy=(1-{\mathrm{tan}}^{2}2x)dx$$
C. $$ dy=2(1-{\mathrm{tan}}^{2}2x)dx$$
D. $$ dy=2(1+{\mathrm{tan}}^{2}2x)dx$$

\textbf{{ANSWER}}

Ta có : $$ f\text{'}\text{}\left(x\right)=(1+\text{}{\mathrm{tan}}^{2}2x).\left(2x\right)\text{'}=2.(1+{\mathrm{tan}}^{2}2x)$$
Do đó, vi phân của hàm số đã cho là:$$ dy=2(1+{\mathrm{tan}}^{2}2x)dx$$
 Chọn đáp án D.

========================================================================

https://khoahoc.vietjack.com/thi-online/12-cau-trac-nghiem-vi-phan-co-dap-an-phan-2


\textbf{{QUESTION}}

Xét hàm số y=f(x)=√1+cos22x$$ y=f\left(x\right)=\sqrt{1+{\mathrm{cos}}^{2}2x}$$. Chọn câu đúng:
A. df(x)=−sin4x2√1+cos22xdx$$ \text{d}f\left(x\right)=\frac{-\mathrm{sin}4x}{2\sqrt{1+{\mathrm{cos}}^{2}2x}}\text{d}x$$
B. df(x)=−sin4x√1+cos22xdx$$ \text{d}f\left(x\right)=\frac{-\mathrm{sin}4x}{\sqrt{1+{\mathrm{cos}}^{2}2x}}\text{d}x$$
C. df(x)=cos2x√1+cos22xdx$$ \text{d}f\left(x\right)=\frac{\mathrm{cos}2x}{\sqrt{1+{\mathrm{cos}}^{2}2x}}\text{d}x$$
D. df(x)=−sin2x2√1+cos22xdx$$ \text{d}f\left(x\right)=\frac{-\mathrm{sin}2x}{2\sqrt{1+{\mathrm{cos}}^{2}2x}}\text{d}x$$

\textbf{{ANSWER}}

Ta có : dy=f'(x)dx=(1+cos22x)'2√1+cos22xdx$$ \text{d}y={f}^{\text{'}}\left(x\right)\text{d}x=\frac{{\left(1+{\mathrm{cos}}^{2}2x\right)}^{\text{'}}}{2\sqrt{1+{\mathrm{cos}}^{2}2x}}\text{d}x$$  
=−4cos2x.sin2x2√1+cos22xdx=−2cos2x.sin2x√1+cos22xdx=−sin4x√1+cos22xdx$$ =\frac{-4\mathrm{cos}2x.\mathrm{sin}2x}{2\sqrt{1+{\mathrm{cos}}^{2}2x}}\text{d}x=\frac{-2\mathrm{cos}2x.\mathrm{sin}2x}{\sqrt{1+{\mathrm{cos}}^{2}2x}}dx=\frac{-\mathrm{sin}4x}{\sqrt{1+{\mathrm{cos}}^{2}2x}}\text{d}x$$
 Chọn đáp án B.

========================================================================

https://khoahoc.vietjack.com/thi-online/12-cau-trac-nghiem-vi-phan-co-dap-an-phan-2


\textbf{{QUESTION}}

Cho hàm số y=x+2x−1$$ y=\frac{x+2}{x-1}$$. Vi phân của hàm số là:
A. dy=dx(x−1)2$$ \text{d}y=\frac{\text{d}x}{{\left(x-1\right)}^{2}}$$
B. dy=3dx(x−1)2$$ \text{d}y=\frac{3\text{d}x}{{\left(x-1\right)}^{2}}$$
C. dy=−3dx(x−1)2$$ \text{d}y=\frac{-3\text{d}x}{{\left(x-1\right)}^{2}}$$
D. dy=−dx(x−1)2$$ \text{d}y=-\frac{\text{d}x}{{\left(x-1\right)}^{2}}$$

\textbf{{ANSWER}}

Vi phân của hàm số đã cho là : 
 dy=(x+2x−1)'dx=  (x+​2)'.(x−1)−(x+​2).(x−1)'(x−1)2= 1(x−1)−(x+2).1(x−1)2=−3(x−1)2dx$$ \text{d}y={\left(\frac{x+2}{x-1}\right)}^{\text{'}}\text{d}x=\text{\hspace{0.17em}\hspace{0.17em}}\frac{(x+\text{}2)\text{'}.(x-1)-(x+\text{}2).(x-1)\text{'}}{{(x-1)}^{2}}\phantom{\rule{0ex}{0ex}}=\text{\hspace{0.17em}}\frac{1(x-1)-(x+2).1}{{(x-1)}^{2}}=-\frac{3}{{\left(x-1\right)}^{2}}\text{d}x$$
 Chọn đáp án C.

========================================================================

https://khoahoc.vietjack.com/thi-online/12-cau-trac-nghiem-vi-phan-co-dap-an-phan-2


\textbf{{QUESTION}}

Hàm số y=xsinx+cosx$$ y=x\mathrm{sin}x+\mathrm{cos}x$$  có vi phân là:
A. dy=(xcosx–sinx)dx$$ \text{d}y=\left(x\mathrm{cos}x–\mathrm{sin}x\right)\text{d}x$$
B. dy=(xcosx)dx$$ \text{d}y=\left(x\mathrm{cos}x\right)\text{d}x$$
C. dy=(cosx–sinx)dx$$ \text{d}y=\left(\mathrm{cos}x–\mathrm{sin}x\right)\text{d}x$$
D. dy=(xsinx)dx$$ \text{d}y=\left(x\mathrm{sin}x\right)\text{d}x$$

\textbf{{ANSWER}}

Ta có dy=(xsinx+cosx)'dx=(sinx+xcosx−sinx)dx=(xcosx)dx$$ dy={\left(x\mathrm{sin}x+\mathrm{cos}x\right)}^{\text{'}}\text{d}x=\left(\mathrm{sin}x+x\mathrm{cos}x-\mathrm{sin}x\right)\text{d}x=\left(x\mathrm{cos}x\right)\text{d}x$$
Chọn đáp án B.

========================================================================

https://khoahoc.vietjack.com/thi-online/bai-tap-tim-mot-so-biet-gia-tri-mot-phan-so-cua-no


\textbf{{QUESTION}}

Tìm một số, biết:
a) $$ \frac{2}{3}$$ của nó bằng 7,2                         
b) 1$$ \frac{3}{7}$$ của nó bằng -5
c) $$ \frac{2}{7}$$ của nó bằng 14                          
d) 3$$ \frac{2}{5}$$ của nó bằng $$ \frac{-2}{3}$$

\textbf{{ANSWER}}

$$ a)\mathrm{7,2}:\frac{2}{3}=\frac{54}{5}\begin{array}{cccc}& & & \end{array}b)-5:1\frac{3}{7}=\frac{-7}{2}$$
$$ c)14:\frac{2}{7}=49\begin{array}{cccc}& & & \end{array}\text{d})\frac{-2}{3}:3\frac{2}{5}=\frac{-10}{51}$$

========================================================================

https://khoahoc.vietjack.com/thi-online/bai-tap-tim-mot-so-biet-gia-tri-mot-phan-so-cua-no


\textbf{{QUESTION}}

Tìm một số, biết:
a) 37$$ \frac{3}{7}$$ của nó bằng 36;                           
b)349$$ \frac{4}{9}$$ của nó bằng-62
c) 25$$ \frac{2}{5}$$ của nó bằng 24;                         
 d)358$$ \frac{5}{8}$$ của nó bằng 58.

\textbf{{ANSWER}}

a) 84                     
b) -18
c) 60                     
d) 16

========================================================================

https://khoahoc.vietjack.com/thi-online/bai-tap-tim-mot-so-biet-gia-tri-mot-phan-so-cua-no


\textbf{{QUESTION}}

Trong đậu đen nấu chín tỉ lệ chất đạm chiếm 50%. Tính số kilôgam đậu đen đã nấu chín để có 2 kg chất đạm.

\textbf{{ANSWER}}

Số kilôgam đậu đen đã nấu chín là 1,2: 24100$$ \frac{24}{100}$$= 5kg

========================================================================

https://khoahoc.vietjack.com/thi-online/bai-tap-tim-mot-so-biet-gia-tri-mot-phan-so-cua-no


\textbf{{QUESTION}}

23 quả dưa hấu nặng 412kg. Hỏi quả dưa hấu nặng bao nhiêu kilôgam?

\textbf{{ANSWER}}

6,7kg

========================================================================

https://khoahoc.vietjack.com/thi-online/bai-tap-tim-mot-so-biet-gia-tri-mot-phan-so-cua-no


\textbf{{QUESTION}}

23$$ \frac{2}{3}$$ số tuổi của Mai cách đây 3 năm là 6 tuổi. Hỏi hiện nay Mai bao nhiêu

\textbf{{ANSWER}}

9 tuổi

========================================================================

https://khoahoc.vietjack.com/thi-online/15-cau-trac-nghiem-toan-9-chan-troi-sang-tao-bai-tap-cuoi-chuong-x-co-dap-an


\textbf{{QUESTION}}

I. Nhận biết
Gọi $l,h,R$ lần lượt là độ dài đường sinh, chiều cao và bán kính của hình trụ. Đẳng thức luôn đúng là
A. $R = h.$
B. $l = h.$
C. ${l^2} = {h^2} + {R^2}.$
D. ${R^2} = {h^2} + {l^2}.$

\textbf{{ANSWER}}

Đáp án đúng là: B
Trong hình trụ, ta có độ dài đường sinh luôn bằng chiều cao của hình trụ.
Tức là, $l = h.$
Vậy ta chọn phương án B.

========================================================================

https://khoahoc.vietjack.com/thi-online/giai-sbt-toan-hoc-11-ctst-bai-4-phuong-trinh-bat-phuong-trinh-mu-va-logarit-co-dap-an


\textbf{{QUESTION}}

Giải các phương trình sau: 
a) $$ {3}^{2x+1}=\frac{1}{27}$$ ;                              
b) 52x = 10;
c) 3x = 18;                               
d) $$ 0,{2}^{x-1}=\frac{1}{\sqrt{125}}$$ ;
e) 53x  = 25x – 2;                                 
g) $$ {\left(\frac{1}{8}\right)}^{x+1}={\left(\frac{1}{32}\right)}^{x-1}$$ .

\textbf{{ANSWER}}

a) 32x + 1 = 3– 3 
⇔ 2x + 1= –3 (do 3 > 1) 
⇔ x = – 2. 
Vậy phương trình có nghiệm là x = 2.
b) 52x =10 
⇔ 2x = log5 10  
 $$ \Leftrightarrow x=\frac{1}{2}{\mathrm{log}}_{5}10$$
Vậy phương trình có nghiệm là  $$ x=\frac{1}{2}{\mathrm{log}}_{5}10$$
c) 3x = 18 ⇔ x = log3 18 
Vậy phương trình có nghiệm là x = log3 18.
d)   $$ 0,{2}^{x-1}=\frac{1}{\sqrt{125}}$$
  $$ \Leftrightarrow {5}^{1-x}={5}^{-\frac{3}{2}}$$
 $$ \Leftrightarrow 1-x=\frac{-3}{2}$$(do 5 > 1)
$$ \Leftrightarrow x=\frac{5}{2}$$
 
Vậy phương trình có nghiệm là $$ x=\frac{5}{2}$$
e) 53x = 25x–2  
⇔ 53x = 52x–4 
⇔ 3x = 2x – 4 (do 5 > 1)
⇔ x = – 4.
Vậy phương trình có nghiệm là x = – 4.
g) $$ {\left(\frac{1}{8}\right)}^{x+1}={\left(\frac{1}{32}\right)}^{x-1}$$
$$ \Leftrightarrow {\left({2}^{-3}\right)}^{x+1}={\left({2}^{-5}\right)}^{x-1}$$
$$ \Leftrightarrow {2}^{-3(x+1)}={2}^{-5x+5}$$
 
 
⇔ –3x – 3 = –5x + 5 (do 2 > 1)
⇔ 2x = 8 ⇔ x = 4.
Vậy phương trình có nghiệm là x = 4.

========================================================================

https://khoahoc.vietjack.com/thi-online/giai-sbt-toan-hoc-11-ctst-bai-4-phuong-trinh-bat-phuong-trinh-mu-va-logarit-co-dap-an


\textbf{{QUESTION}}

Giải các phương trình sau:
a) log3(2x−1)=3$$ {\mathrm{log}}_{3}(2x-1)=3$$  ;                       
b) log49x=0,25$$ {\mathrm{log}}_{49}x=0,25$$  ;
c) log2(3x+1)=log2(2x−4)$$ {\mathrm{log}}_{2}(3x+1)={\mathrm{log}}_{2}(2x-4)$$ ;  
d) log5(x−1)+log5(x−3)=log5(2x+10)$$ {\mathrm{log}}_{5}(x-1)+{\mathrm{log}}_{5}(x-3)={\mathrm{log}}_{5}(2x+10)$$ ;
e) log x + log (x – 3) = 1;               
g) log2(log81x)=−2$$ {\mathrm{log}}_{2}\left({\mathrm{log}}_{81}x\right)=-2$$ .

\textbf{{ANSWER}}

a) Điều kiện: 2x – 1 > 0 $$ \Leftrightarrow x>\frac{1}{2}$$
Ta có:    $$ {\mathrm{log}}_{3}(2x-1)=3$$
  $$ \Leftrightarrow 2x-1={3}^{3}=27$$
  $$ \Leftrightarrow x=14$$ (nhận)
Vậy tập nghiệm của phương trình là: S = {14}.
b) Điều kiện: $$ x>0$$
Ta có: $$ {\mathrm{log}}_{49}x=0,25$$
$$ \Leftrightarrow {\mathrm{log}}_{{7}^{2}}x=\frac{1}{4}$$
$$ \Leftrightarrow \frac{1}{2}{\mathrm{log}}_{7}x=\frac{1}{4}$$
$$ \Leftrightarrow {\mathrm{log}}_{7}x=\frac{1}{2}$$
 
 
 
 $$ \Leftrightarrow x=\sqrt{7}$$ (nhận)
Vậy tập nghiệm của phương trình là: $$ S=\left\{\sqrt{7}\right\}$$
c) Điều kiện: $$ \left\{\begin{array}{l}x>0\\ {\mathrm{log}}_{81}x>0\end{array}\right.\Rightarrow \left\{\begin{array}{l}x>0\\ x>{81}^{0}=1\end{array}\right.\Rightarrow x>1$$
Ta có:  $$ {\mathrm{log}}_{2}(3x+1)={\mathrm{log}}_{2}(2x-4)$$
⇔ 3x + 1 = 2x – 4 (do 2 >1) 
⇔ x = – 5 (loại).
Vậy phương trình đã cho vô nghiệm.
d) Điều kiện: $$ \left\{\begin{array}{l}x-1>0\\ x-3>0\\ 2x+10>0\end{array}\right.\Rightarrow \left\{\begin{array}{l}x>-1\\ x>3\\ x>-5\end{array}\right.\Rightarrow x>3$$
Ta có:  $$ {\mathrm{log}}_{5}(x-1)+{\mathrm{log}}_{5}(x-3)={\mathrm{log}}_{5}(2x+10)$$
  $$ \Leftrightarrow {\mathrm{log}}_{5}\left[(x-1)(x-3)\right]={\mathrm{log}}_{5}(2x+10)$$
$$ \Leftrightarrow {\mathrm{log}}_{5}\left({x}^{2}-4x+3\right)={\mathrm{log}}_{5}(2x+10)$$
  x2 ­– 4x + 3 = 2x + 10 (do 2 >1) 
  x2 – 6x – 7 = 0.
 x = 7 (nhận) hoặc x = –1 (loại)   
Kết hợp điều kiện, vậy tập nghiệm của phương trình là: S = {7}.
e) Điều kiện: $$ \left\{\begin{array}{l}x>0\\ x-3>0\end{array}\right.\Rightarrow \left\{\begin{array}{l}x>0\\ x>3\end{array}\right.\Rightarrow x>3$$
Ta có: log x + log (x – 3) = 1
⇔ log [x(x – 3)] = 1
⇔ log (x2 – 3x)=1
⇔ x2 – 3x – 10 = 0 (do 10 >1)  
⇔ x = 5 (nhận) hoặc x = –2 (loại)
Kết hợp điều kiện, vậy tập nghiệm của phương trình là: S = {5}.
g) Điều kiện: $$ \left\{\begin{array}{l}x>0\\ {\mathrm{log}}_{81}x>0\end{array}\right.\Rightarrow \left\{\begin{array}{l}x>0\\ x>{81}^{0}=1\end{array}\right.\Rightarrow x>1$$
Ta có:  $$ {\mathrm{log}}_{2}\left({\mathrm{log}}_{81}x\right)=-2$$
 $$ \Leftrightarrow {\mathrm{log}}_{81}x={2}^{-2}\Leftrightarrow x={81}^{{2}^{-2}}=3$$ (nhận)
Kết hợp điều kiện, vậy tập nghiệm của phương trình là: S = {3}.

========================================================================

https://khoahoc.vietjack.com/thi-online/giai-sbt-toan-hoc-11-ctst-bai-4-phuong-trinh-bat-phuong-trinh-mu-va-logarit-co-dap-an


\textbf{{QUESTION}}

Giải các bất phương trình sau:
a) $$ {4}^{x}<2\sqrt{2}$$ ;                               
b) $$ {\left(\frac{1}{\sqrt{3}}\right)}^{x-1}\ge \frac{1}{9}$$ ;
c) $$ 5\text{\hspace{0.33em}}.\text{\hspace{0.33em}}{\left(\frac{1}{2}\right)}^{x}<40$$ ;                          
d) 42x < 8x –1;
e) $$ {\left(\frac{1}{5}\right)}^{2-x}\le {\left(\frac{1}{25}\right)}^{x}$$ ;                       
g) 0,25x – 2 > 0,5x + 1.

\textbf{{ANSWER}}

a) Ta có: $$ {4}^{x}<2\sqrt{2}$$
 
 
 
 $$ \Leftrightarrow {2}^{2x}<2\sqrt{2}$$ $$ \Leftrightarrow 2x<{\mathrm{log}}_{2}2\sqrt{2}$$
$$ \Leftrightarrow 2x<\frac{3}{2}$$
$$ \Leftrightarrow x<\frac{3}{4}$$
Vậy tập nghiệm của bất phương trình là:  $$ S=\left(-\infty ;\frac{3}{4}\right)$$
b) Ta có: $$ {\left(\frac{1}{\sqrt{3}}\right)}^{x-1}\ge \frac{1}{9}$$
 
  $$ \Leftrightarrow {3}^{-\frac{1}{2}(x-1)}\ge {3}^{-2}$$ 
$$ \Leftrightarrow -\frac{1}{2}(x-1)\ge -2$$ (do 3 > 1)
$$ \Leftrightarrow x\le 5$$
 
Vậy tập nghiệm của bất phương trình là:  $$ S=(-\infty ;5]$$
c) $$ 5\text{\hspace{0.33em}}.\text{\hspace{0.33em}}{\left(\frac{1}{2}\right)}^{x}<40$$
$$ \Leftrightarrow {2}^{-x}<8$$
$$ \Leftrightarrow {2}^{-x}<{2}^{3}$$
$$ \Leftrightarrow x>-3$$
 
 
 
Vậy tập nghiệm của bất phương trình là: $$ S=\left(-3;\text{\hspace{0.33em}}+\text{\hspace{0.33em}}\infty \right)$$.
d) 42x < 8x – 1  
⇔ 24x < 23x – 3 
⇔ 4x < 3x – 3 (do 2 > 1)
⇔ x < – 3.
Vậy tập nghiệm của bất phương trình là: $$ S=\left(-\infty ;\text{\hspace{0.33em}}-3\right)$$
e) $$ {\left(\frac{1}{5}\right)}^{2-x}\le {\left(\frac{1}{25}\right)}^{x}$$
$$ \Leftrightarrow {5}^{x-2}\le {5}^{-2x}$$
 
 $$ \Leftrightarrow x-2\le -2x$$ (do 5 >1)
$$ \Leftrightarrow 3x\le 2$$$$ \Leftrightarrow x\le \frac{2}{3}$$
  
Vậy tập nghiệm của bất phương trình là: $$ S=\left(-\infty ;\text{\hspace{0.33em}}\frac{2}{3}\right]$$.
g) 0,25x – 2 > 0,5x + 1  
⇔ 0,52(x - 2) > 0,5x + 1
⇔ 2(x –2) < x +1 (do 0 < 0,5 < 1)
⇔ x < 5.
Vậy tập nghiệm của bất phương trình là: $$ S=\left(-\infty ;\text{\hspace{0.33em}}5\right)$$

========================================================================

https://khoahoc.vietjack.com/thi-online/giai-sbt-toan-hoc-11-ctst-bai-4-phuong-trinh-bat-phuong-trinh-mu-va-logarit-co-dap-an


\textbf{{QUESTION}}

Giải các bất phương trình sau:
a) log3(x+4)<2 ;                        
b) log12x≥4  ;
c)  log0,25(x−1)≤−1                     
d)  log5(x2−24x)≥2
e) 2log14(x+1)≥log14(3x+7)
g)  2log3(x+1)≤1+log3(x+7)

\textbf{{ANSWER}}

a) Điều kiện: x > –4
Ta có: log3(x+4)<2$$ {\mathrm{log}}_{3}(x+4)<2$$  ⇔ x + 4 < 9 ⇔ x < 5
Kết hợp điều kiện, vậy tập nghiệm của bất phương trình là: S = (–4; 5).
b) Điều kiện: x > 0
Ta có:  log12x≥4$$ {\mathrm{log}}_{\frac{1}{2}}x\ge 4$$⇔x≤(12)4⇔x≤116$$ \Leftrightarrow x\le {\left(\frac{1}{2}\right)}^{4}\Leftrightarrow x\le \frac{1}{16}$$    
Kết hợp điều kiện, vậy tập nghiệm của bất phương trình là: S=(0;116]$$ S=\left(0;\frac{1}{16}\right]$$ ;
c) Điều kiện: x > 1
Ta có:   log0,25(x−1)≤−1$$ {\mathrm{log}}_{0,25}(x-1)\le -1$$
  ⇔x−1≥(0,25)−1$$ \Leftrightarrow x-1\ge {\left(0,25\right)}^{-1}$$(do 0 < 0, 5 < 1) 
⇔x−1≥4$$ \Leftrightarrow x-1\ge 4$$
⇔x≥5$$ \Leftrightarrow x\ge 5$$                                                     
 
Kết hợp điều kiện, vậy tập nghiệm của bất phương trình là: S=[5; + ∞)$$ S=\left[5;\text{\hspace{0.33em}}+\text{\hspace{0.33em}}\infty \right)$$
d) Điều kiện: x2−24x>0⇔[x<0x>24$$ {x}^{2}-24x>0\Leftrightarrow \left[\begin{array}{l}x<0\\ x>24\end{array}\right.$$
Ta có: log5(x2−24x)≥2$$ {\mathrm{log}}_{5}({x}^{2}-24x)\ge 2$$
⇔x2−24x≥25$$ \Leftrightarrow {x}^{2}-24x\ge 25$$
 
 ⇔x2−24x−25≥0$$ \Leftrightarrow {x}^{2}-24x-25\ge 0$$ (Do 5 > 1)
⇔[x≤−1x≥25$$ \Leftrightarrow \left[\begin{array}{l}x\le -1\\ x\ge 25\end{array}\right.$$
 
Kết hợp điều kiện, vậy tập nghiệm của bất phương trình là: S=(−∞;−1]∪[25;+∞)$$ S=\left(-\infty ;-1\right]\cup \left[25;+\infty \right)$$
e) Điều kiện: {x+1>03x+7>0⇒{x>−1x>−73⇒x>−1$$ \left\{\begin{array}{l}x+1>0\\ 3x+7>0\end{array}\right.\Rightarrow \left\{\begin{array}{l}x>-1\\ x>\frac{-7}{3}\end{array}\right.\Rightarrow x>-1$$
Ta có: 2log14(x+1)≥log14(3x+7)$$ 2{\mathrm{log}}_{\frac{1}{4}}(x+1)\ge {\mathrm{log}}_{\frac{1}{4}}(3x+7)$$
  ⇔log14(x+1)2≥log14(3x+7)$$ \Leftrightarrow {\mathrm{log}}_{\frac{1}{4}}{(x+1)}^{2}\ge {\mathrm{log}}_{\frac{1}{4}}(3x+7)$$
⇔x2+2x+1≤3x+7$$ \Leftrightarrow {x}^{2}+2x+1\le 3x+7$$
  (do cơ số 0<12<1$$ 0<\frac{1}{2}<1$$ )
    ⇔x2−x−6≤0$$ \Leftrightarrow {x}^{2}-x-6\le 0$$⇔−2≤x≤3$$ \Leftrightarrow -2\le x\le 3$$ 
Kết hợp điều kiện, vậy tập nghiệm của phương trình là: S = (−1; 3].
g)  Điều kiện: {x+1>0x+7>0⇒{x>−1x>−7⇒x>−1$$ \left\{\begin{array}{l}x+1>0\\ x+7>0\end{array}\right.\Rightarrow \left\{\begin{array}{l}x>-1\\ x>-7\end{array}\right.\Rightarrow x>-1$$
Ta có: 2log3(x+1)≤1+log3(x+7)$$ 2{\mathrm{log}}_{3}(x+1)\le 1+{\mathrm{log}}_{3}(x+7)$$
  ⇔log3(x+1)2≤log33+log3(x+7)$$ \Leftrightarrow {\mathrm{log}}_{3}{(x+1)}^{2}\le {\mathrm{log}}_{3}3+{\mathrm{log}}_{3}(x+7)$$
⇔log3(x+1)2≤log3(3(x+7))$$ \Leftrightarrow {\mathrm{log}}_{3}{(x+1)}^{2}\le {\mathrm{log}}_{3}\left(3(x+7)\right)$$
 
  ⇔(x+1)2≤3x+21$$ \Leftrightarrow {(x+1)}^{2}\le 3x+21$$ (do cơ số 2>1$$ 2>1$$ )
 
 
  ⇔(x+1)2≤3x+21$$ \Leftrightarrow {(x+1)}^{2}\le 3x+21$$
  ⇔x2+2x+1≤3x+21$$ \Leftrightarrow {x}^{2}+2x+1\le 3x+21$$
⇔x2−x−20≤0$$ \Leftrightarrow {x}^{2}-x-20\le 0$$
⇔−4≤x≤5$$ \Leftrightarrow -4\le x\le 5$$
Kết hợp điều kiện, vậy tập nghiệm của phương trình là: S = (–1; 5].

========================================================================

https://khoahoc.vietjack.com/thi-online/giai-sbt-toan-hoc-11-ctst-bai-4-phuong-trinh-bat-phuong-trinh-mu-va-logarit-co-dap-an


\textbf{{QUESTION}}

Giải các phương trình sau:
a) 4x  –  5.2x + 4 = 0;                       
b) (19)x−2.(13)x−1−27=0$$ {\left(\frac{1}{9}\right)}^{x}-2.{\left(\frac{1}{3}\right)}^{x-1}-27=0$$

\textbf{{ANSWER}}

a) 4x  –  5.2x + 4 = 0;
Đặt  t = 2x  (t > 0).
Khi đó: t2 – 5t + 4 = 0 ⇔[t=4t=1$$ \Leftrightarrow \left[\begin{array}{l}t=4\\ t=1\end{array}\right.$$
  ⇒[2x=42x=1 $$ \Rightarrow \left[\begin{array}{l}{2}^{x}=4\\ {2}^{x}=1\text{\hspace{0.33em}}\end{array}\right.$$⇔[x=log24=2x=log21=0$$ \Leftrightarrow \left[\begin{array}{l}x={\mathrm{log}}_{2}4=2\\ x={\mathrm{log}}_{2}1=0\end{array}\right.$$
Kết hợp với điều kiện, vậy phương trình có nghiệm x = 0 hoặc x = 2.
b) (19)x−2.(13)x−1−27=0$$ {\left(\frac{1}{9}\right)}^{x}-2.{\left(\frac{1}{3}\right)}^{x-1}-27=0$$
⇔(13)2x−2(13)x(13)−1−27=0$$ \Leftrightarrow {\left(\frac{1}{3}\right)}^{2x}-2{\left(\frac{1}{3}\right)}^{x}{\left(\frac{1}{3}\right)}^{-1}-27=0$$
⇔(13)2x−6(13)x−27=0$$ \Leftrightarrow {\left(\frac{1}{3}\right)}^{2x}-6{\left(\frac{1}{3}\right)}^{x}-27=0$$
 
 
Đặt t=(13)x$$ t={\left(\frac{1}{3}\right)}^{x}$$  (t > 0).
Khi đó, ta có: t2−6t+27=0$$ {t}^{2}-6t+27=0$$ ⇔ t = 9 (nhận) hoặc t = –3 (loại)
Do đó (13)x=9$$ {\left(\frac{1}{3}\right)}^{x}=9$$  ⇔ 3–x = 32 ⇔ x = –2.
Vậy nghiệm của phương trình là x = –2.

========================================================================

https://khoahoc.vietjack.com/thi-online/bo-de-on-tap-toan-9-thi-vao-10-nam-2018-co-dap-an/104555


\textbf{{QUESTION}}

Giải hệ phương trình sau: $$ \left\{\begin{array}{l}2x+3y=7\\ x-y=1\end{array}\right.$$

\textbf{{ANSWER}}

$$ \left\{\begin{array}{l}2x+3y=7\\ x-y=1\end{array}\right.\Leftrightarrow \left\{\begin{array}{l}2x+3y=7\\ 3x-3y=3\end{array}\right.\Leftrightarrow \left\{\begin{array}{l}5x=10\\ 2x+3y=7\end{array}\right.\Leftrightarrow \left\{\begin{array}{l}x=2\\ 2.2+3y=7\end{array}\right.\Leftrightarrow \left\{\begin{array}{l}x=2\\ y=1\end{array}\right.$$
Vậy hệ phương trình có nghiệm duy nhất là: (x;y)=(2;1)

========================================================================

https://khoahoc.vietjack.com/thi-online/bo-de-on-tap-toan-9-thi-vao-10-nam-2018-co-dap-an/104555


\textbf{{QUESTION}}

Giải hệ phương trình {x-y=33x-2y=8.$$ \left\{\begin{array}{l}x-y=3\\ 3x-2y=8\end{array}\right..$$

\textbf{{ANSWER}}

{2x-y=33x+2y=8⇔{7x=142x-y=3⇔{x=24-y=3⇔{x=2y=1.$$ \left\{\begin{array}{l}2x-y=3\\ 3x+2y=8\end{array}\right.\Leftrightarrow \left\{\begin{array}{l}7x=14\\ 2x-y=3\end{array}\right.\Leftrightarrow \left\{\begin{array}{l}x=2\\ 4-y=3\end{array}\right.\Leftrightarrow \left\{\begin{array}{l}x=2\\ y=1\end{array}\right..$$
Vậy hệ phương trình có nghiệm duy nhất là (x;y)=(2;1)

========================================================================

https://khoahoc.vietjack.com/thi-online/bo-de-on-tap-toan-9-thi-vao-10-nam-2018-co-dap-an/104555


\textbf{{QUESTION}}

Giải hệ phương trình {3x+y=102x−3y=3$$ \left\{\begin{array}{l}3x+y=10\\ 2x-3y=3\end{array}\right.$$

\textbf{{ANSWER}}

Hệ phương trình
{3x+y=102x−3y=3⇔{y=10−3x2x−3(10−3x)=3⇔{y=10−3x11x−30=3⇔{y=1x=3$$ \left\{\begin{array}{l}3x+y=10\\ 2x-3y=3\end{array}\right.\Leftrightarrow \left\{\begin{array}{l}y=10-3x\\ 2x-3(10-3x)=3\end{array}\right.\Leftrightarrow \left\{\begin{array}{l}y=10-3x\\ 11x-30=3\end{array}\right.\Leftrightarrow \left\{\begin{array}{l}y=1\\ x=3\end{array}\right.$$
Vậy hệ phương trình có nghiệm duy nhất là (x;y)=(3;1)

========================================================================

https://khoahoc.vietjack.com/thi-online/bo-de-on-tap-toan-9-thi-vao-10-nam-2018-co-dap-an/104555


\textbf{{QUESTION}}

Giải hệ phương trình {(x+y)+(x+2y)=−23(x+y)+(x−2y)=1$$ \left\{\begin{array}{l}\left(x+y\right)+\left(x+2y\right)=-2\\ 3\left(x+y\right)+\left(x-2y\right)=1\end{array}\right.$$
Giải hệ phương trình {(x+y)+(x+2y)=−23(x+y)+(x−2y)=1$$ \left\{\begin{array}{l}\left(x+y\right)+\left(x+2y\right)=-2\\ 3\left(x+y\right)+\left(x-2y\right)=1\end{array}\right.$$
{(x+y)+(x+2y)=−23(x+y)+(x−2y)=1
{(x+y)+(x+2y)=−23(x+y)+(x−2y)=1
{(x+y)+(x+2y)=−23(x+y)+(x−2y)=1
{
{
(x+y)+(x+2y)=−23(x+y)+(x−2y)=1
(x+y)+(x+2y)=−23(x+y)+(x−2y)=1
(x+y)+(x+2y)=−2
(x+y)+(x+2y)=−2
(x+y)
(
x+y
x
+
y
)
+
(x+2y)
(
x+2y
x
+
2
y
)
=
−
2
3(x+y)+(x−2y)=1
3(x+y)+(x−2y)=1
3
(x+y)
(
x+y
x
+
y
)
+
(x−2y)
(
x−2y
x
−
2
y
)
=
1

\textbf{{ANSWER}}

{(x+y)+(x+2y)=−23(x+y)+(x−2y)=1⇔{2x+3y=−24x+y=1⇔{2x+3y=−2−12x−3y=−3⇔{2x+3y=−2−10x=−5⇔{2.12+3y=−2x=12⇔{x=12y=−1$$ \left\{\begin{array}{l}\left(x+y\right)+\left(x+2y\right)=-2\\ 3\left(x+y\right)+\left(x-2y\right)=1\end{array}\right.\Leftrightarrow \left\{\begin{array}{l}2x+3y=-2\\ 4x+y=1\end{array}\right.\Leftrightarrow \left\{\begin{array}{l}2x+3y=-2\\ -12x-3y=-3\end{array}\Leftrightarrow \left\{\begin{array}{l}2x+3y=-2\\ -10x=-5\end{array}\Leftrightarrow \left\{\begin{array}{l}2.\frac{1}{2}+3y=-2\\ x=\frac{1}{2}\end{array}\right.\right.\right.\phantom{\rule{0ex}{0ex}}\Leftrightarrow \left\{\begin{array}{l}x=\frac{1}{2}\\ y=-1\end{array}\right.$$
Vậy hệ phương trình có nghiệm (x;y)=(12;-1$$ \frac{1}{2};-1$$)

========================================================================

https://khoahoc.vietjack.com/thi-online/bo-de-on-tap-toan-9-thi-vao-10-nam-2018-co-dap-an/104555


\textbf{{QUESTION}}

Cho hệ phương trình: {mx−y=nnx+my=1$$ \left\{\begin{array}{l}mx-y=n\\ nx+my=1\end{array}\right.$$ (I) (m, n là tham số)
a) Giải hệ phương trình khi m=−12$$ m=-\frac{1}{2}$$; n=13$$ n=\frac{1}{3}$$.

\textbf{{ANSWER}}

Thay m=−12$$ m=-\frac{1}{2}$$, n=13$$ n=\frac{1}{3}$$ vào hệ phương trình (I) ta được: {−12x−y=1313x−12y=1$$ \left\{\begin{array}{l}-\frac{1}{2}x-y=\frac{1}{3}\\ \frac{1}{3}x-\frac{1}{2}y=1\end{array}\right.$$
Vậy nghiệm của hệ phương trình là (x; y)=(107;  −2221)$$ (x;\text{\hspace{0.17em}}y)=\left(\frac{10}{7};\text{\hspace{0.17em}\hspace{0.17em}}-\frac{22}{21}\right)$$

========================================================================

https://khoahoc.vietjack.com/thi-online/bai-2-tap-hop-cac-so-tu-nhien


\textbf{{QUESTION}}

Điền vào chỗ trống để ba số ở mỗi dòng là ba số tự nhiên liên tiếp tăng dần:
28, …, …
…, 100, …

\textbf{{ANSWER}}

Để có 3 số tự nhiên liên tiếp tăng dần, ta phải :
- Điền vào chỗ trống 2 số liền sau của 28 là 29 ; 30 ( 28 ; 29 ; 30 )
- Điền vào chỗ trống số liền trước và liền sau của 100 là 99 ; 101 ( 99 ; 100 ; 101 )

========================================================================

https://khoahoc.vietjack.com/thi-online/bai-2-tap-hop-cac-so-tu-nhien


\textbf{{QUESTION}}

Viết số tự nhiên liền sau mỗi số:
17;         99 ;         a (với a ∈ N)

\textbf{{ANSWER}}

Số tự nhiên liền sau của 17 là 18
Số tự nhiên liền sau của 99 là 100
Số tự nhiên liền sau của a (với a ∈ N) là a + 1.

========================================================================

https://khoahoc.vietjack.com/thi-online/bai-2-tap-hop-cac-so-tu-nhien


\textbf{{QUESTION}}

Viết số tự nhiên liền trước mỗi số:
35 ;         1000 ;         b (với b ∈ N*)

\textbf{{ANSWER}}

Số tự nhiên liền trước của 35 là 34.
Số tự nhiên liền trước của 1000 là 999.
Số tự nhiên liền trước của b (b ∈ N*) là b – 1.
Chú ý b ∈ N* nên b ≥ 1, lúc đó b mới có số tự nhiên liền trước. Số 0 không có số tự nhiên liền trước.

========================================================================

https://khoahoc.vietjack.com/thi-online/bai-2-tap-hop-cac-so-tu-nhien


\textbf{{QUESTION}}

Viết tập hợp sau bằng cách liệt kê các phần tử:
A = {x ∈ N | 12 < x < 16}

\textbf{{ANSWER}}

A = {x ∈ N | 12 < x < 16} là tập hợp các số tự nhiên lớn hơn 12 và nhỏ hơn 16.
Các số đó là 13 ; 14 ; 15.
Do đó ta viết A = { 13 ; 14 ; 15}.

========================================================================

https://khoahoc.vietjack.com/thi-online/bai-2-tap-hop-cac-so-tu-nhien


\textbf{{QUESTION}}

Viết tập hợp sau bằng cách liệt kê các phần tử:
B = {x ∈ N* | x < 5}

\textbf{{ANSWER}}

B = {x ∈ N* | x < 5} là tập hợp các số tự nhiên khác 0 và nhỏ hơn 5.
Các số đó là 1 ; 2 ; 3 ; 4.
Do đó ta viết B = {1 ; 2 ; 3 ; 4 }

========================================================================

https://khoahoc.vietjack.com/thi-online/100-cau-trac-nghiem-phep-doi-hinh-co-ban/57632


\textbf{{QUESTION}}

Cho tam giác ABC đều, phép quay tâm A biến B thành C là :
A.$$ {Q}_{\left(A;{120}^{o}\right)}$$
B.$$ {Q}_{\left(A;-{60}^{o}\right)}$$
C.$$ {Q}_{\left(A;{60}^{o}\right)}$$
D.$$ {Q}_{\left(A;{30}^{o}\right)}$$

\textbf{{ANSWER}}

Đáp án C

========================================================================

https://khoahoc.vietjack.com/thi-online/100-cau-trac-nghiem-phep-doi-hinh-co-ban/57632


\textbf{{QUESTION}}

Cho hai đường thẳng d và d’song song. Có bao nhiêu phép đối xứng trục biến đường thẳng (d) thành đường thẳng (d’) :
A. Có duy nhất một phép đối xứng trục.
B. Có 2 phép đối xứng trục.
C. Có vô số phép đối xứng trục.
D. Không có phép đối xứng trục nào

\textbf{{ANSWER}}

Đáp án A
Khi 2 đường  thẳng d và d' song song thì có 1 phép đối xứng  trục biến d thành d'.
Trục đối xứng là đường thẳng song song và cách đều 2 đường thẳng d và d'.

========================================================================

https://khoahoc.vietjack.com/thi-online/100-cau-trac-nghiem-phep-doi-hinh-co-ban/57632


\textbf{{QUESTION}}

Cho hai đường thẳng d và d’cắt nhau. Có bao nhiêu phép đối xứng trục biến đường thẳng (d) thành đường thẳng (d’) :
A. Có duy nhất một phép đối xứng trục
B. Có 2 phép đối xứng trục.
C. Có vô số phép đối xứng trục
D. Không có phép đối xứng trục nào

\textbf{{ANSWER}}

Đáp án B
Khi 2 đường thẳng d và d' cắt nhau thì có 2 phép đối xứng trục biến đường thẳng này thành đường thằng kia.
Trục đối xứng là 2 đường phân giác của 2 góc  tạo bởi 2 đường thẳng đã cho.

========================================================================

https://khoahoc.vietjack.com/thi-online/100-cau-trac-nghiem-phep-doi-hinh-co-ban/57632


\textbf{{QUESTION}}

Hình ngũ giác đều có bao nhiêu trục đối xứng ?
A. Có 5 trục đối xứng.
B. Có 1 trục đối xứng
C. Có vô số trục đối xứng.
D. Không có trục đối xứng nào.

\textbf{{ANSWER}}

Đáp án A
Trục đối xứng là đường thẳng nối 1 đỉnh với trung điểm cạnh đối diện.

========================================================================

https://khoahoc.vietjack.com/thi-online/100-cau-trac-nghiem-phep-doi-hinh-co-ban/57632


\textbf{{QUESTION}}

Cho điểm M(5;2), M’ là ảnh của M qua phép đối xứng tâm I(1;–2) . Tọa độ điểm M’ là:
A. M’(3;6)
B.M’(- 3;–6)
C. M’(–4;–4)
D. M’(−32;−3)

\textbf{{ANSWER}}

Đáp án B
{xM+xM'=2xIyM+yM'=2yI⇒{xM'=2xI-xM= 2.1- 5 = -3 yM'=2yI- yM= 2.(-2)-2 = - 6$$ \left\{\begin{array}{l}{x}_{M}+{x}_{M\text{'}}=2{x}_{I}\\ {y}_{M}+{y}_{M\text{'}}=2{y}_{I}\end{array}\right.\Rightarrow \left\{\begin{array}{l}{x}_{M\text{'}}=2{x}_{I}-{x}_{M}=\quad 2.1-\quad 5\quad =\quad -3\\ \quad {y}_{M\text{'}}=2{y}_{I}-\quad {y}_{M}=\quad 2.(-2)-2\quad =\quad -\quad 6\end{array}\right.$$
=> M’(–3;–6)

========================================================================

https://khoahoc.vietjack.com/thi-online/10-bai-stap-doi-don-vi-giua-do-va-radian-co-loi-giai


\textbf{{QUESTION}}

Số đo theo đơn vị rađian của góc 55° là
A. $$ \frac{11\pi }{36}$$;
B. $$ \frac{13\pi }{36}$$;
C. $$ \frac{-11\pi }{36}$$;

\textbf{{ANSWER}}

Hướng dẫn giải
Đáp án đúng là: A
Ta có: 55° = 55 . $$ \frac{\pi }{180}$$ = $$ \frac{11\pi }{36}$$.

========================================================================

https://khoahoc.vietjack.com/thi-online/10-bai-stap-doi-don-vi-giua-do-va-radian-co-loi-giai


\textbf{{QUESTION}}

Số đo góc π5$$ \frac{\pi }{5}$$ = …°. Giá trị thích hợp để điền vào chỗ trống là
A. 40;
B. 45;
C. 36;

\textbf{{ANSWER}}

Hướng dẫn giải
Đáp án đúng là: C
Ta có: π5$$ \frac{\pi }{5}$$ = π5$$ \frac{\pi }{5}$$ . (180π)°$$ \left(\frac{180}{\pi }\right)\begin{array}{c}°\\ \end{array}$$= 36°.

========================================================================

https://khoahoc.vietjack.com/thi-online/10-bai-stap-doi-don-vi-giua-do-va-radian-co-loi-giai


\textbf{{QUESTION}}

Đáp án nào sau đây là đúng?     
A. π9$$ \frac{\pi }{9}$$ = 25°;
B. π9$$ \frac{\pi }{9}$$ = 15°;
C. π9$$ \frac{\pi }{9}$$ = 18°;

\textbf{{ANSWER}}

Hướng dẫn giải
Đáp án đúng là: D
Ta có: π9$$ \frac{\pi }{9}$$ = π9$$ \frac{\pi }{9}$$. (180π)°$$ \left(\frac{180}{\pi }\right)\begin{array}{c}°\\ \end{array}$$= 20°.

========================================================================

https://khoahoc.vietjack.com/thi-online/10-bai-stap-doi-don-vi-giua-do-va-radian-co-loi-giai


\textbf{{QUESTION}}

Đổi số đo góc 105° sang rađian ta được
A. 7π12$$ \frac{7\pi }{12}$$;
B. 9π12$$ \frac{9\pi }{12}$$;
C. 5π8$$ \frac{5\pi }{8}$$;

\textbf{{ANSWER}}

Hướng dẫn giải
Đáp án đúng là: A
Ta có: 105° = 105 . π180$$ \frac{\pi }{180}$$ = 7π12$$ \frac{7\pi }{12}$$.

========================================================================

https://khoahoc.vietjack.com/thi-online/10-bai-stap-doi-don-vi-giua-do-va-radian-co-loi-giai


\textbf{{QUESTION}}

Cho biết 22°30' = abπ$$ \frac{a}{b}\pi $$ (với ab$$ \frac{a}{b}$$ là phân số tối giản). Giá trị a + b là
A. 7;
B. 9;
C. 8;

\textbf{{ANSWER}}

Hướng dẫn giải
Đáp án đúng là: B
Ta có: 22°30' = 22,5° = 22,5 . π180$$ \frac{\pi }{180}$$ = π8$$ \frac{\pi }{8}$$, suy ra a = 1, b = 8.
Vậy a + b = 1 + 8 = 9.

========================================================================

https://khoahoc.vietjack.com/thi-online/tuyen-chon-de-thi-thu-thptqg-mon-toan-cuc-hay-chon-loc/50383


\textbf{{QUESTION}}

Cho hàm số $$ \mathrm{y}=\sqrt{2{\mathrm{x}}^{2}+1}$$. Mệnh đề nào sau đây đúng?
A. Hàm số nghịch biến trên khoảng (-1;1)
B. Hàm số đồng biến trên khoảng $$ \left(0;+\infty \right)$$
C. Hàm số đồng biến trên khoảng $$ \left(-\infty ;0\right)$$
D. Hàm số nghịch biến trên khoảng $$ \left(0;+\infty \right)$$

\textbf{{ANSWER}}

Đáp án B
$$ \mathrm{y}\text{'}=\frac{2\mathrm{x}}{\sqrt{2{\mathrm{x}}^{2}+1}}>0\Leftrightarrow \mathrm{x}>0$$
Vậy hàm số đồng biến trên khoảng $$ \left(0;+\infty \right)$$.

========================================================================

https://khoahoc.vietjack.com/thi-online/15-cau-trac-nghiem-toan-7-ket-noi-tri-thuc-bai-20-ti-le-thuc-co-dap-an/111032


\textbf{{QUESTION}}

Giá trị của x trong tỉ lệ thức $$ \frac{3}{x}$$ = $$ \frac{9}{12}$$ là:
A. 3
B. 4
C. 5
D. 6

\textbf{{ANSWER}}

Đáp án đúng là: B
Áp dụng tính chất của tỉ lệ thức ta có:
$$ \frac{3}{x}$$ = $$ \frac{9}{12}$$ ⇔ 3.12 = 9x
⇒ 9x = 36
⇒ x = 36 : 9
⇒ x = 4
Vậy x = 4. Chọn đáp án B.

========================================================================

https://khoahoc.vietjack.com/thi-online/15-cau-trac-nghiem-toan-7-ket-noi-tri-thuc-bai-20-ti-le-thuc-co-dap-an/111032


\textbf{{QUESTION}}

A. $$ \frac{-21}{-15}=\frac{14}{10}$$;

\textbf{{ANSWER}}

Đáp án đúng là: C
Ta có: (– 21).10 = (– 15).14
Khi đó ta có các tỉ lệ thức là: $$ \frac{-21}{-15}=\frac{14}{10};\text{\hspace{0.17em}\hspace{0.17em}}\frac{-21}{14}=\frac{-15}{10};\text{\hspace{0.17em}\hspace{0.17em}}\frac{10}{-15}=\frac{14}{-21};\text{\hspace{0.17em}\hspace{0.17em}\hspace{0.17em}}\frac{10}{14}=\frac{-15}{-21}$$.
Vậy $$ \frac{-21}{-15}=\frac{14}{10}$$
Chọn đáp án C.

========================================================================

https://khoahoc.vietjack.com/thi-online/15-cau-trac-nghiem-toan-7-ket-noi-tri-thuc-bai-20-ti-le-thuc-co-dap-an/111032


\textbf{{QUESTION}}

Cho x5$$ \frac{x}{5}$$ = y7$$ \frac{y}{7}$$ và x + y = 24. Tính giá trị biểu thức x – y ?
A. -2
B. 2
C. 4
D. -4

\textbf{{ANSWER}}

Đáp án đúng là: D
Áp dụng tính chất của tỉ lệ thức ta có: X5$$ \frac{X}{5}$$ = y7$$ \frac{y}{7}$$ ⇒ xy$$ \frac{x}{y}$$ = 57$$ \frac{5}{7}$$.
Đặt x = 5k, y = 7k
Do x + y = 24 nên ta có 5k + 7k = 12k = 24
⇔ k = 24 : 12 = 2
⇒ x = 5.2 =10
⇒ y = 24 – 10 = 14
Như vậy x – y = 10 – 14 = –4
Vậy đáp án đúng là D.

========================================================================

https://khoahoc.vietjack.com/thi-online/15-cau-trac-nghiem-toan-7-ket-noi-tri-thuc-bai-20-ti-le-thuc-co-dap-an/111032


\textbf{{QUESTION}}

Có bao nhiêu giá trị x thỏa mãn tỉ lệ thức 36x$$ \frac{36}{x}$$ = x9$$ \frac{x}{9}$$.
A. 1
B. 2
C. 0
D. 3

\textbf{{ANSWER}}

Đáp án đúng là: B
Áp dụng tính chất của tỉ lệ thức ta có:
36x$$ \frac{36}{x}$$ = x9$$ \frac{x}{9}$$ ⇔ 36.9 = x.x
⇒ x2 = 324
⇒ x = 18 và x = –18
Vậy có 2 giá trị x thoả mãn yêu cầu. Chọn đáp án B.

========================================================================

https://khoahoc.vietjack.com/thi-online/15-cau-trac-nghiem-toan-7-ket-noi-tri-thuc-bai-20-ti-le-thuc-co-dap-an/111032


\textbf{{QUESTION}}

Để cày hết một cánh đồng trong 12 ngày phải sử dụng 18 máy cày. Hỏi muốn cày hết cánh đồng đó trong 6 ngày thì phải sử dụng bao nhiêu máy cày (biết năng suất của các máy cày là như nhau)?

\textbf{{ANSWER}}

Đáp án đúng là: D
Gọi số máy cày cần tìm là x (máy) (x ∈ ℕ)
Ta có: 12.18 = 6.x
⇔ x = (12 . 18) : 6 = 36.
Vậy để cày hết cánh đồng đó trong 6 ngày thì cần 36 máy cày.

========================================================================

https://khoahoc.vietjack.com/thi-online/bo-5-de-thi-cuoi-ki-2-toan-11-canh-dieu-cau-truc-moi-co-dap-an/164316


\textbf{{QUESTION}}

A. TRẮC NGHIỆM NHIỀU PHƯƠNG ÁN LỰA CHỌN. Thí sinh trả lời từ câu 1 đến câu 12.
Mỗi câu hỏi thí sinh chỉ chọn một phương án.
Cho số thực a >0. Biểu thức $$ \mathrm{P}={\mathrm{a}}^{\frac{1}{3}}.\sqrt[3]{{\mathrm{a}}^{5}}$$ bằng 
A. $$ {\mathrm{a}}^{2}$$
B. $$ {\mathrm{a}}^{\frac{4}{3}}$$
C. $$ {\mathrm{a}}^{\frac{14}{15}}$$
D. $$ {\mathrm{a}}^{5}$$

\textbf{{ANSWER}}

Đáp án đúng là: A
$$ \mathrm{P}={\mathrm{a}}^{\frac{1}{3}}.\sqrt[3]{{\mathrm{a}}^{5}}={\mathrm{a}}^{\frac{1}{3}}.{\mathrm{a}}^{\frac{5}{3}}={\mathrm{a}}^{2}$$.

========================================================================

https://khoahoc.vietjack.com/thi-online/bo-5-de-thi-cuoi-ki-2-toan-11-canh-dieu-cau-truc-moi-co-dap-an/164316


\textbf{{QUESTION}}

A. $$ {\mathrm{log}}_{\mathrm{a}}\frac{\mathrm{x}}{\mathrm{y}}={\mathrm{log}}_{\mathrm{a}}\mathrm{x}-{\mathrm{log}}_{\mathrm{a}}\mathrm{y}$$
B. $$ {\mathrm{log}}_{\mathrm{a}}\left(\mathrm{x}.\mathrm{y}\right)={\mathrm{ylog}}_{\mathrm{a}}\mathrm{x}$$
C. $$ {\mathrm{log}}_{\mathrm{a}}\mathrm{a}=0$$
D. $$ {\mathrm{log}}_{\mathrm{a}}\mathrm{x}=1$$

\textbf{{ANSWER}}

Đáp án đúng là: A
$$ {\mathrm{log}}_{\mathrm{a}}\frac{\mathrm{x}}{\mathrm{y}}={\mathrm{log}}_{\mathrm{a}}\mathrm{x}-{\mathrm{log}}_{\mathrm{a}}\mathrm{y}$$.

========================================================================

https://khoahoc.vietjack.com/thi-online/bo-5-de-thi-cuoi-ki-2-toan-11-canh-dieu-cau-truc-moi-co-dap-an/164316


\textbf{{QUESTION}}

Hàm số y=log3x$$ \mathrm{y}={\mathrm{log}}_{3}\mathrm{x}$$. Mệnh đề nào dưới đây sai?
y=log3x
y=log3x
y
=
log3
log
3
x
A. Hàm số y=log3x$$ \mathrm{y}={\mathrm{log}}_{3}\mathrm{x}$$ có tập xác định là ℝ$$ \mathrm{\mathbb{R} }$$.
A.
 Hàm số y=log3x$$ \mathrm{y}={\mathrm{log}}_{3}\mathrm{x}$$ có tập xác định là ℝ$$ \mathrm{\mathbb{R} }$$.
y=log3x
y=log3x
y
=
log3
log
3
x
 
ℝ
ℝ
ℝ
     
     
B. Hàm số y=log3x$$ \mathrm{y}={\mathrm{log}}_{3}\mathrm{x}$$ đồng biến trên khoảng (0;+∞)$$ \left(0;+\mathrm{\infty }\right)$$.
y=log3x
y=log3x
y
=
log3
log
3
x
 
(0;+∞)
(0;+∞)
(0;+∞)
(
0;+∞
0
;
+
∞
)
     
     
C. Hàm số y=log3x$$ \mathrm{y}={\mathrm{log}}_{3}\mathrm{x}$$ có tập xác định là (0;+∞)$$ \left(0;+\mathrm{\infty }\right)$$.
y=log3x
y=log3x
y
=
log3
log
3
x
 
(0;+∞)
(0;+∞)
(0;+∞)
(
0;+∞
0
;
+
∞
)
     
     
D. Hàm số y=log3x$$ \mathrm{y}={\mathrm{log}}_{3}\mathrm{x}$$ luôn đi qua điểm (1;0)$$ \left(1;0\right)$$.
y=log3x
y=log3x
y
=
log3
log
3
x
 
(1;0)
(1;0)
(1;0)
(
1;0
1
;
0
)

\textbf{{ANSWER}}

Đáp án đúng là: A
Hàm số y=log3x$$ \mathrm{y}={\mathrm{log}}_{3}\mathrm{x}$$ có tập xác định là (0;+∞)$$ \left(0;+\mathrm{\infty }\right)$$.

========================================================================

https://khoahoc.vietjack.com/thi-online/bo-5-de-thi-cuoi-ki-2-toan-11-canh-dieu-cau-truc-moi-co-dap-an/164316


\textbf{{QUESTION}}

Mệnh đề nào dưới đây đúng?
     
     
A. Nếu một đường thẳng vuông góc với một trong hai đường thẳng song song thì cũng vuông góc với đường thẳng còn lại.
     
     
B. Hai đường thẳng vuông góc với nhau thì luôn cắt nhau.
     
     
C. Góc giữa hai đường thẳng bằng góc giữa hai vectơ chỉ phương của chúng.
     
     
D. Hai đường thẳng cùng vuông góc với một đường thẳng thì song song với nhau.

\textbf{{ANSWER}}

Đáp án đúng là: A
Nếu một đường thẳng vuông góc với một trong hai đường thẳng song song thì cũng vuông góc với đường thẳng còn lại.

========================================================================

https://khoahoc.vietjack.com/thi-online/giai-sbt-toan-10-bai-tap-cuoi-chuong-2-co-dap-an-ctst


\textbf{{QUESTION}}

Bạn Danh để dành được 900 nghìn đồng. Trong một đợt ủng hộ trẻ em mồ côi, bạn Danh đã lấy ra x tờ tiền loại 50 nghìn đồng, y tờ tiền loại 100 nghìn đồng để trao tặng. Một bất phương trình mô tả điều kiện ràng buộc đối với x, y là: 
A. 50x + 100y ≤ 900; 
B. 50x + 100y ≥ 900; 
C. 100x + 50y ≤ 900; 
D. x + y = 900.

\textbf{{ANSWER}}

Đáp án đúng là: A
Ta có x tờ tiền loại 50 nghìn đồng thì có giá trị là 50x (nghìn đồng). 
y tờ tiền loại 100 nghìn đồng thì có giá trị là 100y (nghìn đồng). 
Tổng số tiền bạn Danh trao tặng là: 50x + 100y (nghìn đồng). 
Mà bạn Danh có 900 nghìn đồng nên 50x + 100y ≤ 900. 
Vậy bất phương trình mô tả điều kiện ràng buộc đối với x, y là 50x + 100y ≤ 900.

========================================================================

https://khoahoc.vietjack.com/thi-online/giai-sbt-toan-10-bai-tap-cuoi-chuong-2-co-dap-an-ctst


\textbf{{QUESTION}}

Trong các bất phương trình sau, bất phương trình nào không phải là bất phương trình bậc nhất hai ẩn? 
A. 2x – 3y – 2022 ≤ 0; 
B. 5x + y ≥ 2x + 11; 
C. x + 2025 > 0; 
D. xy+1>0$$ \frac{x}{y}+1>0$$.

\textbf{{ANSWER}}

Đáp án đúng là: D
Xét đáp án A, 2x – 3y – 2022 ≤ 0 ⇔ 2x – 3y ≤ 2022, đây là một bất phương trình bậc nhất hai ẩn có dạng ax + by ≤ c (a, b, c là các số thực, a, b không đồng thời bằng 0). 
Xét đáp án B, 5x + y ≥ 2x + 11 ⇔ 3x + y ≥ 11, đây là một bất phương trình bậc nhất hai ẩn có dạng ax + by ≥ c (a, b, c là các số thực, a, b không đồng thời bằng 0).
Xét đáp án C, x + 2025 > 0 ⇔ x + 0y > – 2025, đây là một bất phương trình bậc nhất hai ẩn có dạng ax + by > c (a, b, c là các số thực, a, b không đồng thời bằng 0).
Xét đáp án D, xy+1>0$$ \frac{x}{y}+1>0$$, đây không phải là một bất phương trình bậc nhất hai ẩn vì nó không có một trong các dạng ax + by < c, ax + by > c, ax + by ≤ c, ax + by ≥ c với a, b, c là các số thực, a, b không đồng thời bằng 0.

========================================================================

https://khoahoc.vietjack.com/thi-online/12-bai-tap-giai-he-phuong-trinh-bac-nhat-hai-an-bang-phuong-phap-cong-dai-so-co-loi-giai


\textbf{{QUESTION}}

Bạn hãy đọc đoạn văn 1 trên và trả lời câu hỏi.
Phương trình thích hợp điền vào chỗ trống (1) là:
A. 4x + 4y = 12.
B. 4y = 12.
C. 4x = 12.
D. 4y = 10.

\textbf{{ANSWER}}

Đáp án đúng là: C
Cộng từng vế của cả hai phương trình ta được hệ phương trình:
3x – 2y + x + 2y = 11 + 1 hay 4x = 12.

========================================================================

https://khoahoc.vietjack.com/thi-online/12-bai-tap-giai-he-phuong-trinh-bac-nhat-hai-an-bang-phuong-phap-cong-dai-so-co-loi-giai


\textbf{{QUESTION}}

Bạn hãy đọc đoạn văn 1 trên và trả lời câu hỏi.
Mệnh đề thích hợp điền vào chỗ trống (2) là:
A. x = 2.
B. x = 2,5.
C. x = 3.
D. y = 3.

\textbf{{ANSWER}}

Đáp án đúng là: C
Giải phương trình 4x = 12, ta được x = 3.

========================================================================

https://khoahoc.vietjack.com/thi-online/12-bai-tap-giai-he-phuong-trinh-bac-nhat-hai-an-bang-phuong-phap-cong-dai-so-co-loi-giai


\textbf{{QUESTION}}

Bạn hãy đọc đoạn văn 1 trên và trả lời câu hỏi.
Cặp số thích hợp điền vào chỗ trống (3) là:
A. (3; −1).
B. (1; 3).
C. (−1; 3).
D. (2; −1).

\textbf{{ANSWER}}

Đáp án đúng là: A
Thế x = 3 vào phương trình x + 2y = 1, ta được 2y = −2 hay y = −1.
Vậy cặp nghiệm của hệ phương trình là (3; −1).

========================================================================

https://khoahoc.vietjack.com/thi-online/12-bai-tap-giai-he-phuong-trinh-bac-nhat-hai-an-bang-phuong-phap-cong-dai-so-co-loi-giai


\textbf{{QUESTION}}

Hệ phương trình {3x+2y=55x+2y=7$\left\{ \begin{array}{l}3x + 2y = 5\\5x + 2y = 7{\rm{ }}\end{array} \right.$ có nghiệm là
A. (1; 1).
B. (1; −1).
C. (−1; 1).
D. (−1; −1).

\textbf{{ANSWER}}

Đáp án đúng là: A
Trừ theo vế hai phương trình của hệ, ta được:
3x + 2y – (5x + 2y) = 5 – 7 hay −2x = −2, suy ra x = 1.
Thay x = 1 vaod phương trình thứ nhất của hệ, ta được y = 1.
Vậy nghiệm của hệ phương trình là (1; 1).

========================================================================

https://khoahoc.vietjack.com/thi-online/12-bai-tap-giai-he-phuong-trinh-bac-nhat-hai-an-bang-phuong-phap-cong-dai-so-co-loi-giai


\textbf{{QUESTION}}

Hệ phương trình {4x−3y=0x+3y=9$\left\{ \begin{array}{l}4x - 3y = 0\\x + 3y = 9{\rm{ }}\end{array} \right.$ có nghiệm là
A. (125;−95)$\left( {\frac{{12}}{5}; - \frac{9}{5}} \right)$.
B. (95;−125)$\left( {\frac{9}{5}; - \frac{{12}}{5}} \right)$.
C. (125;95)$\left( {\frac{{12}}{5};\frac{9}{5}} \right)$.
D. (95;125)$\left( {\frac{9}{5};\frac{{12}}{5}} \right)$.

\textbf{{ANSWER}}

Đáp án đúng là: D
Cộng theo vế hai phương trình của hệ, ta được: 4x – 3y + x + 3y = 9 hay 5x = 9, do đó x = 95.$\frac{9}{5}.$
Thay x = 95$\frac{9}{5}$ vào phương trình x + 3y = 9, ta được 95$\frac{9}{5}$ + 3y = 9, suy ra y = 125$\frac{{12}}{5}$.
Vậy nghiệm của hệ phương trình là (95;125)$\left( {\frac{9}{5};\frac{{12}}{5}} \right)$.

========================================================================

https://khoahoc.vietjack.com/thi-online/giai-sgk-toan-8-canh-dieu-bai-6-phep-cong-phep-tru-phan-thuc-dai-so-co-dap-an


\textbf{{QUESTION}}

Ở lớp 6, ta đã biết cách cộng, trừ các phân số. Làm thế nào để cộng, trừ được các phân thức đại số?

\textbf{{ANSWER}}

Sau khi học xong bài này ta sẽ giải quyết bài toán này như sau:
Cộng, trừ được các phân thức đại số, ta thực hiện tương tự như phép cộng, phép trừ các phân số.
• Đối với các phân thức đại số có cùng mẫu thì ta thực hiện cộng (trừ) các tử và giữ nguyên mẫu.
• Đối với các phân thức đại số khác mẫu thì ta quy đồng mẫu thức các phân thức sau đó thực hiện cộng (trừ) các tử và giữ nguyên mẫu.

========================================================================

https://khoahoc.vietjack.com/thi-online/giai-sgk-toan-8-canh-dieu-bai-6-phep-cong-phep-tru-phan-thuc-dai-so-co-dap-an


\textbf{{QUESTION}}

Thực hiện phép tính: − 35+235$$ \frac{-\text{\hspace{0.17em}}3}{5}+\frac{23}{5}$$.

\textbf{{ANSWER}}

Ta có $$ \frac{-\text{\hspace{0.17em}}3}{5}+\frac{23}{5}=\frac{-\text{\hspace{0.17em}}3+33}{5}=\frac{-\text{\hspace{0.17em}}30}{5}=-6$$.

========================================================================

https://khoahoc.vietjack.com/thi-online/giai-sgk-toan-8-canh-dieu-bai-6-phep-cong-phep-tru-phan-thuc-dai-so-co-dap-an


\textbf{{QUESTION}}

Thực hiện phép tính: x−2yx2+xy+x+2yx2+xy$$ \frac{x-2y}{{x}^{2}+xy}+\frac{x+2y}{{x}^{2}+xy}$$.

\textbf{{ANSWER}}

x−2yx2+xy+x+2yx2+xy=x−2y+x+2yx2+xy$$ \frac{x-2y}{{x}^{2}+xy}+\frac{x+2y}{{x}^{2}+xy}=\frac{x-2y+x+2y}{{x}^{2}+xy}$$=(x+x)+(2y−2y)x(x+y)=2xx(x+y)=2x+y$$ =\frac{(x+x)+(2y-2y)}{x(x+y)}=\frac{2x}{x(x+y)}=\frac{2}{x+y}$$

========================================================================

https://khoahoc.vietjack.com/thi-online/giai-sgk-toan-8-canh-dieu-bai-6-phep-cong-phep-tru-phan-thuc-dai-so-co-dap-an


\textbf{{QUESTION}}

Cho hai phân thức: 1x+1;  1x−1.$$ \frac{1}{x+1};\text{\hspace{0.17em}\hspace{0.17em}}\frac{1}{x-1}.$$
a) Quy đồng mẫu thức hai phân thức trên.

\textbf{{ANSWER}}

a) MTC: (x + 1)(x – 1).
Quy đồng mẫu thức hai phân thức đã cho, ta được:
1x+1=x−1(x+1)(x−1);  1x−1=x+1(x−1)(x+1)$$ \frac{1}{x+1}=\frac{x-1}{(x+1)(x-1)};\text{\hspace{0.17em}\hspace{0.17em}}\frac{1}{x-1}=\frac{x+1}{(x-1)(x+1)}$$.

========================================================================

https://khoahoc.vietjack.com/thi-online/giai-sgk-toan-8-canh-dieu-bai-6-phep-cong-phep-tru-phan-thuc-dai-so-co-dap-an


\textbf{{QUESTION}}

b) Từ câu a, hãy thực hiện phép tính: 1x+1+1x−1.$$ \frac{1}{x+1}+\frac{1}{x-1}.$$

\textbf{{ANSWER}}

b) Ta có 1x+1+1x−1=x−1(x+1)(x−1)+x+1(x−1)(x+1)$$ \frac{1}{x+1}+\frac{1}{x-1}=\frac{x-1}{(x+1)(x-1)}+\frac{x+1}{(x-1)(x+1)}$$
=x−1+x+1(x+1)(x−1)=2x(x+1)(x−1)$$ =\frac{x-1+x+1}{(x+1)(x-1)}=\frac{2x}{(x+1)(x-1)}$$.

========================================================================

https://khoahoc.vietjack.com/thi-online/10-cau-trac-nghiem-phep-thu-va-bien-co-co-dap-an-van-dung


\textbf{{QUESTION}}

Một con xúc sắc cân đối, đồng chất được gieo 6 lần. Xác suất để được một số lớn hơn hay bằng 5 xuất hiện ít nhất 5 lần là:
A. $$ \frac{31}{23328}$$.
B. $$ \frac{41}{23328}$$.
C. $$ \frac{51}{23328}$$.
D. $$ \frac{21}{23328}$$.

\textbf{{ANSWER}}

Đáp án cần chọn là: A
Ta có: n(Ω)=$$ {6}^{6}$$
TH1: Số bằng 5 xuất hiện đúng 5 lần ⇒ có 5.6=30 khả năng xảy ra.
TH2: Số bằng 5 xuất hiện đúng 6 lần ⇒ có 1 khả năng xảy ra.
TH3: Số bằng 6 xuất hiện đúng 5 lần ⇒ có 5.6=30 khả năng xảy ra.
TH4: Số bằng 6 xuất hiện đúng 6 lần ⇒ có 1 khả năng xảy ra.
Vậy có30+1+30+1=62 khả năng xảy ra biến cố A.
VậyP(A)= $$ \frac{62}{{6}^{6}}=\frac{31}{23328}$$.

========================================================================

https://khoahoc.vietjack.com/thi-online/10-cau-trac-nghiem-phep-thu-va-bien-co-co-dap-an-van-dung


\textbf{{QUESTION}}

Có hai dãy ghế đối diện nhau, mỗi dãy có ba ghế. Xếp ngẫu nhiên 6 học sinh, gồm 3 nam và 3 nữ, ngồi vào hai dãy ghế đó sao cho mỗi ghế có đúng một học sinh ngồi. Xác suất để mỗi học sinh nam đều ngồi đối diện với một học sinh nữ bằng:
A. 25$$ \frac{2}{5}$$.
B. 120$$ \frac{1}{20}$$.
C. 35$$ \frac{3}{5}$$.
D. 110$$ \frac{1}{10}$$.

\textbf{{ANSWER}}

Đáp án cần chọn là: A
Số phần tử của không gian mẫu là: n(Ω)=6!
Gọi biến cố A: "Các bạn học sinh nam ngồi đối diện các bạn nữ".
Chọn chỗ cho học sinh nam thứ nhất có 6 cách.
Chọn chỗ cho học sinh nam thứ 2 có 4 cách (không ngồi đối diện học sinh nam thứ nhất)
Chọn chỗ cho học sinh nam thứ 3 có 2 cách (không ngồi đối diện học sinh nam thứ nhất, thứ hai).
Xếp chỗ cho 3 học sinh nữ: 3! cách.
⇒ =6.4.2.3! = 288 cách.
⇒P(A)=$$ \frac{288}{6!}=\frac{2}{5}$$.

========================================================================

https://khoahoc.vietjack.com/thi-online/10-cau-trac-nghiem-phep-thu-va-bien-co-co-dap-an-van-dung


\textbf{{QUESTION}}

Một hộp đựng 20 viên bi khác nhau được đánh số từ 1 đến 20. Lấy ba viên bi từ hộp trên rồi cộng số ghi trên đó lại. Hỏi có bao nhiêu cách để lấy kết quả thu được là một số chia hết cho 3?
A. 90.
B. 1200.
C. 384.
D. 1025.

\textbf{{ANSWER}}

Đáp án cần chọn là: C
Chia các số từ 1 đến 20 làm 3 nhóm:
X1={1;4;7;...;19}$$ {X}_{1}=\left\{1;4;7;...;19\right\}$$: chia cho 3 dư 1(có 7 phần tử)
X2={2;5;8;...;20}$$ {X}_{2}=\left\{2;5;8;...;20\right\}$$: chia cho 3 dư 2(có 7 phần tử)
X3={3;6;9;...;18}$$ {X}_{3}=\left\{3;6;9;...;18\right\}$$: chia hết cho 3(có 6 phần tử)
Để kết quả thu được là một số chia hết cho 3 thì số ghi trên viên bi có các trường hợp sau:
+) Cả 3 viên thuộc X1$$ {X}_{1}$$, có: C37$$ {C}_{7}^{3}$$ cách
+) Cả 3 viên thuộc X2$$ {X}_{2}$$, có: C37$$ {C}_{7}^{3}$$ cách
+) Cả 3 viên thuộc X3$$ {X}_{3}$$, có: C36$$ {C}_{6}^{3}$$cách
+) 1 viên thuộc X1$$ {X}_{1}$$, 1 viên thuộc X2$$ {X}_{2}$$, 1 viên thuộc X3$$ {X}_{3}$$, có: 7.7.6 cách
⇒Số cách thỏa mãn là: C37+C37+C36+7.7.6=384$$ {C}_{7}^{3}+{C}_{7}^{3}+{C}_{6}^{3}+7.7.6=384$$.

========================================================================

https://khoahoc.vietjack.com/thi-online/10-cau-trac-nghiem-phep-thu-va-bien-co-co-dap-an-van-dung


\textbf{{QUESTION}}

Chọn ngẫu nhiên một số tự nhiên trong các số tự nhiên có bốn chữ số. Tính xác xuất để số được chọn có ít nhất hai chữ số 8 đứng liền nhau.
A. 0,029.
B. 0,019.
C. 0,021.
D. 0,017.

\textbf{{ANSWER}}

Đáp án cần chọn là: A
* Gọi số tự nhiên có 4 chữ số là abcd$$ \overline{)abcd}$$(a≠0;0≤a, b, c, d≤9; a, b, c, d∈N)
+ a có 9 cách chọn
+b, c, d có 10 cách chọn
Không gian mẫu có số phần tử là n(Ω)=9.103$$ {10}^{3}$$
* Gọi A là biến cố số được chọn có ít nhất hai chữ số 8 đứng liền nhau
TH1: Có hai chữ số 8 đứng liền nhau. Ta chọn 2 chữ số còn lại trong abcd$$ \overline{)abcd}$$
+ 2 chữ số 8 đứng đầu thì có 9.10=90 cách chọn 2 chữ số còn lại
+ 2 chữ số 8 đứng ở giữa thì có 8 cách chọn chữ số hàng nghìn và 9 cách chọn chữ số hàng đơn vị nên có 8.9=72 cách chọn.
+ 2 chữ số 8 đứng ở cuối thì có 9 cách chọn chữ số hàng nghìn và 9 cách chọn chữ số hàng trăm nên có 9.9 cách chọn.
Vậy trường hợp này có 90+72+81=243 số.
TH2: Có ba chữ số 8 đứng liền nhau.
+ 3 chữ số 8 đứng đầu thì có 9 cách chọn chữ số hàng đơn vị
+ 3 chữ số 8 đứng cuối thì có 8 cách chọn chữ số hàng nghìn
Vậy trường hợp này có 9+8=17 số
TH3: Có 4 chữ số 8 đứng liền nhau thì có 1 số
Số phần tử của biến cố A là n(A)=243+17+1=261
Xác suất cần tìm là P(A)= n(A)n(Ω)=2619.103=0,029$$ \frac{n\left(A\right)}{n\left(\Omega \right)}=\frac{261}{9.{10}^{3}}=0,029$$.

========================================================================

https://khoahoc.vietjack.com/thi-online/10-cau-trac-nghiem-phep-thu-va-bien-co-co-dap-an-van-dung


\textbf{{QUESTION}}

Gọi S là tập các số tự nhiên gồm 9 chữ số được lập từ tập X={6;7;8},trong đó chữ số 6 xuất hiện 2 lần, chữ số 7 xuất hiện 3 lần, chữ số 8 xuất hiện 4 lần. Chọn ngẫu nhiên một số từ tập S; tính xác suất để số được chọn là số không có chữ số 7 đứng giữa hai chữ số 6.
A. $$ \frac{2}{5}$$.
B. $$ \frac{11}{12}$$.
C. $$ \frac{4}{5}$$.
D. $$ \frac{55}{432}$$.

\textbf{{ANSWER}}

Đáp án cần chọn là: A
+ Số cách sắp xếp 2 chữ số 6 vào 9 vị trí là $$ {C}_{9}^{2}$$
+ Số cách sắp xếp 3 chữ số 7 vào 7 vị trí còn lại là $$ {C}_{7}^{3}$$
+ Số cách sắp xếp 4 chữ số 8 vào 4 vị trí còn lại là $$ {C}_{4}^{4}$$
Số phần tử của tập S là n(Ω)=$$ {C}_{9}^{2}.{C}_{7}^{3}.{C}_{4}^{4}=1260$$
Gọi A là biến cố “Số được chọn ra từ tập S là số không có chữ số 7 đứng giữa hai chữ số 6”
TH1: Ta xét 2 chữ số 6 thành 1 cặp, ta sẽ sắp xếp cặp này với các chữ số còn lại
Số cách sắp xếp là $$ {C}_{8}^{1}.{C}_{7}^{3}.{C}_{4}^{4}=280$$ cách
TH2: Ta xếp chữ số 8 đứng giữa hai chữ số 6.
Cách 1: Có 1 số 8 đứng giữa hai số 6, khi đó có coi 686 là 1 cụm thì có 7 cách sắp xếp cụm này vào số có 9 chữ số, có $$ {C}_{6}^{3}$$ cách sắp xếp 3 chữ số 8 còn lại $$ {C}_{3}^{3}$$ và cách sắp xếp 3 chữ số 7.
Vậy có $$ 7.{C}_{6}^{3}.{C}_{3}^{3}=140$$ số
Cách 2:  Có 2 số 8 đứng giữa hai số 6, khi đó có coi 6886 là 1 cụm thì có 66 cách sắp xếp cụm này vào số có 9 chữ số, có cách sắp xếp 3 chữ số 8 còn lại và  cách sắp xếp 3 chữ số 7.
Vậy có $$ 6.{C}_{5}^{2}.{C}_{3}^{3}=60$$ số
Cách 3: Có 3 số 8 đứng giữa hai số 6, khi đó có coi 68886 là 1 cụm thì có 5 cách sắp xếp cụm này vào số có 9 chữ số, có $$ {C}_{5}^{2}$$ cách sắp xếp 3 chữ số 8 còn lại và $$ {C}_{3}^{3}$$ cách sắp xếp 3 chữ số 7.
Vậy có $$ 5.{C}_{4}^{1}.{C}_{3}^{3}=20$$ số
Cách 4: Có 4 số 8 đứng giữa hai số 6, khi đó có coi 688886 là 1 cụm thì có 4 cách sắp xếp cụm này vào số có 9 chữ số, có $$ {C}_{4}^{1}$$ cách sắp xếp 3 chữ số 7.
Vậy có $$ 4{C}_{3}^{3}=4$$ số
Vậy biến cố A có 280+140+60+20+4=504 phần tử
Xác suất cần tìm là P(A)= $$ \frac{504}{1260}=\frac{2}{5}$$.

========================================================================

https://khoahoc.vietjack.com/thi-online/de-kiem-tra-1-tiet-toan-7-chuong-4-dai-so-co-dap-an/40610


\textbf{{QUESTION}}

Trong mỗi câu dưới đây, hãy chọn phương án trả lời đúng:
Giá trị của đơn thức $$ A\quad =\quad 2/3\quad {x}^{4}{y}^{2}$$ tại x = 1, y = 2 là:
A. 8/3
B. 4/3
C. 4
D. 2

\textbf{{ANSWER}}

Thay x = 1, y = 2 vào đơn thức ta có
A = 2/3.14.22 = 8/3. Chọn A

========================================================================

https://khoahoc.vietjack.com/thi-online/de-kiem-tra-1-tiet-toan-7-chuong-4-dai-so-co-dap-an/40610


\textbf{{QUESTION}}

Đơn thức đồng dạng với đơn thức -xy2z$$ -x{y}^{2}z$$ là:
A. xyz2$$ xy{z}^{2}$$
B. -√7 xy2z$$ -\sqrt{7}\quad x{y}^{2}z$$
C. -xy2z2$$ -x{y}^{2}{z}^{2}$$
D. 9x2y2z$$ 9{x}^{2}{y}^{2}z$$

\textbf{{ANSWER}}

Chọn B

========================================================================

https://khoahoc.vietjack.com/thi-online/de-kiem-tra-1-tiet-toan-7-chuong-4-dai-so-co-dap-an/40610


\textbf{{QUESTION}}

Thu gọn đơn thức (-2x3y4)(-3x5y6)$$ (-2{x}^{3}{y}^{4})(-3{x}^{5}{y}^{6})$$ ta được đơn thức:
A.-6x8y10$$ -6{x}^{8}{y}^{10}$$
B. 6x10y8$$ 6{x}^{10}{y}^{8}$$
C.- 6x8y10$$ -\quad 6{x}^{8}{y}^{10}$$
D. 6x8y10$$ 6{x}^{8}{y}^{10}$$

\textbf{{ANSWER}}

Ta có (-2x3y4) (-3x5y6) = 6x8y10. Chọn D

========================================================================

https://khoahoc.vietjack.com/thi-online/de-kiem-tra-1-tiet-toan-7-chuong-4-dai-so-co-dap-an/40610


\textbf{{QUESTION}}

Hệ số của đơn thức 8a2x4y(-1/2 b3y2) (a,b là các hằng số) là:
A. -8a2b2
B. -4a3b3
C. 4a2b3
D. -4a2b3

\textbf{{ANSWER}}

Hệ số của đơn thức là: 8a2(-1/2 b3) = -4a2b3. Chọn D

========================================================================

https://khoahoc.vietjack.com/thi-online/de-kiem-tra-1-tiet-toan-7-chuong-4-dai-so-co-dap-an/40610


\textbf{{QUESTION}}

Tổng của các đơn thức 4x2y, -3x2y, 3x2y và 2x2y$$ 4{x}^{2}y,\quad -3{x}^{2}y,\quad 3{x}^{2}y\quad và\quad 2{x}^{2}y$$ là:
A. 6x2y$$ 6{x}^{2}y$$
B. 9x2y$$ 9{x}^{2}y$$
C. 12x2y$$ 12{x}^{2}y$$
D. 5x2y$$ 5{x}^{2}y$$

\textbf{{ANSWER}}

Chọn A
Ta có: 4x2y + (-3x2y) + 3x2y + 2x2y = 6x2y.

========================================================================

https://khoahoc.vietjack.com/thi-online/bo-de-minh-hoa-mon-toan-thpt-quoc-gia-nam-2022-30-de/91351


\textbf{{QUESTION}}

Trong không gian với hệ tọa độ Oxyz, cho mặt phẳng $\left( P \right):2x - 4y + 6z - 1 = 0$. Mặt phẳng $\left( P \right)$ có một vectơ pháp tuyến là:

\textbf{{ANSWER}}

Đáp án A
Mặt phẳng $\left( P \right):2x - 4y + 6z - 1 = 0$ nhận $\overrightarrow a = \left( {2; - 4;6} \right)$ là một vectơ pháp tuyến.
Xét $\overrightarrow n = \left( {1; - 2;3} \right)$. Ta có $\overrightarrow a = 2\overrightarrow n $ nên suy ra $\overrightarrow a $ và $\overrightarrow n $ cùng phương. Vậy $\overrightarrow n = \left( {1; - 2;3} \right)$ cũng là một vectơ pháp tuyến của mặt phẳng $\left( P \right)$.

========================================================================

https://khoahoc.vietjack.com/thi-online/bo-de-minh-hoa-mon-toan-thpt-quoc-gia-nam-2022-30-de/91351


\textbf{{QUESTION}}

Cho a là số thực dương khác 5. Tính $I = {\log _{\frac{a}{5}}}\left( {\frac{{{a^3}}}{{125}}} \right)$.

\textbf{{ANSWER}}

Đáp án D
Phương pháp:
Sử dụng công thức:  ${\log _a}{b^m} = m{\log _a}b\;\left( {0 < a \ne 1,b > 0} \right)$.
Cách giải:
Ta có $I = {\log _{\frac{a}{5}}}\left( {\frac{{{a^3}}}{{125}}} \right) = {\log _{\frac{a}{5}}}{\left( {\frac{a}{5}} \right)^3} = 3{\log _{\frac{a}{5}}}\left( {\frac{a}{5}} \right) = 3.$

========================================================================

https://khoahoc.vietjack.com/thi-online/35-cau-trac-nghiem-toan-6-ket-noi-tri-thuc-bai-tap-cuoi-chuong-3-trang-76-co-dap-an


\textbf{{QUESTION}}

Dùng số âm để diễn tả các thông tin sau:
a) Ở nơi lạnh nhất thế giới, nhiệt độ có thể xuống đến $$ {60}^{\mathrm{o}}\mathrm{C}$$ dưới $$ {0}^{\mathrm{o}}\mathrm{C}$$
b) Do dịch bệnh, một công ty trong một tháng đã bị lỗ 2 triệu đồng.

\textbf{{ANSWER}}

a) Ở nơi lạnh nhất thế giới, nhiệt độ có thể xuống tới - $$ {60}^{\mathrm{o}}\mathrm{C}$$.
b) Do dịch bệnh, một công ty trong một tháng đã thu về - 2 triệu đồng.

========================================================================

https://khoahoc.vietjack.com/thi-online/35-cau-trac-nghiem-toan-6-ket-noi-tri-thuc-bai-tap-cuoi-chuong-3-trang-76-co-dap-an


\textbf{{QUESTION}}

Trong các số a, b, c, d, số nào dương, số nào âm nếu:
 a > 0;         b < 0;        c ≥ 1;         d ≤ -2.

\textbf{{ANSWER}}

+) Vì a > 0 nên a là số dương.
+) Vì b < 0 nên b là số âm
+) Vì c ≥ 1 hay c > 1 nên c là số dương
+) Vì d ≤ -2 hay d < 0 nên d là số âm.
Vậy các số dương là: a, c
        các số âm là: b, d.

========================================================================

https://khoahoc.vietjack.com/thi-online/35-cau-trac-nghiem-toan-6-ket-noi-tri-thuc-bai-tap-cuoi-chuong-3-trang-76-co-dap-an


\textbf{{QUESTION}}

Liệt kê các phần tử của tập hợp sâu rồi tính tổng của chúng:
a) S = {x ∈ Z|- 5 < x ≤ 5}
b) T = {x ∈ Z|- 7 ≤ x < 1}.

\textbf{{ANSWER}}

a) Các số nguyên lớn hơn -5 và nhỏ hơn hoặc bằng 5 là: -4; -3; -2; -1; 0; 1; 2; 3; 4; 5
Do đó S = {-4; -3; -2; -1; 0; 1; 2; 3; 4; 5}
Tổng các chữ số trong tập S là: (-4) + (- 3) + (- 2) + (- 1) + 0 + 1 + 2 + 3 + 4 + 5
= [(- 4) + 4] + [(- 3) + 3] + [(- 2) + 2] + [(- 1) + 1] + 0 + 5
= 0 + 0 + 0 + 0 + 0 + 5 = 5
b) Các số nguyên lớn hơn hoặc bằng -7 và nhỏ hơn 1 là: -7; -6; -5; -4; -3; -2; -1; 0
Do đó T = {-7; -6; -5; -4; -3; -2; -1; 0}
Tổng các chữ số trong tập T là: (-7) + (- 6) + (- 5) + (- 4) + (- 3) + (- 2) + (- 1) + 0
= - (1 + 2 + 3 + 4 + 5 + 6 + 7)
= - 28.

========================================================================

https://khoahoc.vietjack.com/thi-online/35-cau-trac-nghiem-toan-6-ket-noi-tri-thuc-bai-tap-cuoi-chuong-3-trang-76-co-dap-an


\textbf{{QUESTION}}

Tính một cách hợp lí:
a) 15.(-236) + 15.235; 
b) 237.(-28) + 28.137; 
c) 38.(27 - 44) - 27. (38 - 44).

\textbf{{ANSWER}}

a) 15.(-236) + 15.235 
= 15.(-236 + 235) 
= 15.[ - (236 – 235)]
= 15.(-1) = - 15
b) 237.(-28) + 28.137 
= (- 237).28 + 28.137 
= 28.(- 237 + 137)
= 28.[- (237 – 137)] 
= 28.(- 100) = - 2 800
c) 38.(27 - 44) - 27.(38 - 44) 
= 38.27 - 38. 44 - 27.38 + 27.44
= (38.27 – 27.38) + (27.44 – 38.44)
= 0 + 44.(27 – 38) 
= 44.(27 - 38) 
= 44.(-11) = - 484.

========================================================================

https://khoahoc.vietjack.com/thi-online/35-cau-trac-nghiem-toan-6-ket-noi-tri-thuc-bai-tap-cuoi-chuong-3-trang-76-co-dap-an


\textbf{{QUESTION}}

Tính giá trị của biểu thức (-35). x - (-15). 37  trong mỗi trường hợp sau:
a) x = 15;
b) x = - 37.

\textbf{{ANSWER}}

a) Thay x = 15 vào biểu thức P ta được:
P = (-35). x - (-15). 37 
   = (-35). 15 - (-15). 37 
   = (-35). 15 + 15. 37
   = 15. (- 35 + 37) 
   = 15. 2 = 30
b) Thay x = - 37 vào biểu thức P ta được:
P = (-35). (-37) - (-15). 37 
= 35. 37 + 15. 37 
= 37. (15 + 35) 
= 37. 50 = 1 850.

========================================================================

https://khoahoc.vietjack.com/thi-online/16-cau-trac-nghiem-toan-6-chan-troi-sang-tao-bai-2-ba-diem-thang-hang-ba-diem-khong-thang-hang-co-da


\textbf{{QUESTION}}

A. Nếu ba điểm cùng thuộc một đường thẳng thì ba điểm đó không thẳng hàng
B. Nếu ba điểm không cùng thuộc một đường thẳng thì ba điểm đó thẳng hàng
C. Nếu ba điểm cùng thuộc một đường thẳng thì ba điểm đó thẳng hàng
D. Cả ba đáp án trên đều sai.

\textbf{{ANSWER}}

Từ định nghĩa ba điểm thẳng hàng ta thấy đáp án C đúng.
Đáp án cần chọn là: C

========================================================================

https://khoahoc.vietjack.com/thi-online/12-bai-tap-chung-minh-dang-thuc-luong-giac-co-loi-giai


\textbf{{QUESTION}}

Cho góc α thỏa mãn 0° ≤ α ≤ 180°. Chứng minh rằng 
sin4 α − cos4 α = 2 sin2 α − 1.

\textbf{{ANSWER}}

Hướng dẫn giải:
Cách 1. Ta có ${\cos ^4}\alpha = {\left( {{{\cos }^2}\alpha } \right)^2} = {\left( {1 - {{\sin }^2}\alpha } \right)^2} = 1 - 2{\sin ^2}\alpha + {\sin ^4}\alpha $
Do đó: sin4 α − cos4 α = sin4 α – (1 – 2sin2 α + sin4 α) = 2 sin2 α − 1.
Vậy ta được điều phải chứng minh. 
Cách 2. Ta có sin4 α − sin4 α = (sin2 α + cos2 α)( sin2 α − cos2 α)
 = 1. [sin2 α – (1 − sin2 α)] = 2 sin2 α − 1.
Vậy sin4 α − cos4 α = 2 sin2 α − 1.
Cách 3. Ta sử dụng phép biến đổi tương đương
sin4 α − cos4 α = 2 sin2 α − 1 
⇔ sin4 α − 2 sin2 α + 1 − cos4 α = 0
⇔ (1 − sin2 α)2 − cos4 α = 0 
⇔ cos4 α − cos4 α = 0 (luôn đúng).
Vậy đẳng thức được chứng minh.

========================================================================

https://khoahoc.vietjack.com/thi-online/12-bai-tap-chung-minh-dang-thuc-luong-giac-co-loi-giai


\textbf{{QUESTION}}

Cho tam giác ABC. Chứng minh rằng: cosA = − cos(B + C).

\textbf{{ANSWER}}

Hướng dẫn giải:
Áp dụng định lí tổng ba góc trong tam giác ABC, ta có ˆA$\widehat A$+ ˆB$\widehat B$+ ˆC$\widehat C$ = 180°.
Suy ra: 180° −ˆA$\widehat A$= ˆB$\widehat B$+ ˆC$\widehat C$. 
Do đó: cos(180° – A) = cos(B + C).
Lại có: cos(180° – A) = – cosA            (quan hệ giữa hai góc bù nhau). 
Khi đó ta có: – cosA = cos(B + C) ⇔ cosA = – cos(B + C). 
Vậy đẳng thức được chứng minh.

========================================================================

https://khoahoc.vietjack.com/thi-online/12-bai-tap-chung-minh-dang-thuc-luong-giac-co-loi-giai


\textbf{{QUESTION}}

Chọn hệ thức đúng được suy ra từ hệ thức cos2 α + sin2 α = 1 với 0° ≤ α ≤ 180°?

\textbf{{ANSWER}}

Hướng dẫn giải:
Đáp án đúng là: D.
Từ hệ thức cos2 α + sin2 α = 1, ta suy ra được: 
${\cos ^2}\frac{\alpha }{2} + {\sin ^2}\frac{\alpha }{2} = 1$; ${\cos ^2}\frac{\alpha }{3} + {\sin ^2}\frac{\alpha }{3} = 1$; ${\cos ^2}\frac{\alpha }{4} + {\sin ^2}\frac{\alpha }{4} = 1$; ${\cos ^2}\frac{\alpha }{5} + {\sin ^2}\frac{\alpha }{5} = 1$.
Suy ra: $5\left( {{{\cos }^2}\frac{\alpha }{5} + {{\sin }^2}\frac{\alpha }{5}} \right) = 5.1 = 5$.

========================================================================

https://khoahoc.vietjack.com/thi-online/12-bai-tap-chung-minh-dang-thuc-luong-giac-co-loi-giai


\textbf{{QUESTION}}

Cho tam giác ABC, tìm đẳng thức sai trong các đẳng thức sau ?
A. sin A = sin (B + C);
B. tan A = tan (B + C);
C. cosA2=sinB+C2$\cos \frac{A}{2} = \sin \frac{{B + C}}{2}$;
cosA2=sinB+C2
cosA2=sinB+C2
cos

A2
A
A
2
2


2
2
2
=
sin

B+C2
B+C
B+C
B
+
C
2
2


2
2
2
D. tan A = − tan (B + C).

\textbf{{ANSWER}}

Hướng dẫn giải:
Đáp án đúng là: B.
Tam giác ABC có: ˆA$\widehat A$+ ˆB$\widehat B$+ ˆC$\widehat C$ = 180° (định lí tổng ba góc trong tam giác).
Suy ra: 180° −ˆA$\widehat A$= ˆB$\widehat B$+ ˆC$\widehat C$ và 
Do đó sin A = sin (180° − A) = sin (B + C), suy ra khẳng định A đúng.
Lại có ˆA+ˆB+ˆC2=180∘2=90∘$\frac{{\widehat A + \widehat B + \widehat C}}{2} = \frac{{180^\circ }}{2} = 90^\circ $ ⇒ˆA2+ˆB+ˆC2=90∘$ \Rightarrow \frac{{\widehat A}}{2} + \frac{{\widehat B + \widehat C}}{2} = 90^\circ $
Do đó:cosA2=sinB+C2$\cos \frac{A}{2} = \sin \frac{{B + C}}{2}$ (hai góc phụ nhau), suy ra khẳng định C đúng.
Mặt khác tan A = − tan (180° −ˆA$\widehat A$) = − tan (B + C), suy ra khẳng định D đúng và B sai.

========================================================================

https://khoahoc.vietjack.com/thi-online/12-bai-tap-chung-minh-dang-thuc-luong-giac-co-loi-giai


\textbf{{QUESTION}}

Cho góc x với 0° < x < 90°. Trong các đẳng thức dưới đây, đẳng thức đúng là?

\textbf{{ANSWER}}

Hướng dẫn giải:
Đáp án đúng là: A.
Do 0° < x < 90° nên tanx > 0 và cotx > 0.
Ta có tanx . cotx = 1, suy ra cotx = $\frac{1}{{tanx}}$.
 Khi đó: $\frac{{1 + \cot x}}{{1 - \cot x}}$ = $\frac{{1 + \frac{1}{{\tan x}}}}{{1 - \frac{1}{{\tan x}}}} = \frac{{\frac{{\tan x + 1}}{{\tan x}}}}{{\frac{{\tan x - 1}}{{\tan x}}}} = \frac{{\tan x + 1}}{{\tan x - 1}}$.
Vậy $\frac{{1 + \cot x}}{{1 - \cot x}} = \frac{{\tan x + 1}}{{\tan x - 1}}$.

========================================================================

https://khoahoc.vietjack.com/thi-online/15-cau-trac-nghiem-toan-7-ket-noi-tri-thuc-bai-24-bieu-thuc-dai-so-co-dap-an/111074


\textbf{{QUESTION}}

Bạn A có 100 nghìn đồng. Bạn ấy mua một mua x chiếc bút với giá 3 nghìn đồng và x – 5 cuốn vở viết với giá 10 nghìn đồng vừa hết 100 nghìn đồng. Hỏi bạn ấy đã mua bao nhiêu cuốn vở và bao nhiêu chiếc bút?

\textbf{{ANSWER}}

Đáp án đúng là: A
Số tiền bạn A cần để mua bút là:
3x (nghìn đồng)
Số tiền bạn A cần để mua vở là:
10(x – 5) (nghìn đồng)
Tổng số tiền bạn ấy cần để mua là 100 nghìn đồng nên ta có:
3x + 10(x – 5) = 100
15x – 50 = 100
15x = 150
x = 10.
Vậy bạn A mua là 10 chiếc bút và 5 cuốn vở.

========================================================================

https://khoahoc.vietjack.com/thi-online/15-cau-trac-nghiem-toan-7-ket-noi-tri-thuc-bai-24-bieu-thuc-dai-so-co-dap-an/111074


\textbf{{QUESTION}}

Giả sử C là giá gốc của một đôi giày và r là thuế giá trị gia tăng. Biểu thức biểu diễn giá bán thị trường của đôi giày đó là gì? Nếu C = 800 nghìn đồng và r = 10% thì giá bán của đôi giày đó là bao nhiêu?

\textbf{{ANSWER}}

Đáp án đúng là: C
Giá bán thị trường sẽ bằng giá gốc cộng với giá trị gia tăng của đôi giày.
Suy ra biểu thức biểu diễn giá bán thị trường của đôi giày là: C + Cr.
Khi đó giá bán của đôi giày khi C = 800 nghìn đồng và r = 10% là:
800 + 800. 10% = 800 + 80 = 880( nghìn đồng).

========================================================================

https://khoahoc.vietjack.com/thi-online/15-cau-trac-nghiem-toan-7-ket-noi-tri-thuc-bai-24-bieu-thuc-dai-so-co-dap-an/111074


\textbf{{QUESTION}}

Gọi x là chiều rộng của hình hộp chữ nhật, viết biểu thức đại số biểu thị thể tích của một hình hộp chữ nhật có chiều dài hơn chiều rộng 4 cm và hơn chiều cao 2cm. Tính kích thước của hình hộp chữ nhật khi biết chu vi mặt đáy của hình hộp chữ nhật là: 28 cm.

\textbf{{ANSWER}}

Đáp án đúng là: A
x (cm) là chiều rộng của hình hộp chữ nhật.
Khi đó chiều dài của hình chữ nhật là: x + 4 (cm).
Chiều cao của hình hộp chữ nhật là: x + 4 – 2 = x + 2 (cm)
Vậy biểu thức biểu thị thể tích của hình hộp chữ nhật là:
x(x + 4)(x + 2)
Chu vi mặt đáy của hình hộp chữ nhật bằng 28 cm nên ta có:
2(x + x + 4) = 28
2(2x + 4) = 28
 4x + 8 = 28
4x = 28 – 8
4x= 20
x = 5
Vậy kích thước của hình hộp chữ nhật lần lượt là: 9 cm; 5 cm; 7 cm thì thỏa mãn yêu cầu bài toán.

========================================================================

https://khoahoc.vietjack.com/thi-online/giai-vth-toan-6-kntt-bai-1-tap-hop-co-dap-an


\textbf{{QUESTION}}

Ta đã biết N là tập hợp các số tự nhiên. Gọi P là tập hợp các phân số. Khi đó:
A)  $$ 5\in \mathbb{N}$$ và $$ 5\in P$$
B) $$ 5\in \mathbb{N}$$ và $$ 5\notin P$$
C) $$ 5\notin \mathbb{N}$$ và $$ 5\in P$$
D) $$ 5\notin \mathbb{N}$$ và $$ 5\notin P$$

\textbf{{ANSWER}}

Lời giải:
Đáp án đúng là A do 5 là số tự nhiên đồng thời 5 được viết dưới dạng phân số là $$ \frac{5}{1}$$.

========================================================================

https://khoahoc.vietjack.com/thi-online/giai-vth-toan-6-kntt-bai-1-tap-hop-co-dap-an


\textbf{{QUESTION}}

Gọi M là tập hợp các số tự nhiên có hai chữ số. Khi đó:
A) M = {n | n có hai chữ số}
B) M = {n có hai chữ số | n ∈$$ \in $$ℕ*$$ {\mathbb{N}}^{*}$$}
C) M = {n | n∈ℕ$$ \in \mathbb{N}$$ có hai chữ số}
D) M = {n| n∈ℕ$$ \in \mathbb{N}$$ }

\textbf{{ANSWER}}

Lời giải:
Vì n là số tự nhiên nên n ∈ℕ$$ \in \mathbb{N}$$ và n có hai chữ số.
Đáp án C

========================================================================

https://khoahoc.vietjack.com/thi-online/giai-vth-toan-6-kntt-bai-1-tap-hop-co-dap-an


\textbf{{QUESTION}}

Cho hai tập hợp A = {a; b; c; x; y} và B = {b; d; y; t; u; v}.
Dùng kí hiệu ∈ hoặc ∉ để trả lời các câu hỏi: Mỗi phần tử a, b, x, u thuộc tập nào và không thuộc tập hợp nào?

\textbf{{ANSWER}}

Lời giải:
a ∈ A và a ∉ B
b ∈ A và b ∉ B
x ∈ A và x ∉ B
u ∈ A và u ∈ B.

========================================================================

https://khoahoc.vietjack.com/thi-online/giai-vth-toan-6-kntt-bai-1-tap-hop-co-dap-an


\textbf{{QUESTION}}

Cho tập hợp U = {x ∈ N| x chia hết cho 3}.
Trong các số 3; 5; 6; 0; 7 số nào thuộc và số nào không thuộc tập hợp U?

\textbf{{ANSWER}}

Lời giải:
Trong các số đã cho, các số 3; 6; 0 chia hết cho 3, các số 5 và 7 không chia hết cho 3.
Do đó: 
Các số thuộc tập U là 3; 6; 0.
Các số không thuộc tập U là 5; 7.

========================================================================

https://khoahoc.vietjack.com/thi-online/giai-vth-toan-6-kntt-bai-1-tap-hop-co-dap-an


\textbf{{QUESTION}}

Bằng cách liệt kê các phần tử, hãy viết các tập hợp sau:
a) Tập hợp K các số tự nhiên nhỏ hơn 7;
b) Tập hợp D tên các tháng (dương lịch) có 30 ngày;
c) Tập hợp M các chữ cái tiếng Việt trong từ “ĐIỆN BIÊN PHỦ”.

\textbf{{ANSWER}}

Lời giải:
a) K = {0; 1; 2; 3; 4; 5; 6}.
b) D = {Tháng Tư, tháng Sáu, tháng Chín, Tháng Mười một}.
c) M = {Đ; I; Ê; N; B; P; H; U}.

========================================================================

https://khoahoc.vietjack.com/thi-online/10-bai-tasp-phat-bieu-dinh-ly-dinh-ly-dao-duoi-dang-dieu-kien-can-dieu-kien-du-co-loi-giai


\textbf{{QUESTION}}

Trong các mệnh đề sau, mệnh đề nào có phát biểu là định lý?
A. Nếu một tứ giác là hình chữ nhật thì tứ giác đó có bốn cạnh bằng nhau;
B. Nếu một số tự nhiên tận cùng là 5 thì số đó chia hết cho 5;
C. Nếu một tự nhiên chia hết cho 3 thì nó chia hết cho 9;
D. Nếu một tứ giác có hai đường chéo vuông góc với nhau thì tứ giác đó là hình thoi.

\textbf{{ANSWER}}

Đáp án đúng là: B.
A. Vì hình chữ nhật có hai cặp cạnh đối bằng nhau nên mệnh đề ở câu A sai.
Do đó mệnh đề trên không phải là định lý.
B. Mệnh đề ở câu B đúng do dấu hiệu để một số chia hết cho 5 là số đó có chữ số tận cùng là 0 hoặc 5.
Vì vậy mệnh đề câu B là định lý.
C. Ta có một số chia hết cho 9 thì nó cũng chia hết cho 3, tuy nhiên một số chia hết cho 3 thì nó chưa chắc chia hết cho 9.
Chẳng hạn số 3 chia hết cho 3 nhưng nó không chia hết cho 9.
Nên mệnh đề ở câu C sai và nó không phải là định lý.
D. Mệnh đề ở câu D sai do một tứ giác có hai đường chéo vuông góc với nhau thì chưa chắc nó đã là hình thoi.
Vì vậy mệnh đề trên không phải là định lý.

========================================================================

https://khoahoc.vietjack.com/thi-online/10-bai-tasp-phat-bieu-dinh-ly-dinh-ly-dao-duoi-dang-dieu-kien-can-dieu-kien-du-co-loi-giai


\textbf{{QUESTION}}

Cho định lý sau: “Nếu hai tam giác bằng nhau thì hai tam giác đó đồng dạng”.
Phát biểu định lý trên dưới dạng điều kiện cần.
A. Hai tam giác bằng nhau kéo theo hai tam giác đó đồng dạng;
B. Hai tam giác bằng nhau là điều kiện cần để hai tam giác đó đồng dạng;
C. Hai tam giác đồng dạng là điều kiện cần để hai tam giác đó bằng nhau;
D. Hai tam giác bằng nhau tương đương với hai tam giác đó đồng dạng.

\textbf{{ANSWER}}

Đáp án đúng là: C.
Ta có:
P: “Hai tam giác bằng nhau”.
Q: “Hai tam giác đó đồng dạng”.
Ta thấy định lý trên có dạng P ⇒ Q có thể được phát biểu dưới dạng điều kiện cần như sau:
Q là điều kiện cần để có P.
Do đó định lý đã cho được phát biểu dưới dạng điều kiện cần là:
Hai tam giác đồng dạng là điều kiện cần để hai tam giác đó bằng nhau.

========================================================================

https://khoahoc.vietjack.com/thi-online/10-bai-tasp-phat-bieu-dinh-ly-dinh-ly-dao-duoi-dang-dieu-kien-can-dieu-kien-du-co-loi-giai


\textbf{{QUESTION}}

Cho định lý sau: “Nếu một số tự nhiên chỉ chia hết cho 1 và chính nó thì số đó là số nguyên tố”.
Phát biểu định lý trên dưới dạng điều kiện đủ.
A. Một số tự nhiên chỉ chia hết cho 1 và chính nó khi và chỉ khi số đó là số nguyên tố;
B. Một số tự nhiên chỉ chia hết cho 1 và chính nó là điều kiện đủ để số đó là số nguyên tố;
C. Một số tự nhiên là số nguyên tố là điều kiện đủ để số đó chia hết cho 1 và chính nó;
D. Điều kiện cần và đủ để một số tự nhiên chỉ chia hết cho 1 và chính nó là số đó là số nguyên tố.

\textbf{{ANSWER}}

Đáp án đúng là: B.
Ta có:
P: “ Nếu một số tự nhiên chỉ chia hết cho 1 và chính nó”.
Q: “Số đó là số nguyên tố”.
Ta thấy định lý trên có dạng P ⇒ Q có thể được phát biểu dưới dạng điều kiện đủ như sau:
P là điều kiện đủ để có Q.
Do đó định lý đã cho được phát biểu dưới dạng điều kiện đủ là:
Một số tự nhiên chỉ chia hết cho 1 và chính nó là điều kiện đủ để số đó là số nguyên tố.

========================================================================

https://khoahoc.vietjack.com/thi-online/10-bai-tasp-phat-bieu-dinh-ly-dinh-ly-dao-duoi-dang-dieu-kien-can-dieu-kien-du-co-loi-giai


\textbf{{QUESTION}}

Cho các mệnh đề sau:
(1) Nếu tích của hai số a và b lớn hơn 0 thì a và b đều dương.
(2) Nếu a, b là hai số nguyên dương thì tích của chúng cũng là một số nguyên dương.
(3) Nếu tích của hai số a và b là một số nguyên âm thì trong hai số đó phải có một số nguyên dương và một số nguyên âm.
Trong các mệnh đề trên, có bao nhiêu mệnh đề là định lý?
A. 0;
B. 1;
C. 2;
D. 3.

\textbf{{ANSWER}}

Đáp án đúng là: C.
(1) Ta có ví dụ sau :
a = – 2, b = – 4
a.b = (– 2).(– 4) = 8
Từ ví dụ trên ta thấy mặc dù tích của a và b là một số dương nhưng a và b đều là các số âm.
Do đó mệnh đề (1) sai và không phải là định lý.
(2) Ta có ví dụ sau:
Tích của hai số nguyên dương là một số nguyên dương là mệnh đề đúng (tích của hai số nguyên cùng dấu là số nguyên dương).
Do đó mệnh đề (2) là định lý.
(3) Ta có ví dụ sau :
Tích của hai số nguyên khác dấu thì là số nguyên âm. 
Do đó mệnh đề (3) là định lý.
Vậy có hai mệnh đề là định lý.

========================================================================

https://khoahoc.vietjack.com/thi-online/10-bai-tasp-phat-bieu-dinh-ly-dinh-ly-dao-duoi-dang-dieu-kien-can-dieu-kien-du-co-loi-giai


\textbf{{QUESTION}}

Cho định lý sau: “Một tam giác là tam giác đều thì tam giác đó có ba đường phân giác bằng nhau”.
Phát biểu định lý đảo của định lý trên dưới dạng điều kiện cần.
A. Một tam giác là tam giác đều là điều kiện cần để tam giác đó có ba đường phân giác bằng nhau;
B. Một tam giác có ba đường phân giác bằng nhau là điều kiện cần để tam giác đó là tam giác đều;
C. Một tam giác là tam giác đều khi và chỉ khi tam giác đó có ba đường phân giác bằng nhau;
D. Một tam giác là tam giác đều là điều kiện cần và đủ để tam giác đó có ba đường phân giác bằng nhau.

\textbf{{ANSWER}}

Đáp án đúng là: A.
Xét mệnh đề “Một tam giác là tam giác đều thì tam giác đó có ba đường phân giác bằng nhau” ta có:
P: “Một tam giác là tam giác đều”
Q: “Tam giác đó có ba đường phân giác bằng nhau”
Định lý đảo Q ⇒ P của định lý trên được phát biểu như sau:
“Một tam giác có ba đường phân giác bằng nhau thì tam giác đó là tam giác đều”.
Xét định lý đảo trên ta có :
A: “Một tam giác có ba đường phân giác bằng nhau”.
B: “Tam giác đó là tam giác đều”.
Ta thấy định lý trên có dạng A ⇒ B có thể được phát biểu dưới dạng điều kiện cần như sau:
B là điều kiện cần để có A.
Do đó định lý đã cho được phát biểu dưới dạng điều kiện cần là:
“Một tam giác là tam giác đều là điều kiện cần để tam giác đó có ba đường phân giác bằng nhau”.

========================================================================

https://khoahoc.vietjack.com/thi-online/bai-tap-toan-9-bai-2-ti-so-luong-giac-cua-goc-nhon


\textbf{{QUESTION}}

Tính giá trị của các biểu thức

\textbf{{ANSWER}}

1. $$ A=3a+b-a=2a+b$$
2. $$ B=4{a}^{2}.{\left(\frac{\sqrt{2}}{2}\right)}^{2}-3{a}^{2}+{\left(2a.\frac{\sqrt{2}}{2}\right)}^{2}=2{a}^{2}-3{a}^{2}+2{a}^{2}={a}^{2}$$

========================================================================

https://khoahoc.vietjack.com/thi-online/bai-tap-toan-9-bai-2-ti-so-luong-giac-cua-goc-nhon


\textbf{{QUESTION}}

Tính giá trị của biểu thức A=8−cos2300+2sin2450−√3tan3600$$ A=8-c{\text{os}}^{2}{30}^{0}+2{\mathrm{sin}}^{2}{45}^{0}-\sqrt{3}{\mathrm{tan}}^{3}{60}^{0}$$

\textbf{{ANSWER}}

Ta có: A=8−(√32)2+2(√22)−√3(√3)3=−34$$ A=8-{\left(\frac{\sqrt{3}}{2}\right)}^{2}+2\left(\frac{\sqrt{2}}{2}\right)-\sqrt{3}{\left(\sqrt{3}\right)}^{3}=-\frac{3}{4}$$

========================================================================

https://khoahoc.vietjack.com/thi-online/bai-tap-toan-9-bai-2-ti-so-luong-giac-cua-goc-nhon


\textbf{{QUESTION}}

Tính giá trị của biểu thức A=(a2+1)sin00+bcos900$$ A=\left({a}^{2}+1\right)\mathrm{sin}{0}^{0}+bc{\text{os90}}^{0}$$

\textbf{{ANSWER}}

Ta có: A=(a2+1).0+b.0=0$$ A=\left({a}^{2}+1\right).0+b.0=0$$

========================================================================

https://khoahoc.vietjack.com/thi-online/bai-tap-toan-9-bai-2-ti-so-luong-giac-cua-goc-nhon


\textbf{{QUESTION}}

Tính giá trị của biểu thức A=a2sin900−b2cos00acot450−b−2acot900

\textbf{{ANSWER}}

Ta có A=a2−b2a−b=a+b$$ A=\frac{{a}^{2}-{b}^{2}}{a-b}=a+b$$

========================================================================

https://khoahoc.vietjack.com/thi-online/10-bai-tap-bai-toan-thuc-tien-liaen-quan-den-phuong-trinh-luong-giac-co-loi-giai


\textbf{{QUESTION}}

Số giờ có ánh sáng mặt trời của một thành phố A ở vĩ độ 40° Bắc trong ngày thứ t của một năm không nhuận được cho bởi hàm số: 
$$ d\left(t\right)=3\mathrm{sin}\left[\frac{\pi }{182}\left(t-80\right)\right]+12$$ , với t ∈ ℤ và 0 ≤ t ≤ 365.
Thành phố A có đúng 15 giờ có ánh sáng mặt trời vào ngày nào trong năm?
A. 171;          
B. 170;          
C. 169;

\textbf{{ANSWER}}

Hướng dẫn giải:
Đáp án đúng là: A 
Giả sử thành phố A có đúng 15 giờ có ánh sáng mặt trời vào ngày thứ t0.
Ta có: $$ d\left({t}_{0}\right)=3\mathrm{sin}\left[\frac{\pi }{182}\left({t}_{0}-80\right)\right]+12$$
Mà d(t0) = 15 nên ta có:
$$ 3\mathrm{sin}\left[\frac{\pi }{182}\left({t}_{0}-80\right)\right]+12=15$$
⇔$$ 3\mathrm{sin}\left[\frac{\pi }{182}\left({t}_{0}-80\right)\right]=3$$
⇔$$ \mathrm{sin}\left[\frac{\pi }{182}\left({t}_{0}-80\right)\right]=1$$
⇔$$ \frac{\pi }{182}\left({t}_{0}-80\right)=\frac{\pi }{2}+k2\pi $$, k ∈ ℤ  
⇔ t0 – 80 = 91 + 364k, k ∈ ℤ
⇔ t0 = 364k + 171, k ∈ ℤ
Mà 0 ≤ t0 ≤ 365
⇔ –171 ≤ 364k ≤ 194
⇔ –0,47 ≤ k ≤ 0,53 
Mà k ∈ ℤ nên k = 0 
Nếu k = 0 thì t0 = 171
Vậy thành phố A có đúng 15 giờ có ánh sáng mặt trời vào ngày thứ 171.

========================================================================

https://khoahoc.vietjack.com/thi-online/10-bai-tap-bai-toan-thuc-tien-liaen-quan-den-phuong-trinh-luong-giac-co-loi-giai


\textbf{{QUESTION}}

Số giờ có ánh sáng mặt trời của một thành phố A ở vĩ độ 40° Bắc trong ngày thứ t của một năm không nhuận được cho bởi hàm số: 
$$ d\left(t\right)=3\mathrm{sin}\left[\frac{\pi }{182}\left(t-80\right)\right]+12$$ , với t ∈ ℤ và 0 ≤ t ≤ 365.
Vào ngày nào trong năm thì thành phố A có đúng 9 giờ có ánh sáng mặt trời?
A. 351;            
B. 352;                   
C. 353;

\textbf{{ANSWER}}

Hướng dẫn giải:
Đáp án đúng là: C 
Giả sử thành phố A có đúng 9 giờ có ánh sáng mặt trời vào ngày thứ t0.
Ta có: $$ d\left({t}_{0}\right)=3\mathrm{sin}\left[\frac{\pi }{182}\left({t}_{0}-80\right)\right]+12$$
Mà d(t0) = 9 nên ta có:
$$ 3\mathrm{sin}\left[\frac{\pi }{182}\left({t}_{0}-80\right)\right]+12=9$$
⇔$$ 3\mathrm{sin}\left[\frac{\pi }{182}\left({t}_{0}-80\right)\right]=-3$$
⇔$$ \mathrm{sin}\left[\frac{\pi }{182}\left({t}_{0}-80\right)\right]=-1$$
⇔$$ \frac{\pi }{182}\left({t}_{0}-80\right)=-\frac{\pi }{2}+k2\pi $$ , k ∈ ℤ    
⇔ t0 – 80 = –91 + 364k , k ∈ ℤ
⇔ t0 = 364k – 11, k ∈ ℤ
Mà 0 ≤ t0 ≤ 365
⇔ 11 ≤ 364k ≤ 376
⇔ 0,03 ≤ k ≤ 1,03 
Mà k ∈ ℤ nên k = 1
Nếu k = 1 thì t0 = 353
Vậy thành phố A có đúng 9 giờ có ánh sáng mặt trời vào ngày thứ 353.

========================================================================

https://khoahoc.vietjack.com/thi-online/15-cau-trac-nghiem-toan-10-chan-troi-sang-tao-cac-phep-toan-tren-tap-hop-co-dap-an


\textbf{{QUESTION}}

Điền vào chỗ trống: “Tập hợp các phần tử thuộc A hoặc thuộc B gọi là ….”
A. hợp của hai tập hợp;
B. giao của hai tập hợp;
C. hai tập hợp bằng nhau; 
D. phần bù của hai tập hợp.

\textbf{{ANSWER}}

Đáp án đúng là: A
Tập hợp các phần tử thuộc A hoặc thuộc B gọi là hợp của hai tập hợp, kí hiệu A ∪ B.

========================================================================

https://khoahoc.vietjack.com/thi-online/15-cau-trac-nghiem-toan-10-chan-troi-sang-tao-cac-phep-toan-tren-tap-hop-co-dap-an


\textbf{{QUESTION}}

Giao của hai tập hợp A và B kí hiệu như thế nào?
A. A ∪ B;
B. A = B;
C. A ∩ B;
D. A ⊆ B.

\textbf{{ANSWER}}

Đáp án đúng là: C
Giao của hai tập hợp A và B kí hiệu là A ∩ B.

========================================================================

https://khoahoc.vietjack.com/thi-online/15-cau-trac-nghiem-toan-10-chan-troi-sang-tao-cac-phep-toan-tren-tap-hop-co-dap-an


\textbf{{QUESTION}}

Điền vào chỗ trống: “Hiệu của tập hợp A và tập hợp B là ….”
A. tập hợp các phần tử thuộc B nhưng không thuộc A;
B. tập hợp các phần tử thuộc A nhưng không thuộc B;
C. tập hợp các phần tử thuộc B và thuộc A;
D. tập hợp các phần tử thuộc B hoặc thuộc A.

\textbf{{ANSWER}}

Đáp án đúng là: B
Hiệu của tập hợp A và tập hợp B là tập hợp các phần tử thuộc A nhưng không thuộc B, kí hiệu là A\B

========================================================================

https://khoahoc.vietjack.com/thi-online/20-cau-trac-nghiem-toan-10-canh-dieu-bat-phuong-trinh-bac-nhat-hai-an-co-dap-an-phan-2/103648


\textbf{{QUESTION}}

Một công ty nhập về 1 tấn gỗ để sản xuất bàn và ghế. Biết một cái bàn cần 30 kg gỗ và một cái ghế cần 15 kg gỗ. Gọi x và y lần lượt là số bàn và số ghế mà công ty sản xuất. Viết bất phương trình bậc nhất hai ẩn x, y sao cho lượng bàn ghế mà công ty sản xuất không vượt quá 1 tấn gỗ ?

\textbf{{ANSWER}}

Hướng dẫn giải
Đáp án đúng là: B
Đổi 1 tấn = 1000 kg
Số gỗ để sản xuất x bàn là 30x (kg).
Số gỗ để sản xuất y ghế là 15y (kg)
Số gỗ để sản xuất x bàn và y ghế là 30x + 15y (kg)
Vì lượng bàn ghế mà công ty sản xuất không được vượt quá 1 tấn gỗ nên 
30x + 15y ≤ 1000.

========================================================================

https://khoahoc.vietjack.com/thi-online/20-cau-trac-nghiem-toan-10-canh-dieu-bat-phuong-trinh-bac-nhat-hai-an-co-dap-an-phan-2/103648


\textbf{{QUESTION}}

Một người thợ được thuê làm một cái hồ bơi và một vườn hoa trên mảnh đất có diện tích là 200 m2 và phải để diện tích lối đi tối thiểu là 50 m2. Diện tích của hồ bơi và vườn hoa thỏa mãn các điều kiện trên lần lượt là:

\textbf{{ANSWER}}

Hướng dẫn giải
Đáp án đúng là: D
Gọi x (m2) là diện tích của hồ bơi và y (m2) là diện tích của vườn hoa (0 < x, y < 200).
Diện tích lối đi là 200 − x − y (m2).
Vì diện tích lối đi tối thiếu là 50 m2 nên ta có bất phương trình: 
200 − x − y ≥ 50 (*).
+ Thay cặp số (100; 70) vào bất phương trình (*) ta được
200 − 100 − 70 = 30 < 50, không thỏa mãn.
+ Thay cặp số (100; 60) vào bất phương trình (*) ta được
200 − 100 − 60 = 40 < 50, không thỏa mãn.
+ Thay cặp số (90; 80) vào bất phương trình (*) ta được
200 − 90 − 80 = 30 < 50, không thỏa mãn.
+ Thay cặp số (100; 40) vào bất phương trình (*) ta được
200 − 100 − 40 = 60 ≥ 50, thỏa mãn.
Vậy diện tích của hồ bơi và vườn hoa lần lượt là 100 m2 và 40 m2.

========================================================================

https://khoahoc.vietjack.com/thi-online/20-cau-trac-nghiem-toan-10-canh-dieu-bat-phuong-trinh-bac-nhat-hai-an-co-dap-an-phan-2/103648


\textbf{{QUESTION}}

Nhà mạng Mobiphone tính phí cước gọi nội mạng là 1 580 đồng/phút và cước gọi ngoại mạng là 2 500 đồng/ phút. Số phút có thể sử dụng gọi nội mạng và số phút có thể sử dụng gọi ngoại mạng sao cho tổng số tiền phải trả trong 1 tháng ít hơn 250 nghìn đồng lần lượt là: 
A. 50 phút và 70 phút;

\textbf{{ANSWER}}

Hướng dẫn giải
Đáp án đúng là: D
Gọi x (phút) là số phút gọi nội mạng và y (phút) là số phút gọi ngoại mạng (x, y ≥ 0)
1,58x (nghìn đồng) là số tiền phải trả gọi nội mạng trong 1 tháng
2,5y (nghìn đồng) là số tiền phải trả gọi ngoại mạng trong 1 tháng
Để số tiền cước điện thoại trong một tháng ít hơn 250 nghìn đồng ta có: 1,58x + 2,5y < 250. (*)
+ Thay cặp số 50 phút và 70 phút và bất phương trình (*) ta được
1,58 . 50 + 2,5 . 70 = 254 > 250, không thỏa mãn.
+ Thay cặp số 60 phút và 70 phút và bất phương trình (*) ta được
1,58 . 60 + 2,5 . 70 = 269,8 > 250, không thỏa mãn.
+ Thay cặp số 40 phút và 80 phút và bất phương trình (*) ta được
1,58 . 40 + 2,5 . 80 = 263,2 > 250, không thỏa mãn.
+ Thay cặp số 80 phút và 40 phút và bất phương trình (*) ta được
1,58 . 80 + 2,5 . 40 = 226,4 < 250, thỏa mãn.
Vậy có thể sử dụng 80 phút nội mạng và 40 phút ngoại mạng trong 1 tháng để tổng số tiền phải trả trong 1 tháng ít hơn 250 nghìn đồng.

========================================================================

https://khoahoc.vietjack.com/thi-online/20-cau-trac-nghiem-toan-10-canh-dieu-bat-phuong-trinh-bac-nhat-hai-an-co-dap-an-phan-2/103648


\textbf{{QUESTION}}

Một hộ gia đình tính chi phí sử dụng đèn và máy lạnh trong nhà. Biết đèn sử dụng trong 1 giờ tốn 500 đồng và máy lạnh sử dụng trong 1 giờ tốn 1 nghìn đồng. Hỏi số giờ sử dụng đèn trong 1 ngày và số giờ sử dụng máy lạnh trong 1 ngày để tổng số tiền điện trong một tháng (30 ngày) ít hơn 1 triệu đồng lần lượt là bao nhiêu ? (Biết căn nhà có 3 cái đèn và 2 cái máy lạnh)

\textbf{{ANSWER}}

Hướng dẫn giải
Đáp án đúng là: A
Gọi x (giờ) là số giờ sử dụng đèn trong 1 ngày và y (giờ) là số giờ sử dụng máy lạnh trong 1 ngày (x, y ≥ 0)
0,5x . 3 . 30 (nghìn đồng) là số tiền phải trả khi sử dụng đèn trong 1 tháng.
y . 2 . 30 (nghìn đồng) là số tiền phải trả khi sử dụng máy lạnh trong 1 tháng.
Ta có: 1 triệu = 1 000 nghìn đồng.
Để tổng số tiền điện trong một tháng ít hơn 1 triệu đồng thì : 
0,5x . 3 . 30 + y . 2 . 30 < 1000 ⇔ 45x + 60y < 1000 (*).
Thay cặp số 15 giờ và 5 giờ vào bất phương trình trên ta được
45 . 15 + 60 . 5 = 975 < 1000, thỏa mãn.
Vậy có thể sử dụng đèn 15 giờ/ngày và sử dụng máy lạnh 5 giờ/ngày để tiền điện phải trả trong 1 tháng nhỏ hơn 1 triệu đồng.

========================================================================

https://khoahoc.vietjack.com/thi-online/20-cau-trac-nghiem-toan-10-canh-dieu-bat-phuong-trinh-bac-nhat-hai-an-co-dap-an-phan-2/103648


\textbf{{QUESTION}}

Một xưởng sản xuất bánh ngọt nhập về 100 kg đường để làm hai loại bánh kem và bánh donut. Biết một cái bánh kem cần 500 g đường và một cái bánh donut cần 200 g đường. Hỏi có thể sản xuất bao nhiêu cái bánh kem và bao nhiêu cái bánh donut sao cho không vượt quá số đường nhập về ? 
A. 150 cái bánh kem và 200 cái bánh donut;

\textbf{{ANSWER}}

Hướng dẫn giải
Đáp án đúng là: C
Đổi 500 g = 0,5 kg, 200 g = 0,2 kg. 
Gọi x (bánh) là số bánh kem có thể sản xuất và y (bánh) là số bánh donut có thể sản xuất (x, y ∈ ℕ*). 
0,5x (kg) là lượng đường cần sử dụng để làm bánh kem.
0,2y (kg) là lượng đường cần sử dụng để làm bánh donut. 
Để sản xuất bánh kem và bánh donut sao cho không vượt quá 100 kg đường thì ta có: 0,5x + 0,2y ≤ 100. (*)
Thay cặp 100 cái bánh kem và 150 cái bánh donut vào bất phương trình (*) ta được
0,5 . 100 + 0,2 . 150 = 80 < 100. 
Tương tự, thay các cặp số (150; 200), (200; 100), (150; 150) ta thấy không thỏa mãn (*).
Vậy có thể sản xuất 100 cái bánh kem và 150 cái bánh donut để lượng đường sử dụng không quá 100kg.

========================================================================

https://khoahoc.vietjack.com/thi-online/15-cau-trac-nghiem-toan-7-chan-troi-sang-tao-bai-3-tam-giac-can-co-dap-an


\textbf{{QUESTION}}

Khẳng định nào sau đây là sai?
A. Tam giác đều có ba góc bằng nhau và bằng 60°;

\textbf{{ANSWER}}

Đáp án đúng là: C
Hướng dẫn giải
Tam giác đều là tam giác có ba cạnh bằng nhau. Vậy đáp án B đúng
Tam giác đều có mỗi góc bằng nhau và bằng 60°. Vậy đáp án A đúng
Tam giác đều cũng là tam giác cân nhưng tam giác cân chưa chắc là tam giác đều vì nó chỉ có hai cạnh bên bằng nhau.
Vậy đáp án D đúng, C sai.

========================================================================

https://khoahoc.vietjack.com/thi-online/15-cau-trac-nghiem-toan-7-chan-troi-sang-tao-bai-3-tam-giac-can-co-dap-an


\textbf{{QUESTION}}

Cho tam giác ABC cân tại B. Chọn kết luận đúng nhất.

\textbf{{ANSWER}}

Đáp án đúng là: C
Hướng dẫn giải
Ta có tam giác ABC cân tại B nên AB = BC. Do đó đáp án A và D sai.
Tam giác ABC chưa thể kết luận là tam giác đều vì thiếu điều kiện. Vậy đáp án B sai.
Tam giác ABC cân tại B có ˆA=ˆC$$ \widehat{A}=\widehat{C}$$ (hai góc ở đáy). Vậy đáp án C đúng.

========================================================================

https://khoahoc.vietjack.com/thi-online/15-cau-trac-nghiem-toan-7-chan-troi-sang-tao-bai-3-tam-giac-can-co-dap-an


\textbf{{QUESTION}}

Tam giác cân là tam giác:
A. có hai đường cao bằng nhau;

\textbf{{ANSWER}}

Đáp án đúng là: C
Hướng dẫn giải
Tam giác cân là tam giác có hai cạnh bên bằng nhau và hai góc ở đáy bằng nhau.
Vậy đáp án C đúng.

========================================================================

https://khoahoc.vietjack.com/thi-online/15-cau-trac-nghiem-toan-7-chan-troi-sang-tao-bai-3-tam-giac-can-co-dap-an


\textbf{{QUESTION}}

Cho tam giác ABC cân tại A, biết góc B = 50°. Tính số đo các góc còn lại của tam giác đó.

\textbf{{ANSWER}}

Đáp án đúng là: B
Hướng dẫn giải
Ta có tam giác ABC cân tại A suy ra ˆB=ˆC$$ \widehat{B}=\widehat{C}$$ = 50°.
Xét tam giác ABC có: 
 ˆA+ˆB+ˆC$$ \widehat{A}+\widehat{B}+\widehat{C}$$= 180° (tổng ba góc trong một tam giác).
Suy ra ˆA$$ \widehat{A}$$ = 180° − (ˆB+ˆC$$ \widehat{B}+\widehat{C}$$) 
= 180° − (50° + 50°) 
= 180° – 100° = 80°.
Vậy ˆA$$ \widehat{A}$$ = 80°; ˆB=50o;  ˆC=50o$$ \widehat{B}={50}^{o};\text{\hspace{0.17em}\hspace{0.17em}}\widehat{C}={50}^{o}$$ .

========================================================================

https://khoahoc.vietjack.com/thi-online/15-cau-trac-nghiem-cap-so-nhan-co-dap-an-nhan-biet


\textbf{{QUESTION}}

Cho cấp số nhân $$ \left({\mathrm{u}}_{\mathrm{n}}\right)$$, biết: $$ {\mathrm{u}}_{1}=-2;{\mathrm{u}}_{2}=8$$. Lựa chọn đáp án đúng.
A. q = -4
B. q = 4
C. q = -12
D. q = 10

\textbf{{ANSWER}}

Đáp án A
Vì $$ \left({\mathrm{u}}_{\mathrm{n}}\right)$$ là cấp số nhân nên $$ \mathrm{q}=\frac{{\mathrm{u}}_{2}}{{\mathrm{u}}_{1}}=\frac{8}{-2}=-4$$

========================================================================

https://khoahoc.vietjack.com/thi-online/15-cau-trac-nghiem-cap-so-nhan-co-dap-an-nhan-biet


\textbf{{QUESTION}}

Dãy số 1,2,4,8,16,... là một cấp số nhân với:
A. Công bội là 3 và số hạng đầu tiên là 1.
B. Công bội là 2 và số hạng đầu tiên là 1.
C. Công bội là 4 và số hạng đầu tiên là 2.
D. Công bội là 2 và số hạng đầu tiên là 2.

\textbf{{ANSWER}}

Đáp án B
Cấp số nhân: 1,2,4,8,16,32,…⇒{u1=1q=u2u1=2$$ \Rightarrow \left\{\begin{array}{l}{\mathrm{u}}_{1}=1\\ \mathrm{q}=\frac{{\mathrm{u}}_{2}}{{\mathrm{u}}_{1}}=2\end{array}\right.$$

========================================================================

https://khoahoc.vietjack.com/thi-online/15-cau-trac-nghiem-cap-so-nhan-co-dap-an-nhan-biet


\textbf{{QUESTION}}

Cho cấp số nhân (un)$$ \left({\mathrm{u}}_{\mathrm{n}}\right)$$ có số hạng đầu u1=12$$ {\mathrm{u}}_{1}=\frac{1}{2}$$ và công bội q=3$$ \mathrm{q}=3$$. Tính u5$$ {\mathrm{u}}_{5}$$
A. 812$$ \frac{81}{2}$$
B. 1632$$ \frac{163}{2}$$
C. 272$$ \frac{27}{2}$$
D. 552$$ \frac{55}{2}$$

\textbf{{ANSWER}}

Đáp án A
Ta có: u5=u1q4=12.34=812$$ {\mathrm{u}}_{5}={\mathrm{u}}_{1}{\mathrm{q}}^{4}=\frac{1}{2}.{3}^{4}=\frac{81}{2}$$

========================================================================

https://khoahoc.vietjack.com/thi-online/15-cau-trac-nghiem-cap-so-nhan-co-dap-an-nhan-biet


\textbf{{QUESTION}}

Cho cấp số nhân (un)$$ \left({\mathrm{u}}_{\mathrm{n}}\right)$$ biết: u1=3,u5=48$$ {\mathrm{u}}_{1}=3,{\mathrm{u}}_{5}=48$$. Lựa chọn đáp án đúng.
A. u3=12$$ {\mathrm{u}}_{3}=12$$
B. u3=-12$$ {\mathrm{u}}_{3}=-12$$
C. u3=16$$ {\mathrm{u}}_{3}=16$$
D. u3=-16$$ {\mathrm{u}}_{3}=-16$$

\textbf{{ANSWER}}

Đáp án A
Ta có:
u5=u1.q4⇔48=3.q4⇔q4=16⇔q2=4⇒u3=u1.q2=3.4=12$$ {\mathrm{u}}_{5}={\mathrm{u}}_{1}.{\mathrm{q}}^{4}\Leftrightarrow 48=3.{\mathrm{q}}^{4}\phantom{\rule{0ex}{0ex}}\Leftrightarrow {\mathrm{q}}^{4}=16\Leftrightarrow {\mathrm{q}}^{2}=4\phantom{\rule{0ex}{0ex}}\Rightarrow {\mathrm{u}}_{3}={\mathrm{u}}_{1}.{\mathrm{q}}^{2}=3.4=12$$

========================================================================

https://khoahoc.vietjack.com/thi-online/15-cau-trac-nghiem-cap-so-nhan-co-dap-an-nhan-biet


\textbf{{QUESTION}}

Cho cấp số nhân (un)$$ \left({\mathrm{u}}_{\mathrm{n}}\right)$$ với u1=−2,q=−5$$ {\mathrm{u}}_{1}=-2,\mathrm{q}=-5$$. Viết bốn số hạng đầu tiên của cấp số nhân.
A. −2;10;50;−250
B. −2;10;−50;250
C. −2;−10;−50;−250
D. −2;10;50;250

\textbf{{ANSWER}}

Đáp án B
Ta có:
{u1=−2q=−5⇒{u1=−2u2=u1.q=−2.(−5)=10u3=u2.q=10.(−5)=−50u4=u3.q=−50.(−5)=250$$ \left\{\begin{array}{l}{\mathrm{u}}_{1}=-2\\ \mathrm{q}=-5\end{array}\right.\phantom{\rule{0ex}{0ex}}\Rightarrow \left\{\begin{array}{l}{\mathrm{u}}_{1}=-2\\ {\mathrm{u}}_{2}={\mathrm{u}}_{1}.\mathrm{q}=-2.(-5)=10\\ {\mathrm{u}}_{3}={\mathrm{u}}_{2}.\mathrm{q}=10.(-5)=-50\\ {\mathrm{u}}_{4}={\mathrm{u}}_{3}.\mathrm{q}=-50.(-5)=250\end{array}\right.$$

========================================================================

https://khoahoc.vietjack.com/thi-online/bo-25-de-thi-hoc-ki-1-toan-12-nam-2022-2023-tiep-theo-co-dap-an/116960


\textbf{{QUESTION}}

B. ${\log _a}\left( {xy} \right) = {\log _a}x + {\log _a}y$
D. ${\log _a}\left( {x + y} \right) = {\log _a}x + {\log _a}y$

\textbf{{ANSWER}}

Đáp án B
Phương pháp: 
Sử dụng các công thức logarit. 
Cách giải: 
Trong 4 mệnh đề trên chỉ có mệnh đề ${\log _a}\left( {xy} \right) = {\log _a}x + {\log _a}y$ đúng.

========================================================================

https://khoahoc.vietjack.com/thi-online/bo-25-de-thi-hoc-ki-1-toan-12-nam-2022-2023-tiep-theo-co-dap-an/116960


\textbf{{QUESTION}}

Có tất cả bao nhiêu giá trị nguyên của tham số thực m thuộc đoạn $\left[ { - 2017;2017} \right]$ để hàm số $y = {x^3} - 6{x^2} + mx + 1$ đồng biến trên khoảng $\left( {0; + \infty } \right)$?
D. 2006

\textbf{{ANSWER}}

Đáp án D
Phương pháp: 
Do hàm số $y = {x^3} - 6{x^2} + mx + 1$ đồng biến trên khoảng $\left( {0; + \infty } \right)$ tương đương với hàm số đồng biến trên $\left[ {0; + \infty } \right) \Leftrightarrow y' \ge 0\,\,\forall x \in \left[ {0; + \infty } \right)$ 
Cách giải:
Do hàm số $y = {x^3} - 6{x^2} + mx + 1$ đồng biến trên khoảng $\left( {0; + \infty } \right)$ tương đương với hàm số đồng biến trên $\left[ {0; + \infty } \right)$
Ta có $y' = 3{x^2} - 12x + m \ge 0,\,\,\,\forall x \in \left[ {0; + \infty } \right)$ 
$ \Leftrightarrow m \ge - 3{x^2} + 12x,\,\,\forall x \in \left[ {0; + \infty } \right)$ 
$ \Leftrightarrow m \ge \mathop {max}\limits_{\left[ {0; + \infty } \right)} \left( { - 3{x^2} + 12x} \right)$ 
Xét hàm số $y = - 3{x^2} + 12x$ có hoành độ đỉnh là ${x_0} = - \frac{b}{{2a}} = 2$ 
Và $y\left( 2 \right) = 12,\,\,y\left( 0 \right) = 0$. Suy ra $\mathop {max}\limits_{\left[ {0; + \infty } \right)} \left( { - 3{x^2} + 12x} \right) = y\left( 2 \right) = 12$ 
Vậy giá trị m cần tìm là $m \in \left\{ {12;13;14;...;2017} \right\}$. Suy ra có $2017 - 12 + 1$ giá trị nguyên của tham số m cần tìm.

========================================================================

https://khoahoc.vietjack.com/thi-online/bai-tap-theo-tuan-toan-8-tuan-5


\textbf{{QUESTION}}

Tính nhanh: 
$$ 6,4.18+18.3,6$$

\textbf{{ANSWER}}

$$ 6,4.18+18.3,6=18.\left(6,4+3,6\right)=18.10=180$$

========================================================================

https://khoahoc.vietjack.com/thi-online/bai-tap-theo-tuan-toan-8-tuan-5


\textbf{{QUESTION}}

48.139−48.21−48.18
 48.139−48.21−48.18
 
48.139
−
48.21
−
48.18

\textbf{{ANSWER}}

48.139−48.21−48.18=48.(139−21−18)=48.100=4800$$ 48.139-48.21-48.18=48.\left(139-21-18\right)=48.100=4800$$

========================================================================

https://khoahoc.vietjack.com/thi-online/bai-tap-theo-tuan-toan-8-tuan-5


\textbf{{QUESTION}}

Phân tích đa thức thành nhân tử: 
x2(x−1)−x+1 $$ {x}^{2}\left(x-1\right)-x+1\quad $$
x2(x−1)−x+1 
x2(x−1)−x+1 
x2
x
2
(x−1)
(
x−1
x
−
1
)
−
x
+
1

\textbf{{ANSWER}}

x2.(x−1)−x+1=x2.(x−1)−(x−1)=(x−1)(x2−1)=(x−1)2(x+1)$$ \begin{array}{l}{\mathrm{x}}^{2}.\left(\mathrm{x}-1\right)-\mathrm{x}+1={\mathrm{x}}^{2}.\left(\mathrm{x}-1\right)-\left(\mathrm{x}-1\right)=\left(\mathrm{x}-1\right)\left({\mathrm{x}}^{2}-1\right)={\left(\mathrm{x}-1\right)}^{2}\left(\mathrm{x}+1\right)\end{array}$$

========================================================================

https://khoahoc.vietjack.com/thi-online/bai-tap-theo-tuan-toan-8-tuan-5


\textbf{{QUESTION}}

(a+b)3−(a−b)3
(a+b)3−(a−b)3
(a+b)3
(a+b)
(
a+b
a
+
b
)
3
−
(a−b)3
(a−b)
(
a−b
a
−
b
)
3

\textbf{{ANSWER}}

(a+b)3−(a−b)3=a3+3a2b+3ab2+b3−a3+3a2b−3ab2−b3=6a2b$$ {\left(\mathrm{a}+\mathrm{b}\right)}^{3}-{\left(\mathrm{a}-\mathrm{b}\right)}^{3}={\mathrm{a}}^{3}+3{\mathrm{a}}^{2}\mathrm{b}+3{\mathrm{ab}}^{2}+{\mathrm{b}}^{3}-{\mathrm{a}}^{3}+3{\mathrm{a}}^{2}\mathrm{b}-3{\mathrm{ab}}^{2}-{\mathrm{b}}^{3}=6{\mathrm{a}}^{2}\mathrm{b}$$

========================================================================

https://khoahoc.vietjack.com/thi-online/bai-tap-theo-tuan-toan-8-tuan-5


\textbf{{QUESTION}}

6x(x−3)+9−3x2
 6x(x−3)+9−3x2
 
6
x
(x−3)
(
x−3
x
−
3
)
+
9
−
3
x2
x
2

\textbf{{ANSWER}}

6x(x−3)+9−3x2=6x2−18x+9−3x2=3x2−18x+9=3(x2−6x+3)$$ 6\mathrm{x}\left(\mathrm{x}-3\right)+9-3{\mathrm{x}}^{2}=6{\mathrm{x}}^{2}-18\mathrm{x}+9-3{\mathrm{x}}^{2}=3{\mathrm{x}}^{2}-18\mathrm{x}+9=3\left({\mathrm{x}}^{2}-6\mathrm{x}+3\right)$$

========================================================================

https://khoahoc.vietjack.com/thi-online/10-bai-tasduong-phan-giac-va-duong-phan-giac-doi-voi-tam-giac-dac-biet-tam-giac-can-tam-g


\textbf{{QUESTION}}

Cho điểm E nằm trên tia phân giác góc A của tam giác ABC. Khẳng định nào sau đây là đúng?
A. E nằm trên tia phân giác góc B;
B. E cách đều hai cạnh AB, AC;
C. E nằm trên tia phân giác góc C;

\textbf{{ANSWER}}

Hướng dẫn giải:
Đáp án đúng là: B
Điểm E nằm trên tia phân giác góc A của tam giác ABC thì  E cách đều hai cạnh AB, AC.

========================================================================

https://khoahoc.vietjack.com/thi-online/10-bai-tasduong-phan-giac-va-duong-phan-giac-doi-voi-tam-giac-dac-biet-tam-giac-can-tam-g


\textbf{{QUESTION}}

Cho tam giác ABC có hai đường phân giác CD và BE cắt nhau tại I. Khi đó
A. AI là trung tuyến kẻ từ A;
B. AI là đường cao kẻ từ A;
C. AI là trung trực cạnh BC;

\textbf{{ANSWER}}

Hướng dẫn giải:
Đáp án đúng là: D
Hai đường phân giác CD và BE cắt nhau tại I mà nên I là giao điểm của ba đường phân giác của ∆ABC, do đó AI là phân giác của góc A.

========================================================================

https://khoahoc.vietjack.com/thi-online/10-bai-tasduong-phan-giac-va-duong-phan-giac-doi-voi-tam-giac-dac-biet-tam-giac-can-tam-g


\textbf{{QUESTION}}

Em hãy điền cụm từ thích hợp nhất vào chỗ trống:
"Ba đường phân giác của tam giác giao nhau tại 1 điểm. Điểm này cách đều ... của tam giác đó".
A. ba đỉnh;
B. ba cạnh;
C. hai đỉnh;

\textbf{{ANSWER}}

Hướng dẫn giải:
Đáp án đúng là: B
Ba đường phân giác của một tam giác cùng đi qua một điểm. Điểm này cách đều ba cạnh của tam giác.

========================================================================

https://khoahoc.vietjack.com/thi-online/23-cau-trac-nghiem-dao-ham-cua-cac-ham-so-luong-giac-co-dap-an-phan-2


\textbf{{QUESTION}}

Tính đạo hàm của hàm số y = x.cosx
A. cosx – x.sinx
B. sinx + x.cosx
C. cosx+ x. sinx
D. cosx + sinx

\textbf{{ANSWER}}

Chọn A
Ta áp dụng đạo hàm của 1 tích :
$$ \begin{array}{l}y\text{'}=\left(x\right)\text{'}.c\text{osx + x. (cosx)' }\\ \text{=1.cosx + x. (- sinx})\\ =c\text{osx- x.sin x}\end{array}$$

========================================================================

https://khoahoc.vietjack.com/thi-online/23-cau-trac-nghiem-dao-ham-cua-cac-ham-so-luong-giac-co-dap-an-phan-2


\textbf{{QUESTION}}

Tính đạo hàm của hàm số sau: $$ y={\mathrm{sin}}^{3}\left(2x+1\right)$$. 
A.$$ {\mathrm{sin}}^{2}\left(2x+1\right)\mathrm{cos}\left(2x+1\right).$$
B.$$ 12{\mathrm{sin}}^{2}\left(2x+1\right)\mathrm{cos}\left(2x+1\right).$$
C.$$ 3{\mathrm{sin}}^{2}\left(2x+1\right)\mathrm{cos}\left(2x+1\right).$$
D.$$ 6{\mathrm{sin}}^{2}\left(2x+1\right)\mathrm{cos}\left(2x+1\right).$$

\textbf{{ANSWER}}

Chọn D
Bước đầu tiên áp dung công thức $$ {\left({u}^{\alpha }\right)}^{/}$$với $$ u=\mathrm{sin}\left(2x+1\right)$$ 
  Vậy 
$$ \begin{array}{l}y\text{'}={\left({\mathrm{sin}}^{3}\left(2x+1\right)\right)}^{/}\\ =3{\mathrm{sin}}^{2}\left(2x+1\right).{\left(\mathrm{sin}\left(2x+1\right)\right)}^{/}.\end{array}$$ 
* Tính $$ {\left(\mathrm{sin}\left(2x+1\right)\right)}^{/}$$: Áp dụng $$ {\left(\mathrm{sin}u\right)}^{/}$$, với $$ u=\left(2x+1\right)$$ 
Ta được: 
$$ \begin{array}{l}{\left(\mathrm{sin}\left(2x+1\right)\right)}^{/}\\ =\mathrm{cos}\left(2x+1\right).{\left(2x+1\right)}^{/}\\ =2\mathrm{cos}\left(2x+1\right).\end{array}$$
 
 $$ \begin{array}{l}\Rightarrow y\text{'}=3.{\mathrm{sin}}^{2}\left(2x+1\right).2\mathrm{cos}\left(2x+1\right)\\ =6{\mathrm{sin}}^{2}\left(2x+1\right)\mathrm{cos}\left(2x+1\right).\end{array}$$

========================================================================

https://khoahoc.vietjack.com/thi-online/23-cau-trac-nghiem-dao-ham-cua-cac-ham-so-luong-giac-co-dap-an-phan-2


\textbf{{QUESTION}}

Tính đạo hàm của hàm số sau $$ y=\mathrm{sin}\sqrt{2+{x}^{2}}$$
A.$$ \mathrm{cos}\sqrt{2+{x}^{2}}.$$
B.$$ \frac{1}{\sqrt{2+{x}^{2}}}.\mathrm{cos}\sqrt{2+{x}^{2}}.$$
C.$$ \frac{1}{2}.\mathrm{cos}\sqrt{2+{x}^{2}}.$$
D.$$ \frac{x}{\sqrt{2+{x}^{2}}}.\mathrm{cos}\sqrt{2+{x}^{2}}.$$

\textbf{{ANSWER}}

Chọn D. 
Áp dụng công thức $$ {\left(\mathrm{sin}u\right)}^{/}$$ Với $$ u=\sqrt{2+{x}^{2}}$$
$$ \begin{array}{l}y\text{'}=\mathrm{cos}\sqrt{2+{x}^{2}}.{\left(\sqrt{2+{x}^{2}}\right)}^{/}\\ =\mathrm{cos}\sqrt{2+{x}^{2}}.\frac{{\left(2+{x}^{2}\right)}^{/}}{2\sqrt{2+{x}^{2}}}\\ =\frac{x}{\sqrt{2+{x}^{2}}}.\mathrm{cos}\sqrt{2+{x}^{2}}.\end{array}$$

========================================================================

https://khoahoc.vietjack.com/thi-online/23-cau-trac-nghiem-dao-ham-cua-cac-ham-so-luong-giac-co-dap-an-phan-2


\textbf{{QUESTION}}

Tính đạo hàm của hàm số sau y=√sinx+2x$$ y=\sqrt{\mathrm{sin}x+2x}$$
A.cosx+22√sinx+2x.$$ \frac{\mathrm{cos}x+2}{2\sqrt{\mathrm{sin}x+2x}}.$$
B.cosx+2√sinx+2x.$$ \frac{\mathrm{cos}x+2}{\sqrt{\mathrm{sin}x+2x}}.$$
C.22√sinx+2x.$$ \frac{2}{2\sqrt{\mathrm{sin}x+2x}}.$$
D.cosx2√sinx+2x.$$ \frac{\mathrm{cos}x}{2\sqrt{\mathrm{sin}x+2x}}.$$

\textbf{{ANSWER}}

Chọn A. 
Áp dụng (√u)/$$ {\left(\sqrt{u}\right)}^{/}$$, với  u=sinx+2x$$ u=\mathrm{sin}x+2x$$
y'=(sinx+2x)/2√sinx+2x=cosx+22√sinx+2x.$$ \begin{array}{l}y\text{'}=\frac{{\left(\mathrm{sin}x+2x\right)}^{/}}{2\sqrt{\mathrm{sin}x+2x}}\\ =\frac{\mathrm{cos}x+2}{2\sqrt{\mathrm{sin}x+2x}}.\end{array}$$

========================================================================

https://khoahoc.vietjack.com/thi-online/23-cau-trac-nghiem-dao-ham-cua-cac-ham-so-luong-giac-co-dap-an-phan-2


\textbf{{QUESTION}}

Hàm số y=f(x)=2cos(πx)$$ y=f\left(x\right)=\frac{2}{\mathrm{cos}\left(\pi x\right)}$$ có f'(3)$$ f\text{'}\left(3\right)$$ bằng
A.2π$$ 2\pi $$
B.8π3$$ \frac{8\pi }{3}$$
C.0 
D.4√33$$ \frac{4\sqrt{3}}{3}$$

\textbf{{ANSWER}}

Chọn C
f'(x)=[2cos(πx)]'=2.(cos(πx))'.−1cos2(πx)=2π[−sin(πx)]. −1cos2(πx)=2.πsin(πx)cos2(πx)$$ \begin{array}{l}f\text{'}\left(x\right)=\left[\frac{2}{\mathrm{cos}\left(\pi x\right)}\right]\text{'}\\ =2.\left(\mathrm{cos}\left(\pi x\right)\right)\text{'}.\frac{-1}{{\mathrm{cos}}^{2}\left(\pi x\right)}\\ =2\pi \left[-\mathrm{sin}\left(\pi x\right)\right].\text{\hspace{0.17em}}\frac{-1}{{\mathrm{cos}}^{2}\left(\pi x\right)}\\ =2.\pi \frac{\mathrm{sin}\left(\pi x\right)}{{\mathrm{cos}}^{2}\left(\pi x\right)}\end{array}$$
f'(3)=2π.sin3πcos23π=0$$ f\text{'}\left(3\right)=2\pi .\frac{\mathrm{sin}3\pi }{{\mathrm{cos}}^{2}3\pi }=0$$

========================================================================

https://khoahoc.vietjack.com/thi-online/10-sbai-tap-su-dung-nhi-thuc-newton-de-tinh-gia-tri-gan-dung-co-loi-giai


\textbf{{QUESTION}}

Sử dụng 3 số hạng đầu tiên của khai triển nhị thức Newton, giá trị gần đúng của biểu thức (3 + 0,03)4 là
A. 84,2886;
B. 86,2886;
C. 44,2868;

\textbf{{ANSWER}}

Đáp án đúng là: A
Ta có (3 + 0,03)4 ≈ 34 + 4 . 33 . 0,03 + 6 . 32 . 0,032 = 84,2886.

========================================================================

https://khoahoc.vietjack.com/thi-online/10-sbai-tap-su-dung-nhi-thuc-newton-de-tinh-gia-tri-gan-dung-co-loi-giai


\textbf{{QUESTION}}

Sử dụng 2 số hạng đầu tiên của khai triển nhị thức Newton, giá trị gần đúng của biểu thức (3 + 0,03)5 là 
A. 254,15;
B. 255,15;
C. 256,15;

\textbf{{ANSWER}}

Đáp án đúng là: B
Ta có (3 + 0,03)5 ≈ 35 + 5 . 34 . 0,03 = 255,15.

========================================================================

https://khoahoc.vietjack.com/thi-online/10-sbai-tap-su-dung-nhi-thuc-newton-de-tinh-gia-tri-gan-dung-co-loi-giai


\textbf{{QUESTION}}

Sử dụng 4 số hạng đầu tiên của khai triển nhị thức Newton, giá trị gần đúng của biểu thức (3 − 0,02)5 là 
A. 235,00728;
B. 236,00768;
C. 235,00125;

\textbf{{ANSWER}}

Đáp án đúng là: A
(3 − 0,02)5 ≈ 35 + 5 . 34 . (−0,02) + 10 . 33 . (−0,02)2 + 10 . 32 . (−0,02)3 = 235,00728.

========================================================================

https://khoahoc.vietjack.com/thi-online/10-sbai-tap-su-dung-nhi-thuc-newton-de-tinh-gia-tri-gan-dung-co-loi-giai


\textbf{{QUESTION}}

Sử dụng 3 số hạng đầu tiên của khai triển nhị thức Newton, giá trị gần đúng của biểu thức (3 − 0,02)4 là 
A. 72,8616;
B. 74,8616;
C. 76,8616;

\textbf{{ANSWER}}

Đáp án đúng là: D
Ta có (3 − 0,02)4 ≈ 34 + 4 . 33 . (−0,02) + 6 . 32 . (−0,02)2 = 78,8616.

========================================================================

https://khoahoc.vietjack.com/thi-online/10-sbai-tap-su-dung-nhi-thuc-newton-de-tinh-gia-tri-gan-dung-co-loi-giai


\textbf{{QUESTION}}

Sai số tuyệt đối khi dùng 3 số hạng đầu tiên để tính giá trị gần đúng của biểu thức (2 + 0,05)4 là
A. 0,00100625;
B. 0,00200625;
C. 0,00300625;

\textbf{{ANSWER}}

Đáp án đúng là: A
Ta có: (2 + 0,05)4 ≈ 24 + 4 . 23 . 0,05 + 6 . 22 . 0,052 = 17,66.
Sử dụng máy tính cầm tay, ta kiểm tra được: (2 + 0,05)4 = 17,66100625.

========================================================================

https://khoahoc.vietjack.com/thi-online/bo-de-minh-hoa-mon-toan-thpt-quoc-gia-nam-2022-30-de/91460


\textbf{{QUESTION}}

Trong không gian Oxyz, cho mặt phẳng $\left( P \right):2x - y + z + 4 = 0$. Khi đó mặt phẳng (P) có một vectơ pháp tuyến là

\textbf{{ANSWER}}

Đáp án A
Phương pháp
Mặt phẳng $\left( P \right):ax + by + cz + d = 0$ có một vectơ pháp tuyến $\overrightarrow n = \left( {a;b;c} \right)$ 
Cách giải

========================================================================

https://khoahoc.vietjack.com/thi-online/bo-de-minh-hoa-mon-toan-thpt-quoc-gia-nam-2022-30-de/91460


\textbf{{QUESTION}}

Cho a là số thực dương bất kì khác 1. Tính $S = {\log _a}\left( {{a^3}\sqrt[4]{a}} \right)$.

\textbf{{ANSWER}}

Đáp án C
Phương pháp
Sử dụng các công thức lũy thừa thu gọn biểu thức dưới dấu logarit và sử dụng công thức ${\log _a}{a^n} = n.$ 
Cách giải
Ta có: $S = {\log _a}\left( {{a^3}\sqrt[4]{a}} \right) = {\log _a}\left( {{a^3}.{a^{\frac{1}{4}}}} \right) = {\log _a}^{\frac{{13}}{4}} = \frac{{13}}{4}.$

========================================================================

https://khoahoc.vietjack.com/thi-online/de-kiem-tra-hoc-ki-2-toan-12-co-dap-an-moi-nhat/101678


\textbf{{QUESTION}}

Trong không gian Oxyz, cho hai vectơ $$ \overrightarrow{u}=\left(1;\text{\hspace{0.33em}}3;\text{\hspace{0.33em}}-2\right)$$ và $$ \overrightarrow{v}=\left(2;\text{\hspace{0.33em}}1;\text{\hspace{0.33em}}-1\right)$$ . Tọa độ của vectơ $$ \overrightarrow{u}-\overrightarrow{v}$$  là
A. (-1; 2; -1);
B. (1; -2; 1);
C. (3; 4; -3);

\textbf{{ANSWER}}

Đáp án đúng là: A
Ta có: $$ \overrightarrow{u}-\overrightarrow{v}=\left(1;\text{\hspace{0.33em}}3;\text{\hspace{0.33em}}-2\right)-\left(2;\text{\hspace{0.33em}}1;\text{\hspace{0.33em}}-1\right)=\left(-1;\text{\hspace{0.33em}}2;\text{\hspace{0.33em}}-1\right)$$
Vậy suy ra tọa độ của vectơ $$ \overrightarrow{u}-\overrightarrow{v}$$  là (-1; 2; -1).

========================================================================

https://khoahoc.vietjack.com/thi-online/de-thi-thu-thpt-quoc-gia-mon-toan-co-chon-loc-va-loi-giai-chi-tiet-20-de/75218


\textbf{{QUESTION}}

A. $$ \left(0;-2;-1\right).$$
B. $$ \left(2;1;-1\right).$$
C. $$ \left(1;1;4\right).$$
D. $$ \left(-2;-1;-4\right).$$

\textbf{{ANSWER}}

Đáp án B
Ta thấy chỉ có điểm $$ \left(2;1;-1\right)$$ không thuộc mặt phẳng $$ \left(P\right).$$

========================================================================

https://khoahoc.vietjack.com/thi-online/giai-sbt-toan-10-bai-25-nhi-thuc-newton-co-dap-an


\textbf{{QUESTION}}

Khai triển các đa thức
(x – 2)4;

\textbf{{ANSWER}}

Hướng dẫn giải
(x – 2)4 = [x + (– 2)4]
= $C_4^0.{x^4} + C_4^1.{x^3}.( - 2) + C_4^2.{x^2}.{( - 2)^2} + C_4^3.x.{( - 2)^3} + C_4^4.{( - 2)^4}$
= 1.x4 + 4.x3.(–2) + 6.x2.4 + 4.x.(–8) + 1.16
= x4 – 8x3 + 24x2 – 32x + 16.

========================================================================

https://khoahoc.vietjack.com/thi-online/giai-sbt-toan-10-bai-25-nhi-thuc-newton-co-dap-an


\textbf{{QUESTION}}

(x + 2)5;

\textbf{{ANSWER}}

Hướng dẫn giải
(x+2)5=C05.x5+C15.x4.2+C25.x3.22+C35.x2.23+C45.x.24+C55.25${(x + 2)^5} = C_5^0.{x^5} + C_5^1.{x^4}.2 + C_5^2.{x^3}{.2^2} + C_5^3.{x^2}{.2^3} + C_5^4.x{.2^4} + C_5^5{.2^5}$
= 1.x5 + 5.x4­.2 + 10.x3.4 + 10.x2.8 + 5.x.16 + 1.32
= x5 + 10x4 + 40x3 + 80x2 + 80x + 32.

========================================================================

https://khoahoc.vietjack.com/thi-online/giai-sbt-toan-10-bai-25-nhi-thuc-newton-co-dap-an


\textbf{{QUESTION}}

(2x + 3y)4

\textbf{{ANSWER}}

Hướng dẫn giải
= C04.(2x)4+C14.(2x)3.3y+C24.(2x)2.(3y)2+C34.2x.(3y)3+C44.(3y)4$C_4^0.{(2x)^4} + C_4^1.{(2x)^3}.3y + C_4^2.{(2x)^2}.{(3y)^2} + C_4^3.2x.{(3y)^3} + C_4^4.{(3y)^4}$
= 1.16x4 + 4.8x3.3y + 6.4x2.9y2 + 4.2x.27y3  + 1.81y4 
= 16x4 + 96x3y + 216x2y2­ + 216xy3 + 81y4.

========================================================================

https://khoahoc.vietjack.com/thi-online/giai-sbt-toan-10-bai-25-nhi-thuc-newton-co-dap-an


\textbf{{QUESTION}}

(2x – y)5.

\textbf{{ANSWER}}

Hướng dẫn giải
(2x – y)5 = [2x + (– y)5] 
=C05.(2x)5+C15.(2x)4.(−y)+C25.(2x)3.(−y)2+C35.(2x)2.(−y)3+C45.2x.(−y)4+C55.(−y)5$ = C_5^0.{(2x)^5} + C_5^1.{(2x)^4}.( - y) + C_5^2.{(2x)^3}.{( - y)^2} + C_5^3.{(2x)^2}.{( - y)^3} + C_5^4.2x.{( - y)^4} + C_5^5.{( - y)^5}$
= 1.32x5 + 5.16x4­.(–y) + 10.8x3.y2 + 10.4x2.(–y)3 + 5.2x.y4 + 1.(–y)5 
= 32x5 – 80x4y + 80x3y2 – 40x2y3 + 10xy4 – y5.

========================================================================

https://khoahoc.vietjack.com/thi-online/giai-sbt-toan-10-bai-25-nhi-thuc-newton-co-dap-an


\textbf{{QUESTION}}

Trong khai triển của (5x – 2)5, số mũ của x được sắp xếp theo luỹ thừa tăng dần, hãy tìm hạng tử thứ hai.

\textbf{{ANSWER}}

Hướng dẫn giải
Áp dụng công thức khai triển của (a + b)5 với a = 5x, b = –2, ta có:
(5x – 2)5 
= C05.(5x)5+C15.(5x)4.(−2)+C25.(5x)3.(−2)2+C35.(5x)2.(−2)3+C45.5x.(−2)4+C55.(−2)5$C_5^0.{(5x)^5} + C_5^1.{(5x)^4}.( - 2) + C_5^2.{(5x)^3}.{( - 2)^2} + C_5^3.{(5x)^2}.{( - 2)^3} + C_5^4.5x.{( - 2)^4} + C_5^5.{( - 2)^5}$
= 1 . 3 125x5 + 5 . 625x4­.(–2) + 10 . 125x3.4 + 10 . 25x2.(–8) + 5 . 5x.16 + 1.(–32)
= 3 125x5 – 6 250x4 + 5 000x3 – 2 000x2 + 400x – 32
= – 32 + 400x – 2 000x2 + 5 000x3 – 6 250x4 + 3 125x5 
Vậy, số hạng thứ hai trong khai triển theo số mũ tăng dần của x là 400x.

========================================================================

https://khoahoc.vietjack.com/thi-online/giai-sbt-toan-11-canh-dieu-bai-2-phep-tinh-logarit-co-dap-an


\textbf{{QUESTION}}

Cho a > 0, a ≠ 2. Giá trị của  $$ {\mathrm{log}}_{\frac{a}{2}}\left(\frac{{a}^{2}}{4}\right)$$ bằng:
A. $$ \frac{1}{2};$$
B. 2;
C. $$ -\frac{1}{2};$$
D. – 2.

\textbf{{ANSWER}}

Đáp án đúng là: B
Với a > 0, a ≠ 2 ta có:  $$ {\mathrm{log}}_{\frac{a}{2}}\left(\frac{{a}^{2}}{4}\right)={\mathrm{log}}_{\frac{a}{2}}\left(\frac{{a}^{2}}{{2}^{2}}\right)={\mathrm{log}}_{\frac{a}{2}}{\left(\frac{a}{2}\right)}^{2}=2.$$

========================================================================

https://khoahoc.vietjack.com/thi-online/15-cau-trac-nghiem-toan-8-ket-noi-tri-thuc-bai-5-phep-chia-da-thuc-cho-don-thuc-co-dap-an


\textbf{{QUESTION}}

Giá trị của số tự nhiên thỏa mãn điều kiện gì để phép chia $$ {x}^{n+3}{y}^{6}:{x}^{9}{y}^{n}$$ là phép chia hết?
A. n < 6
B. n = 5
C. n > 6

\textbf{{ANSWER}}

Đáp án đúng là: D
Để phép chia $$ {x}^{n+3}{y}^{6}:{x}^{9}{y}^{n}$$ là phép chia hết: $$ \left\{\begin{array}{c}9\le n+3\\ n\le 6\\ n\in \mathbb{N}\end{array}\right.\Leftrightarrow \left\{\begin{array}{c}n\ge 6\\ n\le 6\\ n\in \mathbb{N}\end{array}\right.\Leftrightarrow n=6$$

========================================================================

https://khoahoc.vietjack.com/thi-online/15-cau-trac-nghiem-toan-8-ket-noi-tri-thuc-bai-5-phep-chia-da-thuc-cho-don-thuc-co-dap-an


\textbf{{QUESTION}}

Đa thức 7x3y2z−2x4y3$$ {\text{7x}}^{\text{3}}{\text{y}}^{\text{2}}\text{z}-{\text{2x}}^{\text{4}}{\text{y}}^{\text{3}}$$ chia hết cho đơn thức nào dưới đây?
A. 3x4$$ 3{x}^{4}$$
B. -3x4$$ -3{x}^{4}$$
C. −2x3y$$ -2{x}^{3}y$$

\textbf{{ANSWER}}

Đáp án đúng là: C
Đa thức $$ {\text{7x}}^{\text{3}}{\text{y}}^{\text{2}}\text{z}-{\text{2x}}^{\text{4}}{\text{y}}^{\text{3}}$$ chia hết cho đơn thức $$ -2{x}^{3}y$$

========================================================================

https://khoahoc.vietjack.com/thi-online/15-cau-trac-nghiem-toan-8-ket-noi-tri-thuc-bai-5-phep-chia-da-thuc-cho-don-thuc-co-dap-an


\textbf{{QUESTION}}

Thực hiện phép chia (2x4y−6x2y7):(2x2)$$ \left({\text{2x}}^{\text{4}}\text{y}-{\text{6x}}^{\text{2}}{\text{y}}^{\text{7}}\right):\left({\text{2x}}^{\text{2}}\right)$$ ta được đa thức ax2y+by7$$ {\text{ax}}^{\text{2}}\text{y}+{\text{by}}^{\text{7}}$$(a, b là hằng số). Khi đó a + b bằng
A. – 3.
B. – 4.
C. – 2.

\textbf{{ANSWER}}

Đáp án đúng là: C
(2x4y−6x2y7):(2x2)=x2y−3y7⇒{a=1b=−3⇒a+b=−2$$ \left({\text{2x}}^{\text{4}}\text{y}-{\text{6x}}^{\text{2}}{\text{y}}^{\text{7}}\right):\left({\text{2x}}^{\text{2}}\right)={x}^{2}y-3{y}^{7}\Rightarrow \left\{\begin{array}{c}a=1\\ b=-3\end{array}\right.\Rightarrow a+b=-2$$

========================================================================

https://khoahoc.vietjack.com/thi-online/15-cau-trac-nghiem-toan-7-ket-noi-tri-thuc-bai-11-dinh-li-va-chung-minh-dinh-ly-co-dap-an


\textbf{{QUESTION}}

Định lý là
A. một khẳng định được suy ra từ những khẳng định đúng đã biết;
B. một khẳng định được suy ra từ những khẳng định không đúng đã biết;
C. một tính chất được suy ra từ những khẳng định đúng;
D. một tính chất được suy ra từ những khẳng định chưa biết.

\textbf{{ANSWER}}

Đáp án đúng là: A
Định lý là một khẳng định được suy ra từ những khẳng định đúng đã biết.

========================================================================

https://khoahoc.vietjack.com/thi-online/15-cau-trac-nghiem-toan-7-ket-noi-tri-thuc-bai-11-dinh-li-va-chung-minh-dinh-ly-co-dap-an


\textbf{{QUESTION}}

Định lí thường được phát biểu dưới dạng:
A. Thì … là…;
B. Do … nên …,
C. Vì … nên …;
D. Nếu … thì ….

\textbf{{ANSWER}}

Đáp án đúng là: D
Mỗi định lí thường được phát biểu dưới dạng “Nếu … thì …”.

========================================================================

https://khoahoc.vietjack.com/thi-online/15-cau-trac-nghiem-toan-7-ket-noi-tri-thuc-bai-11-dinh-li-va-chung-minh-dinh-ly-co-dap-an


\textbf{{QUESTION}}

Trong định lý “Hai đường thẳng phân biệt cùng song song với một đường thứ ba thì chúng song song với nhau”, thì có
A. Giả thiết là “hai đường thẳng phân biệt cùng song song với một đường thẳng thứ ba”;
B. Giả thiết là “chúng song song với nhau”;
C. Giả thiết là “hai đường thẳng phân biệt cùng song song với một đường thứ ba thì chúng song song với nhau”; 
D. Giả thiết là “hai đường thẳng phân biệt cùng song song”.

\textbf{{ANSWER}}

Đáp án đúng là: A.
Giả thiết là vế đứng trước “thì”.
Do đó giả thiết là “hai đường thẳng phân biệt cùng song song với một đường thẳng thứ ba”.
Vậy chọn đáp án A.

========================================================================

https://khoahoc.vietjack.com/thi-online/15-cau-trac-nghiem-toan-7-ket-noi-tri-thuc-bai-11-dinh-li-va-chung-minh-dinh-ly-co-dap-an


\textbf{{QUESTION}}

Chọn phát biểu sai. 
A. Giả thiết của định lí là điều cho biết;
B. Kết luận của định lí là điều suy ra;
C. Giả thiết của định lí là điều suy ra;
D. Cả A và B đều đúng.

\textbf{{ANSWER}}

Đáp án đúng là: C
Giả thiết của định lí là phần cho biết. Kết luận của định lí là điều suy ra.
Vậy chọn đáp án C.

========================================================================

https://khoahoc.vietjack.com/thi-online/15-cau-trac-nghiem-toan-7-ket-noi-tri-thuc-bai-11-dinh-li-va-chung-minh-dinh-ly-co-dap-an


\textbf{{QUESTION}}

Điền vào chỗ trống những nội dung thích hợp để được định lí đúng.
Cho đoạn thẳng AB nếu M là trung điểm của AB thì ...
A. M thuộc AB;
B. M không thuộc AB và cách đều A, B;
C. M thuộc AB và cách đều A, B;
D. M không thuộc AB.

\textbf{{ANSWER}}

Đáp án đúng là: C
Nếu M là trung điểm của đoạn thẳng AB thì M là điểm thuộc AB và cách đều A, B.
Vậy chọn đáp án C.

========================================================================

https://khoahoc.vietjack.com/thi-online/giai-bai-tap-hinh-hoc-12/23810


\textbf{{QUESTION}}

Các đỉnh, cạnh, mặt của một đa diện phải thỏa mãn những tính chất nào?

\textbf{{ANSWER}}

Các đỉnh, cạnh, mặt của một đa diện phải thỏa mãn những tính chất:
- Mỗi đỉnh là đỉnh chung của ít nhất ba cạnh, ba mặt;
- Mỗi cạnh là cạnh chung của đúng hai mặt;
- Hai mặt bất kì hoặc không có điểm chung, hoặc có một đỉnh chung, hoặc có đúng một cạnh chung.

========================================================================

https://khoahoc.vietjack.com/thi-online/20-cau-trac-nghiem-he-toa-do-trong-khong-gian-co-dap-an-thong-hieu


\textbf{{QUESTION}}

Trong không gian Oxyz, điểm N đối xứng với M(3,-1,2) qua trục Oy là:
A. $$ N\left(-3;1;-2\right)$$
B. $$ N\left(3;1;2\right)$$
C. $$ N\left(-3;-1;-2\right)$$
D. $$ N\left(3;-1;-2\right)$$

\textbf{{ANSWER}}

Trong không gian Oxyz, điểm N đối xứng với M(3,-1,2) qua trục Oy là $$ N\left(-3;-1;-2\right)$$
Đáp án cần chọn là: C

========================================================================

https://khoahoc.vietjack.com/thi-online/20-cau-trac-nghiem-he-toa-do-trong-khong-gian-co-dap-an-thong-hieu


\textbf{{QUESTION}}

Khi chiếu điểm M(-4,3,-2) lên trục Ox được điểm N thì:
A.$$ \overline{ON}=-4$$
B. $$ \overline{ON}=3$$
C.$$ \overline{ON}=4$$
D. $$ \overline{ON}=2$$

\textbf{{ANSWER}}

Khi chiếu điểm M(-4,3,-2) lên trục Ox được điểm N có tọa độ N (-4,0,0) nên $$ \overline{ON}=-4$$ 
Đáp án cần chọn là: A

========================================================================

https://khoahoc.vietjack.com/thi-online/20-cau-trac-nghiem-he-toa-do-trong-khong-gian-co-dap-an-thong-hieu


\textbf{{QUESTION}}

Trong không gian với hệ tọa độ Oxyz, cho điểm A(- 2,3,4) . Khoảng cách từ điểm A đến trục Ox là:
A.4
 B.3
C. 5
D.2

\textbf{{ANSWER}}

Gọi H là hình chiếu của A lên Ox  ⇒H(−2;0;0)⇒→AH=(0;−3;−4)$$ \Rightarrow H\left(-2;0;0\right)\Rightarrow \overrightarrow{AH}=\left(0;-3;-4\right)$$
Vậy khoảng cách từ A đến trục Ox là:  AH=√(−3)2+(−4)2=5$$ AH=\sqrt{{\left(-3\right)}^{2}+{\left(-4\right)}^{2}}=5$$
Đáp án cần chọn là: C

========================================================================

https://khoahoc.vietjack.com/thi-online/20-cau-trac-nghiem-he-toa-do-trong-khong-gian-co-dap-an-thong-hieu


\textbf{{QUESTION}}

Trong không gian với hệ tọa độ Oxyz, cho điểm A(1,2,3) . Tìm tọa độ điểm A1$$ {A}_{1}$$ là hình chiếu vuông góc của A lên mặt phẳng (Oyz)
A. A1(1;0;0)$$ {A}_{1}\left(1;0;0\right)$$
B. A1(0;2;3)$$ {A}_{1}\left(0;2;3\right)$$
C. A1(1;0;3)$$ {A}_{1}\left(1;0;3\right)$$
D. A1(1;2;0)$$ {A}_{1}\left(1;2;0\right)$$

\textbf{{ANSWER}}

Phương trình mặt phẳng (Oxz): x = 0.
Tọa độ điểm A1 là hình chiếu vuông góc của A lên mặt phẳng (Oyz) là:A1(0;2;3)$$ {A}_{1}\left(0;2;3\right)$$
Đáp án cần chọn là: B

========================================================================

https://khoahoc.vietjack.com/thi-online/20-cau-trac-nghiem-he-toa-do-trong-khong-gian-co-dap-an-thong-hieu


\textbf{{QUESTION}}

Trong không gian với hệ tọa độ Oxyz, cho điểm A(-3,2,-1). Tọa độ điểm A’ đối xứng với A qua gốc tọa độ O là:
A.A'(3;−2;1)$$ A\text{'}\left(3;-2;1\right)$$
B. A'(3;2;−1)$$ A\text{'}\left(3;2;-1\right)$$
C. A'(3;−2;−1)$$ A\text{'}\left(3;-2;-1\right)$$
D. A'(3;2;1)$$ A\text{'}\left(3;2;1\right)$$

\textbf{{ANSWER}}

Tọa độ điểm A’ đối xứng với A(−3;2;−1)$$ A\left(-3;2;-1\right)$$ qua gốc tọa độ O là:A'(3;−2;1)$$ A\text{'}\left(3;-2;1\right)$$
Đáp án cần chọn là: A

========================================================================

https://khoahoc.vietjack.com/thi-online/20-cau-trac-nghiem-toan-12-canh-dieu-bai-2-nguyen-ham-cua-mot-so-ham-so-so-cap-co-dap-an


\textbf{{QUESTION}}

I. Nhận biết
Chọn mệnh đề đúng trong các mệnh đề dưới đây.
A. $\int {{x^\alpha }dx = \frac{{{x^{\alpha + 1}}}}{{\alpha + 1}} + C{\rm{ }}\left( {\alpha \ne - 1} \right)} .$
B. $\int {{x^\alpha }dx = \frac{{{x^{\alpha + 1}}}}{{\alpha + 1}} + C{\rm{ }}} .$
C. $\int {{x^\alpha }dx = \frac{{{x^{\alpha - 1}}}}{{\alpha - 1}} + C{\rm{ }}\left( {\alpha \ne - 1} \right){\rm{. }}} $
D. $\int {{x^\alpha }dx = \frac{{{x^\alpha }}}{\alpha } + C{\rm{ }}} .$

\textbf{{ANSWER}}

Đáp án đúng là: A
Ta có: $\int {{x^\alpha }dx = \frac{{{x^{\alpha + 1}}}}{{\alpha + 1}} + C{\rm{ }}\left( {\alpha \ne - 1} \right)} .$

========================================================================

https://khoahoc.vietjack.com/thi-online/20-cau-trac-nghiem-toan-12-canh-dieu-bai-2-nguyen-ham-cua-mot-so-ham-so-so-cap-co-dap-an


\textbf{{QUESTION}}

Trong các khẳng định sau, khẳng định nào sai?
A. ∫f(x)g(x)dx=∫f(x)dx.∫g(x)dx.$\int {f\left( x \right)g\left( x \right)dx{\rm{ }}}  = \int {f\left( x \right)dx.\int {g\left( x \right)dx.} } $
B. ∫f′(x)dx=f(x)+C.$\int {f'\left( x \right)dx{\rm{ }}}  = f\left( x \right) + C.$
C. ∫sinxdx=−cosx+C.$\int {\sin xdx{\rm{ }}}  =  - \cos x + C.$
D. ∫1xdx=ln|x|+C.$\int {\frac{1}{x}dx{\rm{ }}}  = \ln \left| x \right| + C.$

\textbf{{ANSWER}}

Đáp án đúng là: A

========================================================================

https://khoahoc.vietjack.com/thi-online/20-cau-trac-nghiem-toan-12-canh-dieu-bai-2-nguyen-ham-cua-mot-so-ham-so-so-cap-co-dap-an


\textbf{{QUESTION}}

Hàm số F(x)$F\left( x \right)$ là một nguyên hàm của hàm số y=lnx$y = \ln x$ nếu
A. F′(x)=1lnx,∀x∈(0;+∞).$F'\left( x \right) = \frac{1}{{\ln x}},\forall x \in \left( {0; + \infty } \right).$
B. F′(x)=lnx,∀x∈(0;+∞).$F'\left( x \right) = \ln x,\forall x \in \left( {0; + \infty } \right).$
C. F′(x)=1x,∀x∈(0;+∞).$F'\left( x \right) = \frac{1}{x},\forall x \in \left( {0; + \infty } \right).$
D. F′(x)=1x∀x∈(−∞;0).$F'\left( x \right) = \frac{1}{x}\forall x \in \left( { - \infty ;0} \right).$

\textbf{{ANSWER}}

Đáp án đúng là: C

========================================================================

https://khoahoc.vietjack.com/thi-online/20-cau-trac-nghiem-toan-12-canh-dieu-bai-2-nguyen-ham-cua-mot-so-ham-so-so-cap-co-dap-an


\textbf{{QUESTION}}

Trong các mệnh đề sau, chọn mệnh đề đúng.
A. 1x+C$\frac{1}{x} + C$ là họ nguyên hàm của hàm số y=lnx$y = \ln x$ trên (0;+∞).$\left( {0; + \infty } \right).$ 
B.  3x2$3{x^2}$ là một nguyên hàm của hàm số y=x3$y = {x^3}$ trên (−∞;+∞).$\left( { - \infty ; + \infty } \right).$
C. 15x4$\frac{1}{5}{x^4}$ là một nguyên hàm của hàm số y=45x3.$y = \frac{4}{5}{x^3}.$
D. Hàm số y=2x$y = 2x$là nguyên hàm của hàm số y=x2.$y = {x^2}.$

\textbf{{ANSWER}}

Đáp án đúng là: C
Ta có: ∫45x3dx=15x4+C.$\int {\frac{4}{5}{x^3}dx = \frac{1}{5}{x^4} + C.} $
Với C = 0 thì ta có 15x4$\frac{1}{5}{x^4}$ là một nguyên hàm của hàm số y=45x3.$y = \frac{4}{5}{x^3}.$

========================================================================

https://khoahoc.vietjack.com/thi-online/20-cau-trac-nghiem-toan-12-canh-dieu-bai-2-nguyen-ham-cua-mot-so-ham-so-so-cap-co-dap-an


\textbf{{QUESTION}}

Công thức nguyên hàm nào sau đây là công thức sai?
A. ∫dxx=lnx+C.$\int {\frac{{dx}}{x} = \ln x + C.} $
C. ∫axdx=axlna+C(0<a≠1).$\int {{a^x}dx = \frac{{{a^x}}}{{\ln a}} + C{\rm{ }}\left( {0 < a \ne 1} \right).} $
D. ∫(n+1)xndx=nn+1+C(n∈Z+).$\int {\left( {n + 1} \right){x^n}dx = {n^{n + 1}} + C{\rm{ }}\left( {n \in {\mathbb{Z}^ + }} \right).} $

\textbf{{ANSWER}}

Đáp án đúng là: A
Ta có: ∫dxx=ln|x|+C$\int {\frac{{dx}}{x} = \ln \left| x \right| + C} $ với x≠0.$x \ne 0.$

========================================================================

https://khoahoc.vietjack.com/thi-online/15-cau-trac-nghiem-toan-7-ket-noi-tri-thuc-bai-3-luy-thua-voi-so-mu-tu-nhien-cua-1-so-huu-ti-co-dap


\textbf{{QUESTION}}

Trong số những khẳng định sau, khẳng định nào sai?
A. $$ {\text{x}}^{\text{m}}{\text{.x}}^{\text{n}}{\text{ = x}}^{m+n}$$;
B. $$ {\text{x}}^{\text{0}}\text{ = 1}$$;
C. $$ {\text{x}}^{1}\text{ = 1}$$;

\textbf{{ANSWER}}

Hướng dẫn giải
Đáp án đúng là: C
Theo quy ước: $$ {\text{x}}^{\text{0}}\text{ = 1}$$ với x ≠ 0; $$ {\text{x}}^{1}\text{ = 1}$$ nên khẳng định sai là C.

========================================================================

https://khoahoc.vietjack.com/thi-online/15-cau-trac-nghiem-toan-7-ket-noi-tri-thuc-bai-3-luy-thua-voi-so-mu-tu-nhien-cua-1-so-huu-ti-co-dap


\textbf{{QUESTION}}

Trong số những khẳng định sau, khẳng định nào sai?
A. (x.y)n = xn.yn$$ {\text{(x.y)}}^{\text{n}}{\text{ = x}}^{\text{n}}{\text{.y}}^{\text{n}}$$;
B. (xy)n= xnyn$$ {\left(\frac{\text{x}}{\text{y}}\right)}^{\text{n}}\text{= }\frac{{\text{x}}^{\text{n}}}{{\text{y}}^{\text{n}}}$$;
C. xm.xn = xm.n$$ {\text{x}}^{\text{m}}{\text{.x}}^{\text{n}}{\text{ = x}}^{\text{m.n}}$$;

\textbf{{ANSWER}}

Hướng dẫn giải
Đáp án đúng là: C
Theo định nghĩa: $$ {\text{x}}^{\text{m}}{\text{.x}}^{\text{n}}{\text{ = x}}^{m+n}$$ nên khẳng định sai là khẳng định C.

========================================================================

https://khoahoc.vietjack.com/thi-online/15-cau-trac-nghiem-toan-7-ket-noi-tri-thuc-bai-3-luy-thua-voi-so-mu-tu-nhien-cua-1-so-huu-ti-co-dap


\textbf{{QUESTION}}

Tính giá trị biểu thức (12)5.(12)3$$ {\left(\frac{1}{2}\right)}^{5}.{\left(\frac{1}{2}\right)}^{3}$$.
A. 1215$$ \frac{1}{{2}^{15}}$$;
B. 128$$ \frac{1}{{2}^{8}}$$;
C. 28$$ {2}^{8}$$;

\textbf{{ANSWER}}

Hướng dẫn giải
Đáp án đúng là: B
Theo định nghĩa: xm.xn = xm+n$$ {\text{x}}^{\text{m}}{\text{.x}}^{\text{n}}{\text{ = x}}^{\text{m+n}}$$ nên ta có (12)5.(12)3=(12)5+3=(12)8=128$$ {\left(\frac{1}{2}\right)}^{5}.{\left(\frac{1}{2}\right)}^{3}={\left(\frac{1}{2}\right)}^{5+3}={\left(\frac{1}{2}\right)}^{8}=\frac{1}{{2}^{8}}$$.
Vậy đáp án đúng là B.

========================================================================

https://khoahoc.vietjack.com/thi-online/15-cau-trac-nghiem-toan-7-ket-noi-tri-thuc-bai-3-luy-thua-voi-so-mu-tu-nhien-cua-1-so-huu-ti-co-dap


\textbf{{QUESTION}}

Tính giá trị biểu thức 22.794$$ \frac{{2}^{2}{.7}^{9}}{4}$$.
A. 711$$ {7}^{11}$$;
B. 79$$ {7}^{9}$$;
C. 792$$ \frac{{7}^{9}}{2}$$;

\textbf{{ANSWER}}

Hướng dẫn giải
Đáp án đúng là: B
Theo định nghĩa: (x.y)n = xn.yn$$ {\text{(x.y)}}^{\text{n}}{\text{ = x}}^{\text{n}}{\text{.y}}^{\text{n}}$$ nên ta có 22.794=22.7922=22.7922=79$$ \frac{{2}^{2}{.7}^{9}}{4}=\frac{{2}^{2}{.7}^{9}}{{2}^{2}}=\frac{{2}^{2}{.7}^{9}}{{2}^{2}}={7}^{9}$$.
Vậy đáp án đúng là B.

========================================================================

https://khoahoc.vietjack.com/thi-online/15-cau-trac-nghiem-toan-7-ket-noi-tri-thuc-bai-3-luy-thua-voi-so-mu-tu-nhien-cua-1-so-huu-ti-co-dap


\textbf{{QUESTION}}

Tính giá trị biểu thức 8.(23)4$$ 8.{\left({2}^{3}\right)}^{4}$$.
A. 214$$ {2}^{14}$$;
B. 210$$ {2}^{10}$$;
C. 215$$ {2}^{15}$$;

\textbf{{ANSWER}}

Hướng dẫn giải
Đáp án đúng là: C
Theo quy ước: (xm)n = xm.n$$ {{\text{(x}}^{\text{m}}\text{)}}^{\text{n}}{\text{ = x}}^{\text{m.n}}$$ và xm.xn = xm+n$$ {\text{x}}^{\text{m}}{\text{.x}}^{\text{n}}{\text{ = x}}^{m+n}$$ nên ta có 8.(23)4=23.23.4=23+12=215$$ 8.{\left({2}^{3}\right)}^{4}={2}^{3}{.2}^{3.4}={2}^{3+12}={2}^{15}$$.
Vậy đáp án đúng là C.

========================================================================

https://khoahoc.vietjack.com/thi-online/30-cau-trac-nghiem-toan-10-bai-on-tap-cuoi-chuong-1-co-dap-an-phan-2/105018


\textbf{{QUESTION}}

Cho A = {x ∈ ℝ | |x – m| ≤ 25}; B = {x ∈ ℝ | |x| ≥ 2020}.
Có bao nhiêu giá trị nguyên m thỏa mãn A ∩ B = ∅.
C. 3989;
D. 2020.

\textbf{{ANSWER}}

Hướng dẫn giải
Đáp án đúng là: C
Ta có: 
A = {x ∈ ℝ | |x – m| ≤ 25} ⟹ A = [m – 25; m + 25]
B = {x ∈ ℝ | |x| ≥ 2020} ⟹ B = (-∞; -2020] ∪ [2020; +∞)
Để A ∩ B = ∅ thì -2020 < m – 25 và m + 25 < 2020 (1)
Khi đó (1) ⟺$$ \left\{\begin{array}{c}m-25>-2020\\ m+25<2020\end{array}\right.\Leftrightarrow \left\{\begin{array}{c}m>-1995\\ m<1995\end{array}\right.$$ ⟹ -1995 < m < 1995.
Vậy có 3989 giá trị nguyên m thỏa mãn.

========================================================================

https://khoahoc.vietjack.com/thi-online/30-cau-trac-nghiem-toan-10-bai-on-tap-cuoi-chuong-1-co-dap-an-phan-2/105018


\textbf{{QUESTION}}

Cho mệnh đề chứa biến P(x) = {x ∈ ℤ : |x2 – 2x – 3| = x2 + |2x + 3|}. Trong đoạn [-2020; 2021] có bao nhiêu giá trị của x để mệnh đề chứa biến P(x) là mệnh đề đúng?
B. 2021;
C. 2023

\textbf{{ANSWER}}

Hướng dẫn giải
Đáp án đúng là: A
Số giá trị nguyên để mệnh đề P(x) là mệnh đề đúng chính là số nghiệm nguyên của phương trình |x2 – 2x – 3| = x2 + |2x + 3| (1).
+ Nếu x ≥   $$ -\frac{3}{2}$$ thì ta có:
(1) ⟺ |x2 – 2x – 3| = x2 + |2x + 3| ⇔  $$ \left[\begin{array}{c}{x}^{2}–\text{ }2x\text{ }–\text{ }3\text{ }=\text{ }{x}^{2}+\text{ }2x\text{ }+\text{ }3\\ -{x}^{2}\text{+ }2x\text{ + }3\text{ }=\text{ }{x}^{2}+\text{ }2x\text{ }+\text{ }3\end{array}\right.\Leftrightarrow \left[\begin{array}{c}x=-\frac{3}{2}\\ x=0\end{array}\right.$$  .Mà x ∈ ℤ và x ∈ [-2020; 2021] nên x = 0 thỏa mãn. 
+ Nếu x < $$ -\frac{3}{2}$$ thì ta có (1) ⟺ |x2 – 2x – 3| = x2 – 2x – 3. Sử dụng định nghĩa giá trị tuyệt đối, kết hợp với điều kiện, ta có nghiệm của (1) trong trường hợp này:
(1) ⇔  $$ \left\{\begin{array}{c}{x}^{2}–\text{ }2x\text{ }–\text{ }3\ge 0\\ x<-\frac{3}{2}\end{array}\right.\Leftrightarrow \left\{\begin{array}{c}\left[\begin{array}{c}x\le -1\\ x\ge 3\end{array}\right.\\ x<-\frac{3}{2}\end{array}\right.\Leftrightarrow x<-\frac{3}{2}$$
Mà x ∈ [-2020;2021] nên x ∈ {-2; -3; …; -2020}.
Do đó tập nghiệm của phương trình là S = {0; -2; -3; …; -2020}.
Vậy có 2020 số nguyên thỏa mãn yêu cầu bài toán.

========================================================================

https://khoahoc.vietjack.com/thi-online/10-bai-tap-ap-sdung-cong-thuc-luong-giac-vao-cac-bai-toan-rut-gon-chung-minh-dang-thuc-luong-giac-co


\textbf{{QUESTION}}

Giá trị biểu thức $$ \frac{\mathrm{sin}\frac{\pi }{15}cos\frac{\pi }{10}+\mathrm{sin}\frac{\pi }{10}cos\frac{\pi }{15}}{cos\frac{2\pi }{15}cos\frac{\pi }{5}-\mathrm{sin}\frac{2\pi }{15}\mathrm{sin}\frac{\pi }{5}}$$ bằng
A. 1;
B. −1;
C. $$ -\frac{3}{2}$$;

\textbf{{ANSWER}}

Hướng dẫn giải:
Đáp án đúng là: A
$$ \frac{\mathrm{sin}\frac{\pi }{15}cos\frac{\pi }{10}+\mathrm{sin}\frac{\pi }{10}cos\frac{\pi }{15}}{cos\frac{2\pi }{15}cos\frac{\pi }{5}-\mathrm{sin}\frac{2\pi }{15}\mathrm{sin}\frac{\pi }{5}}$$$$ =\frac{\mathrm{sin}\left(\frac{\pi }{15}+\frac{\pi }{10}\right)}{cos\left(\frac{2\pi }{15}+\frac{\pi }{5}\right)}=\frac{\mathrm{sin}\left(\frac{\pi }{6}\right)}{cos\left(\frac{\pi }{3}\right)}=\frac{\frac{1}{2}}{\frac{1}{2}}=1$$.

========================================================================

https://khoahoc.vietjack.com/thi-online/trac-nghiem-toan-6-bai-10-co-dap-an-nhan-hai-so-nguyen-va-tinh-chat


\textbf{{QUESTION}}

Kết quả của phép tính ( - 125).8 là:
A. 1000
B. −1000
C. −100
D. −10000

\textbf{{ANSWER}}

Đáp án cần chọn là: B
(−125).8=−(125.8)=−1000

========================================================================

https://khoahoc.vietjack.com/thi-online/trac-nghiem-toan-6-bai-10-co-dap-an-nhan-hai-so-nguyen-va-tinh-chat


\textbf{{QUESTION}}

Kết quả của phép tính ( - 25).16 là:
A. 100
B. −400
C. 400
D. 4000

\textbf{{ANSWER}}

Đáp án cần chọn là: B
(−25).16=−(25.16)=−400

========================================================================

https://khoahoc.vietjack.com/thi-online/trac-nghiem-toan-6-bai-10-co-dap-an-nhan-hai-so-nguyen-va-tinh-chat


\textbf{{QUESTION}}

Chọn câu sai
A. (−5).25=−125
B. 6.(−15)=−90    
C. 125.(−20)=−250  
D. 225.(−18)=−4050

\textbf{{ANSWER}}

Đáp án cần chọn là: C
Đáp án A: (−5).25=−125nên A đúng.
Đáp án B: 6.(−15)=−90nên B đúng.
Đáp án C: 125.(−20)=−2500≠−250nên C sai.
Đáp án D: 225.(−18)=−4050nên D đúng.

========================================================================

https://khoahoc.vietjack.com/thi-online/trac-nghiem-toan-6-bai-10-co-dap-an-nhan-hai-so-nguyen-va-tinh-chat


\textbf{{QUESTION}}

Chọn câu sai
A. (−6).20=−120
B. 14.(−5)=−80
C. (−35).8=−280
D. 25.(−20)=−500

\textbf{{ANSWER}}

Đáp án cần chọn là: B
Đáp án A: (−6).20=−120nên A đúng.
Đáp án B: 14.(−5)=−70nên B sai.
Đáp án C: (−35).8=−280nên C đúng.
Đáp án D: 25.(−20)=−500nên D đúng.

========================================================================

https://khoahoc.vietjack.com/thi-online/200-trac-nghiem-ung-dung-dao-ham-de-khao-sat-ham-so-nang-cao/1182


\textbf{{QUESTION}}

Tìm m để giá trị lớn nhất của hàm số $$ y=\left|{x}^{2}+2x+m-4\right|$$  trên đoạn [-2; 1]  đạt giá trị nhỏ nhất. Giá trị của m là
A. 4
B. 3
C. 1
D. 2

\textbf{{ANSWER}}

$$ y=\left|{x}^{2}+2x+m-4\right|=\left|{(x+1)}^{2}+m-5\right|\phantom{\rule{0ex}{0ex}}$$
Ta có: $$ {(x+1)}^{2}+m-5\quad \ge \quad m\quad -\quad 5\quad \forall m$$
$$ {(x+1)}^{2}+m-5=\quad m\quad -\quad 1$$ với m = 1
 Suy ra: $$ \left[{(x+1)}^{2}+m-5\right]\in \left[m-5;m-1\right]$$
Giá trị lớn nhất của hàm số  $$ y=\left|{x}^{2}+2x+m-4\right|$$ trên đoạn[ -2; 1] đạt giá trị nhỏ nhất khi
 $$ \left\{\begin{array}{l}m-5<0\\ \begin{array}{c}m-1>0\\ 5-m=m-1\end{array}\end{array}\right.\Leftrightarrow m=3\phantom{\rule{0ex}{0ex}}$$
Chọn B.

========================================================================

https://khoahoc.vietjack.com/thi-online/thu-tu-thuc-hien-cac-phep-tinh/57643


\textbf{{QUESTION}}

Thực hiện phép tính:
a, $$ 3.{5}^{2}-16:{2}^{2}$$
b, 15.141 + 59.15
c, $$ {2}^{3}.17-{2}^{3}.14$$
d, $$ 20-\left[30-{\left(5-1\right)}^{2}\right]$$

\textbf{{ANSWER}}

a, $$ 3.{5}^{2}-16:{2}^{2}$$ = 3.25 $$ -$$ 16:4 = 75$$ -$$4 = 71
b, 15.141 + 59.15 = 15.(141+59) = 15.200 = 3000
c, $$ {2}^{3}.17-{2}^{3}.14$$ = 8.17$$ -$$8.14 = 8.(17$$ -$$14) = 8.3 = 24
d, $$ 20-\left[30-{\left(5-1\right)}^{2}\right]$$ = $$ 20-\left(30-{4}^{2}\right)$$ = 20$$ -$$(30$$ -$$16) = 20$$ -$$14 = 6

========================================================================

https://khoahoc.vietjack.com/thi-online/thu-tu-thuc-hien-cac-phep-tinh/57643


\textbf{{QUESTION}}

Thực hiện phép tính:
a, 35-102:52+23.7$$ {3}^{5}-{10}^{2}:{5}^{2}+{2}^{3}.7$$
b, 36.57-36.35+64.62-64.40$$ 36.57-36.35+64.62-64.40$$

\textbf{{ANSWER}}

a, $$ {3}^{5}-{10}^{2}:{5}^{2}+{2}^{3}.7$$ = 243$$ -$$100:25+8.7 = 243$$ -$$4+56 = 239+56 = 295
b, $$ 36.57-36.35+64.62-64.40$$ = 36(57$$ -$$35)+64(62$$ -$$40)
= 36.22+64.22 = 22.(36+64) = 22.100 = 2200

========================================================================

https://khoahoc.vietjack.com/thi-online/thu-tu-thuc-hien-cac-phep-tinh/57643


\textbf{{QUESTION}}

Thực hiện phép tính:
a, 600:{450:[450-(4.53-23.52)]}$$ 600:\{450:\left[450-\left(4.{5}^{3}-{2}^{3}.{5}^{2}\right)\right]\}$$
b, 18.{420:6-[150-(68.2-23.5)]}$$ 18.\{420:6-\left[150-(68.2-{2}^{3}.5)\right]\}$$

\textbf{{ANSWER}}

a, 600:{450:[450-(4.53-23.52)]}$$ 600:\{450:\left[450-\left(4.{5}^{3}-{2}^{3}.{5}^{2}\right)\right]\}$$
= 600:{450:[450-$$ -$$(4.125-$$ -$$8.25)]} = 600:{450:[450-$$ -$$(500-$$ -$$200)]}
= 600:{450:[450-$$ -$$300]} = 600:{450:150} = 600:3 = 200
b, 18.{420:6-[150-(68.2-23.5)]}$$ 18.\{420:6-\left[150-(68.2-{2}^{3}.5)\right]\}$$
= 18.{70-$$ -$$[150-$$ -$$(68.2-$$ -$$8.5)]} = 18.{70-$$ -$$[150-$$ -$$(136-$$ -$$40)]}
= 18.{70-$$ -$$[150-$$ -$$96]} = 18.{70-$$ -$$54} = 18.16 = 288

========================================================================

https://khoahoc.vietjack.com/thi-online/thu-tu-thuc-hien-cac-phep-tinh/57643


\textbf{{QUESTION}}

Tính giá trị biểu thức sau bằng cách hợp lí:
a, A = 27.36+73.99+27.14-$$ -$$49.73
b, B = (45.10.56+255.28):(28.54+57.25)$$ \left({4}^{5}.10.{5}^{6}+{25}^{5}.{2}^{8}\right):\left({2}^{8}.{5}^{4}+{5}^{7}.{2}^{5}\right)$$

\textbf{{ANSWER}}

a, A = 27.36+73.99+27.14$$ -$$49.73
= 27(36+14)+73(99$$ -$$49)
= 27.50 + 73.50 = 50(27+73) = 50.100 = 5000
b, B = $$ \left({4}^{5}.10.{5}^{6}+{25}^{5}.{2}^{8}\right):\left({2}^{8}.{5}^{4}+{5}^{7}.{2}^{5}\right)$$
Ta có: $$ {4}^{5}={\left({2}^{2}\right)}^{5}={2}^{2.5}={2}^{10}$$; $$ {25}^{5}={\left({5}^{2}\right)}^{5}={5}^{2.5}={5}^{10}$$
=> $$ {4}^{5}.10.{5}^{6}+{25}^{5}.{2}^{8}={2}^{10}.2.5.{5}^{6}+{5}^{10}.{2}^{8}$$ = $$ {2}^{11}.{5}^{7}+{5}^{10}.{2}^{8}$$ = $$ {2}^{8}.{2}^{3}.{5}^{7}+{5}^{7}.{5}^{3}.{2}^{8}$$ = $$ {2}^{8}.{5}^{7}.\left({2}^{3}+{5}^{3}\right)$$
Ta lại có: $$ {2}^{8}.{5}^{4}+{5}^{7}.{2}^{5}={2}^{5}.{2}^{3}.{5}^{4}+{5}^{4}.{5}^{3}.{2}^{5}$$ = $$ {2}^{5}.{5}^{4}.\left({2}^{3}+{5}^{3}\right)$$
Suy ra B = $$ {2}^{8}.{5}^{7}.\left({2}^{3}+{5}^{3}\right)$$:[$$ {2}^{5}.{5}^{4}.\left({2}^{3}+{5}^{3}\right)$$] = $$ {2}^{8}.{5}^{7}:\left({2}^{5}.{5}^{4}\right)$$
= $$ \left({2}^{8}:{2}^{5}\right).\left({5}^{7}:{5}^{4}\right)$$ = $$ {2}^{3}.{5}^{3}={\left(2.5\right)}^{3}={10}^{3}=1000$$

========================================================================

https://khoahoc.vietjack.com/thi-online/thu-tu-thuc-hien-cac-phep-tinh/57643


\textbf{{QUESTION}}

Tìm x, biết:
a, 70-5(x-3)=45
b, 10+2x=45:43
c, 60-3(x-2)=51
d, 4x-20=25:23

\textbf{{ANSWER}}

a, 70-5(x-3)=45$$ 70-5(x-3)=45$$
⇔5.(x-3)=70-45$$ \Leftrightarrow 5.\left(x-3\right)=70-45$$
⇔5(x-3)=35$$ \Leftrightarrow 5\left(x-3\right)=35$$
⇔x-3=35:5⇔x-3=7$$ \Leftrightarrow x-3=35:5\Leftrightarrow x-3=7$$
⇔x=10$$ \Leftrightarrow x=10$$
b, 10+2x=45:43$$ 10+2x={4}^{5}:{4}^{3}$$
⇔10+2x=42⇔10+2x=16$$ \Leftrightarrow 10+2x={4}^{2}\Leftrightarrow 10+2x=16$$
⇔2x=4⇔x=2$$ \Leftrightarrow 2x=4\Leftrightarrow x=2$$
c, 60-3(x-2)=51$$ 60-3\left(x-2\right)=51$$
⇔3(x-2)=60-51$$ \Leftrightarrow 3\left(x-2\right)=60-51$$
⇔3(x-2)=9$$ \Leftrightarrow 3\left(x-2\right)=9$$
⇔x-2=3⇔x=5$$ \Leftrightarrow x-2=3\Leftrightarrow x=5$$
d, 4x-20=25:23$$ 4x-20={2}^{5}:{2}^{3}$$
⇔4x-20=22⇔4x-20=4$$ \Leftrightarrow 4x-20={2}^{2}\Leftrightarrow 4x-20=4$$
⇔4x=24⇔x=6$$ \Leftrightarrow 4x=24\Leftrightarrow x=6$$

========================================================================

https://khoahoc.vietjack.com/thi-online/11-cau-trac-nghiem-toan-6-chan-troi-sang-tao-bai-12-uoc-chung-uoc-chung-lon-nhat-co-dap-an


\textbf{{QUESTION}}

Số x là ước chung của số a và số b nếu:
A. x∈Ư(a) và x∈B(b)
B. x⊂Ư(a)và x⊂Ư(b)
C. x∈Ư(a) và  x∈Ư(b)  
D. x∉Ư(a) và x∉Ư(b)

\textbf{{ANSWER}}

Số x là ước chung của a, b nếu x vừa là ước của a vừa là ước của b.
Đáp án cần chọn là: C

========================================================================

https://khoahoc.vietjack.com/thi-online/11-cau-trac-nghiem-toan-6-chan-troi-sang-tao-bai-12-uoc-chung-uoc-chung-lon-nhat-co-dap-an


\textbf{{QUESTION}}

8 là ước chung của
A. 12 và 32
B. 24 và 56
C. 14 và 48
D. 18 và 24

\textbf{{ANSWER}}

24:8 = 3;
56:8 = 7
=>8 là ước chung của 24 và 56.
Đáp án cần chọn là: B

========================================================================

https://khoahoc.vietjack.com/thi-online/11-cau-trac-nghiem-toan-6-chan-troi-sang-tao-bai-12-uoc-chung-uoc-chung-lon-nhat-co-dap-an


\textbf{{QUESTION}}

Tìm ƯCLN(18; 60)
A. 6                 
B. 30                           
C. 12                         
D. 18

\textbf{{ANSWER}}

Ta có: 
18 = 2.32; 60 = 22.3.5
Nên ƯCLN(18;60) = 2.3 = 6
Đáp án cần chọn là: A

========================================================================

https://khoahoc.vietjack.com/thi-online/10-cau-trac-nghiem-toan-8-bai-1-lien-he-giua-thu-tu-va-phep-cong-co-dap-an-nhan-biet


\textbf{{QUESTION}}

Cho m bất kỳ, chọn câu đúng?
A. m - 3 > m - 4
B. m - 3 < m - 4
C. m - 3 = m - 4
D. Cả A, B, C đều sai

\textbf{{ANSWER}}

Đáp án A
Vì -3 > -4  nên “cộng vào hai vế của bất đẳng thức với cùng một số m bất kỳ” ta được m - 3 > m - 4

========================================================================

https://khoahoc.vietjack.com/thi-online/10-cau-trac-nghiem-toan-8-bai-1-lien-he-giua-thu-tu-va-phep-cong-co-dap-an-nhan-biet


\textbf{{QUESTION}}

Biết rằng m > n với m, n bất kỳ, chọn câu đúng?
A. m - 3 > n - 3
B. m - 3 < n - 3
C. m - 3 = n - 3
D. Cả A, B, C đều sai

\textbf{{ANSWER}}

Đáp án A
Vì m > n nên “cộng vào hai vế của bất đẳng thức với cùng một số -3” ta được:
m - 3 > n - 3

========================================================================

https://khoahoc.vietjack.com/thi-online/10-cau-trac-nghiem-toan-8-bai-1-lien-he-giua-thu-tu-va-phep-cong-co-dap-an-nhan-biet


\textbf{{QUESTION}}

Cho x - 3 ≤ y - 3, so sánh x và y. Chọn đáp án đúng nhất?
A. x < y
B. x = y
C. x > y
D. x ≤ y

\textbf{{ANSWER}}

Đáp án D
Cộng cả hai vế của bất đẳng thức x - 3 ≤ y - 3 với 3 ta được:
x - 3 ≤ y - 3 ⇒$$ \Rightarrow $$ x - 3 + 3 ≤ y - 3 + 3 ⇒$$ \Rightarrow $$ x ≤ y

========================================================================

https://khoahoc.vietjack.com/thi-online/10-cau-trac-nghiem-toan-8-bai-1-lien-he-giua-thu-tu-va-phep-cong-co-dap-an-nhan-biet


\textbf{{QUESTION}}

Cho x - 5 ≤ y - 5. So sánh x và y?
A. x < y
B. x = y
C. x > y
D. x ≤ y

\textbf{{ANSWER}}

Đáp án D
Cộng hai vế của bất đẳng thức x - 5 ≤ y - 5 với 5 ta được:
x - 5 + 5 ≤ y - 5 + 5 ⇒ x ≤ y

========================================================================

https://khoahoc.vietjack.com/thi-online/10-cau-trac-nghiem-toan-8-bai-1-lien-he-giua-thu-tu-va-phep-cong-co-dap-an-nhan-biet


\textbf{{QUESTION}}

Cho a > b khi đó
A. a - b > 0
B. a - b < 0
C. a - b = 0
D. a - b ≤ 0

\textbf{{ANSWER}}

Đáp án A
Từ a > b, cộng -b vào hai vế ta được a - b > b - b, tức là a - b > 0

========================================================================

https://khoahoc.vietjack.com/thi-online/de-thi-hoc-ki-1-toan-9-co-dap-an-nam-2022-2023/116666


\textbf{{QUESTION}}

Giải các phương trình và hệ phương trình sau:

\textbf{{ANSWER}}

Ta có : x2 + 7x = 0 <=>
 
 
x(x + 7) = 0 <=> 

x = 0 hoặc x = –7

========================================================================

https://khoahoc.vietjack.com/thi-online/de-thi-hoc-ki-1-toan-9-co-dap-an-nam-2022-2023/116666


\textbf{{QUESTION}}

b) 
 
x2 + x = 2√3(x + 1)
√3
√3
√3
√
√
3

3
3

\textbf{{ANSWER}}

Ta có : x2 + x = 2√3$$ \sqrt{3}$$(x + 1) x2 – (2√3$$ \sqrt{3}$$ – 1)x – 2  = 0

========================================================================

https://khoahoc.vietjack.com/thi-online/de-thi-hoc-ki-1-toan-9-co-dap-an-nam-2022-2023/116666


\textbf{{QUESTION}}

c) 
 
– x4 + 5x2 + 36 = 0

\textbf{{ANSWER}}

x4 – 5x2 –36 = 0, Đặt t = x2≥0. Phương trình đã cho có dạng: t2 – 5t – 36 = 0
 △= 25 – 4.1(–36) = 169 ⇒√Δ=13. PT có 2 nghiệm t = 9 (nhận) , t = – 4 < 0 (loại)

========================================================================

https://khoahoc.vietjack.com/thi-online/de-thi-hoc-ki-1-toan-9-co-dap-an-nam-2022-2023/116666


\textbf{{QUESTION}}

{2x−3y=193x+4y=−14
{2x−3y=193x+4y=−14
{2x−3y=193x+4y=−14
{
{
2x−3y=193x+4y=−14
2x−3y=193x+4y=−14
2x−3y=19
2x−3y=19
2
x
−
3
y
=
19
3x+4y=−14
3x+4y=−14
3
x
+
4
y
=
−
14

\textbf{{ANSWER}}

{2x−3y=193x+4y=−14⇔{8x−12y=769x+12y=−42⇔{17x=342x−3y=19⇔{x=2y=−5$$ \left\{\begin{array}{l}2x-3y=19\\ 3x+4y=-14\end{array}\right.\Leftrightarrow \left\{\begin{array}{l}8x-12y=76\\ 9x+12y=-42\end{array}\right.\Leftrightarrow \left\{\begin{array}{l}17x=34\\ 2x-3y=19\end{array}\right.\Leftrightarrow \left\{\begin{array}{l}x=2\\ y=-5\end{array}\right.$$

========================================================================

https://khoahoc.vietjack.com/thi-online/15-cau-trac-nghiem-toan-7-ket-noi-tri-thuc-bai-6-so-vo-ti-can-bac-hai-so-hoc-co-dap-an-phan-2/111173


\textbf{{QUESTION}}

Tính: A = −4$$ \sqrt{\frac{1}{16}}$$ + 3$$ \sqrt{\frac{1}{9}}$$ − 5$$ \sqrt{0,04}$$ bằng
A. 2;
B. 1;
C. −1;

\textbf{{ANSWER}}

Hướng dẫn giải
Đáp án đúng là: C
Ta có: $$ \sqrt{\frac{1}{16}}=\frac{1}{4}$$ ; $$ \sqrt{\frac{1}{9}}=\frac{1}{3}$$ ; $$ \sqrt{0,04}$$ = 0,2 = $$ \frac{1}{5}$$         
Vậy A = −4.$$ \frac{1}{4}$$ + 3.$$ \frac{1}{3}$$ − 5.$$ \frac{1}{5}$$ = −1 + 1 – 1 = −1

========================================================================

https://khoahoc.vietjack.com/thi-online/15-cau-trac-nghiem-toan-7-ket-noi-tri-thuc-bai-6-so-vo-ti-can-bac-hai-so-hoc-co-dap-an-phan-2/111173


\textbf{{QUESTION}}

Giá trị của x không âm trong biểu thức x2 – 64 = 0 bằng:
A. x = 3;
B. x = 4;
C. x = 6;

\textbf{{ANSWER}}

Hướng dẫn giải
Đáp án đúng là: D
x2 – 64 = 0 
Û x2 = 64 
Û x = 8 hoặc x = −8 
Vì x không âm nên x = 8 thỏa mãn.
Vậy x = 8.

========================================================================

https://khoahoc.vietjack.com/thi-online/15-cau-trac-nghiem-toan-7-ket-noi-tri-thuc-bai-6-so-vo-ti-can-bac-hai-so-hoc-co-dap-an-phan-2/111173


\textbf{{QUESTION}}

Cho A = 3 và B = √12. Mệnh đề nào đúng?
A. A > B;
B. A < B;
C. A = B;
D. A ≤ B.

\textbf{{ANSWER}}

Hướng dẫn giải
Đáp án đúng là: B
Ta có: 3 = √9$$ \sqrt{9}$$ mà 9 < 12 nên √9<√12$$ \sqrt{9}<\sqrt{12}$$ 
Do đó 3 < √12$$ \sqrt{12}$$ hay A < B.

========================================================================

https://khoahoc.vietjack.com/thi-online/20-cau-trac-nghiem-toan-10-chan-troi-sang-tao-dinh-li-cosin-va-dinh-li-sin-co-dap-an-phan-2


\textbf{{QUESTION}}

Cho tam giác ABC bất kì có BC = a, AC = b và AB = c. Đẳng thức nào đúng?

\textbf{{ANSWER}}

Hướng dẫn giải
Đáp án đúng là: D
Theo định lí côsin ta có:
a2 = b2 + c2 – 2bc.cosA;
b2 = a2 + c2 – 2ac.cosB;
c2 = b2 + a2 – 2ab.cosC.
Do đó phương án D là đúng.
Vậy ta chọn phương án D.

========================================================================

https://khoahoc.vietjack.com/thi-online/bai-tap-lam-tron-so-va-uoc-luong-co-dap-an


\textbf{{QUESTION}}

Một bồn hoa có dạng hình tròn với bán kính 0,8m. 
Hỏi diện tích của bồn hoa khoảng bao nhiêu mét vuông?

\textbf{{ANSWER}}

Diện tích bồn hoa là: 0,8.0,8.3,14 = 2,0096 m2
Vậy diện tích bồn hoa khoảng 2m2.

========================================================================

https://khoahoc.vietjack.com/thi-online/bai-tap-lam-tron-so-va-uoc-luong-co-dap-an


\textbf{{QUESTION}}

Hóa đơn tiền điện tháng 9/2020 của gia đình cô Hạnh là 574 880 đồng. Trong thực tế, cô Hạnh đã trả tiền mặt cho người thu tiền điện số tiền là 575 000 đồng. Tại sao cô Hạnh không thể trả cho người thu tiền điện số tiền chính xác là 574 880 đồng?

\textbf{{ANSWER}}

Trên thực tế, hiện nay các đồng tiền có mệnh giá nhỏ nhất là 200 đồng nên số tiền 880 đồng sẽ không có mệnh giá tiền nào phù hợp để trả.
Do vậy cô Hạnh không thể trả chính xác số tiền 574 880 đồng bằng tiền mặt.

========================================================================

https://khoahoc.vietjack.com/thi-online/bai-tap-lam-tron-so-va-uoc-luong-co-dap-an


\textbf{{QUESTION}}

Khoảng cách từ sân vận động Old Trafford ở Greater Manchester đến tháp đồng hồ Big Ben ở London (Vương Quốc Anh) là 201 dặm. (Nguồn: https://www.google.com). Tính khoảng cách đó theo đơn vị ki-lô-mét (làm tròn đến hàng đơn vị), biết 1 dặm = 1,609344 km.

\textbf{{ANSWER}}

Khoảng cách từ sân vận động Old Trafford ở Greater Manchester đến tháp đồng hồ Big Ben ở London (Vương Quốc Anh) theo đơn vị ki-lô-mét là:
201 . 1,609344 = 323,478144 ≈ 323 (km).

========================================================================

https://khoahoc.vietjack.com/thi-online/11-cau-trac-nghiem-toan-6-chan-troi-sang-tao-bai-12-uoc-chung-uoc-chung-lon-nhat-co-dap-an


\textbf{{QUESTION}}

Số x là ước chung của số a và số b nếu:
A. x∈Ư(a) và x∈B(b)
B. x⊂Ư(a)và x⊂Ư(b)
C. x∈Ư(a) và  x∈Ư(b)  
D. x∉Ư(a) và x∉Ư(b)

\textbf{{ANSWER}}

Số x là ước chung của a, b nếu x vừa là ước của a vừa là ước của b.
Đáp án cần chọn là: C

========================================================================

https://khoahoc.vietjack.com/thi-online/11-cau-trac-nghiem-toan-6-chan-troi-sang-tao-bai-12-uoc-chung-uoc-chung-lon-nhat-co-dap-an


\textbf{{QUESTION}}

8 là ước chung của
A. 12 và 32
B. 24 và 56
C. 14 và 48
D. 18 và 24

\textbf{{ANSWER}}

24:8 = 3;
56:8 = 7
=>8 là ước chung của 24 và 56.
Đáp án cần chọn là: B

========================================================================

https://khoahoc.vietjack.com/thi-online/11-cau-trac-nghiem-toan-6-chan-troi-sang-tao-bai-12-uoc-chung-uoc-chung-lon-nhat-co-dap-an


\textbf{{QUESTION}}

Tìm ƯCLN(18; 60)
A. 6                 
B. 30                           
C. 12                         
D. 18

\textbf{{ANSWER}}

Ta có: 
18 = 2.32; 60 = 22.3.5
Nên ƯCLN(18;60) = 2.3 = 6
Đáp án cần chọn là: A

========================================================================

https://khoahoc.vietjack.com/thi-online/10-cau-trac-nghiem-toan-8-bai-1-lien-he-giua-thu-tu-va-phep-cong-co-dap-an-nhan-biet


\textbf{{QUESTION}}

Cho m bất kỳ, chọn câu đúng?
A. m - 3 > m - 4
B. m - 3 < m - 4
C. m - 3 = m - 4
D. Cả A, B, C đều sai

\textbf{{ANSWER}}

Đáp án A
Vì -3 > -4  nên “cộng vào hai vế của bất đẳng thức với cùng một số m bất kỳ” ta được m - 3 > m - 4

========================================================================

https://khoahoc.vietjack.com/thi-online/10-cau-trac-nghiem-toan-8-bai-1-lien-he-giua-thu-tu-va-phep-cong-co-dap-an-nhan-biet


\textbf{{QUESTION}}

Biết rằng m > n với m, n bất kỳ, chọn câu đúng?
A. m - 3 > n - 3
B. m - 3 < n - 3
C. m - 3 = n - 3
D. Cả A, B, C đều sai

\textbf{{ANSWER}}

Đáp án A
Vì m > n nên “cộng vào hai vế của bất đẳng thức với cùng một số -3” ta được:
m - 3 > n - 3

========================================================================

https://khoahoc.vietjack.com/thi-online/10-cau-trac-nghiem-toan-8-bai-1-lien-he-giua-thu-tu-va-phep-cong-co-dap-an-nhan-biet


\textbf{{QUESTION}}

Cho x - 3 ≤ y - 3, so sánh x và y. Chọn đáp án đúng nhất?
A. x < y
B. x = y
C. x > y
D. x ≤ y

\textbf{{ANSWER}}

Đáp án D
Cộng cả hai vế của bất đẳng thức x - 3 ≤ y - 3 với 3 ta được:
x - 3 ≤ y - 3 ⇒ x - 3 + 3 ≤ y - 3 + 3 ⇒ x ≤ y

========================================================================

https://khoahoc.vietjack.com/thi-online/10-cau-trac-nghiem-toan-8-bai-1-lien-he-giua-thu-tu-va-phep-cong-co-dap-an-nhan-biet


\textbf{{QUESTION}}

Cho x - 5 ≤ y - 5. So sánh x và y?
A. x < y
B. x = y
C. x > y
D. x ≤ y

\textbf{{ANSWER}}

Đáp án D
Cộng hai vế của bất đẳng thức x - 5 ≤ y - 5 với 5 ta được:
x - 5 + 5 ≤ y - 5 + 5 ⇒ x ≤ y

========================================================================

https://khoahoc.vietjack.com/thi-online/10-cau-trac-nghiem-toan-8-bai-1-lien-he-giua-thu-tu-va-phep-cong-co-dap-an-nhan-biet


\textbf{{QUESTION}}

Cho a > b khi đó
A. a - b > 0
B. a - b < 0
C. a - b = 0
D. a - b ≤ 0

\textbf{{ANSWER}}

Đáp án A
Từ a > b, cộng -b vào hai vế ta được a - b > b - b, tức là a - b > 0

========================================================================

https://khoahoc.vietjack.com/thi-online/de-thi-hoc-ki-1-toan-9-co-dap-an-nam-2022-2023/116666


\textbf{{QUESTION}}

Giải các phương trình và hệ phương trình sau:

\textbf{{ANSWER}}

Ta có : x2 + 7x = 0 <=>
 
 
x(x + 7) = 0 <=> 

x = 0 hoặc x = –7

========================================================================

https://khoahoc.vietjack.com/thi-online/de-thi-hoc-ki-1-toan-9-co-dap-an-nam-2022-2023/116666


\textbf{{QUESTION}}

b) 
 
x2 + x = 2√3(x + 1)
√3
√3
√3
√
√
3

3
3

\textbf{{ANSWER}}

Ta có : x2 + x = 2√3$$ \sqrt{3}$$(x + 1) x2 – (2√3$$ \sqrt{3}$$ – 1)x – 2  = 0

========================================================================

https://khoahoc.vietjack.com/thi-online/de-thi-hoc-ki-1-toan-9-co-dap-an-nam-2022-2023/116666


\textbf{{QUESTION}}

c) 
 
– x4 + 5x2 + 36 = 0

\textbf{{ANSWER}}

x4 – 5x2 –36 = 0, Đặt t = x2≥$$ \ge $$0. Phương trình đã cho có dạng: t2 – 5t – 36 = 0
 △$$ \triangle $$= 25 – 4.1(–36) = 169 ⇒√Δ=13$$ \Rightarrow \sqrt{\Delta }=13$$. PT có 2 nghiệm t = 9 (nhận) , t = – 4 < 0 (loại)

========================================================================

https://khoahoc.vietjack.com/thi-online/de-thi-hoc-ki-1-toan-9-co-dap-an-nam-2022-2023/116666


\textbf{{QUESTION}}

{2x−3y=193x+4y=−14
{2x−3y=193x+4y=−14
{2x−3y=193x+4y=−14
{
{
2x−3y=193x+4y=−14
2x−3y=193x+4y=−14
2x−3y=19
2x−3y=19
2
x
−
3
y
=
19
3x+4y=−14
3x+4y=−14
3
x
+
4
y
=
−
14

\textbf{{ANSWER}}

{2x−3y=193x+4y=−14⇔{8x−12y=769x+12y=−42⇔{17x=342x−3y=19⇔{x=2y=−5$$ \left\{\begin{array}{l}2x-3y=19\\ 3x+4y=-14\end{array}\right.\Leftrightarrow \left\{\begin{array}{l}8x-12y=76\\ 9x+12y=-42\end{array}\right.\Leftrightarrow \left\{\begin{array}{l}17x=34\\ 2x-3y=19\end{array}\right.\Leftrightarrow \left\{\begin{array}{l}x=2\\ y=-5\end{array}\right.$$

========================================================================

https://khoahoc.vietjack.com/thi-online/de-thi-hoc-ki-1-toan-9-co-dap-an-nam-2022-2023/116666


\textbf{{QUESTION}}

Cho phương trình: x2 – (m + 5)x + 2m + 6 = 0 (x là ẩn số). 
a) Chứng minh rằng, phương trình đã cho luôn luôn có hai nghiệm với mọi giá trị của m.

\textbf{{ANSWER}}

Xét phương trình : x2 – (m + 5)x + 2m + 6 = 0 (x là ẩn số). 
Ta có: △ = (m + 5)2 – 4(2m + 6) = (m + 1)2 ≥0, ∀m

========================================================================

https://khoahoc.vietjack.com/thi-online/15-cau-trac-nghiem-toan-7-ket-noi-tri-thuc-bai-6-so-vo-ti-can-bac-hai-so-hoc-co-dap-an-phan-2/111173


\textbf{{QUESTION}}

Tính: A = −4$$ \sqrt{\frac{1}{16}}$$ + 3$$ \sqrt{\frac{1}{9}}$$ − 5$$ \sqrt{0,04}$$ bằng
A. 2;
B. 1;
C. −1;

\textbf{{ANSWER}}

Hướng dẫn giải
Đáp án đúng là: C
Ta có: $$ \sqrt{\frac{1}{16}}=\frac{1}{4}$$ ; $$ \sqrt{\frac{1}{9}}=\frac{1}{3}$$ ; $$ \sqrt{0,04}$$ = 0,2 = $$ \frac{1}{5}$$         
Vậy A = −4.$$ \frac{1}{4}$$ + 3.$$ \frac{1}{3}$$ − 5.$$ \frac{1}{5}$$ = −1 + 1 – 1 = −1

========================================================================

https://khoahoc.vietjack.com/thi-online/15-cau-trac-nghiem-toan-7-ket-noi-tri-thuc-bai-6-so-vo-ti-can-bac-hai-so-hoc-co-dap-an-phan-2/111173


\textbf{{QUESTION}}

Giá trị của x không âm trong biểu thức x2 – 64 = 0 bằng:
A. x = 3;
B. x = 4;
C. x = 6;

\textbf{{ANSWER}}

Hướng dẫn giải
Đáp án đúng là: D
x2 – 64 = 0 
Û x2 = 64 
Û x = 8 hoặc x = −8 
Vì x không âm nên x = 8 thỏa mãn.
Vậy x = 8.

========================================================================

https://khoahoc.vietjack.com/thi-online/15-cau-trac-nghiem-toan-7-ket-noi-tri-thuc-bai-6-so-vo-ti-can-bac-hai-so-hoc-co-dap-an-phan-2/111173


\textbf{{QUESTION}}

Cho A = 3 và B = √12. Mệnh đề nào đúng?
A. A > B;
B. A < B;
C. A = B;
D. A ≤ B.

\textbf{{ANSWER}}

Hướng dẫn giải
Đáp án đúng là: B
Ta có: 3 = √9 mà 9 < 12 nên √9<√12 
Do đó 3 < √12 hay A < B.

========================================================================

https://khoahoc.vietjack.com/thi-online/20-cau-trac-nghiem-toan-10-chan-troi-sang-tao-dinh-li-cosin-va-dinh-li-sin-co-dap-an-phan-2


\textbf{{QUESTION}}

Cho tam giác ABC bất kì có BC = a, AC = b và AB = c. Đẳng thức nào đúng?

\textbf{{ANSWER}}

Hướng dẫn giải
Đáp án đúng là: D
Theo định lí côsin ta có:
a2 = b2 + c2 – 2bc.cosA;
b2 = a2 + c2 – 2ac.cosB;
c2 = b2 + a2 – 2ab.cosC.
Do đó phương án D là đúng.
Vậy ta chọn phương án D.

========================================================================

https://khoahoc.vietjack.com/thi-online/bai-tap-lam-tron-so-va-uoc-luong-co-dap-an


\textbf{{QUESTION}}

Một bồn hoa có dạng hình tròn với bán kính 0,8m. 
Hỏi diện tích của bồn hoa khoảng bao nhiêu mét vuông?

\textbf{{ANSWER}}

Diện tích bồn hoa là: 0,8.0,8.3,14 = 2,0096 m2
Vậy diện tích bồn hoa khoảng 2m2.

========================================================================

https://khoahoc.vietjack.com/thi-online/bai-tap-lam-tron-so-va-uoc-luong-co-dap-an


\textbf{{QUESTION}}

Hóa đơn tiền điện tháng 9/2020 của gia đình cô Hạnh là 574 880 đồng. Trong thực tế, cô Hạnh đã trả tiền mặt cho người thu tiền điện số tiền là 575 000 đồng. Tại sao cô Hạnh không thể trả cho người thu tiền điện số tiền chính xác là 574 880 đồng?

\textbf{{ANSWER}}

Trên thực tế, hiện nay các đồng tiền có mệnh giá nhỏ nhất là 200 đồng nên số tiền 880 đồng sẽ không có mệnh giá tiền nào phù hợp để trả.
Do vậy cô Hạnh không thể trả chính xác số tiền 574 880 đồng bằng tiền mặt.

========================================================================

https://khoahoc.vietjack.com/thi-online/bai-tap-lam-tron-so-va-uoc-luong-co-dap-an


\textbf{{QUESTION}}

Khoảng cách từ sân vận động Old Trafford ở Greater Manchester đến tháp đồng hồ Big Ben ở London (Vương Quốc Anh) là 201 dặm. (Nguồn: https://www.google.com). Tính khoảng cách đó theo đơn vị ki-lô-mét (làm tròn đến hàng đơn vị), biết 1 dặm = 1,609344 km.

\textbf{{ANSWER}}

Khoảng cách từ sân vận động Old Trafford ở Greater Manchester đến tháp đồng hồ Big Ben ở London (Vương Quốc Anh) theo đơn vị ki-lô-mét là:
201 . 1,609344 = 323,478144 ≈ 323 (km).

========================================================================

https://khoahoc.vietjack.com/thi-online/10-bai-tap-bai-toan-thuc-tien-lien-quan-den-ham-so-mu-va-ham-so-logarit-co-loi-giai


\textbf{{QUESTION}}

Số lượng của loại vi khuẩn A trong một phòng thí nghiệm được tính theo công thức s(t) = s(0).2t, trong đó s(0) là số lượng vi khuẩn A lúc ban đầu, s(t) là số lượng vi khuẩn A có sau t phút. Biết sau 3 phút thì số lượng vi khuẩn A là 625 nghìn con. Hỏi sau bao lâu, kể từ lúc ban đầu, số lượng vi khuẩn A là 10 triệu con?

\textbf{{ANSWER}}

Đáp án đúng là: C
Sau 3 phút thì số lượng vi khuẩn A là 625 nghìn con nên ta có:
625 000 = s(0).23 Þ s(0) = 78125 (con vi khuẩn).
Tại thời điểm t số lượng vi khuẩn A là 10 triệu con nên ta có:
 $$ s\left(t\right)=s\left(0\right){.2}^{t}\Leftrightarrow {2}^{t}=\frac{s\left(t\right)}{s\left(0\right)}\Leftrightarrow {2}^{t}=\frac{10\text{\hspace{0.33em}}000\text{\hspace{0.33em}}000}{78\text{\hspace{0.33em}}125}\Leftrightarrow {2}^{t}=128\Leftrightarrow t=7$$.
Vậy sau 7 phút kể từ lúc ban đầu, số lượng vi khuẩn A là 10 triệu con.

========================================================================

https://khoahoc.vietjack.com/thi-online/10-bai-tap-bai-toan-thuc-tien-lien-quan-den-ham-so-mu-va-ham-so-logarit-co-loi-giai


\textbf{{QUESTION}}

Bà Mai gửi tiết kiệm ngân hàng Vietcombank số tiền 50 triệu đồng với lãi suất 0,79% một tháng, theo phương thức lãi kép. Tính số tiền cả vốn lẫn lãi bà Mai nhận được sau 2 năm? (làm tròn đến hàng nghìn)

\textbf{{ANSWER}}

Đáp án đúng là: A
Nhận xét: đây là bài toán lãi kép với chu kỳ là một tháng.
Áp dụng công thức: S = A . (1 + r)n với A = 50 triệu đồng, r% = 0,79% và n = 24 tháng.
Ta có: 
S = 50 . (1 + 0,0079)24 » 60,393 (triệu đồng) = 60 393 000 đồng.

========================================================================

https://khoahoc.vietjack.com/thi-online/10-bai-tap-bai-toan-thuc-tien-lien-quan-den-ham-so-mu-va-ham-so-logarit-co-loi-giai


\textbf{{QUESTION}}

Bạn An gửi tiết kiệm một số tiền ban đầu là 1 000 000 đồng với lãi suất 0,58%/tháng (không kỳ hạn). Hỏi bạn An phải gửi bao nhiêu tháng thì được cả vốn lẫn lãi bằng hoặc vượt quá 1 300 000 đồng?

\textbf{{ANSWER}}

Đáp án đúng là: A
Ta có:  n=log1,0058(13000001000000)≈45,3662737(tháng).
Để nhận được số tiền cả vốn lẫn lãi bằng hoặc vượt quá 1 300 000 đồng thì bạn An phải gửi ít nhất là 46 tháng.

========================================================================

https://khoahoc.vietjack.com/thi-online/10-bai-tap-bai-toan-thuc-tien-lien-quan-den-ham-so-mu-va-ham-so-logarit-co-loi-giai


\textbf{{QUESTION}}

Một người được lãnh lương khởi điểm là 3 triệu đồng/tháng. Cứ 3 tháng thì lương người đó được tăng thêm  7%/tháng. Hỏi sau 36 năm người đó lĩnh được tất cả số tiền là bao nhiêu?

\textbf{{ANSWER}}

Đáp án đúng là: B
Áp dụng công thức: Tổng số tiền nhận được sau kn tháng là  Skn=Ak(1+r)k−1r.
 S36=3.106.12.(1,07)12−10,07≈643984245,8≈644 000 000 (đồng)
Vậy sau 36 năm người đó lĩnh được tất cả số tiền là 644 triệu đồng.

========================================================================

https://khoahoc.vietjack.com/thi-online/10-bai-tap-bai-toan-thuc-tien-lien-quan-den-ham-so-mu-va-ham-so-logarit-co-loi-giai


\textbf{{QUESTION}}

Một người có 58 000 000đ gửi tiết kiệm ngân hàng (theo hình thức lãi kép ) trong 8 tháng thì lĩnh về được 61 329 000đ. Lãi suất hàng tháng là:

\textbf{{ANSWER}}

Đáp án đúng là: B
Áp dụng công thức lãi kép ta được: 
 r%=61329000580000008-1≈0,7%
Vậy lãi suất hàng tháng là 0,7%.

========================================================================

https://khoahoc.vietjack.com/thi-online/12-bai-tap-giai-phuong-trinh-chua-an-o-mau-co-loi-giai


\textbf{{QUESTION}}

Phương trình $\frac{4}{{x - 1}} = \frac{x}{{x - 2}}$ có nghiệm là
A. Phương trình vô nghiệm.
B. Phương trình vô số nghiệm.
C. x = 1.
D. x = 2.

\textbf{{ANSWER}}

Đáp án đúng là: A
Điều kiện xác định: x ≠ 1 và x ≠ 2.
Giải phương trình, ta có:
$\frac{4}{{x - 1}} = \frac{x}{{x - 2}}$ suy ra $\frac{{4\left( {x - 2} \right)}}{{\left( {x - 1} \right)\left( {x - 2} \right)}} = \frac{{x\left( {x - 1} \right)}}{{\left( {x - 1} \right)\left( {x - 2} \right)}}$
Suy ra 4(x – 2) = x(x – 1) hay x2 – 5x + 8 = 0.
Ta có: ∆ = 52 – 4.8 = 25 – 32 = −6 < 0.
Do đó, phương trình x2 – 5x + 8 = 0 vô nghiệm.
Vậy phương trình $\frac{4}{{x - 1}} = \frac{x}{{x - 2}}$ vô nghiệm.

========================================================================

https://khoahoc.vietjack.com/thi-online/12-bai-tap-giai-phuong-trinh-chua-an-o-mau-co-loi-giai


\textbf{{QUESTION}}

Nghiệm của phương trình 11x=9x+1+2x−4 là
A. x = 44.
B. x = 22.
C. x = −44.
D. x = 4.

\textbf{{ANSWER}}

Đáp án đúng là: A
Điều kiện xác định: x ≠ 0, x ≠ −1, x ≠ 4.
Giải phương trình, ta có: 11x=9x+1+2x−4$\frac{{11}}{x} = \frac{9}{{x + 1}} + \frac{2}{{x - 4}}$
11.(x+1).(x−4)x(x+1).(x−4)=9x(x−4)x(x+1).(x−4)+2x(x+1)x(x+1).(x−4)$\frac{{11.\left( {x + 1} \right).\left( {x - 4} \right)}}{{x\left( {x + 1} \right).\left( {x - 4} \right)}} = \frac{{9x\left( {x - 4} \right)}}{{x\left( {x + 1} \right).\left( {x - 4} \right)}} + \frac{{2x\left( {x + 1} \right)}}{{x\left( {x + 1} \right).\left( {x - 4} \right)}}$
11(x + 1)(x – 4) = 9x(x – 4) + 2x(x + 1)
11(x2 – 3x – 4) = 9x2 – 36x + 2x2 + 2x
11x2 – 33x – 44 = 11x2 – 34x
x – 44 = 0
x = 44 (thỏa mãn).
Vậy phương trình có nghiệm là x = 44.

========================================================================

https://khoahoc.vietjack.com/thi-online/12-bai-tap-giai-phuong-trinh-chua-an-o-mau-co-loi-giai


\textbf{{QUESTION}}

Phương trình x−1x−2−3+x=1x−2 có nghiệm là
A. x = 4 hoặc x = −2.
B. x = −4 hoặc x = 2.
C. x = −4.
D. x = 2.

\textbf{{ANSWER}}

Đáp án đúng là: C
Điều kiện xác định: x ≠ 2.
Ta có: x−1x−2−3+x=1x−2
x−1+(x−3)(x−2)x−2=1x−2
x – 1 + (x + 3)(x – 2) = 1
x – 1 + x2 + x – 6 – 1 = 0
x2 + 2x – 8 = 0
x2 – 2x + 4x – 8 = 0
x(x – 2) + 4(x – 2) = 0
(x + 4)(x – 2) = 0
Suy ra x + 4 = 0 hoặc x – 2 = 0 khi x = −4 (thỏa mãn) hoặc x = 2 (loại).
Vậy nghiệm của phương trình là x = −4.

========================================================================

https://khoahoc.vietjack.com/thi-online/12-bai-tap-giai-phuong-trinh-chua-an-o-mau-co-loi-giai


\textbf{{QUESTION}}

Phương trình 34(x−5)+1550−2x2=76x+30 có nghiệm là
A. x = 5.
B. x = −5.
C. Phương trình vô nghiệm.
D. Phương trình vô số nghiệm.

\textbf{{ANSWER}}

Đáp án đúng là: C
Điều kiện xác định: x ≠ 5 và x ≠ −5.
Ta có: 34(x−5)+1550−2x2=76x+30
34(x−5)−152(x−5)(x+5)=76(x+5)
9(x+5)12(x−5)(x+5)−15.612(x−5)(x+5)=14(x−5)6(x+5)(x−5)
9(x + 5) – 90 = 14(x – 5)
9x + 45 – 90 = 14x – 70
5x = 25
x = 5 (loại).
Vậy phương trình không có nghiệm thỏa mãn.

========================================================================

https://khoahoc.vietjack.com/thi-online/12-bai-tap-giai-phuong-trinh-chua-an-o-mau-co-loi-giai


\textbf{{QUESTION}}

Nghiệm của phương trình 2x2−4−1x(x−2)+x−4x(x+2)=0 là
A. x = 2 hoặc x = 3.
B. x = 2 hoặc x = −3.
C. x = 2.
D. x = 3.

\textbf{{ANSWER}}

Đáp án đúng là: D
Điều kiện xác định: x ≠ 0, x ≠ 2 và x ≠ −2.
Ta có: 2x2−4−1x(x−2)+x−4x(x+2)=0
2xx(x−2)(x+2)−x+2x(x−2)(x+2)+(x−4)(x−2)x(x+2)(x−2)=0
2x – x – 2 + x2 – 6x + 8 = 0
x2 – 5x + 6 = 0
x2 – 2x − 3x + 6 = 0
x(x – 2) – 3(x – 2) = 0
(x – 2)(x – 3) = 0
Suy ra x – 2 = 0 hoặc x – 3 = 0 khi x = 2 (loại) hoặc x = 3 (thỏa mãn).
Vậy nghiệm của phương trình là x = 3.

========================================================================

https://khoahoc.vietjack.com/thi-online/sach-bai-tap-toan-7-tap-2/23592


\textbf{{QUESTION}}

Hãy xếp các đơn thức sau thành nhóm các đơn thức đồng dạng với nhau: -5x2yz;    3xy2z;    2/3 x2yz;    10x2y2z;    - 2/3 xy2z;    5/7 x2y2z

\textbf{{ANSWER}}

Nhóm các đơn thức đồng dạng:

========================================================================

https://khoahoc.vietjack.com/thi-online/sach-bai-tap-toan-7-tap-2/23592


\textbf{{QUESTION}}

Các cặp đơn thức sau có đồng dạng hay không? 2/3 x2y và -2/3.x2y

\textbf{{ANSWER}}

2/3 x2y và -2/3.x2y là hai đơn thức đồng dạng

========================================================================

https://khoahoc.vietjack.com/thi-online/sach-bai-tap-toan-7-tap-2/23592


\textbf{{QUESTION}}

Các cặp đơn thức sau có đồng dạng hay không? 2xy và 3/4 xy

\textbf{{ANSWER}}

2xy và 3/4 xy là 2 đơn thức đồng dạng

========================================================================

https://khoahoc.vietjack.com/thi-online/sach-bai-tap-toan-7-tap-2/23592


\textbf{{QUESTION}}

Các cặp đơn thức sau có đồng dạng hay không? 5x và 5x2

\textbf{{ANSWER}}

5x và 5x2 không phải là 2 đơn thức đồng dạng

========================================================================

https://khoahoc.vietjack.com/thi-online/sach-bai-tap-toan-7-tap-2/23592


\textbf{{QUESTION}}

Tính tổng: x2 + 5x2 + (-3x2)

\textbf{{ANSWER}}

x2 + 5x2 + (-3x2) = (1 + 5 – 3)x2 = 3x2

========================================================================

https://khoahoc.vietjack.com/thi-online/15-cau-trac-nghiem-toan-10-chan-troi-sang-tao-tap-hop-co-dap-an


\textbf{{QUESTION}}

Để chỉ phần tử a thuộc tập số A, ta kí hiệu như thế nào?

\textbf{{ANSWER}}

Hướng dẫn giải 
Đáp án đúng là: A
Để chỉ phần tử a thuộc tập số A, ta kí hiệu a ∈ A.

========================================================================

https://khoahoc.vietjack.com/thi-online/15-cau-trac-nghiem-toan-10-chan-troi-sang-tao-tap-hop-co-dap-an


\textbf{{QUESTION}}

Chọn khẳng định đúng trong các khẳng định sau:

\textbf{{ANSWER}}

Hướng dẫn giải 
Đáp án đúng là: C
Phương án A sai vì tập rỗng chứa 0 phần tử;
Phương án B sai vì phần tử a không thuộc tập A kí hiệu là a ∉ A, 
Phương án C đúng do tập rỗng là tập con của mọi tập hợp vì tập rỗng không có phần tử nào.
Phương án D sai vì tập hợp có thể có vô số phần tử, ví dụ như các tập số tự nhiên, tập số thực,….

========================================================================

https://khoahoc.vietjack.com/thi-online/15-cau-trac-nghiem-toan-10-chan-troi-sang-tao-tap-hop-co-dap-an


\textbf{{QUESTION}}

Người ta thường kí hiệu tập hợp số như thế nào?

\textbf{{ANSWER}}

Hướng dẫn giải 
Đáp án đúng là: D
Người ta thường kí hiệu tập hợp số như sau: ℕ là tập hợp các số tự nhiên, ℤ là tập hợp các số nguyên, ℝ là tập hợp các số thực.

========================================================================

https://khoahoc.vietjack.com/thi-online/15-cau-trac-nghiem-toan-10-chan-troi-sang-tao-tap-hop-co-dap-an


\textbf{{QUESTION}}

Có mấy cách xác định tập hợp?
A. 1
B. 2
C. 3
D. 4

\textbf{{ANSWER}}

Hướng dẫn giải 
Đáp án đúng là: B
Có 2 cách xác định tập hợp: Liệt kê các phần tử và Chỉ ra tính chất đặc trưng của tập hợp.

========================================================================

https://khoahoc.vietjack.com/thi-online/15-cau-trac-nghiem-toan-10-chan-troi-sang-tao-tap-hop-co-dap-an


\textbf{{QUESTION}}

Cách viết tập hợp nào đúng trong các cách viết sau để xác định tập hợp A các ước dương của 12:

\textbf{{ANSWER}}

Hướng dẫn giải 
Đáp án đúng là: A
Tập hợp A là tập hợp các ước dương của 12 nên ta có:
A = {1; 2; 3; 4; 6; 12}, cũng có thể viết A = {x| x ∈ ℕ, x là ước của 12}.
Do đó ta chọn phương án A.

========================================================================

https://khoahoc.vietjack.com/thi-online/10-bai-tap-van-dung-cac-phep-toan-gioi-han-de-tim-gioi-han-cua-day-so-dang-chua-can


\textbf{{QUESTION}}

Giá trị của $$ \underset{n\to +\infty }{\mathrm{lim}}\left(\sqrt{n+5}-\sqrt{n+1}\right)$$  là

\textbf{{ANSWER}}

Đáp án đúng là: D
Ta có: $$ \underset{n\to +\infty }{\mathrm{lim}}\left(\sqrt{n+5}-\sqrt{n+1}\right)=\underset{n\to +\infty }{\mathrm{lim}}\frac{n+5-n-1}{\sqrt{n+5}+\sqrt{n+1}}$$

========================================================================

https://khoahoc.vietjack.com/thi-online/10-bai-tap-van-dung-cac-phep-toan-gioi-han-de-tim-gioi-han-cua-day-so-dang-chua-can


\textbf{{QUESTION}}

Giá trị của  limn→+∞(√n2−n+1−n)$$ \underset{n\to +\infty }{\mathrm{lim}}\left(\sqrt{{n}^{2}-n+1}-n\right)$$ là

\textbf{{ANSWER}}

Đáp án đúng là: A
Ta có: $$ \underset{n\to +\infty }{\mathrm{lim}}\left(\sqrt{{n}^{2}-n+1}-n\right)=\underset{n\to +\infty }{\mathrm{lim}}\frac{{n}^{2}-n+1-{n}^{2}}{\sqrt{{n}^{2}-n+1}+n}$$
$$ =\underset{n\to +\infty }{\mathrm{lim}}\frac{-n+1}{\sqrt{{n}^{2}-n+1}+n}=\underset{n\to +\infty }{\mathrm{lim}}\frac{-1+\frac{1}{n}}{\sqrt{1-\frac{1}{n}+\frac{1}{{n}^{2}}}+1}=-\frac{1}{2}.$$

========================================================================

https://khoahoc.vietjack.com/thi-online/10-bai-tap-van-dung-cac-phep-toan-gioi-han-de-tim-gioi-han-cua-day-so-dang-chua-can


\textbf{{QUESTION}}

Giá trị của  limn→+∞(√n2+2n−√n2−2n) $$ \underset{n\to +\infty }{\mathrm{lim}}\left(\sqrt{{n}^{2}+2n}-\sqrt{{n}^{2}-2n}\right)\quad $$là

\textbf{{ANSWER}}

Đáp án đúng là: B

========================================================================

https://khoahoc.vietjack.com/thi-online/10-bai-tap-van-dung-cac-phep-toan-gioi-han-de-tim-gioi-han-cua-day-so-dang-chua-can


\textbf{{QUESTION}}

Có bao nhiêu giá trị của a để limn→+∞(√n2+a2n−√n2+(a+2)n+1)=0$$ \underset{n\to +\infty }{\mathrm{lim}}\left(\sqrt{{n}^{2}+{a}^{2}n}-\sqrt{{n}^{2}+(a+2)n+1}\right)=0$$ ?

\textbf{{ANSWER}}

Đáp án đúng là: C

========================================================================

https://khoahoc.vietjack.com/thi-online/10-bai-tap-van-dung-cac-phep-toan-gioi-han-de-tim-gioi-han-cua-day-so-dang-chua-can


\textbf{{QUESTION}}

Có bao nhiêu giá trị nguyên của a thỏa mãn limn→+∞(√n2−8n−n+a2)=0$$ \underset{n\to +\infty }{\mathrm{lim}}\left(\sqrt{{n}^{2}-8n}-n+{a}^{2}\right)=0$$ ?

\textbf{{ANSWER}}

Đáp án đúng là: B

========================================================================

https://khoahoc.vietjack.com/thi-online/giai-sgk-toan-8-kntt-bai-21-phan-thuc-dai-so-co-dap-an


\textbf{{QUESTION}}

Trong một cuộc đua xe đạp, các vận động viên phải hoàn thành ba chặng đường đua bao gồm 9 km leo dốc; 5 km xuống dốc và 36 km đường bằng phẳng. Vận tốc của một vận động viên trên chặng đường bằng phẳng hơn vận tốc leo dốc 5 km/h và kém vận tốc xuống dốc 10 km/h. Nếu biết vận tốc của vận động viên trên chặng đường bằng phẳng thì có tính được thời gian hoàn thành cuộc đua của vận động viên đó không?

\textbf{{ANSWER}}

Nếu biết vận tốc của vận động viên trên chặng đường bằng phẳng thì sẽ tính được thời gian hoàn thành cuộc đua của vận động viên đó.
Vì giả sử x (km/h) là vận tốc của vận động viên trên chặng đường bằng phẳng thì vận tốc leo dốc là x – 5 (km/h) và vận tốc khi xuống dốc là x + 10 (km/h). Quãng đường đã biết từ đó tính được thời gian của từng chặng và thời gian hoàn thành cuộc đua của vận động viên.

========================================================================

https://khoahoc.vietjack.com/thi-online/giai-sgk-toan-8-kntt-bai-21-phan-thuc-dai-so-co-dap-an


\textbf{{QUESTION}}

Trong tình huống mở đầu, giả sử vận tốc trung bình của một vận động viên đi xe đạp trên 36 km đường bằng phẳng là x (km/h). Hãy viết biểu thức biểu thị thời gian vận động viên đó hoàn thành chặng leo dốc, chặng xuống dốc, chặng đường bằng phẳng.

\textbf{{ANSWER}}

Giả sử x (km/h) là vận tốc của vận động viên trên chặng đường bằng phẳng thì vận tốc leo dốc là x – 5 (km/h) và vận tốc khi xuống dốc là x + 10 (km/h). 
+ Thời gian vận động viên đó hoàn thành chặng leo dốc là:   t1=9x−5 $$ {t}_{1}=\frac{9}{x-5}\quad $$(h)
+ Thời gian vận động viên đó hoàn thành chặng xuống dốc là:   t2=5x+10 $$ {t}_{2}=\frac{5}{x+10}\quad $$(h)
+ Thời gian vận động viên đó hoàn thành chặng đường bằng phẳng là:   t3=36x$$ {t}_{3}=\frac{36}{x}$$(h).

========================================================================

https://khoahoc.vietjack.com/thi-online/giai-sgk-toan-8-kntt-bai-21-phan-thuc-dai-so-co-dap-an


\textbf{{QUESTION}}

Viết biểu thức biểu thị tỉ số giữa chiều rộng và chiều dài của một hình chữ nhật có chiều rộng là x (cm) và chiều dài là y (cm)

\textbf{{ANSWER}}

Biểu thức biểu thị tỉ số giữa chiều rộng và chiều dài là: xy$$ \frac{x}{y}$$  .

========================================================================

https://khoahoc.vietjack.com/thi-online/giai-sgk-toan-8-kntt-bai-21-phan-thuc-dai-so-co-dap-an


\textbf{{QUESTION}}

Trong các cặp phân thức sau, cặp phân thức nào có cùng mẫu thức?
a) −20x3y2$$ \frac{-20x}{3{y}^{2}}$$   và 4x35y2 $$ \frac{4{x}^{3}}{5{y}^{2}}\quad $$ ;                
b) 5x−10x2+1$$ \frac{5x-10}{{x}^{2}+1}$$  và 5x−20x2−1$$ \frac{5x-20}{{x}^{2}-1}$$ ;        
c)   5x+104x−8 $$ \frac{5x+10}{4x-8}\quad $$và 4−2x4(x−2)$$ \frac{4-2x}{4(x-2)}$$  .

\textbf{{ANSWER}}

Cặp phân thức có cùng mẫu thức là 5x+104x−8$$ \frac{5x+10}{4x-8}$$  và 4−2x4(x−2)$$ \frac{4-2x}{4(x-2)}$$  vì 4−2x4(x−2)=4−2x4x−8$$ \frac{4-2x}{4(x-2)}=\frac{4-2x}{4x-8}$$ .

========================================================================

https://khoahoc.vietjack.com/thi-online/giai-sgk-toan-8-kntt-bai-21-phan-thuc-dai-so-co-dap-an


\textbf{{QUESTION}}

Tròn: 3−2x3+1x$$ \frac{3-2x}{3+\frac{1}{x}}$$  không phải là phân thức. Vuông: 3−2x3+1x$$ \frac{3-2x}{3+\frac{1}{x}}$$  là phân thức đại số. Theo em, bạn nào đúng?

\textbf{{ANSWER}}

Tròn đúng, vuông sai vì  3+1x $$ 3+\frac{1}{x}\quad $$không phải là đa thức.

========================================================================

https://khoahoc.vietjack.com/thi-online/21-cau-trac-nghiem-toan-6-chan-troi-sang-tao-bai-3-so-sanh-phan-so-co-dap-an


\textbf{{QUESTION}}

Điền dấu thích hợp vào chỗ chấm: $\frac{{ - 5}}{{13}}...\frac{{ - 7}}{{13}}$
A. >
B. <    
C. = 
D.Tất cả các đáp án trên đều sai

\textbf{{ANSWER}}

Vì – 5 >−7 nên  $\frac{{ - 5}}{{13}} >\frac{{ - 7}}{{13}}$
Đáp án cần chọn là: A

========================================================================

https://khoahoc.vietjack.com/thi-online/21-cau-trac-nghiem-toan-6-chan-troi-sang-tao-bai-3-so-sanh-phan-so-co-dap-an


\textbf{{QUESTION}}

A. 11231125>1$\frac{{1123}}{{1125}} >1$ 
B. −154−156<1$\frac{{ - 154}}{{ - 156}} < 1$ 

C. −123345>0$\frac{{ - 123}}{{345}} >0$ 
D. −657−324<0$\frac{{ - 657}}{{ - 324}} < 0$

\textbf{{ANSWER}}

Đáp án A: Vì 1123 < 1125  nên 11231125<1$\frac{{1123}}{{1125}} < 1$ ⇒A sai.
Đáp án B: −154−156=154156$\frac{{ - 154}}{{ - 156}} = \frac{{154}}{{156}}$ 
Vì 154 < 156  nên 154156<1$\frac{{154}}{{156}} < 1$ hay −154−156<1$\frac{{ - 154}}{{ - 156}} < 1$ ⇒B đúng.
Đáp án C: −123345<0$\frac{{ - 123}}{{345}} < 0$  vì nó là phân số âm.⇒C sai.
Đáp án D: −657−324>0$\frac{{ - 657}}{{ - 324}} >0$  vì nó là phân số dương. =>D sai.
Đáp án cần chọn là: B

========================================================================

https://khoahoc.vietjack.com/thi-online/21-cau-trac-nghiem-toan-6-chan-troi-sang-tao-bai-3-so-sanh-phan-so-co-dap-an


\textbf{{QUESTION}}

Sắp xếp các phân số 2941;2841;2940 theo thứ tự tăng dần ta được
A. 2941;2841;2940
B. 2940;2941;2841
C. 2841;2941;2940
D. 2841;2940;2941

\textbf{{ANSWER}}

Ta có:
+) 28 < 29 nên 2841<2941$\frac{{28}}{{41}} < \frac{{29}}{{41}}$
+) 41 >40 nên 2941<2940$\frac{{29}}{{41}} < \frac{{29}}{{40}}$
Do đó 2841<2941<2940$\frac{{28}}{{41}} < \frac{{29}}{{41}} < \frac{{29}}{{40}}$
Đáp án cần chọn là: C

========================================================================

https://khoahoc.vietjack.com/thi-online/21-cau-trac-nghiem-toan-6-chan-troi-sang-tao-bai-3-so-sanh-phan-so-co-dap-an


\textbf{{QUESTION}}

Có bao nhiêu phân số lớn hơn 16  nhưng nhỏ hơn 14  mà có tử số là 5.
A.9 
B.10  
C.11
D.12

\textbf{{ANSWER}}

Gọi phân số cần tìm là 5x(x∈N∗)$\frac{5}{x}(x \in N * )$
Ta có: 16<5x<14$\frac{1}{6} < \frac{5}{x} < \frac{1}{4}$
⇒530<5x<520⇒30>x>20$ \Rightarrow \frac{5}{{30}} < \frac{5}{x} < \frac{5}{{20}} \Rightarrow 30 >x >20$hay x∈{21;22;...;29}$x \in \left\{ {21;22;...;29} \right\}$ </>
Số giá trị của xx là: (29 − 21) : 1 + 1 = 9
Vậy có tất cả 9 phân số thỏa mãn bài toán.
Đáp án cần chọn là: A

========================================================================

https://khoahoc.vietjack.com/thi-online/21-cau-trac-nghiem-toan-6-chan-troi-sang-tao-bai-3-so-sanh-phan-so-co-dap-an


\textbf{{QUESTION}}

So sánh 25.7+2525.52−25.3$\frac{{{2^5}.7 + {2^5}}}{{{2^5}{{.5}^2} - {2^5}.3}}$ và 34.5−3634.13+34$\frac{{{3^4}.5 - {3^6}}}{{{3^4}.13 + {3^4}}}$ với 1.
A. A < 1 < B
B. A = B = 1
C. A >1 >B
D. 1 >A >B

\textbf{{ANSWER}}

25.7+2525.52−25.3=25.(7+1)25.(52−3)=25.(7+1)25.(25−3)=25.825.22=822=411$\frac{{{2^5}.7 + {2^5}}}{{{2^5}{{.5}^2} - {2^5}.3}} = \frac{{{2^5}.(7 + 1)}}{{{2^5}.({5^2} - 3)}} = \frac{{{2^5}.(7 + 1)}}{{{2^5}.(25 - 3)}} = \frac{{{2^5}.8}}{{{2^5}.22}} = \frac{8}{{22}} = \frac{4}{{11}}$
34.5−3634.13+34=34.(5−32)34.(13+1)=34.(5−9)34.14=34.(−4)34.14=−414=−27$\frac{{{3^4}.5 - {3^6}}}{{{3^4}.13 + {3^4}}} = \frac{{{3^4}.(5 - {3^2})}}{{{3^4}.(13 + 1)}} = \frac{{{3^4}.(5 - 9)}}{{{3^4}.14}} = \frac{{{3^4}.( - 4)}}{{{3^4}.14}} = \frac{{ - 4}}{{14}} = \frac{{ - 2}}{7}$
MSC = 77
411=4.711.7=2877;−27=−2.117.11=−2277$\frac{4}{{11}} = \frac{{4.7}}{{11.7}} = \frac{{28}}{{77}};\frac{{ - 2}}{7} = \frac{{ - 2.11}}{{7.11}} = \frac{{ - 22}}{{77}}$
Do đó −2277<2877<1$\frac{{ - 22}}{{77}} < \frac{{28}}{{77}} < 1$ hay B < A < 1
Đáp án cần chọn là: D
</>

========================================================================

https://khoahoc.vietjack.com/thi-online/bai-tap-he-phuong-trinh-bac-nhat-ba-an-co-dap-an


\textbf{{QUESTION}}

Khái niệm hệ phương trình bậc nhất ba ẩn
Xét hệ phương trình với ẩn là x, y, z sau:
$$ \left\{\begin{array}{l}x+y+z=2\\ x+2y+3z=1\\ 2x+y+3z=-1\end{array}\right.$$
a) Mỗi phương trình của hệ trên có bậc mấy đối với các ẩn x, y, z?
b) Thử lại rằng bộ ba số (x; y, z) = (1; 3;–2) thoả mãn cả ba phương trình của hệ.
c) Bằng cách thay trực tiếp vào hệ, hãy kiểm tra bộ ba số (1; 1; 2) có thoả mãn hệ phương trình đã cho không.

\textbf{{ANSWER}}

a) Mỗi phương trình của hệ trên có bậc nhất đối với các ẩn x, y, z.
b) Bộ số (x; y; z) = (1; 3;–2) có thoả mãn cả ba phương trình của hệ.
Thử lại:
1 + 3 + (–2) = 2;
1 + 2 . 3 + 3 . (–2) = 1;
2 . 1 + 3 + 3 . (–2) = –1.
c) Bộ số (x; y; z) = (1; 3;–2) không thoả mãn hệ phương trình đã cho. Vì khi thay vào phương trình thứ nhất của hệ ta được 1 + 1 + 2 = 2, đây là đẳng thức sai.

========================================================================

https://khoahoc.vietjack.com/thi-online/bai-tap-he-phuong-trinh-bac-nhat-ba-an-co-dap-an


\textbf{{QUESTION}}

Hệ nào dưới đây là hệ phương trình bậc nhất ba ẩn? Kiểm tra xem bộ ba số (–3; 2;–1) có phải là nghiệm của hệ phương trình bậc nhất ba ẩn đó không.
a) {x+2y−3z=12x−3y+7z=153x2−4y+z=−3$$ \left\{\begin{array}{l}x+2y-3z=1\\ 2x-3y+7z=15\\ 3{x}^{2}-4y+z=-3\end{array}\right.$$;
b) {−x+y+z=42x+y−3z=−13x−2z=−7$$ \left\{\begin{array}{l}-x+y+z=4\\ 2x+y-3z=-1\\ 3x-2z=-7\end{array}\right.$$.

\textbf{{ANSWER}}

a) Bộ ba số (–3; 2;–1) không là nghiệm của hệ phương trình bậc nhất đã cho.
Vì khi thay bộ số này vào phương trình thứ nhất của hệ ta được (–3) + 2 . 2 – 3 . (–1) = 1, đây là đẳng thức sai.
b) Bộ ba số (–3; 2;–1) có là nghiệm của hệ phương trình bậc nhất đã cho.
Vì khi thay bộ số này vào từng phương trình thì chúng đều có nghiệm đúng:
–(–3) + 2 + (–1) = 4;
2 . (–3) + 2 – 3 . (–1) = –1;
3 . (–3) – 2 . (–1) = –7.

========================================================================

https://khoahoc.vietjack.com/thi-online/bai-tap-he-phuong-trinh-bac-nhat-ba-an-co-dap-an


\textbf{{QUESTION}}

Hệ bậc nhất ba ẩn có dạng tam giác.
Cho hệ phương trình:
{x+y−2z=3y+z=72z=4.
Hệ phương trình dạng tam giác có cách giải rất đơn giản.
Từ phương trình cuối hãy tính z, sau đó thay vào phương trình thứ hai để tìm y, cuối cùng thay y và z tìm được vào phương trình đầu để tìm x.

\textbf{{ANSWER}}

+) Từ phương trình cuối ta tính được z = 2.
+) Thay z = 2 vào phương trình thứ hai ta được y + 2 = 7, suy ra y = 5.
+) Thay y = 5 và z = 2 vào phương trình đầu ta được x + 5 – 2 . 2 = 3, suy ra x = 2.

========================================================================

https://khoahoc.vietjack.com/thi-online/bai-tap-he-phuong-trinh-bac-nhat-ba-an-co-dap-an


\textbf{{QUESTION}}

Giải hệ phương trình:

\textbf{{ANSWER}}

+) Từ phương trình đầu ta tính được x = 32
+) Thay x = 32 vào phương trình thứ hai ta được 32 + y = 2, suy ra y = 12
+) Thay x = 32 và y = 12 vào phương trình thứ ba ta được 2.32−2.12+z=−1, suy ra z = –3.
Vậy nghiệm của hệ đã cho là (x; y; z) = (32;12;−3).

========================================================================

https://khoahoc.vietjack.com/thi-online/bai-tap-he-phuong-trinh-bac-nhat-ba-an-co-dap-an


\textbf{{QUESTION}}

Giải hệ phương trình bậc nhất ba ẩn bằng phương pháp Gauss. Cho hệ phương trình:
{x+y−2z=3−x+y+6z=132x+y−9z=−5.
a) Khử ẩn x của phương trình thứ hai bằng cách cộng phương trình này với phương trình thứ nhất. Viết phương trình nhận được (phương trình này không còn chứa ẩn x và là phương trình thứ hai của hệ mới, tương đương với hệ ban đầu).
b) Khử ẩn x của phương trình thứ ba bằng cách nhân phương trình thứ nhất với –2 và cộng với phương trình thứ ba. Viết phương trình thứ ba mới nhận được. Từ đó viết hệ mới nhận được sau hai bước trên (đã khử x ở hai phương trình cuối).
c) Làm tương tự đối với hệ mới nhận được ở câu b), từ phương trình thứ hai và thứ ba khử ẩn y ở phương trình thứ ba. Viết hệ dạng tam giác nhận được.
d) Giải hệ dạng tam giác nhận được ở câu c). Từ đó suy ra nghiệm của hệ đã cho.

\textbf{{ANSWER}}

a) Cộng phương trình thứ hai với phương trình thứ nhất, ta được:
(x + y – 2z) + (–x + y + 6z) = 3 = 13 ⇔ 2y + 4z = 16 ⇔ y + 2z = 8.
b) Nhân phương trình thứ nhất với –2 và cộng với phương trình thứ ba, ta được:
–2(x + y – 2z) + (2x + y – 9z) = –2 . 3 + (–5) ⇔ –y – 5z = –11 ⇔ y + 5z = 11.
Hệ mới nhận được sau hai bước trên là: {x+y−2z=3y+2z=8y+5z=11.
c) Lấy phương trình thứ hai trừ phương trình thứ ba, ta được:
(y + 2z) – (y + 5z) = 8 – 11 ⇔ –3z = –3 ⇔ z = 1.
Hệ tam giác nhận được là: {x+y−2z=3y+2z=8z=1.
d) {x+y−2z=3y+2z=8z=1⇔{x+y−2z=3y+2.1=8z=1⇔{x+y−2z=3y=6z=1
⇔{x+6−2.1=3y=6z=1⇔{x=−1y=6z=1.
Vậy nghiệm của hệ đã cho là (x; y; z) = (–1; 6; 1).

========================================================================

https://khoahoc.vietjack.com/thi-online/de-thi-thu-thpt-quoc-gia-nam-hoc-2019-mon-toan/23932


\textbf{{QUESTION}}

Đường tiệm cận đứng và tiệm cận ngang của đồ thị hàm số $$ y=\frac{1-x}{-x+2}$$ có phương trình lần lượt là
A. x=1,y=2
B. x=2,y=1
C. x=2,y=$$ \frac{1}{2}$$
D. x=2,y=-1

\textbf{{ANSWER}}

Chọn B.
Ta có TCĐ x=2 và TCN y=1.

========================================================================

https://khoahoc.vietjack.com/thi-online/11-cau-trac-nghiem-toan-8-bai-4-nhung-hang-dang-thuc-dang-nho-tiep-co-dap-an-thong-hieu


\textbf{{QUESTION}}

Tìm x biết x3 + 3x2 + 3x + 1 = 0
A. x=-1
B. x = 1
C. x = -2
D. x = 0

\textbf{{ANSWER}}

Ta có x3 + 3x2 + 3x + 1 = 0 $$ \Leftrightarrow $$ (x + 1)3 = 0
$$ \Leftrightarrow $$ x + 1 = 0
$$ \Leftrightarrow $$ x = -1
Vậy x = -1
Đáp án cần chọn là: A

========================================================================

https://khoahoc.vietjack.com/thi-online/11-cau-trac-nghiem-toan-8-bai-4-nhung-hang-dang-thuc-dang-nho-tiep-co-dap-an-thong-hieu


\textbf{{QUESTION}}

Tìm x biết x3 – 12x2 + 48x – 64 = 0
A. x = -4
B. x = 4
C. x = -8
D. x = 8

\textbf{{ANSWER}}

Ta có x3 – 12x2 + 48x – 64 = 0
⇔$$ \Leftrightarrow $$ x3 – 3.x2.4 + 3.x.42 – 43 = 0
⇔$$ \Leftrightarrow $$ (x – 4)3 = 0 ⇔$$ \Leftrightarrow $$ x – 4 = 0 ⇔$$ \Leftrightarrow $$ x = 4
Vậy x = 4
Đáp án cần chọn là: B

========================================================================

https://khoahoc.vietjack.com/thi-online/11-cau-trac-nghiem-toan-8-bai-4-nhung-hang-dang-thuc-dang-nho-tiep-co-dap-an-thong-hieu


\textbf{{QUESTION}}

Cho x thỏa mãn (x + 2)(x2 – 2x + 4) – x(x2 – 2) = 14. Chọn câu đúng.
A. x = -3
B. x = 11
C. x = 3
D. x = 4

\textbf{{ANSWER}}

Ta có (x + 2) (x2 – 2x + 4) – x(x2 – 2) = 14
⇔$$ \Leftrightarrow $$x3 + 23 – (x3 – 2x) = 14
⇔$$ \Leftrightarrow $$x3 + 8 – x3 + 2x = 14
⇔$$ \Leftrightarrow $$ 2x = 6 ⇔$$ \Leftrightarrow $$ x = 3
Vậy x = 3
Đáp án cần chọn là: C

========================================================================

https://khoahoc.vietjack.com/thi-online/11-cau-trac-nghiem-toan-8-bai-4-nhung-hang-dang-thuc-dang-nho-tiep-co-dap-an-thong-hieu


\textbf{{QUESTION}}

Chọn câu đúng.
A. 8 + 12y + 6y2 + y3 = (8 + y3)
B. a3 + 3a2 + 3a + 1 = (a + 1)3
C. (2x – y)3 = 2x3 – 6x2y + 6xy – y3
D. (3a + 1)3 = 3a3 + 9a2 + 3a + 1

\textbf{{ANSWER}}

Ta có 8 + 12y + 6y2 + y3 = 23 + 3.22y + 3.2.y2 + y3 = (2 + y)3 ≠ (8 + y3) nên A sai
+ Xét (2x – y)3 = (2x)3 – 3(2x)2.y + 3.2x.y2 – y3
= 8x3 – 12x2y + 6xy2 – y3 ≠ 2x3 – 6x2y + 6xy – y3 nên C sai
+ Xét (3a + 1)3 = (3a)3 + 3.(3a)2.1 + 3.3a.12 + 1
= 27a3 + 27a2 + 9a + 1 ≠ 3a3 + 9a2 + 3a + 1 nên D sai
+ Xét a3 + 3a2 + 3a + 1 = (a + 1)3 nên B đúng
Đáp án cần chọn là: B

========================================================================

https://khoahoc.vietjack.com/thi-online/11-cau-trac-nghiem-toan-8-bai-4-nhung-hang-dang-thuc-dang-nho-tiep-co-dap-an-thong-hieu


\textbf{{QUESTION}}

Chọn câu sai.
A. (-b – a)3 = -a3 – 3ab(a + b) – b3
B. (c – d)3 = c3 – d3 + 3cd(d – c)
C. (y – 2)3 = y3 – 8 – 6y(y + 2)
D. (y – 1)3 = y3 – 1- 3y(y – 1)

\textbf{{ANSWER}}

Ta có (-b – a)3 = [-(a + b)3] = -(a + b)3
                   = -(a3 + 3a2b + 3ab2 + b3)
                   =  -a3 - 3a2b - 3ab2 - b3
                   = -a3 – 3ab(a + b) – b3 nên A đúng
+ Xét (c – d)3 = c3 – 3c2d + 3cd2 - d3 = c3 – d3 + 3cd(d – c) nên B đúng
+ Xét (y – 1)3 = y3 – 3y2.1 + 3y.12 – 13 = y3 – 1 – 3y(y – 1) nên D đúng
+ Xét (y – 2)3 = y3 – 3y2.2 +3y.22 – 23 = y3 – 6y2 + 12y – 8
= y3 – 8 – 6y(y – 2) ≠ y3 – 8 – 6y(y + 2) nên C sai
Đáp án cần chọn là: C

========================================================================

https://khoahoc.vietjack.com/thi-online/10-bai-tap-nhan-don-thuc-voi-don-thuc-don-thuc-voi-da-thuc-co-loi-giai


\textbf{{QUESTION}}

Kết quả phép tính 5x2y6z5 . 2xy3 là
A. 10x2y18z5;

\textbf{{ANSWER}}

Hướng dẫn giải:
Đáp án đúng là: C
Ta có: 5x2y6z5 . 2xy3 = (5.2) . (x2.x) . (y6.y3) . z5 = 10x3y9z5.

========================================================================

https://khoahoc.vietjack.com/thi-online/10-bai-tap-nhan-don-thuc-voi-don-thuc-don-thuc-voi-da-thuc-co-loi-giai


\textbf{{QUESTION}}

A. – 3xy + 6x;

\textbf{{ANSWER}}

Hướng dẫn giải:
Đáp án đúng là: A
Ta có: ‐34x.(4y-8)=(‐34x).4y-(‐34x).8=‐3xy+6x$$ ‐\frac{3}{4}x.\left(4y-8\right)=\left(‐\frac{3}{4}x\right).4y-\left(‐\frac{3}{4}x\right).8=‐3xy+6x$$

========================================================================

https://khoahoc.vietjack.com/thi-online/giai-vth-toan-8-kntt-bai-34-ba-truong-hop-dong-dang-cua-hai-tam-giac-co-dap-an


\textbf{{QUESTION}}

Chọn phương án đúng.
Cho tam giác ABC có AB = 4 cm, BC = 5 cm, CA = 6 cm. Bộ ba độ dài nào dưới đây là độ dài ba cạnh của một tam giác đồng dạng với tam giác ABC với tỉ số đồng dạng là 2.
A. 2 cm, 2,5 cm, 3 cm.
B. 4 cm, 5 cm, 6 cm.
C. 8 cm, 10 cm, 12 cm.
D. 6 cm, 8 cm, 10 cm.

\textbf{{ANSWER}}

Đáp án đúng là: C
Ta có: $$ \frac{8}{4}=\frac{10}{5}=\frac{12}{5}=2$$ nên bộ ba độ dài trong đáp án C là độ dài ba cạnh của tam giác thỏa mãn yêu cầu.

========================================================================

https://khoahoc.vietjack.com/thi-online/tong-hop-de-thi-giua-hoc-ki-2-toan-9-hay-nhat-nam-2023-co-dap-an/117304


\textbf{{QUESTION}}

Cho biểu thức  $$ A=\left(\frac{\sqrt{x}+x}{\sqrt{x}+1}+1\right)\left(\frac{\sqrt{x}-x}{\sqrt{x}-1}+1\right);$$   với   $$ x\ge 0\text{\hspace{0.17em}\hspace{0.17em}\hspace{0.17em}\hspace{0.17em}\hspace{0.17em}\hspace{0.17em}}x\ne 1$$
a)     Rút gọn biểu thức A .
b)    Tìm giá trị của biểu thức A biết $$ X=\sqrt{4-2\sqrt{3}}$$

\textbf{{ANSWER}}

$$ \begin{array}{l}a)A=\left(\frac{\sqrt{x}+x}{\sqrt{x}+1}+1\right)\left(\frac{\sqrt{x}-x}{\sqrt{x}-1}+1\right)\left(\begin{array}{l}x\ge 0\\ x\ne 1\end{array}\right)\\ =\left[\frac{\sqrt{x}\left(\sqrt{x}+1\right)}{\sqrt{x}+1}+1\right]\left[\frac{-\sqrt{x}\left(\sqrt{x}-1\right)}{\sqrt{x}-1}+1\right]=\left(1+\sqrt{x}\right)\left(1-\sqrt{x}\right)=1-x\\ b)x=\sqrt{4-2\sqrt{3}}=\sqrt{{\left(\sqrt{3}-1\right)}^{2}}=\sqrt{3}-1\Rightarrow A=1-\sqrt{3}+1=2-\sqrt{3}\end{array}$$

========================================================================

https://khoahoc.vietjack.com/thi-online/tong-hop-de-thi-giua-hoc-ki-2-toan-9-hay-nhat-nam-2023-co-dap-an/117304


\textbf{{QUESTION}}

y=12x2     (P)
y=12x2     (P)
y
=
12
1
1
2
2


2
2
2
x2
x
2
     
(P)
(
P
)



Tìm giá trị của m   sao cho điểm C(−2;m)   ∈(p)  
 m 
 
 
 C(−2;m)   ∈(p) 
C(−2;m)   ∈(p)
C(−2;m)   ∈(p)
C
(−2;m)
(
−2;m
−
2
;
m
)
   
∈
(p)
(
p
)

\textbf{{ANSWER}}

Học sinh tự vẽ (P)
C(−2;m)∈(P)⇒(−2)2.12=m⇔m=2$$ C\left(-2;m\right)\in \left(P\right)\Rightarrow {\left(-2\right)}^{2}.\frac{1}{2}=m\Leftrightarrow m=2$$

========================================================================

https://khoahoc.vietjack.com/thi-online/tong-hop-de-thi-giua-hoc-ki-2-toan-9-hay-nhat-nam-2023-co-dap-an/117304


\textbf{{QUESTION}}

Giải bài toán sau bằng cách lập hệ phương trình :
 Hai tổ sản xuất cùng may một loại áo . Nếu tổ thứ nhất may trong  3 ngày , tổ thứ hai may trong  3 ngày thì cả hai tổ may được 1310 chiếc áo . Biết rằng trong một ngày ,tổ thứ nhất may được nhiều hơn tổ thứ hai  là 10 chiếc áo . Hỏi mỗi tổ trong một ngày may được bao nhiêu chiếc áo?

\textbf{{ANSWER}}

Gọi x, y lần lượt là số áo tổ 1, tổ 2 may được (x,y∈ℕ*,x>10)$$ \left(x,y\in \mathbb{N}*,x>10\right)$$
Theo bài ta có hệ phương trình {3x+5y=1310x−y=10⇔{x=170y=160(tm)$$ \left\{\begin{array}{l}3x+5y=1310\\ x-y=10\end{array}\right.\Leftrightarrow \left\{\begin{array}{l}x=170\\ y=160\end{array}\right.\left(tm\right)$$
Vậy tổ 1: 170 chiếc áo,   tổ 2: 160 chiếc áo

========================================================================

https://khoahoc.vietjack.com/thi-online/giai-sbt-toan-10-bai-4-nhi-thuc-newton-co-dap-an


\textbf{{QUESTION}}

Trong các phát biểu sau, phát biểu nào sai?
A. (a + b)4 = a4 + 4a3b + 6a2b2 + 4ab3 + b4.
B. (a – b)4 = a4 – 4a3b + 6a2b2 – 4ab3 + b4.
C. (a + b)4 = b4 + 4b3a + 6b2a2 + 4ba3 + a4.
D. (a + b)4 = a4 + b4.

\textbf{{ANSWER}}

Lời giải
Đáp án đúng là D
Công thức khai triển nhị thức Newton (a + b)4 là:
(a + b)4 = a4 + 4a3b + 6a2b2 + 4ab3 + b4 = b4 + 4b3a + 6b2a2 + 4ba3 + a4.
Do đó phương án A, C đúng, phương án D sai.
Công thức khai triển nhị thức Newton (a – b)4 là:
(a + b)4 = a4 – 4a3b + 6a2b2 – 4ab3 + b4.
Do đó phương án B đúng.
Vậy ta chọn phương án D.

========================================================================

https://khoahoc.vietjack.com/thi-online/giai-sbt-toan-10-bai-4-nhi-thuc-newton-co-dap-an


\textbf{{QUESTION}}

Trong các phát biểu sau, phát biểu nào đúng?
A. (a + b)5 = a5 + 5a4b + 10a3b2 + 10a2b3 + 5ab4 + b5.
B. (a – b)5 = a5 – 5a4b + 10a3b2 + 10a2b3 – 5ab4 + b5.
C. (a + b)5 = a5 + b5.
D. (a – b)5 = a5 – b5.

\textbf{{ANSWER}}

Lời giải
Công thức khai triển nhị thức Newton (a + b)5 là:
(a + b)5 = a5 + 5a4b + 10a3b2 + 10a2b3 + 5ab4 + b5.
Do đó phương án A đúng, phương án C sai.
Công thức khai triển nhị thức Newton (a – b)5 là:
(a – b)5 = a5 – 5a4b + 10a3b2 – 10a2b3 + 5ab4 – b5.
Do đó các phương án B, D sai.
Vậy ta chọn phương án A.

========================================================================

https://khoahoc.vietjack.com/thi-online/giai-sbt-toan-10-bai-4-nhi-thuc-newton-co-dap-an


\textbf{{QUESTION}}

Hệ số của x3 trong khai triển biểu thức (2x – 1)4 là:
A. 32.
B. –32.
C. 8.
D. –8.

\textbf{{ANSWER}}

Lời giải
Đáp án đúng là B
Ta có: (2x – 1)4 = (2x)4 – 4.(2x)3.1 + 6.(2x)2.12 – 4.(2x).13 + 14 
= 16x4 – 32x3 + 24x2 – 8x + 1
Số hạng chứa x3 trong khai triển biểu thức (2x – 1)4 là –32x3.
Vậy hệ số của x3 trong khai triển biểu thức (2x – 1)4 là –32.
Do đó ta chọn phương án B.

========================================================================

https://khoahoc.vietjack.com/thi-online/giai-sbt-toan-10-bai-4-nhi-thuc-newton-co-dap-an


\textbf{{QUESTION}}

Hệ số của x trong khai triển biểu thức (x – 2)5 là:
A. 32.
B. –32.
C. 80.
D. –80.

\textbf{{ANSWER}}

Lời giải
Đáp án đúng là C
Ta có: (x – 2)5 = x5 – 5x4.2 + 10x3.22 – 10x2.23 + 5x.24 – 25
= x5 – 10x4 + 40x3 – 80x2 + 80x – 32
Số hạng chứa x trong khai triển biểu thức (x – 2)5 là 80x.
Vậy hệ số của x trong khai triển biểu thức (x – 2)5 là 80.
Do đó ta chọn phương án C.

========================================================================

https://khoahoc.vietjack.com/thi-online/giai-sbt-toan-10-bai-4-nhi-thuc-newton-co-dap-an


\textbf{{QUESTION}}

Khai triển các biểu thức sau:
(4x + 1)4;

\textbf{{ANSWER}}

Lời giải
(4x + 1)4 = (4x)4 + 4.(4x)3.1 + 6.(4x)2.12 + 4.4x.13 + 14
= 256x4 + 256x3 + 96x2 + 16x + 1.

========================================================================

https://khoahoc.vietjack.com/thi-online/20-cau-trac-nghiem-toan-10-ket-noi-tri-thuc-phuong-trinh-duong-thang-co-dap-an-phan-2/110843


\textbf{{QUESTION}}

Viết phương trình tham số của đường thẳng đi qua hai điểm A(−1; 3) và B(3; 1)
A. $$ \left\{\begin{array}{l}x=-1+2t\\ y=3+t\end{array}\right.$$
B. $$ \left\{\begin{array}{l}x=-1-2t\\ y=3-t\end{array}\right.$$
C. $$ \left\{\begin{array}{l}x=3+2t\\ y=1+t\end{array}\right.$$
D. $$ \left\{\begin{array}{l}x=-1+2t\\ y=3-t\end{array}\right.$$

\textbf{{ANSWER}}

Hướng dẫn giải
Đáp án đúng là: D
Ta có: $$ \overrightarrow{AB}=(4;-2)$$.
Chọn vectơ chỉ phương $$ \overrightarrow{u}=\frac{1}{2}\overrightarrow{AB}$$ = (2; −1).
Do đó, phương trình đường thẳng đi qua điểm A(−1; 3) và nhận $$ \overrightarrow{u}(2;-1)$$ làm vectơ chỉ phương là: $$ \left\{\begin{array}{l}x=-1+2t\\ y=3-t\end{array}\right.$$.

========================================================================

https://khoahoc.vietjack.com/thi-online/20-cau-trac-nghiem-toan-10-ket-noi-tri-thuc-phuong-trinh-duong-thang-co-dap-an-phan-2/110843


\textbf{{QUESTION}}

Cho đường thẳng d có phương trình tham số là: {x=3+ty=4−2t. Khi đó phương trình tổng quát của đường thẳng d là: 
A. x – 2y + 5 = 0;      
B. 3x + 4y + 5 = 0;                          
C. 2x + y – 10 = 0 ;

\textbf{{ANSWER}}

Hướng dẫn giải
Đáp án đúng là: C
Cách 1: Từ phương trình tham số của đường thẳng d ta có đường thẳng d đi qua điểm M(3; 4) và có vectơ chỉ phương →u(1;−2)$$ \overrightarrow{u}(1;-2)$$ nên có vectơ pháp tuyến là →n(2;1)$$ \overrightarrow{n}(2;1)$$. Khi đó phương trình tổng quát của đường thẳng d là: 2.(x – 3) + (y – 4) = 0 ⇔ 2x + y – 10 = 0.
Cách 2: Xét phương trình tham số {x=3+ty=4−2t⇔{t=x−3t=y−4−2$$ \left\{\begin{array}{l}x=3+t\\ y=4-2t\end{array}\right.\Leftrightarrow \left\{\begin{array}{l}t=x-3\\ t=\frac{y-4}{-2}\end{array}\right.$$
 ⇔x−3=y−4−2⇔−2(x−3)=y−4⇔2x+y−10=0$$ \Leftrightarrow x-3=\frac{y-4}{-2}\Leftrightarrow -2\left(x-3\right)=y-4\Leftrightarrow 2x+y-10=0$$.
Vậy phương trình tổng quát của đường thẳng d là: 2x + y – 10 = 0.

========================================================================

https://khoahoc.vietjack.com/thi-online/20-cau-trac-nghiem-toan-10-ket-noi-tri-thuc-phuong-trinh-duong-thang-co-dap-an-phan-2/110843


\textbf{{QUESTION}}

Cho đường thẳng ∆ có phương trình 3x – 4y + 2 = 0. Điểm nào sau đây không nằm trên đường thẳng ∆?
A. M1(2;2)$$ {M}_{1}(2;2)$$;           
B. M2(3;4)$$ {M}_{2}(3;4)$$;         
C. M3(−2;−1)$$ {M}_{3}(-2;-1)$$;

\textbf{{ANSWER}}

Hướng dẫn giải
Đáp án đúng là: B
+ Xét điểm M1(2;2)$$ {M}_{1}(2;2)$$
Với x = 2 và y = 2 ta có: 3.2 – 4.2 + 2 = 0 nên M1 ∈ ∆.
+ Xét điểm M2(3;4)$$ {M}_{2}(3;4)$$
Với x = 3 và y = 4 ta có: 3.3 – 4.4 + 2 = – 5 ≠ 0 nên M2 ∉ ∆.
+ Xét điểm M3(−2;−1)$$ {M}_{3}(-2;-1)$$
Với x = −2 và y = −1 ta có: 3.( −2) – 4.( −1) + 2 = 0 nên M3 ∈ ∆.
+ Xét điểm M4(0;12)$$ {M}_{4}\left(0;\frac{1}{2}\right)$$
Với x = 0 và y =12$$ \frac{1}{2}$$ ta có: 3.0 – 4. 12$$ \frac{1}{2}$$ + 2 = 0 nên M4 ∈ ∆.
Vậy điểm M2 không thuộc đường thẳng ∆

========================================================================

https://khoahoc.vietjack.com/thi-online/20-cau-trac-nghiem-toan-10-ket-noi-tri-thuc-phuong-trinh-duong-thang-co-dap-an-phan-2/110843


\textbf{{QUESTION}}

Phương trình đường thẳng d đi qua điểm M(−2; 3) và song song với đường thẳng EF với E(0; −1), F(−3; 0) là: 
A. {x=−2−ty=3+3t
B. {x=−2+3ty=3+t
C. {x=−2-3ty=3+t
D. {x=−2−ty=3−3t

\textbf{{ANSWER}}

Hướng dẫn giải
Đáp án đúng là: C
Ta có: →EF=(−3;1)$$ \overrightarrow{EF}=(-3;1)$$
Vì đường thẳng d song song với đường thẳng EF nên đường thẳng d nhận vectơ →EF$$ \overrightarrow{EF}$$ làm vectơ chỉ phương 
Vậy phương trình tham số của đường thẳng d đi qua điểm M(−2; 3) nhận →EF=(−3;1)$$ \overrightarrow{EF}=(-3;1)$$ làm vectơ chỉ phương là:
 {x=−2−3ty=3+t$$ \left\{\begin{array}{l}x=-2-3t\\ y=3+t\end{array}\right.$$.

========================================================================

https://khoahoc.vietjack.com/thi-online/20-cau-trac-nghiem-toan-10-ket-noi-tri-thuc-phuong-trinh-duong-thang-co-dap-an-phan-2/110843


\textbf{{QUESTION}}

Cho đường thẳng ∆ có phương trình tổng quát là x + 2y + 5 = 0. Phương trình tham số của đường thẳng ∆ là: 
A. {x=1+ty=−3+2t
B. {x=1+2ty=−3−t
C. 2x – y – 5 = 0;

\textbf{{ANSWER}}

Hướng dẫn giải
Đáp án đúng là: B
Đường thẳng ∆ có vectơ pháp tuyến là →n=(1;2). Do đó vectơ chỉ phương của đường thẳng ∆ là →u=(2;−1).
Chọn x = 1 ⇒ y = – 3. Ta có điểm M(1; – 3) là điểm thuộc đường thẳng ∆.
Vậy phương trình tham số của đường thẳng ∆ là: {x=1+2ty=−3−t.

========================================================================

https://khoahoc.vietjack.com/thi-online/30-de-thi-thu-thpt-quoc-gia-mon-toan-nam-2022-co-loi-giai/116468


\textbf{{QUESTION}}

Hàm số nào dưới đây đồng biến trên khoảng $$ \left(0;\text{\hspace{0.17em}}+\infty \right)$$?
C. $$ y={\mathrm{log}}_{\frac{2}{\pi }}x$$.

\textbf{{ANSWER}}

Chọn A
Ta có $$ \frac{3}{e}>1$$ nên hàm số $$ y={\mathrm{log}}_{\frac{3}{e}}x$$ đồng biến trên $$ \left(0;\text{\hspace{0.17em}}+\infty \right)$$.

========================================================================

https://khoahoc.vietjack.com/thi-online/30-de-thi-thu-thpt-quoc-gia-mon-toan-nam-2022-co-loi-giai/116468


\textbf{{QUESTION}}

Với giá trị nào của tham số m thì phương trình 4x−(m+1).2x+1+3m−4=0 có hai nghiệm x1, x2 thỏa mãn x1+x= 3 ?

\textbf{{ANSWER}}

Chọn B
Đặt t=2x$$ t={2}^{x}$$, ĐK t > 0.
Phương trình trở thành t2−2(m+1)t+3m−4=0$$ {t}^{2}-2(m+1)t+3m-4=0$$(*), có Δ'=m2−m+5>0, ∀m$$ \Delta \text{'}={m}^{2}-m+5>0,\text{\hspace{0.17em}}\forall m$$.
Phương trình (*) có hai nghiệm dương {t1+t2>0t1t2>0⇔{m+1>03m−4>0⇔m>34$$ \left\{\begin{array}{c}{t}_{1}+{t}_{2}>0\\ {t}_{1}{t}_{2}>0\end{array}\right.\Leftrightarrow \left\{\begin{array}{c}m+1>0\\ 3m-4>0\end{array}\Leftrightarrow m>\frac{3}{4}\right.$$.
Khi đó
x1=log2t1, x2=log2t2⇒x1+x2=3⇔log2t1+log2t2=3⇔log2t1t2=3⇔t1t2=23⇔3m−4=8⇔m=4$$ \begin{array}{l}{x}_{1}={\mathrm{log}}_{2}{t}_{1},\text{\hspace{0.17em}}{x}_{2}={\mathrm{log}}_{2}{t}_{2}\\ \Rightarrow {x}_{1}+{x}_{2}=3\\ \Leftrightarrow {\mathrm{log}}_{2}{t}_{1}+{\mathrm{log}}_{2}{t}_{2}=3\Leftrightarrow {\mathrm{log}}_{2}{t}_{1}{t}_{2}=3\Leftrightarrow {t}_{1}{t}_{2}={2}^{3}\\ \Leftrightarrow 3m-4=8\Leftrightarrow m=4\end{array}$$

========================================================================

https://khoahoc.vietjack.com/thi-online/15-cau-trac-nghiem-toan-7-chan-troi-sang-tao-bai-3-lam-tron-so-va-uoc-luong-ket-qua-co-dap-an-phan-2/103423


\textbf{{QUESTION}}

Làm tròn số $$ \sqrt{3}$$ với độ chính xác d = 0,002 ta được:

\textbf{{ANSWER}}

Hướng dẫn giải
Đáp án đúng là: D
Ta có:
$$ \sqrt{3}=1,\mathrm{73205...}$$ 
Làm tròn với độ chính xác d = 0,005 tức là ta làm tròn số đến hàng phần trăm.
Áp dụng quy tắc làm tròn ta được số 1,73.
Vậy ta chọn phương án D.

========================================================================

https://khoahoc.vietjack.com/thi-online/15-cau-trac-nghiem-toan-7-chan-troi-sang-tao-bai-3-lam-tron-so-va-uoc-luong-ket-qua-co-dap-an-phan-2/103423


\textbf{{QUESTION}}

Ước lượng phép tính 6 122 . 99 ta được:

\textbf{{ANSWER}}

Hướng dẫn giải
Đáp án đúng là: A
Làm tròn các thừa số đến hàng trăm ta được:
6122 ≈ 6 100 và 99 ≈ 100.
Nên ta có 
6 122 . 99 ≈ 6 100 . 100 = 610 000 ≈ 600 000.
Ta chọn phương án A.

========================================================================

https://khoahoc.vietjack.com/thi-online/15-cau-trac-nghiem-toan-7-chan-troi-sang-tao-bai-3-lam-tron-so-va-uoc-luong-ket-qua-co-dap-an-phan-2/103423


\textbf{{QUESTION}}

Một số sau khi làm tròn đến hàng nghìn cho kết quả là 21 000. Số đó có thể lớn nhất là bao nhiêu?

\textbf{{ANSWER}}

Hướng dẫn giải
Đáp án đúng là: C
Số lớn nhất có thể làm tròn đến hàng nghìn cho kết quả 21 000 là 20 999.
Vậy ta chọn phương án C.

========================================================================

https://khoahoc.vietjack.com/thi-online/15-cau-trac-nghiem-toan-7-chan-troi-sang-tao-bai-3-lam-tron-so-va-uoc-luong-ket-qua-co-dap-an-phan-2/103423


\textbf{{QUESTION}}

Thực hiện phép tính (12,345 + 2,7) – (5,22 – 2,55) rồi làm tròn kết quả đến chữ số thập phân thứ nhất, ta được kết quả là:

\textbf{{ANSWER}}

Hướng dẫn giải
Đáp án đúng là: B
Ta có:(12,345 + 2,7) – (5,22 – 2,55)
= 15,045 – 2,67 = 12,375 
Làm tròn số 12,375 đến chữ số thập phân thứ nhất ta được 12,4.
Vậy ta chọn phương án B.

========================================================================

https://khoahoc.vietjack.com/thi-online/15-cau-trac-nghiem-toan-7-chan-troi-sang-tao-bai-3-lam-tron-so-va-uoc-luong-ket-qua-co-dap-an-phan-2/103423


\textbf{{QUESTION}}

Gọi x là số làm tròn đến hàng chục của số a = 3333. Khi đó ta có:

\textbf{{ANSWER}}

Hướng dẫn giải:
Đáp án đúng là: B
• Ta có a = 3333, làm tròn số đến hàng chục ta được x = 3330.
Ta có |a – x| = |3333 – 3330| = |3| = 3.
Mà 2 < 3 < 6 nên 2 < |a – x| < 6.
Do đó A và C là sai.
• Ta có:
x – 5 = 3330 – 5 = 3325;
x + 5 = 3330 + 5 = 3335.
Nên suy ra : x – 5 ≤ a ≤ x + 5.
Do đó A đúng.
Vậy ta chọn phương án B.

========================================================================

https://khoahoc.vietjack.com/thi-online/de-on-thi-vao-10-mon-toan-co-dap-an-moi-nhat/86646


\textbf{{QUESTION}}

Cho biểu thức: $P = \frac{{x\sqrt x + 1}}{{\sqrt x + 1}} - \sqrt x $
1) Tìm điều kiện xác định và rút gọn biểu thức P?
2) Tính giá trị của P tại x thỏa mãn ${x^2} - \frac{{\sqrt 5 }}{{\sqrt 5 - 2}}x - \left( {6 + 2\sqrt 5 } \right) = 0?$

\textbf{{ANSWER}}

1) Điều kiện xác định: $x \ge 0$.
Ta có: $P = \frac{{x\sqrt x + 1}}{{\sqrt x + 1}} - \sqrt x = \frac{{{{\left( {\sqrt x } \right)}^3} + {1^3}}}{{\sqrt x + 1}} - \sqrt x $
$ = \frac{{\left( {\sqrt x + 1} \right)\left( {x - \sqrt x + 1} \right)}}{{\sqrt x + 1}} - \sqrt x $
$ = x - \sqrt x + 1 - \sqrt x = x - 2\sqrt x + 1.$
Vậy $P = x - 2\sqrt x + 1.$
Cách 2: Đặt $a = \sqrt x \left( {a \ge 0} \right).$
Ta có: $P = \frac{{{a^3} + 1}}{{a + 1}} - a = \frac{{\left( {a + 1} \right)\left( {{a^2} - a + 1} \right)}}{{a + 1}} - a = {a^2} - 2a + 1 = x - 2\sqrt x + 1.$
Nhận xét: Bài toán rút gọn biểu thức áp dụng quy tắc tìm điều kiện và các phương pháp phân tích đa thức thành nhân tử.
2) Ta có: ${x^2} - \frac{{\sqrt 5 }}{{\sqrt 5 - 2}}x - \left( {6 + 2\sqrt 5 } \right) = 0 \Leftrightarrow {x^2} - \left( {5 + 2\sqrt 5 } \right)x - \left( {6 + 2\sqrt 5 } \right) = 0$
$ \Leftrightarrow \left( {x + 1} \right)\left[ {x - \left( {6 + 2\sqrt 5 } \right)} \right] = 0 \Leftrightarrow \left[ \begin{array}{l}x = - 1\\x = 6 + 2\sqrt 5 \end{array} \right. \Rightarrow x = 6 + 2\sqrt 5 $ (vì $x \ge 0$)
Nên ta có $P = \left( {6 + 2\sqrt 5 } \right) - 2\sqrt {6 + 2\sqrt 5 } + 1 = 7 + 2\sqrt 5 - 2\sqrt {{{\left( {\sqrt 5 + 1} \right)}^2}} $
$ = 7 + 2\sqrt 5 - 2\sqrt {{{\left( {\sqrt 5 + 1} \right)}^2}} = 7 + 2\sqrt 5 - 2\sqrt 5 - 2 = 5.$
Vậy $P = 5$.
Nhận xét: Bài toán tìm giá trị của biểu thức khi biết biến thỏa mãn một điều kiện nào đó. Ta tìm biến rồi thay vào biểu thức để tìm giá trị.

========================================================================

https://khoahoc.vietjack.com/thi-online/de-on-thi-vao-10-mon-toan-co-dap-an-moi-nhat/86646


\textbf{{QUESTION}}

1)  Giải bài toán sau bằng cách lập phương trình hoặc hệ phương trình.
Một xe mô-tô đi từ A đến B (cách nhau 60km) theo thời gian đã định. Nửa quãng đường đầu xe đi với vận tốc nhanh hơn vận tốc dự định 10km/h và nửa quãng đường sau xe đi với vận tốc chậm hơn vận tốc dự định 6km/h. Biết rằng xe về đến B đúng thời gian quy định, hỏi vận tốc dự định là bao nhiêu?
2) Tìm các giá trị m để hàm số y=(√m−2)x+3$y = \left( {\sqrt m  - 2} \right)x + 3$ đồng biến.

\textbf{{ANSWER}}

1) Gọi x (km/h) là vận tốc dự định.
Thời gian dự định để đến B với vận tốc trên là 60x$\frac{{60}}{x}$ (giờ).
Nửa quãng đường đầu xe đi nhanh hơn với vận tốc dự định 10(km/h) nên tốn 30x+10$\frac{{30}}{{x + 10}}$(giờ).
 Nửa quãng đường sau xe đi chậm hơn với vận tốc dự định 6(km/h) nên tốn 30x−6$\frac{{30}}{{x - 6}}$ (giờ).
Do đến B đúng thời gian quy định nên ta có phương trình
60x=30x+10+30x−6⇔60(x+10)(x−6)=x[30(x−6)+30(x+10)]$\frac{{60}}{x} = \frac{{30}}{{x + 10}} + \frac{{30}}{{x - 6}} \Leftrightarrow 60\left( {x + 10} \right)\left( {x - 6} \right) = x\left[ {30\left( {x - 6} \right) + 30\left( {x + 10} \right)} \right]$
⇔60+240x−3600=60x2+120x⇔120x=3600⇔x=30$ \Leftrightarrow 60 + 240x - 3600 = 60{x^2} + 120x \Leftrightarrow 120x = 3600 \Leftrightarrow x = 30$(km/h).
Nhận xét: Giải bài toán bằng cách lập hệ phương trình từ kiến thức về chuyển động cơ bản và chuyển động trên dòng nước:
“Quãng đường = Vận tốc x Thời gian”
2) Hàm số y=(√m−2)x+3$y = \left( {\sqrt m - 2} \right)x + 3$đồng biến khi {m≥2√m−2>0$\left\{ \begin{array}{l}m \ge 2\\\sqrt m - 2 > 0\end{array} \right.$
⇔{m≥0√m>2⇔{m≥0m>4⇔m>4$ \Leftrightarrow \left\{ \begin{array}{l}m \ge 0\\\sqrt m > 2\end{array} \right. \Leftrightarrow \left\{ \begin{array}{l}m \ge 0\\m > 4\end{array} \right. \Leftrightarrow m > 4$
Vậy m>4$m > 4$.

========================================================================

https://khoahoc.vietjack.com/thi-online/bo-10-de-thi-giua-ki-2-toan-8-chan-troi-sang-tao-cau-truc-moi-co-dap-an


\textbf{{QUESTION}}

Điểm nào dưới đây thuộc đồ thị hàm số $y = 2x - 4$?

\textbf{{ANSWER}}

Đáp án đúng là: A
Thay $x = 0$ vào $y = 2x - 4$, ta được: $y = 2.0 - 4 = - 4.$ 
Do đó, điểm $M\left( {0; - 4} \right)$ thuộc đồ thị hàm số $y = 2x - 4$.

========================================================================

https://khoahoc.vietjack.com/thi-online/6-cau-trac-nghiem-toan-6-ket-noi-tri-thuc-bai-tap-on-cuoi-chuong-9-trang-99-co-dap-an


\textbf{{QUESTION}}

Nam muốn tìm kiếm thông tin để trả lời các câu hỏi dưới đây.
Em hãy gợi ý giúp Nam cách thu thập dữ liệu phù hợp cho mỗi câu hỏi:
a) Năm quốc gia nào có diện tích lớn nhất?
b) Có bao nhiêu bạn trong lớp có đồng hồ đeo tay?
c)Trong tuần trước, tổ nào trong lớp có nhiều lượt đi học muộn nhất?

\textbf{{ANSWER}}

Học sinh có thể thu thập thông tin nhiều cách khác nhau. Dưới đây là ví dụ
a. Khảo sát qua mạng Internet
b. Quan sát trực tiếp các bạn trong lớp 
c. Lập bảng thống kê từ tổ trưởng.

========================================================================

https://khoahoc.vietjack.com/thi-online/6-cau-trac-nghiem-toan-6-ket-noi-tri-thuc-bai-tap-on-cuoi-chuong-9-trang-99-co-dap-an


\textbf{{QUESTION}}

Việt muốn tìm hiểu về đội bóng yêu thích của một số bạn nam. Em hãy giúp Việt:
a) Lập phiếu hỏi để thu thập dữ liệu;
b) Thu thập trong phạm vi lớp em và ghi lại kết quả dưới dạng bảng.
Từ kết quả thu được em có kết luận gì?

\textbf{{ANSWER}}

a) Lập phiếu hỏi:
Bạn yêu thích đội bóng nào?
Manchester United □
Manchester City □
Liverpool □
Khác □
(Với mỗi dấu hỏi tích X vào 1 trong các lựa chọn)
b) Bảng thống kê số lượng học sinh yêu thích đội bóng của các bạn nam trong lớp.
Ví dụ ở một lớp 6 có 20 bạn nam.
Đội bóng
Manchester United
Manchester City
Liverpool
Đội bóng khác
Số học sinh
10
5
3
2
Từ kết quả bảng ta thấy:
Số học sinh yêu thích đội Manchester United nhiều nhất là 10 chiếm một nửa số bạn nam trong lớp yêu thích.

========================================================================

https://khoahoc.vietjack.com/thi-online/de-kiem-tra-1-tiet-toan-9-chuong-3-dai-so-co-dap-an/44373


\textbf{{QUESTION}}

Phần trắc nghiệm
Nội dung câu hỏi 1
Phương trình nào sau đây là phương trình bậc nhất hai ẩn?
A.xy + y = 5       
B.3x - 2y = 0       
C.x + xy = 2      
D.x + y = xy

\textbf{{ANSWER}}

Đáp án là B

========================================================================

https://khoahoc.vietjack.com/thi-online/de-kiem-tra-1-tiet-toan-9-chuong-3-dai-so-co-dap-an/44373


\textbf{{QUESTION}}

Phương trình 2x – 3y = 0 có nghiệm tổng quát là:
A. (x∈ℝ;y=2x3)
B. (x∈ℝ;y=3x2)
C. (x=23;y∈ℝ)
D. (x=2y3;y∈ℝ)

\textbf{{ANSWER}}

Đáp án là A

========================================================================

https://khoahoc.vietjack.com/thi-online/de-kiem-tra-1-tiet-toan-9-chuong-3-dai-so-co-dap-an/44373


\textbf{{QUESTION}}

Cặp số (-2;1) là nghiệm của phương trình nào dưới đây?
A.2x + 0y = -3 
B.0x - 3y = -3
C.2x - 3y = 1       
D.2x - y = 0

\textbf{{ANSWER}}

Đáp án là B

========================================================================

https://khoahoc.vietjack.com/thi-online/de-kiem-tra-1-tiet-toan-9-chuong-3-dai-so-co-dap-an/44373


\textbf{{QUESTION}}

Hệ phương trình {x-3y=-2x+2y=3 có nghiệm là:
A. (x = 1 ; y = -1)
B. (x = -1 ; y = 1)
C. (x = 2 ; y = -1)
D. (x = 1 ; y = 1)

\textbf{{ANSWER}}

Đáp án là D

========================================================================

https://khoahoc.vietjack.com/thi-online/de-kiem-tra-1-tiet-toan-9-chuong-3-dai-so-co-dap-an/44373


\textbf{{QUESTION}}

Phương trình đường thẳng đi qua hai điểm M(-2;1) và N(2;0) là:
A. y=2x+1
B. y= -2x + 4
C. y=-14x+12
D. y = 3x + 7

\textbf{{ANSWER}}

Đáp án là C

========================================================================

https://khoahoc.vietjack.com/thi-online/15-cau-trac-nghiem-toan-7-ket-noi-tri-thuc-bai-26-phep-cong-va-phep-tru-da-thuc-mot-bien-co-dap-an/111081


\textbf{{QUESTION}}

Tìm hệ số tự do của hiệu 2A – B với A = 2x2 – 4x3 + 2x – 5; B = 2x3 – 3x3 + 4x + 5.

\textbf{{ANSWER}}

Đáp án đúng là: A
Hệ số tự do của A là −5.
Hệ số tự do của B là 5.
Hệ số tự do của hiệu 2A – B = 2.(−5) – 5 = −15.

========================================================================

https://khoahoc.vietjack.com/thi-online/15-cau-trac-nghiem-toan-7-ket-noi-tri-thuc-bai-26-phep-cong-va-phep-tru-da-thuc-mot-bien-co-dap-an/111081


\textbf{{QUESTION}}

Cho đa thức H(x) = x3 – 2x2 + 1. Tìm đa thức P(x) sao cho H(x) + P(x) = x4 + 2x3 + x.

\textbf{{ANSWER}}

Đáp án đúng là: C
H(x) + P(x) = x4 + 2x3 + x
Suy ra 
P(x) = x4 + 2x3 + x – H(x) 
= x4 + 2x3 + x – (x3 – 2x2 + 1)
= x4 + 2x3 + x – x3 + 2x2 – 1 
= x4 + (2x3 – x3) + 2x2 + x – 1 
= x4 + x3 + 2x2 + x – 1.
Vậy P(x) = x4 + x3 + 2x2 + x – 1.

========================================================================

https://khoahoc.vietjack.com/thi-online/15-cau-trac-nghiem-toan-7-ket-noi-tri-thuc-bai-26-phep-cong-va-phep-tru-da-thuc-mot-bien-co-dap-an/111081


\textbf{{QUESTION}}

Cho A = x2 + 3x3 – x – 1. Tìm đa thức B sao cho: A – B = −2x2

\textbf{{ANSWER}}

Đáp án đúng là: D 
A – B = −2x2
Suy ra:
B = A + 2x2
= x2 + 3x3 – x – 1 + 2x2
= 3x3 + 3x2 – x – 1 
Vậy B = 3x3 + 3x2 – x – 1.

========================================================================

https://khoahoc.vietjack.com/thi-online/15-cau-trac-nghiem-toan-7-ket-noi-tri-thuc-bai-26-phep-cong-va-phep-tru-da-thuc-mot-bien-co-dap-an/111081


\textbf{{QUESTION}}

Cho hai đa thức P = −5x5 + 3x2 + 3x + 1 và Q = 5x5 + x4 + x3 + 2.
Bậc của mỗi đa thức P + Q và P – Q lần lượt là:

\textbf{{ANSWER}}

Đáp án đúng là: C 
P + Q = (−5x5 + 3x2 + 3x + 1) + (5x5 + x4 + x3 + 2)
= −5x5 + 3x2 + 3x + 1 + 5x5 + x4 + x3 + 2
= (−5x5 + 5x5) + x4 + x3 + 3x2 + 3x + (1 + 2)
= x4 + x3 + 3x2 + 3x + 3.
Bậc của  đa thức P + Q là 4.
P − Q = (−5x5 + 3x2 + 3x + 1) − (5x5 + x4 + x3 + 2)
= −5x5 + 3x2 + 3x + 1 – 5x5 – x4 – x3 – 2 
= (−5x5 – 5x5) – x4 – x3 + 3x2 + 3x + (1 – 2)
= −10x5 – x4 – x3 + 3x2 + 3x – 2 
Bậc của đa thức P + Q là 5.

========================================================================

https://khoahoc.vietjack.com/thi-online/15-cau-trac-nghiem-toan-7-ket-noi-tri-thuc-bai-26-phep-cong-va-phep-tru-da-thuc-mot-bien-co-dap-an/111081


\textbf{{QUESTION}}

Tìm hệ số cao nhất của đa thức f(x) biết f(x) + k(x) = g(x) và k(x) = 2x3 + 4x2 + 2x và g(x) = 2x2 + 3x.

\textbf{{ANSWER}}

Đáp án đúng là: C 
f(x) + k(x) = g(x)
Suy ra 
f(x) = g(x) – k(x) 
= (2x2 + 3x) – (2x3 + 4x2 + 2x)
= 2x2 + 3x – 2x3 – 4x2 – 2x
= −2x3 – 2x2 + x
Vậy hệ số cao nhất của đa thức f(x) là −2

========================================================================

https://khoahoc.vietjack.com/thi-online/top-8-de-kiem-tra-toan-8-hoc-ki-2-chuong-4-dai-so-co-dap-an-cuc-hay/44721


\textbf{{QUESTION}}

Đúng điền Đ, sai điền S vào các chỗ trống ở các khẳng định sau:
 
$$ a)\quad -3\quad +\quad 4\quad \ge \quad 3\quad ....\quad \quad \phantom{\rule{0ex}{0ex}}b)\quad {x}^{2}\quad +\quad 2\quad \ge \quad 2\quad ....$$

\textbf{{ANSWER}}

a) S       b) Đ

========================================================================

https://khoahoc.vietjack.com/thi-online/top-8-de-kiem-tra-toan-8-hoc-ki-2-chuong-4-dai-so-co-dap-an-cuc-hay/44721


\textbf{{QUESTION}}

Nếu -5a > -5b thì:
A. a < b
B. a > b
C. a = b

\textbf{{ANSWER}}

Chọn A

========================================================================

https://khoahoc.vietjack.com/thi-online/giai-sach-bai-tap-toan-9-tap-1/24845


\textbf{{QUESTION}}

Trong các bảng sau ghi các giá trị tương ứng của x và y. Bảng nào xác định y là hàm số của x? Vì sao?

\textbf{{ANSWER}}

Xác định y là hàm số của biến số x vì với mỗi giá trị của x ta xác định được một giá trị tương ứng duy nhất của y.

========================================================================

https://khoahoc.vietjack.com/thi-online/giai-sach-bai-tap-toan-9-tap-1/24845


\textbf{{QUESTION}}

Trong các bảng sau ghi các giá trị tương ứng của x và y. Bảng nào xác định y là hàm số của x? Vì sao?

\textbf{{ANSWER}}

Xác định y không phải là hàm số của biến số x vì với mỗi giá trị của x ta xác định được hai giá trị khác nhau của y.
Vì dụ x = 3 thì y = 6 và y = 4.

========================================================================

https://khoahoc.vietjack.com/thi-online/20-cau-trac-nghiem-toan-12-chan-troi-sang-tao-bai-1-khoang-bien-thien-khoang-tu-phan-vi-cua-mau-so-l


\textbf{{QUESTION}}

I. Nhận biết
Gọi ${Q_1},{Q_2},{Q_3}$ là tứ phân vị thứ nhất, tứ phân vị thứ hai và thứ ba của mẫu số liệu ghép nhóm. Khoảng tứ phân vị của mẫu số liệu ghép nhóm là:
A. $\Delta Q = {Q_1} - {Q_3}.$
B. $\Delta Q = {Q_3} - {Q_1}.$
C. $\Delta Q = {Q_1} - {Q_2}.$
D. $\Delta Q = {Q_2} - {Q_1}.$

\textbf{{ANSWER}}

Đáp án đúng là: B
Khoảng tứ phân vị của mẫu ghép nhóm có công thức là: $\Delta Q = {Q_3} - {Q_1}.$

========================================================================

https://khoahoc.vietjack.com/thi-online/bo-de-thi-toan-thpt-quoc-gia-nam-2022-co-loi-giai-30-de/67731


\textbf{{QUESTION}}

Có bao nhiêu số có bốn chữ số khác nhau được tạo thành từ các chữ số 1,2,3,4,5?
A. $$ {A}_{5}^{4}$$
B. $$ {P}_{5}$$
C. $$ {C}_{5}^{4}$$
D. $$ {P}_{4}$$

\textbf{{ANSWER}}

Chọn A
Số tự nhiên gồm bốn chữ số khác nhau được tạo thành từ các chữ số 1,2,3,4,5 là một chỉnh hợp chập 4 của 5 phần tử
Vậy có $$ {A}_{5}^{4}$$ số cần tìm.

========================================================================

https://khoahoc.vietjack.com/thi-online/10-bai-tap-biesn-co-doc-lap-co-loi-giai


\textbf{{QUESTION}}

Cho hai biến cố A và B độc lập với nhau. Phát biểu nào sau đây là đúng?

\textbf{{ANSWER}}

Đáp án đúng là: B
Cặp biến cố A và B được gọi là độc lập nếu việc xảy ra hay không xảy ra của biến cố này không ảnh hưởng tới xác suất xảy ra của biến cố kia.

========================================================================

https://khoahoc.vietjack.com/thi-online/10-bai-tap-biesn-co-doc-lap-co-loi-giai


\textbf{{QUESTION}}

Xét phép thử gieo ngẫu nhiên một con xúc xắc đồng chất sáu mặt hai lần. Gọi A là biến cố: "Số chấm thu được ở lần giao thứ nhất là số nhỏ hơn 3", B là biến cố: "Số chấm thu được ở lần gieo thứ hai là số lớn hơn hoặc bằng 4" và C là biến cố: "Số chấm thu được ở hai lần gieo là số lẻ”. Có bao nhiêu cặp biến cố độc lập?

\textbf{{ANSWER}}

Đáp án đúng là: B
Ta thấy A và B là các biến cố độc lập vì việc kết quả xảy ra ở lần gieo thứ nhất không làm ảnh hưởng đến kết quả xảy ra ở lần gieo thứ hai.

========================================================================

https://khoahoc.vietjack.com/thi-online/10-bai-tap-biesn-co-doc-lap-co-loi-giai


\textbf{{QUESTION}}

Rút hộp đựng 9 thẻ được đánh số từ 1, 2, 3, . . . , 9. An rút ngẫu nhiên 1 thẻ rồi trả lại hộp. Sau đó Bình rút 1 thẻ từ hộp đó. Biến cố A: “An rút được thẻ số chẵn” và biến cố B: “Bình rút được thẻ số lẻ” là hai biến cố:

\textbf{{ANSWER}}

Đáp án đúng là: B
Dù A có xảy ra (An lấy được thẻ số chẵn) hay A không xảy ra (An lấy được thẻ số lẻ) ta đều có  P(B)=59.
Dù B có xảy ra (Bình lấy được được thẻ số lẻ) hay B không xảy ra (Bình lấy được thẻ số chẵn) ta đều có  P(A)=49.
Do đó việc xảy ra hay không xảy ra của biến cố này không ảnh hưởng tới xác suất xảy ra của biến cố kia. 
Vậy A và B độc lập.

========================================================================

https://khoahoc.vietjack.com/thi-online/10-bai-tap-biesn-co-doc-lap-co-loi-giai


\textbf{{QUESTION}}

Gieo đồng tiền hai lần. Biến cố A “Lần gieo thứ nhất mặt ngửa xuất hiện”. Trong các biến cố dưới đây, biến cố nào độc lập với biến cố A?

\textbf{{ANSWER}}

Đáp án đúng là: A
Kết quả gieo ở lần thứ nhất không làm ảnh hưởng đến kết quả ở lần gieo thứ 2.
Do đó biến cố A: “Lần gieo thứ nhất mặt ngửa xuất hiện” độc lập với biến cố “Lần gieo thứ hai mặt ngửa xuất hiện”.

========================================================================

https://khoahoc.vietjack.com/thi-online/10-bai-tap-biesn-co-doc-lap-co-loi-giai


\textbf{{QUESTION}}

Hộp I chứa bốn cái thẻ được đánh số 1, 2, 3, 4. Hộp II chứa bốn cái thẻ được đánh số 5, 6, 7, 8. Mỗi hộp rút ngẫu nhiên một thẻ. Biến cố A: “Rút từ hộp I thẻ số chẵn”. Trong các biến cố sau, biến cố nào độc lập với biến cố A?

\textbf{{ANSWER}}

Đáp án đúng là: D
Việc xảy ra hay không xảy ra ở lần rút thẻ từ hộp I không ảnh hưởng đến xác suất xảy ra ở lần rút thẻ từ hộp II.
Do đó biến cố “Rút từ hộp II thẻ lẻ” hoặc “Rút từ hộp II thẻ chẵn” độc lập với biến cố A.

========================================================================

https://khoahoc.vietjack.com/thi-online/bai-tap-gia-tri-cua-mot-bieu-thuc-dai-so-co-dap-an


\textbf{{QUESTION}}

Giá trị của biểu thức $$ {x}^{3}\quad +\quad 2{x}^{2}\quad -\quad 3$$ tại x = 2 là
A. 13
B. 10
C. 19
D. 9

\textbf{{ANSWER}}

Thay x = 2 vào biểu thức $$ {x}^{3}\quad +\quad 2{x}^{2}\quad -\quad 3$$ ta được
$$ {2}^{3}\quad +\quad 2.{2}^{2}\quad -\quad 3\quad =\quad 8\quad +\quad 8\quad -\quad 3\quad =\quad 13$$
Chọn đáp án A

========================================================================

https://khoahoc.vietjack.com/thi-online/bai-tap-gia-tri-cua-mot-bieu-thuc-dai-so-co-dap-an


\textbf{{QUESTION}}

Cho biểu thức đại số A = x2 - 3x + 8$$ A\quad =\quad {x}^{2}\quad -\quad 3x\quad +\quad 8$$. Giá trị của A tại x = -2 là:
A. 13
B. 18
C. 19
D. 9

\textbf{{ANSWER}}

Thay vào biểu thức ta có :
(-2)2 - 3.(-2) + 8 = 4 + 6 + 8 = 18$$ {\left(-2\right)}^{2}\quad -\quad 3.(-2)\quad +\quad 8\quad =\quad 4\quad +\quad 6\quad +\quad 8\quad =\quad 18$$
Vậy A = 18 tại x = -2
Chọn đáp án B

========================================================================

https://khoahoc.vietjack.com/thi-online/bai-tap-gia-tri-cua-mot-bieu-thuc-dai-so-co-dap-an


\textbf{{QUESTION}}

Cho biểu thức đại số B = x3 + 6y - 35. Giá trị của B tại x = 3, y = -4 là:
A. 16
B. 86
C. -32
D. -28

\textbf{{ANSWER}}

Thay x = 3, y = -4 vào biểu thức B để tìm giá trị của biểu thức B ta có:
 
33 + 6.(-4) - 35 = 27 - 24 - 35 = 3 - 35 = -32$$ {3}^{3}\quad +\quad 6.(-4)\quad -\quad 35\quad =\quad 27\quad -\quad 24\quad -\quad 35\quad =\quad 3\quad -\quad 35\quad =\quad -32$$
Vậy B = -32 tại x = 3, y = -4
Chọn đáp án C

========================================================================

https://khoahoc.vietjack.com/thi-online/bai-tap-phep-tru-va-phep-chia


\textbf{{QUESTION}}

Tính:
a) 217 - 320 : 4;
b) 5052 : 5- 25 : 5
c) 640 : 32 + 32
d) 2180-180:2:9.

\textbf{{ANSWER}}

a) 217 - 320 : 4 = 217 - 80 = 137
b) 5025 : 5 - 25 : 5 = 1005 - 5 = 1000
c) 640 : 32 + 32 = 20 + 32 = 52
d) 2180 - 180 : 2 : 9 = 2180 - 10 = 2170

========================================================================

https://khoahoc.vietjack.com/thi-online/bai-tap-phep-tru-va-phep-chia


\textbf{{QUESTION}}

Tính:
a) 982 - 420 :20;
b) (328 - 8): 32;
c) 1000: 4 + 6;
d) 930 : 31 - 1.

\textbf{{ANSWER}}

a) 961
b) 10   
c) 256
d) 29

========================================================================

https://khoahoc.vietjack.com/thi-online/de-thi-toan-lop-8-giua-hoc-ki-2-nam-2020-2021-co-dap-an/61222


\textbf{{QUESTION}}

Điều kiện xác định của phương trình $$ \frac{\mathrm{x}}{3\left(\mathrm{x}-1\right)}+\frac{\mathrm{x}}{2\mathrm{x}+4}=\frac{2\mathrm{x}}{\left(\mathrm{x}+2\right)\left(\mathrm{x}-1\right)}$$ là
A. $$ \mathrm{x}\ne 1$$
B. $$ \mathrm{x}\ne 1$$ và $$ \mathrm{x}\ne -2$$
C. D.
D. $$ \mathrm{x}\ne 1$$ và $$ \mathrm{x}\ne 2$$

\textbf{{ANSWER}}

Đáp án B
Phương trình xác định khi $$ \left\{\begin{array}{c}\mathrm{x}-1\ne 0\\ \mathrm{x}+2\ne 0\end{array}\right.\Rightarrow \left\{\begin{array}{c}\mathrm{x}\ne 1\\ \mathrm{x}\ne -2\end{array}\right.$$

========================================================================

https://khoahoc.vietjack.com/thi-online/100-cau-trac-nghiem-cung-va-goc-luong-giac-nang-cao


\textbf{{QUESTION}}

Tính giá trị biểu thức sau : B = cos00 + cos200 + cos 400 + ... + cos1600 + cos1800.
A. -1
B. 0
C. 1
D. 2

\textbf{{ANSWER}}

Chọn B.
Ta có: B = ( cos00 + cos1800) + (cos200 + cos1600)  +...+ cos800 + cos1000)
= (cos00 - cos00)  + (cos200 - cos 200) + ... + (cos800 - cos800) = 0

========================================================================

https://khoahoc.vietjack.com/thi-online/100-cau-trac-nghiem-cung-va-goc-luong-giac-nang-cao


\textbf{{QUESTION}}

Tính giá trị biểu thức sau C = tan 50 tan 100 tan 150 ..tan800 tan850
A. 0
B. 1
C. 2
D. 4

\textbf{{ANSWER}}

Chọn B.
Ta có
C = ( tan50 . tan 850  ) .( tan 150  tan 750 ) ...tan 450
= ( tan50 .cot 50  ) .( tan 150  cot 150 ) ..tan 450 = 1
( do với 2 góc phụ nhau thì tan góc này bằng cot  góc kia)

========================================================================

https://khoahoc.vietjack.com/thi-online/10-bai-tap-xac-dinh-mien-snghiem-cua-bat-phuong-trinh-hai-an-co-loi-giai


\textbf{{QUESTION}}

Bất phương trình nào sau đây là bất phương trình bậc nhất hai ẩn ? 
A. 2x – 4y + 7 ≥ 0;
B. 5x3 – 4y3 – 2 ≤ 0;
C. x3 – 2y < 0;
D. x2 + 3 > 0.

\textbf{{ANSWER}}

Đáp án đúng là: A
Xét bất phương trình 2x – 4y + 7 ≥ 0 
Bất phương trình có hai ẩn x, y có lũy thừa bậc cao nhất là bậc một và các hệ số a = 2, b = –4, c = 7.
Do đó, đây là một bất phương trình bậc nhất hai ẩn.

========================================================================

https://khoahoc.vietjack.com/thi-online/10-bai-tap-xac-dinh-mien-snghiem-cua-bat-phuong-trinh-hai-an-co-loi-giai


\textbf{{QUESTION}}

Bất phương trình nào sau đây là bất phương trình bậc nhất hai ẩn ? 
A. 2x2 + 1 ≥ y + 2x2;
B. 2x – 6y + 5 < 2x – 6y + 3;
C. 4x2 < 2x + 5y – 6;
D. 2x3 + 1 ≥ y + 2x2.

\textbf{{ANSWER}}

Đáp án đúng là: A
Xét bất phương trình 2x2 + 1 ≥ y + 2x2 ≥ 0 ⇔ 2x2 + 1 – 2x2 – y ≥ 0 ⇔ –y + 1 ≥ 0 (1)
Bất phương trình (1) có hai ẩn x, y có lũy thừa bậc cao nhất là bậc một và các hệ số a = 0, b = –1, c = 1.
Do đó, đây là một bất phương trình bậc nhất hai ẩn.
Chú ý: Đáp án B không thỏa mãn vì ta biến đổi đưa về được 5 < 3 (vô lí). 
Đáp án C, D bậc của các ẩn không phải bậc nhất.

========================================================================

https://khoahoc.vietjack.com/thi-online/10-bai-tap-xac-dinh-mien-snghiem-cua-bat-phuong-trinh-hai-an-co-loi-giai


\textbf{{QUESTION}}

Bất phương trình nào sau đây là bất phương trình bậc nhất hai ẩn dạng ax + by + c < 0 ? 
A. 4x + 5y – 3 ≥ 0;
B. 4x – 2y + 3 < 0;
C. y – 3 ≥ 0;
D. 2x + 6 ≤ 0.

\textbf{{ANSWER}}

Đáp án đúng là: B
Xét bất phương trình 4x – 2y + 3 < 0
Bất phương trình có hai ẩn x, y có lũy thừa bậc cao nhất là bậc một và các hệ số a = 4, b = –2, c = 3.
Do đó, đây là một bất phương trình bậc nhất hai ẩn dạng ax + by + c < 0.

========================================================================

https://khoahoc.vietjack.com/thi-online/10-bai-tap-xac-dinh-mien-snghiem-cua-bat-phuong-trinh-hai-an-co-loi-giai


\textbf{{QUESTION}}

Bất phương trình nào sau đây là bất phương trình bậc nhất hai ẩn dạng ax + by + c > 0 ? 
A. x + 5y – 3 ≤ 0;
B. x – 2y + 3 ≥ 0;
C. x – 3 ≥ 0;
D. x + 6 > 0.

\textbf{{ANSWER}}

Đáp án đúng là: D
Xét bất phương trình x + 6 > 0
Bất phương trình có hai ẩn x, y có lũy thừa bậc cao nhất là bậc một và các hệ số a = 1, b = 0, c = 6.
Do đó, đây là một bất phương trình bậc nhất hai ẩn dạng ax + by + c > 0.

========================================================================

https://khoahoc.vietjack.com/thi-online/10-bai-tap-xac-dinh-mien-snghiem-cua-bat-phuong-trinh-hai-an-co-loi-giai


\textbf{{QUESTION}}

Bất phương trình nào sau đây là bất phương trình bậc nhất hai ẩn dạng ax + by + c < 0 ? 
A. 2x + 5y – 3 > 0;
B. x – 2y ≥ 2y;
C. x – 3 ≥ 3y – 1;
D. x + 6 < 4x + 2y.

\textbf{{ANSWER}}

Đáp án đúng là: D
Xét bất phương trình x + 6 < 4x + 2y ⇔ x + 6 – 4x – 2y < 0 ⇔ –3x – 2y + 6 < 0
Bất phương trình có hai ẩn x, y có lũy thừa bậc cao nhất là bậc một và các hệ số a = –3, b = –2, c = 6.
Do đó, đây là một bất phương trình bậc nhất hai ẩn dạng ax + by + c < 0.

========================================================================

https://khoahoc.vietjack.com/thi-online/bo-de-thi-thu-mon-toan-thpt-quoc-gia-nam-2022-co-loi-giai-30-de/75403


\textbf{{QUESTION}}

Cho cấp số cộng $$ \left({u}_{n}\right)$$ với $$ {u}_{1}=-3$$ và $$ {u}_{2}=3.$$ Công sai d của cấp số cộng đó bằng
A. -6
B. 0
C. 6
D. -9

\textbf{{ANSWER}}

Chọn C.
$$ d={u}_{2}-{u}_{1}=3-\left(-3\right)=6.$$

========================================================================

https://khoahoc.vietjack.com/thi-online/bo-de-thi-thu-mon-toan-thpt-quoc-gia-nam-2022-co-loi-giai-30-de/75403


\textbf{{QUESTION}}

Trong không gian Oxyz, hình chiếu vuông góc của điểm A(2; 3; 4) trên trục Oz có tọa độ là
A. (2; 0; 4)
B. (0; 3; 4)
C. (2; 3; 0)
D. (0; 0; 4)

\textbf{{ANSWER}}

Chọn D.
Tọa độ hình chiếu vuông góc của điểm A(2; 3; 4) trên trục Oz là (0; 0; 4).

========================================================================

https://khoahoc.vietjack.com/thi-online/bo-de-thi-thu-mon-toan-thpt-quoc-gia-nam-2022-co-loi-giai-30-de/75403


\textbf{{QUESTION}}

A. 8πa2.
B. 2πa2.
C. πa2.
D. 4πa2.

\textbf{{ANSWER}}

Chọn D.
Sxq=2πrl=2.π.2a.a=4πa2

========================================================================

https://khoahoc.vietjack.com/thi-online/bo-de-thi-thu-mon-toan-thpt-quoc-gia-nam-2022-co-loi-giai-30-de/75403


\textbf{{QUESTION}}

A. maxy1;2=32.
B. maxy1;2=0.
C. maxy1;2=2.
D. maxy1;2=52.

\textbf{{ANSWER}}

Chọn A.
Hàm số xác định với x∈1;2, khi đó ta có
y'=1+1x2>0,∀x∈1;2.
 
⇒ Hàm số luôn đồng biến trên [1; 2]
⇒max1;2y=y2=2−12=32.

========================================================================

https://khoahoc.vietjack.com/thi-online/bo-de-thi-thu-mon-toan-thpt-quoc-gia-nam-2022-co-loi-giai-30-de/75403


\textbf{{QUESTION}}

Số giao điểm của đồ thị hàm số y=x−1x2+x với trục Ox là:

\textbf{{ANSWER}}

Chọn B.
Số giao điểm của đồ thị hàm số y=x−1x2+x với trục Ox bằng số nghiệm của phương trình x−1x2+x=0⇔xx−1x+1=0

⇒x=1x=0x=−1.
Vậy số giao điểm là 3.

========================================================================

https://khoahoc.vietjack.com/thi-online/bai-tap-on-tap-toan-7-chuong-2-ham-so-va-do-thi-co-dap-an


\textbf{{QUESTION}}

Cho y tỉ lệ thuận với x theo hệ số tỉ lệ k = 2
a) Hãy biểu diễn y theo x
b) Hỏi x tỉ lệ thuận với y theo hệ số tỉ lệ nào?

\textbf{{ANSWER}}

a) Vì y tỉ lệ thuận với x theo hệ số tỉ lệ k = 2 nên y = 2x
b) Từ y = 2x suy ra $$ x\quad =\quad \frac{1}{2}y$$
Vậy x tỉ lệ thuận với y theo hệ số tỉ lệ bằng $$ \frac{1}{2}$$

========================================================================

https://khoahoc.vietjack.com/thi-online/10-bai-tap-tim-uoc-chung-lon-nhat-cua-hai-hay-nhieu-so-co-loi-giai


\textbf{{QUESTION}}

Tìm ƯCLN(18, 60)?
A. 6;
B. 30;
C. 12;
D. 18.

\textbf{{ANSWER}}

Đáp án đúng là: A
Phân tích 18; 60 ra thừa số nguyên tố ta được:
18 = 2.32
60 = 22.3.5
Ta thấy 2 và 3 là các thừa số nguyên tố chung của 18 và 60. Số mũ nhỏ nhất của 2 là 1, số mũ nhỏ nhất của 3 là 1 nên:
ƯCLN(18, 60) = 2.3 = 6.

========================================================================

https://khoahoc.vietjack.com/thi-online/10-bai-tap-tim-uoc-chung-lon-nhat-cua-hai-hay-nhieu-so-co-loi-giai


\textbf{{QUESTION}}

Gọi a là ƯCLN của 56 và 140, b là ƯCLN của 28 và 14. Giá trị a.b là:
A. 196;
B. 392;
C. 98;
D. 56.

\textbf{{ANSWER}}

Đáp án đúng là: B
Phân tích 56; 140 ra thừa số nguyên tố ta được:
56 = 23.7
140 = 22.5.7
Ta thấy 2 và 7 là các thừa số nguyên tố chung của 56 và 140. Số mũ nhỏ nhất của 2 là 2, số mũ nhỏ nhất của 7 là 1 nên:
ƯCLN(56, 140) = 22.7 = 28 nên a = 28
Vì 28 chia hết cho 14 nên ƯCLN(28, 14) = 14 nên b = 14
a.b = 28.14 = 392

========================================================================

https://khoahoc.vietjack.com/thi-online/10-bai-tap-tim-uoc-chung-lon-nhat-cua-hai-hay-nhieu-so-co-loi-giai


\textbf{{QUESTION}}

Tìm ƯCLN của 15; 45 và 225?
A. 18;
B. 3;
C. 15;
D. 5.

\textbf{{ANSWER}}

Đáp án đúng là C
Ta thấy: 45 chia hết cho 15; 225 chia hết cho 15 nên ƯCLN của 15; 45; 225 là 15.

========================================================================

https://khoahoc.vietjack.com/thi-online/10-bai-tap-tim-uoc-chung-lon-nhat-cua-hai-hay-nhieu-so-co-loi-giai


\textbf{{QUESTION}}

ƯCLN của a và b là:
A. Bằng b nếu a chia hết cho b;
B. Bằng a nếu a chia hết cho b;
C. Là ước chung nhỏ nhất của a và b;
D. Là hiệu của 2 số a và b.

\textbf{{ANSWER}}

Đáp án đúng là: A
Trong các số đã cho, nếu số nhỏ nhất là ước của các số còn lại thì ƯCLN của các số đã cho chính là số nhỏ nhất ấy.

========================================================================

https://khoahoc.vietjack.com/thi-online/10-bai-tap-tim-uoc-chung-lon-nhat-cua-hai-hay-nhieu-so-co-loi-giai


\textbf{{QUESTION}}

Cho a = 32.5.7 và b = 24.3.7. Tìm ƯCLN của a và b?
A. ƯCLN(a, b) = 8.7;     
B. ƯCLN(a, b) = 32.72;
C. ƯCLN(a, b) = 24.5;     
D. ƯCLN(a, b) = 24.32.5.7.

\textbf{{ANSWER}}

Đáp án đúng là: A
a = 32.5.7 = 25.5.7
b = 24.3.7 = 23.3.3.7 = 23.32.7
Ta thấy 2 và 7 là các thừa số nguyên tố chung của a và b. Số mũ nhỏ nhất của 2 là 3, số mũ nhỏ nhất của 7 là 1 nên:
ƯCLN(a, b) = 23.7 = 8.7

========================================================================

https://khoahoc.vietjack.com/thi-online/de%CC%80-kie%CC%89m-tra-chuong-1/58498


\textbf{{QUESTION}}

Thực hiện phép tính:
a, $$ 5.{2}^{2}-18:{3}^{2}$$
b, 17.85+15.17 – 120
c, $$ {2}^{3}.17-{2}^{3}.14$$
d, 20 – [30 – $$ {\left(5-1\right)}^{2}$$]

\textbf{{ANSWER}}

a, $$ 5.{2}^{2}-18:{3}^{2}$$
= 5.4 – 18:9
= 20 – 2 = 18
b, 17.85+15.17 – 120
= 1445+255 – 120
= 1580
c, $$ {2}^{3}.17-{2}^{3}.14$$
= $$ {2}^{3}.(17-14)$$ = 8.3 = 24
d, 20 – [30 – $$ {\left(5-1\right)}^{2}$$]
= 20 – (30 – 16) = 6

========================================================================

https://khoahoc.vietjack.com/thi-online/de%CC%80-kie%CC%89m-tra-chuong-1/58498


\textbf{{QUESTION}}

Thực hiện phép tính:
a, 75-(3.52-4.23)
b, 2.52+3:710-54:33
c, 150+50:5-2.32
d, 5.32-32:42

\textbf{{ANSWER}}

a, 75-(3.52-4.23)
= 75 – (3.25 – 4.8)
= 75 – 43 = 32
b, 2.52+3:710-54:33
= 50 + 3 – 2 = 51
c, 150+50:5-2.32
= 150 + 10 – 18 = 142
d, 5.32-32:42
= 5.9 – 32:16 = 45 – 2 = 43

========================================================================

https://khoahoc.vietjack.com/thi-online/de%CC%80-kie%CC%89m-tra-chuong-1/58498


\textbf{{QUESTION}}

Thực hiện phép tính:
a, 27.75  + 25.27 – 150
b, 142 – [50 – (23.10 – 23.5)]
c, 375:{32 – [4+(5.32 – 42]} – 14
d, {210:[16+3.(6+3.22)]} – 3

\textbf{{ANSWER}}

a, 27.75  + 25.27 – 150
= 2025 + 675 – 150 = 2550
b, 142 – [50 – (23.10 – 23.5)]
= 142 – [50 – (80 – 40)] = 132
c, 375:{32 – [4+(5.32 – 42]} – 14
= 375:{32 – [4+(45 – 42)]} – 14
= 375:(32 – 7) – 14 = 15 – 14 = 1
d, {210:[16+3.(6+3.22)]} – 3
= [210:(16+3.18)] – 3
= 210 : 70 – 3 = 3 – 3 = 0

========================================================================

https://khoahoc.vietjack.com/thi-online/de%CC%80-kie%CC%89m-tra-chuong-1/58498


\textbf{{QUESTION}}

Thực hiện phép tính:
a, 80-4.52-3.23
b, 56:54+23.22-12018
c, [36.4 – 4.82-7.112]:4 – 20190
d, 303 – 3.{[655 – (18:2+1).43+5]}:100

\textbf{{ANSWER}}

a, 80-4.52-3.23
= 80 – (4.25 – 3.8)
= 80 – 76 = 4
b, 56:54+23.22-12018
= 52+25-1
= 25 + 32 – 1 = 56
c, [36.4 – 4.82-7.112]:4 – 20190
= (144 – 4.52):4 – 1
= (144 – 100):4 – 1
= 11 – 1 = 10
d, 303 – 3.{[655 – (18:2+1).43+5]}:100
= 303 – 3.(655 – 10.64 + 5):1
= 303 – 10 = 293

========================================================================

https://khoahoc.vietjack.com/thi-online/de%CC%80-kie%CC%89m-tra-chuong-1/58498


\textbf{{QUESTION}}

Cho:
 a, A = 5.415.99-4.320.89 và B = 5.29.619-7.229.276. Tính A:B
b, C = 2181.729 + 243.81.27 và D = 32.92.243+18.243.324+723.729. Tính C:D

\textbf{{ANSWER}}

a, A:B = 229.318:228.318 = 2
b, C:D = 729.2910:729.2910 = 1

========================================================================

https://khoahoc.vietjack.com/thi-online/10-bai-tap-xac-dinh-cac-ket-qua-thuan-loi-cho-mot-bien-co-lien-quan-toi-hanh-dong-thuc-nghiem-co-loi


\textbf{{QUESTION}}

Bạn Mai thực nghiệm gieo một con xúc xắc 6 mặt. Số kết quả thuận lợi của biến cố “Số chấm xuất hiện trên con xúc xắc là số nguyên tố” là:

\textbf{{ANSWER}}

Đáp án đúng là: B
Trong các kết quả có thể là {1; 2; 3; 4; 5; 6} của thực nghiệm, các kết quả để biến cố “Số chấm xuất hiện trên con xúc xắc là số nguyên tố” xảy ra là {2; 3; 5}, nên có 3 kết quả thuận lợi cho biến cố.

========================================================================

https://khoahoc.vietjack.com/thi-online/10-bai-tap-xac-dinh-cac-ket-qua-thuan-loi-cho-mot-bien-co-lien-quan-toi-hanh-dong-thuc-nghiem-co-loi


\textbf{{QUESTION}}

Một hộp đựng 20 tấm thẻ ghi số 1; 2; 3; …; 20. Bạn Nga rút ngẫu nhiên một tấm thẻ trong hộp. Các kết quả thuận lợi cho biến cố “Rút được tấm thẻ ghi số chia hết cho 3” là:

\textbf{{ANSWER}}

Đáp án đúng là: A
Trong các kết quả có thể là {1; 2; 3; …; 20} của hành động, các kết quả để biến cố “Rút được tấm thẻ ghi số chia hết cho 3” xảy ra là {3; 6; 9; 12; 15; 18}, đây là các kết quả thuận lợi cho biến cố.

========================================================================

https://khoahoc.vietjack.com/thi-online/10-bai-tap-xac-dinh-cac-ket-qua-thuan-loi-cho-mot-bien-co-lien-quan-toi-hanh-dong-thuc-nghiem-co-loi


\textbf{{QUESTION}}

Bạn My có 6 cuốn sách, trong đó có 2 cuốn tiểu thuyết, 3 cuốn truyện ngắn và 1 cuốn truyện tranh. Các cuốn sách này được xếp tùy ý trên giá sách. Bạn Long đến chơi và lấy ngẫu nhiên một cuốn sách trên giá sách của My. Số kết quả thuận lợi của biến cố “Long lấy được một cuốn sách không phải tiểu thuyết” là:

\textbf{{ANSWER}}

Đáp án đúng là: D
Trong các kết quả có thể của hành động, các kết quả thuận lợi để biến cố “Long lấy được một cuốn sách không phải tiểu thuyết”  xảy ra là 3 cuốn truyện ngắn và 1 cuốn truyện tranh, nên có 4 kết quả thuận lợi cho biến cố.

========================================================================

https://khoahoc.vietjack.com/thi-online/10-bai-tap-xac-dinh-cac-ket-qua-thuan-loi-cho-mot-bien-co-lien-quan-toi-hanh-dong-thuc-nghiem-co-loi


\textbf{{QUESTION}}

Bạn My có 6 cuốn sách, trong đó có 2 cuốn tiểu thuyết, 3 cuốn truyện ngắn và 1 cuốn truyện tranh. Các cuốn sách này được xếp tùy ý trên giá sách. Bạn Long đến chơi và lấy ngẫu nhiên một cuốn sách trên giá sách của My. Số kết quả thuận lợi của biến cố “Long lấy được một cuốn sách tiểu thuyết hoặc truyện ngắn” là:

\textbf{{ANSWER}}

Đáp án đúng là: C
Trong các kết quả có thể của hành động, các kết quả để biến cố “Long lấy được một cuốn sách tiểu thuyết hoặc truyện ngắn” xảy ra là 2 cuốn tiểu thuyết và 3 cuốn truyện ngắn, nên có 5 kết quả thuận lợi cho biến cố.

========================================================================

https://khoahoc.vietjack.com/thi-online/10-bai-tap-xac-dinh-cac-ket-qua-thuan-loi-cho-mot-bien-co-lien-quan-toi-hanh-dong-thuc-nghiem-co-loi


\textbf{{QUESTION}}

Trong tủ quần áo của bạn Hải có 3 chiếc áo sơ mi trơn màu trắng, 2 chiếc áo sơ mi kẻ sọc và 2 chiếc áo sơ mi trơn màu xanh, tất cả các áo đều khác nhau. Hải chọn ngẫu nhiên một chiếc áo trong tủ để mặc. Số kết quả thuận lợi cho biến cố “Chiếc áo được chọn không phải áo kẻ sọc” là:

\textbf{{ANSWER}}

Đáp án đúng là: C
Trong các kết quả có thể của hành động, các kết quả để biến cố “Chiếc áo được chọn không phải áo kẻ sọc” xảy ra là 3 chiếc áo sơ mi trơn màu trắng, 2 chiếc áo sơ mi trơn màu xanh, nên có 5 kết quả thuận lợi cho biến cố.

========================================================================

https://khoahoc.vietjack.com/thi-online/giai-sbt-toan-11-canh-dieu-bai-2-hai-duong-thang-song-song-trong-khong-gian-co-dap-an


\textbf{{QUESTION}}

Hai đường thẳng chéo nhau khi và chỉ khi: 
A. Hai đường thẳng cùng nằm trong một mặt phẳng và không có điểm chung. 
B. Hai đường thẳng không có điểm chung. 
C. Hai đường thẳng không cùng nằm trong một mặt phẳng nào. 
D. Hai đường thẳng cùng chéo nhau với đường thẳng thứ ba.

\textbf{{ANSWER}}

Đáp án đúng là: C
Hai đường thẳng chéo nhau khi và chỉ khi chúng không đồng phẳng hay hai đường thẳng đó không cùng nằm trong một mặt phẳng nào.

========================================================================

https://khoahoc.vietjack.com/thi-online/sach-bai-tap-toan-7-tap-1/23468


\textbf{{QUESTION}}

Tính : (1/5)5.55

\textbf{{ANSWER}}

(1/5)5.55 = ((1/5).5)5 = 15= 1

========================================================================

https://khoahoc.vietjack.com/thi-online/sach-bai-tap-toan-7-tap-1/23468


\textbf{{QUESTION}}

Tính: (0,125)3.512

\textbf{{ANSWER}}

(0,125)3.512 = 0,1253.83 = (0,125.8)3 = 13 = 1

========================================================================

https://khoahoc.vietjack.com/thi-online/sach-bai-tap-toan-7-tap-1/23468


\textbf{{QUESTION}}

Tính: (0,25)4.1024

\textbf{{ANSWER}}

(0,25)4.1024 = (0,25)4.256.4 = (0,25)4.44.4 = (0,25.4)4.4 = 1.4 = 4

========================================================================

https://khoahoc.vietjack.com/thi-online/sach-bai-tap-toan-7-tap-1/23468


\textbf{{QUESTION}}

Tính: 1203403

\textbf{{ANSWER}}

1203403=120403=33=27

========================================================================

https://khoahoc.vietjack.com/thi-online/sach-bai-tap-toan-7-tap-1/23468


\textbf{{QUESTION}}

Tính: 39041304

\textbf{{ANSWER}}

39041304=3901304=34=81

========================================================================

https://khoahoc.vietjack.com/thi-online/24-cau-trac-nghiem-toan-9-bai-7-phuong-trinh-quy-ve-phuong-trinh-bac-hai-co-dap-an-phan-2


\textbf{{QUESTION}}

Phương trình $$ {x}^{4}\quad -\quad 6{x}^{2}\quad –\quad 7\quad =\quad 0$$ có bao nhiêu nghiệm?
A. 0
B. 1
C. 2
D. 4

\textbf{{ANSWER}}

Đặt $$ {x}^{2}$$ = t (t $$ \ge $$  0) ta được phương trình $$ {t}^{2}$$ – 6t – 7 = 0 (*)
Nhận thấy a – b + c = 1 + 6 – 7 = 0 nên phương trình (*) có hai nghiệm $$ {t}_{1}\quad =-1\quad \left(L\right);\quad {t}_{2}=7\quad \left(N\right)$$
Thay lại cách đặt ta có $$ {x}^{2}=7\Leftrightarrow x=\pm \sqrt{7}$$
Vậy phương trình đã cho có hai nghiệm
 Đáp án: C

========================================================================

https://khoahoc.vietjack.com/thi-online/15-cau-trac-nghiem-toan-7-chan-troi-sang-tao-bai-4-quy-tac-dau-ngoac-va-quy-tac-chuyen-ve-phan-2-co/103288


\textbf{{QUESTION}}

Một mảnh vườn hình chữ nhật có độ dài hai cạnh là 5,5 m và 3,5 m. Xung quanh các cạnh của mảnh vườn, người ta cắm các cọc gỗ, cứ 0,5 m cắm một cọc gỗ. Số lượng cọc cần sử dụng là:
A. 40
B. 38
C. 36
D. 34

\textbf{{ANSWER}}

Hướng dẫn giải
Đáp án đúng là: C
Chu vi mảnh vườn đó là: 
2.(5,5 + 3,5) = 18 (m)
Số lượng cọc cần sử dụng là: 
18 : 0,5 = 36 (cái)
Ta chọn phương án C.

========================================================================

https://khoahoc.vietjack.com/thi-online/15-cau-trac-nghiem-toan-7-chan-troi-sang-tao-bai-4-quy-tac-dau-ngoac-va-quy-tac-chuyen-ve-phan-2-co/103288


\textbf{{QUESTION}}

Giá trị của biểu thức −38.12+16.−38+13:−83$$ \frac{-3}{8}.\frac{1}{2}+\frac{1}{6}.\frac{-3}{8}+\frac{1}{3}:\frac{-8}{3}$$ là:

\textbf{{ANSWER}}

Hướng dẫn giải
Đáp án đúng là: B
Ta có: $$ \frac{-3}{8}.\frac{1}{2}+\frac{1}{6}.\frac{-3}{8}+\frac{1}{3}:\frac{-8}{3}$$
$$ =\frac{-3}{8}.\frac{1}{2}+\frac{-3}{8}.\frac{1}{6}+\frac{1}{3}.\frac{3}{-8}$$
$$ =\frac{-3}{8}.\frac{1}{2}+\frac{-3}{8}.\frac{1}{6}+\frac{-3}{8}.\frac{1}{3}$$
$$ =\frac{-3}{8}.\left(\frac{1}{2}+\frac{1}{6}+\frac{1}{3}\right)$$
$$ =\frac{-3}{8}.\left(\frac{3}{6}+\frac{1}{6}+\frac{2}{6}\right)$$
$$ =\frac{-3}{8}.\frac{6}{6}=\frac{-3}{8}.$$


Ta chọn phương án B.

========================================================================

https://khoahoc.vietjack.com/thi-online/15-cau-trac-nghiem-toan-7-chan-troi-sang-tao-bai-4-quy-tac-dau-ngoac-va-quy-tac-chuyen-ve-phan-2-co/103288


\textbf{{QUESTION}}

Giá niêm yết của một chiếc ti vi ở cửa hàng là 15 triệu đồng. Nhân dịp lễ, cửa hàng giảm giá 5% và khi thanh toán bằng thẻ khách hàng được giảm thêm 2%. Số tiền khách hàng phải trả khi thanh toán bằng thẻ là:

\textbf{{ANSWER}}

Hướng dẫn giải
Đáp án đúng là: A
Số tiền được giảm là:
15. 5% + 15. 2% = 15. 7% = 1,05 (triệu đồng)
Số tiền khách hàng phải thanh toán là: 15 – 1,05 = 13,95 (triệu đồng)
Ta chọn phương án A.

========================================================================

https://khoahoc.vietjack.com/thi-online/15-cau-trac-nghiem-phuong-trinh-duong-thang-co-dap-an-nhan-biet


\textbf{{QUESTION}}

Cho phương trình: ax + by + c = 0 (1) với a2 + b2 > 0. Mệnh đề nào sau đây sai?
A. (1) là phương trình tổng quát của đường thẳng có vectơ pháp tuyến là $$ \overrightarrow{n}=\left(a;b\right)$$
B. a = 0 thì (1) là phương trình đường thẳng song song hoặc trùng với trục Ox
C. b = 0 thì (1) là phương trình đường thẳng song song hoặc trùng với trục Oy
D. Điểm M0 (x0; y0) thuộc đường thẳng (1) khi và chỉ khi ax0 + by0 + c ≠0

\textbf{{ANSWER}}

+ Phương trình (1) là phương trình tổng quát của đường thẳng có vectơ pháp tuyến là   nên A đúng.
+ Nếu a = 0 thì by + c = 0 ⇔ y =$$ -\frac{c}{b}$$ nên nó là phương trình đường thẳng song song hoặc trùng với Ox (y = 0) nên B đúng.
+ Nếu b = 0 thì ax + c = 0 ⇔ x =$$ -\frac{c}{a}$$ nên nó là phương trình đường thẳng song song hoặc trùng với Oy (x = 0) nên C đúng.
+ Ta có điểm M0 (x0; y0) thuộc đường thẳng (1) khi và chỉ khi ax0 + by0 + c = 0 nên D sai.
Đáp án cần chọn là: D

========================================================================

https://khoahoc.vietjack.com/thi-online/15-cau-trac-nghiem-phuong-trinh-duong-thang-co-dap-an-nhan-biet


\textbf{{QUESTION}}

Đường thẳng (d) có vec tơ pháp tuyến →n=(a;b)$$ \overrightarrow{n}=\left(a;b\right)$$. Mệnh đề nào sau đây sai?
A. →u1=(-b;a)$$ \overrightarrow{{u}_{1}}=\left(-b;a\right)$$ là vec tơ chỉ phương của (d)
B. →u2=(-b;a)$$ \overrightarrow{{u}_{2}}=\left(-b;a\right)$$ là vec tơ chỉ phương của (d)
C.  →n'=(ka;kb); k∈R$$ \quad \overrightarrow{n\text{'}}=\left(ka;kb\right);\quad k\in R$$ 
D. (d) có hệ số góc k=-ba(b≠0)$$ k=\frac{-b}{a}\left(b\ne 0\right)$$

\textbf{{ANSWER}}

Phương trình tổng quát đường thẳng có vecto pháp tuyến $$ \overrightarrow{n}=\left(a;b\right)$$ là:
ax + by + c = 0 ⇔ $$ y=-\frac{a}{b}x-\frac{c}{b}$$ (b ≠ 0)
Suy ra hệ số góc $$ k=-\frac{a}{b}$$
Đáp án cần chọn là: D

========================================================================

https://khoahoc.vietjack.com/thi-online/15-cau-trac-nghiem-phuong-trinh-duong-thang-co-dap-an-nhan-biet


\textbf{{QUESTION}}

Mệnh đề nào sau đây sai? Đường thẳng (d) được xác định khi biết.
A. Một vecto pháp tuyến hoặc một vec tơ chỉ phương
B. Hệ số góc và một điểm thuộc đường thẳng
C. Một điểm thuộc (d) và biết (d) song song với một đường thẳng cho trước
D. Hai điểm phân biệt thuộc (d)

\textbf{{ANSWER}}

Nếu chỉ có vecto pháp tuyến hoặc một vecto chỉ phương thì thiếu điểm đi qua để viết phương trình đường thẳng.
Đáp án cần chọn là: A

========================================================================

https://khoahoc.vietjack.com/thi-online/15-cau-trac-nghiem-phuong-trinh-duong-thang-co-dap-an-nhan-biet


\textbf{{QUESTION}}

Tìm một vec tơ chỉ phương của đường thẳng d:{x=-1+2ty=3-5t
A. →u=(2;-5)
B. →u=(5;2)
C. →u=(-1;3)
D. →u=(-3;1)

\textbf{{ANSWER}}

Vec tơ chỉ phương của đường thẳng d là →u=(2;-5)$$ \overrightarrow{u}=\left(2;-5\right)$$
Đáp án cần chọn là: A

========================================================================

https://khoahoc.vietjack.com/thi-online/15-cau-trac-nghiem-phuong-trinh-duong-thang-co-dap-an-nhan-biet


\textbf{{QUESTION}}

Cho đường thẳng (d): 2x +3y - 4 = 0. Vec tơ nào sau đây là vec tơ pháp tuyến của (d)?
A. →n=(2;3)$$ \overrightarrow{n}=\left(2;3\right)$$
B. →n=(3;-2)$$ \overrightarrow{n}=\left(3;-2\right)$$
C. →n=(3;-2)$$ \overrightarrow{n}=\left(3;-2\right)$$
D. →n=(-3;-2)$$ \overrightarrow{n}=\left(-3;-2\right)$$

\textbf{{ANSWER}}

D: 2x + 3y – 4 = 0 có vec tơ pháp tuyến là: →n=(2;3)$$ \overrightarrow{n}=\left(2;3\right)$$
Đáp án cần chọn là: A

========================================================================

https://khoahoc.vietjack.com/thi-online/30-de-thi-thpt-quoc-gia-mon-toan-nam-2022-co-loi-giai/67666


\textbf{{QUESTION}}

Có bao nhiêu cách chọn ba học sinh từ một nhóm gồm 15 học sinh?
A. $$ {15}^{3}.$$
B. $$ {3}^{15}.$$
C. $$ {A}_{15}^{3}.$$
D. $$ {C}_{15}^{3}$$

\textbf{{ANSWER}}

Số cách chọn ba học sinh từ một nhóm gồm 15 học sinh là $$ {C}_{15}^{3}.$$
Chọn đáp án D.

========================================================================

https://khoahoc.vietjack.com/thi-online/bai-11-dau-hieu-chia-het-cho-2-cho-5


\textbf{{QUESTION}}

Trong các số sau, số nào chia hết cho 2, số nào không chia hết cho 2: 328; 1437; 895; 1234.

\textbf{{ANSWER}}

Số chia hết cho 2 là 328 và 1234 vì hai số này có tận cùng các chữ số chẵn
Số không chia hết cho 2 là 1437 và 895 vì hai số này có tận cùng là các chữ số lẻ

========================================================================

https://khoahoc.vietjack.com/thi-online/bai-11-dau-hieu-chia-het-cho-2-cho-5


\textbf{{QUESTION}}

Điền chữ số vào dấu * để được số (37*) chia hết cho 5.

\textbf{{ANSWER}}

Ta có thể điền chữ số 0 hoặc 5 vào dấu * để được số 370 và 375 là hai số chia hết cho 5 vì các số có chữ số tận cùng là 0 hoặc 5 thì chia hết cho 5

========================================================================

https://khoahoc.vietjack.com/thi-online/bai-11-dau-hieu-chia-het-cho-2-cho-5


\textbf{{QUESTION}}

Trong các số sau, số nào chia hết cho 2, số nào chia hết cho 5?
652; 850; 1546; 785; 6321

\textbf{{ANSWER}}

– 652 có chữ số tận cùng là 2 nên chia hết cho 2
– 850 có chữ số tận cùng là 0 nên chia hết cho 2 và 5
– 1546 có chữ số tận cùng là 6 nên chia hết cho 2
– 785 có chữ số tận cùng là 5 nên chia hết cho 5
– 6321 có chữ số tận cùng là 1 nên không chia hết cho 2 và 5

========================================================================

https://khoahoc.vietjack.com/thi-online/bai-11-dau-hieu-chia-het-cho-2-cho-5


\textbf{{QUESTION}}

Cho các số 2141; 1345; 4620; 234. Trong các số đó:
Số nào chia hết cho 2 mà không chia hết cho 5?

\textbf{{ANSWER}}

Số chia hết cho 2 mà không chia hết cho 5 là 234

========================================================================

https://khoahoc.vietjack.com/thi-online/bai-11-dau-hieu-chia-het-cho-2-cho-5


\textbf{{QUESTION}}

Cho các số 2141; 1345; 4620; 234. Trong các số đó:
Số nào chia hết cho 5 mà không chia hết cho 2?

\textbf{{ANSWER}}

Số chia hết cho 5 mà không chia hết cho 2 là 1345.

========================================================================

https://khoahoc.vietjack.com/thi-online/16-cau-trac-nghiem-toan-9-bai-4-duong-thang-song-song-va-duong-thang-cat-nhau-co-dap-an-phan-3


\textbf{{QUESTION}}

Viết phương trình đường thẳng d biết d cắt trục tung tại điểm có tung độ bằng −2 và cắt trục hoành tại điểm có hoành độ 1
A. y = 2x + 2 
B. y = −2x – 2 
C. y = 3x – 2 
D. y = 2x – 2

\textbf{{ANSWER}}

Gọi phương trình đường thẳng d cần tìm là y = ax + b (a khác 0)
Vì d cắt trục tung tại điểm có tung độ bằng −2 và cắt trục hoành tại điểm có hoành độ 1 nên d đi qua hai điểm A (0; 2); B (1; 0).
Thay tọa độ điểm A vào phương trình đường thẳng d ta được:
a.0 + b = −2 = > b = −2
Thay tọa độ điểm B và b = −2 vào phương trình đường thẳng d ta được
a.1 – 2 = 0 <= > a = 2
Vậy phương trình đường thẳng cần tìm là y = 2x − 2
Đáp án cần chọn là: D

========================================================================

https://khoahoc.vietjack.com/thi-online/16-cau-trac-nghiem-toan-9-bai-4-duong-thang-song-song-va-duong-thang-cat-nhau-co-dap-an-phan-3


\textbf{{QUESTION}}

Viết phương trình đường thẳng d biết d cắt trục tung tại điểm có tung độ bằng 3 và cắt trục hoành tại điểm có hoành độ −4
A. y=−34x+3$$ y=-\frac{3}{4}x+3$$
B. y=34x+3$$ y=\frac{3}{4}x+3$$
C. y=−34x−3$$ y=-\frac{3}{4}x-3$$
D. y=34x−3$$ y=\frac{3}{4}x-3$$

\textbf{{ANSWER}}

Gọi phương trình đường thẳng d cần tìm là y = ax + b (a khác 0)
Vì d cắt trục tung tại điểm có tung độ bằng 3 và cắt trục hoành tại điểm có hoành độ −4 nên d đi qua hai điểm A(0;3); B(−4;0).
Thay tọa độ điểm A vào phương trình đường thẳng d ta được:
$$ a.0\quad +\quad b\quad =\quad 3\Rightarrow b\quad =\quad 3$$
Thay tọa độ điểm B vào phương trình đường thẳng d ta được
$$ a.(-4)\quad +\quad 3\quad =\quad 0\quad \Rightarrow a=\frac{3}{4}$$
Vậy phương trình đường thẳng cần tìm là $$ y=\frac{3}{4}x+3$$
Đáp án cần chọn là: B

========================================================================

https://khoahoc.vietjack.com/thi-online/giai-sbt-toan-8-kntt-ham-so-bac-nhat-va-do-thi-cua-ham-so-bac-nhat-co-dap-an


\textbf{{QUESTION}}

Cho hàm số y = (1 – 2m)x + 3. 
Với những giá trị nào của m thì hàm số đã cho là hàm số bậc nhất ?

\textbf{{ANSWER}}

hàm số y = (1 – 2m)x + 3 là hàm số bậc nhất thì 1 – 2m ≠ 0 hay m ≠ $\frac{1}{2}$.

========================================================================

https://khoahoc.vietjack.com/thi-online/giai-sbt-toan-8-kntt-ham-so-bac-nhat-va-do-thi-cua-ham-so-bac-nhat-co-dap-an


\textbf{{QUESTION}}

Cho hàm số y = (1 – 2m)x + 3. 
Tìm m, biết đồ thị hàm số đã cho đi qua điểm (–1; 4).

\textbf{{ANSWER}}

Đồ thị hàm số đã cho đi qua điểm (–1; 4) nên ta có khi x = –1 thì y = 4.
Thay vào công thức hàm số ta có: 
4 = (1 – 2m).(–1) + 3 
4 = –1 + 2m + 3
2m = 2 
m = 1.
Vậy m = 1.

========================================================================

https://khoahoc.vietjack.com/thi-online/giai-sbt-toan-8-kntt-ham-so-bac-nhat-va-do-thi-cua-ham-so-bac-nhat-co-dap-an


\textbf{{QUESTION}}

Cho hàm số y = (1 – 2m)x + 3. 
Với giá trị m tìm được ở câu b, hãy hoàn thành bảng giá trị sau vào vở:
x
–2
–1
0
1
2
y
5
4
3
2
1

\textbf{{ANSWER}}

Với m = 1 ta có công thức hàm số y = –x + 3.
Ta có: 
Khi x = –2 thì y = –(–2) + 3 = 5; 
Khi x = –1 thì y = –(–1) + 3 = 4;
Khi x = 0 thì y = –0 + 3 = 3;
Khi x = 1 thì y = –1 + 3 = 2;
Khi x = 2 thì y = –2 + 3 = 1.
Do đó, ta có bảng dưới đây 
x
–2
–1
0
1
2
y
5
4
3
2
1

========================================================================

https://khoahoc.vietjack.com/thi-online/10-bai-tasp-doc-va-viet-cac-chu-so-bang-so-la-ma-co-loi-giai


\textbf{{QUESTION}}

Số La Mã XXIII đọc là
A. Mười ba;
B. Hai mươi ba;
C. Mười bảy;
D. Ba mươi hai.

\textbf{{ANSWER}}

Đáp án đúng là: B
Theo cách ghi số La Mã, ta có:
XXIII có năm thành phần là: X; X; I; I; I nên XXIII = 23 
Đọc là “hai mươi ba”

========================================================================

https://khoahoc.vietjack.com/thi-online/10-bai-tasp-doc-va-viet-cac-chu-so-bang-so-la-ma-co-loi-giai


\textbf{{QUESTION}}

Số Lã Mã XXIV đọc là
A. hai mươi tư;
B. hai mươi sáu;
C. mười sáu;
D. mười bốn.

\textbf{{ANSWER}}

Đáp án đúng là: A
Theo cách ghi số La Mã, ta có:
XXIV có ba thành phần là X; X; IV nên XXIV = 24 
Đọc là “hai mươi tư”

========================================================================

https://khoahoc.vietjack.com/thi-online/10-bai-tasp-doc-va-viet-cac-chu-so-bang-so-la-ma-co-loi-giai


\textbf{{QUESTION}}

Các số 18; 25 được viết thành số La Mã lần lượt là
A. XXVIII; XXV;
B. XVIII; XV;
C. XVIII; XVX;
D. XVIII; XXV.

\textbf{{ANSWER}}

Đáp án đúng là: D
Số 18, ta tách 18 = 10 + 5 + 3. Số 10 là X, số 5 là V, số 3 là III. Nên số 18 trong số La Mã viết là XVIII.
Số 25, ta tách 25 = 10 + 10 + 5. Số 10 là X, số 5 là V. Nên số 25 trong số La Mã viết là XXV.

========================================================================

https://khoahoc.vietjack.com/thi-online/10-bai-tasp-doc-va-viet-cac-chu-so-bang-so-la-ma-co-loi-giai


\textbf{{QUESTION}}

Các số 29; 13 được viết thành số La Mã lần lượt là
A. XXVIIII; XIV;
B. XXIX; XIII;
C. XXXI; XVIII;
D. XXIX; XIII.

\textbf{{ANSWER}}

Đáp án đúng là: D
Số 29, ta tách 29 = 10 + 10 + 9. Số 10 là X, số 9 là IX. Nên số 29 trong số La Mã viết là XXIX.
Số 13, ta tách 13 = 10 + 3. Số 10 là X, số 3 là III. Nên số 13 trong số La Mã viết là XIII.

========================================================================

https://khoahoc.vietjack.com/thi-online/10-bai-tasp-doc-va-viet-cac-chu-so-bang-so-la-ma-co-loi-giai


\textbf{{QUESTION}}

Số La Mã XXII tương ứng với số tự nhiên nào dưới đây?
A. 22;
B. 12;
C. 21;
D. 2.

\textbf{{ANSWER}}

Đáp án đúng là: A
Số La Mã XXII tương ứng với số 22.

========================================================================

https://khoahoc.vietjack.com/thi-online/15-cau-trac-nghiem-toan-7-ket-noi-tri-thuc-bai-3-luy-thua-voi-so-mu-tu-nhien-cua-1-so-huu-ti-co-dap/111112


\textbf{{QUESTION}}

Tính giá trị biểu thức A = $$ 1-\frac{3}{4}+{\left(\frac{3}{4}\right)}^{2}-{\left(\frac{3}{4}\right)}^{3}+{\left(\frac{3}{4}\right)}^{4}-\mathrm{...}-{\left(\frac{3}{4}\right)}^{2021}+{\left(\frac{3}{4}\right)}^{2022}$$
A. $$ \frac{4}{7}+\frac{3}{7}.{\left(\frac{3}{4}\right)}^{2022}$$
B. $$ \frac{4}{7}+\frac{3}{7}.{\left(\frac{3}{4}\right)}^{2023}$$
C. $$ \frac{4}{7}+{\left(\frac{3}{4}\right)}^{2022}$$
D. $$ \frac{4}{7}-\frac{3}{7}.{\left(\frac{3}{4}\right)}^{2022}$$

\textbf{{ANSWER}}

Hướng dẫn giải
Đáp án đúng là: A
A = $$ 1-\frac{3}{4}+{\left(\frac{3}{4}\right)}^{2}-{\left(\frac{3}{4}\right)}^{3}+{\left(\frac{3}{4}\right)}^{4}-\mathrm{...}-{\left(\frac{3}{4}\right)}^{2021}+{\left(\frac{3}{4}\right)}^{2022}$$
 $$ \frac{3}{4}\text{A =}$$$$ \frac{3}{4}-{\left(\frac{3}{4}\right)}^{2}+{\left(\frac{3}{4}\right)}^{3}-{\left(\frac{3}{4}\right)}^{4}+{\left(\frac{3}{4}\right)}^{5}-\mathrm{...}-{\left(\frac{3}{4}\right)}^{2022}+{\left(\frac{3}{4}\right)}^{2023}$$
$$ \frac{3}{4}\text{A + A =}\left[\frac{3}{4}-{\left(\frac{3}{4}\right)}^{2}+{\left(\frac{3}{4}\right)}^{3}-{\left(\frac{3}{4}\right)}^{4}+{\left(\frac{3}{4}\right)}^{5}-\mathrm{...}-{\left(\frac{3}{4}\right)}^{2022}+{\left(\frac{3}{4}\right)}^{2023}\right]$$+$$ \left[1-\frac{3}{4}+{\left(\frac{3}{4}\right)}^{2}-{\left(\frac{3}{4}\right)}^{3}+{\left(\frac{3}{4}\right)}^{4}-\mathrm{...}-{\left(\frac{3}{4}\right)}^{2021}+{\left(\frac{3}{4}\right)}^{2022}\right]$$
$$ \frac{7}{4}\text{A = }$$$$ 1+{\left(\frac{3}{4}\right)}^{2023}$$
 
A = $$ \left[1+{\left(\frac{3}{4}\right)}^{2023}\right]:\frac{7}{4}=\left[1+{\left(\frac{3}{4}\right)}^{2023}\right].\frac{4}{7}=\frac{4}{7}+\frac{4}{7}.\frac{3}{4}.{\left(\frac{3}{4}\right)}^{2022}=\frac{4}{7}+\frac{3}{7}.{\left(\frac{3}{4}\right)}^{2022}$$
Vậy đáp án đúng là A.

========================================================================

https://khoahoc.vietjack.com/thi-online/tong-hop-de-thi-thu-thpt-quoc-gia-mon-toan-cuc-hay-moi-nhat


\textbf{{QUESTION}}

Đường thẳng nào dưới đây là tiệm cận ngang của đồ thị hàm số $$ y=\frac{2x-3}{x+1}?$$
A. y = -2
B. y = -1
C. x = 2
D. y = 2

\textbf{{ANSWER}}

Đáp án D

========================================================================

https://khoahoc.vietjack.com/thi-online/giai-vth-toan-6-kntt-bai-tap-cuoi-chuong-7-co-dap-an


\textbf{{QUESTION}}

Tính giá trị của các biểu thức sau:
a) 15,3 - 21,5 – 3. 1,5; 
b) 2(42 – 2. 4,1) + 1,25: 5.

\textbf{{ANSWER}}

Lời giải:
a) 15,3 - 21,5 – 3. 1,5 = 15,3 – 21,5 – 4,5
= 15,3 – (21,5 + 4,5) = 15,3 – 26 = -10,7
b) 2(42 – 2. 4,1) + 1,25: 5 = 2(16 – 8,2) + 1,25:5 = 2.7,8 – 1,25:5 = 15,6 – 0,25 = 15,35

========================================================================

https://khoahoc.vietjack.com/thi-online/giai-vth-toan-6-kntt-bai-tap-cuoi-chuong-7-co-dap-an


\textbf{{QUESTION}}

Tìm x, biết:
a) x - 5,01 = 7,02 – 2.1,5;
b) x: 2,5 = 1,02 + 3.1,5.

\textbf{{ANSWER}}

Lời giải:
a) Ta có: 7,02 – 2.1,5 = 7,02 – 3 = 4,02 nên điều kiện có thể viết lại thành x – 5,01 = 4,02 do đó x = 5,01 + 4,02 = 9,03.
b) Tương tự, vì 1,02 + 3.1,5 = 1,02 + 4,5 = 5,52 nên điều kiện được viết lại thành x: 2,5 = 5,52. Do đó x = 5,52.2,5 = 13,8.

========================================================================

https://khoahoc.vietjack.com/thi-online/giai-vth-toan-6-kntt-bai-tap-cuoi-chuong-7-co-dap-an


\textbf{{QUESTION}}

Làm tròn số.
a) 127,459 đến hàng phần mười;
b) 152,025 đến hàng chục; 
c) 15 025 796 đến hàng nghìn.

\textbf{{ANSWER}}

Lời giải:
a) Làm tròn 127,459 đến hàng phần mười được kết quả 127,5.
b) Làm tròn 152,025 đến hàng chục được kết quả 150.
c) Làm tròn 15 025 796 đến hàng nghìn được kết quả 15 026 000.

========================================================================

https://khoahoc.vietjack.com/thi-online/giai-vth-toan-6-kntt-bai-tap-cuoi-chuong-7-co-dap-an


\textbf{{QUESTION}}

Năm 2002, Thumbelina được Tổ chức Kỉ lục Thế giới Guinness chính thức xác nhận là con ngựa thấp nhất thế giới với chiều cao khoảng 44,5 cm. Còn Big Jake trở nên nổi tiếng vào năm 2010 khi được Tổ chức Kỉ lục Thế giới Guinness trao danh hiệu là con ngựa cao nhất thế giới, nó cao gấp khoảng 4,72 lần con ngựa Thumbelina.
(Theo guinnessworidrecords.com)
Hỏi chiều cao của con Big Jake là bao nhiêu (làm tròn kết quả đến hàng đơn vị)?

\textbf{{ANSWER}}

Lời giải:
Con ngựa Big Jake cao khoảng 44,5.4,72 = 210,04 (cm) ≈ 210cm.

========================================================================

https://khoahoc.vietjack.com/thi-online/bai-tap-cuoi-chuong-5-co-dap-an-2


\textbf{{QUESTION}}

a) Có bao nhiêu cách xếp 20 học sinh theo một hàng dọc?
A. 2020.
B. 20!.
C. 20.
D. 1.
b) Số cách chọn ra 3 học sinh từ một lớp có 40 học sinh là:
A. $A_{40}^3$. 
B. 403. 
C. 340. 
D. $C_{40}^3$.

\textbf{{ANSWER}}

Hướng dẫn giải
a) Đáp án đúng là: B. 
Mỗi cách xếp 20 học sinh theo một hàng dọc là một hoán vị của 20 phần tử, do đó có 20! cách xếp 20 học sinh theo một hàng dọc. 
b) Đáp án đúng là: D. 
Mỗi cách chọn 3 học sinh từ 40 học sinh là một tổ hợp chập 3 của 40, do đó có $C_{40}^3$ cách chọn 3 học sinh từ 40 học sinh.

========================================================================

https://khoahoc.vietjack.com/thi-online/12-bai-tap-tinh-do-dai-dien-tich-goc-lien-quan-den-tinh-chat-hai-tiep-tuyen-cat-nhau-co-loi-giai


\textbf{{QUESTION}}

Cho hai tiếp tuyến tuyến của đường tròn cắt nhau tại một điểm. Chọn khẳng định sai trong các khẳng định dưới đây.
A. Khoảng cách từ điểm đó đến hai tiếp tuyến là bằng nhau.
B. Tia nối điểm đó tới tâm là tia phân giác của góc tạo bởi hai bán kính.
C. Tia nối từ tâm tới điểm đó là tia phân giác của góc tạo bởi hai bán kính.
D. Tia nối từ điểm đó tới tâm là tia phân giác của góc tạo bởi tiếp tuyến.

\textbf{{ANSWER}}

Đáp án đúng là: B
Nếu hai tiếp tuyến của đường tròn cắt nhau tại một điểm thì:
- Điểm đó cách đều hai tiếp điểm.
- Tia kẻ từ điểm đó đi qua tâm là tia phân giác của các góc tạo bởi hai tiếp tuyến.
- Tia kẻ từ tâm đi qua điểm đó là tia phân giác của góc tạo bởi hai bán kính đi qua tiếp điểm.

========================================================================

https://khoahoc.vietjack.com/thi-online/12-bai-tap-tinh-do-dai-dien-tich-goc-lien-quan-den-tinh-chat-hai-tiep-tuyen-cat-nhau-co-loi-giai


\textbf{{QUESTION}}

“Cho hai tiếp tuyến của một đường trong cắt nhau tại một điểm. Tia nối từ điểm đó tới tia phân giác của góc tạo bởi…… Tia nối từ tâm tới điểm đó là tia phân giác của góc tạo bởi …..”. Hai cụm từ thích hợp vào chỗ trống lần lượt là:
A. Hai tiếp tuyến, hai bán kính đi qua tiếp điểm.
B. Hai bán kính đi qua tiếp điểm, hai tiếp điểm.
C. Hai tiếp tuyến, hai dây cung.
D. Hai dây cung, hai bán kính.

\textbf{{ANSWER}}

Đáp án đúng là: A

========================================================================

https://khoahoc.vietjack.com/thi-online/10-bai-tap-bai-toan-thuc-tien-lien-quan-den-ham-so-luong-giac-co-loi-giai


\textbf{{QUESTION}}

Hằng ngày, mực nước của con kênh lên xuống theo thủy triều. Độ sâu h (mét) của mực nước trong kênh được tính tại thời điểm t (giờ) trong một ngày bởi công thức $$ h\left(t\right)=3\mathrm{cos}\left(\frac{\pi t}{8}+\frac{\pi }{4}\right)+12$$. Mực nước của con kênh cao nhất khi
A. t = 13 (giờ);
B. t = 14 (giờ);
C. t = 15 (giờ);

\textbf{{ANSWER}}

Hướng dẫn giải:
Đáp án đúng là: B
Mực nước của con kênh cao nhất khi h đạt giá trị lớn nhất
$$ \Leftrightarrow \mathrm{cos}\left(\frac{\pi t}{8}+\frac{\pi }{4}\right)=1\Leftrightarrow \frac{\pi t}{8}+\frac{\pi }{4}=k2\pi \text{\hspace{0.17em}\hspace{0.17em}}\left(k\in \mathbb{Z}\right)\Leftrightarrow t=16k-2\text{\hspace{0.17em}\hspace{0.17em}}\left(k\in \mathbb{Z}\right)$$ (1).
Mặt khác 0 ≤ t ≤ 24, kết hợp với (1) ta được: $$ \frac{1}{8}\le k\le \frac{13}{8},k\in \mathbb{Z}$$. Do đó k = 1.
Với k = 1 thì t = 14 (giờ).
Vậy mực nước của con kênh cao nhất khi t = 14 giờ.

========================================================================

https://khoahoc.vietjack.com/thi-online/bai-tap-bieu-dien-thap-phan-cua-so-huu-ti-co-dap-an


\textbf{{QUESTION}}

Viết các số hữu tỉ $$ \frac{1}{10}$$  và $$ \frac{1}{9}$$  dưới dạng số thập phân ta được: $$ \frac{1}{10}=0,1$$  và $$ \frac{1}{9}=0,\mathrm{111...}$$ .
Hai số thập phân 0,1 và 0,111… khác nhau như thế nào?
Biểu diễn thập phân của số hữu tỉ như thế nào?

\textbf{{ANSWER}}

Để trả lời được hai câu hỏi trên, chúng ta cùng tìm hiểu hai mục trong bài ở trang 27 và trang 28.

========================================================================

https://khoahoc.vietjack.com/thi-online/27-cau-trac-nghiem-toan-9-on-tap-chuong-iii-co-dap-an


\textbf{{QUESTION}}

Cặp số (x; y) = (1; 3) là nghiệm của hệ phương trình bậc nhất hai ẩn nào trong các hệ phương trình sau:
A. $$ \left\{\begin{array}{l}x-y=-2\\ x+y=4\end{array}\right.$$
B. $$ \left\{\begin{array}{l}2x-y=0\\ x+y=4\end{array}\right.$$
C. $$ \left\{\begin{array}{l}x+y=4\\ 2x+y=4\end{array}\right.$$
D. $$ \left\{\begin{array}{l}{x}^{2}+{y}^{2}=10\\ x-y=2\end{array}\right.$$

\textbf{{ANSWER}}

Hệ phương trình có chứa phương trình bậc hai là hệ phương trình ở đáp án D nên loại D
+ Với hệ phương trình A:
$$ \left\{\begin{array}{l}x-y=-2\\ x+y=4\end{array}\right.\Rightarrow \left\{\begin{array}{l}1-3=-2\\ 1+3-4\end{array}\right.\Leftrightarrow \left\{\begin{array}{l}-2=-2\\ 4=4\end{array}\right.$$(luôn đúng) nên (1; 3) là nghiệm của hệ phương trình $$ \left\{\begin{array}{l}x-y=-2\\ x+y=4\end{array}\right.$$
+ Với hệ phương trình B:  $$ \left\{\begin{array}{l}2x-y=0\\ x+y=4\end{array}\right.$$
Thay x = 1; y = 3 ta được $$ \left\{\begin{array}{l}2.1-3=0\\ 1+3=4\end{array}\right.\Leftrightarrow \left\{\begin{array}{l}-1=0\\ 1+3=4\end{array}\right.$$(vô lý) nên loại B.
+ Với hệ phương trình C: $$ \left\{\begin{array}{l}x+y=4\\ 2x+y=4\end{array}\right.$$
Thay x = 1; y = 3 ta được $$ \left\{\begin{array}{l}1+3=4\\ 2.1+3=4\end{array}\right.\Leftrightarrow \left\{\begin{array}{l}4=4\\ 5=4\end{array}\right.$$(vô lý) nên loại C.
Đáp án:A

========================================================================

https://khoahoc.vietjack.com/thi-online/27-cau-trac-nghiem-toan-9-on-tap-chuong-iii-co-dap-an


\textbf{{QUESTION}}

Với m = 1 thì hệ phương trình {x−y=m+1x+2y=2m+3$$ \left\{\begin{array}{l}x-y=m+1\\ x+2y=2m+3\end{array}\right.$$có cặp nghiệm (x; y) là:
A. (3; 1)
B. (1; 3)
C. (−1; −3)
D. (−3; −1)

\textbf{{ANSWER}}

Thay m = 1 vào hệ phương trình đã cho ta được: 
{x−y=2x+2y=5⇔{2x−2y=4x+2y=5⇔{3x=9x+2y=5⇔{x=3y=1$$ \left\{\begin{array}{l}x-y=2\\ x+2y=5\end{array}\right.\Leftrightarrow \left\{\begin{array}{l}2x-2y=4\\ x+2y=5\end{array}\right.\Leftrightarrow \left\{\begin{array}{l}3x=9\\ x+2y=5\end{array}\right.\Leftrightarrow \left\{\begin{array}{l}x=3\\ y=1\end{array}\right.$$
Đáp án:A

========================================================================

https://khoahoc.vietjack.com/thi-online/27-cau-trac-nghiem-toan-9-on-tap-chuong-iii-co-dap-an


\textbf{{QUESTION}}

Cặp số (x; y) là nghiệm của hệ phương trình {3x−4y=−22x+y=6$$ \left\{\begin{array}{l}3x-4y=-2\\ 2x+y=6\end{array}\right.$$là:
A. (−1; −2)
B. (2; 2)
C. (2; −1)
D. (3; 2)

\textbf{{ANSWER}}

{3x−4y=−22x+y=6⇔{3x−4y=−28x+4y=24⇔{3x−4y=−211x=22⇔{x=2y=3x+24⇔{x=2y=2$$ \left\{\begin{array}{l}3x-4y=-2\\ 2x+y=6\end{array}\right.\Leftrightarrow \left\{\begin{array}{l}3x-4y=-2\\ 8x+4y=24\end{array}\right.\phantom{\rule{0ex}{0ex}}\Leftrightarrow \left\{\begin{array}{l}3x-4y=-2\\ 11x=22\end{array}\right.\Leftrightarrow \left\{\begin{array}{l}x=2\\ y=\frac{3x+2}{4}\end{array}\right.\Leftrightarrow \left\{\begin{array}{l}x=2\\ y=2\end{array}\right.$$
Đáp án:B

========================================================================

https://khoahoc.vietjack.com/thi-online/27-cau-trac-nghiem-toan-9-on-tap-chuong-iii-co-dap-an


\textbf{{QUESTION}}

Với giá trị nào của m thì hệ phương trình {45x+12y=m+1x−y=2$$ \left\{\begin{array}{l}\frac{4}{5}x+\frac{1}{2}y=m+1\\ x-y=2\end{array}\right.$$nhận (3; 1) là nghiệm:
A. m=12$$ m=\frac{1}{2}$$
B. m=1910$$ m=\frac{19}{10}$$
C. m=310$$ m=\frac{3}{10}$$
D. Không có giá trị m

\textbf{{ANSWER}}

Nhận thấy {x=3y=1$$ \left\{\begin{array}{l}x=3\\ y=1\end{array}\right.$$thỏa mãn x – y = 2 nên ta thay {x=3y=1$$ \left\{\begin{array}{l}x=3\\ y=1\end{array}\right.$$vào phương trình 45x+12y=m+1$$ \frac{4}{5}x+\frac{1}{2}y=m+1$$ta được 125+12=m+1⇔m=1910$$ \frac{12}{5}+\frac{1}{2}=m+1\Leftrightarrow m=\frac{19}{10}$$
Đáp án:B

========================================================================

https://khoahoc.vietjack.com/thi-online/27-cau-trac-nghiem-toan-9-on-tap-chuong-iii-co-dap-an


\textbf{{QUESTION}}

Tìm cặp giá trị (a; b) để hai hệ phương trình sau tương đương (I)  và {ax−y=22ax+by=7$$ \left\{\begin{array}{l}ax-y=2\\ 2ax+by=7\end{array}\right.$$(II)
A. (−1; −1)
B. (1; 2)
C. (−1; 1)
D. (1; 1)

\textbf{{ANSWER}}

Giải hệ phương trình (I) ⇔{x=1+2y1+2y+y=4⇔{x=1+2y3y=3⇔{x=3y=1$$ \Leftrightarrow \left\{\begin{array}{l}x=1+2y\\ 1+2y+y=4\end{array}\right.\Leftrightarrow \left\{\begin{array}{l}x=1+2y\\ 3y=3\end{array}\right.\Leftrightarrow \left\{\begin{array}{l}x=3\\ y=1\end{array}\right.$$
Hai phương trình tương đương⇔$$ \Leftrightarrow $$hai phương trình có cùng tập nghiệm hay (3; 1) cũng là nghiệm của phương trình (II)
Thay {x=3y=1$$ \left\{\begin{array}{l}x=3\\ y=1\end{array}\right.$$vào hệ phương trình (II) ta được {3a−1=26a+b=7⇔{a=1b=1$$ \left\{\begin{array}{l}3a-1=2\\ 6a+b=7\end{array}\right.\Leftrightarrow \left\{\begin{array}{l}a=1\\ b=1\end{array}\right.$$
Đáp án:D

========================================================================

https://khoahoc.vietjack.com/thi-online/87-cau-chuyen-de-toan-12-bai-3-phuong-trinh-duong-thang-co-dap-an/116585


\textbf{{QUESTION}}

Trong không gian Oxyz, cho hai đường thẳng $$ \Delta :\frac{x+3}{1}=\frac{y-1}{1}=\frac{z+2}{4}$$  và mặt phẳng $$ \left(P\right):x+y-2z+6=0$$ . Biết $$ ∆$$ cắt mặt phẳng $$ \left(P\right)$$  tại $$ A,M$$  thuộc $$ ∆$$ sao cho $$ AM=2\sqrt{3}$$ . Tính khoảng cách từ M tới mặt phẳng (P).
A. $$ \sqrt{2}$$ .

\textbf{{ANSWER}}

Đường thẳng $$ \Delta :\frac{x+3}{1}=\frac{y-1}{1}=\frac{z+2}{4}$$  có vectơ chỉ phương $$ \overrightarrow{u}=\left(1;1;4\right)$$ .
Mặt phẳng $$ \left(P\right):x+y-2z+6=0$$  có vectơ chỉ phương $$ \overrightarrow{n}=\left(1;1;-2\right)$$ .
$$ \mathrm{sin}\left(\Delta ,\left(P\right)\right)=\left|\mathrm{cos}\left(\overrightarrow{u},\overrightarrow{n}\right)\right|=\frac{\left|\overrightarrow{u}.\overrightarrow{n}\right|}{\left|\overrightarrow{u}\right|.\left|\overrightarrow{n}\right|}=\sqrt{\frac{1}{3}}=\mathrm{sin}\phi $$
 
Suy ra $$ d\left(M,\Delta \right)=MH=MA.\mathrm{sin}\phi =2\sqrt{3}.\sqrt{\frac{1}{3}}=2$$ .
Chọn B.

========================================================================

https://khoahoc.vietjack.com/thi-online/87-cau-chuyen-de-toan-12-bai-3-phuong-trinh-duong-thang-co-dap-an/116585


\textbf{{QUESTION}}

Trong không gian Oxyz, cho đường thẳng d là giao tuyến của hai mặt phẳng (P):x−z.sinα+cosα=0; (Q):y−z.cosα−sinα=0; α∈(0;π2) .
Góc giữa d và trục Oz là:
A. 30° .

\textbf{{ANSWER}}

Mặt phẳng (P)$$ \left(P\right)$$  có vectơ pháp tuyến là →n(P)=(1;0;−sinα)$$ \overrightarrow{{n}_{\left(P\right)}}=\left(1;0;-\mathrm{sin}\alpha \right)$$ .
Mặt phẳng (Q)$$ \left(Q\right)$$  có vectơ pháp tuyến là →n(Q)=(0;1;−cosα)$$ \overrightarrow{{n}_{\left(Q\right)}}=\left(0;1;-\mathrm{cos}\alpha \right)$$ .
d là giao tuyến của (P) và (Q) nên vectơ chỉ phương của d là:
→u(d)=[→n(P),→n(Q)]=(sinα;cosα;1)$$ \overrightarrow{{u}_{\left(d\right)}}=\left[\overrightarrow{{n}_{\left(P\right)}},\overrightarrow{{n}_{\left(Q\right)}}\right]=\left(\mathrm{sin}\alpha ;\mathrm{cos}\alpha ;1\right)$$
Vectơ chỉ phương của (Oz) là →u(Oz)=(0;0;1)$$ \overrightarrow{{u}_{\left(Oz\right)}}=\left(0;0;1\right)$$ .
Suy ra cos(d,Oz)=|0.sinα+0.cosα+1.1|√sin2α+cos2α+12.√0+0+12=1√2⇒(d,Oz)=45°$$ \mathrm{cos}\left(d,Oz\right)=\frac{\left|0.\mathrm{sin}\alpha +0.\mathrm{cos}\alpha +1.1\right|}{\sqrt{{\mathrm{sin}}^{2}\alpha +{\mathrm{cos}}^{2}\alpha +{1}^{2}}.\sqrt{0+0+{1}^{2}}}=\frac{1}{\sqrt{2}}\Rightarrow \left(d,Oz\right)=45°$$ .
Vậy góc giữa d và trục (Oz)$$ \left(Oz\right)$$  là 45°$$ 45°$$ .
Chọn B.

========================================================================

https://khoahoc.vietjack.com/thi-online/87-cau-chuyen-de-toan-12-bai-3-phuong-trinh-duong-thang-co-dap-an/116585


\textbf{{QUESTION}}

Trong không gian Oxyz, d là đường thẳng đi qua điểm A(1,-1,2), song song với mặt phẳng (P):2x−y−z+3=0$$ \left(P\right):2x-y-z+3=0$$ , đồng thời tạo với đường thẳng Δ:x+11=y−1−2=z2$$ \Delta :\frac{x+1}{1}=\frac{y-1}{-2}=\frac{z}{2}$$  một góc lớn nhất. Phương trình đường thẳng d là
A. x−1−4=y+15=z−23$$ \frac{x-1}{-4}=\frac{y+1}{5}=\frac{z-2}{3}$$ .

\textbf{{ANSWER}}

Mặt phẳng (P):2x−y−z+3=0$$ \left(P\right):2x-y-z+3=0$$  có một vectơ pháp tuyến là →n(P)=(2;−1;−1)$$ {\overrightarrow{n}}_{{}_{\left(P\right)}}=\left(2;-1;-1\right)$$ .
Đường thẳng Δ:x+11=y−1−2=z2$$ \Delta :\frac{x+1}{1}=\frac{y-1}{-2}=\frac{z}{2}$$  có một vectơ chỉ phương là →uΔ=(1;−2;2)$$ {\overrightarrow{u}}_{{}_{\Delta }}=\left(1;-2;2\right)$$ .
Giả sử đường thẳng d có vectơ chỉ phương là →ud$$ {\overrightarrow{u}}_{{}_{d}}$$ .
Do 0°≤(d,Δ)≤90°$$ 0°\le \left(d,\Delta \right)\le 90°$$  mà theo giả thiết d tạo ∆$$ ∆$$ góc lớn nhất nên (d,Δ)=90°⇒→ud⊥→uΔ$$ \left(d,\Delta \right)=90°\Rightarrow {\overrightarrow{u}}_{d}\bot {\overrightarrow{u}}_{\Delta }$$ .
Lại có d//(P)$$ d\text{//}\left(P\right)$$  nên →ud⊥→n(P)$$ {\overrightarrow{u}}_{d}\bot {\overrightarrow{n}}_{\left(P\right)}$$ . Do đó chọn →ud=[→uΔ,→n(P)]=(4;5;3)$$ {\overrightarrow{u}}_{d}=\left[{\overrightarrow{u}}_{\Delta },{\overrightarrow{n}}_{\left(P\right)}\right]=\left(4;5;3\right)$$ .
Vậy phương trình đường thẳng d là x−14=y+15=z−23$$ \frac{x-1}{4}=\frac{y+1}{5}=\frac{z-2}{3}$$ .
Chọn D.

========================================================================

https://khoahoc.vietjack.com/thi-online/10-bai-tap-tism-dieu-kien-cua-tham-so-m-de-ham-so-lien-tuc-co-loi-giai


\textbf{{QUESTION}}

Giá trị của m để hàm số  $$ f\left(x\right)=\left\{\begin{array}{c}x-m\text{   khi }x\ne 5\\ 3\text{          khi }x=5\end{array}\right.$$ liên tục tại x = 5 là

\textbf{{ANSWER}}

Đáp án đúng là: B
Ta có f(5) = 3
Để hàm số liên tục tại x = 5 thì
 $$ \underset{x\to 5}{\mathrm{lim}}f\left(x\right)=\underset{x\to 5}{\mathrm{lim}}\left(x-m\right)=5-m=f\left(5\right)=3$$ hay m = 2.
Vậy hàm số liên tục tại x = 5 tại m = 2.

========================================================================

https://khoahoc.vietjack.com/thi-online/10-bai-tap-tism-dieu-kien-cua-tham-so-m-de-ham-so-lien-tuc-co-loi-giai


\textbf{{QUESTION}}

Giá trị m để hàm số  f(x)={2x−7   khi x≠4x−m    khi x=4$$ f\left(x\right)=\left\{\begin{array}{c}2x-7\text{   khi }x\ne 4\\ x-m\text{    khi }x=4\end{array}\right.$$ liên tục tại x = 4 là

\textbf{{ANSWER}}

Đáp án đúng là: C
Ta có f(4) = 4 – m.
Để hàm số liên tục tại x = 4 thì
 limx→4f(x)=limx→4(2x−7)=1=f(4)=4−m$$ \underset{x\to 4}{\mathrm{lim}}f\left(x\right)=\underset{x\to 4}{\mathrm{lim}}\left(2x-7\right)=1=f\left(4\right)=4-m$$ hay m = 3.
Vậy hàm số liên tục tại x = 4 tại m = 3.

========================================================================

https://khoahoc.vietjack.com/thi-online/10-bai-tap-tism-dieu-kien-cua-tham-so-m-de-ham-so-lien-tuc-co-loi-giai


\textbf{{QUESTION}}

Giá trị của m để hàm số  f(x)={x2−1x−1   khi x>1mx−4 khi x≤1$$ f\left(x\right)=\left\{\begin{array}{c}\frac{{x}^{2}-1}{x-1}\text{   khi x}>1\\ mx-4\text{ khi }x\le 1\end{array}\right.$$ liên tục tại x = 1 là

\textbf{{ANSWER}}

Đáp án đúng là: A
•  limx→1+f(x)=limx→1+x2−1x−1=limx→1+(x−1)(x+1)x−1=limx→1+(x+1)=2$$ \underset{x\to {1}^{+}}{\mathrm{lim}}f\left(x\right)=\underset{x\to {1}^{+}}{\mathrm{lim}}\frac{{x}^{2}-1}{x-1}=\underset{x\to {1}^{+}}{\mathrm{lim}}\frac{\left(x-1\right)\left(x+1\right)}{x-1}=\underset{x\to {1}^{+}}{\mathrm{lim}}\left(x+1\right)=2$$.
•  limx→1−f(x)=limx→1−(mx−4)=m−4$$ \underset{x\to {1}^{-}}{\mathrm{lim}}f\left(x\right)=\underset{x\to {1}^{-}}{\mathrm{lim}}\left(mx-4\right)=m-4$$.
Để hàm số liên tục tại x = 1 thì  limx→1+f(x)=limx→1−f(x)⇔$$ \underset{x\to {1}^{+}}{\mathrm{lim}}f\left(x\right)\underset{x\to {1}^{-}}{=\mathrm{lim}}f\left(x\right)\Leftrightarrow $$m – 4 = 2 hay m = 6.
Vậy hàm số liên tục tại x = 1 khi m = 6.

========================================================================

https://khoahoc.vietjack.com/thi-online/10-bai-tap-tism-dieu-kien-cua-tham-so-m-de-ham-so-lien-tuc-co-loi-giai


\textbf{{QUESTION}}

Cho hàm số  f(x)={x2−xx2+2x−3 khi x≠1mm+2         khi x=1$$ f\left(x\right)=\left\{\begin{array}{c}\frac{{x}^{2}-x}{{x}^{2}+2x-3}\text{ khi x}\ne 1\\ \frac{m}{m+2}\text{         khi }x=1\end{array}\right.$$. Giá trị của m để liên tục tại x = 1 là

\textbf{{ANSWER}}

Đáp án đúng là: C

========================================================================

https://khoahoc.vietjack.com/thi-online/10-bai-tap-tism-dieu-kien-cua-tham-so-m-de-ham-so-lien-tuc-co-loi-giai


\textbf{{QUESTION}}

Để hàm số   f(x)={x2−m  khi x≠113          khi x=1$$ f\left(x\right)=\left\{\begin{array}{c}{x}^{2}-m\text{  khi x}\ne 1\\ \frac{1}{3}\text{          khi }x=1\end{array}\right.$$ liên tục tại x = 1 thì giá trị của m bằng
A. 12$$ \frac{1}{2}$$
B. 23$$ \frac{2}{3}$$
C. 34$$ \frac{3}{4}$$
D. 45$$ \frac{4}{5}$$

\textbf{{ANSWER}}

Đáp án đúng là: B
Ta có  f(1)=13$$ f\left(1\right)=\frac{1}{3}$$.
Để hàm số liên tục tại x = 1 thì
 limx→1f(x)=x2−m=1−m=f(1)=13$$ \underset{x\to 1}{\mathrm{lim}}f\left(x\right)={x}^{2}-m=1-m=f\left(1\right)=\frac{1}{3}$$ hay  m=23$$ m=\frac{2}{3}$$.
Vậy hàm số liên tục tại x = 1 khi  m=23$$ m=\frac{2}{3}$$.

========================================================================

https://khoahoc.vietjack.com/thi-online/giai-sbt-toan-10-bai-2-xac-suat-cua-bien-co-co-dap-an


\textbf{{QUESTION}}

Gieo một con xúc xắc 4 mặt cân đối và đồng chất ba lần. Tính xác suất của các biến cố:
a) “Tổng các số xuất hiện ở đỉnh phía trên của con xúc xắc trong ba lần gieo lớn hơn 2”;

\textbf{{ANSWER}}

a) Biến cố “Tổng các số xuất hiện ở đỉnh phía trên của con xúc xắc trong ba lần gieo lớn hơn 2” đây là biến cố chắc chắn nên ta có P(A) = 1.

========================================================================

https://khoahoc.vietjack.com/thi-online/giai-sbt-toan-10-bai-2-xac-suat-cua-bien-co-co-dap-an


\textbf{{QUESTION}}

b) “Có đúng một lần số xuất hiện ở đỉnh phía trên của con xúc xắc là 2”.

\textbf{{ANSWER}}

b) Số phần tử của không gian mẫu n(Ω) = 43 = 64
Gọi B là biến cố: “Có đúng một lần số xuất hiện ở đỉnh phía trên của con xúc xắc là 2”
Trường hợp 1. Lần thứ nhất số xuất hiện ở đỉnh phía trên của con xúc xắc là 2.
Vì có đúng một lần số xuất hiện ở đỉnh phía trên của con xúc xắc là 2 nên số xuất hiện ở hai lần sau phải khác 2 nên mỗi lần có 3 kết quả xảy ra.
Ta có 1.32 = 9 kết quả thuận lợi.
Trường hợp 2. Lần thứ hai số xuất hiện ở đỉnh phía trên của con xúc xắc là 2
Tương tự trường hợp 1 có 1.32 = 9 kết quả thuận lợi
Trường hợp 3. Lần thứ ba số xuất hiện ở đỉnh phía trên của con xúc xắc là 2
Tương tự trường hợp 1 có 1.32 = 9 kết quả thuận lợi
Số phần tử của biến cố B là: n(B) = 9 + 9 + 9 = 27
Xác suất của biến cố B là: P(B) = 2764$$ \frac{27}{64}$$

========================================================================

https://khoahoc.vietjack.com/thi-online/giai-sbt-toan-10-bai-2-xac-suat-cua-bien-co-co-dap-an


\textbf{{QUESTION}}

Tung một đồng xu cân đối và đồng chất bốn lần. Tính xác suất của các biến cố:
a) “Cả bốn lần đều xuất hiện mặt giống nhau”;

\textbf{{ANSWER}}

Số phần tử của không gian mẫu n(Ω) = 24 = 16.
a) Gọi A là biến cố: “Cả bốn lần đều xuất hiện mặt giống nhau”.
A = {SSSS; NNNN}.
Số phần tử của biến cố A là: n(A) = 2.
Xác suất của biến cố A là: P(A) = 216=18$$ \frac{2}{16}=\frac{1}{8}$$.

========================================================================

https://khoahoc.vietjack.com/thi-online/giai-sbt-toan-10-bai-2-xac-suat-cua-bien-co-co-dap-an


\textbf{{QUESTION}}

b) “Có đúng một lần xuất hiện mặt sấp, ba lần xuất hiện mặt ngửa”.

\textbf{{ANSWER}}

b) Gọi B là biến cố: “Có đúng một lần xuất hiện mặt sấp, ba lần xuất hiện mặt ngửa”.
B = {SNNN; NSNN; NNSN; NNNS}.
Số phần tử của biến cố B là: n(B) = 4.
Xác suất của biến cố B là: P(B) = 416=14$$ \frac{4}{16}=\frac{1}{4}$$.

========================================================================

https://khoahoc.vietjack.com/thi-online/toan-9-tap-1-phan-dai-so


\textbf{{QUESTION}}

Tìm các căn bậc hai của mỗi số sau: 9

\textbf{{ANSWER}}

Căn bậc hai của 9 là 3 và -3 (vì 32 = 9 và (-3)2 = 9)

========================================================================

https://khoahoc.vietjack.com/thi-online/toan-9-tap-1-phan-dai-so


\textbf{{QUESTION}}

Tìm các căn bậc hai của mỗi số sau: 49$$ \frac{4}{9}$$

\textbf{{ANSWER}}

Ta có: (23)2=49$$ {\left(\frac{2}{3}\right)}^{2}=\frac{4}{9}$$  và (−23)2=49$$ {\left(-\frac{2}{3}\right)}^{2}=\frac{4}{9}$$
Do đó căn bậc hai của 49$$ \frac{4}{9}$$  và  23$$ \frac{2}{3}$$ và -23$$ \frac{-2}{3}$$

========================================================================

https://khoahoc.vietjack.com/thi-online/toan-9-tap-1-phan-dai-so


\textbf{{QUESTION}}

Tìm các căn bậc hai của mỗi số sau: 0,25

\textbf{{ANSWER}}

Căn bậc hai của 0,25 là 0,5 và -0,5 (vì 0,52 = 0,25 và (-0,5)2 = 0,25)

========================================================================

https://khoahoc.vietjack.com/thi-online/toan-9-tap-1-phan-dai-so


\textbf{{QUESTION}}

Tìm các căn bậc hai của mỗi số sau: 2

\textbf{{ANSWER}}

Căn bậc hai của 2 là √2 và -√2 (vì (√2)2 = 2 và(-√2)2 = 2 )

========================================================================

https://khoahoc.vietjack.com/thi-online/toan-9-tap-1-phan-dai-so


\textbf{{QUESTION}}

Tìm căn bậc hai số học của mỗi số sau: 49

\textbf{{ANSWER}}

√49 = 7, vì 7 > 0 và 72 = 49

========================================================================

https://khoahoc.vietjack.com/thi-online/giai-sbt-toan-9-chan-troi-sang-tao-bai-1-khong-gian-mau-va-bien-co-co-dap-an


\textbf{{QUESTION}}

Một hộp chứa 3 quả bóng bàn và 2 quả bóng gôn. Trong các hoạt động sau, hoạt động nào là phép thử ngẫu nhiên?
a) Chọn ra đồng thời 5 quả bóng từ hộp.
b) Chọn ra lần lượt 5 quả bóng từ hộp, bóng lấy ra không được trả lại hộp.
c) Chọn ra đồng thời 2 quả bóng gôn từ hộp.
d) Chọn ra đồng thời 2 quả bóng bàn từ hộp.

\textbf{{ANSWER}}

⦁ Hoạt động a) không là phép thử ngẫu nhiên vì hoạt động này chỉ có một kết quả có thể xảy ra, đó là lấy được đồng thời 3 quả bóng bàn và 2 quả bóng gôn.
⦁ Hoạt động b) có nhiều kết quả có thể xảy ra, chẳng hạn 5 quả bóng lần lượt được lấy ra là: bóng bàn, bóng gôn, bóng bàn, bóng gôn, bóng bàn; bóng bàn, bóng bàn, bóng gôn, bóng bàn, bóng gôn; …
Do đó ta không thể biết trước được kết quả của nó, nhưng ta có thể biết tất cả các kết quả có thể xảy ra của nó. Vì vậy, hoạt động b) là phép thử ngẫu nhiên.
⦁ Hoạt động c) không là phép thử ngẫu nhiên vì hoạt động này chỉ có một kết quả có thể xảy ra, do trong hộp chỉ có 2 quả bóng gôn.
⦁ Hoạt động d) có nhiều kết quả có thể xảy ra đối với 2 quả bóng bàn đồng thời được lấy ra là: quả bóng bàn A và quả bóng bàn B; quả bóng bàn A và quả bóng bàn C; quả bóng bàn B và quả bóng bàn C.
Do đó ta không thể biết trước được kết quả của nó, nhưng ta có thể biết tất cả các kết quả có thể xảy ra của nó. Vì vậy, hoạt động d) là phép thử ngẫu nhiên.

========================================================================

https://khoahoc.vietjack.com/thi-online/giai-sbt-toan-9-chan-troi-sang-tao-bai-1-khong-gian-mau-va-bien-co-co-dap-an


\textbf{{QUESTION}}

Một hộp đựng 4 tấm thẻ ghi các số 5; 6; 8; 9. Lấy ngẫu nhiên lần lượt 2 tấm thẻ từ hộp. Tấm thẻ lấy ra lần đầu không được trả lại hộp.
a) Xác định không gian mẫu của phép thử. Không gian mẫu của phép thử có bao nhiêu phần tử?
b) Liệt kê các kết quả thuận lợi cho biến cố A: “Tích các số ghi trên hai tấm thẻ là số lẻ”.

\textbf{{ANSWER}}

a) Kí hiệu (i; j) là kết quả thẻ lấy ra lần đầu ghi số i và thẻ lấy ra lần sau ghi số j.
Không gian mẫu của phép thử là
Ω = {(5; 6); (5; 8); (5; 9); (6; 5); (6; 8); (6; 9); (8; 5); (8; 6); (8; 9); (9; 5); (9; 6); (9; 8)}.
Không gian mẫu của phép thử có 12 phần tử.
b) Các kết quả thuận lợi cho biến cố A: “Tích các số ghi trên hai tấm thẻ là số lẻ” là: (5; 9) và (9; 5). Có 2 kết quả thuận lợi cho biến cố A.

========================================================================

https://khoahoc.vietjack.com/thi-online/giai-sbt-toan-9-chan-troi-sang-tao-bai-1-khong-gian-mau-va-bien-co-co-dap-an


\textbf{{QUESTION}}

Một hộp chứa 2 cây bút xanh và 1 cây bút tím.
a) Liệt kê các phần tử của không gian mẫu của phép thử chọn ngẫu nhiên đồng thời 2 cây bút từ hộp.
b) Liệt kê các phần tử của không gian mẫu của phép thử chọn ngẫu nhiên lần lượt 2 cây bút từ hộp, cây bút lấy ra lần thứ nhất không được trả lại hộp trước khi lấy cây bút thứ hai.
c) Liệt kê các phần tử của không gian mẫu của phép thử chọn ngẫu nhiên lần lượt 2 cây bút từ hộp, cây bút lấy ra lần thứ nhất được trả lại hộp trước khi lấy cây bút thứ hai.

\textbf{{ANSWER}}

Kí hiệu hai cây bút xanh là X1, X2 và cây bút tím là T.
a) Các phần tử của không gian mẫu của phép thử chọn ngẫu nhiên đồng thời 2 cây bút từ hộp là: (X1; X2); (X1; T); (X2; T).
b) Các phần tử của không gian mẫu của phép thử chọn ngẫu nhiên lần lượt 2 cây bút từ hộp, cây bút lấy ra lần thứ nhất không được trả lại hộp trước khi lấy cây bút thứ hai là: (X1; X2); (X1; T); (X2; X1); (X2; T); (T; X1); (T; X2).
c) Các phần tử của không gian mẫu của phép thử chọn ngẫu nhiên lần lượt 2 cây bút từ hộp, cây bút lấy ra lần thứ nhất được trả lại hộp trước khi lấy cây bút thứ hai là (X1; X1); (X1; X2); (X1; T); (X2; X1); (X2; X2); (X2; T); (T; X1); (T; X2); (T; T).

========================================================================

https://khoahoc.vietjack.com/thi-online/giai-sbt-toan-9-chan-troi-sang-tao-bai-1-khong-gian-mau-va-bien-co-co-dap-an


\textbf{{QUESTION}}

Hộp thứ nhất chứa 2 tấm thẻ cùng loại được đánh số 1; 2. Hộp thứ hai chứa 3 tấm thẻ cùng loại được đánh số 3; 4; 5. Bạn Hà lấy ngẫu nhiên 1 tấm thẻ từ hộp thứ nhất và 1 tấm thẻ từ hộp thứ hai.
a) Hãy xác định không gian mẫu của phép thử. Không gian mẫu của phép thử có bao nhiêu phần tử?
b) Liệt kê các kết quả thuận lợi cho biến cố A: “Các số trên hai thẻ lấy ra đều là số lẻ”. Có bao nhiêu kết quả thuận lợi cho biến cố A?

\textbf{{ANSWER}}

a) Kí hiệu (i; j) là kết quả thẻ lấy ra từ hộp thứ nhất được đánh số i, thẻ lấy ra từ hộp thứ hai được đánh số j.
Không gian mẫu của phép thử là Ω = {(1; 3); (1; 4); (1; 5); (2; 3); (2; 4); (2; 5)}.
Không gian mẫu của phép thử có 6 phần tử.
b) Các kết quả thuận lợi cho biến cố A là (1; 3) và (1; 5).
Có 2 kết quả thuận lợi cho biến cố A.

========================================================================

https://khoahoc.vietjack.com/thi-online/giai-sbt-toan-9-chan-troi-sang-tao-bai-1-khong-gian-mau-va-bien-co-co-dap-an


\textbf{{QUESTION}}

Một nhóm học sinh gồm 2 bạn lớp 9A là Đăng, Phước và 3 bạn lớp 9B là Dung, Thọ và Thuý. Thầy giáo chọn ngẫu nhiên 1 học sinh lớp 9A và 1 học sinh lớp 9B từ nhóm trên.
a) Hãy xác định không gian mẫu của phép thử. Không gian mẫu của phép thử có bao nhiêu phần tử?
b) Liệt kê các kết quả thuận lợi cho biến cố A: “Tên của hai bạn được chọn đều có chữ cái n”. Có bao nhiêu kết quả thuận lợi cho biến cố A?

\textbf{{ANSWER}}

a) Không gian mẫu của phép thử gồm các kết quả là: Đăng và Dung; Đăng và Thọ; Đăng và Thuý; Phước và Dung; Phước và Thọ; Phước và Thuý.
Không gian mẫu của phép thử có 6 phần tử.
b) Kết quả thuận lợi cho biến cố A là: Đăng và Dung.
Có đúng 1 kết quả thuận lợi cho biến cố A.

========================================================================

https://khoahoc.vietjack.com/thi-online/giai-vth-toan-8-kntt-bai-30-ket-qua-co-the-va-ket-qua-thuan-loi-co-dap-an


\textbf{{QUESTION}}

Chọn phương án đúng
Rút ngẫu nhiên một tấm thẻ từ một hộp chứa các tấm thẻ ghi số 1; 2; …; 30. Số kết quả thuận lợi cho biến cố “Rút được tấm thẻ ghi số chia hết cho 5” là
A. 5.
B. 6.
C. 7.
D. 4.

\textbf{{ANSWER}}

Đáp án đúng là: B
Các tấm thẻ ghi số chia hết cho 5 là: 5; 10; 15; 20; 25; 30.
Vậy có 6 kết quả thuận lợi cho biến cố “Rút được tấm thẻ ghi số chia hết cho 5”.

========================================================================

https://khoahoc.vietjack.com/thi-online/giai-vth-toan-8-kntt-bai-30-ket-qua-co-the-va-ket-qua-thuan-loi-co-dap-an


\textbf{{QUESTION}}

Chọn phương án đúng
Một lớp học có 30 học sinh, trong đó có 12 học sinh nữ. Trong lớp có 2 học sinh nữ cận thị và 6 học sinh nam không cận thị. Chọn ngẫu nhiên một học sinh trong lớp. Số kết quả thuận lợi cho biến cố “Học sinh đó cận thị” là 
A. 13.
B. 15.
C. 14.
D. 16.

\textbf{{ANSWER}}

Đáp án đúng là: C
Số học sinh nam trong lớp là: 30 – 12 = 18 (học sinh)
Số học sinh nam cận thị là: 18 – 6 = 12 (học sinh)
Tổng số học sinh cận thị là: 2 + 12 = 14 (học sinh).
Vậy số kết quả thuận lợi cho biến cố “Học sinh đó cận thị” là 14.

========================================================================

https://khoahoc.vietjack.com/thi-online/giai-vth-toan-8-kntt-bai-30-ket-qua-co-the-va-ket-qua-thuan-loi-co-dap-an


\textbf{{QUESTION}}

Vuông thực nghiệm giao một con xúc xắc.
a) Liệt kê các kết quả có thể của thực nghiệm trên.
b) Liệt kê các kết quả thuận lợi cho các biến cố sau:
• A: “Số chấm xuất hiện trên con xúc xắc là hợp số”;
• B: “Số chấm xuất hiện trên con xúc xắc nhỏ hơn 5”;
• C: “Số chấm xuất hiện trên con xúc xắc là số lẻ”.

\textbf{{ANSWER}}

a) Các kết quả có thể là các số chấm có thể xuất hiện trên con xúc xắc, đó là 1; 2; 3; 4; 5; 6 chấm.
b) Các kết quả thuận lợi cho biến cố A là các hợp số trong tập các kết quả có thể, đó là 4; 6 chấm.
Các kết quả thuận lợi cho biến cố B là các số nhỏ hơn 5 trong tập các kết quả có thể, đó là 1; 2; 3; 4 chấm.
Các kết quả thuận lợi cho biến cố C là các số lẻ trong tập các kết quả có thể, đó là 1; 3; 5 chấm.

========================================================================

https://khoahoc.vietjack.com/thi-online/giai-vth-toan-8-kntt-bai-30-ket-qua-co-the-va-ket-qua-thuan-loi-co-dap-an


\textbf{{QUESTION}}

Một hộp đựng 12 tấm thẻ, được ghi số 1; 2;...; 12. Bạn Nam rút ngẫu nhiên một tấm thẻ trong hộp.
a) Liệt kê các kết quả có thể của hành động trên.
b) Liệt kê các kết quả thuận lợi cho các biến cố sau: 
• A: “Rút được tấm thẻ ghi số chẵn”;
• B: “Rút được tấm thẻ ghi số nguyên tố”;
• C: “Rút được tấm thẻ ghi số chính phương”.

\textbf{{ANSWER}}

a) Các kết quả có thể là tấm thẻ ghi số 1; 2; ...; 12.
b) Các kết quả thuận lợi cho biến cố A là các tấm thẻ ghi số chẵn trong tập các kết quả có thể. Đó là các tấm thẻ ghi số 2; 4; 6; 8; 10; 12.
Các kết quả thuận lợi cho biến cố B là các tấm thẻ ghi số nguyên tố trong tập các kết quả có thể. Đó là các tấm thẻ ghi số 2; 3; 5; 7; 11.
Các kết quả thuận lợi cho biến cố C là các tấm thẻ ghi số chính phương trong tập các kết quả có thể. Đó là các tấm thẻ ghi số 1; 4; 9.

========================================================================

https://khoahoc.vietjack.com/thi-online/giai-vth-toan-8-kntt-bai-30-ket-qua-co-the-va-ket-qua-thuan-loi-co-dap-an


\textbf{{QUESTION}}

Lớp 8B có 16 học sinh nam, 22 học sinh nữ, trong đó có 13 học sinh nam thuận tay phải, kí hiệu là A1, A2,…, A13; 3 học sinh nam thuận tay trái, kí hiệu là B1, B2, B3; 20 học sinh nữ thuận tay phải, kí hiệu là C1, C2,…, C20 và 2 học sinh nữ thuận tay trái, kí hiệu là D1, D2. Chọn ngẫu nhiên một học sinh trong lớp.
a) Liệt kê tất cả các kết quả có thể.
b) Liệt kê các kết quả thuận lợi cho biến cố E: “Học sinh đó là nam thuận tay phải”.
c) Liệt kê các kết quả thuận lợi cho biến cố F: “Học sinh đó thuận tay trái”.

\textbf{{ANSWER}}

a) Các kết quả có thể là A1; A2;…; A13; B1; B2; B3; C1; C2;…; C20; D1; D2.
b) Các kết quả thuận lợi cho biến cố E là các học sinh nam thuận tay phải. Đó là A1; A2;…; A13.
c) Các kết quả thuận lợi cho biến cố F là các học sinh thuận tay trái. Đó là B1; B2; B3; D1; D2.

========================================================================

https://khoahoc.vietjack.com/thi-online/bo-de-thi-thu-mon-toan-thpt-quoc-gia-nam-2022-co-loi-giai-30-de/81443


\textbf{{QUESTION}}

A. $$ \mathrm{log}\left(a.b\right)=\mathrm{log}a.\mathrm{log}b.$$
B. $$ \mathrm{log}\left(a.b\right)=\mathrm{log}a+\mathrm{log}b.$$
C. $$ \mathrm{log}\frac{a}{b}=\frac{\mathrm{log}a}{\mathrm{log}b}$$
D. $$ \mathrm{log}\frac{a}{b}=\mathrm{log}b-\mathrm{log}a$$

\textbf{{ANSWER}}

Chọn B.

========================================================================

https://khoahoc.vietjack.com/thi-online/bo-de-thi-thu-mon-toan-thpt-quoc-gia-nam-2022-co-loi-giai-30-de/81443


\textbf{{QUESTION}}

A. 90
B. 9ln10
C. 40
D. 9ln10

\textbf{{ANSWER}}

Chọn B.
Ta có I=1∫010xdx=10xln10|10=9ln10.$$ I=\underset{0}{\overset{1}{\int }}{10}^{x}dx=\frac{{10}^{x}}{\mathrm{ln}10}\left|\begin{array}{l}1\\ 0\end{array}\right.=\frac{9}{\mathrm{ln}10}.$$

========================================================================

https://khoahoc.vietjack.com/thi-online/bo-de-thi-thu-mon-toan-thpt-quoc-gia-nam-2022-co-loi-giai-30-de/81443


\textbf{{QUESTION}}

Có bao nhiêu cách xếp 4 học sinh thành một hàng dọc

\textbf{{ANSWER}}

Chọn C.
Số cách xếp 4 học sinh thành một hàng dọc là số hoán vị của 4 phần tử P4=4!=24 cách.

========================================================================

https://khoahoc.vietjack.com/thi-online/bo-de-thi-thu-mon-toan-thpt-quoc-gia-nam-2022-co-loi-giai-30-de/81443


\textbf{{QUESTION}}

Họ nguyên hàm của hàm số fx=cosx−1sin2x là
A. −sinx+cotx+C.
B. sinx+cotx+C.
C. −sinx-cotx+C.
D. sinx-cotx+C.

\textbf{{ANSWER}}

Chọn B.
Áp dụng bảng nguyên hàm, ta được họ nguyên hàm của hàm số f(x)=cosx−1sin2x là

========================================================================

https://khoahoc.vietjack.com/thi-online/bai-14-so-nguyen-to-hop-so-bang-so-nguyen-to


\textbf{{QUESTION}}

Trong các số 7, 8, 9, số nào là số nguyên tố, số nào là hợp số ? Vì sao ?

\textbf{{ANSWER}}

- Số 7 là số nguyên tố vì 7 là số tự nhiên lớn hơn 1 và có hai ước là 1 và chính nó
- Số 8 là hợp số vì 8 là số tự nhiên lớn hơn 1 và có nhiều hơn hai ước đó là 1; 2; 4; 8
- Số 9 là hợp số vì 9 là số tự nhiên lớn hơn 1 và có nhiều hai ước là 1; 3; 9

========================================================================

https://khoahoc.vietjack.com/thi-online/bai-14-so-nguyen-to-hop-so-bang-so-nguyen-to


\textbf{{QUESTION}}

Các số sau là số nguyên tố hay hợp số?
312; 213; 435; 417; 3311; 67

\textbf{{ANSWER}}

*Phương pháp kiểm tra một số a là số nguyên tố: Chia lần lượt a cho các số nguyên tố (2; 3; 5; 7; 11; 13; …) mà bình phương không vượt quá a
– 312 chia hết cho 2 nên không phải số nguyên tố.
– 213 có 2 + 1 + 3 = 6 nên chia hết cho 3. Do đó 213 không phải số nguyên tố.
– 435 chia hết cho 5 nên không phải số nguyên tố.
- 417 chia hết cho 3 nên không phải số nguyên tố.
– 3311 chia hết cho 11 nên không phải số nguyên tố.
– 67 không chia hết cho 2; 3; 5; 7 nên 67 là số nguyên tố. (chỉ chia đến 7 vì các số nguyên tố khác lớn hơn 7 thì bình phương của chúng lớn hơn 67).

========================================================================

https://khoahoc.vietjack.com/thi-online/bo-5-de-thi-giua-ki-2-toan-10-canh-dieu-cau-truc-moi-co-dap-an/160777


\textbf{{QUESTION}}

Giả sử một công việc được chia thành hai công đoạn. Công đoạn thứ nhất có 2 cách thực hiện và ứng với mỗi cách đó có 6 cách thực hiện công đoạn thứ hai. Khi đó, công việc có thể thực hiện theo bao nhiêu cách?

\textbf{{ANSWER}}

Đáp án đúng là: C
Theo quy tắc nhân ta có công việc đó thực hiện theo $2.6 = 12$ cách.

========================================================================

https://khoahoc.vietjack.com/thi-online/bo-5-de-thi-giua-ki-2-toan-10-canh-dieu-cau-truc-moi-co-dap-an/160777


\textbf{{QUESTION}}

Với k,n là các số tự nhiên và 1≤k≤n, công thức nào sau đây là đúng?

\textbf{{ANSWER}}

Đáp án đúng là: B
Akn=n!(n−k)!$A_n^k = \frac{{n!}}{{(n - k)!}}$.

========================================================================

https://khoahoc.vietjack.com/thi-online/bo-5-de-thi-giua-ki-2-toan-10-canh-dieu-cau-truc-moi-co-dap-an/160777


\textbf{{QUESTION}}

Cho k,n$k,n$ là các số nguyên dương thoả mãn n≥k$n \ge k$. Trong các phát biểu sau, phát biểu nào đúng?
A. Akn=n(n−1)…(n−k+1)$A_n^k = n(n - 1) \ldots (n - k + 1)$.                                    
B. Akn=n(n−1)…k$A_n^k = n(n - 1) \ldots k$.

\textbf{{ANSWER}}

Đáp án đúng là: A
$A_n^k = n(n - 1) \ldots (n - k + 1)$.

========================================================================

https://khoahoc.vietjack.com/thi-online/bo-5-de-thi-giua-ki-2-toan-10-canh-dieu-cau-truc-moi-co-dap-an/160777


\textbf{{QUESTION}}

Cho tập hợp A$A$ có n$n$ phần tử ( n≥1$n \ge 1$) và số nguyên dương k$k$ thoả mãn k≤n$k \le n$. Một tổ hợp chập k$k$ của n$n$ phần tử là:
A. Tất cả kết quả của việc lấy k$k$ phần tử từ n$n$ phần tử của tập hợp A$A$ và sắp xếp chúng theo một thứ tự nào đó.
B. Tất cả tập con gồm k$k$ phần tử được lấy ra từ n$n$ phần tử của tập hợp A$A$.
C. Mỗi kết quả của việc lấy k$k$ phần tử từ n$n$ phần tử của tập hợp A$A$ và sắp xếp chúng theo một thứ tự nào đó.

\textbf{{ANSWER}}

Đáp án đúng là: D
Mỗi tập con gồm k$k$ phần tử được lấy ra từ n$n$ phần tử của tập hợp A$A$ được gọi là một tổ hợp chập k$k$ của n$n$ phần tử .

========================================================================

https://khoahoc.vietjack.com/thi-online/bo-5-de-thi-giua-ki-2-toan-10-canh-dieu-cau-truc-moi-co-dap-an/160777


\textbf{{QUESTION}}

Hệ số của x4 trong khai triển biểu thức (x+2)5 là:

\textbf{{ANSWER}}

Đáp án đúng là: D
Số hạng chứa x4${x^4}$ trong khai triển biểu thức (x+2)5${(x + 2)^5}$ là 5⋅x4⋅2=10x4$5 \cdot {x^4} \cdot 2 = 10{x^4}$. 
Vậy hệ số của x4${x^4}$ là 10.

========================================================================

https://khoahoc.vietjack.com/thi-online/chuong-2-de-kiem-tra-danh-gia/58809


\textbf{{QUESTION}}

a, Thực hiện phép tính A = $$ \sqrt{7-4\sqrt{3}}+\frac{1}{2-\sqrt{3}}$$
b, Rút gọn biểu thức B = $$ {\mathrm{sin}}^{2}{19}^{0}+{\mathrm{cos}}^{2}{19}^{0}+\mathrm{tan}{19}^{0}-cot{71}^{0}$$

\textbf{{ANSWER}}

a, A = $$ \sqrt{7-4\sqrt{3}}+\frac{1}{2-\sqrt{3}}$$ = $$ 2-\sqrt{3}+2+\sqrt{3}$$ = 4
b, B = $$ {\mathrm{sin}}^{2}{19}^{0}+{\mathrm{cos}}^{2}{19}^{0}+\mathrm{tan}{19}^{0}-cot{71}^{0}$$
= $$ {\mathrm{sin}}^{2}{19}^{0}+{\mathrm{cos}}^{2}{19}^{0}+\mathrm{tan}{19}^{0}-\mathrm{tan}{19}^{0}$$ = 1

========================================================================

https://khoahoc.vietjack.com/thi-online/chuong-2-de-kiem-tra-danh-gia/58809


\textbf{{QUESTION}}

a, Cho biểu thức A = 3x-1+1√x+1$$ \frac{3}{x-1}+\frac{1}{\sqrt{x}+1}$$. Tìm x với A = 12$$ \frac{1}{2}$$
b, Tính P = A:1√x+1$$ \frac{1}{\sqrt{x}+1}$$. Tìm x với P<0
c, Tìm giá trị nhỏ nhất của biểu thức M = x+12√x-1.1P$$ \frac{x+12}{\sqrt{x}-1}.\frac{1}{P}$$

\textbf{{ANSWER}}

a, Tìm được A = 1√x-1$$ \frac{1}{\sqrt{x}-1}$$; với x≥0, x≠1. Ta có A =  12$$ \frac{1}{2}$$ => x = 9
b, Tìm được P =  √x+2√x-1$$ \frac{\sqrt{x}+2}{\sqrt{x}-1}$$. Ta có P<0 và điều kiện x≥0, x≠1 ta tìm được 0≤x≤1
c, M = x+12√x-1.1P$$ \frac{x+12}{\sqrt{x}-1}.\frac{1}{P}$$ = x+12√x+2=(√x+2)2√x+2+4$$ \frac{x+12}{\sqrt{x}+2}=\frac{{\left(\sqrt{x}+2\right)}^{2}}{\sqrt{x}+2}+4$$ ≥ 4
Vậy M min = 4 <=> x = 4

========================================================================

https://khoahoc.vietjack.com/thi-online/chuong-2-de-kiem-tra-danh-gia/58809


\textbf{{QUESTION}}

Cho hai hàm số y = 2x + l và y = x – 1 có đồ thị lần lượt là đường thẳng d1$$ {d}_{1}$$ và d2$$ {d}_{2}$$
a, Vẽ d1$$ {d}_{1}$$ và d2$$ {d}_{2}$$ trên cùng một hệ trục tọa độ Oxy
b, Tìm tọa độ giao điểm C của d1$$ {d}_{1}$$ và d2$$ {d}_{2}$$ bằng đồ thị và bằng phép toán
c, Gọi A và B lần lượt là giao điểm của d1$$ {d}_{1}$$ và d2$$ {d}_{2}$$ với trục hoàng. Tính diện tích của tam giác ABC

\textbf{{ANSWER}}

a, HS Tự làm
b, Tìm được C(–2; –3) là tọa độ giao điểm của d1$$ {d}_{1}$$ và d2$$ {d}_{2}$$
c, Kẻ OH⊥$$ \bot $$AB (CH⊥$$ \bot $$Ox)
SABC=12CH.AB=94$$ {S}_{ABC}=\frac{1}{2}CH.AB=\frac{9}{4}$$ (đvdt)

========================================================================

https://khoahoc.vietjack.com/thi-online/10-bai-tap-tim-so-schua-biet-trong-phep-tinh-co-loi-giai


\textbf{{QUESTION}}

Tìm x, biết: $184 - {\left( {x - 6} \right)^2} = 1339:13$
A. x = 15;
B. x = 86;
C. x = 46;
D. x = 45.

\textbf{{ANSWER}}

Đáp án đúng là: A
.$184 - {\left( {x - 6} \right)^2} = 1339:13$.
184 – (x – 6)2 = 103
(x – 6)2 = 184 – 103
(x – 6)2 = 81
(x – 6)2 = 92
x – 6 = 9
x = 9 + 6
x = 15
Vậy x = 15.

========================================================================

https://khoahoc.vietjack.com/thi-online/10-bai-tap-tim-so-schua-biet-trong-phep-tinh-co-loi-giai


\textbf{{QUESTION}}

Cho 2x – 138 = 23.22. Giá trị của x là
A. 58;
B. 53;
C. 35;
D. 85.

\textbf{{ANSWER}}

Đáp án đúng là: D
2x – 138 = 23.22
2x – 138 = 8.4
2x – 138 = 32
2x = 32 + 138
2x = 170
x = 170:2
x = 85
Vậy x = 85

========================================================================

https://khoahoc.vietjack.com/thi-online/10-bai-tap-tim-so-schua-biet-trong-phep-tinh-co-loi-giai


\textbf{{QUESTION}}

A. 31;
B. 13;
C. 37;
D. 73.

\textbf{{ANSWER}}

Đáp án đúng là: A
105 – 3(x – 5) = 35:32
105 – 3(x – 5) = 33
3(x – 5) = 105 – 27
3(x – 5) = 78
x – 5 = 78 : 3
x – 5 = 26
x = 26 + 5
x = 31
Vậy x = 31.

========================================================================

https://khoahoc.vietjack.com/thi-online/10-bai-tap-tim-so-schua-biet-trong-phep-tinh-co-loi-giai


\textbf{{QUESTION}}

Tìm x, biết: 62 – (2x – 3):3 = 23
A. x = 8;
B. x = 60;
C. x = 6;
D. x = 80.

\textbf{{ANSWER}}

Đáp án đúng là: B
62 – (2x – 3):3 = 23
(2x – 3):3 = 62 – 23
(2x – 3):3 = 39
2x – 3 = 39.3
2x – 3 = 117
2x = 117 + 3
x = 120 : 2
x = 60
Vậy x = 60

========================================================================

https://khoahoc.vietjack.com/thi-online/10-bai-tap-tim-so-schua-biet-trong-phep-tinh-co-loi-giai


\textbf{{QUESTION}}

Cho (3x – 2)3 = 64. Giá trị của x là
A. x = 2;
B. x = 3;
C. x = 5;
D. x = 4.

\textbf{{ANSWER}}

Đáp án đúng là: A
(3x – 2)3 = 43
3x – 2 = 4
3x = 4 + 2
3x = 6
x = 2
Vậy x = 2

========================================================================

https://khoahoc.vietjack.com/thi-online/35-de-minh-hoa-thpt-quoc-gia-mon-toan-nam-2022-co-loi-giai-x/75250


\textbf{{QUESTION}}

Cho tập hợp $$ A$$ gồm 12 phần tử. Số tập con gồm 4 phần tử của tập hợp $$ A$$ là 
A. $$ {A}_{12}^{8}.$$
B. $$ {C}_{12}^{4}$$
C. $$ 4!$$
D. $$ {A}_{12}^{4}$$

\textbf{{ANSWER}}

Số cách chọn 4 phần tử từ 12 phần tử bằng: $$ {C}_{12}^{4}.$$
Chọn đáp án B.

========================================================================

https://khoahoc.vietjack.com/thi-online/35-de-minh-hoa-thpt-quoc-gia-mon-toan-nam-2022-co-loi-giai-x/75250


\textbf{{QUESTION}}

Cho cấp số cộng (un)$$ \left({u}_{n}\right)$$, có u1=−2,u4=4.$$ {u}_{1}=-\mathrm{2,}{u}_{4}=4.$$ Số hạng u6$$ {u}_{6}$$ là
A. 8
B. 6
C. 10
D. 12

\textbf{{ANSWER}}

Áp dụng công thức của cấp số cộng un=u1+(n−1)d,$$ {u}_{n}={u}_{1}+\left(n-1\right)d,$$ ta có
u4=u1+3d⇔4=−2+3d⇔d=2.$$ {u}_{4}={u}_{1}+3d\Leftrightarrow 4=-2+3d\Leftrightarrow d=2.$$
Vậy u6=u1+5d=−2+5(2)=8.$$ {u}_{6}={u}_{1}+5d=-2+5\left(2\right)=8.$$
Chọn đáp án A.

========================================================================

https://khoahoc.vietjack.com/thi-online/bo-20-de-thi-hoc-ki-1-toan-11-nam-2022-2023-co-dap-an/116580


\textbf{{QUESTION}}

Tính tổng S tất cả các hệ số trong khai triển ${\left( {3x - 4} \right)^{17}}.$

A. $S = - 1.$
A.
 $S = - 1.$

B. $S = 1.$
B. $S = 1.$

C. $S = 0.$
C. $S = 0.$

D. $S = 8192.$
D. $S = 8192.$

\textbf{{ANSWER}}

Đáp án A
Phương pháp
+ Sử dụng khai triển nhị thức Newton: ${\left( {a + b} \right)^n} = \sum\limits_{k = 0}^n {C_n^k{a^k}{b^{n - k}}} .$ 
+ Thay $x = 1$ để tính tổng các hệ số của khai triển.
Cách giải:
Ta có: ${\left( {3x - 4} \right)^{17}} = \sum\limits_{k = 0}^{17} {C_{17}^k{{.3}^k}.{{\left( { - 4} \right)}^{17 - k}}{x^k}} $ (*).
Hệ số ${a_k} = \sum\limits_{k = 0}^{17} {C_{17}^k{{.3}^k}.{{\left( { - 4} \right)}^{17 - k}}.} $ 
Thay $x = 1$ vào (*) ta được tổng các hệ số: $S = {\left( {3.1 - 4} \right)^{17}} = - 1.$

========================================================================

https://khoahoc.vietjack.com/thi-online/bo-20-de-thi-hoc-ki-1-toan-11-nam-2022-2023-co-dap-an/116580


\textbf{{QUESTION}}

Làng Duyên Yên, xã Ngọc Thanh, huyện Kim Động, tỉnh Hưng Yên nổi tiếng với trò chơi dân gian đánh đu. Trong trò chơi này, khi người chơi nhún đều thì cây đu sẽ đưa người chơi dao động qua lại ở vị trí cân bằng. Nghiên cứu trò chơi này, người ta thấy rằng khoảng cách h (tính bằng mét) từ người chơi đu đến vị trí cân bằng được biểu diễn qua thời gian t(t≥0 và được tính bằng giây) bởi hệ thức h=|d| với d=3cos[π3(2t−1)]. Trong đó quy ước rằng d>0 khi vị trí cân bằng ở phía sau lưng người chơi đu và d<0 trong trường hợp trái lại. Tìm thời điểm đầu tiên sau 10 giây mà người chơi đu ở xa vị trí cân bằng nhất.
t
t
t
t
t≥0
t≥0
t≥0
t
≥
0
h=|d|
h=|d|
h=|d|
h
=
|d|
|
d
|
d=3cos[π3(2t−1)].
d=3cos[π3(2t−1)].
d=3cos[π3(2t−1)].
d
=
3
cos

[π3(2t−1)]
[
[
π3(2t−1)
π3
π
π
3
3


3
3
3
(2t−1)
(
2t−1
2
t
−
1
)
]
]
.
d>0
d>0
d>0
d
>
0
d<0
d<0
d<0
d
<
0
A.
 Giây thứ 13.
B. Giây thứ 12,5.
C. Giây thứ 10,5.
D. Giây thứ 11.

\textbf{{ANSWER}}

Đáp án D
Phương pháp:
Vị trí xa cân bằng nhất là ở biên nên cho hmax${h_{\max }}$, tìm t nhỏ nhất thỏa mãn.
Cách giải:
Vị trí xa vị trí cân bằng nhất nên ta có:
|3cos(π3(2t−1))|=3⇔[cos(π3(2t−1))=1cos(π3(2t−1))=−1$\left| {3\cos \left( {\frac{\pi }{3}\left( {2t - 1} \right)} \right)} \right| = 3 \Leftrightarrow \left[ \begin{array}{l}\cos \left( {\frac{\pi }{3}\left( {2t - 1} \right)} \right) = 1\\\cos \left( {\frac{\pi }{3}\left( {2t - 1} \right)} \right) = - 1\end{array} \right.$ 
⇔sin(π3(2t−1))=0⇔π3(2t−1)=kπ⇔t=3k+12$ \Leftrightarrow \sin \left( {\frac{\pi }{3}\left( {2t - 1} \right)} \right) = 0 \Leftrightarrow \frac{\pi }{3}\left( {2t - 1} \right) = k\pi \Leftrightarrow t = \frac{{3k + 1}}{2}$ 
Vị trí sau giây thứ 10 nên: t>10⇒3k+12>10⇔k>193⇔k≥7$t > 10 \Rightarrow \frac{{3k + 1}}{2} > 10 \Leftrightarrow k > \frac{{19}}{3} \Leftrightarrow k \ge 7$ (Do k∈Z$k \in \mathbb{Z}$ ).
k≥7⇒t≥3.7+12=11.$k \ge 7 \Rightarrow t \ge \frac{{3.7 + 1}}{2} = 11.$ 
Vậy thời điểm đầu tiên sau 10 giây mà người chơi đu ở vị trí cân bằng nhất là giây thứ 11.

========================================================================

https://khoahoc.vietjack.com/thi-online/bo-20-de-thi-hoc-ki-1-toan-11-nam-2022-2023-co-dap-an/116580


\textbf{{QUESTION}}

Bạn An muốn mua một chiếc áo sơ mi cỡ 39 hoặc 40. Biết áo cỡ 39 có 3 màu khác nhau, cỡ 40 có 5 màu khác nhau. Hỏi bạn An có bao nhiêu lựa chọn để mua một chiếc áo?
A.
 8.
B. 3.
C. 5.
D. 15.

\textbf{{ANSWER}}

Đáp án A
Phương pháp
Sử dụng quy tắc cộng.
Cách giải:
Có 3 cách chọn cỡ 39.
Có 5 cách chọn cỡ 40.
Suy ra có 8 cách chọn cỡ 39 hoặc 40.

========================================================================

https://khoahoc.vietjack.com/thi-online/bo-20-de-thi-hoc-ki-1-toan-11-nam-2022-2023-co-dap-an/116580


\textbf{{QUESTION}}

Số đường chéo của đa giác 10 cạnh là:
A.
 35.
B. 710.
710.
710.
710.
710
710
7
10
10
.
C. 45.
D. 1010.
1010.
1010.
1010.
1010
1010
10
10
10
.

\textbf{{ANSWER}}

Đáp án A
Phương pháp:
Sử dụng tổ hợp.
Cách giải:
Đa giác có 10 cạnh suy ra sẽ có 10 đỉnh.
Chọn 2 trong 10 đỉnh ta được các đoạn thẳng chứa cả cạnh và đường chéo của đa giác là C210. 
Suy ra đa giác có các đường chéo là C210−10=35.

========================================================================

https://khoahoc.vietjack.com/thi-online/bo-20-de-thi-hoc-ki-1-toan-11-nam-2022-2023-co-dap-an/116580


\textbf{{QUESTION}}

Từ các chữ số của tập A={1;2;3;4;5;6} có thể lập được bao nhiêu số tự nhiên có 3 chữ số mà các chữ đôi một khác nhau?
A={1;2;3;4;5;6}
A={1;2;3;4;5;6}
A={1;2;3;4;5;6}
A
=
{1;2;3;4;5;6}
{
1;2;3;4;5;6
1
;
2
;
3
;
4
;
5
;
6
}
A.
 125.
B. 120. 
 
C. 6.
D. 10.

\textbf{{ANSWER}}

Đáp án B
Phương pháp:
Sử dụng quy tắc nhân.
Cách giải:
Tập A có 6 phần tử.
Có 6 cách chọn chữ số hàng trăm.
Có 5 cách chọn chữ số hàng chục.
Có 4 cách chọn chữ số hàng đơn vị.
Suy ra có tất cả 6.5.4=120 số tự nhiên thỏa mãn bài toán.
Chọn B.

========================================================================

https://khoahoc.vietjack.com/thi-online/10-bai-taps-nhan-biet-hon-so-duong-co-loi-giai


\textbf{{QUESTION}}

Hỗn số $$ 11\frac{2}{9}$$  đọc là:

\textbf{{ANSWER}}

Đáp án đúng là: C
Hỗn số $$ 11\frac{2}{9}$$   đọc là mười một hai phần chín.

========================================================================

https://khoahoc.vietjack.com/thi-online/10-bai-taps-nhan-biet-hon-so-duong-co-loi-giai


\textbf{{QUESTION}}

Số nào dưới đây là hỗn số dương?
A. 937$$ 9\frac{3}{7}$$
B. 387$$ 3\frac{8}{7}$$
C. −254$$ -2\frac{5}{4}$$
D. 2−69$$ 2\frac{-6}{9}$$

\textbf{{ANSWER}}

Đáp án đúng là: A
Đáp án A: 937$$ 9\frac{3}{7}$$  là một hỗn số dương vì có phần nguyên và phần phân số là các số dương và phần phân số nhỏ hơn 1.

========================================================================

https://khoahoc.vietjack.com/thi-online/10-bai-taps-nhan-biet-hon-so-duong-co-loi-giai


\textbf{{QUESTION}}

Phần nguyên của hỗn số 31529$$ 3\frac{15}{29}$$  là: 
A. 1529$$ \frac{15}{29}$$
B. 329$$ \frac{3}{29}$$
C. 15
D. 3

\textbf{{ANSWER}}

Đáp án đúng là: D
Phần nguyên của hỗn số 31529$$ 3\frac{15}{29}$$   là 3.

========================================================================

https://khoahoc.vietjack.com/thi-online/10-bai-taps-nhan-biet-hon-so-duong-co-loi-giai


\textbf{{QUESTION}}

Phần phân số của hỗn số 458   là:
A. 48
B. 54
C. 58
D. 45

\textbf{{ANSWER}}

Đáp án đúng là: C
Phần phân số của hỗn số   458 $$ 4\frac{5}{8}\quad $$là  58$$ \frac{5}{8}$$.

========================================================================

https://khoahoc.vietjack.com/thi-online/10-bai-taps-nhan-biet-hon-so-duong-co-loi-giai


\textbf{{QUESTION}}

Chuyển hỗn số 3511  thành phân số ta được
A. 1911
B. 3811
C. 3833
D. 1933

\textbf{{ANSWER}}

Đáp án đúng là: B
 3511=3⋅11+511=3811$$ 3\frac{5}{11}=\frac{3\cdot 11+5}{11}=\frac{38}{11}$$.

========================================================================

https://khoahoc.vietjack.com/thi-online/giai-sgk-toan-8-ctst-bai-2-giai-bai-toan-bang-cach-lap-phuong-trinh-bac-nhat-co-dap-an


\textbf{{QUESTION}}

Sau khi giảm giá 15% thì đôi giày thể thao có giá là 1 275 000 đồng. Hỏi lúc chưa giảm giá thì đôi giày có giá là bao nhiêu?

\textbf{{ANSWER}}

Giảm giá 15% suy ra sau khi giảm giá, đôi giày có giá bằng 85% giá gốc ban đầu.
Giá đôi giày lúc chưa giảm giá là: 
(1 275 000 : 85%) × 100% = 1500 000 (đồng)
Vậy giá đôi giày khi chưa giảm giá là: 1 500 000 đồng.

========================================================================

https://khoahoc.vietjack.com/thi-online/giai-sgk-toan-8-ctst-bai-2-giai-bai-toan-bang-cach-lap-phuong-trinh-bac-nhat-co-dap-an


\textbf{{QUESTION}}

Một mảnh vườn hình chữ nhật có chiều rộng là x (m), chiều dài hơn chiều rộng 20 m. Hãy viết biểu thức với biến x biểu thị: 
a) Chiều dài của hình chữ nhật;
b) Chu vi của hình chữ nhật;
c) Diện tích của hình chữ nhật.

\textbf{{ANSWER}}

a) Chiều dài của hình chữ nhật là: x + 20 (m)
b) Chu vi của hình chữ nhật là: (x + x + 20).2 = 4x + 40 (m)
c) Diện tích của hình chữ nhật là: x(x + 20) = x2 + 20x (m2)

========================================================================

https://khoahoc.vietjack.com/thi-online/giai-sgk-toan-8-ctst-bai-2-giai-bai-toan-bang-cach-lap-phuong-trinh-bac-nhat-co-dap-an


\textbf{{QUESTION}}

Tiền lương cơ bản của anh Minh mỗi tháng là x (triệu đồng). Tiền phụ cấp mỗi tháng là 3 500 000 đồng.
a) Viết biểu thức biểu thị tiền lương mỗi tháng của anh Minh. Biết tiền lương mỗi tháng bằng tổng tiền lương cơ bản và tiền phụ cấp.

\textbf{{ANSWER}}

a) Biểu thức biểu thị tiền lương mỗi tháng của anh Minh: x + 3 500 000 (đồng)

========================================================================

https://khoahoc.vietjack.com/thi-online/giai-sgk-toan-8-ctst-bai-2-giai-bai-toan-bang-cach-lap-phuong-trinh-bac-nhat-co-dap-an


\textbf{{QUESTION}}

b) Tháng Tết, anh Minh được thưởng 1 tháng lương cùng với 60% tiền phụ cấp. Viết biểu thức chỉ số tiền anh Minh được nhận ở tháng Tết.

\textbf{{ANSWER}}

b) Biểu thức chỉ số tiền anh Minh được nhận ở tháng Tết là: 
(x + 3 500 000) + (x + 0,8.3 500 000) = 2x + 6 300 000 (đồng)

========================================================================

https://khoahoc.vietjack.com/thi-online/giai-sgk-toan-8-ctst-bai-2-giai-bai-toan-bang-cach-lap-phuong-trinh-bac-nhat-co-dap-an


\textbf{{QUESTION}}

Thay dấu   ?   $$ \overline{)\text{\hspace{0.17em}\hspace{0.17em}}?\text{\hspace{0.17em}\hspace{0.17em}}}\quad $$bằng các dữ liệu thích hợp để hoàn thành lời giải bài toán.
Một người đi xe gắn máy từ A đến B với tốc độ 40 km/h. Lúc về người đó đi với tốc độ 50 km/h nên thời gian về ít hơn thời gian đi là 30 phút. Tìm chiều dài quãng đường AB.
Gọi chiều dài quãng đường AB là x (km). Điều kiện x >   ?  $$ \overline{)\text{\hspace{0.17em}\hspace{0.17em}}?\text{\hspace{0.17em}\hspace{0.17em}}}$$  .
Thời gian đi là: x40$$ \frac{x}{40}$$ giờ.
Thời gian về là:   ?  $$ \overline{)\text{\hspace{0.17em}\hspace{0.17em}}?\text{\hspace{0.17em}\hspace{0.17em}}}$$ .
Ta có: 30 phút = 12$$ \frac{1}{2}$$  giờ.
Vì thời gian về ít hơn thời gian đi là 12$$ \frac{1}{2}$$  giờ nên ta có phương trình: x40−  ?  =12$$ \frac{x}{40}-\overline{)\text{\hspace{0.17em}\hspace{0.17em}}?\text{\hspace{0.17em}\hspace{0.17em}}}=\frac{1}{2}$$
 
Giải phương trình, ta được x =   ?  $$ \overline{)\text{\hspace{0.17em}\hspace{0.17em}}?\text{\hspace{0.17em}\hspace{0.17em}}}$$  thỏa mãn điều kiện x >   ?  $$ \overline{)\text{\hspace{0.17em}\hspace{0.17em}}?\text{\hspace{0.17em}\hspace{0.17em}}}$$ .
Vậy chiều dài của quãng đường AB là   ?  $$ \overline{)\text{\hspace{0.17em}\hspace{0.17em}}?\text{\hspace{0.17em}\hspace{0.17em}}}$$ .

\textbf{{ANSWER}}

Gọi chiều dài quãng đường AB là x (km). Điều kiện x > 0.
Thời gian đi là:x40$$ \frac{x}{40}$$   giờ
Thời gian về là: x50$$ \frac{x}{50}$$  giờ
Ta có: 30 phút =  12$$ \frac{1}{2}$$  giờ
Vì thời gian về ít hơn thời gian đi là 12$$ \frac{1}{2}$$ giờ nên ta có phương trình: x40−x50=12$$ \frac{x}{40}-\frac{x}{50}=\frac{1}{2}$$
 
Giải phương trình, ta được x = 100 thỏa mãn điều kiện x > 0.
Vậy chiều dài của quãng đường AB là 100 km.

========================================================================

https://khoahoc.vietjack.com/thi-online/de-thi-giua-ki-1-toan-10-chan-troi-sang-tao-co-dap-an/106943


\textbf{{QUESTION}}

Trong các bất phương trình dưới đây, bất phương trình nào là bất phương trình bậc nhất hai ẩn?

\textbf{{ANSWER}}

Đáp án đúng là D

========================================================================

https://khoahoc.vietjack.com/thi-online/giai-sbt-toan-8-ctst-cac-truong-hop-dong-dang-cua-hai-tam-giac-co-dap-an


\textbf{{QUESTION}}

Tam giác ABC có độ dài AB = 9 cm, AC = 12 cm, BC = 14 cm. Tam giác A'B'C' đồng dạng với tam giác ABC và có chu vi bằng 61.25 cm. Hãy tính độ dài các cạnh của tam giác A'B'C'.

\textbf{{ANSWER}}

Ta có ∆A’B’C’ ᔕ ∆ABC, suy ra 
$\frac{{A'B'}}{{AB}} = \frac{{A'C'}}{{AC}} = \frac{{B'C'}}{{BC}}$hay $\frac{{A'B'}}{9} = \frac{{A'C'}}{{12}} = \frac{{B'C'}}{{14}}$.
Áp dụng tính chất tỉ lệ thức, có: 
$\frac{{A'B'}}{9} = \frac{{A'C'}}{{12}} = \frac{{B'C'}}{{14}}$= $\frac{{A'B' + A'C' + B'C'}}{{9 + 12 + 14}} = \frac{{61,25}}{{35}} = \frac{7}{4}$.
Suy ra $\frac{{A'B'}}{9} = \frac{7}{4}$ ; $\frac{{A'C'}}{{12}} = \frac{7}{4}$ và $\frac{{B'C'}}{{14}} = \frac{7}{4}$.
Do đó $A'B' = \frac{{7.9}}{4} = 15,75$; $A'C' = \frac{{7.12}}{4} = 21$ và $B'C' = \frac{{7.14}}{4} = 24,5$.
Vậy A’B’ = 15,75 cm ; A’C’ = 21 cm và B’C’ = 24,5 cm.

========================================================================

https://khoahoc.vietjack.com/thi-online/21-cau-trac-nghiem-toan-6-ket-noi-tri-thuc-bai-17-cac-dang-toan-ve-phep-nhan-chia-so-nguyen-boi-va-u


\textbf{{QUESTION}}

Tính nhanh 
$\left( { - 5} \right).125.\left( { - 8} \right).20.\left( { - 2} \right)\;$

 ta được kết quả là
A. 
−200000
B. 
−2000000
C. 
200000
D.
−100000

\textbf{{ANSWER}}

Trả lời:
$\begin{array}{l}( - 5).125.( - 8).20.( - 2)\\ = [125.( - 8)].[( - 5).20].( - 2)\\ = - (125.8).[ - (5.20)].( - 2)\\ = ( - 1000).( - 100).( - 2)\\ = 100000.( - 2)\\ = - 200000\end{array}$ 
Đáp án cần chọn là: A

========================================================================

https://khoahoc.vietjack.com/thi-online/21-cau-trac-nghiem-toan-6-ket-noi-tri-thuc-bai-17-cac-dang-toan-ve-phep-nhan-chia-so-nguyen-boi-va-u


\textbf{{QUESTION}}

Giá trị biểu thức M=(−192873).(−2345).(−4)5.0$M = \left( { - 192873} \right).\left( { - 2345} \right).{\left( { - 4} \right)^5}.0\;$ là

\textbf{{ANSWER}}

Trả lời:
Vì trong tích có một thừa số bằng 0 nên M=0$M = 0$ 
Đáp án cần chọn là: C

========================================================================

https://khoahoc.vietjack.com/thi-online/21-cau-trac-nghiem-toan-6-ket-noi-tri-thuc-bai-17-cac-dang-toan-ve-phep-nhan-chia-so-nguyen-boi-va-u


\textbf{{QUESTION}}

Tính hợp lý 
A=−43.18−82.43−43.100$A = - 43.18 - 82.43 - 43.100$
A=−43.18−82.43−43.100
A=−43.18−82.43−43.100
A
=
−
43.18
−
82.43
−
43.100
A. 
0
B. 
−86000
C. 
−8600
D. -4300

\textbf{{ANSWER}}

Trả lời:
A=−43.18−82.43−43.100A=43.(−18−82−100)A=43.[−(18+82+100)]A=43.(−200)A=−8600$\begin{array}{l}A = - 43.18 - 82.43 - 43.100\\A = 43.( - 18 - 82 - 100)\\A = 43.[ - (18 + 82 + 100)]\\A = 43.( - 200)\\A = - 8600\end{array}$ 
Đáp án cần chọn là: C

========================================================================

https://khoahoc.vietjack.com/thi-online/21-cau-trac-nghiem-toan-6-ket-noi-tri-thuc-bai-17-cac-dang-toan-ve-phep-nhan-chia-so-nguyen-boi-va-u


\textbf{{QUESTION}}

Cho 
Q=−135.17−121.17−256.(−17)
Q=−135.17−121.17−256.(−17)
Q=−135.17−121.17−256.(−17)
Q
=
−
135.17
−
121.17
−
256.
(−17)
(
−17
−
17
)
 chọn câu đúng.
A. 
−17
B. 
0
C. 
1700
D. 
−1700

\textbf{{ANSWER}}

Trả lời:
Q=−135.17−121.17−256.(−17)Q=−135.17−121.17+256.17Q=17.(−135−121+256)Q=17.(−256+256)Q=17.0Q=0$\begin{array}{l}Q = - 135.17 - 121.17 - 256.( - 17)\\Q = - 135.17 - 121.17 + 256.17\\Q = 17.( - 135 - 121 + 256)\\Q = 17.( - 256 + 256)\\Q = 17.0\\Q = 0\end{array}$ 
Đáp án cần chọn là: B

========================================================================

https://khoahoc.vietjack.com/thi-online/21-cau-trac-nghiem-toan-6-ket-noi-tri-thuc-bai-17-cac-dang-toan-ve-phep-nhan-chia-so-nguyen-boi-va-u


\textbf{{QUESTION}}

Tìm x∈Z biết (x+1)+(x+2)+...+(x+99)+(x+100)=0

\textbf{{ANSWER}}

Trả lời:
(x+1)+(x+2)+...+(x+99)+(x+100)=0(x+x+....+x)+(1+2+...+100)=0100x+(100+1).100:2=0$\begin{array}{l}(x + 1) + (x + 2) + ... + (x + 99) + (x + 100) = 0\\(x + x + .... + x) + (1 + 2 + ... + 100) = 0\\100x + (100 + 1).100:2 = 0\end{array}$ 
100x+5050=0100x=−5050x=−50,5$\begin{array}{l}100x + 5050 = 0\\100x = - 5050\\x = - 50,5\end{array}$ 
Mà x∈Z$x \in Z$  nên không có x thỏa mãn.
Đáp án cần chọn là: B

========================================================================

https://khoahoc.vietjack.com/thi-online/bo-5-de-thi-giua-ki-2-toan-11-ket-noi-tri-thuc-cau-truc-moi-de-so-4


\textbf{{QUESTION}}

PHẦN I. TRẮC NGHIỆM KHÁCH QUAN
A. TRẮC NGHIỆM NHIỀU PHƯƠNG ÁN LỰA CHỌN. Thí sinh trả lời từ câu 1 đến câu 12.
Mỗi câu hỏi thí sinh chỉ chọn một phương án.
Với $a$ là số thực dương tùy ý, tích ${a^2}.{a^{\frac{1}{2}}}$ bằng
A. ${a^{\frac{5}{2}}}$.
B. $a$.
C. ${a^{\frac{3}{2}}}$.
D. ${a^{\frac{1}{4}}}$.

\textbf{{ANSWER}}

Đáp án đúng là: A
Ta có ${a^2}.{a^{\frac{1}{2}}} = {a^{\frac{5}{2}}}$.

========================================================================

https://khoahoc.vietjack.com/thi-online/bo-5-de-thi-giua-ki-2-toan-11-ket-noi-tri-thuc-cau-truc-moi-de-so-4


\textbf{{QUESTION}}

Đặt a=log25$a = {\log _2}5$. Khi đó log2532${\log _{25}}32$ bằng
A. 52a$\frac{5}{{2a}}$.
B. 5a2$\frac{{5a}}{2}$.
C. 25a$\frac{2}{{5a}}$.
D. 2a5$\frac{{2a}}{5}$.

\textbf{{ANSWER}}

Đáp án đúng là: A
log2532=log5225=52log52=52log25=52a${\log _{25}}32 = {\log _{{5^2}}}{2^5} = \frac{5}{2}{\log _5}2 = \frac{5}{{2{{\log }_2}5}} = \frac{5}{{2a}}$.

========================================================================

https://khoahoc.vietjack.com/thi-online/bo-5-de-thi-giua-ki-2-toan-11-ket-noi-tri-thuc-cau-truc-moi-de-so-4


\textbf{{QUESTION}}

Hàm số nào dưới đây đồng biến trên tập xác định của nó?
A. y=(1e)x$y = {\left( {\frac{1}{e}} \right)^x}$.
B. y=(√1π)x$y = {\left( {\sqrt {\frac{1}{\pi }} } \right)^x}$.
C. y=(13)x$y = {\left( {\frac{1}{3}} \right)^x}$.
D. y=(2024√π)x$y = {\left( {\sqrt[{2024}]{\pi }} \right)^x}$.

\textbf{{ANSWER}}

Đáp án đúng là: D
Vì $\sqrt[{2024}]{\pi } > 1$ nên hàm số $y = {\left( {\sqrt[{2024}]{\pi }} \right)^x}$ đồng biến trên tập xác định của nó.

========================================================================

https://khoahoc.vietjack.com/thi-online/bo-5-de-thi-giua-ki-2-toan-11-ket-noi-tri-thuc-cau-truc-moi-de-so-4


\textbf{{QUESTION}}

Tập nghiệm $S$ của bất phương trình $\log x < 1$ là
A. $S = \left( { - \infty ;10} \right)$.
B. $S = \left( {0;10} \right)$.
C. $S = \left( {10; + \infty } \right)$.
D. $S = \left( { - \infty ;1} \right)$.

\textbf{{ANSWER}}

Đáp án đúng là: B
Điều kiện: $x > 0$.
Ta có $\log x < 1$$ \Leftrightarrow x < 10$.
Kết hợp điều kiện ta có tập nghiệm của bất phương trình là $S = \left( {0;10} \right)$.

========================================================================

https://khoahoc.vietjack.com/thi-online/bo-5-de-thi-giua-ki-2-toan-11-ket-noi-tri-thuc-cau-truc-moi-de-so-4


\textbf{{QUESTION}}

Cho ${\log _a}b = 2$ với $a,b$ là số thực dương và $a$ khác 1. Tính giá trị biểu thức $T = {\log _{{a^2}}}{b^6} + {\log _a}\sqrt b $.
A. $T = 7$.
B. $T = 6$.
C. $T = 5$.
D. $T = 8$.

\textbf{{ANSWER}}

Đáp án đúng là: A
$T = {\log _{{a^2}}}{b^6} + {\log _a}\sqrt b $$ = 3{\log _a}b + \frac{1}{2}{\log _a}b$$ = 3.2 + \frac{1}{2}.2 = 7$.

========================================================================

https://khoahoc.vietjack.com/thi-online/luyen-tap-ve-tia


\textbf{{QUESTION}}

Điền vào chỗ trống trong các phát biểu sau:
a/ Hình gồm điểm O và một phần đường thẳng bị chia ra bởi điểm O được gọi là ....
b/ Mỗi điểm trên đường thẳng là .... của hai tia đối nhau.
c/ Ox và Oy là hai tia đối nhau nếu ...

\textbf{{ANSWER}}

a/ Một tia gốc O một nửa đường thẳng gốc O
b/ Gốc chung
c/ Hai tia Ox, Oy chung gốc và tạo thành đường thẳng xy.

========================================================================

https://khoahoc.vietjack.com/thi-online/bai-tap-toan-9-chu-de-2-giai-he-phuong-trinh-co-dap-an/109598


\textbf{{QUESTION}}

Giải hệ phương trình:     $$ \left\{\begin{array}{l}8{\text{x}}^{3}{y}^{3}+27=18{y}^{3}\\ 4{\text{x}}^{2}y+6\text{x}={y}^{2}\end{array}\right.$$

\textbf{{ANSWER}}

Dễ thấy y=0   không là nghiệm của mỗi phương trình. 
Chia cả 2 vế phương trình (1)+ 27 = 18 y63  cho $$ {y}^{3}$$  , phương trình (2) cho y2  ta được $$ \left\{\begin{array}{l}8{x}^{3}+\frac{27}{{y}^{3}}=18\\ 4.\frac{{x}^{2}}{{y}^{}}+6.\frac{x}{{y}^{2}}=1\end{array}\right.$$
Đặt $$ \left\{\begin{array}{l}2x=a\\ \frac{3}{y}=b\end{array}\right.$$  ta có hệ     $$ \left\{\begin{array}{l}{a}^{3}+{b}^{3}=18\\ {a}^{2}b+a{b}^{2}=3\end{array}\right.\Leftrightarrow \left\{\begin{array}{l}a+b=3\\ ab=1\end{array}\right.$$
a; b là nghiệm của phương trình   $$ {X}^{2}-3X+1=0$$
Từ đó suy ra hệ có 2 nghiệm: $$ ({x}_{1},{y}_{1})=\left(\frac{3+\sqrt{5}}{4};\frac{3+\sqrt{5}}{6}\right);({x}_{2},{y}_{2})=\left(\frac{3-\sqrt{5}}{4};\frac{3-\sqrt{5}}{6}\right)$$

========================================================================

https://khoahoc.vietjack.com/thi-online/bai-tap-toan-9-chu-de-2-giai-he-phuong-trinh-co-dap-an/109598


\textbf{{QUESTION}}

Giải hệ phương trình: {x2−2xy+x−2y+3=0y2−x2+2xy+2x−2=0.

\textbf{{ANSWER}}

{x2−2xy+x−2y+3=0   (1)y2−x2+2xy+2x−2=0  (2)⇔{2x2−4xy+2x−4y+6=0 y2−x2+2xy+2x−2=0 $$ \left\{\begin{array}{l}{x}^{2}-2xy+x-2y+3=0\text{   (1)}\\ {y}^{2}-{x}^{2}+2xy+2x-2=0\text{  (2)}\end{array}\right.\Leftrightarrow \left\{\begin{array}{l}2{x}^{2}-4xy+2x-4y+6=0\text{ }\\ {y}^{2}-{x}^{2}+2xy+2x-2=0\text{ }\end{array}\right.$$
Cộng 2 vế của hệ phương trình ta được x2+y2−2xy+4x−4y+4=0$$ {x}^{2}+{y}^{2}-2xy+4x-4y+4=0$$
⇔(x−y+2)2=0$$ \Leftrightarrow {\left(x-y+2\right)}^{2}=0$$
 
⇔y=x+2$$ \Leftrightarrow y=x+2$$.Thay vào pt (1) ta được x2+5x+1=0⇔x=−5±√212$$ {x}^{2}+5x+1=0\Leftrightarrow x=\frac{-5\pm \sqrt{21}}{2}$$

========================================================================

https://khoahoc.vietjack.com/thi-online/bai-tap-toan-9-chu-de-2-giai-he-phuong-trinh-co-dap-an/109598


\textbf{{QUESTION}}

Giải hệ phương trình: {√x+√y+4√xy=16x+y=10
{√x+√y+4√xy=16x+y=10
{√x+√y+4√xy=16x+y=10
{√x+√y+4√xy=16x+y=10
{
{
√x+√y+4√xy=16x+y=10
√x+√y+4√xy=16x+y=10
√x+√y+4√xy=16
√x+√y+4√xy=16
√x
√
√
x

x
x
+
√y
√
√
y

y
y
+4
√xy
√
√
xy

xy
xy
=16
x+y=10
x+y=10
x+y=10

\textbf{{ANSWER}}

{√x+√y+4√xy=16x+y=10(I)( Điều kiện:x; y≥0)
Đặt S= √x+√y  ; P = √xy   ( S≥0;P≥0 ) hệ (I) có dạng:
 
 
{S+4P=16   S2-2P=10      ⇔{S+4P=16   2S2-4P=20      ⇔{S+4P=16   2S2 +S-36=0      
⇔{S=4(tm);S=-92( loai)   P=3     ⇔{S=4   P=3     
Khi đó √x;√y  là 2 nghiệm của phương trình:   t2–
Giải phương trình ta được   t1= 3; t2= 1( thỏa mãn )
    TH1:x=3y=1⇔x=9y=1             TH2:x=1y=3⇔x=1y=9
( thỏa mãn)                                          (thỏa mãn)

========================================================================

https://khoahoc.vietjack.com/thi-online/bai-tap-toan-9-chu-de-2-giai-he-phuong-trinh-co-dap-an/109598


\textbf{{QUESTION}}

Giải hệ phương trình: x2+y2=11x+xy+y=3+42

\textbf{{ANSWER}}

- Đặt  S=x+y; P=xy   được: S2−2P=11S+P=3+42 ⇔S2−2P=112S+2P=6+82
Cộng hai vế của hệ phương trình ta được phương trình:
S2+2S−(17+82)=0
 
- Giải phương trình được S1=3+2  ; S2=−5−2
 S1=3+2 được P1=32  ;S2=−5−2  được P2=8+52
Với S1=3+2 ; P1=32 có x, y là hai nghiệm của phương trình:            
X2−(3+2)X+32=0       
Giải phương trình được X1=3;X2=2  .
Với    S2=−5−2 ,  được P2=8+52  có x, y là hai nghiệm của phương trình:X2+(5+2)X+8+52=0.
 Phương trình này vô nghiệm.
Vậy hệ có hai nghiệm: x=3y=2 ; x=2y=3 .

========================================================================

https://khoahoc.vietjack.com/thi-online/bai-tap-toan-9-chu-de-2-giai-he-phuong-trinh-co-dap-an/109598


\textbf{{QUESTION}}

Giải hệ phương trình: 3+2x+3−2y=x+43+2x−3−2y=x

\textbf{{ANSWER}}

Điều kiện: x≥−32  ;    y≤32
Trừ từng vế hai phương trình của hệ ta được phương trình:
    3−2y=2⇔3–2y=4 ⇔y =−12  (t/mãn đk)
Cộng từng vế hai phương trình của hệ đã cho ta được phương trình:
x+12=0⇔x=−13+2x= x+2⇔(thỏa mãn điều kiện)
Vậy hệ phương trình có nghiệm là: x; y = (−1 ;−12)

========================================================================

https://khoahoc.vietjack.com/thi-online/giai-sbt-toan-10-bai-2-hoan-vi-chinh-hop-va-to-hop-co-dap-an


\textbf{{QUESTION}}

Sau khi biên soạn 9 câu hỏi trắc nghiệm, cô giáo có thể tạo ra bao nhiêu đề kiểm tra khác nhau bằng cách đảo thứ tự các câu hỏi đó.

\textbf{{ANSWER}}

Mỗi cách sắp xếp thứ tự câu hỏi để được một đề ta được một hoán vị của 9 câu hỏi đó. Do đó số đề khác nhau có thể tạo ra là 9! = 362880 đề.

========================================================================

https://khoahoc.vietjack.com/thi-online/giai-sbt-toan-10-bai-2-hoan-vi-chinh-hop-va-to-hop-co-dap-an


\textbf{{QUESTION}}

Cô giáo đã biên soạn 10 câu hỏi trắc nghiệm. Từ 10 câu hỏi này, cô giáo chọn ra 6 câu hỏi và sắp xếp theo thứ tự để tạo nên một đề trắc nghiệm. Cô giáo có thể tạo bao nhiêu đề kiểm tra trắc nghiệm khác nhau?

\textbf{{ANSWER}}

Mỗi đề được tạo ra là một chỉnh hợp chập 6 của 10 câu hỏi. Do đó số đề có thể được tạo ra là A610 =  10!4!= 151200 đề.

========================================================================

https://khoahoc.vietjack.com/thi-online/giai-sbt-toan-10-bai-2-hoan-vi-chinh-hop-va-to-hop-co-dap-an


\textbf{{QUESTION}}

Một giải đấu có 4 đội bóng A, B, C và D tham gia. Các đội đấu vòng tròn một lượt để tính điểm và xếp hạng.
a) Có tất cả bao nhiêu trận đấu?

\textbf{{ANSWER}}

a) Cứ hai đội bất kì thì có một trận đấu. Do đó, số trận đấu của đội bằng số tổ hợp chập 2 của 4 đội, tức bằng  C24=4!2!.2! = 6 trận đấu.

========================================================================

https://khoahoc.vietjack.com/thi-online/giai-sbt-toan-10-bai-2-hoan-vi-chinh-hop-va-to-hop-co-dap-an


\textbf{{QUESTION}}

b) Có tất cả bao nhiêu khả năng có thể xảy ra về đội vô địch và á quân?

\textbf{{ANSWER}}

b) Mỗi kết quả của giải đấu về đội vô địch và á quân là một chỉnh hợp chập 2 của 4 đội, tức bằng A42 = 4.3 = 12 khả năng có thể xảy ra về đội vô địch và á quân,

========================================================================

https://khoahoc.vietjack.com/thi-online/15-cau-trac-nghiem-toan-7-chan-troi-sang-tao-bai-1-tap-hop-so-huu-ti-co-dap-an-phan-2


\textbf{{QUESTION}}

Số nào dưới đây đang ở dạng phân số của số hữu tỉ?

\textbf{{ANSWER}}

Hướng dẫn giải
Đáp án đúng là: A
Số hữu tỉ là số viết được dưới dạng phân số $$ \frac{a}{b}$$, với a, b ∈ ℤ; b ≠ 0.
Do đó ta sẽ đi tìm phân số có tử số, mẫu số đều là số nguyên và mẫu số khác 0.
Trong tất cả các phương án chỉ có phân số $$ \frac{2020}{2021}$$ thỏa mãn điều kiện trên nên là số đang ở dạng phân số của số hữu tỉ.
Ta chọn phương án A.

========================================================================

https://khoahoc.vietjack.com/thi-online/15-cau-trac-nghiem-toan-7-chan-troi-sang-tao-bai-1-tap-hop-so-huu-ti-co-dap-an-phan-2


\textbf{{QUESTION}}

Tập hợp các số hữu tỉ được kí hiệu là:

\textbf{{ANSWER}}

Hướng dẫn giải
Đáp án đúng là: A
Tập hợp các số hữu tỉ được kí hiệu là ℚ.
Ta chọn phương án A.

========================================================================

https://khoahoc.vietjack.com/thi-online/15-cau-trac-nghiem-toan-7-chan-troi-sang-tao-bai-1-tap-hop-so-huu-ti-co-dap-an-phan-2


\textbf{{QUESTION}}

Trong các số sau, số nào là số hữu tỉ dương?
A. 2−3;
B. −32;
C. -−3-2;
D. −3-2.

\textbf{{ANSWER}}

Hướng dẫn giải
Đáp án đúng là: D.
Số hữu tỉ dương là số hữu tỉ lớn hơn 0.
Ta có: 2−3=−23<0;$$ \frac{2}{-3}=-\frac{2}{3}<0;$$ −32=−32<0;$$ \frac{-3}{2}=-\frac{3}{2}<0;$$ −−3−2=−32<0;$$ -\frac{-3}{-2}=-\frac{3}{2}<0;$$−3−2=32>0.$$ \frac{-3}{-2}=\frac{3}{2}>0.$$
Do đó −3−2$$ \frac{-3}{-2}$$ là số hữu tỉ dương.
Ta chọn phương án A.

========================================================================

https://khoahoc.vietjack.com/thi-online/15-cau-trac-nghiem-toan-7-chan-troi-sang-tao-bai-1-tap-hop-so-huu-ti-co-dap-an-phan-2


\textbf{{QUESTION}}

Cho hai số hữu tỉ 3,5 và 5,25, trên trục số nằm ngang, điểm 3,5 ở:

\textbf{{ANSWER}}

Hướng dẫn giải
Đáp án đúng là: B
Với hai số hữu tỉ bất kì x, y, nếu x < y thì trên trục số nằm ngang, điểm x ở bên trái điểm y.
Ta có 3,5 < 5,25 nên 3,5 nằm ở bên trái điểm 5,25.
Ta chọn phương án B.

========================================================================

https://khoahoc.vietjack.com/thi-online/15-cau-trac-nghiem-toan-7-chan-troi-sang-tao-bai-1-tap-hop-so-huu-ti-co-dap-an-phan-2


\textbf{{QUESTION}}

Chọn khẳng định sai:
A. Số đối của số –3,5 là 3,5;

\textbf{{ANSWER}}

Hướng dẫn giải
Đáp án đúng là: C
Số đối của số –3,5 là 3,5=72=312.
Do đó phương án C sai.
Ta chọn phương án C.

========================================================================

https://khoahoc.vietjack.com/thi-online/giai-sgk-toan-11-cd-chu-de-2-tinh-the-tich-mot-so-hinh-khoi-trong-thuc-tien-co-dap-an


\textbf{{QUESTION}}

Giáo viên thực hiện những nhiệm vụ sau:
– Chia lớp thành những nhóm học sinh;
– Giao nhiệm vụ các nhóm tính toán chi phí vật liệu làm bao bì chứa cùng một loại sản phẩm và lựa chọn kiểu bao bì có chi phí thấp nhất.

\textbf{{ANSWER}}

Giáo viên chia lớp thành các nhóm và chuyển giao nhiệm vụ. 
Học sinh thực hiện nhiệm vụ theo hướng dẫn của giáo viên.

========================================================================

https://khoahoc.vietjack.com/thi-online/giai-sgk-toan-11-cd-chu-de-2-tinh-the-tich-mot-so-hinh-khoi-trong-thuc-tien-co-dap-an


\textbf{{QUESTION}}

Mỗi nhóm học sinh trao đổi, thảo luận để xác định rõ: Nhiệm vụ của nhóm và thời gian hoàn thành nhiệm vụ đó; nhiệm vụ của từng thành viên trong nhóm và thời gian hoàn thành nhiệm vụ đó.

\textbf{{ANSWER}}

– Nhiệm vụ của nhóm: tính toán chi phí vật liệu làm bao bì chứa cùng một loại sản phẩm và lựa chọn kiểu bao bì có chi phí thấp nhất.
– Thời gian dự kiến hoàn thành là 30 phút kể từ lúc nhận nhiệm vụ của giáo viên.
– Các nhiệm vụ chi tiết:
+ Sưu tầm các sản phẩm cùng loại nhưng có hình dạng bao bì khác nhau (như: có dạng khối hộp chữ nhật, khối trụ, khối chóp, khối chóp cụt đều, ...): cả nhóm thực hiện và chuẩn bị trước khi đến buổi học.
+ Lựa chọn, thống nhất loại sản phẩm và những hình dạng bao bì của loại sản phẩm đó để nghiên cứu thực hành: cả nhóm thực hiện trong 10 phút.
+ Đo các kích thước và tính thể tích, diện tích toàn phần của các bao bì sản phẩm: 02 học sinh thực hiện trong 15 phút.
+ Tìm hiểu về giá thành sản phẩm bao bì, dự đoán chi phí bao bì: 02 học sinh thực hiện trong 15 phút.

========================================================================

https://khoahoc.vietjack.com/thi-online/giai-sbt-toan-9-kntt-bai-21-giai-toan-bang-cach-lap-phuong-trinh-co-dap-an


\textbf{{QUESTION}}

Một bức ảnh hình chữ nhật có chiều rộng 8 cm và chiều dài 12 cm. Bức ảnh được phóng to bằng cách tăng chiều dài và chiều rộng thêm một đoạn bằng nhau để tăng gấp đôi diện tích của bức ảnh. Tìm kích thước của bức ảnh mới.

\textbf{{ANSWER}}

Diện tích ban đầu của bức ảnh là: 8 . 12 = 96 (cm2)
Gọi độ dài đoạn tăng thêm của mỗi chiều là x (cm) (x > 0).
Diện tích bức ảnh sau khi phóng to là:
(8 + h)(12 + h) = h2 + 20h + 96 (cm2)
Diện tích của bức ảnh sau khi phóng to gấp đôi diện tích bức ảnh ban đầu nên ta có:
h2 + 20h + 96 = 2 . 96
h2 + 20h – 96 = 0
Ta có ∆ = 202 – 4 . 1 . (–96) = 784 > 0 nên phương trình có hai nghiệm phân biệt:
$$ {x}_{1}=\frac{-20+\sqrt{784}}{2.1}=4>0$$ (thỏa mãn điều kiện);
$$ {x}_{2}=\frac{-20-\sqrt{784}}{2.1}=-24<0$$ (không thỏa mãn điều kiện).
Do đó người ta đã tăng mỗi chiều của bức ảnh thêm 4 cm.
Chiều dài bức ảnh mới là: 12 + 4 = 16 (cm) 
Chiều rộng bức ảnh mới là: 8 + 4 = 12 (cm) 
Vậy chiều dài và chiều rộng bức ảnh mới lần lượt là 16 cm và 12 cm.

========================================================================

https://khoahoc.vietjack.com/thi-online/giai-sgk-toan-8-kntt-mot-vai-ung-dung-cua-ham-so-bac-nhat-trong-tai-chinh-co-dap-an


\textbf{{QUESTION}}

Chi phí sử dụng truyền hình cáp của hai công ty dịch vụ truyền hình A và B như sau:
 
Công ty A
Công ty B
Chi phí lắp đặt ban đầu
150 000 đồng
Miễn phí
Cước hằng tháng
110 000 đồng
120 000 đồng
 
a) Viết công thức tính chi phí sử dụng truyền hình cáp y (nghìn đồng) của mỗi công ty A và B theo số tháng sử dụng là x (tháng).

\textbf{{ANSWER}}

a) Với số tháng sử dụng là x tháng. Ta có:
Công thức tính chi phí sử dụng truyền hình cáp công ty A là
y(A) = 110x + 150 (nghìn đồng).
Công thức tính chi phí sử dụng truyền hình cáp công ty B là
y(B) = 120x (nghìn đồng).

========================================================================

https://khoahoc.vietjack.com/thi-online/15-cau-trac-nghiem-toan-7-ket-noi-tri-thuc-bai-9-hai-duong-thang-song-song-va-dau-hieu-nhan-biet-co


\textbf{{QUESTION}}

Cho tia Ot nằm trong góc mOn, $\widehat {mOt} = \widehat {tOn}$ thì
A. Ot là tia phân giác của góc mOn;
B. Ot là tia nằm phía trong của góc mOn;
C. Ot là tia nằm phía ngoài của góc mOn;
D. Ot là tia nằm giữa hai cạnh Om và On.

\textbf{{ANSWER}}

Đáp án đúng là: A
Ot là tia nằm trong góc mOn và tạo với hai cạnh của góc đó hai góc $\widehat {mOt} = \widehat {tOn}$ nên Ot là tia phân giác của góc mOn.
Do đó chọn đáp án A.

========================================================================

https://khoahoc.vietjack.com/thi-online/15-cau-trac-nghiem-toan-7-ket-noi-tri-thuc-bai-9-hai-duong-thang-song-song-va-dau-hieu-nhan-biet-co


\textbf{{QUESTION}}

Nếu đường thẳng z cắt hai đường thẳng x, y và x // y thì ta có :
A. Hai góc bù nhau bằng nhau;
B. Hai góc kề nhau bằng nhau;
C. Hai góc đồng vị bằng nhau;
D. Hai góc kề bù bằng nhau.

\textbf{{ANSWER}}

Đáp án đúng là: C
Nếu một đường thẳng cắt hai đường thẳng song song thì:
Hai góc đồng vị bằng nhau;
Hai góc so le trong bằng nhau.

========================================================================

https://khoahoc.vietjack.com/thi-online/bai-tap-chuyen-de-toan-6-dang-1-cach-ghi-so-tu-nhien-co-dap-an/107182


\textbf{{QUESTION}}

Có bao nhiêu số có năm chữ số?

\textbf{{ANSWER}}

Số lớn nhất có năm chữ số là: 99 999. Số nhỏ nhất có năm chữ số là :10 000. Số các số có năm chữ số là : $$ 99\text{ }999\text{ - }10\text{ }000\text{ }+\text{ }1\text{ }=\text{ }90\text{ }000$$ .

========================================================================

https://khoahoc.vietjack.com/thi-online/bai-tap-chuyen-de-toan-6-dang-1-cach-ghi-so-tu-nhien-co-dap-an/107182


\textbf{{QUESTION}}

Có bao nhiêu số có sáu chữ số ?

\textbf{{ANSWER}}

Số các số có sáu chữ số là : 900 000 số.

========================================================================

https://khoahoc.vietjack.com/thi-online/bai-tap-chuyen-de-toan-6-dang-1-cach-ghi-so-tu-nhien-co-dap-an/107182


\textbf{{QUESTION}}

Tính số các số tự nhiên chẵn có bốn chữ số.

\textbf{{ANSWER}}

Các số tự nhiên chẵn có bốn chữ số là 1000 ; 1002 ; 1004 ; … ; 9998, trong đó số lớn nhất (số cuối) là 9998, số nhỏ nhất (số đầu) là 1000, khoảng cách giữa hai số liên tiếp là :
1002 – 1000 = 1004 – 1002 = … = 2   . 
Theo công thức nêu trên, số các số tự nhiên chẵn có bốn chữ số là :  (9998–1000)2+1=4500   (số)

========================================================================

https://khoahoc.vietjack.com/thi-online/bai-tap-chuyen-de-toan-6-dang-1-cach-ghi-so-tu-nhien-co-dap-an/107182


\textbf{{QUESTION}}

Muốn viết tất cả các số tự nhiên từ 100 đến 999 phải dùng bao nhiêu chữ số 9 ?

\textbf{{ANSWER}}

Ta lần lượt tính các chữ số 9 ở hàng đơn vị, ở hàng chục và ở hàng trăm. 
Các số chứa chữ số 9 ở hàng đơn vị: 109, 119, … , 999 gồm  999–10910+1 =90  (số)
Các số chứa chữ số 9 ở hàng chục: 
190, 191,… , 199 gồm  (số) 
290, 291 ,… , 299 gồm 10 số … 
990, 991,999 gồm 10 số. 
Các số chứa chữ số 9 ở hàng chục có : 10.9 = 90 (số) 
Các số chứa chữ số 9 ở hàng trăm : 900, 901,… , 999 gồm  (số) 
Vậy tất cả có: 90 + 90 + 100 = 280 (chữ số 9).

========================================================================

https://khoahoc.vietjack.com/thi-online/de-thi-giua-ki-1-toan-7-ctst-co-dap-an/106324


\textbf{{QUESTION}}

Trong các câu sau, câu nào đúng?
B. Số 0 là số hữu tỉ dương;
C. Số nguyên âm không phải là số hữu tỉ âm;

\textbf{{ANSWER}}

Đáp án đúng là: A
Số hữu tỉ âm nhỏ hơn số hữu tỉ dương. Đúng.
Số 0 là số hữu tỉ dương. Sai vì số 0 không là số hữu tỉ dương, cũng không là số hữu tỉ âm.
Số nguyên âm không phải là số hữu tỉ âm. Sai vì mỗi số nguyên là một số hữu tỉ.
Tập hợp ℚ gồm các số hữu tỉ dương và các số hữu tỉ âm. Sai vì tập hợp ℚ gồm các số hữu tỉ dương, số 0 và các số hữu tỉ âm.

========================================================================

https://khoahoc.vietjack.com/thi-online/de-thi-giua-ki-1-toan-7-ctst-co-dap-an/106324


\textbf{{QUESTION}}

Số đối của số hữu tỉ 94 là
B.-9-4;
C. 49;

\textbf{{ANSWER}}

Đáp án đúng là: A
Số đối của số hữu tỉ 94$$ \frac{9}{4}$$ là -94$$ -\frac{9}{4}$$.

========================================================================

https://khoahoc.vietjack.com/thi-online/de-thi-giua-ki-1-toan-7-ctst-co-dap-an/106324


\textbf{{QUESTION}}

Cho a = 2−9 và b = -13.
Khẳng định nào sau đây là đúng?
A. a = b;
B. a > b;

\textbf{{ANSWER}}

Đáp án đúng là: B
Ta có:
2−9=−29$$ \frac{2}{-9}=\frac{-2}{9}$$
−13=(−1).33.3=−39$$ \frac{-1}{3}=\frac{\left(-1\right).3}{3.3}=\frac{-3}{9}$$ (quy đồng mẫu số)
Vì ‒2 > ‒3 nên −29>−39$$ \frac{-2}{9}>\frac{-3}{9}$$
Hay 2−9$$ \frac{2}{-9}$$ > -13$$ \frac{-1}{3}$$.
Vậy 2−9$$ \frac{2}{-9}$$ > -13$$ \frac{-1}{3}$$.

========================================================================

https://khoahoc.vietjack.com/thi-online/10-bai-tap-phsan-loai-du-lieu-co-loi-giai


\textbf{{QUESTION}}

Cho các dữ liệu sau đây, dữ liệu nào là dữ liệu không là số?

\textbf{{ANSWER}}

Đáp án đúng là: C
Dữ liệu ở đáp án A, B và D là số nên là số liệu.
Dữ liệu ở đáp án C là dữ liệu không là số.

========================================================================

https://khoahoc.vietjack.com/thi-online/10-bai-tap-phsan-loai-du-lieu-co-loi-giai


\textbf{{QUESTION}}

Cho các dữ liệu sau đây, dữ liệu nào là số liệu?

\textbf{{ANSWER}}

Đáp án đúng là: D
Dữ liệu ở đáp án A, B và C là dữ liệu không là số.
Dữ liệu ở đáp án D là số liệu.

========================================================================

https://khoahoc.vietjack.com/thi-online/10-bai-tap-phsan-loai-du-lieu-co-loi-giai


\textbf{{QUESTION}}

Cho các dãy số liệu sau về 4 bạn Huy, Phương, Nhi và Hy, dãy số liệu nào là số liệu liên tục?

\textbf{{ANSWER}}

Đáp án đúng là: A
Dữ liệu ở đáp án A là số liệu liên tục (số liệu thu được từ phép đo thời gian).
Dữ liệu ở đáp án B, C và D là số liệu rời rạc (số liệu đếm số phần tử của tập các hoạt động tình nguyện đã tham gia, tập các thành viên trong gia đình, tập các chữ cái trong tên.

========================================================================

https://khoahoc.vietjack.com/thi-online/10-bai-tap-phsan-loai-du-lieu-co-loi-giai


\textbf{{QUESTION}}

Dữ liệu nào sau đây là số liệu liên tục?

\textbf{{ANSWER}}

Đáp án đúng là: C
Dữ liệu ở đáp án A là dữ liệu không là số.
Dữ liệu ở đáp án C là số liệu liên tục (số liệu thu được từ phép đo cân nặng).
Dữ liệu ở đáp án B, D là số liệu rời rạc.

========================================================================

https://khoahoc.vietjack.com/thi-online/10-bai-tap-phsan-loai-du-lieu-co-loi-giai


\textbf{{QUESTION}}

Khẳng định nào sau đây là đúng?

\textbf{{ANSWER}}

Đáp án đúng là: C
Dữ liệu ở đáp án A là dữ liệu không là số.
Dữ liệu ở đáp án B là dữ liệu không là số, không thể sắp thứ tự.
Dữ liệu ở đáp án C là số liệu rời rạc.
Dữ liệu ở đáp án D là số liệu liên tục.

========================================================================

https://khoahoc.vietjack.com/thi-online/9-cau-trac-nghiem-toan-9-bai-7-bien-doi-don-gian-bieu-thuc-chua-can-co-dap-an-tiep-thong-hieu


\textbf{{QUESTION}}

Trục căn thức ở mẫu biểu thức $$ \frac{2a}{2-\sqrt{a}}$$ với $$ a\ge 0;\quad a\ne 4$$ ta được
A. $$ \frac{-2a\sqrt{a}+4a}{4-a}$$
B. $$ \frac{2a\sqrt{a}-4a}{4-a}$$
C. $$ \frac{2a\sqrt{a}+4a}{4-a}$$
D. $$ -\frac{2a\sqrt{a}+4a}{4-a}$$

\textbf{{ANSWER}}

Ta có: $$ \frac{2a}{2-\sqrt{a}}=\frac{2a\left(2+\sqrt{a}\right)}{\left(2-\sqrt{a}\right)\left(2+\sqrt{a}\right)}=\frac{2a\sqrt{a}+4a}{4-a}$$
Đáp án cần chọn là: C

========================================================================

https://khoahoc.vietjack.com/thi-online/9-cau-trac-nghiem-toan-9-bai-7-bien-doi-don-gian-bieu-thuc-chua-can-co-dap-an-tiep-thong-hieu


\textbf{{QUESTION}}

Trục căn thức ở mẫu biểu thức 36+√3a$$ \frac{3}{6+\sqrt{3a}}$$với a≥0; a≠12$$ a\ge 0;\quad a\ne 12$$ ta được:
A. 6+√3a12+a$$ \frac{6+\sqrt{3a}}{12+a}$$
B. 6−√3a12+a$$ \frac{6-\sqrt{3a}}{12+a}$$
C. 6+√3a12−a$$ \frac{6+\sqrt{3a}}{12-a}$$
D. 6−√3a12−a$$ \frac{6-\sqrt{3a}}{12-a}$$

\textbf{{ANSWER}}

Ta có: 36+√3a=3(6−√3a)(6+√3a)(6−√3a)=3(6−√3a)62−(√3a)2=3(6−√3a)36−3a=6−√3a12−a$$ \frac{3}{6+\sqrt{3a}}=\frac{3\left(6-\sqrt{3a}\right)}{\left(6+\sqrt{3a}\right)\left(6-\sqrt{3a}\right)}=\frac{3\left(6-\sqrt{3a}\right)}{{6}^{2}-{\left(\sqrt{3a}\right)}^{2}}=\frac{3\left(6-\sqrt{3a}\right)}{36-3a}=\frac{6-\sqrt{3a}}{12-a}$$
Đáp án cần chọn là: D

========================================================================

https://khoahoc.vietjack.com/thi-online/9-cau-trac-nghiem-toan-9-bai-7-bien-doi-don-gian-bieu-thuc-chua-can-co-dap-an-tiep-thong-hieu


\textbf{{QUESTION}}

Trục căn thức ở mẫu biểu thức 6√x+√2y với x≥0; y≥0 ta được:
A. 6(√x−√2y)x−4y
B. 6(√x+√2y)x−2y
C. 6(√x−√2y)x−2y
D. 6(√x+√2y)x+2y

\textbf{{ANSWER}}

Ta có: 6√x+√2y=6(√x−√2y)(√x+√2y)(√x−√2y)=6(√x−√2y)x−2y$$ \frac{6}{\sqrt{x}+\sqrt{2y}}=\frac{6\left(\sqrt{x}-\sqrt{2y}\right)}{\left(\sqrt{x}+\sqrt{2y}\right)\left(\sqrt{x}-\sqrt{2y}\right)}=\frac{6\left(\sqrt{x}-\sqrt{2y}\right)}{x-2y}$$
Đáp án cần chọn là: C

========================================================================

https://khoahoc.vietjack.com/thi-online/9-cau-trac-nghiem-toan-9-bai-7-bien-doi-don-gian-bieu-thuc-chua-can-co-dap-an-tiep-thong-hieu


\textbf{{QUESTION}}

Trục căn thức ở mẫu biểu thức 43√x+2√y$$ \frac{4}{3\sqrt{x}+2\sqrt{y}}$$ với x≥0; y≥0; x≠49y$$ x\ge 0;\quad y\ge 0;\quad x\ne \frac{4}{9}y$$ ta được:
A. 3√x−2√y9x−4y$$ \frac{3\sqrt{x}-2\sqrt{y}}{9x-4y}$$
B. 12√x−8√y3x+2y$$ \frac{12\sqrt{x}-8\sqrt{y}}{3x+2y}$$
C. 12√x+8√y9x+4y$$ \frac{12\sqrt{x}+8\sqrt{y}}{9x+4y}$$
D. 12√x−8√y9x−4y$$ \frac{12\sqrt{x}-8\sqrt{y}}{9x-4y}$$

\textbf{{ANSWER}}

Ta có: $$ \frac{4}{3\sqrt{x}+2\sqrt{y}}=\frac{4\left(3\sqrt{x}-2\sqrt{y}\right)}{\left(3\sqrt{x}+2\sqrt{y}\right)\left(3\sqrt{x}-2\sqrt{y}\right)}=\frac{4\left(3\sqrt{x}-2\sqrt{y}\right)}{{\left(3\sqrt{x}\right)}^{2}-{\left(2\sqrt{y}\right)}^{2}}=\frac{12\sqrt{x}-8\sqrt{y}}{9x-4y}$$
Đáp án cần chọn là: D

========================================================================

https://khoahoc.vietjack.com/thi-online/9-cau-trac-nghiem-toan-9-bai-7-bien-doi-don-gian-bieu-thuc-chua-can-co-dap-an-tiep-thong-hieu


\textbf{{QUESTION}}

Giá trị biểu thức 32√6+2√23−4√32là giá trị nào sau đây?
A. √66
B. √6
C. √62
D. √63

\textbf{{ANSWER}}

Ta có: 32√6+2√23−4√32=32√6+2√63−4√62=√6(32+23−42)=√66
Đáp án cần chọn là: A

========================================================================

https://khoahoc.vietjack.com/thi-online/15-cau-trac-nghiem-toan-8-canh-dieu-bai-2-cac-phep-tinh-voi-da-thuc-nhieu-bien-co-dap-an


\textbf{{QUESTION}}

Tích của đa thức 6xy và đa thức $$ 2{x}^{2}-3y$$ là đa thức
A. $$ 12{x}^{2}y+18x{y}^{2}$$

\textbf{{ANSWER}}

Đáp án đúng là: B
Ta có: $$ 6xy\left(2{x}^{2}-3y\right)=12{x}^{3}y-18x{y}^{2}$$.

========================================================================

https://khoahoc.vietjack.com/thi-online/10-bai-tap-giai-cac-basi-toan-thuc-tien-bang-cach-lap-phuong-trinh-co-loi-giai


\textbf{{QUESTION}}

Mẹ hơn con 24 tuổi. Sau 2 năm nữa thì tuổi mẹ gấp 3 lần tuổi con. Tuổi của con hiện nay là
C. 15;

\textbf{{ANSWER}}

Đáp án đúng là: B
Gọi số tuổi của con hiện tại là x (tuổi) (x ∈ ℕ)
Suy ra số tuổi của mẹ là x + 24 (tuổi)
Theo bài ra ta có phương trình:  3(x + 2) = x + 24 + 2
3x + 6 = x + 26
2x – 20 = 0
x = 10
Vậy hiện tại tuổi của con là 10 tuổi.

========================================================================

https://khoahoc.vietjack.com/thi-online/10-bai-tap-giai-cac-basi-toan-thuc-tien-bang-cach-lap-phuong-trinh-co-loi-giai


\textbf{{QUESTION}}

Một hình chữ nhật có chiều dài hơn chiều rộng 3 cm. Chu vi hình chữ nhật là 100 cm. Chiều rộng hình chữ nhật là

\textbf{{ANSWER}}

Đáp án đúng là: A
Gọi chiều rộng hình chữ nhật là x (cm) (x > 0)
Suy ra chiều dài hình chữ nhật là x + 3 (cm)
Do chu vi hình chữ nhật là 100cm nên ta có:
2[x + (x + 3)] = 100 
2x + 3 = 50 
x = 23,5 (TMĐK)
Vậy chiều rộng hình chữ nhật là 23,5 cm.

========================================================================

https://khoahoc.vietjack.com/thi-online/10-bai-tap-giai-cac-basi-toan-thuc-tien-bang-cach-lap-phuong-trinh-co-loi-giai


\textbf{{QUESTION}}

Học kỳ một, số học sinh giỏi của lớp 8A chiếm  18 học sinh cả lớp. Sang kỳ hai, lớp 8A có thêm 3 học sinh giỏi nữa và lúc này số học sinh giỏi chiếm  15 học sinh cả lớp. Số học sinh lớp 8A là

\textbf{{ANSWER}}

Đáp án đúng là: B
Gọi số học sinh cả lớp là x (x ∈ ℕ*)
Vì học kỳ một số học sinh giỏi chiếm  18 học sinh cả lớp nên số học sinh giỏi kỳ một là  18x(học sinh)
Vì học kỳ hai có thêm 3 học sinh giỏi nữa nên số học sinh giỏi kỳ hai là  18x+3 (học sinh)
Mặt khác số học sinh giỏi kỳ hai bằng  15 số học sinh cả lớp nên số học sinh giỏi kỳ hai là  15x=18x+3 (học sinh) 
Theo đề bài, ta có phương trình:  15x=18x+3
Giải phương trình: 
 15x=18x+315x−18x=3(15−18)x=3
340x=3
                              x = 40 (TM)
Vậy số học sinh lớp 8A là 40 học sinh.

========================================================================

https://khoahoc.vietjack.com/thi-online/25-cau-trac-nghiem-toan-8-on-tap-chuong-2-da-giac-dien-tich-da-giac-co-dap-an


\textbf{{QUESTION}}

Đa giác đều là đa giác
A. Có tất cả các cạnh bằng nhau
B. Có tất cả các góc bằng nhau
C. Có tất cả các cạnh bằng nhau và các góc bằng nhau
D. Cả ba câu trên đều đúng

\textbf{{ANSWER}}

Theo định nghĩa: Đa giác đều là đa giác có tất cả các cạnh bằng nhau và các góc bằng nhau.
Đáp án cần chọn là: C

========================================================================

https://khoahoc.vietjack.com/thi-online/25-cau-trac-nghiem-toan-8-on-tap-chuong-2-da-giac-dien-tich-da-giac-co-dap-an


\textbf{{QUESTION}}

Hãy chọn câu đúng
A. Diện tích tam giác vuông bằng nửa tích hai cạnh góc vuông
B. Diện tích hình chữ nhật bằng nửa tích hai kích thước của nó
C. Diện tích hình vuông có cạnh a là 2a
D. Tất cả các đáp án trên đều đúng

\textbf{{ANSWER}}

Diện tích hình chữ nhật bằng tích hai kích thước của nó
Diện tích hình vuông có cạnh a là a2
Diện tích tam giác vuông bằng nửa tích hai cạnh góc vuông của tam giác vuông đó.
Đáp án cần chọn là: A

========================================================================

https://khoahoc.vietjack.com/thi-online/25-cau-trac-nghiem-toan-8-on-tap-chuong-2-da-giac-dien-tich-da-giac-co-dap-an


\textbf{{QUESTION}}

Một đa giác lồi 10 cạnh thì có số đường chéo là
A. 35
B. 30
C. 70
D. 27

\textbf{{ANSWER}}

Số đường chéo của hình 10 cạnh là:
10(10−3)2=35$$ \frac{10(10-3)}{2}=35$$ đường
Đáp án cần chọn là: A

========================================================================

https://khoahoc.vietjack.com/thi-online/25-cau-trac-nghiem-toan-8-on-tap-chuong-2-da-giac-dien-tich-da-giac-co-dap-an


\textbf{{QUESTION}}

Số đo mỗi góc của hình 9 cạnh đều là
A. 1200
B. 600
C. 1400
D. 1350

\textbf{{ANSWER}}

Số đo góc của đa giác đều 9 cạnh:
(9−2).18009=1400
Đáp án cần chọn là: C

========================================================================

https://khoahoc.vietjack.com/thi-online/25-cau-trac-nghiem-toan-8-on-tap-chuong-2-da-giac-dien-tich-da-giac-co-dap-an


\textbf{{QUESTION}}

Một tam giác có độ dài ba cạnh là 12cm, 5cm, 13cm. Diện tích tam giác đó là
A. 60cm2
B. 30cm2
C. 45cm2
D. 32,5cm2

\textbf{{ANSWER}}

Ta có: 52 + 122 = 169; 132 = 169
=> 52 + 122 = 132
Do đó đây tam giác đã cho là tam giác vuông có hai cạnh góc vuông là 5cm và 12cm.
Diện tích của nó là:12.12.5=30 (cm2)
Đáp án cần chọn là: B

========================================================================

https://khoahoc.vietjack.com/thi-online/12-bai-taps-tinh-gia-tri-va-rut-gon-bieu-thuc-luong-giac-co-loi-giai


\textbf{{QUESTION}}

Tính $A = \sin 60^\circ + \cos 150^\circ - \cot 45^\circ $.

\textbf{{ANSWER}}

Hướng dẫn giải:
Áp dụng bảng giá trị lượng giác của một số góc đặc biệt, ta có:
$A = 4\sin 60^\circ + 3\cos 150^\circ - \cot 45^\circ = 4.\frac{{\sqrt 3 }}{2} + 3.\left( { - \frac{{\sqrt 3 }}{2}} \right) - 1 = \frac{{\sqrt 3 - 2}}{2}$.

========================================================================

https://khoahoc.vietjack.com/thi-online/12-bai-taps-tinh-gia-tri-va-rut-gon-bieu-thuc-luong-giac-co-loi-giai


\textbf{{QUESTION}}

Tính giá trị của biểu thức 
B=cos0∘+cos20∘+cos40∘+...+cos160∘+cos180∘.

\textbf{{ANSWER}}

Hướng dẫn giải:
Ta có:
B=cos0∘+cos20∘+cos40∘+...+cos160∘+cos180∘$B = \cos 0^\circ + \cos 20^\circ + \cos 40^\circ + ... + \cos 160^\circ + \cos 180^\circ $
=(cos0∘+cos180∘)+(cos20∘+cos160∘)+...+(cos80∘+cos100∘)$ = \left( {\cos 0^\circ + \cos 180^\circ } \right) + \left( {\cos 20^\circ + \cos 160^\circ } \right) + ... + \left( {\cos 80^\circ + \cos 100^\circ } \right)$
=(cos0∘+cos(180∘−0∘))+(cos20∘+cos(180∘−20∘))+...+(cos80∘+cos(180∘−80∘))$ = \left( {\cos 0^\circ + \cos \left( {180^\circ - 0^\circ } \right)} \right) + \left( {\cos 20^\circ + \cos \left( {180^\circ - 20^\circ } \right)} \right) + ... + \left( {\cos 80^\circ + \cos \left( {180^\circ - 80^\circ } \right)} \right)$
=(cos0∘−cos0∘)+(cos20∘−cos20∘)+...+(cos80∘−cos80∘)$ = \left( {\cos 0^\circ - \cos 0^\circ } \right) + \left( {\cos 20^\circ - \cos 20^\circ } \right) + ... + \left( {\cos 80^\circ - \cos 80^\circ } \right)$ 
= 0

========================================================================

https://khoahoc.vietjack.com/thi-online/12-bai-taps-tinh-gia-tri-va-rut-gon-bieu-thuc-luong-giac-co-loi-giai


\textbf{{QUESTION}}

Tính giá trị biểu thức sau: A=asin90∘+bcos90∘+ccos180∘.
A=asin90∘+bcos90∘+ccos180∘
A=asin90∘+bcos90∘+ccos180∘
A
=
a
sin

90∘
90
∘
+
b
cos

90∘
90
∘
+
c
cos

180∘
180
∘
A. a – b;
B. a + b – c;
C. a – b + c;
D. a − c.

\textbf{{ANSWER}}

Hướng dẫn giải:
Đáp án đúng là: D.
Áp dụng bảng giá trị lượng giác của một số góc đặc biệt, ta có:
A=asin90∘+bcos90∘+ccos180∘
= a . 1 + b . 0 + c . (– 1) = a – c.

========================================================================

https://khoahoc.vietjack.com/thi-online/12-bai-taps-tinh-gia-tri-va-rut-gon-bieu-thuc-luong-giac-co-loi-giai


\textbf{{QUESTION}}

Kết quả của phép tính B=5−sin290∘+2cos260∘−3tan245∘ là:
B=5−sin290∘+2cos260∘−3tan245∘
B=5−sin290∘+2cos260∘−3tan245∘
B
=
5
−
sin2
sin2
sin
2
90∘
90
∘
+
2
cos2
cos2
cos
2
60∘
60
∘
−
3
tan2
tan2
tan
2
45∘
45
∘
A. 1;
B. 2;
C. 3;
D. 4.

\textbf{{ANSWER}}

Hướng dẫn giải:
Đáp án đúng là: C.
Ta có: B=5−sin290∘+2cos260∘−3tan245∘
=5−(sin90∘)2+2(cos60∘)2−3(tan45∘)2
Áp dụng bảng giá trị lượng giác của một số góc đặc biệt, ta có:
B=5−12+2.(12)2−3.(√22)2=5−1+12−32=3.

========================================================================

https://khoahoc.vietjack.com/thi-online/12-bai-taps-tinh-gia-tri-va-rut-gon-bieu-thuc-luong-giac-co-loi-giai


\textbf{{QUESTION}}

Rút gọn biểu thức C=sin45∘+3cos60∘−4tan30∘+5cot120∘+6sin135∘ ta được kết quả là

\textbf{{ANSWER}}

Hướng dẫn giải:
Đáp án đúng là A. 
Áp dụng bảng giá trị lượng giác của một số góc đặc biệt, ta có:
C=sin45∘+3cos60∘−4tan30∘+5cot120∘+6sin135∘
=√22+3.12−4.√33−5.√33+6.√22=32+7√22−3√3.

========================================================================

https://khoahoc.vietjack.com/thi-online/5920-cau-trac-nghiem-tong-hop-mon-toan-2023-co-dap-an


\textbf{{QUESTION}}

Cho tam giác ABC, M, N, P được xác định bởi véctơ $$ \overrightarrow{MA}=-\frac{3}{4}\overrightarrow{BM},\text{ }\overrightarrow{AN}=-3\overrightarrow{CN},\text{ }\overrightarrow{CP}=\frac{1}{4}\overrightarrow{PB}$$.
Chứng minh M, N, P thẳng hàng?

\textbf{{ANSWER}}

Ta có: $$ \overrightarrow{MA}=-\frac{3}{4}\overrightarrow{BM}=\frac{3}{4}\overrightarrow{MB};\overrightarrow{AN}=-3\overrightarrow{CN}=3\overrightarrow{NC};\overrightarrow{CP}=\frac{1}{4}\overrightarrow{PB}$$
Ta lại có: $$ \overrightarrow{MN}=\overrightarrow{MA}+\overrightarrow{AN}=\frac{3}{4}\overrightarrow{MB}+3\overrightarrow{NC}=\frac{3}{4}\overrightarrow{MP}+\frac{3}{4}\overrightarrow{PB}+3\overrightarrow{NP}+3\overrightarrow{PC}$$
$$ \Leftrightarrow \overrightarrow{MN}=\frac{3}{4}\overrightarrow{MP}+3\overrightarrow{CP}+3\overrightarrow{NP}-3\overrightarrow{CP}=\frac{3}{4}\overrightarrow{MP}+3\overrightarrow{NP}$$
$$ \Leftrightarrow \overrightarrow{MP}+\overrightarrow{PN}=\frac{3}{4}\overrightarrow{MP}+3\overrightarrow{NP}$$
$$ \Leftrightarrow \frac{1}{4}\overrightarrow{MP}=4\overrightarrow{NP}\Leftrightarrow \overrightarrow{MP}=16\overrightarrow{NP}$$
Do đó M, N, P thằng hàng.

========================================================================

https://khoahoc.vietjack.com/thi-online/5920-cau-trac-nghiem-tong-hop-mon-toan-2023-co-dap-an


\textbf{{QUESTION}}

Cho a,b ≠ -2 thỏa mãn (2a + 1) (2b + 1) = 9
Tính giá trị biểu thức M=12+a+12+b$$ M=\frac{1}{2+a}+\frac{1}{2+b}$$.

\textbf{{ANSWER}}

Ta có :
2b+1=92a+1⇒b=4-a2a+1⇒b+2=4-a2a+1+2=3a+62a+1$$ 2b+1=\frac{9}{2a+1}\Rightarrow b=\frac{4-a}{2a+1}\Rightarrow b+2=\frac{4-a}{2a+1}+2=\frac{3a+6}{2a+1}$$
M=1a+2+2a+13a+6=1a+2+2a+13(a+2)=3+2a+13(a+2)=2(a+2)3(a+2)=23$$ M=\frac{1}{a+2}+\frac{2a+1}{3a+6}=\frac{1}{a+2}+\frac{2a+1}{3\left(a+2\right)}=\frac{3+2a+1}{3\left(a+2\right)}=\frac{2\left(a+2\right)}{3\left(a+2\right)}=\frac{2}{3}$$

========================================================================

https://khoahoc.vietjack.com/thi-online/5920-cau-trac-nghiem-tong-hop-mon-toan-2023-co-dap-an


\textbf{{QUESTION}}

Chứng minh các bất đẳng thức: 1a+ 1b≥4a+b với a > 0, b > 0

\textbf{{ANSWER}}

Xét hiệu
1a+ 1b−4a+b=b(a+b)+a(a+b)−4abab(a+b)=a2−2ab+b2ab(a+b)=(a−b)2ab(a+b)≥0, vì a, b > 0
Xảy ra đẳng thức khi và chỉ khi a = b.

========================================================================

https://khoahoc.vietjack.com/thi-online/5920-cau-trac-nghiem-tong-hop-mon-toan-2023-co-dap-an


\textbf{{QUESTION}}

Cho a lớn hơn 0 và b lớn hơn 0. Chứng minh rằng  (1a+1b)(a+b)≥4

\textbf{{ANSWER}}

(1a+1b)(a+b)≥4
⇔1+ba+ab+1≥4
⇔b2+a2ab≥2
Vì a > 0 và b > 0  ⇒ ab > 0
Vậy b2+a2ab≥2⇔b2+a2≥2ab
⇔(a−b)2≥0.
Vậy bất đẳng thức được chứng minh.

========================================================================

https://khoahoc.vietjack.com/thi-online/10-bai-tap-gioi-han-cua-ham-so-tai-mot-diesm-va-tai-vo-cuc-co-loi-giai


\textbf{{QUESTION}}

Giới hạn  $$ \underset{x\to 3}{\mathrm{lim}}\left({x}^{2}+x+5\right)$$ bằng

\textbf{{ANSWER}}

Đáp án đúng là: A
 $$ \underset{x\to 3}{\mathrm{lim}}\left({x}^{2}+x+5\right)={3}^{2}+3+5=17$$.

========================================================================

https://khoahoc.vietjack.com/thi-online/10-bai-tap-gioi-han-cua-ham-so-tai-mot-diesm-va-tai-vo-cuc-co-loi-giai


\textbf{{QUESTION}}

Giới hạn  limx→23x−5(x−2)2$$ \underset{x\to 2}{\mathrm{lim}}\frac{3x-5}{{\left(x-2\right)}^{2}}$$ bằng

\textbf{{ANSWER}}

Đáp án đúng là: C
 limx→23x−5(x−2)2$$ \underset{x\to 2}{\mathrm{lim}}\frac{3x-5}{{\left(x-2\right)}^{2}}$$ = +∞ vì  {limx→2(3x−5)=1>0limx→2(x−2)2=0(x−2)>0∀x≠2$$ \left\{\begin{array}{c}\begin{array}{c}\underset{x\to 2}{\mathrm{lim}}\left(3x-5\right)=1>0\\ \underset{x\to 2}{\mathrm{lim}}{\left(x-2\right)}^{2}=0\end{array}\\ \left(x-2\right)>0\forall x\ne 2\end{array}\right.$$

========================================================================

https://khoahoc.vietjack.com/thi-online/10-bai-tap-gioi-han-cua-ham-so-tai-mot-diesm-va-tai-vo-cuc-co-loi-giai


\textbf{{QUESTION}}

Tính giới hạn  limx→∞(x4+2x2+3), ta thu được kết quả nào sau đây?

\textbf{{ANSWER}}

Đáp án đúng là: C
 limx→∞(x4+2x2+3)=limx→∞x4(1+2x2+3x4)=+∞$$ \underset{x\to \infty }{\mathrm{lim}}\left({x}^{4}+2{x}^{2}+3\right)=\underset{x\to \infty }{\mathrm{lim}}{x}^{4}\left(1+\frac{2}{{x}^{2}}+\frac{3}{{x}^{4}}\right)=+\infty $$
Vì  {limx→∞x4=+∞limx→∞(1+2x2+3x4)=1>0$$ \left\{\begin{array}{c}\underset{x\to \infty }{\mathrm{lim}}{x}^{4}=+\infty \\ \underset{x\to \infty }{\mathrm{lim}}\left(1+\frac{2}{{x}^{2}}+\frac{3}{{x}^{4}}\right)=1>0\end{array}\right.$$

========================================================================

https://khoahoc.vietjack.com/thi-online/10-bai-tap-gioi-han-cua-ham-so-tai-mot-diesm-va-tai-vo-cuc-co-loi-giai


\textbf{{QUESTION}}

Giới hạn  limx→2x2−5x+6x2−6x+8 bằng
D. 12

\textbf{{ANSWER}}

Đáp án đúng là: D
 limx→2x2−5x+6x2−6x+8=limx→2(x−2)(x−3)(x−2)(x−4)=limx→2x−3x−4=12.

========================================================================

https://khoahoc.vietjack.com/thi-online/10-bai-tap-gioi-han-cua-ham-so-tai-mot-diesm-va-tai-vo-cuc-co-loi-giai


\textbf{{QUESTION}}

Tính giới hạn  limx→13√x−1x−1, ta thu được kết quả nào sau đây?
C. 12
D. 13

\textbf{{ANSWER}}

Đáp án đúng là: D
 limx→13√x−1x−1=limx→13√x−1(3√x−1)(3√x2+3√x+1)
 =limx→113√x2+3√x+1=13.

========================================================================

https://khoahoc.vietjack.com/thi-online/bai-tap-tap-hop-phan-tu-cua-tap-hop


\textbf{{QUESTION}}

Viết tập hợp các chữ cái trong từ "GIÁO VIÊN".

\textbf{{ANSWER}}

Hướng dẫn
A = { G, I, A, O, V, Ê, N}.

========================================================================

https://khoahoc.vietjack.com/thi-online/bai-tap-tap-hop-phan-tu-cua-tap-hop


\textbf{{QUESTION}}

Viết tập hợp các chữ cái trong từ "HỌC SINH".

\textbf{{ANSWER}}

Hướng dẫn
B = {H, O, C, S, I, N}.

========================================================================

https://khoahoc.vietjack.com/thi-online/bai-tap-tap-hop-phan-tu-cua-tap-hop


\textbf{{QUESTION}}

Viết tập hợp M các số tự nhiên lớn hơn 9 và nhỏ hơn 16 bằng hai cách.

\textbf{{ANSWER}}

Hướng dẫn
Cách 1. M = {10;11;12;13;14;15}
Cách 2. M={x∈N|9<x<16)$$ M=\{x\in N|9<x<16)$$.

========================================================================

https://khoahoc.vietjack.com/thi-online/bai-tap-tap-hop-phan-tu-cua-tap-hop


\textbf{{QUESTION}}

Viết tập hợp N các số tự nhiên lớn hơn 5 và nhỏ hơn 12 bằng hai cách.

\textbf{{ANSWER}}

Hướng dẫn
Cách 1. M = {6;7;8;9;10;11}
Cách 2. M={x∈N|5<x<12).

========================================================================

https://khoahoc.vietjack.com/thi-online/16-cau-trac-nghiem-toan-10-ket-noi-tri-thuc-menh-de-co-dap-an


\textbf{{QUESTION}}

Trong các câu sau, câu nào là mệnh đề?
A. Đi ngủ đi!
B. Trung Quốc là nước đông dân nhất thế giới. 
C. Bạn học trường nào? 
D. Không được làm việc riêng trong giờ học.

\textbf{{ANSWER}}

Đáp án đúng là: B
Chỉ có câu “Trung Quốc là nước đông dân nhất thế giới” có thể xác định được tính đúng sai nên đáp án B là mệnh đề.

========================================================================

https://khoahoc.vietjack.com/thi-online/16-cau-trac-nghiem-toan-10-ket-noi-tri-thuc-menh-de-co-dap-an


\textbf{{QUESTION}}

Trong các câu sau, câu nào không phải là mệnh đề?
A. Buồn ngủ quá!;
B. Hình thoi có hai đường chéo vuông góc với nhau;
C. 8 là số chính phương;
D. Băng Cốc là thủ đô của Mianma.

\textbf{{ANSWER}}

Đáp án đúng là: A
Đáp án A là câu cảm thán không xác định được tính đúng sai. Do đó không phải là mệnh đề.

========================================================================

https://khoahoc.vietjack.com/thi-online/16-cau-trac-nghiem-toan-10-ket-noi-tri-thuc-menh-de-co-dap-an


\textbf{{QUESTION}}

Trong các câu sau, có bao nhiêu câu là mệnh đề?
a) Hãy đi nhanh lên!
b) Hà Nội là thủ đô của Việt Nam.
c) 4 + 5 + 7 = 15.
d) Năm 2018 là năm nhuận.
A. 4;
B. 3;
C. 1;
D. 2.

\textbf{{ANSWER}}

Đáp án đúng là: B
Câu a) là câu cảm thán không xác định được tính đúng, sai nên câu a không phải là mệnh đề. 
Các câu b), c), d) đều có thể xác định được tính đúng sai. Do đó các câu b), c), d) đều là mệnh đề.
Vậy có tất cả 3 mệnh đề.

========================================================================

https://khoahoc.vietjack.com/thi-online/16-cau-trac-nghiem-toan-10-ket-noi-tri-thuc-menh-de-co-dap-an


\textbf{{QUESTION}}

Câu nào sau đây không là mệnh đề?
A. x > 2;
B. 3 < 1;
C. 4 – 5 = 1;
D. Tam giác đều là tam giác có ba cạnh bằng nhau.

\textbf{{ANSWER}}

Đáp án đúng là: A
Vì x > 2 là mệnh đề chứa biến không xác định được tính đúng sai nên không phải mệnh đề.

========================================================================

https://khoahoc.vietjack.com/thi-online/16-cau-trac-nghiem-toan-10-ket-noi-tri-thuc-menh-de-co-dap-an


\textbf{{QUESTION}}

Trong các mệnh đề sau, mệnh đề nào là mệnh đề đúng?
A. Tổng của hai số tự nhiên là một số chẵn khi và chỉ khi cả hai số đều là số chẵn;
B. Tích của hai số tự nhiên là một số chẵn khi và chỉ khi cả hai số đều là số chẵn;
C. Tổng của hai số tự nhiên là một số lẻ khi và chỉ khi cả hai số đều là số lẻ;
D. Tích của hai số tự nhiên là một số lẻ khi và chỉ khi cả hai số đều là số lẻ.

\textbf{{ANSWER}}

Đáp án đúng là: D
A là mệnh đề sai: Ví dụ: 1 + 3 = 4 là số chẵn nhưng 1, 3 là số lẻ.
B là mệnh đề sai: Ví dụ: 2.3 = 6 là số chẵn nhưng 3 là số lẻ.
C là mệnh đề sai: Ví dụ: 1 + 3 = 4 là số chẵn nhưng 1, 3 là số lẻ.

========================================================================

https://khoahoc.vietjack.com/thi-online/34-cau-trac-nghem-toan-6-bai-15-tim-mot-so-biet-gia-tri-mot-phan-so-cua-no-co-dap-an


\textbf{{QUESTION}}

Khánh có 45 cái kẹo. Khánh cho Linh $$ \frac{2}{3}$$  số kẹo đó. Hỏi Khánh cho Linh bao nhiêu cái kẹo?
A. 30 cái kẹo
B. 36 cái kẹo  
C. 40 cái kẹo          
D. 18 cái kẹo.

\textbf{{ANSWER}}

Đáp án A
Khánh cho Linh số kẹo là:
$$ 45.\frac{2}{3}=30$$ (cái kẹo)
Vậy Khánh cho Linh 30 cái kẹo.

========================================================================

https://khoahoc.vietjack.com/thi-online/34-cau-trac-nghem-toan-6-bai-15-tim-mot-so-biet-gia-tri-mot-phan-so-cua-no-co-dap-an


\textbf{{QUESTION}}

Một tổ công nhân có 42 người, số nữ chiếm 23  tổng số. Hỏi tổ có bao nhiêu công nhân nữ?
A. 28
B. 21 
C. 20
D. 18

\textbf{{ANSWER}}

Đáp án A
Số công nhân nữ của tổ là: 42.23=28  (công nhân)

========================================================================

https://khoahoc.vietjack.com/thi-online/34-cau-trac-nghem-toan-6-bai-15-tim-mot-so-biet-gia-tri-mot-phan-so-cua-no-co-dap-an


\textbf{{QUESTION}}

Biết 35  số học sinh giỏi của lớp 6A là 12 học sinh. Hỏi lớp 6A có bao nhiêu học sinh giỏi?
A. 12 học sinh giỏi 
B. 15 học sinh giỏi 
C. 14 học sinh giỏi 
D. 20 học sinh giỏi

\textbf{{ANSWER}}

Đáp án D 
Lớp 6A có số học sinh giỏi là:
12:352=20 (học sinh giỏi)
Vậy lớp 6A có 20 học sinh giỏi.

========================================================================

https://khoahoc.vietjack.com/thi-online/34-cau-trac-nghem-toan-6-bai-15-tim-mot-so-biet-gia-tri-mot-phan-so-cua-no-co-dap-an


\textbf{{QUESTION}}

Biết 25  số vở Lan có là 20 quyển. Hỏi Lan có bao nhiêu quyển vở?
A. 34
B. 25
C. 45
D. 50

\textbf{{ANSWER}}

Đáp án D 
Lan có số quyển vở là:  20:25=50  (quyển vở).

========================================================================

https://khoahoc.vietjack.com/thi-online/34-cau-trac-nghem-toan-6-bai-15-tim-mot-so-biet-gia-tri-mot-phan-so-cua-no-co-dap-an


\textbf{{QUESTION}}

Một lớp học có 30 học sinh, trong đó có 6 em học giỏi toán. Hãy tính tỉ số phần trăm của số học sinh giỏi toán so với số học sinh cả lớp?
A. 25%         
B. 35%
C. 20% 
D. 40%

\textbf{{ANSWER}}

Đáp án C
Tỉ số phần trăm của số học sinh giỏi toán so với số học sinh cả lớp là:
6.10030%=20%
 
Vậy số học sinh giỏi Toán chiếm 20% số học sinh cả lớp.

========================================================================

https://khoahoc.vietjack.com/thi-online/bai-tap-phuong-trinh-dua-duoc-ve-dang-ax-b-0-co-loi-giai-chi-tiet


\textbf{{QUESTION}}

Nghiệm của phương trình 4( x - 1 ) - ( x + 2 ) = - x là? 
A. x = 2.   
B. x = 3/2. 
C. x = 1.   
D. x = - 1.

\textbf{{ANSWER}}

Ta có: 4( x - 1 ) - ( x + 2 ) = - x
⇔ 4x - 4 - x - 2 = - x
⇔ 4x - x + x = 2 + 4 ⇔ 4x = 6 ⇔ x = 3/2.
Vậy phương trình đã cho có nghiệm là x = 3/2.
Chọn đáp án B.

========================================================================

https://khoahoc.vietjack.com/thi-online/giai-sbt-toan-10-bai-21-duong-tron-trong-mat-phang-toa-do-co-dap-an


\textbf{{QUESTION}}

Tìm tâm và bán kính của đường tròn (C) trong các trường hợp sau:
 (x – 2)2 + (y – 8)2 = 49;

\textbf{{ANSWER}}

Hướng dẫn giải
Phương trình đường tròn có dạng: (x – a)2 + (y – b)2 = R2
Với (a; b) là tọa độ tâm I và R > 0 là bán kính của đường tròn
Xét (x – 2)2 + (y – 8)2 = 49 có: 
a = 2, b = 8, R2 = 49 ⇒ R = 7 
Vậy đường tròn (C) có tâm I(2; 8) và bán kính R = 7.

========================================================================

https://khoahoc.vietjack.com/thi-online/giai-sbt-toan-10-bai-21-duong-tron-trong-mat-phang-toa-do-co-dap-an


\textbf{{QUESTION}}

(x + 3)2 + (y – 4)2 = 23.

\textbf{{ANSWER}}

Hướng dẫn giải
Phương trình đường tròn có dạng: (x – a)2 + (y – b)2 = R2
Với (a; b) là tọa độ tâm I và R > 0 là bán kính của đường tròn
Xét(x + 3)2 + (y – 4)2 = 23 có: 
a = –3, b = 4, R2 = 23 ⇒ R = √23 
Vậy đường tròn (C) có tâm I(–3; 4) và bán kính R = √23.

========================================================================

https://khoahoc.vietjack.com/thi-online/giai-sbt-toan-10-bai-21-duong-tron-trong-mat-phang-toa-do-co-dap-an


\textbf{{QUESTION}}

Phương trình nào dưới đây là phương trình của một đường tròn? Khi đó hãy tìm tâm và bán kính của nó.
x2 + 2y2 – 4x – 2y + 1 = 0.

\textbf{{ANSWER}}

Hướng dẫn giải
Phương trình đã cho không là phương trình của đường tròn do hệ số của x2 và y2 không bằng nhau

========================================================================

https://khoahoc.vietjack.com/thi-online/giai-sbt-toan-10-bai-21-duong-tron-trong-mat-phang-toa-do-co-dap-an


\textbf{{QUESTION}}

x2 + y2 – 4x + 3y + 2xy = 0.

\textbf{{ANSWER}}

Hướng dẫn giải
Phương trình đã cho không là phương trình của đường tròn do trong phương trình của đường tròn không có thành phần tích xy.

========================================================================

https://khoahoc.vietjack.com/thi-online/giai-sbt-toan-10-bai-21-duong-tron-trong-mat-phang-toa-do-co-dap-an


\textbf{{QUESTION}}

x2 + y2 – 8x – 6y + 26 = 0.

\textbf{{ANSWER}}

Hướng dẫn giải
Phương trình đã cho có các hệ số a = 4, b = 3, c = 26, ta có:
a2 + b2 – c = 42 + 32 – 26 = –1 < 0 
do đó nó không là phương trình của đường tròn.

========================================================================

https://khoahoc.vietjack.com/thi-online/bai-tap-toan-7-chuong-1-on-tap-hoc-ki-i


\textbf{{QUESTION}}

Hai đường thẳng m và n vuông góc với nhau thì tạo thành
A. một góc vuông.
B. hai góc vuông.
C. ba góc vuông.
D. bốn góc vuông

\textbf{{ANSWER}}

Đáp án là D

========================================================================

https://khoahoc.vietjack.com/thi-online/bai-tap-toan-7-chuong-1-on-tap-hoc-ki-i


\textbf{{QUESTION}}

Cho ba đường thẳng a , b , c . Câu nào sau đây sai
A. Nếu a // b , b // c thì a // c.
B. Nếu a ⊥ b , b // c thì a ^ c.
C. Nếu a ⊥ b , b ⊥c thì a ⊥ c
D. Nếu a ⊥ b , b ⊥ c thì a // c

\textbf{{ANSWER}}

Đáp án là B

========================================================================

https://khoahoc.vietjack.com/thi-online/30-cau-trac-nghiem-toan-7-chan-troi-sang-tao-bai-tap-cuoi-chuong-3-co-dap-an


\textbf{{QUESTION}}

Tính độ dài của một chiếc hộp hình lập phương, biết rằng diện tích sơn 4 mặt bên của hộp đó là 144 cm2.
A. 4 cm;       
B. 8 cm;           
C. 6 cm;          
D. 5 cm.

\textbf{{ANSWER}}

Đáp án đúng là: C
Gọi a (cm) là độ dài cạnh của chiếc hộp hình lập phương (a > 0).
Diện tích xung quanh của hộp hình lập phương là: 144 = 4 . a2.
Suy ra a2 = 36.
Do đó a = ± 6 mà a > 0 nên a = 6.
Vậy độ dài cạnh của hộp hình phương là 6 cm.

========================================================================

https://khoahoc.vietjack.com/thi-online/30-cau-trac-nghiem-toan-7-chan-troi-sang-tao-bai-tap-cuoi-chuong-3-co-dap-an


\textbf{{QUESTION}}

Một bể nước dạng hình hộp chữ nhật có kích thước các số đo trong lòng bể là: chiều dài 4 m, chiều rộng 3 m, chiều cao 2,5 m. Biết 34 bể đang chứa nước. Hỏi thể tích phần bể không chứa nước là bao nhiêu?
A. 30 m3;       
B. 22,5 m3;    
C. 7,5 m3;     
D. 5,7 m3.

\textbf{{ANSWER}}

Đáp án đúng là: C
Vì bể nước có dạng hình hộp chữ nhật nên ta tính được thể tích bể nước là:
V = 4 . 3 . 2,5 = 30 (m3)
Vì 34 bể đang chứa nước nên thể tích phần bể chứa nước là: 
Vchứa nước = 34V = 34 . 30 = 22,5 (m3)
Thể tích phần bể không chứa nước là: 
Vkhông chứa nước = V - Vchứa nước = 30 – 22,5 = 7,5 (m3)
Vậy thể tích phần bể không chứa nước là 7,5 m3.

========================================================================

https://khoahoc.vietjack.com/thi-online/30-cau-trac-nghiem-toan-7-chan-troi-sang-tao-bai-tap-cuoi-chuong-3-co-dap-an


\textbf{{QUESTION}}

Hình lập phương A có cạnh bằng 23cạnh hình lập phương B. Hỏi thể tích hình lập phương A bằng bao nhiêu phần thể tích hình lập phương B?
A. 29;
B. 278;
C. 827;
D. 49.

\textbf{{ANSWER}}

Đáp án đúng là: B
Gọi a là chiều dài một cạnh của hình lập phương A.
Vì hình lập phương A có cạnh bằng 23cạnh của hình lập phương B nên chiều dài một cạnh của hình lập phương B là 23a.
Thể tích hình lập phương A là: VA = a3.
Thể tích hình lập phương B là:
VB = (23a)3= (23)3.a3= 2333.a3= 278a3
Vậy thể tích hình lập phương A bằng 278 thể tích hình lập phương B.

========================================================================

https://khoahoc.vietjack.com/thi-online/30-cau-trac-nghiem-toan-7-chan-troi-sang-tao-bai-tap-cuoi-chuong-3-co-dap-an


\textbf{{QUESTION}}

Một chiếc hộp hình lập phương được sơn 4 mặt bên cả mặt trong và mặt ngoài. Diện tích phải sơn tổng cộng là 1 152 cm2. Tính thể tích của hình lập phương đó.
A. 1 782 cm3;  
B. 1 728 cm3;
C. 144 cm3;    
D. 1 827 cm3.

\textbf{{ANSWER}}

Đáp án đúng là: B
Chiếc hộp hình lập phương có 4 mặt bên đều là hình vuông, mỗi hình vuông được sơn 2 mặt nên diện tích mỗi hình vuông là: 1152 : 8 = 144 (cm2).
Gọi a (cm) là độ dài cạnh của hình vuông.
Diện tích của hình vuông là 144 = a2.
Suy ra a = ± 12 mà a > 0 nên a = 12 (cm).
Vậy cạnh của chiếc hộp hình lập phương là 12 cm.
Thể tích của chiếc hộp hình lập phương là:
123 = 1 728 (cm3).
Vậy thể tích chiếc hộp hình lập phương là 1 728 cm3.

========================================================================

https://khoahoc.vietjack.com/thi-online/de-on-luyen-thi-thpt-quoc-gia-mon-toan-cuc-hay-co-loi-giai-chi-tiet/23862


\textbf{{QUESTION}}

Trong mặt phẳng tọa độ Oxy, đường thẳng đi qua hai điểm $$ A\left(1;1\right)$$và $$ B\left(-3;5\right)$$ nhận vectơ nào sau đây làm vectơ chỉ phương?
A. $$ \overrightarrow{a}=\left(1;-1\right)$$
B. $$ \overrightarrow{b}=\left(1;1\right)$$
C. $$ \overrightarrow{c}=\left(-2;6\right)$$
D. $$ \overrightarrow{d}=\left(3;1\right)$$

\textbf{{ANSWER}}

Chọn đáp án A
Nếu $$ \overrightarrow{u}$$ là một vectơ chỉ phương của đường thẳng Δ thì $$ k\overrightarrow{u}\left(k\ne 0\right)$$ cũng là một vectơ chỉ phương.
Đường thẳng đi qua hai điểm A và B nhận vectơ $$ \overrightarrow{AB}=\left(-4;4\right)=-4\left(1;-1\right)=-4\overrightarrow{a}$$ làm một vectơ chỉ phương nên vectơ $$ \overrightarrow{a}-\left(1;-1\right)$$  là một vectơ chỉ phương.

========================================================================

https://khoahoc.vietjack.com/thi-online/15-cau-trac-nghiem-toan-8-canh-dieu-bai-6-phep-cong-phep-tru-phan-thuc-dai-so-co-dap-an


\textbf{{QUESTION}}

Với $$ \text{B}\ne 0$$, kết quả của phép cộng $$ \frac{\text{A}}{\text{B}}\text{+}\frac{\text{C}}{\text{B}}$$ là
A. $$ \frac{\text{A.C}}{\text{B}}$$

\textbf{{ANSWER}}

Đáp án đúng là: B
Ta có $$ \frac{\text{A}}{\text{B}}\text{+}\frac{\text{C}}{\text{B}}\text{=}\frac{\text{A + C}}{\text{B}}$$.

========================================================================

https://khoahoc.vietjack.com/thi-online/15-cau-trac-nghiem-toan-8-canh-dieu-bai-6-phep-cong-phep-tru-phan-thuc-dai-so-co-dap-an


\textbf{{QUESTION}}

Thực hiện phép tính sau: x2x+2−4x+2   (x≠−2)$$ \frac{{\text{x}}^{2}}{\text{x}+2}-\frac{4}{\text{x}+2}\text{\hspace{0.17em}\hspace{0.17em}\hspace{0.17em}}\left(\text{x}\ne -2\right)$$

\textbf{{ANSWER}}

Đáp án đúng là: D
x2x+2−4x+2=x2−4x+2=(x−2)(x+2)x+2$$ \frac{{\text{x}}^{2}}{\text{x}+2}-\frac{4}{\text{x}+2}=\frac{{\text{x}}^{2}-4}{\text{x}+2}=\frac{\left(\text{x}-2\right)\left(\text{x}+2\right)}{\text{x}+2}$$
=(x−2)(x+2):(x+2)(x+2):(x+2)=x−21=x−2$$ =\frac{\left(\text{x}-2\right)\left(\text{x}+2\right):\left(\text{x}+2\right)}{\left(\text{x}+2\right):\left(\text{x}+2\right)}=\frac{\text{x}-2}{1}=\text{x}-2$$.

========================================================================

https://khoahoc.vietjack.com/thi-online/100-cau-trac-nghiem-duong-thang-mat-phang-trong-khong-gian-co-ban/6466


\textbf{{QUESTION}}

Tính chất nào sau đây không phải của hình chóp cụt:
A. Hai đáy là hai đa giác có các cạnh tương ứng song song và các tỉ số các cặp cạnh tương ứng bằng nhau
B. Các mặt bên là các hình thang .
C. Các mặt bên là các hình thang cân
D. Các đường thẳng chứa các cạnh bên đồng quy tại một điểm

\textbf{{ANSWER}}

Đáp án C
Các mặt bên là các hình thang

========================================================================

https://khoahoc.vietjack.com/thi-online/100-cau-trac-nghiem-duong-thang-mat-phang-trong-khong-gian-co-ban/6466


\textbf{{QUESTION}}

Phát biểu nào sau đây đúng:
A. Phép chiếu song song biến 3 điểm thẳng hàng thành 3 điểm thẳng hàng và không làm thay đổi thứ tự 3 điểm đó
B. Phép chiếu song song biến đường thẳng thành đường thẳng, biến tia thành tia, biến đoạn thẳng thành đoạn thẳng
C. Phép chiếu song song biến 2 đường thẳng song song thành 2 đường thẳng song song hoặc trùng nhau.
D. Phép chiếu song song không làm thay đổi tỷ số độ dài của hai đoạn thẳng nằm trên 2 đường thẳng song song

\textbf{{ANSWER}}

Đáp án C
Phép chiếu song song biến 2 đường thẳng song song thành 2 đường thẳng song song hoặc trùng nhau

========================================================================

https://khoahoc.vietjack.com/thi-online/87-cau-bai-tap-chuyen-de-toan-11-bai-1-gioi-han-cua-day-so-co-dap-an/105471


\textbf{{QUESTION}}

Tìm các giới hạn sau:
a) $$ \mathrm{lim}\frac{{n}^{5}+{n}^{4}-n-2}{4{n}^{3}+6{n}^{2}+9}$$

\textbf{{ANSWER}}

Hướng dẫn giải
a)   $$ \mathrm{lim}\frac{{n}^{5}+{n}^{4}-n-2}{4{n}^{3}+6{n}^{2}+9}=\mathrm{lim}\frac{{n}^{5}\left(1+\frac{1}{n}-\frac{1}{{n}^{4}}-\frac{2}{{n}^{5}}\right)}{{n}^{3}\left(4+\frac{6}{n}+\frac{9}{{n}^{3}}\right)}.$$
  
Mà $$ \mathrm{lim}{n}^{2}=+\infty $$  và   $$ \mathrm{lim}\left(\frac{1+\frac{1}{n}-\frac{1}{{n}^{4}}-\frac{2}{{n}^{5}}}{4+\frac{6}{n}+\frac{9}{{n}^{3}}}\right)=\frac{1}{4}>0$$
Nên  $$ \mathrm{lim}\frac{{n}^{5}+{n}^{4}-n-2}{4{n}^{3}+6{n}^{2}+9}=+\infty .$$

========================================================================

https://khoahoc.vietjack.com/thi-online/87-cau-bai-tap-chuyen-de-toan-11-bai-1-gioi-han-cua-day-so-co-dap-an/105471


\textbf{{QUESTION}}

Tìm các giới hạn sau: b, lim−3√n6−7n3−5n+8n+12$$ \mathrm{lim}\frac{-\sqrt[3]{{n}^{6}-7{n}^{3}-5n+8}}{n+12}$$

\textbf{{ANSWER}}

b)   
  =lim−n2.3√1−7n3−5n5+8n6n(1+12n)$$ =\mathrm{lim}\frac{-{n}^{2}.\sqrt[3]{1-\frac{7}{{n}^{3}}-\frac{5}{{n}^{5}}+\frac{8}{{n}^{6}}}}{n\left(1+\frac{12}{n}\right)}$$
=lim(−n).3√1−7n3−5n5+8n61+12n$$ =\mathrm{lim}\left(-n\right).\frac{\sqrt[3]{1-\frac{7}{{n}^{3}}-\frac{5}{{n}^{5}}+\frac{8}{{n}^{6}}}}{1+\frac{12}{n}}$$
Mà  lim(−n)=−∞ $$ \mathrm{lim}\left(-n\right)=-\infty \quad $$và   lim3√1−7n3−5n5+8n61+12n=11=1>0$$ \mathrm{lim}\frac{\sqrt[3]{1-\frac{7}{{n}^{3}}-\frac{5}{{n}^{5}}+\frac{8}{{n}^{6}}}}{1+\frac{12}{n}}=\frac{1}{1}=1>0$$

========================================================================

https://khoahoc.vietjack.com/thi-online/87-cau-bai-tap-chuyen-de-toan-11-bai-1-gioi-han-cua-day-so-co-dap-an/105471


\textbf{{QUESTION}}

Tìm các giới hạn sau:

\textbf{{ANSWER}}

Hướng dẫn giải
a)   lim(√2n+3−√n+1)=lim√n.(√2+3n−√1+1n).
Do lim√n=+∞   và   lim(√2+3n−√1+1n)=√2−1>0.
Nên lim(√2n+3−√n+1)=+∞ .

========================================================================

https://khoahoc.vietjack.com/thi-online/87-cau-bai-tap-chuyen-de-toan-11-bai-1-gioi-han-cua-day-so-co-dap-an/105471


\textbf{{QUESTION}}

Tìm các giới hạn sau:
b, lim(n2+1)(2n+3)√n4−n2+1.

\textbf{{ANSWER}}

b)   
Do lim(1+1n2)(2+3n)=1.2=2>0;lim√1n2−1n4+1n6=0  và   √1n2−1n4+1n6>0.

========================================================================

https://khoahoc.vietjack.com/thi-online/87-cau-bai-tap-chuyen-de-toan-11-bai-1-gioi-han-cua-day-so-co-dap-an/105471


\textbf{{QUESTION}}

Tìm các giới hạn sau:
a, lim(5n−3n+1).

\textbf{{ANSWER}}

Hướng dẫn giải
a)   lim(5n−3n+1)=lim5n.[1−3.(35)n].
Do lim5n=+∞  và lim[1−3.(35)n]=1−3.0=1>0  nên lim(5n−3n+1)=+∞ .

========================================================================

https://khoahoc.vietjack.com/thi-online/bai-tap-nhi-thuc-newton-co-dap-an


\textbf{{QUESTION}}

Khai triển (a + b)n, n $$ \in $$ {1; 2; 3; 4; 5}.
Trong Bài 25 SGK Toán 10 (bộ sách Kết nối tri thức với cuộc sống), ta đã biết:
(a + b)1 = a + b
(a + b)2 = a2 + 2ab + b2
(a + b)3 = a3 + 3a2b + 3ab2 + b3
(a + b)4 = a4 + 4a3b + 6a2b2 + 4ab3 + b4
(a + b)5 = a5 + 5a4b + 10a3b2 + 10a2b3 + 5ab4 + b5
Với n $$ \in $$ {1; 2: 3; 4; 5}, trong khai triển của mỗi nhị thức (a + b)n:
a) Có bao nhiêu số hạng?
b) Tổng số mũ của a và b trong mỗi số hạng bằng bao nhiêu?
c) Số mũ của a và b thay đổi thế nào khi chuyển từ số hạng này đến số hạng tiếp theo, tính từ trái sang phải?

\textbf{{ANSWER}}

a) Có n + 1 số hạng, số hạng đầu tiên là an và số hạng cuối cùng là bn.
b) Tổng số mũ của a và b trong mỗi số hạng đều bằng n.
c) Số mũ của a giảm 1 đơn vị và số mũ của b tăng 1 đơn vị khi chuyền từ số hạng này đến số hạng tiếp theo, tính từ trái sang phải.

========================================================================

https://khoahoc.vietjack.com/thi-online/bai-tap-toan-8-chu-de-4-khai-niem-hai-tam-giac-dong-dang-co-dap-an/108850


\textbf{{QUESTION}}

Cho hai tam giác ABC và $$ A\text{'}B\text{'}C\text{'}$$ đồng dạng với nhau theo tỉ số k, chứng minh rằng tỉ số chu vi của hai tam giác ABC và $$ A\text{'}B\text{'}C\text{'}$$ cũng bằng k.

\textbf{{ANSWER}}

$$ \Delta ABC\Delta A\text{'}B\text{'}C\text{'}\Rightarrow \frac{AB}{A\text{'}B\text{'}}=\frac{AC}{A\text{'}C\text{'}}=\frac{BC}{B\text{'}C\text{'}}=k$$ 
Áp dụng tính chất dãy tỉ số bằng nhau ta có :
$$ \frac{AB}{A\text{'}B\text{'}}=\frac{AC}{A\text{'}C\text{'}}=\frac{BC}{B\text{'}C\text{'}}=\frac{AB+AC+BC}{A\text{'}B\text{'}+A\text{'}C\text{'}+B\text{'}C\text{'}}=k=\frac{{C}_{\Delta ABC}}{{C}_{\Delta A\text{'}B\text{'}C\text{'}}}$$  
Với $$ {C}_{\Delta ABC}$$ là chu vi tam giác ABC và $$ {C}_{\Delta A\text{'}B\text{'}C\text{'}}$$ là chu vi tam giác $$ A\text{'}B\text{'}C\text{'}$$

========================================================================

https://khoahoc.vietjack.com/thi-online/bai-tap-toan-8-chu-de-4-khai-niem-hai-tam-giac-dong-dang-co-dap-an/108850


\textbf{{QUESTION}}

Cho tam giác ABC có cạnh BC=10cm, CA=14cm, AB=6cm. Tam giác ABC đồng dạng với tam giác DEF có cạnh nhỏ nhất là 9cm. Tính các cạnh còn lại của tam giác DEF.

\textbf{{ANSWER}}

ΔABCΔDEF⇒ABDE=ACDF=BCEF$$ \Delta ABC\Delta DEF\Rightarrow \frac{AB}{DE}=\frac{AC}{DF}=\frac{BC}{EF}$$ .
ΔABC$$ \Delta ABC$$ cạnh nhỏ nhất là cạnh AB=6 cm$$ AB=6\text{\hspace{0.17em}}cm$$ . Nên cạnh nhỏ nhất của ΔDEF$$ \Delta DEF$$ là DE=9 cm$$ DE=9\text{\hspace{0.17em}}cm$$
Ta có: ABDE=ACDF=BCEF=69=14DF=10EF$$ \frac{AB}{DE}=\frac{AC}{DF}=\frac{BC}{EF}=\frac{6}{9}=\frac{14}{DF}=\frac{10}{EF}$$ 
Từ đó tính được DF=21 cm; EF=15 cm$$ DF=21\text{\hspace{0.17em}}cm;\text{\hspace{0.17em}}EF=15\text{\hspace{0.17em}}cm$$

========================================================================

https://khoahoc.vietjack.com/thi-online/15-cau-trac-nghiem-toan-9-ket-noi-tri-thuc-bai-26-xac-suat-cua-bien-co-lien-quan-toi-phep-thu-co-dap


\textbf{{QUESTION}}

I. Nhận biết
Cho phép thử $T$, xét biến cố $E$. Kết quả của phép thử $T$ làm cho biến cố $E$ xảy ra được gọi là
A. Kết quả đúng với $E$.
B. Kết quả phù hợp với $E$.
C. Kết quả của $E$.
D. Kết quả thuận lợi cho $E$.

\textbf{{ANSWER}}

Đáp án đúng là: D
Kết quả của phép thử $T$ làm cho biến cố $E$ xảy ra được gọi là kết quả thuận lợi cho $E$.

========================================================================

https://khoahoc.vietjack.com/thi-online/15-cau-trac-nghiem-toan-9-ket-noi-tri-thuc-bai-26-xac-suat-cua-bien-co-lien-quan-toi-phep-thu-co-dap


\textbf{{QUESTION}}

Giả sử các kết quả có thể của phép thử T là đồng khả năng xảy ra. Khi đó xác suất của biến cố E có liên quan tới T được ký hiệu là
A. A(E).
B. P(E).
C. Q(E).
D. n(E).

\textbf{{ANSWER}}

Đáp án đúng là: B
Xác suất của biến cố E$E$ có liên quan tới T được ký hiệu là P(E)$P\left( E \right)$.

========================================================================

https://khoahoc.vietjack.com/thi-online/15-cau-trac-nghiem-toan-9-ket-noi-tri-thuc-bai-26-xac-suat-cua-bien-co-lien-quan-toi-phep-thu-co-dap


\textbf{{QUESTION}}

Trong các phép thử sau, phép thử mà các kết quả không có cùng khả năng xảy ra là
A. Gieo một đồng xu cân đối và đồng chất.
B. Lấy ngẫu nhiên 1 viên bi từ một hộp có 10 viên bi giống nhau được đánh số từ 1 đến 10.
C. Lấy ngẫu nhiên 1 tấm thẻ từ một hộp chứa 2 tấm thẻ ghi số 5 và 5 tấm thẻ ghi số 2 và xem số của nó.
D. Các kết quả của các phép thử trên đều có cùng khả năng xảy ra.

\textbf{{ANSWER}}

Đáp án đúng là: C
 Các kết quả của phép thử A có cùng khả năng xảy ra vì khả năng gieo ra mặt sấp và ngửa là như nhau. 
 Các kết quả của phép thử B có cùng khả năng xảy ra vì các viên bi giống nhau nên khả năng được lựa chọn của các viên bi là như nhau. 
 Các kết quả của phép thử C không cùng khả năng xảy ra vì không thể khẳng định các thẻ lấy ra có cùng khối lượng, kích thước.

========================================================================

https://khoahoc.vietjack.com/thi-online/15-cau-trac-nghiem-toan-9-ket-noi-tri-thuc-bai-26-xac-suat-cua-bien-co-lien-quan-toi-phep-thu-co-dap


\textbf{{QUESTION}}

Xét một phép thử có không gian mẫu \(\Omega \) và \(A\) là một biến cố của phép thử đó. Xác suất của biến cố \(A\) là
A. \(P\left( A \right) = \frac{{n\left( A \right)}}{{n\left( \Omega \right)}}\).
B. \[P\left( A \right) = \frac{{n\left( \Omega \right)}}{{n\left( A \right)}}\].
C. \(P\left( A \right) = n\left( A \right).n\left( \Omega \right)\).
D. \(n\left( A \right) = n\left( \Omega \right)\).

\textbf{{ANSWER}}

Đáp án đúng là: A
Xác suất của biến cố \(A\) bằng tỉ số giữa số kết quả thuận lợi cho \(A\) và số phần tử của tập hợp \(\Omega \).
Xác suất của biến cố \(A\) bằng \(P\left( A \right) = \frac{{n\left( A \right)}}{{n\left( \Omega \right)}}\).

========================================================================

https://khoahoc.vietjack.com/thi-online/15-cau-trac-nghiem-toan-9-ket-noi-tri-thuc-bai-26-xac-suat-cua-bien-co-lien-quan-toi-phep-thu-co-dap


\textbf{{QUESTION}}

Chọn ngẫu nhiên một số tự nhiên từ 1 đến 10. Xác suất của biến cố \(A\): “Số được chọn là 10” là
A. \(\frac{4}{{10}}\).
B. \(\frac{3}{{10}}\).
C. \(\frac{2}{{10}}\).
D. \(\frac{1}{{10}}\).

\textbf{{ANSWER}}

Đáp án đúng là: D
Không gian mẫu của phép thử là \(\Omega = \left\{ {1;\,\,2;\,\,3;\,\,4;\,\,5;\,\,6;\,\,7;\,\,8;\,\,9;\,\,10} \right\}\).
Khả năng được chọn của các số là như nhau nên các kết quả của phép thử có cùng khả năng xảy ra.
Có 1 kết quả thuận lợi cho biến cố C là: 10.
Vậy xác suất xảy ra biến cố \(A\) là \(P\left( A \right) = \frac{1}{{10}}\).

========================================================================

https://khoahoc.vietjack.com/thi-online/bai-tap-khi-nao-thi-am-mb-ab-chon-loc-co-dap-an


\textbf{{QUESTION}}

Điểm P nằm giữa hai điểm M và N thì:
A. PN + MN = PN     
B. MP + MN = PN
C. MP + PN = MN     
D. MP - PN = MN

\textbf{{ANSWER}}

Đáp án là C
Điểm P nằm giữa hai điểm M và N thì: MP + PN = MN

========================================================================

https://khoahoc.vietjack.com/thi-online/bai-tap-khi-nao-thi-am-mb-ab-chon-loc-co-dap-an


\textbf{{QUESTION}}

Cho hai điểm A và B nằm trên tia Ox sao cho OA = 6cm, OB = 2cm. Hỏi trong ba điểm O, A, B điểm nào nằm giữa hai điểm còn lại?
A. Điểm O     
B. Điểm B
C. Điểm A     
D. Không có điểm nào nằm giữa hai điểm còn lại

\textbf{{ANSWER}}

Đáp án là B
Vì A, B đều thuộc tia Ox và OB < OA (2cm < 6cm) nên B nằm giữa A và O.

========================================================================

https://khoahoc.vietjack.com/thi-online/bo-5-de-cuoi-ki-2-toan-8-canh-dieu-cau-truc-moi-co-dap-an/164896


\textbf{{QUESTION}}

A. TRẮC NGHIỆM (7,0 điểm)
Phần 1. (3,0 điểm) Câu trắc nghiệm nhiều phương án lựa chọn
Trong mỗi câu hỏi từ câu 1 đến câu 12, hãy viết chữ cái in hoa đứng trước phương án đúng duy nhất vào bài làm.
Nhà bạn Mai mở tiệm kem, bạn ấy đã lập bảng tìm hiểu các khách hàng trong sáng chủ nhật và thu được kết quả như sau:
Loại kem
Số lượng bán
Dâu
10
Vani
5
Sầu riêng
6
Xoài
14
Từ bảng trên của bạn Mai, em hãy cho biết bạn Mai đang điều tra về vấn đề gì?

\textbf{{ANSWER}}

Đáp án đúng là: B
Ta xét từng vấn đề trên:
- Loại kem nhà Mai được khách hàng yêu thích nhất là Xoài vì đã bán 14.
- Người ăn kem nhiều nhất không có dữ liệu.
- Số loại kem của nhà Mai không có là không có dữ liệu.
- Khách hàng thân thiết là không có dữ liệu.
Do đó, chọn đáp án B.

========================================================================

https://khoahoc.vietjack.com/thi-online/bo-5-de-cuoi-ki-2-toan-8-canh-dieu-cau-truc-moi-co-dap-an/164896


\textbf{{QUESTION}}

Biểu diễn tỉ lệ của các phần trong tổng thể ta dùng biểu đồ nào sau đây?
A.
 Biểu đồ tranh.                                                
                                                
B. Biểu đồ đoạn thẳng.                                      
                                      
C.
 Biểu đồ hình quạt tròn.                                  
                                  
D. Biều đồ cột.

\textbf{{ANSWER}}

Đáp án đúng là: C
Để biểu diễn tỉ lệ của các phần trong tổng thể ta dùng biểu đồ hình quạt tròn.

========================================================================

https://khoahoc.vietjack.com/thi-online/bo-5-de-cuoi-ki-2-toan-8-canh-dieu-cau-truc-moi-co-dap-an/164896


\textbf{{QUESTION}}

Gieo đồng thời hai con xúc xắc, số các kết quả có thể xảy ra là

A.
 
10.$10.$
10.$10.$
10.
10.
10.
                       
                       
B. 
20.$20.$
20.$20.$
20.
20.
20.
                       
                       
C. 
12.$12.$
12.$12.$
12.
12.
12.
                       
                       
D. 
36.$36.$
36.$36.$
36.
36.
36.

\textbf{{ANSWER}}

Đáp án đúng là: D
Gieo đồng thời hai con xúc xắc, số các kết quả có thể xảy ra là: $6.6 = 36$.

========================================================================

https://khoahoc.vietjack.com/thi-online/bo-5-de-cuoi-ki-2-toan-8-canh-dieu-cau-truc-moi-co-dap-an/164896


\textbf{{QUESTION}}

Đội múa của trường gồm có 7 bạn nữ lớp 8A, 5 nam lớp 8A, 2 bạn nữ lớp 8B. Chọn ngẫu nhiên một bạn đội múa để múa chính. Số kết quả thuận lợi cho biến cố “Chọn được bạn nam” là

A.
 
3.$3.$
3.$3.$
3.
3.
3.
                         
                         
B. 
7.$7.$
7.$7.$
7.
7.
7.
                         
                         
C. 
5.$5.$
5.$5.$
5.
5.
5.
                         
                         
D. 
4.$4.$
4.$4.$
4.
4.
4.

\textbf{{ANSWER}}

Đáp án đúng là: C
Số kết quả thuận lợi cho biến cố “Chọn được bạn nam” là 5.

========================================================================

https://khoahoc.vietjack.com/thi-online/bo-5-de-cuoi-ki-2-toan-8-canh-dieu-cau-truc-moi-co-dap-an/164896


\textbf{{QUESTION}}

Phương trình nào là phương trình bậc nhất một ẩn trong các phương trình sau:

A. 
2x=0.
2x=0.
2x=0.
2x=0.
2
x
=
0.
                 
                 
B. 
3x2+1=0.
3x2+1=0.
3x2+1=0.
3x2+1=0.
3
x2
x2
x
2
+
1
=
0.
   
   
C. 
0x+2=0.
0x+2=0.
0x+2=0.
0x+2=0.
0
x
+
2
=
0.
          
          
D. 
1x=0.
1x=0.
1x=0.
1x=0.
1x
1
1
x
x


x
x
x
=
0.

\textbf{{ANSWER}}

Đáp án đúng là: A
Phương trình bậc nhất một ẩn có dạng ax+b=0(a≠0)$ax + b = 0{\rm{ }}\left( {a \ne 0} \right)$.
Do đó, phương trình 2x=0$2x = 0$ là phương trình bậc nhất một ẩn.

========================================================================

https://khoahoc.vietjack.com/thi-online/20-cau-trac-nghiem-toan-10-canh-dieu-menh-de-toan-hoc-co-dap-an-phan-2/103635


\textbf{{QUESTION}}

Trong các câu sau, câu nào là mệnh đề đúng?

\textbf{{ANSWER}}

Hướng dẫn giải
Đáp án đúng là: B
Mệnh đề A là một mệnh đề sai vì b ≤ a < 0 thì b2 ≥ a2. 
Mệnh đề B là mệnh đề đúng. Vì a ⁝ 9 thì a = 9n, n ∈ ℤ, mà 9 ⁝ 3, do đó a ⁝ 3. 
Câu C chưa là mệnh đề vì chưa khẳng định được tính đúng, sai. 
Mệnh đề D là mệnh đề sai vì chưa đủ điều kiện để khẳng định một tam giác là đều.

========================================================================

https://khoahoc.vietjack.com/thi-online/20-cau-trac-nghiem-toan-10-canh-dieu-menh-de-toan-hoc-co-dap-an-phan-2/103635


\textbf{{QUESTION}}

Trong các mệnh đề sau, mệnh đề nào có mệnh đề đảo đúng ?

\textbf{{ANSWER}}

Hướng dẫn giải
Đáp án đúng là: C
+ Mệnh đề đảo của mệnh đề A là: “Nếu tứ giác có hai đường chéo vuông góc với nhau thì tứ giác đó là hình thang cân”, mệnh đề này là mệnh đề sai. 
+ Mệnh đề đảo của mệnh đề B là: “Nếu hai tam giác có các góc tương ứng bằng nhau thì hai tam giác đó bằng nhau”, mệnh đề này là mệnh đề sai. 
+ Mệnh đề đảo của mệnh đề C là: “Nếu tam giác có ít nhất một góc (trong) nhỏ hơn 60° thì tam giác đó không phải là tam giác đều”, mệnh đề này là mệnh đề đúng, do tam giác đều thì có ba góc bằng nhau và bằng 60°, nên nếu tam giác có góc trong nhỏ hơn 60° thì chắc chắn tam giác đó không đều. 
+ Mệnh đề đảo của mệnh đề D là: “Nếu hai số tự nhiên có tổng chia hết cho 11 thì hai số tự nhiên đó cùng chia hết cho 11”, mệnh đề này là mệnh đề sai, thật vậy, chẳng hạn ta có hai số là 5 và 6, có 5 + 6 = 11 ⁝ 11 nhưng 5 và 6 đều không chia hết cho 11.

========================================================================

https://khoahoc.vietjack.com/thi-online/20-cau-trac-nghiem-toan-10-canh-dieu-menh-de-toan-hoc-co-dap-an-phan-2/103635


\textbf{{QUESTION}}

Cho a ∈ ℤ. Mệnh đề nào dưới đây đúng ?
B. a ⁝ 3 ⇔ a ⁝ 9;

\textbf{{ANSWER}}

Hướng dẫn giải
Đáp án đúng là: A
Đáp án B sai vì có 3 ⁝ 3 nhưng 3\cancel⋮9. 
Đáp án C sai vì có 2 ⁝ 2 nhưng 2\cancel⋮4. 
Đáp án D sai vì 6 ⁝ 3 và 6 ⁝ 6 nhưng 6\cancel⋮18.

========================================================================

https://khoahoc.vietjack.com/thi-online/20-cau-trac-nghiem-toan-10-canh-dieu-menh-de-toan-hoc-co-dap-an-phan-2/103635


\textbf{{QUESTION}}

Phủ định của mệnh đề: “Có ít nhất một số vô tỉ là số thập phân vô hạn tuần hoàn” là mệnh đề nào sau đây:

\textbf{{ANSWER}}

Hướng dẫn giải
Đáp án đúng là: C
Phủ định của “có ít nhất” là “mọi”.
Phủ định của “tuần hoàn” là “không tuần hoàn”.
Vậy phủ định của mệnh đề: “Có ít nhất một số vô tỉ là số thập phân vô hạn tuần hoàn” là mệnh đề: “Mọi số vô tỉ đều là số thập phân vô hạn không tuần hoàn”.

========================================================================

https://khoahoc.vietjack.com/thi-online/20-cau-trac-nghiem-toan-10-canh-dieu-menh-de-toan-hoc-co-dap-an-phan-2/103635


\textbf{{QUESTION}}

Cho mệnh đề chứa biến P(n): “n2 – 1 chia hết cho 4” với n là số nguyên. Xét xem các mệnh đề P(5) và P(2) đúng hay sai?

\textbf{{ANSWER}}

Hướng dẫn giải
Đáp án đúng là: C
Ta có: Với n = 5, n2 – 1 = 52 – 1 = 25 – 1 = 24 chia hết cho 4, do đó P(5) đúng. 
Với n = 2, n2 – 1 = 22 – 1 = 4 – 1 = 3 không chia hết cho 4, do đó P(2) sai. 
Vậy P(5) đúng và P(2) sai.

========================================================================

https://khoahoc.vietjack.com/thi-online/giai-vth-toan-7-bai-15-cac-truong-hop-bang-nhau-cua-tam-giac-vuong-co-dap-an


\textbf{{QUESTION}}

Hai tam giác vuông bằng nhau khi và chỉ khi điều nào dưới đây xảy ra?
A. Một cạnh góc vuông và góc nhọn của tam giác này bằng một cạnh góc vuông và góc nhọn của tam giác kia.
B. Một cạnh góc vuông và góc nhọn kề cạnh ấy của tam giác này bằng một cạnh góc vuông và góc kề của tam giác kia.
C. Hai góc nhọn của tam giác này bằng hai góc nhọn của tam giác kia.
D. Hai cạnh cảu tam giác này bằng hai cạnh của tam giác kia.

\textbf{{ANSWER}}

Đáp án đúng là B
Hai tam giác vuông bằng nhau khi và chỉ khi một cạnh góc vuông và góc nhọn kề cạnh ấy của tam giác này bằng một cạnh góc vuông và góc kề của tam giác kia.

========================================================================

https://khoahoc.vietjack.com/thi-online/giai-vth-toan-7-bai-15-cac-truong-hop-bang-nhau-cua-tam-giac-vuong-co-dap-an


\textbf{{QUESTION}}

Biết rằng ABC và MNP là tam giác vuông tại đỉnh A, M và AB = PM, ˆC=ˆN. Câu nào dưới đây là đúng?
A. ∆ABC = ∆MPN;
B. ∆ABC = ∆MNP;
C. ∆ABC = ∆PMN;
D. ∆ABC = ∆NMP.

\textbf{{ANSWER}}

Đáp án đúng là A
Xét hai tam giác ABC và MPN, ta có:
^BAC=^PMN=90°$$ \widehat{BAC}=\widehat{PMN}=90°$$
AB = MP (theo giả thiết)
^ABC=180°−^ACB=180°−^MNP=^MPN$$ \widehat{ABC}=180°-\widehat{ACB}=180°-\widehat{MNP}=\widehat{MPN}$$ (vì ˆC=ˆN$$ \widehat{C}=\widehat{N}$$)
Vậy ∆ABC = ∆MPN (cạnh góc vuông – góc nhọn).

========================================================================

https://khoahoc.vietjack.com/thi-online/giai-vth-toan-7-bai-15-cac-truong-hop-bang-nhau-cua-tam-giac-vuong-co-dap-an


\textbf{{QUESTION}}

Biết rằng ABC và MNP là các tam giác vuông tại đỉnh A, M và BC = PN, ˆC=50°,ˆP=40°. Câu nào dưới đây là đúng?
A. ∆ABC = ∆MPN;
B. ∆ABC = ∆MNP;
C. AB = MN;
D. AC = MP.

\textbf{{ANSWER}}

Xét tam giác ABC vuông tại A có tổng hai góc nhọn trong tam giác bằng 90° nên ta có: 
ˆC+ˆB=50°⇒ˆB=180°−ˆC=180°−50°=40°=ˆP
Hai tam giác ABC và MPN có: 
ˆA=ˆM=90°
ˆB=ˆP (chứng minh trên)
AB = MP (theo giả thiết)
Vậy ∆ABC = ∆MPN (cạnh góc vuông – góc nhọn)
Suy ra AB = MP, AC = MN (các cặp cạnh tương ứng)
Do đó A đúng; B, C, D sai.

========================================================================

https://khoahoc.vietjack.com/thi-online/giai-vth-toan-7-kntt-bai-tap-cuoi-chuong-6-co-dap-an


\textbf{{QUESTION}}

Lập tất cả các tỉ lệ thức có thể được từ bốn số: 0,2; 0,3; 0,8; 1,2.

\textbf{{ANSWER}}

Từ bốn số đã cho ta lập được đẳng thức: 0,2 . 1,2 = 0,3 . 0,8.
Từ đẳng thức này ta lập được bốn tỉ lệ thức:
$\frac{{0,2}}{{0,3}} = \frac{{0,8}}{{1,2}}$; $\frac{{0,2}}{{0,8}} = \frac{{0,3}}{{1,2}}$; $\frac{{0,3}}{{0,2}} = \frac{{1,2}}{{0,8}}$; $\frac{{1,2}}{{0,3}} = \frac{{0,8}}{{0,2}}$.

========================================================================

https://khoahoc.vietjack.com/thi-online/giai-vth-toan-7-kntt-bai-tap-cuoi-chuong-6-co-dap-an


\textbf{{QUESTION}}

Tìm thành phần chưa biết x trong tỉ lệ thức: x2,5=1015.

\textbf{{ANSWER}}

Từ x2,5=1015 suy ra x = =2,5.1015=53. Vậy x = 53.

========================================================================

https://khoahoc.vietjack.com/thi-online/giai-vth-toan-7-kntt-bai-tap-cuoi-chuong-6-co-dap-an


\textbf{{QUESTION}}

Từ tỉ lệ thức ab=cd (với a, b, c, d khác 0) có thể suy ra những tỉ lệ thức nào nữa?

\textbf{{ANSWER}}

Từ ab=cd suy ra đẳng thức ad = bc. Từ đẳng thức này suy ra, ngoài tỉ lệ thức đã cho ab=cd, các tỉ lệ thức sau: ac=bd; db=ca; dc=ba.

========================================================================

https://khoahoc.vietjack.com/thi-online/giai-vth-toan-7-kntt-bai-tap-cuoi-chuong-6-co-dap-an


\textbf{{QUESTION}}

Inch (đọc là in-sơ và viết tắt là in) là tên của một đơn vị chiều dài trong Hệ đo lường Mỹ. Biết rằng 1 in = 2,54 cm.
Hỏi một người cao 170 cm sẽ có chiều cao là bao nhiêu inch (làm tròn kết quả đến hàng đơn vị)?

\textbf{{ANSWER}}

Một người cao 170 cm sẽ có chiều cao xấp xỉ bằng 1702,54≈ 67 (in).
1702,54
1702,54
1702,54
1702,54
170
170
170
2,54
2,54


2,54
2,54
2,54
2
,
54

========================================================================

https://khoahoc.vietjack.com/thi-online/giai-vth-toan-7-kntt-bai-tap-cuoi-chuong-6-co-dap-an


\textbf{{QUESTION}}

Inch (đọc là in-sơ và viết tắt là in) là tên của một đơn vị chiều dài trong Hệ đo lường Mỹ. Biết rằng 1 in = 2,54 cm.
Chiều cao của một người tính theo xentimét có tỉ lệ thuận với chiều cao của người đó tính theo inch không? Nếu có thì hệ số tỉ lệ là bao nhiêu?

\textbf{{ANSWER}}

Chiều cao của một người tính theo xentimét tỉ lệ thuận với chiều cao của người đó tính theo inch và hệ số tỉ lệ là 2,54.

========================================================================

https://khoahoc.vietjack.com/thi-online/luy-thua-voi-so-mu-tu-nhien-nhan-hai-luy-thua-cung-co-so/57478


\textbf{{QUESTION}}

Tính giá trị các lũy thừa sau:
a, $$ {2}^{6}$$
b, $$ {5}^{3}$$
c, $$ {4}^{4}$$
d, $$ {15}^{2}$$
e, $$ {100}^{2}$$
f, $$ {20}^{3}$$

\textbf{{ANSWER}}

a, $$ {2}^{6}$$ = 64
b, $$ {5}^{3}$$ = 125
c, $$ {4}^{4}$$ = 256
d, $$ {15}^{2}$$ = 225
e, $$ {100}^{2}$$ = 10000
f, $$ {20}^{3}$$ = 8000

========================================================================

https://khoahoc.vietjack.com/thi-online/luy-thua-voi-so-mu-tu-nhien-nhan-hai-luy-thua-cung-co-so/57478


\textbf{{QUESTION}}

Tính giá trị các biểu thức sau:
a, A = 210-25
b, B = 43-42-4
c, C = 32.23+43.25
d, D = 13+23+33+43+53

\textbf{{ANSWER}}

a, A = 210-25 = 1024 - 32 = 992
b, B = 43-42-4 = 64 - 16 - 4 = 44
c, C = 32.23+43.25 = 9.8 + 64.32 = 2120
d, D = 13+23+33+43+53 = 1 + 8 + 27 + 64 + 125 = 225

========================================================================

https://khoahoc.vietjack.com/thi-online/luy-thua-voi-so-mu-tu-nhien-nhan-hai-luy-thua-cung-co-so/57478


\textbf{{QUESTION}}

Viết các tổng sau thành một bình phương của một Số tự nhiên:
a) 2+33+42+132$$ 2+{3}^{3}+{4}^{2}+{13}^{2}$$
b) 13+23+33+43+53+63$$ {1}^{3}+{2}^{3}+{3}^{3}+{4}^{3}+{5}^{3}+{6}^{3}$$

\textbf{{ANSWER}}

a) 2+33+42+132$$ 2+{3}^{3}+{4}^{2}+{13}^{2}$$ = 196 = 142$$ {14}^{2}$$
b, 13+23+33+43+53+63$$ {1}^{3}+{2}^{3}+{3}^{3}+{4}^{3}+{5}^{3}+{6}^{3}$$ = 441 = 212$$ {21}^{2}$$

========================================================================

https://khoahoc.vietjack.com/thi-online/luy-thua-voi-so-mu-tu-nhien-nhan-hai-luy-thua-cung-co-so/57478


\textbf{{QUESTION}}

Tính giá trị các biểu thức sau và viết kết quả dưới dạng một lũy thừa của một số:
a, A = 22.52-32-10
b, B = 33.32+22+32
c, C = 5.43+24.5
d, D = 53+63+73+79.22

\textbf{{ANSWER}}

a, A = 22.52-32-10$$ {2}^{2}.{5}^{2}-{3}^{2}-10$$ = 81 = 34=(92)$$ {3}^{4}=\left({9}^{2}\right)$$
b, B = 33.32+22+32$$ {3}^{3}.{3}^{2}+{2}^{2}+{3}^{2}$$ = 256 = 162(=28=44)$$ {16}^{2}\left(={2}^{8}={4}^{4}\right)$$
c, C = 5.43+24.5$$ 5.{4}^{3}+{2}^{4}.5$$ = 400 = 202$$ {20}^{2}$$
d, D = 53+63+73+79.22$$ {5}^{3}+{6}^{3}+{7}^{3}+79.{2}^{2}$$ = 1000 = 103$$ {10}^{3}$$

========================================================================

https://khoahoc.vietjack.com/thi-online/122-cau-chuyen-de-toan-12-bai-2-cuc-tri-cua-ham-so-co-dap-an


\textbf{{QUESTION}}

Hàm số $$ f\left(x\right)={x}^{3}-3{x}^{2}-9x+1$$đạt cực tiểu tại điểm
B. $$ x=3.$$
C. $$ x=1$$
D. $$ x=-3.$$

\textbf{{ANSWER}}

Hàm số đã cho xác định trên R.
Ta có: $$ {f}^{\text{'}}\left(x\right)=3{x}^{2}-6x-9.$$
Từ đó: $$ {f}^{\text{'}}\left(x\right)=0\Leftrightarrow \left[\begin{array}{l}x=-1\\ x=3\end{array}\right..$$
Ta có: $$ {{f}^{\text{'}}}^{\text{'}}\left(x\right)=6x-6$$ . Khi đó: $$ {{f}^{\text{'}}}^{\text{'}}\left(-1\right)=-12<0;{{f}^{\text{'}}}^{\text{'}}\left(3\right)=12>0.$$
 
Vậy hàm số đạt cực tiểu tại điểm x=3

========================================================================

https://khoahoc.vietjack.com/thi-online/18-cau-trac-nghiem-toan-6-chan-troi-sang-tao-bai-tap-on-tap-chuong-8-hinh-hoc-phang-cac-hinh-hinh-ho


\textbf{{QUESTION}}

A.1
B.2
C.3
D.Vô số

\textbf{{ANSWER}}

Có 1 và chỉ 1 đường thẳng đi qua 2 điểm phân biệt cho trước. Vậy có duy nhất 1 đường thẳng đi qua hai điểm A và B.
Đáp án cần chọn là: A

========================================================================

https://khoahoc.vietjack.com/thi-online/uoc-va-boi/57805


\textbf{{QUESTION}}

Tìm số tự nhiên n sao cho:
a) 3$$ \vdots $$n;
 b) 3$$ \vdots $$(n + l);
c) ( n +3)$$ \vdots $$( n + 1)
d) (2n + 3)$$ \vdots $$( n – 2)

\textbf{{ANSWER}}

a) 3$$ \vdots $$n ó n$$ \in $$Ư (3). Ta có Ư (3) = {1;3}. Vậy n$$ \in $${ 1;3}.
b) 3$$ \vdots $$(n + l) ó (n + l)$$ \in $$Ư (3). Ta có  Ư (3) = {1;3}.
Vậy (n + l)$$ \in $${l ;3} => n$$ \in $${0; 2}.
c) Ta có: (n - 3)$$ \vdots $$(n - 1) và (n - 1)$$ \vdots $$(n -1);
Áp dụng tính chất chia hết của tổng (hiệu) ta có:
(n + 3) - (n + 1 )$$ \vdots $$( n+ l) ó 2$$ \vdots $$( n + 1) <=> ( n +1)$$ \in $$Ư (2) = {1;2}
Từ đó  n$$ \in $${0;l}.
d) Ta có (2n + 3)$$ \vdots $$(n - 2) và (n - 2)$$ \vdots $$(n - 2) =>2 (n - 2)$$ \vdots $$(n - 2);
Áp dụng tính chất chia hết của tổng (hiệu) ta có
(2n + 3)(n - 2)$$ \vdots $$(n - 2) <=> 7$$ \vdots $$(n - 2) ó (n - 2)$$ \in $$Ư(97) = {1;7}.
Từ đó n$$ \in $${3;9}

========================================================================

https://khoahoc.vietjack.com/thi-online/uoc-va-boi/57805


\textbf{{QUESTION}}

Tìm số tự nhiên n sao cho:
a) 7⋮$$ \vdots $$n;
b) 7⋮$$ \vdots $$(n - l);
c) ( 2n +6)⋮$$ \vdots $$( 2n - 1)
d) (3n + 7)⋮$$ \vdots $$( n - 2)

\textbf{{ANSWER}}

Tương tự câu 1. HS tự làm

========================================================================

https://khoahoc.vietjack.com/thi-online/10-bai-tap-tim-cac-boi-cua-mot-so-nguyen-cho-truoc-co-loi-giai


\textbf{{QUESTION}}

Các bội của 9 là:
A. -9; 9; 0; 23; -23;…;
B. 132; -132; 16;…;
C. -1; 1; 9; -9;…;
D. 0; 9; -9; 18; -18; ....

\textbf{{ANSWER}}

Đáp án đúng là: D
Lần lượt nhân 9 với 0; 1; 2; 3;… ta được các bội tự nhiên của 6 là: 0; 9; 18; …
Do đó các bội của 9 lần lượt là: 0; 9; -9; 18; -18;…

========================================================================

https://khoahoc.vietjack.com/thi-online/10-bai-tap-tim-cac-boi-cua-mot-so-nguyen-cho-truoc-co-loi-giai


\textbf{{QUESTION}}

Tập hợp tất cả các bội của 7 lớn hơn -50 và nhỏ hơn 50 là:
A. {0; 7; 14; 21; 28; 35; 42; 49; -7; -14; -21; -28; -35; -42; -49};
B. {7; 14; 21; 28; 35; 42; 49; -7; -14; -21; -28; -35; -42; -49};
C. {0; 7; 14; 21;28; 35; 42; 49};
D. {0; 7; 14; 21; 28; 35; 42; 49; -7; -14; -21; -28; -35; -42; -49; -56; ...}.

\textbf{{ANSWER}}

Đáp án đúng là: A
Lần lượt nhân 7 với 0; 1; 2; 3;… ta được các bội dương của 7 là: 0; 7; 14; 21; 28; 35; 42; 49;…
Do đó các bội của 7 lần lượt là: 0; 7; 14; 21; 28; 35; 42; 49; -7; -14; -21; -28; -35; -42; -49;…
Vậy tập hợp tất cả các bội của 7 có giá trị lớn hơn -50 và nhỏ hơn 50 là:
{0; 7; 14; 21; 28; 35; 42; 49; -7; -14; -21; -28; -35; -42; -49}

========================================================================

https://khoahoc.vietjack.com/thi-online/10-bai-tap-tim-cac-boi-cua-mot-so-nguyen-cho-truoc-co-loi-giai


\textbf{{QUESTION}}

Số các bội của 13 lớn hơn -40 và nhỏ hơn 40 là:
A. 7;
B. 6;
C. 8;
D. 5.

\textbf{{ANSWER}}

Đáp án đúng là: A
Lần lượt nhân 13 với 0; 1; 2; 3; 4 … ta được các bội tự nhiên của 13 là: 0; 13; 26; 39; 52; …
Các bội của 13 là: 0; 13; -13; 26; -26; 39; -39; 52; -52; …
Do đó các bội của 13 lớn hơn -40 và nhỏ hơn 40 lần lượt là: -39; -26; -13; 0; 13; 26; 39
Vậy có 7 bội của -13 lớn hơn -40 và nhỏ hơn 40.
Chị đang chưa hiểu bài này em muốn tìm bội của 13 hay -13…chị thấy lúc thì làm là 13 lúc thì làm là -13.

========================================================================

https://khoahoc.vietjack.com/thi-online/10-bai-tap-tim-cac-boi-cua-mot-so-nguyen-cho-truoc-co-loi-giai


\textbf{{QUESTION}}

Số bội lớn hơn -30 và nhỏ hơn 24 của 6 là:
A. 11;
B. 10;
C. 9;
D. 8.

\textbf{{ANSWER}}

Đáp án đúng là: C
Lần lượt nhân 6 với 0; 1; 2; 3; 4; 5; … ta được các bội tự nhiên của 6 là: 0; 6; 12; 18; 24; …
Các bội của 6 là: 0; 6; -6; 12; -12; 18; -18; 24; -24; 30; -30…
Các bội lớn hơn -30 và nhỏ hơn 24 của 6 là: : 0; 6; -6; 12; -12; 18; -18; 24; -24
Vậy số bội lớn hơn -30 và nhỏ hơn 24 của 6 là: 9

========================================================================

https://khoahoc.vietjack.com/thi-online/10-bai-tap-tim-cac-boi-cua-mot-so-nguyen-cho-truoc-co-loi-giai


\textbf{{QUESTION}}

Tổng các bội dương nhỏ hơn 20 của 5 là:
A. 15;
B. 20;
C. 25;
D. 30.

\textbf{{ANSWER}}

Đáp án đúng là: D
Lần lượt nhân 5 với 0; 1; 2; 3; 4; …ta được các bội tự nhiên của 5 là: 0; 5; 10; 15; 20; …
Các bội dương nhỏ hơn 20 của 5 là: 5; 10; 15
Tổng các bội dương nhỏ hơn 20 của 5 là: 5 + 10 + 15 = 30.

========================================================================

https://khoahoc.vietjack.com/thi-online/chuong-3-bai-4-goc-tao-boi-tia-tiep-tuyen-va-day-cung/59059


\textbf{{QUESTION}}

Cho các đường tròn (O; R) và (O’; R’) tiếp xúc trong với nhau tại A (R > R’). Vẽ đường kính AB của (O), AB cắt (O’) tại điểm thứ hai C. Từ B vẽ tiếp tuyến BP với đường tròn (O’), BP cắt (O) tại Q. Đường thẳng AP cắt (O) tại điểm thứ hai R. Chứng minh:
a, AP là phân giác của $$ \hat{BAQ}$$
b, CP và BR song song với nhau

\textbf{{ANSWER}}

a, Sử dụng AQ//O'P
=> $$ \hat{QAP}=\hat{O\text{'}AP}$$ => Đpcm
b, CP//BR (cùng vuông góc AR)

========================================================================

https://khoahoc.vietjack.com/thi-online/10-bai-tap-do-thi-ham-so-bac-nhat-va-cac-bai-toan-lien-quan-co-loi-giai


\textbf{{QUESTION}}

Đồ thị hàm số y = ax (a ≠ 0) đi qua điểm nào trong các điểm dưới đây

\textbf{{ANSWER}}

Đáp án đúng là: D
• Với x = 0 thì y = 0, đồ thị hàm số đi qua điểm (0; 0).
• Với x = 1 thì y = a, đồ thị hàm số đi qua điểm (1; a).
• Với x = 2 thì y = 2a, đồ thị hàm số đi qua điểm (2; 2a).
Vậy đồ thị hàm số đi qua cả 3 điểm (0; 0); (1; a); (2; 2a).

========================================================================

https://khoahoc.vietjack.com/thi-online/10-bai-tap-do-thi-ham-so-bac-nhat-va-cac-bai-toan-lien-quan-co-loi-giai


\textbf{{QUESTION}}

Đồ thị hàm số y = ax + b (a ≠ 0) đi qua điểm nào trong các điểm dưới đây

\textbf{{ANSWER}}

Đáp án đúng là: A
Với x = 0 thì y = b, đồ thị hàm số đi qua điểm (0; b).

========================================================================

https://khoahoc.vietjack.com/thi-online/10-bai-tap-do-thi-ham-so-bac-nhat-va-cac-bai-toan-lien-quan-co-loi-giai


\textbf{{QUESTION}}

Đồ thị hàm số bậc nhất có dạng

\textbf{{ANSWER}}

Đáp án đúng là: C
Đồ thị của hàm số bậc nhất có dạng một đường thẳng.

========================================================================

https://khoahoc.vietjack.com/thi-online/10-bai-tap-do-thi-ham-so-bac-nhat-va-cac-bai-toan-lien-quan-co-loi-giai


\textbf{{QUESTION}}

Điểm nào dưới đây thuộc đồ thị hàm số y = 2x + 1?

\textbf{{ANSWER}}

Đáp án đúng là: C
Với x = 1 thì y = 2 . 1 + 1 = 3.
Do đó, điểm (1; 3) thuộc đồ thị hàm số y = 2x + 1.

========================================================================

https://khoahoc.vietjack.com/thi-online/10-bai-tap-do-thi-ham-so-bac-nhat-va-cac-bai-toan-lien-quan-co-loi-giai


\textbf{{QUESTION}}

Điểm nào dưới đây thuộc đồ thị hàm số y = 5x – 3?

\textbf{{ANSWER}}

Đáp án đúng là: D
Với x = 1 thì y = 5. 1 – 3 = 2.
Do đó, điểm (1; 2) thuộc đồ thị hàm số y = 5x – 3.

========================================================================

https://khoahoc.vietjack.com/thi-online/giai-sbt-toan-9-ctst-bai-1-bat-dang-thuc


\textbf{{QUESTION}}

Dùng các dấu >, <, ≥, ≤ để diễn tả:
a) Giá bán thấp nhất T của một chiếc điện thoại là 6 triệu đồng.
b) Điểm trung bình tối thiểu G để đạt học lực giỏi là 8.
c) Thời gian tối đa t để hoàn thành một dự án là 12 tháng.

\textbf{{ANSWER}}

a) Giá bán thấp nhất T của một chiếc điện thoại là 6 triệu đồng, tức là T ≥ 6 (triệu đồng);
b) Điểm trung bình tối thiểu G để đạt học lực giỏi là 8, tức là G ≥ 8 (điểm);
c) Thời gian tối đa t để hoàn thành một dự án là 12 tháng, tức là t ≤ 12 (tháng).

========================================================================

https://khoahoc.vietjack.com/thi-online/giai-sbt-toan-9-ctst-bai-1-bat-dang-thuc


\textbf{{QUESTION}}

Điền vào chỗ chấm dấu >, =, hoặc < để tạo thành một phát biểu đúng.
a) Nếu 17 > 10 và 10 > p thì 17 ... p.
b) Nếu –11 > x và x > y thì –11 ... y.
c) Nếu a < 100 và b > 100 thì b ... a.
d) Nếu x + 1 = y thì x ... y.
e) Nếu 3x = 3y thì x ... y.

\textbf{{ANSWER}}

a) Nếu 17 > 10 và 10 > p thì 17 > p.
b) Nếu –11 > x và x > y thì –11 > y.
c) Nếu a < 100 và b > 100 thì b > a.
d) Nếu x + 1 = y thì x < y.
e) Nếu 3x = 3y thì x = y.

========================================================================

https://khoahoc.vietjack.com/thi-online/giai-sbt-toan-9-ctst-bai-1-bat-dang-thuc


\textbf{{QUESTION}}

Hãy cho biết các bất đẳng thức được tạo thành khi:
a) Cộng hai vế của bất đẳng thức p + 2 > 5 với –2;
b) Cộng hai vế của bất đẳng thức x + 10 ≤ y + 11 với 9;
c) Nhân hai vế của bất đẳng thức 13x<5 với 3, rồi tiếp tục cộng với –15;
d) Cộng vào hai vế của bất đẳng thức 2m ≤ –3 với –1, rồi tiếp tục nhân với −12.

\textbf{{ANSWER}}

a) p + 2 > 5
p + 2 + (‒ 2) > 5 + (‒ 2)
p > 3.
b) x + 10 ≤ y + 11
x + 10 + 9 ≤ y + 11 + 9
x + 19 ≤ y + 20.
c) 13x<5$$ \frac{1}{3}x<5$$
13x⋅3<5⋅3$$ \frac{1}{3}x\cdot 3<5\cdot 3$$
x < 15
x + (‒15) < 15 + (‒15)
x ‒ 15 < 0.
d) 2m ≤ –3
2m + (‒1) ≤ ‒3 + (‒1)
2m ‒ 1 ≤ ‒4
(2m−1)⋅(−12)≥−4⋅(−12)−m+12≥2.$$ \begin{array}{l}\left(2m-1\right)\cdot \left(-\frac{1}{2}\right)\ge -4\cdot \left(-\frac{1}{2}\right)\\ -m+\frac{1}{2}\ge 2.\end{array}$$

========================================================================

https://khoahoc.vietjack.com/thi-online/giai-sbt-toan-9-ctst-bai-1-bat-dang-thuc


\textbf{{QUESTION}}

So sánh hai số m và n trong mỗi trường hợp sau:
a) m + 15 < n + 15;
b) –17m ≥ –17n;
c) m7−5≤n7−5;
d) –0,7n + 10 > –0,7m + 10.

\textbf{{ANSWER}}

a) m + 15 < n + 15
m + 15 – 15 < n + 15 – 15
m < n.
b) –17m ≥ –17n
–17m⋅−117≥–17n⋅−117
m ≤ n.
c) m7−5≤n7−5;m7≤n7m7⋅7≤n7⋅7
m ≤ n.
d) –0,7n + 10 > –0,7m + 10
‒0,7n > ‒0,7m
−0,7n⋅1−0,7<−0,7m⋅1−0,7
n < m.

========================================================================

https://khoahoc.vietjack.com/thi-online/giai-sbt-toan-9-ctst-bai-1-bat-dang-thuc


\textbf{{QUESTION}}

Cho a > 0 và b > 0. Chứng tỏ a + b > 0.

\textbf{{ANSWER}}

Cộng hai vế của a > 0 với b ta được a + b > b, do b > 0 nên a + b > 0.

========================================================================

https://khoahoc.vietjack.com/thi-online/35-cau-trac-nghiem-toan-6-ket-noi-tri-thuc-bai-tap-cuoi-chuong-1-trang-28-co-dap-an


\textbf{{QUESTION}}

Viết số tự nhiên a sau đây: Mười lăm tỉ hai trăm sáu mươi bảy triệu không trăm hai mươi mốt nghìn chín trăm linh tám.
a) Số a có bao nhiêu chữ số? Viết tập hợp các chữ số của a
b) Số a có bao nhiêu triệu, chữ số hàng triệu là chữ số nào?
c) Trong a có hai chữ số 1 nằm ở những hàng nào? Mỗi chữ số ấy có giá trị bằng bao nhiêu?

\textbf{{ANSWER}}

Số “mười lăm tỉ hai trăm sáu mươi bảy triệu không trăm hai mươi mốt nghìn chín trăm linh tám” được viết là 15 267 021 908.
Vậy số tự nhiên a là 15 267 021 908.
a) Số a có 11 chữ số. Số tự nhiên a viết bởi các chữ số: 1; 5; 2; 6; 7; 0; 2; 1; 9; 0; 8
Chữ số 0 xuất hiện 2 lần, chữ số 1 xuất hiện 2 lần, chữ số 2 xuất hiện 2 lần nhưng trong tập hợp, mỗi phần tử (hay mỗi số) ta chỉ viết (liệt kê) một lần.
Gọi A là tập hợp các chữ số của a.
Do đó tập hợp các chữ số của a là A = {0; 1; 2; 5; 6; 7; 8; 9}.
b) Số a có 267 triệu, chữ số hàng triệu là chữ số 7.
c) Trong a có 2 chữ số 1, tính từ trái qua phải:
+) Chữ số 1 thứ nhất nằm ở hàng chục tỉ có giá trị là: 10 000 000 000
+) Chữ số 1 thứ hai nằm ở hàng nghìn có giá trị là 1 000.

========================================================================

https://khoahoc.vietjack.com/thi-online/35-cau-trac-nghiem-toan-6-ket-noi-tri-thuc-bai-tap-cuoi-chuong-1-trang-28-co-dap-an


\textbf{{QUESTION}}

a) Số 2 020 là số liền sau của số nào? Là số liền trước của số nào?
b) Cho số tự nhiên a khác 0. Số liền trước của số tự nhiên a là số nào? Số liền sau số tự nhiên a là số nào?
c) Trong các số tự nhiên, số nào không có số liền sau? Số nào không có số liền trước?

\textbf{{ANSWER}}

a) Số 2 020 là số liền sau của số 2 019
Số 2 020 là số liền trước của số 2 021
b) Số liền trước của số tự nhiên a khác 0 là số a – 1. Số liền sau của số tự nhiên a là a + 1
c) Trong các số tự nhiên, không có số nào không có số liền sau. Số 0 không có số liền trước vì trong tập hợp số tự nhiên số 0 là số nhỏ nhất.

========================================================================

https://khoahoc.vietjack.com/thi-online/bo-de-kiem-tra-chuyen-de-toan-8-chuong-1-co-dap-an/109539


\textbf{{QUESTION}}

A. $$ 6{x}^{3}{y}^{2}+\text{ }3{x}^{4}y\text{ }+\text{ }3{x}^{2}{y}^{2}$$
B. $$ -6{x}^{3}{y}^{2}-\text{ }3{x}^{4}y\text{ }+\text{ }3{x}^{2}y$$
C. $$ 6{x}^{2}y\text{ }+{x}^{2}+\text{ }{y}^{2}$$
D. $$ x\text{ }–\text{ }2y\text{ }+\text{ }2{x}^{2}$$

\textbf{{ANSWER}}

Chọn đáp án B

========================================================================

https://khoahoc.vietjack.com/thi-online/bo-de-kiem-tra-chuyen-de-toan-8-chuong-1-co-dap-an/109539


\textbf{{QUESTION}}

Kết quả của phép tính (x + 2y).(x − 2y) bằng :
B. x2− 4xy + 4y2 
C. x2− 4y2
D. Kết quả khác

\textbf{{ANSWER}}

Chọn đáp án C

========================================================================

https://khoahoc.vietjack.com/thi-online/bo-de-kiem-tra-chuyen-de-toan-8-chuong-1-co-dap-an/109539


\textbf{{QUESTION}}

A. 5z
B. 5x5y8z
C. -5xy2z
D. 5x6y15z

\textbf{{ANSWER}}

Chọn đáp án B

========================================================================

https://khoahoc.vietjack.com/thi-online/bo-de-kiem-tra-chuyen-de-toan-8-chuong-1-co-dap-an/109539


\textbf{{QUESTION}}

Giá trị của biểu thức x3− 6x2+ 12x − 8 tại x = -3 là :
A. 14
B. 1200
C. 1000
D. 16

\textbf{{ANSWER}}

Chọn đáp án D

========================================================================

https://khoahoc.vietjack.com/thi-online/bo-de-kiem-tra-chuyen-de-toan-8-chuong-1-co-dap-an/109539


\textbf{{QUESTION}}

Trong các hằng đẳng thức sau, hằng đẳng thức nào là “lập phương của một hiệu”:
A. (a3+ b3) = a − ba2+ ab + b2
B. a + b3= a3+ 3a2b + 3ab2+ b3
C. a − b3= a3− 3a2b + 3ab2− b3
D. (a3+ b3) = a + ba2− ab + b2

\textbf{{ANSWER}}

Chọn đáp án C

========================================================================

https://khoahoc.vietjack.com/thi-online/de-kiem-tra-hoc-ki-1-toan-7-co-dap-an-moi-nhat


\textbf{{QUESTION}}

Trong các phân số sau, phân số nào biểu diễn số hữu tỉ $$ -\frac{3}{4}$$?
A. $$ \frac{-6}{2}$$
B. $$ \frac{8}{-6}$$
C. $$ \frac{9}{-12}$$
D. $$ \frac{-12}{9}$$

\textbf{{ANSWER}}

Rút gọn các phân số ở các đáp án A, B, C, D, ta được:
A. $$ \frac{-6}{2}=-3$$
B. $$ \frac{8}{-6}=\frac{-4}{3}$$
C. $$ \frac{9}{-12}=-\frac{3}{4}$$
D. $$ \frac{-12}{9}=\frac{-4}{3}$$
Vậy chọn C. $$ \frac{9}{-12}$$.

========================================================================

https://khoahoc.vietjack.com/thi-online/de-kiem-tra-hoc-ki-1-toan-7-co-dap-an-moi-nhat


\textbf{{QUESTION}}

Kết quả phép tính −38+56 là:
A. 1124
B. 2248
C. −1124
D. −2248

\textbf{{ANSWER}}

Ta có: −38+56=−924+2024=−9+2024=1124$$ \frac{-3}{8}+\frac{5}{6}=\frac{-9}{24}+\frac{20}{24}=\frac{-9+20}{24}=\frac{11}{24}$$.
Đáp án A

========================================================================

https://khoahoc.vietjack.com/thi-online/de-kiem-tra-hoc-ki-1-toan-7-co-dap-an-moi-nhat


\textbf{{QUESTION}}

Cho biết x và y là hai đại lượng tỉ lệ thuận, khi x = 5 thì y = 15. Hệ số tỉ lệ k của y đối với x là:
A. 3
B. 75
C. 13
D. 10.

\textbf{{ANSWER}}

Công thức liên hệ của hai đại lượng x, y tỉ lệ thuận: y = k . x (với k là hệ số tỉ lệ).
Do đó k=yx=155=3.
Đáp án A

========================================================================

https://khoahoc.vietjack.com/thi-online/de-kiem-tra-hoc-ki-1-toan-7-co-dap-an-moi-nhat


\textbf{{QUESTION}}

Nếu góc xOy có số đo bằng 47o thì số đo của góc đối đỉnh với góc xOy bằng bao nhiêu?
A. 133o
B. 43o
C. 74o
D. 47o

\textbf{{ANSWER}}

Dựa vào tính chất: Hai góc đối đỉnh thì bằng nhau.
Ta có góc xOy có số đo bằng 47o nên góc đối đỉnh với góc xOy cũng có số đo bằng 47o.
Đáp án D

========================================================================

https://khoahoc.vietjack.com/thi-online/16-cau-trac-nghiem-toan-7-chan-troi-sang-tao-bai2-dien-tich-xung-quanh-va-the-tich-cua-hinh-hop-chu


\textbf{{QUESTION}}

Hình hộp chữ nhật có ba kích thước lần lượt là: a, 2a, 2a. Thể tích của hình hộp chữ nhật đó là:

\textbf{{ANSWER}}

Hướng dẫn giải
Đáp án đúng là: D
Thể tích của hình hộp chữ nhật là: 
V = a. 2a. 2a = 4a3 (đơn vị thể tích).
Ta chọn đáp án D.

========================================================================

https://khoahoc.vietjack.com/thi-online/16-cau-trac-nghiem-toan-7-chan-troi-sang-tao-bai2-dien-tich-xung-quanh-va-the-tich-cua-hinh-hop-chu


\textbf{{QUESTION}}

Thể tích của hình lập phương có cạnh bằng 2 cm là:

\textbf{{ANSWER}}

Hướng dẫn giải
Đáp án đúng là: A
Thể tích của hình lập phương có cạnh bằng 2 cm là:
V = 23 = 8 (cm3).
Ta chọn đáp án A.

========================================================================

https://khoahoc.vietjack.com/thi-online/16-cau-trac-nghiem-toan-7-chan-troi-sang-tao-bai2-dien-tich-xung-quanh-va-the-tich-cua-hinh-hop-chu


\textbf{{QUESTION}}

Diện tích hai mặt đáy của hình lập phương có cạnh là a bằng:
A. 2a3;
B. 2a2;
C. 4a2;
D. 4a3.

\textbf{{ANSWER}}

Hướng dẫn giải
Đáp án đúng là: B
Diện tích một mặt đáy của hình lập phương có cạnh bằng a là a2 (đơn vị diện tích).
Khi đó diện tích hai mặt đáy là 2a2 (đơn vị diện tích).
Ta chọn đáp án B.

========================================================================

https://khoahoc.vietjack.com/thi-online/16-cau-trac-nghiem-toan-7-chan-troi-sang-tao-bai2-dien-tich-xung-quanh-va-the-tich-cua-hinh-hop-chu


\textbf{{QUESTION}}

Diện tích xung quanh của hình hộp chữ nhật có đáy với kích thước hai cạnh là 2 cm, 3 cm và chiều cao 4 cm là:

\textbf{{ANSWER}}

Hướng dẫn giải
Đáp án đúng là: D
Diện tích xung quanh của hình hộp chữ nhật có đáy với kích thước hai cạnh là 2 cm, 3 cm và chiều cao 4 cm là:
2.(2 + 3).4 = 40 cm2.
Vậy diện tích xung quanh của hình hộp chữ nhật đã cho là 40 cm2.

========================================================================

https://khoahoc.vietjack.com/thi-online/16-cau-trac-nghiem-toan-7-chan-troi-sang-tao-bai2-dien-tich-xung-quanh-va-the-tich-cua-hinh-hop-chu


\textbf{{QUESTION}}

Diện tích xung quanh của con xúc xắc có các cạnh bằng 2 cm bằng

\textbf{{ANSWER}}

Hướng dẫn giải
Đáp án đúng là: C
Con xúc xắc là một khối lập phương nên diện tích xung quanh của con xúc xắc này là: 4.22 = 16 (cm2).
Vậy diện tích xung quanh của con xúc xắc là 16 cm2.

========================================================================

https://khoahoc.vietjack.com/thi-online/11-cau-trac-nghiem-toan-8-bai-1-mo-dau-ve-phuong-trinh-co-dap-an-van-dung


\textbf{{QUESTION}}

Phương trình nào sau đây vô nghiệm?
A. x – 1 = 0 
B. 4x2 + 1 = 0
C. x2 – 3 = 6
D. x2 + 6x = -9

\textbf{{ANSWER}}

+) x – 1 = 0  x = 1
+) 4x2 + 1 = 0  4x2 = -1 (vô nghiệm vì 4x2 ≥ 0; Ɐx)
+) x2 – 3 = 6          x2 = 9  x = ± 3
+) x2 + 6x = -9  x2 + 6x + 9 = 0  (x + 3)2 = 0  x + 3 = 0  x = -3
Vậy phương trình 4x2 + 1 = 0 vô nghiệm
Đáp án cần chọn là: B

========================================================================

https://khoahoc.vietjack.com/thi-online/11-cau-trac-nghiem-toan-8-bai-1-mo-dau-ve-phuong-trinh-co-dap-an-van-dung


\textbf{{QUESTION}}

Phương trình 3x2-12x+4$$ \frac{3{x}^{2}-12}{x+4}$$ có tập nghiệm là
A. S = {±4}
B. S = {±2}
C. S = {2}
D. S = {4}

\textbf{{ANSWER}}

ĐKXĐ: x + 4 ≠ 0  x ≠ -4
Phương trình đã cho  3x2 – 12 = 0  x2 = 4  x = ±2 (tm)
Vậy tập nghiệm của phương trình là S = {±2}
Đáp án cần chọn là: B

========================================================================

https://khoahoc.vietjack.com/thi-online/11-cau-trac-nghiem-toan-8-bai-1-mo-dau-ve-phuong-trinh-co-dap-an-van-dung


\textbf{{QUESTION}}

Phương trình nào sau đây vô nghiệm?
A. 2x – 1 = 0
B. -x2 + 4 = 0
C. x2 + 3 = -6
D. 4x2 +4x = -1

\textbf{{ANSWER}}

+) 2x – 1 = 0 ⇔ x =  12
+) -x2 + 4 = 0 ⇔ x2 = 4 ⇔ x = ±2
+) x2 + 3 = -6 ⇔ x2 = -9 (vô nghiệm vì -9 < 0)
+) 4x2 + 4x = -1 ⇔ 4x2 +4x + 1 = 0  ⇔(2x + 1)2 = 0 ⇔ 2x + 1 = 0 ⇔ x =  -12
Đáp án cần chọn là: C

========================================================================

https://khoahoc.vietjack.com/thi-online/bai-tap-cuoi-chuong-6-co-dap-an-1


\textbf{{QUESTION}}

Cho mẫu số liệu: 
1        2        4        5        9        10      11
Số trung bình cộng của mẫu số liệu trên là:

\textbf{{ANSWER}}

Hướng dẫn giải
Đáp án đúng là: C.
Số trung bình cộng của mẫu số liệu đã cho là: 
$\overline x = \frac{{1 + 2 + 4 + 5 + 9 + 10 + 11}}{7} = 6$.

========================================================================

https://khoahoc.vietjack.com/thi-online/30-cau-trac-nghiem-toan-7-ket-noi-tri-thuc-bai-on-tap-cuoi-chuong-3-co-dap-an-phan-2/111222


\textbf{{QUESTION}}

Định lí là
A. Một khẳng định được suy ra từ những khẳng định đúng đã biết;
B. Một khẳng định được suy ra từ những khẳng định đúng chưa biết;
C. Một khẳng định được suy ra từ những khẳng định đã biết;

\textbf{{ANSWER}}

Hướng dẫn giải
Đáp án đúng là: A
Định lí là một khẳng định được suy ra từ những khẳng định đúng đã biết.

========================================================================

https://khoahoc.vietjack.com/thi-online/30-cau-trac-nghiem-toan-7-ket-noi-tri-thuc-bai-on-tap-cuoi-chuong-3-co-dap-an-phan-2/111222


\textbf{{QUESTION}}

Chứng minh định lí là 
A. Dùng lập luận để từ giả thiết suy ra kết luận;
B. Dùng hình vẽ để từ giả thiết suy ra kết luận;
C. Dùng lập luận để từ kết quả suy ra giả thiết;

\textbf{{ANSWER}}

Hướng dẫn giải
Đáp án đúng là: A
Chứng minh định lí là dùng lập luận để từ giả thiết suy ra kết luận.

========================================================================

https://khoahoc.vietjack.com/thi-online/30-cau-trac-nghiem-toan-7-ket-noi-tri-thuc-bai-on-tap-cuoi-chuong-3-co-dap-an-phan-2/111222


\textbf{{QUESTION}}

Hoàn thành định lí sau: “ Hai góc đối đỉnh thì...”
A. Bù nhau;
B. Bằng nhau;
C. Phụ nhau;
D. Khác nhau.

\textbf{{ANSWER}}

Hướng dẫn giải
Đáp án đúng là: B
Hai góc đối đỉnh thì bằng nhau.

========================================================================

https://khoahoc.vietjack.com/thi-online/30-cau-trac-nghiem-toan-7-ket-noi-tri-thuc-bai-on-tap-cuoi-chuong-3-co-dap-an-phan-2/111222


\textbf{{QUESTION}}

Giả thiết của định lí là:
A. Điều được suy ra;
B. Điều được lập luận;
C. Điều được cho biết;
D. Điều được tổng kết.

\textbf{{ANSWER}}

Hướng dẫn giải
Đáp án đúng là: C
Giả thiết của định lí là điều được cho biết.

========================================================================

https://khoahoc.vietjack.com/thi-online/30-cau-trac-nghiem-toan-7-ket-noi-tri-thuc-bai-on-tap-cuoi-chuong-3-co-dap-an-phan-2/111222


\textbf{{QUESTION}}

Trong các khẳng định sau, khẳng định nào đúng?
A. Trong một định lí, giả thiết là điều được suy ra;
B. Trong một định lí, kết luận là điều được suy ra;
C. Trong một định lí, có thể không có giả thiết;

\textbf{{ANSWER}}

Hướng dẫn giải
Đáp án đúng là: B
Trong một định lí, giả thiết là điều được cho biết. Khẳng định A sai.
Trong một định lí, kết luận là điều được suy ra. Khẳng định B đúng.
Trong một định lí, phải có cả giả thiết và kết luận. Khẳng định C và D sai.
Chọn đáp án B.

========================================================================

https://khoahoc.vietjack.com/thi-online/16-cau-trac-nghiem-toan-7-chan-troi-sang-tao-bai2-dien-tich-xung-quanh-va-the-tich-cua-hinh-hop-chu/105050


\textbf{{QUESTION}}

Một bể nước dạng hình hộp chữ nhật có các kích thước trong lòng bể là: chiều dài 4 m, chiều rộng 3,5 m và chiều cao 2,5 m. Biết $\frac{2}{5}$ bể đang chứa nước. Thể tích phần bể không chứa nước là

\textbf{{ANSWER}}

Hướng dẫn giải
Đáp án đúng là: A
Vì bể nước có hình hộp chữ nhật nên thể tích của bể nước là:
V = 4. 3,5 . 2,5 = 35 (m3).
Vì $\frac{2}{5}$ bể đang chứa nước nên thể tích phần bể chứa nước là: 
$\frac{2}{5}V = \frac{2}{5}.35$ = 14 (m3).
Vậy thể tích phần bể không chứa nước là:
35 – 14 = 21 (m3).
Ta chọn đáp án A.

========================================================================

https://khoahoc.vietjack.com/thi-online/16-cau-trac-nghiem-toan-7-chan-troi-sang-tao-bai2-dien-tich-xung-quanh-va-the-tich-cua-hinh-hop-chu/105050


\textbf{{QUESTION}}

Hình lập phương A có độ dài cạnh bằng 13 độ dài cạnh của hình lập phương B. Hỏi thể tích hình lập phương A bằng bao nhiêu phần thể tích hình lập phương B?

\textbf{{ANSWER}}

Hướng dẫn giải
Đáp án đúng là: C
Gọi chiều dài một cạnh của hình lập phương B là a.
Vì hình lập phương A có độ dài cạnh bằng 13$\frac{1}{3}$ độ dài cạnh của hình lập phương B nên độ dài cạnh của hình lập phương A là 13a$\frac{1}{3}a$.
Thể tích của hình lập phương B là: VB = a3 (đơn vị thể tích).
Thể tích của hình lập phương A là: 
VA=(13a)3=127a3${V_A} = {\left( {\frac{1}{3}a} \right)^3} = \frac{1}{{27}}{a^3}$ (đơn vị thể tích).
Suy ra VA=127VB${V_A} = \frac{1}{{27}}{V_B}$
Vậy thể tích của hình lập phương A bằng 127$\frac{1}{{27}}$ thể tích hình lập phương B.

========================================================================

https://khoahoc.vietjack.com/thi-online/16-cau-trac-nghiem-toan-7-chan-troi-sang-tao-bai2-dien-tich-xung-quanh-va-the-tich-cua-hinh-hop-chu/105050


\textbf{{QUESTION}}

Diện tích xung quanh của một hình lập phương là 196 cm3. Độ dài cạnh của hình lập phương đó là:

\textbf{{ANSWER}}

Hướng dẫn giải
Đáp án đúng là: B
Gọi độ dài cạnh của hình lập phương là a (cm) (a > 0).
Khi đó diện tích xung quanh của hình lập phương là 4a2 (cm2).
Mà diện tích xung quanh của hình lập phương bằng 196 cm3.
Do đó 4a2 = 196 nên a2 = 49.
Lại có 72 = 49 nên độ dài cạnh của hình lập phương bằng 7 cm.
Ta chọn đáp án B.

========================================================================

https://khoahoc.vietjack.com/thi-online/16-cau-trac-nghiem-toan-7-chan-troi-sang-tao-bai2-dien-tich-xung-quanh-va-the-tich-cua-hinh-hop-chu/105050


\textbf{{QUESTION}}

Cho hình hộp chữ nhật có chiều dài bằng 8 cm, chiều rộng bằng 12chiều dài và chiều cao bằng 2 lần chiều rộng. Thể tích của hình hộp chữ nhật đó là:

\textbf{{ANSWER}}

Hướng dẫn giải
Đáp án đúng là: A
Chiều rộng của hình hộp chữ nhật là: 8.12=4 (cm).
Chiều cao của hình hộp chữ nhật là: 4.2 = 8 (cm).
Vậy thể tích của hình hộp chữ nhật là: 8.4.8 = 256 (cm3).

========================================================================

https://khoahoc.vietjack.com/thi-online/16-cau-trac-nghiem-toan-7-chan-troi-sang-tao-bai2-dien-tich-xung-quanh-va-the-tich-cua-hinh-hop-chu/105050


\textbf{{QUESTION}}

Một chiếc hộp hình lập phương không có nắp được sơn cả mặt trong và mặt ngoài. Diện tích phải sơn tổng cộng là 1690 cm2. Thể tích của hình lập phương đó là:

\textbf{{ANSWER}}

Hướng dẫn giải
Đáp án đúng là: A
Một chiếc hộp hình lập phương không có nắp gồm có 5 mặt đều là hình vuông, mỗi hình vuông được sơn hai mặt nên chiếc hộp này được sơn tất cả 10 mặt là hình vuông.
Diện tích mỗi hình vuông là: 
1690 : 10 = 169 (cm2).
Gọi độ dài cạnh của hình vuông là a (cm) (a > 0).
Khi đó diện tích của hình vuông đó là a2 (cm2).
Do đó a2 = 169.
Mà 132 = 169 nên độ dài cạnh của hình lập phương đó là 13 cm.
Thể tích của hình lập phương là: 133 = 2197 (cm3)
Vậy thể tích của hình lập phương đó là: 2197 cm3.

========================================================================

https://khoahoc.vietjack.com/thi-online/10-bai-tap-tinh-xac-suat-cua-bien-co-hop-cua-hai-bien-co-bat-ki-bang-cach-su-dun


\textbf{{QUESTION}}

Cho hai biến cố A và B. Khẳng định nào sau đây là sai?

\textbf{{ANSWER}}

Đáp án đúng là: B
Với hai biến cố A, B bất kì ta có P(A È B) = P(A) + P(B) – P(A Ç B);
Nếu A và B là hai biến cố xung khắc thì P(A È B) = P(A) + P(B).
Vậy khẳng định B là sai.

========================================================================

https://khoahoc.vietjack.com/thi-online/10-bai-tap-tinh-xac-suat-cua-bien-co-hop-cua-hai-bien-co-bat-ki-bang-cach-su-dun


\textbf{{QUESTION}}

Cho A và B là hai biến cố thỏa mãn P(A) = 0,4; P(B) = 0,5; và P(A È B) = 0,6. Xác suất của biến cố A Ç B là

\textbf{{ANSWER}}

Đáp án đúng là: B
Ta có: P(A È B) = P(A) + P(B) – P(A Ç B) 
Do đó P(A Ç B) = P(A) + P(B) – P(A È B) = 0,4 + 0,5 – 0,6 = 0,3.

========================================================================

https://khoahoc.vietjack.com/thi-online/10-bai-tap-tinh-xac-suat-cua-bien-co-hop-cua-hai-bien-co-bat-ki-bang-cach-su-dun


\textbf{{QUESTION}}

Cho A và B là hai biến cố. Biết  P(A)=12;  P(B)=34 và  P(A∩B)=14. Biến cố A È B là biến cố

\textbf{{ANSWER}}

Đáp án đúng là: C
Với A và B là hai biến cố bất kì ta luôn có:
P(A∪B)=P(A)+P(B)−P(A∩B)=12+34−14=1.
Vậy A È B là biến cố chắc chắn.

========================================================================

https://khoahoc.vietjack.com/thi-online/10-bai-tap-tinh-xac-suat-cua-bien-co-hop-cua-hai-bien-co-bat-ki-bang-cach-su-dun


\textbf{{QUESTION}}

Một lớp học có 100 học sinh, trong đó có 40 học sinh giỏi ngoại ngữ; 30 học sinh giỏi tin học và 20 học sinh giỏi cả ngoại ngữ và tin học. Học sinh nào giỏi ít nhất một trong hai môn sẽ được thêm điểm trong kết quả học tập của học kì. Chọn ngẫu nhiên một trong các học sinh trong lớp, xác suất để học sinh đó được tăng điểm là
A. 310$$ \frac{3}{10}$$
B. 12$$ \frac{1}{2}$$
C. 25$$ \frac{2}{5}$$
D. 35$$ \frac{3}{5}$$

\textbf{{ANSWER}}

Đáp án đúng là: B
Gọi A là biến cố “học sinh được chọn được tăng điểm”.
Gọi B là biến cố “học sinh được chọn học giỏi ngoại ngữ”.
Gọi C là biến cố “học sinh được chọn học giỏi tin học”.
Khi đó A = B È C và BC là biến cố “học sinh chọn học giỏi cả ngoại ngữ lẫn tin học”.
Ta có:  P(A)=P(B)+P(C)−P(BC)=30100+40100−20100=12.$$ P\left(A\right)=P\left(B\right)+P\left(C\right)-P\left(BC\right)=\frac{30}{100}+\frac{40}{100}-\frac{20}{100}=\frac{1}{2}.$$

========================================================================

https://khoahoc.vietjack.com/thi-online/10-bai-tap-tinh-xac-suat-cua-bien-co-hop-cua-hai-bien-co-bat-ki-bang-cach-su-dun


\textbf{{QUESTION}}

Một khu phố có 50 hộ gia đình trong đó có 18 hộ nuôi chó, 16 hộ nuôi mèo và 7 hộ nuôi cả chó và mèo. Chọn ngẫu nhiên một hộ trong khu phố trên. Xác suất để hộ đó nuôi chó hoặc nuôi mèo là

\textbf{{ANSWER}}

Đáp án đúng là: B
Gọi các biến cố A: "Chọn được hộ nuôi chó", và B: "Chọn được hộ nuôi mèo". Ta có: 
 P(A)=1850=925,  P(B)=1650=825,  P(A∩B)=750.
Xác suất để chọn được hộ nuôi chó hoặc nuôi mèo là: 
 P(A∪B)=P(A)+P(B)−P(A∩B)=925+825−750=2750=0,54.

========================================================================

https://khoahoc.vietjack.com/thi-online/bai-tap-toan-8-chu-de-3-rut-gon-phan-thuc-co-dap-an/107800


\textbf{{QUESTION}}

Tìm giá trị nguyên của u để tại đó giá trị của mỗi biểu thức sau là một số nguyên $$ \frac{3}{u-2}$$ với $$ u\quad \ne 2$$

\textbf{{ANSWER}}

Để $$ \frac{3}{u-2}$$nguyên thì $$ 3\vdots u-2\Rightarrow u-2\in Ư\left(3\right)=\{\pm 1;\pm 3\}$$
Ta có bảng
u-2
-1
1
-3
3
u
1 (TM)
3 (TM)
5 (TM)
-1 (TM)
 
Vậy $$ x\in \{1;3;5;-1\}$$  thì $$ \frac{3}{u-2}$$  nguyên.

========================================================================

https://khoahoc.vietjack.com/thi-online/bai-tap-toan-8-chu-de-3-rut-gon-phan-thuc-co-dap-an/107800


\textbf{{QUESTION}}

Tìm giá trị nguyên của u để tại đó giá trị của mỗi biểu thức sau là một số nguyên 3u2−2u+13u+1$$ \frac{3{u}^{2}-2u+1}{3u+1}$$ với u ≠-13$$ u\quad \ne \frac{-1}{3}$$

\textbf{{ANSWER}}

3u2-2u+13u+1 nguyên thì 2⋮3u+1⇒3u+1∈Ư (2)={±1;±2}$$ \frac{3{u}^{2}-2u+1}{3u+1}\quad nguyên\quad thì\quad 2\vdots 3u+1\Rightarrow 3u+1\in Ư\quad \left(2\right)=\{\pm 1;\pm 2\}$$
Ta có bảng
3u+1
-1
1
-2
2
u
 
(KTM)
0
(TM)
-1
(TM)
 
(KTM)

========================================================================

https://khoahoc.vietjack.com/thi-online/bai-tap-toan-8-chu-de-3-rut-gon-phan-thuc-co-dap-an/107800


\textbf{{QUESTION}}

A=x2−2x+2019x2
A=x2−2x+2019x2
A
=
x2−2x+2019x2
x2−2x+2019
x2−2x+2019
x2
x
2
−
2
x
+
2019
x2
x2


x2
x2
x2
x
2
x>0
x>0
x
>
0

\textbf{{ANSWER}}

A=2019x2-2.x.2019+201922019x2=(x-2019)22019x2+2018x22019x2=(x-2019)22019x2+20182019
Vì (x-2019)22019x2⩾0∀x>0⇒(x-2019)22019x2+20182019⩾20182019Dấu "=" xảy ra khi x-2019=0⇒x=2019Vậy Min A=20182019 khi x=2019

========================================================================

https://khoahoc.vietjack.com/thi-online/bai-tap-toan-8-chu-de-3-rut-gon-phan-thuc-co-dap-an/107800


\textbf{{QUESTION}}

Tìm giá trị lớn nhất của   B=3x2+9x+173x2+9x+7

\textbf{{ANSWER}}

B=3x2+9x+173x2+9x+7=1+103x2+9x+7=1+103(x+32)2+14
Để B lớn nhất ⇒3.(x+32)2+14 nhỏ nhất
Mà 3.(x+32)2+14≥14 vì 3.(x+32)2≥0∀x∈RDấu "=" xảy ra khi x+32=0⇒x=-32Vậy Max B=41 khi x=-32

========================================================================

https://khoahoc.vietjack.com/thi-online/giai-vbt-toan-7-cd-bai-4-lam-tron-va-uoc-luong-co-dap-an


\textbf{{QUESTION}}

Ở nhiều tình huống thực tiễn, ta cần tìm một số thực khác xấp xỉ với số thực đã cho để thuận tiện hơn trong ghi nhớ, đo đạc hay tính toán. Số thực tìm được như thế được gọi là…………………………………………………………………………………………

\textbf{{ANSWER}}

Ở nhiều tình huống thực tiễn, ta cần tìm một số thực khác xấp xỉ với số thực đã cho để thuận tiện hơn trong ghi nhớ, đo đạc hay tính toán. Số thực tìm được như thế được gọi là số làm tròn của số thực đã cho.

========================================================================

https://khoahoc.vietjack.com/thi-online/giai-vbt-toan-7-cd-bai-4-lam-tron-va-uoc-luong-co-dap-an


\textbf{{QUESTION}}

Ta nói số a được làm tròn đến số b với độ chính xác d nếu ……………………………

\textbf{{ANSWER}}

Ta nói số a được làm tròn đến số b với độ chính xác d nếu khoảng cách giữa điểm a và điểm b trên trục số không vượt quá d.

========================================================================

https://khoahoc.vietjack.com/thi-online/giai-vbt-toan-7-cd-bai-4-lam-tron-va-uoc-luong-co-dap-an


\textbf{{QUESTION}}

Quãng đường từ sân vận động Old Trafford ở Greater Manchester đến tháp đồng hồ Big Ben ở London (Vương quốc Anh) khoảng 200 dặm.
(Nguồn: https://google.com/maps)
Tính độ dài quãng đường đó theo đơn vị ki–lô–mét (làm tròn kết quả đến hàng đơn vị), biết 1 dặm = 1,609344 km.

\textbf{{ANSWER}}

Quãng đường từ sân vận động Old Trafford ở Greater Manchester đến tháp đồng hồ Big Ben ở London (Vương quốc Anh) tính theo đơn vị ki–lô–mét là:
200 . 1,609344 = 321,8688 ≈ 322 (km).

========================================================================

https://khoahoc.vietjack.com/thi-online/giai-vbt-toan-7-cd-bai-4-lam-tron-va-uoc-luong-co-dap-an


\textbf{{QUESTION}}

a) Làm tròn số 23 615 với độ chính xác 5.

\textbf{{ANSWER}}

a) Để làm tròn số 23 615 với độ chính xác 5, ta sẽ làm tròn số đó đến hàng chục. Nhận thấy chữ số hàng đơn vị là 5 nên ta tăng thêm chữ số hàng chục một đơn vị và thay chữ số hàng đơn vị bởi chữ số 0.
Vậy 23 615 ≈ 23 620.

========================================================================

https://khoahoc.vietjack.com/thi-online/trac-nghiem-chuyen-de-toan-8-chu-de-2-kiem-tra-hoc-ki-2-co-dap-an/102990


\textbf{{QUESTION}}

Giải phương trình sau đây :
8( 3x - 2 ) - 14x = 2( 4 – 7x ) + 15x

\textbf{{ANSWER}}

8( 3x - 2 ) - 14x = 2( 4 – 7x ) + 15x
⇔ 24x – 16 -14x = 8 – 14x + 15x
⇔ 10x -16 = 8 + x
⇔ 9x = 24
⇔ x = 24/9

========================================================================

https://khoahoc.vietjack.com/thi-online/10-bai-tap-tong-cusa-n-so-hang-dau-tien-cua-mot-cap-so-nhan-co-loi-giai


\textbf{{QUESTION}}

Tổng   $$ \text{S}=3+{\left(-3\right)}^{2}+{3}^{3}+{\left(-3\right)}^{4}+\dots +{\left(-3\right)}^{20}\quad $$có giá trị là
$$ A.\frac{3\left({3}^{20}-1\right)}{2}$$
$$ B.\text{\hspace{0.33em}}{3}^{30}-1$$
$$ C.\frac{2\left({3}^{20}-1\right)}{3}$$
$$ D.10\cdot \left({3}^{20}-3\right)$$

\textbf{{ANSWER}}

Đáp án đúng là: A
Ta có: $$ \text{S}=3+{\left(-3\right)}^{2}+{3}^{3}+{\left(-3\right)}^{4}+\dots +{\left(-3\right)}^{20}$$
$$ \Leftrightarrow \text{S}=3+{3}^{2}+{3}^{3}+{3}^{4}+\dots +{3}^{20}$$
 
Nhận xét: Dãy số 3; 32; 33; 34; ... ; 320 là cấp số nhân với số hạng đầu là u1 = 3 và công bội q = 3.
Do đó, tổng 20 số hạng của dãy số là: $$ S=\frac{3\cdot \left(1-{3}^{20}\right)}{1-3}=\frac{3\left({3}^{20}-1\right)}{2}$$ .

========================================================================

https://khoahoc.vietjack.com/thi-online/10-bai-tap-tong-cusa-n-so-hang-dau-tien-cua-mot-cap-so-nhan-co-loi-giai


\textbf{{QUESTION}}

Tổng   S=(−1)+(−1)2+(−1)3+⋯+(−1)41 có giá trị là

\textbf{{ANSWER}}

Đáp án đúng là: B
Ta có dãy số (–1); (–1)2; (–1)3; …; (–1)41 là cấp số nhân gồm 41 số hạng với số hạng đầu là u1 = −1 và công bội q = −1.
Do đó S=(−1)⋅[1−(−1)41]1−(−1)=(−1)⋅[1+1]1+1=−1$$ S=\frac{(-1)\cdot \left[1-{(-1)}^{41}\right]}{1-(-1)}=\frac{(-1)\cdot \left[1+1\right]}{1+1}=-1$$ .

========================================================================

https://khoahoc.vietjack.com/thi-online/12-cau-trac-nghiem-gia-tri-luong-giac-cua-mot-goc-bat-ki-tu-0o-den-150o-co-dap-an-nhan-biet


\textbf{{QUESTION}}

Trong các đẳng thức sau đây, đẳng thức nào đúng?
A. $$ \mathrm{sin}\quad ({180}^{\circ }\quad -\quad \alpha )\quad =\quad -\quad \mathrm{sin}\quad \alpha $$
B. $$ \mathrm{cos}\quad ({180}^{\circ }\quad -\quad \alpha )\quad =\quad \mathrm{cos}\quad \alpha $$
C. $$ \mathrm{tan}\quad ({180}^{\circ }\quad -\quad \alpha )\quad =\quad \mathrm{tan}\quad \alpha $$
D. $$ cot\quad ({180}^{\circ }\quad -\quad \alpha )\quad =\quad -cot\quad \alpha $$

\textbf{{ANSWER}}

Đáp án D
Vì α và $$ {180}^{\circ }-\alpha $$ là hai góc bù nhau nên $$ cot({180}^{\circ }-\alpha )=-cot\quad \alpha $$

========================================================================

https://khoahoc.vietjack.com/thi-online/12-cau-trac-nghiem-gia-tri-luong-giac-cua-mot-goc-bat-ki-tu-0o-den-150o-co-dap-an-nhan-biet


\textbf{{QUESTION}}

Trong các đẳng thức sau, đẳng thức nào đúng?
A. sin(180∘−α)=− cos α
B. sin(180∘−α)=−sin α
C. sin(180∘−α)=sin α
D. sin(180∘−α)=cosα

\textbf{{ANSWER}}

Đáp án C
Hai góc bù nhau α và (1800−α)$$ (1800-\alpha )$$ thì cho có giá trị của sin bằng nhau

========================================================================

https://khoahoc.vietjack.com/thi-online/12-cau-trac-nghiem-gia-tri-luong-giac-cua-mot-goc-bat-ki-tu-0o-den-150o-co-dap-an-nhan-biet


\textbf{{QUESTION}}

Cho góc α tù. Điều khẳng định nào sau đây là đúng?
A. sin α < 0
B. cos α > 0
C. tan α > 0
D. cot α < 0

\textbf{{ANSWER}}

Đáp án D
Vì α tù nên sin α > 0, cos α < 0, tan α < 0, cot α < 0

========================================================================

https://khoahoc.vietjack.com/thi-online/12-cau-trac-nghiem-gia-tri-luong-giac-cua-mot-goc-bat-ki-tu-0o-den-150o-co-dap-an-nhan-biet


\textbf{{QUESTION}}

Hai góc nhọn α và β phụ nhau, hệ thức nào sau đây là sai?
A. sin α = cos β
B. tan α = cot β
C. cotβ=1cotα$$ cot\beta =\frac{1}{\mathrm{cot}\alpha }$$
D. cos α = − sin β

\textbf{{ANSWER}}

Đáp án D
Vì α, β là hai góc phụ nhau nên cos α = sin β
Do đó đáp án D sai.
Ngoài ra các đáp án A, B, C đều đúng theo tính chất của hai góc phụ nhau (sin góc này bằng cos góc kia, tan góc này bằng cot góc kia)

========================================================================

https://khoahoc.vietjack.com/thi-online/12-cau-trac-nghiem-gia-tri-luong-giac-cua-mot-goc-bat-ki-tu-0o-den-150o-co-dap-an-nhan-biet


\textbf{{QUESTION}}

Cho hai góc nhọn α và β phụ nhau. Hệ thức nào sau đây là sai?
A. sin α = − cos β
B. cos α = sin β
C. tan α = cot β
D. cot α = tan β

\textbf{{ANSWER}}

Đáp án A
Hai góc nhọn α và β phụ nhau thì sin α = cos β; cos α = sin β; tan α = cot β; cot α = tan β

========================================================================

https://khoahoc.vietjack.com/thi-online/10-bai-tap-bai-toasn-thuc-tien-lien-quan-den-the-tich-co-loi-giai


\textbf{{QUESTION}}

Một hồ bơi hình hộp chữ nhật có đáy là hình vuông cạnh bằng 50 m. Lượng nước trong hồ cao 1,5m. Thể tích nước trong hồ là:

\textbf{{ANSWER}}

Đáp án đúng là: A
Thể tích nước trong hồ là: V = 1,5.502 = 3750 m3.

========================================================================

https://khoahoc.vietjack.com/thi-online/10-bai-tap-bai-toasn-thuc-tien-lien-quan-den-the-tich-co-loi-giai


\textbf{{QUESTION}}

Một khối gỗ có dạng là lăng trụ, biết diện tích đáy và chiều cao lần lượt là 0,25 m2 và 1,2 m. Mỗi mét khối gỗ này trị giá 5 triệu đồng. Hỏi khối gỗ đó có giá trị bao nhiêu tiền?

\textbf{{ANSWER}}

Đáp án đúng là: C
Thể tích khối gỗ là: V = S.h = 0,25.1,2 = 0,3 (m3).
Vậy khối gỗ đó có giá là: 0,3.5 000 000 = 1 500 000 đồng.

========================================================================

https://khoahoc.vietjack.com/thi-online/20-cau-trac-nghiem-toan-12-ket-noi-tri-thuc-bai-12-tich-phan-co-dap-an


\textbf{{QUESTION}}

I. Nhận biết
Cho hàm số $y = f\left( x \right)$ liên tục trên đoạn $\left[ {a;b} \right]$. Gọi $F\left( x \right)$ là một nguyên hàm của hàm số $f\left( x \right)$ trên đoạn $\left[ {a;b} \right]$. Chọn mệnh đề đúng.
A. $\int\limits_a^b {f\left( x \right)dx = F\left( b \right) - F\left( a \right).} $
B. $\int\limits_a^b {f\left( x \right)dx = F\left( b \right) + F\left( a \right).} $
C. $\int\limits_a^b {f\left( x \right)dx = F\left( a \right) - F\left( b \right).} $
D. $\int\limits_a^b {f\left( x \right)dx = {F^2}\left( b \right) - {F^2}\left( a \right).} $

\textbf{{ANSWER}}

Đáp án đúng là: A
Ta có: $\int\limits_a^b {f\left( x \right)dx = \left. {F\left( x \right)} \right|_a^b = F\left( b \right) - F\left( a \right).} $

========================================================================

https://khoahoc.vietjack.com/thi-online/20-cau-trac-nghiem-toan-12-ket-noi-tri-thuc-bai-12-tich-phan-co-dap-an


\textbf{{QUESTION}}

Cho hàm số y=f(x)$y = f\left( x \right)$ liên tục trên đoạn [a;b]$\left[ {a;b} \right]$. Gọi F(x)$F\left( x \right)$ là một nguyên hàm của hàm số f(x)$f\left( x \right)$ trên đoạn [a;b]$\left[ {a;b} \right]$. Chọn mệnh đề sai.
A. $\int\limits_a^b {f\left( x \right)dx = F\left( b \right) - F\left( a \right).} $
B. a∫af(x)dx=1.$\int\limits_a^a {f\left( x \right)dx = 1.} $
C. a∫af(x)dx=0.$\int\limits_a^a {f\left( x \right)dx = 0.} $
D. b∫af(x)dx=−a∫bf(x)dx.$\int\limits_a^b {f\left( x \right)dx = - } \int\limits_b^a {f\left( x \right)dx} .$

\textbf{{ANSWER}}

Đáp án đúng là: B

========================================================================

https://khoahoc.vietjack.com/thi-online/20-cau-trac-nghiem-toan-12-ket-noi-tri-thuc-bai-12-tich-phan-co-dap-an


\textbf{{QUESTION}}

Cho hàm số y=f(x) có đạo hàm f′(x) và f′(x) liên tục trên đoạn [a;b]. Gọi F(x) là một nguyên hàm của hàm số f(x) trên đoạn [a;b]. Chọn mệnh đề đúng.
A. b∫af′(x)dx=f(b)−f(a).
B. b∫af′(x)dx=F(b)−F(a).
C. b∫aF(x)dx=f(b)−f(a).
D. b∫af′(x)dx=f′(b)−f′(a).

\textbf{{ANSWER}}

Đáp án đúng là: A
Ta có: b∫af′(x)dx=f(x)|ba=f(b)−f(a).$\int\limits_a^b {f'\left( x \right)dx = \left. {f\left( x \right)} \right|_a^b = f\left( b \right) - f\left( a \right).} $

========================================================================

https://khoahoc.vietjack.com/thi-online/20-cau-trac-nghiem-toan-12-ket-noi-tri-thuc-bai-12-tich-phan-co-dap-an


\textbf{{QUESTION}}

Cho hàm số y=f(x), y=g(x) liên tục trên [a;b],k là hằng số . Xét các mệnh đề sau:
a) b∫a[f(x)+g(x)]dx=b∫af(x)dx+b∫ag(x)dx.
b) b∫af(x).g(x)dx=b∫af(x)dx.b∫ag(x)dx.
c) b∫akf(x)dx=kb∫af(x)dx.
d) b∫af(x)g(x)dx=b∫af(x)dxb∫ag(x)dx.
Số mệnh đề đúng là
A. 1.
B. 2.
C. 3.
D. 4.

\textbf{{ANSWER}}

Đáp án đúng là: B
Ta có mệnh đề a và c là mệnh đề đúng.

========================================================================

https://khoahoc.vietjack.com/thi-online/20-cau-trac-nghiem-toan-12-ket-noi-tri-thuc-bai-12-tich-phan-co-dap-an


\textbf{{QUESTION}}

Cho hàm số \[y = f\left( x \right)\] liên tục trên \[\mathbb{R}\] và \[a,b,c \in \mathbb{R}\] thỏa mãn \[a < b < c\]. Trong các mệnh đề dưới đây, mệnh đề đúng là
A. \[\int\limits_a^c {f\left( x \right)dx = \int\limits_a^b {f\left( x \right)dx.\int\limits_b^c {f\left( x \right)dx.} } } \]
B. \[\int\limits_a^c {f\left( x \right)dx = \int\limits_a^b {f\left( x \right)dx + \int\limits_b^c {f\left( x \right)dx.} } } \]
C. \[\int\limits_a^c {f\left( x \right)dx = \int\limits_a^b {f\left( x \right)dx - \int\limits_b^c {f\left( x \right)dx.} } } \]
D. \[\int\limits_a^c {f\left( x \right)dx = \int\limits_a^b {f\left( x \right)dx + \int\limits_c^b {f\left( x \right)dx.} } } \]

\textbf{{ANSWER}}

Đáp án đúng là: B
Với hàm số y=f(x) liên tục trên R và a,b,c∈R thỏa mãn a<b<c thì
c∫af(x)dx=b∫af(x)dx+c∫bf(x)dx.

========================================================================

https://khoahoc.vietjack.com/thi-online/10-bai-tap-bai-toan-sthuc-tien-lien-quan-den-ham-so-mu-va-ham-so-logarit-co-loi-giai


\textbf{{QUESTION}}

Số lượng của loại vi khuẩn A trong một phòng thí nghiệm được tính theo công thức s(t) = s(0).2t, trong đó s(0) là số lượng vi khuẩn A lúc ban đầu, s(t) là số lượng vi khuẩn A có sau t phút. Biết sau 3 phút thì số lượng vi khuẩn A là 625 nghìn con. Hỏi sau bao lâu, kể từ lúc ban đầu, số lượng vi khuẩn A là 10 triệu con?

\textbf{{ANSWER}}

Đáp án đúng là: C
Sau 3 phút thì số lượng vi khuẩn A là 625 nghìn con nên ta có:
625 000 = s(0).23 Þ s(0) = 78125 (con vi khuẩn).
Tại thời điểm t số lượng vi khuẩn A là 10 triệu con nên ta có:
 $$ s\left(t\right)=s\left(0\right){.2}^{t}\Leftrightarrow {2}^{t}=\frac{s\left(t\right)}{s\left(0\right)}\Leftrightarrow {2}^{t}=\frac{10\text{\hspace{0.33em}}000\text{\hspace{0.33em}}000}{78\text{\hspace{0.33em}}125}\Leftrightarrow {2}^{t}=128\Leftrightarrow t=7$$.
Vậy sau 7 phút kể từ lúc ban đầu, số lượng vi khuẩn A là 10 triệu con.

========================================================================

https://khoahoc.vietjack.com/thi-online/10-bai-tap-bai-toan-sthuc-tien-lien-quan-den-ham-so-mu-va-ham-so-logarit-co-loi-giai


\textbf{{QUESTION}}

Bà Mai gửi tiết kiệm ngân hàng Vietcombank số tiền 50 triệu đồng với lãi suất 0,79% một tháng, theo phương thức lãi kép. Tính số tiền cả vốn lẫn lãi bà Mai nhận được sau 2 năm? (làm tròn đến hàng nghìn)

\textbf{{ANSWER}}

Đáp án đúng là: A
Nhận xét: đây là bài toán lãi kép với chu kỳ là một tháng.
Áp dụng công thức: S = A . (1 + r)n với A = 50 triệu đồng, r% = 0,79% và n = 24 tháng.
Ta có: 
S = 50 . (1 + 0,0079)24 » 60,393 (triệu đồng) = 60 393 000 đồng.

========================================================================

https://khoahoc.vietjack.com/thi-online/10-bai-tap-bai-toan-sthuc-tien-lien-quan-den-ham-so-mu-va-ham-so-logarit-co-loi-giai


\textbf{{QUESTION}}

Bạn An gửi tiết kiệm một số tiền ban đầu là 1 000 000 đồng với lãi suất 0,58%/tháng (không kỳ hạn). Hỏi bạn An phải gửi bao nhiêu tháng thì được cả vốn lẫn lãi bằng hoặc vượt quá 1 300 000 đồng?

\textbf{{ANSWER}}

Đáp án đúng là: A
Ta có:  n=log1,0058(13000001000000)≈45,3662737(tháng).
Để nhận được số tiền cả vốn lẫn lãi bằng hoặc vượt quá 1 300 000 đồng thì bạn An phải gửi ít nhất là 46 tháng.

========================================================================

https://khoahoc.vietjack.com/thi-online/10-bai-tap-bai-toan-sthuc-tien-lien-quan-den-ham-so-mu-va-ham-so-logarit-co-loi-giai


\textbf{{QUESTION}}

Một người được lãnh lương khởi điểm là 3 triệu đồng/tháng. Cứ 3 tháng thì lương người đó được tăng thêm  7%/tháng. Hỏi sau 36 năm người đó lĩnh được tất cả số tiền là bao nhiêu?

\textbf{{ANSWER}}

Đáp án đúng là: B
Áp dụng công thức: Tổng số tiền nhận được sau kn tháng là  Skn=Ak(1+r)k−1r.
 S36=3.106.12.(1,07)12−10,07≈643984245,8≈644 000 000 (đồng)
Vậy sau 36 năm người đó lĩnh được tất cả số tiền là 644 triệu đồng.

========================================================================

https://khoahoc.vietjack.com/thi-online/10-bai-tap-bai-toan-sthuc-tien-lien-quan-den-ham-so-mu-va-ham-so-logarit-co-loi-giai


\textbf{{QUESTION}}

Một người có 58 000 000đ gửi tiết kiệm ngân hàng (theo hình thức lãi kép ) trong 8 tháng thì lĩnh về được 61 329 000đ. Lãi suất hàng tháng là:

\textbf{{ANSWER}}

Đáp án đúng là: B
Áp dụng công thức lãi kép ta được: 
 r%=61329000580000008-1≈0,7%
Vậy lãi suất hàng tháng là 0,7%.

========================================================================

https://khoahoc.vietjack.com/thi-online/tong-hop-de-thi-chinh-thuc-vao-10-mon-toan-nam-2021-co-dap-an-phan-1/104435


\textbf{{QUESTION}}

Rút gọn các biểu thức sau :

\textbf{{ANSWER}}

$$ a)A=3\sqrt{2}-\sqrt{32}+\sqrt{50}=3\sqrt{2}-4\sqrt{2}+5\sqrt{2}=4\sqrt{2}$$
Vậy $$ A=4\sqrt{2}$$

========================================================================

https://khoahoc.vietjack.com/thi-online/tong-hop-de-thi-chinh-thuc-vao-10-mon-toan-nam-2021-co-dap-an-phan-1/104435


\textbf{{QUESTION}}

b)B=(1√x−2−√xx−4):1√x+2(x≥0x≠4)

\textbf{{ANSWER}}

b)B=(1√x−2−√xx−4):1√x+2$$ B=\left(\frac{1}{\sqrt{x}-2}-\frac{\sqrt{x}}{x-4}\right):\frac{1}{\sqrt{x}+2}$$
Với x≥0,x≠4$$ x\ge \mathrm{0,}x\ne 4$$ ta có :
B=(1√x−2−√xx−4):1√x+2=√x+2−√x(√x−2)(√x+2).√x+21=2√x−2$$ B=\left(\frac{1}{\sqrt{x}-2}-\frac{\sqrt{x}}{x-4}\right):\frac{1}{\sqrt{x}+2}=\frac{\sqrt{x}+2-\sqrt{x}}{\left(\sqrt{x}-2\right)\left(\sqrt{x}+2\right)}.\frac{\sqrt{x}+2}{1}=\frac{2}{\sqrt{x}-2}$$

========================================================================

https://khoahoc.vietjack.com/thi-online/giai-sbt-toan-10-bai-3-duong-tron-trong-mat-phang-toa-do-co-dap-an


\textbf{{QUESTION}}

Phương trình nào trong các phương trình sau đây là phương trình đường tròn? Tìm toạ độ tâm và bán kính của đường tròn đó.
a) x2 + y2 + 2x + 2y – 9 = 0;

\textbf{{ANSWER}}

a) x2 + y2 + 2x + 2y – 9 = 0 (1)
Phương trình (1) có dạng x2 + y2 – 2ax – 2by + c = 0 với a = – 1; b = – 1; c = – 9.
Ta có a2 + b2 – c = (– 1)2 + (– 1)2 – (– 9) = 11 > 0.
Vậy (1) là phương trình đường tròn tâm I(– 1; – 1) bán kính R = $$ \sqrt{11}$$.

========================================================================

https://khoahoc.vietjack.com/thi-online/giai-sbt-toan-10-bai-3-duong-tron-trong-mat-phang-toa-do-co-dap-an


\textbf{{QUESTION}}

b) x2 + y2 – 6x – 2y + 1 = 0;

\textbf{{ANSWER}}

b) x2 + y2 – 6x – 2y + 1 = 0 (2).
Phương trình (2) có dạng x2 + y2 – 2ax – 2by + c = 0 với a = 3; b = 1; c = 1.
Ta có a2 + b2 – c = 32 + 12 – 1 = 9 > 0.
Vậy (2) là phương trình đường tròn tâm I(3; 1) bán kính R = 3.

========================================================================

https://khoahoc.vietjack.com/thi-online/giai-sbt-toan-10-bai-3-duong-tron-trong-mat-phang-toa-do-co-dap-an


\textbf{{QUESTION}}

c) x2 + y2 + 8x + 4y + 2022 = 0;

\textbf{{ANSWER}}

c) x2 + y2 + 8x + 4y + 2022 = 0 (3).
Phương trình (3) có dạng x2 + y2 – 2ax – 2by + c = 0 với a = – 4; b = – 2; c = 2022.
Ta có a2 + b2 – c = (– 4)2 + (– 2)2 – 2022 = – 2002 < 0.
Vậy (3) không là phương trình đường tròn.

========================================================================

https://khoahoc.vietjack.com/thi-online/giai-sbt-toan-10-bai-3-duong-tron-trong-mat-phang-toa-do-co-dap-an


\textbf{{QUESTION}}

d) 3x2 + 2y2 + 5x + 7y – 1 = 0.

\textbf{{ANSWER}}

d) 3x2 + 2y2 + 5x + 7y – 1 = 0 (4).
Phương trình (4) không phải là phương trình đường tròn vì không thể đưa về dạng (x – a)2 + (y – b)2 = R2 hoặc dạng x2 + y2 – 2ax – 2by + c = 0.

========================================================================

https://khoahoc.vietjack.com/thi-online/giai-sbt-toan-10-bai-3-duong-tron-trong-mat-phang-toa-do-co-dap-an


\textbf{{QUESTION}}

Viết phương trình đường tròn (C) trong các trường hợp sau:
a) (C) có tâm O(0; 0) và có bán kính R = 9;

\textbf{{ANSWER}}

a) Đường trong (C) tâm O(0; 0) và có bán kính R = 9 có phương trình là:
x2 + y2 = 81

========================================================================

https://khoahoc.vietjack.com/thi-online/trac-nghiem-chuyen-de-toan-8-chu-de-1-mo-dau-ve-phuong-trinh-co-dap-an


\textbf{{QUESTION}}

Tìm tập nghiệm của các phương trình sau đây?
a) - 3x = $$ \frac{-\quad 7}{2}$$

\textbf{{ANSWER}}

a) Ta có: - 3x = $$ \frac{-\quad 7}{2}$$
⇔ x = $$ \frac{(\quad {\displaystyle \frac{-\quad 7}{2}})\quad }{-\quad 3}=\quad \frac{7}{6}.$$
Vậy tập nghiệm của phương trình là S = { $$ \frac{7}{6}$$ }

========================================================================

https://khoahoc.vietjack.com/thi-online/bai-tap-chuyen-de-toan-6-dang-3-hai-bai-toan-ve-phan-so-co-dap-an/107667


\textbf{{QUESTION}}

Tuấn có 21 viên bi. Tuấn cho Dũng $$ \frac{3}{7}$$  số bi của mình. Hỏi:
a) Dũng được Tuấn cho bao nhiêu viên bi?

\textbf{{ANSWER}}

a) Số bi Dũng được Tuấn cho là: $$ 21.\frac{3}{7}=9$$  (viên bi)

========================================================================

https://khoahoc.vietjack.com/thi-online/bai-tap-chuyen-de-toan-6-dang-3-hai-bai-toan-ve-phan-so-co-dap-an/107667


\textbf{{QUESTION}}

b) Tuấn còn lại bao nhiêu viên bi?

\textbf{{ANSWER}}

b) Số bi Tuấn còn lại là: 21- 9 = 12 (viên bi)

========================================================================

https://khoahoc.vietjack.com/thi-online/bai-tap-chuyen-de-toan-6-dang-3-hai-bai-toan-ve-phan-so-co-dap-an/107667


\textbf{{QUESTION}}

Một quả cam nặng 325g. Hỏi 35  quả cam nặng bao nhiêu?

\textbf{{ANSWER}}

35 quả cam nặng: 325.35=195 (g)

========================================================================

https://khoahoc.vietjack.com/thi-online/bai-tap-chuyen-de-toan-6-dang-3-hai-bai-toan-ve-phan-so-co-dap-an/107667


\textbf{{QUESTION}}

Một quả cam nặng 300g. Hỏi 34$$ \frac{3}{4}$$ quả cam nặng bao nhiêu ?

\textbf{{ANSWER}}

34$$ \frac{3}{4}$$ quả cam nặng:      300 . 34$$ \frac{3}{4}$$  = 225 (g)

========================================================================

https://khoahoc.vietjack.com/thi-online/bai-tap-chuyen-de-toan-6-dang-3-hai-bai-toan-ve-phan-so-co-dap-an/107667


\textbf{{QUESTION}}

Đoạn đường Hà Nôi - Hải Phòng dài 102 km. Một xe lửa xuất phát từ xuất phát từ Hà Nội đã đi được 35 quãng đường. Hỏi xe lửa còn cách Hải Phòng bao nhiêu ki – lô – mét.

\textbf{{ANSWER}}

Đoạn đường xe lửa đã đi được là: 102.35=61,2  (km)
Đoạn đường còn lại cách Hải Phòng số km là: 102−61,2=40,8  (km)

========================================================================

https://khoahoc.vietjack.com/thi-online/10-bai-taps-nhan-don-thuc-voi-don-thuc-don-thuc-voi-da-thuc-co-loi-giai


\textbf{{QUESTION}}

Kết quả phép tính 5x2y6z5 . 2xy3 là
A. 10x2y18z5;

\textbf{{ANSWER}}

Hướng dẫn giải:
Đáp án đúng là: C
Ta có: 5x2y6z5 . 2xy3 = (5.2) . (x2.x) . (y6.y3) . z5 = 10x3y9z5.

========================================================================

https://khoahoc.vietjack.com/thi-online/10-bai-taps-nhan-don-thuc-voi-don-thuc-don-thuc-voi-da-thuc-co-loi-giai


\textbf{{QUESTION}}

A. – 3xy + 6x;

\textbf{{ANSWER}}

Hướng dẫn giải:
Đáp án đúng là: A
Ta có: ‐34x.(4y-8)=(‐34x).4y-(‐34x).8=‐3xy+6x

========================================================================

https://khoahoc.vietjack.com/thi-online/20-cau-trac-nghiem-toan-7-chan-troi-sang-tao-bai-tap-cuoi-chuong-6-co-dap-an-phan-2


\textbf{{QUESTION}}

Cho bốn số 3, 2, x, y với x, y ≠ 0 và 3x = 2y, một tỉ lệ thức đúng được thiết lập từ bốn số trên là

\textbf{{ANSWER}}

Đáp án đúng là: C
Ta có 3x = 2y nên $$ \frac{3}{2}=\frac{y}{x}$$.
Vậy ta chọn phương án C.

========================================================================

https://khoahoc.vietjack.com/thi-online/20-cau-trac-nghiem-toan-7-chan-troi-sang-tao-bai-tap-cuoi-chuong-6-co-dap-an-phan-2


\textbf{{QUESTION}}

Tỉ lệ thức nào sau đây không được lập từ tỉ lệ thức x2=ab?

\textbf{{ANSWER}}

Đáp án đúng là: B
Từ tỉ lệ thức x2=ab$$ \frac{x}{2}=\frac{a}{b}$$, ta suy ra x . b = 2 . a.
Khi đó ta có các tỉ lệ thức sau:
⦁ xa=2b$$ \frac{x}{a}=\frac{2}{b}$$. Suy ra phương án A đúng.
⦁ b2=ax$$ \frac{b}{2}=\frac{a}{x}$$. Suy ra phương án C đúng.
⦁ ba=2x$$ \frac{b}{a}=\frac{2}{x}$$. Suy ra phương án D đúng.
Do đó phương án B sai.
Vậy ta chọn phương án B.

========================================================================

https://khoahoc.vietjack.com/thi-online/20-cau-trac-nghiem-toan-7-chan-troi-sang-tao-bai-tap-cuoi-chuong-6-co-dap-an-phan-2


\textbf{{QUESTION}}

Cho biết đại lượng a tỉ lệ thuận với đại lượng m theo hệ số tỉ lệ là –2. Khi đó đại lượng m tỉ lệ thuận với đại lượng a theo hệ số tỉ lệ là:
A. 2
B. -2

\textbf{{ANSWER}}

Đáp án đúng là: C
Ta có đại lượng a tỉ lệ thuận với đại lượng m theo hệ số tỉ lệ là k = –2.
Suy ra đại lượng m tỉ lệ thuận với đại lượng a theo hệ số tỉ lệ là 1k=−12.
Vậy ta chọn phương án C.

========================================================================

https://khoahoc.vietjack.com/thi-online/20-cau-trac-nghiem-toan-7-chan-troi-sang-tao-bai-tap-cuoi-chuong-6-co-dap-an-phan-2


\textbf{{QUESTION}}

Phát biểu nào sau đây sai?

\textbf{{ANSWER}}

Đáp án đúng là: D
Phương án D sai vì hệ số tỉ lệ k ≠ 0.
Vậy ta chọn phương án D.

========================================================================

https://khoahoc.vietjack.com/thi-online/20-cau-trac-nghiem-toan-7-chan-troi-sang-tao-bai-tap-cuoi-chuong-6-co-dap-an-phan-2


\textbf{{QUESTION}}

Cho x và y là hai đại lượng tỉ lệ nghịch với nhau và khi x = 3 thì y = –9. Công thức biểu diễn y theo x là:

\textbf{{ANSWER}}

Đáp án đúng là: C
Vì x và y là hai đại lượng tỉ lệ nghịch với nhau nên ta có a = xy = 3 . (–9) = –27 ≠ 0.
Do đó ta có công thức biểu diễn y theo x là: y=−27x$$ y=\frac{-27}{x}$$.
Vậy ta chọn phương án C.

========================================================================

https://khoahoc.vietjack.com/thi-online/10-bai-tap-nhan-bsiet-bac-he-so-cao-nhat-he-so-tu-do-cua-da-thuc-mot-bien-co-loi-giai


\textbf{{QUESTION}}

Bậc của đa thức P(x) = 2x4 + x3 – 6x2 + 6x + 3 là

\textbf{{ANSWER}}

Đáp án đúng là: D
Bậc của đa thức P(x) = 2x4 + x3 – 6x2 + 6x + 3 là 4.

========================================================================

https://khoahoc.vietjack.com/thi-online/10-bai-tap-nhan-bsiet-bac-he-so-cao-nhat-he-so-tu-do-cua-da-thuc-mot-bien-co-loi-giai


\textbf{{QUESTION}}

Cho đa thức 6x5 – x4 + 5x2 – x + 2, hệ số cao nhất của đa thức đó là
A. 5;

\textbf{{ANSWER}}

Đáp án đúng là: C
Hệ số cao nhất của đa thức 6x5 – x4 + 5x2 – x + 2 là 6.

========================================================================

https://khoahoc.vietjack.com/thi-online/10-bai-tap-nhan-bsiet-bac-he-so-cao-nhat-he-so-tu-do-cua-da-thuc-mot-bien-co-loi-giai


\textbf{{QUESTION}}

Hệ số tự do của đa thức x3 – 2x2 + 3 là

\textbf{{ANSWER}}

Đáp án đúng là: C
Hệ số tự do của đa thức x3 – 2x2 + 3 là 3.
Do đó ta chọn đáp án C.

========================================================================

https://khoahoc.vietjack.com/thi-online/10-bai-tap-nhan-bsiet-bac-he-so-cao-nhat-he-so-tu-do-cua-da-thuc-mot-bien-co-loi-giai


\textbf{{QUESTION}}

Cho đa thức – 8x6 + 5x4 + 6x3 – 3x2 + 4, bậc của đa thức đó là

\textbf{{ANSWER}}

Đáp án đúng là: B
Bậc của đa thức – 8x6 + 5x4 + 6x3 – 3x2 + 4 là 6.

========================================================================

https://khoahoc.vietjack.com/thi-online/10-bai-tap-nhan-bsiet-bac-he-so-cao-nhat-he-so-tu-do-cua-da-thuc-mot-bien-co-loi-giai


\textbf{{QUESTION}}

Hệ số tự do của đa thức 7x12  – 8x10 + x11 – x5 + 6x6 + x – 10 là

\textbf{{ANSWER}}

Đáp án đúng là: C
Hệ số tự do của đa thức 7x12  – 8x10 + x11 – x5 + 6x6 + x – 10 là – 10.

========================================================================

https://khoahoc.vietjack.com/thi-online/7881-cau-trac-nghiem-tong-hop-mon-toan-2023-cuc-hay-co-dap-an/123073


\textbf{{QUESTION}}

Tìm x, biết: ${x^2} + 5x + 4 - 5\sqrt {{x^2} + 5x + 28} = 0$.
${x^2} + 5x + 4 - 5\sqrt {{x^2} + 5x + 28} = 0$

\textbf{{ANSWER}}

Lời giải
Đặt $\sqrt {{x^2} + 5x + 28} = t{\rm{ }}\left( {t > 0} \right)$ 
⇒ x2 + 5x = t2 – 28
Phương trình trở thành: t2 – 28 + 4 – 5t = 0 
⇔ t2 – 5t – 24 = 0 $ \Leftrightarrow \left[ \begin{array}{l}t = 8\,\,\,\left( {tm} \right)\\t = - 3\left( {ktm} \right)\end{array} \right.$
Với t = 8 ta có $\sqrt {{x^2} + 5x + 28} = 8$ 
⇔ x2 + 5x + 28 = 64 
⇔ x2 + 5x – 36 = 0
$ \Leftrightarrow \left[ \begin{array}{l}x = 4\\x = - 9\end{array} \right.$
Vậy $x \in \left\{ {4; - 9} \right\}$.

========================================================================

https://khoahoc.vietjack.com/thi-online/7881-cau-trac-nghiem-tong-hop-mon-toan-2023-cuc-hay-co-dap-an/123073


\textbf{{QUESTION}}

Cho định lí “Cho số tự nhiên n, nếu n5 chia hết cho 5 thì n chia hết cho 5”. 
Định lí này được viết dưới dạng P Þ Q. Hãy phát biểu định lí đảo của định lí trên rồi dùng các thuật ngữ “điều kiện cần và đủ” phát biểu gộp cả 2 định lí thuận và đảo.

\textbf{{ANSWER}}

Lời giải
– Định lý đảo: Cho số tự nhiên n, nếu n chia hết cho 5 thì n5 chia hết cho 5.
– Cho số tự nhiên n, n5 chia hết cho 5 là điều kiện cần và đủ để n chia hết cho 5.

========================================================================

https://khoahoc.vietjack.com/thi-online/7881-cau-trac-nghiem-tong-hop-mon-toan-2023-cuc-hay-co-dap-an/123073


\textbf{{QUESTION}}

Viết các số (0,25)8 và (0,125)4 dưới dạng các lũy thừa với cơ số 0,5.

\textbf{{ANSWER}}

Lời giải
Ta có: 
(0,25)8 = [(0,5)2]8 = (0,5)2.8 = (0,5)16
(0,125)4 = [(0,5)3]4 = (0,5)3.4 = (0,5)12

========================================================================

https://khoahoc.vietjack.com/thi-online/7881-cau-trac-nghiem-tong-hop-mon-toan-2023-cuc-hay-co-dap-an/123073


\textbf{{QUESTION}}

Cho một hộp đựng 4 viên bi đỏ, 5 viên bi xanh và 7 viên bi vàng. Lấy ngẫu nhiên một lần ba viên bi. Tính xác suất để trong ba viên bi lấy được chỉ có hai màu.

\textbf{{ANSWER}}

Lời giải
Gọi A là biến cố “ba viên bi lấy được chỉ có hai màu”
Ta có: Số phần tử của không gian mẫu: \(C_{16}^3 = 560\) 
Số cách chọn được ba viên bi chỉ có một màu: \(C_4^3 + C_5^3 + C_7^3 = 49\)
Số cách chọn được ba viên bi có đủ ba màu: \(C_4^1 + C_5^1 + C_7^1 = 140\)
Vậy xác suất cần tìm là: \({\rm P}\left( A \right) = 1 - \frac{{49 + 140}}{{560}} = \frac{{53}}{{80}}\).

========================================================================

https://khoahoc.vietjack.com/thi-online/7881-cau-trac-nghiem-tong-hop-mon-toan-2023-cuc-hay-co-dap-an/123073


\textbf{{QUESTION}}

Một hộp đựng 7 viên bi xanh, 5 viên bi đỏ và 4 viên bi vàng. Có bao nhiêu cách lấy ra 8 viên bi có đủ 3 màu?
A. 12 201;
B. 10 224;
C. 12 422;
D. 14 204.

\textbf{{ANSWER}}

Lời giải
Đáp án đúng là: A
Số cách lấy ra 8 viên bi bất kì: \(C_{16}^8 = 12\,\,870\)
Số cách lấy ra 8 viên bi không có màu vàng mà chỉ có hai màu xanh và đỏ: \(C_7^7C_5^1 + C_7^6C_5^2 + C_7^5C_5^3 + C_7^4C_5^4 + C_7^3C_5^5 = 495\)
Số cách lấy ra 8 viên bi không có màu đỏ mà có hai màu xanh và vàng:
\(C_7^7C_4^1 + C_7^6C_4^2 + C_7^5C_4^3 + C_7^4C_4^4 = 165\)
Số cách lấy ra 8 viên bi không có màu xanh mà chỉ có hai màu đỏ và vàng:
\(C_5^5C_4^3 + C_5^4C_4^4 = 9\)
Số cách lấy ra 8 viên bi có đủ 3 màu:  
12 870 − (495 + 165 + 9) = 12 201 (cách).

========================================================================

https://khoahoc.vietjack.com/thi-online/15-cau-trac-nghiem-toan-9-ket-noi-tri-thuc-bai-25-phep-thu-ngau-nhien-va-khong-gian-mau-co-dap-an


\textbf{{QUESTION}}

I. Nhận biết
Phát biểu đúng nhất trong các phát biểu sau là
A. Tập hợp tất cả các kết quả có thể xảy ra của một phép thử được gọi là không gian mẫu của phép thử đó.
B. Tập hợp tất cả các kết quả không thể xảy ra của một phép thử được gọi là không gian mẫu của phép thử đó.
C. Tập hợp các kết quả có thể xảy ra của một phép thử với khả năng xuất hiện như nhau được gọi là không gian mẫu của phép thử đó.
D. Tập hợp tất cả các kết quả không thể xảy ra của một phép thử với khả năng xuất hiện như nhau được gọi là không gian mẫu của phép thử đó.

\textbf{{ANSWER}}

Đáp án đúng là: A
Tập hợp các kết quả có thể xảy ra của một phép thử được gọi là không gian mẫu của phép thử đó.

========================================================================

https://khoahoc.vietjack.com/thi-online/15-cau-trac-nghiem-toan-9-ket-noi-tri-thuc-bai-25-phep-thu-ngau-nhien-va-khong-gian-mau-co-dap-an


\textbf{{QUESTION}}

Ký hiệu không gian mẫu của phép thử là
A. α.
B. β.
C. ℧
D. Ω.

\textbf{{ANSWER}}

Đáp án đúng là: D
Không gian mẫu của phép thử được ký hiệu là Ω$\Omega $.

========================================================================

https://khoahoc.vietjack.com/thi-online/15-cau-trac-nghiem-toan-9-ket-noi-tri-thuc-bai-25-phep-thu-ngau-nhien-va-khong-gian-mau-co-dap-an


\textbf{{QUESTION}}

Trong các thí nghiệm sau, thí nghiệm nào không phải là phép thử ngẫu nhiên?
A. Gieo đồng xu xem đồng xu đó xuất hiện mặt ngửa hay mặt sấp.
B. Gieo hai đồng xu và xem có mấy đồng tiền lật ngửa.
C. Chọn bất kì 1 học sinh trong lớp và xem là nam hay nữ.
D. Bỏ hai viên bi xanh và ba viên bi đỏ trong một chiếc hộp, sau đó lấy từng viên một để đếm xem có tất cả bao nhiêu viên bi.

\textbf{{ANSWER}}

Đáp án đúng là: D
Theo định nghĩa, ta có phép thử ngẫu nhiên là những phép thử mà ta không thể đoán trước kết quả của nó, mặc dù đã biết được tập hợp tất cả các kết quả của phép thử đó.
Đáp án D không phải là phép thử vì ta biết chắc chắn kết quả chỉ có thể là một số cụ thể số bi xanh và số bi đỏ.

========================================================================

https://khoahoc.vietjack.com/thi-online/15-cau-trac-nghiem-toan-9-ket-noi-tri-thuc-bai-25-phep-thu-ngau-nhien-va-khong-gian-mau-co-dap-an


\textbf{{QUESTION}}

Gieo một con xúc xắc cân đối và đồng chất. Không gian mẫu của phép thử có số phần tử là
A. 3.
B. 4.
C. 5.
D. 6.

\textbf{{ANSWER}}

Đáp án đúng là: D
Không gian mẫu của phép thử là Ω={1;2;3;4;5;6}.
Vậy không gian mẫu của phép thử có 6 phần tử.

========================================================================

https://khoahoc.vietjack.com/thi-online/15-cau-trac-nghiem-toan-9-ket-noi-tri-thuc-bai-25-phep-thu-ngau-nhien-va-khong-gian-mau-co-dap-an


\textbf{{QUESTION}}

Bạn An lấy ngẫu nhiên một số tự nhiên có 1 chữ số. Số phần tử của không gian mẫu của phép thử là
A. 4.
B. 8.
C. 9.
D. 10.

\textbf{{ANSWER}}

Đáp án đúng là: D
Không gian mẫu của phép thử là Ω={0;1;2;3;4;5;6;7;8;9}.
Vậy không gian mẫu của phép thử có 10 phần tử.

========================================================================

https://khoahoc.vietjack.com/thi-online/15-cau-trac-nghiem-hai-mat-phang-vuong-goc-co-dap-an-nhan-biet


\textbf{{QUESTION}}

Cho hai mặt phẳng (P) và (Q) song song với nhau và một điểm M không thuộc (P) và (Q). Qua M có bao nhiêu mặt phẳng vuông góc với (P) và (Q)?
A. 2
B. 3
C. 1
D. Vô số

\textbf{{ANSWER}}

Đáp án D
Gọi d là đường thẳng qua M và vuông góc với (P). Do (P) // (Q) ⇒ d$$ \bot $$(Q)
Giả sử (R) là mặt phẳng chứa d. Mà $$ \left\{\begin{array}{c}\mathrm{d}\bot \left(\mathrm{P}\right)\\ \mathrm{d}\bot \left(\mathrm{Q}\right)\end{array}\right.\Rightarrow \left\{\begin{array}{c}\left(\mathrm{R}\right)\bot \left(\mathrm{P}\right)\\ \left(\mathrm{R}\right)\bot \left(\mathrm{Q}\right)\end{array}\right.$$
Có vô số mặt phẳng (R) chứa d. Do đó có vô số mặt phẳng qua M, vuông góc với (P) và (Q).

========================================================================

https://khoahoc.vietjack.com/thi-online/15-cau-trac-nghiem-hai-mat-phang-vuong-goc-co-dap-an-nhan-biet


\textbf{{QUESTION}}

Trong các mệnh đề sau, mệnh đề nào đúng?
A. Hai mặt phẳng cùng song song với một mặt phẳng thứ ba thì song song với nhau.
B. Qua một đường thẳng cho trước có duy nhất một mặt phẳng vuông góc với một mặt phẳng cho trước.
C. Có duy nhất một mặt phẳng đi qua một điểm cho trước và vuông góc với hai mặt phẳng cắt nhau cho trước.
D. Hai mặt phẳng cùng vuông góc với một mặt phẳng thứ ba thì vuông góc với nhau.

\textbf{{ANSWER}}

Đáp án C
A sai. Hai mặt phẳng cùng song song với một mặt phẳng thứ ba thì song song hoặc trùng nhau.
B sai. Nếu đường thẳng vuông góc với mặt phẳng cho trước thì có vô số mặt phẳng qua đường thẳng và vuông góc với mặt phẳng đó. Nếu đường thẳng không vuông góc với mặt phẳng cho trước thì không có mặt phẳng nào vuông góc với mặt phẳng đó.
D sai. Hai mặt phẳng phân biệt cùng vuông góc với mặt phẳng thứ ba thì song song với nhau hoặc cắt nhau (giao truyến vuông góc với mặt phẳng kia).

========================================================================

https://khoahoc.vietjack.com/thi-online/15-cau-trac-nghiem-hai-mat-phang-vuong-goc-co-dap-an-nhan-biet


\textbf{{QUESTION}}

Trong các mệnh đề sau, mệnh đề nào đúng?
A. Cho hai đường thẳng song song a và b và đường thẳng c sao cho c⊥a, c⊥b. Mọi mặt phẳng (α) chứa c thì đều vuông góc với mặt phẳng (a,b).
B. Cho a⊥(α), mọi mặt phẳng (β) chứa a thì (β)⊥(α).
C. Cho a⊥b, mọi mặt phẳng chứa b đều vuông góc với a.
D. Cho a⊥b, nếu a⊂(α) và b⊂(β) thì (α)⊥(β).

\textbf{{ANSWER}}

Đáp án B
A sai. Trong trường hợp a, b, c đồng phẳng.
C sai. Trong trường hợp a và b cắt nhau, mặt phẳng (a,b) chứa b nhưng không vuông góc với a.
D sai. Trong trường hợp a và b vuông góc nhau và chéo nhau, nếu (α$$ \alpha $$)⊃$$ \supset $$a, (α$$ \alpha $$) // b và (β$$ \mathrm{\beta }$$)⊃$$ \supset $$b, (β$$ \mathrm{\beta }$$) // a thì (α$$ \alpha $$) // (β$$ \mathrm{\beta }$$).

========================================================================

https://khoahoc.vietjack.com/thi-online/15-cau-trac-nghiem-hai-mat-phang-vuong-goc-co-dap-an-nhan-biet


\textbf{{QUESTION}}

Trong các mệnh đề sau, mệnh đề nào sau đây là đúng?
A. Hai mặt phẳng vuông góc với nhau thì mọi đường thẳng nằm trong mặt phẳng này sẽ vuông góc với mặt phẳng kia.
B. Hai mặt phẳng phân biệt cùng vuông góc với một mặt phẳng thì vuông góc với nhau.
C. Hai mặt phẳng phân biệt cùng vuông góc với một mặt phẳng thì song song với nhau.
D. Hai mặt phẳng vuông góc với nhau thì mọi đường thẳng nằm trong mặt phẳng này và vuông góc với giao tuyến của hai mặt phẳng sẽ vuông góc với mặt phẳng kia

\textbf{{ANSWER}}

Đáp án D
A sai. Hai mặt phẳng vuông góc với nhau thì đường thẳng nằm trong mặt phẳng này, vuông góc với giao tuyến thì vuông góc với mặt phẳng kia.
B, C sai. Hai mặt phẳng phân biệt cùng vuông góc với một mặt phẳng thì song song với nhau hoặc cắt nhau (giao truyến vuông góc với mặt phẳng kia).

========================================================================

https://khoahoc.vietjack.com/thi-online/15-cau-trac-nghiem-hai-mat-phang-vuong-goc-co-dap-an-nhan-biet


\textbf{{QUESTION}}

Trong các mệnh đề sau, mệnh đề nào đúng?
A. Hai mặt phẳng phân biệt cùng vuông góc với một mặt phẳng thì song song với nhau.
B. Qua một đường thẳng có duy nhất một mặt phẳng vuông góc với một đường thẳng cho trước.
C. Hai mặt phẳng phân biệt cùng vuông góc với một đường thẳng thì song song với nhau.
D. Qua một điểm có duy nhất một mặt phẳng vuông góc với một mặt phẳng cho trước.

\textbf{{ANSWER}}

Đáp án C
A sai. Hai mặt phẳng phân biệt cùng vuông góc với một mặt phẳng thì song song với nhau hoặc cắt nhau (giao tuyến vuông góc với mặt phẳng thứ 3).
B sai. Qua một đường thẳng chưa chắc đã có mặt phẳng vuông góc với một đường thẳng cho trước (vì nếu hai đường thẳng đã cho không vuông góc với nhau thì không có mặt phẳng nào hết)
D sai. Qua một điểm có vô số mặt phẳng vuông góc với một mặt phẳng cho trước.

========================================================================

https://khoahoc.vietjack.com/thi-online/trac-nghiem-toan-10-bai-tap-cuoi-chuong-4-co-dap-an-1


\textbf{{QUESTION}}

Giá trị của biểu thức M = tan1°.tan2°.tan3°….tan89° là:
A. -1
B. $$ \frac{1}{2}$$ 
C. 1
D. 2

\textbf{{ANSWER}}

Hướng dẫn giải
Đáp án đúng là: C
Ta có:
tan89° = tan(90° ‒ 1°) = cot1°;
tan88° = tan(90° ‒ 2°) = cot2°;
…
tan46° = tan(90° ‒ 44°) = cot44°.
Do đó: M = tan1°.tan2°.tan3°….tan89° 
M = tan1°.tan2°….tan44°.tan45°.cot44°….cot2°.cot1°
M = (tan1°.cot1°).(tan2°.cot2°)…(tan44°.cot44°).tan45°
M = 1.1….1.1
M = 1.
Vậy M = 1.

========================================================================

https://khoahoc.vietjack.com/thi-online/trac-nghiem-toan-10-bai-tap-cuoi-chuong-4-co-dap-an-1


\textbf{{QUESTION}}

Giá trị của biểu thức M=sin60°+tan30°cot120°+cos30°$$ M=\frac{\mathrm{sin}60°+\mathrm{tan}30°}{\mathrm{cot}120°+cos30°}$$ bằng:
A. 1
B. 5
C. √32;$$ \frac{\sqrt{3}}{2};$$
D. 2√3$$ 2\sqrt{3}$$

\textbf{{ANSWER}}

Hướng dẫn giải
Đáp án đúng là: B
Ta có: M=sin60°+tan30°cot120°+cos30°$$ M=\frac{\mathrm{sin}60°+\mathrm{tan}30°}{\mathrm{cot}120°+cos30°}$$
M=√32+√33−√33+√32$$ M=\frac{\frac{\sqrt{3}}{2}+\frac{\sqrt{3}}{3}}{-\frac{\sqrt{3}}{3}+\frac{\sqrt{3}}{2}}$$
M=3√3+2√36:−2√3+3√36$$ M=\frac{3\sqrt{3}+2\sqrt{3}}{6}:\frac{-2\sqrt{3}+3\sqrt{3}}{6}$$
M=5√36:√36$$ M=\frac{5\sqrt{3}}{6}:\frac{\sqrt{3}}{6}$$  
M=5√36.6√3=5.$$ M=\frac{5\sqrt{3}}{6}.\frac{6}{\sqrt{3}}=5.$$ 
Vậy M = 5.

========================================================================

https://khoahoc.vietjack.com/thi-online/trac-nghiem-toan-10-bai-tap-cuoi-chuong-4-co-dap-an-1


\textbf{{QUESTION}}

Cho tam giác ABC có AB = 8, AC = 9, BC = 10. Tam giác ABC là tam giác:

\textbf{{ANSWER}}

Hướng dẫn giải
Đáp án đúng là: A
Áp dụng hệ quả định lí côsin trong tam giác ABC ta có: cosA=AC2+AB2−BC22.AC.AB$$ \mathrm{cos}A=\frac{A{C}^{2}+A{B}^{2}-B{C}^{2}}{2.AC.AB}$$
⇒cosA=92+82−1022.9.8=516>0$$ \Rightarrow \mathrm{cos}A=\frac{{9}^{2}+{8}^{2}-{10}^{2}}{\mathrm{2.9.8}}=\frac{5}{16}>0$$
Suy ra ˆA$$ \widehat{A}$$ là góc nhọn.
Tam giác ABC có cạnh BC lớn nhất đối diện với ˆA$$ \widehat{A}$$ nên ˆA$$ \widehat{A}$$ là góc lớn nhất.
Do đó ˆB,ˆC$$ \widehat{B},\widehat{C}$$ cũng là góc nhọn.
Vậy tam giác ABC là tam giác nhọn.

========================================================================

https://khoahoc.vietjack.com/thi-online/trac-nghiem-toan-10-bai-tap-cuoi-chuong-4-co-dap-an-1


\textbf{{QUESTION}}

Tam giác ABC có các góc ˆA=75°,ˆB=45°. Tỉ số ABAC bằng:
A. 1,2
B. √6;
C. √62
D. √63

\textbf{{ANSWER}}

Hướng dẫn giải
Đáp án đúng là: C
Xét tam giác ABC có ˆA=75°,ˆB=45°$$ \widehat{A}=75°,\widehat{B}=45°$$ ta có: 
ˆA+ˆB+ˆC=180°$$ \widehat{A}+\widehat{B}+\widehat{C}=180°$$ (định lí tổng ba góc trong tam giác)
⇒ˆC=180°−ˆA−ˆB$$ \Rightarrow \widehat{C}=180°-\widehat{A}-\widehat{B}$$ 
⇒ˆC=180°−75°−45°=60°$$ \Rightarrow \widehat{C}=180°-75°-45°=60°$$
Áp dụng định lí sin trong tam giác ABC ta có: ACsinB=ABsinC$$ \frac{AC}{\mathrm{sin}B}=\frac{AB}{\mathrm{sin}C}$$
⇒ABAC=sinCsinB=sin60°sin45°=√32√22=√62.$$ \Rightarrow \frac{AB}{AC}=\frac{\mathrm{sin}C}{\mathrm{sin}B}=\frac{\mathrm{sin}60°}{\mathrm{sin}45°}=\frac{\frac{\sqrt{3}}{2}}{\frac{\sqrt{2}}{2}}=\frac{\sqrt{6}}{2}.$$ 
Vậy ABAC=√62.$$ \frac{AB}{AC}=\frac{\sqrt{6}}{2}.$$

========================================================================

https://khoahoc.vietjack.com/thi-online/trac-nghiem-toan-10-bai-tap-cuoi-chuong-4-co-dap-an-1


\textbf{{QUESTION}}

Tam giác ABC có ˆA=120° với BC = a, AC = b, AB = c thì câu nào sau đây là đúng?

\textbf{{ANSWER}}

Hướng dẫn giải
Đáp án đúng là: C
Áp dụng định lí côsin trong tam giác ABC ta có: 
a2 = b2 + c2 ‒ 2.bc.cosA 
Þ a2 = b2 + c2 ‒ 2.bc.cos120° 
⇒a2= b2+ c2−2.bc.(−12) 
Þ a2 = b2 + c2 + bc.
Vậy a2 = b2 + c2 + bc.

========================================================================

https://khoahoc.vietjack.com/thi-online/bai-tap-toan-8-chu-de-11-phep-chia-cac-phan-thuc-dai-so-co-dap-an/108400


\textbf{{QUESTION}}

Tìm giá trị của x để phân thức A chia hết cho phân thức B biết:
 $$ A=\frac{{x}^{3}-{x}^{2}-x+11}{x-2}$$;$$ B=\frac{x+2}{x-2}$$  .

\textbf{{ANSWER}}

Ta có   $$ A:B=\frac{{x}^{3}-{x}^{2}-x+11}{x-2}:\frac{x+2}{x-2}=\frac{{x}^{3}-{x}^{2}-x+11}{x-2}.\frac{x-2}{x+2}=\frac{{x}^{3}-{x}^{2}-x+11}{x+2}={x}^{2}-3x+5+\frac{1}{x+2}$$
Để phân thức A chia hết cho phân thức B thì 
   $$ 1\vdots \left(x+2\right)$$$$ \Rightarrow $$$$ x+2\in $$Ư 
   $$ \Rightarrow $$$$ x+2\in \left\{-1;1\right\}$$$$ \Rightarrow $$$$ x\in \left\{-3;-1\right\}$$  
Vậy $$ x\in \left\{-3;-1\right\}$$ thì phân thức A chia hết cho phân thức B.

========================================================================

https://khoahoc.vietjack.com/thi-online/bai-tap-toan-8-chu-de-11-phep-chia-cac-phan-thuc-dai-so-co-dap-an/108400


\textbf{{QUESTION}}

Tìm giá trị của x để giá trị của phân thức M=1516x2−1:54x+1  là số nguyên.

\textbf{{ANSWER}}

Ta có M=1516x2−1:54x+1=15(4x−1)(4x+1).4x+15=34x−1
Để giá trị của phân thức M là số nguyên thì
   3⋮(4x−1)⇒4x−1∈Ư 
⇒4x−1∈{−3;−1;1;3}⇒x∈{−12;0;12;1}
Vậy x∈{−12;0;12;1} thì giá trị của phân thức M là số nguyên.

========================================================================

https://khoahoc.vietjack.com/thi-online/de-kiem-tra-15-phut-toan-7-chuong-4-dai-so-co-dap-an/40197


\textbf{{QUESTION}}

Trong mỗi câu dưới đây, hãy chọn phương án trả lời đúng:
Đa thức 2x + 1 = 0 có nghiệm là:
A. x = 1/2
B. x = -1/2
C. x = -2
D. x = -1

\textbf{{ANSWER}}

Chọn B
Ta có 2x + 1 = 0 ⇒ 2x = -1 ⇒ x = -1/2

========================================================================

https://khoahoc.vietjack.com/thi-online/de-kiem-tra-15-phut-toan-7-chuong-4-dai-so-co-dap-an/40197


\textbf{{QUESTION}}

Giá trị x = 2 là nghiệm của đa thức:
A. x2-2x 
B. 2x + x2
C. x2-12x
 D. x + 2

\textbf{{ANSWER}}

Chọn A
Thay x = 2 vào đa thức x2 - 2x ta có 22 - 2.2 = 0 nên x = 2 là nghiệm của x2 - 2x.

========================================================================

https://khoahoc.vietjack.com/thi-online/de-kiem-tra-15-phut-toan-7-chuong-4-dai-so-co-dap-an/40197


\textbf{{QUESTION}}

Nghiệm của đa thức f(x) = x2 + 1 là:
A. x = 1
B. x = -1
C. x = ±1
D. Không có nghiệm

\textbf{{ANSWER}}

Chọn D
Ta có x2 ≥ 0 ⇒ x2 + 1 > 0. Đa thức vô nghiệm.

========================================================================

https://khoahoc.vietjack.com/thi-online/de-kiem-tra-15-phut-toan-7-chuong-4-dai-so-co-dap-an/40197


\textbf{{QUESTION}}

Cho đa thức g(x) = x2 - 4. Nghiệm của g(x) là:
A. x = ±2
B. x = 2
C. x = -2
D. Không có nghiệm

\textbf{{ANSWER}}

Chọn A
Thay x = 2 vào g(x) ta có g(2) = 22 - 4 = 0
Thay x = -2 vào g(x) ta có g(-2) = (-2)2 - 4 = 0
Nên x = ±2 là nghiệm của g(x).

========================================================================

https://khoahoc.vietjack.com/thi-online/de-kiem-tra-15-phut-toan-7-chuong-4-dai-so-co-dap-an/40197


\textbf{{QUESTION}}

Cho hai đa thức f(x) = 2x2 - 5x - 3 và g(x) = -2x2- 2x + 1. Nghiệm của đa thức f(x) + g(x) = 0 là:
A. x=53
B. x=-72
C. x=-27
D. x=-35

\textbf{{ANSWER}}

Chọn C
Ta có f(x) + g(x) = (2x2 - 5x - 3) + (-2x2 - 2x + 1) = -7x - 2
Cho -7x - 2 = 0 ⇒ x = -2/7

========================================================================

https://khoahoc.vietjack.com/thi-online/15-cau-trac-nghiem-toan-9-ket-noi-tri-thuc-bai-1-khai-niem-phuong-trinh-va-he-hai-phuong-trinh-bac-n


\textbf{{QUESTION}}

I. Nhận biết
Trong các phương trình sau, phương trình nào không phải là phương trình bậc nhất hai ẩn?
A. $x + 2y = 1$.
B. $0x - 0y = 5$.
C. $0x - y = 3$.
D. $x + 0y = - 6$.

\textbf{{ANSWER}}

Đáp án đúng là: B
Phương trình $0x - 0y = 5$ không phải là phương trình bậc nhất hai ẩn do $a,{\rm{ }}b$ đồng thời bằng 0.

========================================================================

https://khoahoc.vietjack.com/thi-online/15-cau-trac-nghiem-toan-9-ket-noi-tri-thuc-bai-1-khai-niem-phuong-trinh-va-he-hai-phuong-trinh-bac-n


\textbf{{QUESTION}}

Hệ số a,b,c tương ứng của phương trình bậc nhất hai ẩn x+2y−1=0 là
A. a=1;b=1;c=0.
B. a=1;b=2;c=1.
C. a=1;b=2;c=−1.
D. a=1;b=−2;c=1.

\textbf{{ANSWER}}

Đáp án đúng là: B
Phương trình bậc nhất hai ẩn x+2y−1=0$x + 2y - 1 = 0$ nên a=1;b=2;c=−1$a = 1;\,\,b = 2;\,\,c = - 1$.

========================================================================

https://khoahoc.vietjack.com/thi-online/15-cau-trac-nghiem-toan-9-ket-noi-tri-thuc-bai-1-khai-niem-phuong-trinh-va-he-hai-phuong-trinh-bac-n


\textbf{{QUESTION}}

Cặp số nào sau đây là nghiệm của phương trình 2x−y−1=0?
A. (0;1).
B. (1;0).
C. (1;1).
D. (−1;0).

\textbf{{ANSWER}}

Đáp án đúng là: C
⦁ Thay x=0 và y=1 vào phương trình 2x−y−1=0 ta được 2⋅0−1−1=−2≠0.
Do đó (0;1) không phải là nghiệm phương trình đã cho.
⦁ Thay x=1 và y=0 vào phương trình 2x−y−1=0 ta được 2⋅1−0−1=1≠0.
Do đó (1;0) không phải là nghiệm phương trình.
⦁ Thay x=1 và y=1 vào phương trình 2x−y−1=0 ta được 2⋅1−1−1=0.
Do đó (1;1) là nghiệm phương trình đã cho.
⦁ Thay x=−1 và y=0 vào phương trình 2x−y−1=0 ta được 2⋅(−1)−0−1=−3≠0.
Do đó (−1;0) không phải là nghiệm phương trình đã cho.
Vậy ta chọn phương án C.

========================================================================

https://khoahoc.vietjack.com/thi-online/15-cau-trac-nghiem-toan-9-ket-noi-tri-thuc-bai-1-khai-niem-phuong-trinh-va-he-hai-phuong-trinh-bac-n


\textbf{{QUESTION}}

Trong các hệ phương trình sau, hệ nào không phải là hệ phương trình bậc nhất hai ẩn?
A. \[\left\{ \begin{array}{l}x - y = 2\\2x + y = 1\end{array} \right.\].
B. \[\left\{ \begin{array}{l}2x = 0\\x + 5y = 15\end{array} \right.\].
C. \[\left\{ \begin{array}{l}{x^2} - 4{y^2} = 0\\3x + 2y = 7\end{array} \right.\].
D. \[\left\{ \begin{array}{l}2x - y =  - 5\\3y + 15 = 0\end{array} \right.\].

\textbf{{ANSWER}}

Đáp án đúng là: C
Ta thấy hệ phương trình \[\left\{ \begin{array}{l}{x^2} - 4{y^2} = 0\\3x + 2y = 7\end{array} \right.\] có chứa số hạng có bậc của \(x,\,\,y\) là 2 nên không phải là phương trình bậc nhất hai ẩn.

========================================================================

https://khoahoc.vietjack.com/thi-online/15-cau-trac-nghiem-toan-9-ket-noi-tri-thuc-bai-1-khai-niem-phuong-trinh-va-he-hai-phuong-trinh-bac-n


\textbf{{QUESTION}}

Cho hệ phương trình \[\left\{ \begin{array}{l}x + y = 0\\x + 3y = 4\end{array} \right.\], cặp số nào sau đây là nghiệm của hệ phương trình đã cho?
A. \(\left( {0;\,\,1} \right)\).
B. \(\left( {2;\,\,2} \right)\).
C. \(\left( {3;\,\, - 3} \right)\).
D. \(\left( { - 2;\,\,2} \right)\).

\textbf{{ANSWER}}

Đáp án đúng là: D
⦁ Thay \(x = 0\) và \(y = 1\) vào phương trình \(x + y = 0\) ta được \(0 + 1 = 1 \ne 0\) nên cặp số \(\left( {0;\,\,1} \right)\) không phải là nghiệm của phương trình \(x + y = 0\). Do đó cặp số \(\left( {0;\,\,1} \right)\) không phải là nghiệm của hệ phương trình đã cho.
⦁ Thay \(x = 2\) và \(y = 2\) vào phương trình \(x + y = 0\) ta được \(2 + 2 = 4 \ne 0\) nên cặp số \(\left( {2;\,\,2} \right)\) không phải là nghiệm của phương trình \(x + y = 0\). Do đó cặp số \(\left( {2;\,\,2} \right)\) không phải là nghiệm của hệ phương trình đã cho.
⦁ Thay \(x = 3\) và \(y = - 3\) vào phương trình \[x + 3y = 4\] ta được \[3 + 3 \cdot \left( { - 3} \right) = - 9 \ne 4\] nên cặp số \(\left( {3;\,\, - 3} \right)\) không phải là nghiệm của phương trình \[x + 3y = 4\]. Do đó cặp số \(\left( {3;\,\, - 3} \right)\) không phải là nghiệm của hệ phương trình đã cho.
⦁ Thay \(x = - 2\) và \(y = 2\) vào từng phương trình của hệ phương trình đã cho, ta được:
\( - 2 + 2 = 0\);
\( - 2 + 3 \cdot 2 = 4\).
Do đó cặp số \(\left( { - 2;\,\,2} \right)\) là nghiệm chung của hai phương trình nên \(\left( { - 2;\,\,2} \right)\) là một nghiệm của hệ phương trình đã cho.

========================================================================

https://khoahoc.vietjack.com/thi-online/on-thi-cap-toc-789-vao-10-mon-toan-khu-vuc-da-nang-2024-2025-de-12


\textbf{{QUESTION}}

1) Tính $A = \sqrt 9  + \sqrt {12}  + \sqrt {27}  - 5\sqrt 3 .$
2) Cho biểu thức $B = \left( {\frac{1}{{\sqrt x  + 2}} + \frac{1}{{\sqrt x  - 2}}} \right) \cdot \left( {\frac{{\sqrt x }}{{\sqrt x  - 2}} - \frac{4}{{x - 2\sqrt x }}} \right)$ với $x > 0$ và $x \ne 4.$

\textbf{{ANSWER}}

1) Ta có:
$A = \sqrt 9  + \sqrt {12}  + \sqrt {27}  - 5\sqrt 3 $$ = \sqrt {{3^2}}  + \sqrt {{2^2} \cdot 3}  + \sqrt {{3^2} \cdot 3}  - 5\sqrt 3 $
 $ = 3 + 2\sqrt 3  + 3\sqrt 3  - 5\sqrt 3 $$ = 3 + \left( {2 + 3 - 5} \right) \cdot \sqrt 3 $$ = 3.$
Vậy $A = 3.$
2) Với $x > 0$ và $x \ne 4,$ ta có:
$B = \left( {\frac{1}{{\sqrt x  + 2}} + \frac{1}{{\sqrt x  - 2}}} \right) \cdot \left( {\frac{{\sqrt x }}{{\sqrt x  - 2}} - \frac{4}{{x - 2\sqrt x }}} \right)$
$ = \frac{{\sqrt x  - 2 + \sqrt x  + 2}}{{\left( {\sqrt x  + 2} \right)\left( {\sqrt x  - 2} \right)}} \cdot \left[ {\frac{{\sqrt x }}{{\sqrt x  - 2}} - \frac{4}{{\sqrt x \left( {\sqrt x  - 2} \right)}}} \right]$
$ = \frac{{2\sqrt x }}{{\left( {\sqrt x  + 2} \right)\left( {\sqrt x  - 2} \right)}} \cdot \frac{{x - 4}}{{\sqrt x \left( {\sqrt x  - 2} \right)}}$
$ = \frac{{2\sqrt x \left( {x - 4} \right)}}{{\left( {x - 4} \right) \cdot \sqrt x  \cdot \left( {\sqrt x  - 2} \right)}} = \frac{2}{{\sqrt x  - 2}}.$
Như vậy, với $x > 0$ và $x \ne 4,$ thì $B = \frac{2}{{\sqrt x  - 2}}.$
Khi đó, để $B < 0$ thì $\frac{2}{{\sqrt x  - 2}} < 0,$ tức là $\sqrt x  - 2 < 0,$ suy ra $\sqrt x  < 2,$ nên $x < 4.$
Đối chiếu điều kiện $x > 0$ và $x \ne 4,$ ta được $0 < x < 4.$
Vậy với $0 < x < 4$ thì $B < 0.$

========================================================================

https://khoahoc.vietjack.com/thi-online/27-cau-trac-nghiem-toan-7-ket-noi-tri-thuc-bai-on-tap-cuoi-chuong-1-co-dap-an-phan-2/111134


\textbf{{QUESTION}}

Tính giá trị biểu thức D = $$ \frac{9}{10}-\left(\frac{4}{5}-\frac{9}{4}\right)$$
A. $$ \frac{49}{10}$$;
B. $$ \frac{47}{10}$$;
C. $$ \frac{49}{20}$$;

\textbf{{ANSWER}}

Hướng dẫn giải
Đáp án đúng là: D
Ta có: D = $$ \frac{9}{10}-\left(\frac{4}{5}-\frac{9}{4}\right)$$
D = $$ \frac{9}{10}-\frac{8}{10}+\frac{9}{4}$$
D = $$ \frac{1}{10}+\frac{9}{4}$$
D = $$ \frac{2}{20}+\frac{45}{20}$$
D = $$ \frac{47}{20}$$
Vậy đáp án đúng là D.

========================================================================

https://khoahoc.vietjack.com/thi-online/27-cau-trac-nghiem-toan-7-ket-noi-tri-thuc-bai-on-tap-cuoi-chuong-1-co-dap-an-phan-2/111134


\textbf{{QUESTION}}

Để làm một cái bánh, cần 223$$ 2\frac{2}{3}$$ cốc bột. Hằng đã có 23$$ \frac{2}{3}$$ cốc bột. Hỏi Hằng cần thêm bao nhiêu cốc bột nữa để vừa đủ làm được một cái bánh? 
A. 13$$ \frac{1}{3}$$;
B. 2;
C. 23$$ \frac{2}{3}$$;

\textbf{{ANSWER}}

Hướng dẫn giải
Đáp án đúng là: B
Số cốc bột Hằng cần để vừa đủ làm được một cái bánh là: 
223−23=2+25−23=2+(23−23)=2+(23−23)$$ 2\frac{2}{3}-\frac{2}{3}=2+\frac{2}{5}-\frac{2}{3}=2+\left(\frac{2}{3}-\frac{2}{3}\right)=2+\left(\frac{2}{3}-\frac{2}{3}\right)$$ = 2 (cốc). 
Vậy Hằng cần thêm 2 cốc bột nữa để vừa đủ làm được một cái bánh.

========================================================================

https://khoahoc.vietjack.com/thi-online/27-cau-trac-nghiem-toan-7-ket-noi-tri-thuc-bai-on-tap-cuoi-chuong-1-co-dap-an-phan-2/111134


\textbf{{QUESTION}}

Phân số nào biểu diễn số hữu tỉ –1,5 ?
A. 15$$ \frac{1}{5}$$;
B. 32$$ \frac{3}{2}$$;
C. -32$$ \frac{-3}{2}$$;

\textbf{{ANSWER}}

Hướng dẫn giải
Đáp án đúng là: C
Ta có: – 1,5 = −1510=−32$$ -\frac{15}{10}=-\frac{3}{2}$$.
Vậy phân số biểu diễn số hữu tỉ – 1,5 là −32$$ -\frac{3}{2}$$.

========================================================================

https://khoahoc.vietjack.com/thi-online/10-cau-trac-nghiem-toan-9-bai-4-vi-tri-tuong-doi-cua-duong-thang-va-duong-tron-co-dap-an


\textbf{{QUESTION}}

Đường thẳng và đường tròn có nhiều nhất bao nhiêu điểm chung
A. 1
B. 2
C.3
D. 4

\textbf{{ANSWER}}

Đáp án B
Đường thẳng và đường tròn có nhiều nhất hai điểm chung

========================================================================

https://khoahoc.vietjack.com/thi-online/10-cau-trac-nghiem-toan-9-bai-4-vi-tri-tuong-doi-cua-duong-thang-va-duong-tron-co-dap-an


\textbf{{QUESTION}}

Nếu đường thẳng và đường tròn có duy nhất một điểm chung thì
A. đường thẳng tiếp xúc với đường tròn
B. đường thẳng cắt đường tròn
C. đường thẳng không cắt đường tròn
D. đáp án khác

\textbf{{ANSWER}}

Đáp án A
Đường thẳng và đường tròn chỉ có một điểm chung thì đường thẳng tiếp xúc với đường tròn

========================================================================

https://khoahoc.vietjack.com/thi-online/de-thi-giua-ki-1-toan-7-ctst-co-dap-an/106324

\textbf{{QUESTION}}

Trong các câu sau, câu nào đúng?
B. Số 0 là số hữu tỉ dương;
C. Số nguyên âm không phải là số hữu tỉ âm;

\textbf{{ANSWER}}

Đáp án đúng là: A
Số hữu tỉ âm nhỏ hơn số hữu tỉ dương. Đúng.
Số 0 là số hữu tỉ dương. Sai vì số 0 không là số hữu tỉ dương, cũng không là số hữu tỉ âm.
Số nguyên âm không phải là số hữu tỉ âm. Sai vì mỗi số nguyên là một số hữu tỉ.
Tập hợp ℚ gồm các số hữu tỉ dương và các số hữu tỉ âm. Sai vì tập hợp ℚ gồm các số hữu tỉ dương, số 0 và các số hữu tỉ âm.

========================================================================

https://khoahoc.vietjack.com/thi-online/de-thi-giua-ki-1-toan-7-ctst-co-dap-an/106324

\textbf{{QUESTION}}

Số đối của số hữu tỉ $$ \frac{9}{4}$$ là
B.$$ \frac{-9}{-4}$$;
C. $$ \frac{4}{9}$$;

\textbf{{ANSWER}}

Đáp án đúng là: A
Số đối của số hữu tỉ $$ \frac{9}{4}$$ là $$ -\frac{9}{4}$$.

========================================================================

https://khoahoc.vietjack.com/thi-online/de-thi-giua-ki-1-toan-7-ctst-co-dap-an/106324

\textbf{{QUESTION}}

Cho a = $$ \frac{2}{-9}$$ và b = $$ \frac{-1}{3}$$.
Khẳng định nào sau đây là đúng?
A. a = b;
B. a > b;

\textbf{{ANSWER}}

Đáp án đúng là: B
Ta có:
$$ \frac{2}{-9}=\frac{-2}{9}$$
$$ \frac{-1}{3}=\frac{\left(-1\right).3}{3.3}=\frac{-3}{9}$$ (quy đồng mẫu số)
Vì ‒2 > ‒3 nên $$ \frac{-2}{9}>\frac{-3}{9}$$
Hay $$ \frac{2}{-9}$$ > $$ \frac{-1}{3}$$.
Vậy $$ \frac{2}{-9}$$ > $$ \frac{-1}{3}$$.

